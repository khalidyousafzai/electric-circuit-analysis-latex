\باب{برقرار برقی طاقت}

\حصہ{لمحاتی طاقت}
شکل \حوالہ{شکل_طاقت_پرزے_کو_منتقل} میں بوجھ \عددی{\bZ} کو بدلتی رو منبع  طاقت فراہم کرتا ہے۔اس عمومی دور کے برقرار دباو اور برقرار رو درج ذیل لکھے جا سکتے ہیں۔
\begin{gather}
\begin{aligned}\label{مساوات_طاقت_دباو_رو_عمومی_الف}
v(t)&=V_0\cos(\omega t +\phi_v)\\
i(t)&=I_0\cos(\omega t +\phi_i)
\end{aligned}
\end{gather}
یوں کسی بھی لمحہ بوجھ کو منتقل طاقت درج ذیل ہو گا
\begin{gather}
\begin{aligned}
p(t)&=v(t)i(t)\\
&=V_0 I_0  \cos(\omega t +\phi_v) \cos(\omega t +\phi_i)
\end{aligned}
\end{gather}
جس میں
\begin{align}
\cos \alpha \cos \beta=\frac{\cos(\alpha-\beta)+\cos(\alpha+\beta)}{2}
\end{align}
استعمال کرتے ہوئے
\begin{align}\label{مساوات_طاقت_لمحاتی_طاقت_الف}
p(t)=\frac{V_0 I_0}{2}\left[\cos(\phi_v-\phi_i)+\cos(2\omega t +\phi_v+\phi_i)\right]
\end{align}
ملتا ہے جہاں \عددی{\alpha=\omega t +\phi_v} اور \عددی{\beta=\omega t+\phi_i} لئے گئے ہیں۔آپ دیکھ سکتے ہیں کہ لمحاتی طاقت دو اجزاء کا مجموعہ ہے۔پہلا جزو مستقل طاقت ہے جو وقت کے ساتھ تبدیل نہیں ہوتا جبکہ دوسرا جزو دگنی تعدد کا بدلتی رو طاقت ہے۔  
%
\begin{figure}
\centering
\begin{tikzpicture}
\draw(0,0) to [american voltage source,l={$v(t)$}]++(0,\y) to [short,i={$i(t)$}]++(\x,0) to [european resistor,l={$\bZ$}]++(0,-\y) to [short](0,0);
\end{tikzpicture}
\caption{بدلتی رو دور۔}
\label{شکل_طاقت_پرزے_کو_منتقل}
\end{figure}
%======================
\ابتدا{مثال}\شناخت{مثال_طاقت_عمومی_الف}
شکل \حوالہ{شکل_طاقت_پرزے_کو_منتقل} میں برقرار دباو \عددی{v(t)=15\cos(100 t+45^{\circ})\,\si{\volt}} اور \عددی{\bZ=5\phase{20^{\circ}}\,\si{\ohm}} ہیں۔بوجھ کو منتقل لمحاتی طاقت دریافت کریں۔

حل:دوری سمتیات استعمال کرتے ہوئے
\begin{align*}
\hat{I}&=\frac{15\phase{45^{\circ}}}{5\phase{20^{\circ}}}\\
&=3\phase{25^{\circ}}\,\si{\ampere}
\end{align*}
یعنی
\begin{align*}
i(t)=3\cos(100 t +25^{\circ}) \, \si{\ampere}
\end{align*}
لکھا جا سکتا ہے۔یوں مساوات \حوالہ{مساوات_طاقت_لمحاتی_طاقت_الف} سے لمحاتی طاقت درج ذیل لکھی جا سکتی ہے۔
 \begin{align*}
p(t)&=22.5\left[\cos 20^{\circ}+\cos(200 t +70^{\circ})\right] \\
&=21.143+22.5\cos(200t+70^{\circ})\,\si{\watt}
\end{align*}
دباو، رو اور طاقت کے خط شکل \حوالہ{شکل_طاقت_عمومی_الف} میں دکھائے گئے ہیں۔درج بالا مساوات میں \عددی{\SI{21.143}{\watt}} مستقل طاقت ہے جو وقت کے ساتھ تبدیل نہیں ہوتا جبکہ \عددی{22.5\cos(200t+70^{\circ})\,\si{\watt}} بدلتی رو طاقت ہے جس کی تعدد \عددی{\SI{200}{\radian\per\second}} ہے۔
\begin{figure}
\centering
\begin{subfigure}{0.5\textwidth}
\centering
\begin{tikzpicture}
\begin{axis}[kStyleCircuitsA,small,xlabel=$\omega t$, xtick={90,180,270,360},xticklabels={$90^{\circ}$,$180^{\circ}$,$270^{\circ}$,$360^{\circ}$},]
\addplot[domain=0:370,samples=100]{15*cos(1*x+45)}node[pos=0.7,left]{$v(t)$};
\addplot[domain=0:370,samples=100]{3*cos(1*x+25)}node[pos=0.9,above]{$i(t)$};
\end{axis}%
\end{tikzpicture}%
\end{subfigure}%
\begin{subfigure}{0.5\textwidth}
\centering
\begin{tikzpicture}
\begin{axis}[kStyleCircuitsA,small,xlabel=$\omega t$, xtick={90,180,270,360},xticklabels={$90^{\circ}$,$180^{\circ}$,$270^{\circ}$,$360^{\circ}$},]
\addplot[domain=0:370,samples=100]{21.143+22.5*cos(2*x+25)}node[pos=0.4,left]{$p(t)$};
\end{axis}%
\end{tikzpicture}%
\end{subfigure}%
\caption{مثال \حوالہ{مثال_طاقت_عمومی_الف} کے اشکال۔}
\label{شکل_طاقت_عمومی_الف}
\end{figure}
\انتہا{مثال}
%==============
\ابتدا{مثال}
شکل \حوالہ{شکل_طاقت_پرزے_کو_منتقل} میں \عددی{v(t)=V_0\cos(\omega t +\phi_v)\,\si{\volt}} اور \عددی{\bZ=Z_0\phase{\phi_z}\,\si{\ohm}} ہیں۔رو دریافت کریں۔

حل:دوری سمتیات استعمال کرتے ہوئے
\begin{align*}
\hat{I}&=\frac{V_0\phase{\phi_v}}{Z_0\phase{\phi_z}}\\
&=\frac{V_0}{Z_0}\phase{\phi_v-\phi_z}
\end{align*}
لکھا جا سکتا ہے جس سے وقتی دائرہ کار میں رو درج ذیل حاصل ہوتی ہے۔
\begin{align}
i(t)=\frac{V_0}{Z_0} \cos(\omega t+\phi_v-\phi_z)
\end{align}
مساوات \حوالہ{مساوات_طاقت_دباو_رو_عمومی_الف} میں دیے عمومی رو کے ساتھ موازنہ کرتے ہوئے آپ دیکھ سکتے ہیں کہ \عددی{\phi_i} درحقیقت میں \عددی{\phi_v-\phi_z} کے برابر ہے  جسے درج ذیل لکھا جا سکتا ہے۔
\begin{align}\label{مساوات_طاقت_زاویہ_رکاوٹ_اور_طاقت}
\phi_v-\phi_i=\phi_z
\end{align}
\انتہا{مثال}
%=====================

\حصہ{اوسط طاقت}
دہراتے تفاعل (مثلاً سائن نما تفاعل) کے ایک دوری عرصے پر تکمل کو دوری عرصے سے تقسیم کرنے سے تفاعل کی اوسط قیمت حاصل ہوتی ہے۔یوں مساوات \حوالہ{مساوات_طاقت_دباو_رو_عمومی_الف} میں دیے دباو اور رو کی صورت میں بوجھ کو منتقل اوسط طاقت درج ذیل ہو گی
\begin{gather}
\begin{aligned}\label{مساوات_طاقت_اوسط_تکمل_الف}
P&=\frac{1}{T}\int_{t_0}^{t_0+T} p(t) \dif t\\
&=\frac{V_0 I_0}{T}\int_{t_0}^{t_0+T} \cos(\omega t +\phi_v) \cos(\omega t +\phi_i) \dif t
\end{aligned}
\end{gather}    
جہاں \عددی{t_0} کوئی بھی لمحہ ہو سکتا ہے جبکہ \عددی{T=\tfrac{2\pi}{\omega}} دباو یا رو کا دوری عرصہ ہے۔حقیقت میں ہم ایک دوری عرصے کی بجائے  \عددی{n} مکمل دوری عرصے پر تکمل لیتے ہوئے \عددی{n} دوری عرصے سے تقسیم کرتے ہوئے بھی اوسط قیمت حاصل کر سکتے ہیں۔یوں اوسط طاقت درج ذیل بھی لکھی جا سکتی ہے۔
\begin{align}
P&=\frac{V_0 I_0}{nT}\int_{t_0}^{t_0+nT} \cos(\omega t +\phi_v) \cos(\omega t +\phi_i) \dif t
\end{align} 
مساوات \حوالہ{مساوات_طاقت_لمحاتی_طاقت_الف} کی مدد سے مساوات \حوالہ{مساوات_طاقت_اوسط_تکمل_الف} درج ذیل لکھا جائے گا۔
\begin{gather}
\begin{aligned}\label{مساوات_طاقت_اوسط_تکمل_ب}
P&=\frac{V_0 I_0}{2T}\int_{t_0}^{t_0+T} \left[\cos(\phi_v-\phi_i)+\cos(2\omega t +\phi_v+\phi_i)\right] \dif t\\
&=\frac{V_0 I_0}{2T}\int_{t_0}^{t_0+T} \cos(\phi_v-\phi_i) \dif t+\frac{V_0 I_0}{2T}\int_{t_0}^{t_0+T} \cos(2\omega t +\phi_v+\phi_i)\dif t
\end{aligned}
\end{gather} 
درج بالا تکمل کے دو اجزاء کو باری باری حل کرتے ہیں۔پہلا جزو مستقل ہے لہٰذا اس کو تکمل کے باہر لکھتے ہوئے حل کرتے ہیں۔
\begin{align*}
\frac{V_0 I_0}{2T}\int_{t_0}^{t_0+T} \cos(\phi_v-\phi_i) \dif t&=\frac{V_0 I_0}{2T}\cos(\phi_v-\phi_i) \int_{t_0}^{t_0+T} \dif t\\
&=\left. \frac{V_0 I_0}{2T}\cos(\phi_v-\phi_i) t\right|_{t_0}^{t_0+T}\\
&=\frac{V_0 I_0}{2}\cos(\phi_v-\phi_i)
\end{align*}
اب مساوات \حوالہ{مساوات_طاقت_اوسط_تکمل_ب} کے دوسرے جزو کو حل کرتے ہیں
\begin{align*}
\frac{V_0 I_0}{2T}\int_{t_0}^{t_0+T} \cos(2\omega t +\phi_v+\phi_i)\dif t&=\left. \frac{V_0 I_0}{2T}\frac{ \sin(2\omega t +\phi_v+\phi_i)}{2\omega}\right|_{t_0}^{t_0+T}\\
&=0
\end{align*}
جہاں \عددی{\sin \alpha=\sin(\alpha+T)} کا استعمال کیا گیا ہے۔یوں مساوات \حوالہ{مساوات_طاقت_اوسط_تکمل_ب} سے درج ذیل اوسط طاقت حاصل  ہوتا ہے۔
\begin{align}\label{مساوات_طاقت_عمومی_الف}
P=\frac{V_0 I_0}{2}\cos(\phi_v-\phi_i)
\end{align}
چونکہ \عددی{\cos(\alpha)=\cos(-\alpha)} کے برابر ہے لہٰذا درج بالا مساوات میں کوسائن کا دلیل \عددی{\phi_v-\phi_i} یا \عددی{\phi_i-\phi_v} لکھا جا سکتا ہے۔مساوات \حوالہ{مساوات_طاقت_زاویہ_رکاوٹ_اور_طاقت} کو استعمال کرتے ہوئے درج بالا مساوات کو دوبارہ لکھتے ہیں۔
\begin{align}\label{مساوات_طاقت_عمومی_ب}
P=\frac{V_0 I_0}{2}\cos \phi_z
\end{align}
خالص مزاحمتی رکاوٹ \عددی{\bZ=R\phase{0^{\circ}}} کا زاویہ ہٹاو \عددی{0^{\circ}} ہوتا ہے لہٰذا \عددی{\cos 0^{\circ}=1} لیتے ہوئے مزاحمتی بوجھ کا طاقت
\begin{align}\label{مساوات_طاقت_مزاحمتی_طاقت_الف}
P_{\text{مزاحمتی}}=\frac{V_0 I_0}{2}
\end{align}
ہو گا جہاں \عددی{V_0} سے مراد مزاحمت کے دباو کا حیطہ ہے۔قانون اوہم سے درج بالا کو درج ذیل صورتوں میں بھی لکھا جا سکتا ہے۔
\begin{align}
P_{\text{مزاحمتی}}&=\frac{I^2_0 R}{2} \label{مساوات_طاقت_مزاحمتی_طاقت_ب}\\
P_{\text{مزاحمتی}}&=\frac{V^2_0}{2 R} \label{مساوات_طاقت_مزاحمتی_طاقت_پ}
\end{align}
درج بالا تینوں مساوات کا یک سمتی رو میں مزاحمتی ضیاع کے مساوات کے ساتھ موازنہ کرنے سے معلوم ہوتا ہے کہ موجودہ تینوں مساوات میں کسر کے نچلی جانب دو \عددی{(2)} کا اضافی عدد پایا جاتا ہے۔اس پر آگے جا کر مزید بات ہو گی۔

امالی متعاملیت کی رکاوٹ \عددی{\bZ_L=X_L\phase{90^{\circ}}} جبکہ برق گیر متعاملیت کی رکاوٹ \عددی{\bZ_C=X_C\phase{-90^{\circ}}} ہوتی ہے۔چونکہ \عددی{\cos (\mp 90^{\circ})=0} ہوتا ہے لہٰذا غیر مزاحمتی رکاوٹ کی طاقت صفر ہو گی۔
\begin{align}
P_{\text{متعاملی}} =0
\end{align}
چونکہ خالص متعامل پرزوں کو صفر اوسط طاقت منتقل ہوتی ہے لہٰذا انہیں \اصطلاح{بے ضیاع پرزے}\فرہنگ{بے ضیاع پرزے}\حاشیہب{lossless components}\فرہنگ{lossless components} کہتے ہیں۔دور کا متعامل حصہ، دوری عرصے کے کچھ حصے میں  دور سے طاقت حاصل کرتے ہوئے  ذخیرہ کرتا ہے  جبکہ دوری عرصے کے کسی دوسرے حصے میں اسی طاقت کو دور کو واپس کرتا ہے۔

%==========================
\ابتدا{مثال}\شناخت{مثال_طاقت_مزاحمت_امالہ_الف}
شکل \حوالہ{مشق_طاقت_مزاحمت_امالہ_الف} میں رکاوٹ کی اوسط طاقت دریافت کریں۔
\begin{figure}
\centering
\begin{tikzpicture}
\draw(0,0) to [american voltage source,l={$50\phase{30^{\circ}}\,\si{\volt}$}]++(0,2*\y) to [short,i={$\hat{I}$}]++(\x,0) to [resistor,l={$\SI{3}{\ohm}$}]++(0,-\y) to [inductor,l={$j6 \,\si{\ohm}$}]++(0,-\y) to [short] (0,0);
\end{tikzpicture}
\caption{مثال \حوالہ{مثال_طاقت_مزاحمت_امالہ_الف} کا دور۔}
\label{مشق_طاقت_مزاحمت_امالہ_الف}
\end{figure}

حل:رو درج ذیل ہے۔
\begin{align*}
\hat{I}&=\frac{50\phase{30^{\circ}}}{3+j6}=\frac{50\phase{30^{\circ}}}{3+j6}=\frac{50\phase{30^{\circ}}}{\sqrt{45}\phase{63.435^{\circ}}}=7.454\phase{-33.435^{\circ}}\,\si{\ampere}
\end{align*}
یوں
\begin{align*}
P&=\frac{V_0 I_0}{2}\cos(\phi_v-\phi_i)\\
&=\frac{(50)(7.454)}{2}\cos[30^{\circ}-(-33.435^{\circ})]\\
&=\SI{83.34}{\watt}
\end{align*}
ہو گا۔چونکہ طاقت صرف مزاحمت میں ضائع ہوتی ہے لہٰذا یہی جواب مساوات \حوالہ{مساوات_طاقت_مزاحمتی_طاقت_الف} سے بھی حاصل کیا جا سکتا ہے جہاں \عددی{V_0} سے مراد مزاحمت کے دباو کا حیطہ ہے۔تقسیم دباو سے مزاحمت کا دباو درج ذیل ہے
\begin{align*}
\hat{V}_R&=\left(\frac{3}{3+j6}\right)50\phase{30^{\circ}}=22.361\phase{-33.435^{\circ}}
\end{align*}
جس سے مزاحمت کا اوسط طاقت درج ذیل ہو گا۔
\begin{align*}
P&=\frac{V_0 I_0}{2}=\frac{(22.361)(7.454)}{2}=\SI{83.34}{\watt}
\end{align*}
اسی طرح مساوات \حوالہ{مساوات_طاقت_مزاحمتی_طاقت_ب} اور مساوات \حوالہ{مساوات_طاقت_مزاحمتی_طاقت_پ} بھی استعمال کیے جا سکتے ہیں
\begin{align*}
P&=\frac{I^2_0 R}{2}=\frac{(7.454^2)(3)}{2}=\SI{83.34}{\watt}\\
P&=\frac{V^2_0}{2 R}=\frac{(22.361^2)}{(2)(3)}=\SI{83.34}{\watt}
\end{align*}
\انتہا{مثال}
%==========================
\ابتدا{مثال}\شناخت{مثال_طاقت_مزاحمت_امالہ_ب}
شکل \حوالہ{شکل_طاقت_مزاحمت_امالہ_ب} میں منبع دباو کا اوسط طاقت حاصل کریں۔دور کے بقایا پرزوں کا اوسط طاقت بھی دریافت کریں۔
\begin{figure}
\centering
\begin{tikzpicture}
\draw(0,0) to [american voltage source,i^<={$\hat{I}_m$},l={$10\phase{30^{\circ}}\,\si{\volt}$}]++(0,2*\y) to [short]++(4*\x,0) to [capacitor,i>_={$\hat{I}_C$},l={$-j10\,\si{\ohm}$}]++(0,-2*\y) to [short] (0,0);
\draw(\x,0) to [inductor,i_<={$\hat{I}_L$},*-*,l={$j5\,\si{\ohm}$}]++(0,2*\y);
\draw(2*\x,0) to [resistor,i_<={$\hat{I}_R$},*-*,l={$\SI{2}{\ohm}$}]++(0,2*\y);
\draw(3*\x,0) to [inductor,*-,l={$j2\,\si{\ohm}$}]++(0,\y) to [resistor,i_<={$\hat{I}_Z$},-*,l={$\SI{2}{\ohm}$}]++(0,\y);
\end{tikzpicture}
\caption{مثال \حوالہ{مثال_طاقت_مزاحمت_امالہ_ب} کا دور۔}
\label{شکل_طاقت_مزاحمت_امالہ_ب}
\end{figure} 

حل:پہلے تمام رو دریافت کرتے ہیں۔شکل میں دباو کو دیکھتے ہوئے انفعالی رائج رو کے تحت رو کی سمتیں چننی گئی ہیں۔ 
\begin{align*}
\hat{I}_L&=\frac{10\phase{30^{\circ}}}{j5}=\frac{10\phase{30^{\circ}}}{5\phase{90^{\circ}}}=2\phase{-60^{\circ}}\\
\hat{I}_R&=\frac{10\phase{30^{\circ}}}{2}=\frac{10\phase{30^{\circ}}}{2\phase{0^{\circ}}}=5\phase{30^{\circ}}\\
\hat{I}_Z&=\frac{10\phase{30^{\circ}}}{2+j2}=\frac{10\phase{30^{\circ}}}{\sqrt{8}\phase{45^{\circ}}}=\frac{5}{\sqrt{2}}\phase{-15^{\circ}}\\
\hat{I}_C&=\frac{10\phase{30^{\circ}}}{-j10}=\frac{10\phase{30^{\circ}}}{10\phase{-90^{\circ}}}=1\phase{120^{\circ}}\\
\hat{I}_m&=-\left[\hat{I}_L+\hat{I}_R+\hat{I}_Z+\hat{I}_C\right]=8.27647\phase{-175.01689^{\circ}}
\end{align*}
یوں انفرادی شاخوں کے اوسط طاقت مساوات \حوالہ{مساوات_طاقت_عمومی_الف} یا مساوات \حوالہ{مساوات_طاقت_عمومی_ب} سے درج ذیل ہوں گے۔
\begin{align*}
P_L&=\frac{(30)(2)}{2}\cos(90^{\circ})&=\SI{0}{\watt}\\
P_R&=\frac{(30)(5)}{2}\cos(0^{\circ})&=\SI{75}{\watt}\\
P_Z&=\frac{(30)(\tfrac{5}{\sqrt{2}})}{2}\cos(45^{\circ})&=\SI{37.5}{\watt}\\
P_C&=\frac{(30)(1)}{2}\cos(90^{\circ})&=\SI{0}{\watt}\\
P_m&=\frac{(30)(8.27647)}{2}\cos[(30^{\circ}+175.01689^{\circ})]&=-\SI{112.5}{\watt}
\end{align*}
مثبت جواب طاقت کا ضیاع ہے جبکہ منفی جواب طاقت کی پیداوار ہے۔آپ دیکھ سکتے ہیں کہ منبع کی طاقتی پیداوار \عددی{\SI{112.5}{\watt}} ہے جو دور میں طاقت کے ضیاع 
\begin{align*}
P_L+P_R+P_Z+P_C=0+75+37.5+0=\SI{112.5}{\watt}
\end{align*}
کے عین برابر ہے۔
\انتہا{مثال}
%==========================
\ابتدا{مشق}\شناخت{مشق_طاقت_دریافت_کریں_الف}
شکل \حوالہ{شکل_طاقت_دریافت_کریں_الف} کے تمام مزاحمتوں میں ضائع ہونے والا اوسط طاقت دریافت کریں۔


\begin{figure}
\centering
\begin{tikzpicture}
\draw(0,0) to [american voltage source,l={$22\phase{-30^{\circ}}\,\si{\volt}$}]++(0,2*\y) to [capacitor,l={$-j2\,\si{\ohm}$}]++(\x,0) to [resistor,l={$\SI{4}{\ohm}$}]++(\x,0) to [inductor,l={$j1\,\si{\ohm}$}]++(0,-\y) to [resistor,l={$\SI{5}{\ohm}$}]++(0,-\y) to [short] (0,0);
\draw(2*\x,2*\y) to [short,*-]++(\x,0) to [inductor,l={$j10\,\si{\ohm}$}]++(0,-2*\y) to [short,-*]++(-\x,0);
\end{tikzpicture}
\caption{مشق \حوالہ{مشق_طاقت_دریافت_کریں_الف} کا دور۔}
\label{شکل_طاقت_دریافت_کریں_الف}
\end{figure}

جوابات:\عددی{P_{\SI{4}{\ohm}}=\SI{17.491}{\watt}}، \عددی{P_{\SI{5}{\ohm}}=\SI{14.975}{\watt}}
\انتہا{مشق}
%============================

%==========================
\ابتدا{مشق}\شناخت{مشق_طاقت_دریافت_کریں_ب}
شکل \حوالہ{شکل_طاقت_دریافت_کریں_ب} کے تمام مزاحمتوں میں ضائع ہونے والا اوسط طاقت دریافت کریں۔
\begin{figure}
\centering
\begin{tikzpicture}
\draw(0,0) to [american current source,l={$10\phase{45^{\circ}}\,\si{\ampere}$}]++(0,\yy) to [capacitor,l={$-j2\,\si{\ohm}$}]++(\xx,0) to [resistor,l={$\SI{4}{\ohm}$}]++(0,-\yy) to [short] (0,0);
\draw(0,\yy) to [resistor,*-,l_={$\SI{2}{\ohm}$}]++(-\xx,0) to [inductor,l_={$j4\,\si{\ohm}$}]++(0,-\yy) to [short,-*] (0,0);
\end{tikzpicture}
\caption{مشق \حوالہ{مشق_طاقت_دریافت_کریں_ب} کا دور۔}
\label{شکل_طاقت_دریافت_کریں_ب}
\end{figure}

جوابات:\عددی{P_{\SI{2}{\ohm}}=\SI{50}{\watt}}، \عددی{P_{\SI{4}{\ohm}}=\SI{100}{\watt}}
\انتہا{مشق}
%====================
\ابتدا{مشق}\شناخت{مشق_طاقت_دریافت_کریں_پ}
شکل \حوالہ{شکل_طاقت_دریافت_کریں_پ} کے تمام مزاحمتوں میں ضائع ہونے والا اوسط طاقت دریافت کریں۔
\begin{figure}
\centering
\begin{tikzpicture}
\draw(0,0) to [american voltage source,l={$20\phase{30^{\circ}}\,\si{\volt}$}]++(0,\yy) to [european resistor,l={$2+j2\,\si{\ohm}$}]++(\xx,0) to [european resistor,l={$6+j2\,\si{\ohm}$}]++(0,-\yy) to [short] (0,0);
\draw(\xx,0) to [short,*-]++(\xx,0) to [american voltage source,l_={$40\phase{60^{\circ}}\,\si{\volt}$}]++(0,\yy) to [european resistor,l_={$3-j4\,\si{\ohm}$},-*]++(-\xx,0);
\end{tikzpicture}
\caption{مشق \حوالہ{مشق_طاقت_دریافت_کریں_پ} کا دور۔}
\label{شکل_طاقت_دریافت_کریں_پ}
\end{figure}

جوابات:\عددی{P_{\SI{2}{\ohm}}=\SI{22.72}{\watt}} ،\عددی{P_{\SI{3}{\ohm}}=\SI{5.71}{\watt}}، \عددی{P_{\SI{6}{\ohm}}=\SI{11.42}{\watt}}
\انتہا{مشق}
%==============

ایک سے زیادہ منبع کی صورت میں آپ کسی بھی ترکیب کو استعمال کرتے ہوئے شاخوں کی رو اور جوڑ کے دباو حاصل کرتے ہوئے طاقت دریافت کر سکتے ہیں۔البتہ یاد رہے کہ ترکیب نفاذ سے طاقت کا تخمینہ نہیں لگایا جا سکتا چونکہ طاقت مربع دباو (یا مربع رو) کا تعلق رکھتا ہے جو غیر خطی تعلق ہے۔ 

%===============
\ابتدا{مشق}\شناخت{مشق_طاقت_دریافت_کریں_ت}
شکل \حوالہ{شکل_طاقت_دریافت_کریں_ت} میں اوسط طاقت کی پیداوار اور ضیاع معلوم کریں۔ 
\begin{figure}
\centering
\begin{tikzpicture}
\draw(0,0) to [american voltage source,l={$20\phase{30^{\circ}}\,\si{\volt}$}]++(0,\y) to [resistor,l={$\SI{2}{\ohm}$}]++(\x,0) to [inductor,l={$j4\,\si{\ohm}$}]++(\x,0);
\draw(0,0) to [short]++(2*\x,0) to [american voltage source,l_={$40\phase{0^{\circ}}\,\si{\volt}$}]++(0,\y);
\end{tikzpicture}
\caption{مشق \حوالہ{مشق_طاقت_دریافت_کریں_ت} کا دور۔}
\label{شکل_طاقت_دریافت_کریں_ت}
\end{figure}

\عددی{P_{20\phase{30^{\circ}}}=\SI{-25.36}{\watt}}، \عددی{P_{40\phase{0^{\circ}}}=\SI{-5.36}{\watt}}، \عددی{P_{\SI{2}{\ohm}}=\SI{30.72}{\watt}}
\انتہا{مشق}
%================

\ابتدا{مشق}\شناخت{مشق_طاقت_دریافت_کریں_ٹ}
شکل \حوالہ{شکل_طاقت_دریافت_کریں_ٹ} میں اوسط طاقت کی پیداوار اور ضیاع معلوم کریں۔ 
\begin{figure}
\centering
\begin{tikzpicture}
\draw(0,0) to [american voltage source,l={$30\phase{0^{\circ}}$}]++(0,\y) to [short]++(\x,0) to [inductor,l={$j8\,\si{\ohm}$}]++(0,-\y) to [short] (0,0);
\draw(\x,0) to [short,*-] ++(\x,0) to [inductor,l_={$j6\,\si{\ohm}$}]++(0,\y) to [capacitor,-*,l_={$-j10\,\si{\ohm}$}]++(-\x,0);
\end{tikzpicture}
\caption{مشق \حوالہ{مشق_طاقت_دریافت_کریں_ٹ} کا دور۔}
\label{شکل_طاقت_دریافت_کریں_ٹ}
\end{figure}

جواب:اوسط طاقت کی پیدا وار اور طاقت کا ضیاع صفر واٹ ہیں۔
\انتہا{مشق}
%================

\حصہ{زیادہ سے زیادہ اوسط طاقت منتقل کرنے کا مسئلہ}
یک سمتی رو ادوار میں ہم زیادہ سے زیادہ طاقت منتقل کرنے کے مسئلے پر ہم حصہ \حوالہ{حصہ_مسئلے_زیادہ_سے_زیادہ_طاقت_منتقل} میں غور کر چکے ہیں۔آئیں بدلتی رو کی صورت میں اسی مسئلے پر دوبارہ غور کریں۔

کسی بھی دور کا تھونن مساوی حاصل کیا جا سکتا ہے۔شکل \حوالہ{شکل_طاقت_زیادہ_سے_زیادہ_الف} میں تھونن مساوی دور کے ساتھ بوجھ جوڑا گیا ہے جہاں تھونن دباو کو \عددی{\hat{V}_{\text{کھلا}}} کہا گیا ہے۔ہم جاننا چاہتے ہیں کہ بوجھ کو کس صورت میں زیادہ سے زیادہ اوسط طاقت منتقل ہو گا۔  
\begin{figure}
\centering
\begin{tikzpicture}[american voltages]
\draw(0,0) to [american voltage source,l={$\hat{V}_{\text{کھلا}}$}]++(0,\y) to [european resistor,l={$\bZ_{\text{تھونن}}$}]++(\x,0) to [short,i={$\hat{I}_{\text{بوجھ}}$}]++(\x,0) to [european resistor,v={$\hat{V}_{\text{بوجھ}}$},l={$\bZ_{\text{تھونن}}$}]++(0,-\y) to [short] (0,0);
\end{tikzpicture}
\caption{زیادہ سے زیادہ اوسط طاقت منتقل کرنے کا مسئلہ۔}
\label{شکل_طاقت_زیادہ_سے_زیادہ_الف}
\end{figure}

شکل کو دیکھ کر درج ذیل لکھا جا سکتا ہے
\begin{align}\label{مساوات_طاقت_زیادہ_سے_زیادہ_طاقت_الف}
\hat{I}_{\text{بوجھ}} &=\frac{\hat{V}_{\text{کھلا}}}{\bZ_{\text{تھونن}}+\bZ_{\text{بوجھ}}}
\end{align}
جہاں
\begin{align*}
\bZ_{\text{تھونن}}&=R_{\text{تھونن}}+jX_{\text{تھونن}}\\
\bZ_{\text{بوجھ}}&=R_{\text{بوجھ}}+jX_{\text{بوجھ}}\\
\hat{V}_{\text{کھلا}}&=V_{\text{کھلا}} \phase{\phi_{\text{کھلا}}}
\end{align*}
ہیں۔درج بالا میں امالی رکاوٹ کی صورت میں \عددی{X} کی قیمت مثبت ہو گی جبکہ برق گیر رکاوٹ کی صورت میں اس کی قیمت منفی ہو گی۔یوں مساوات \حوالہ{مساوات_طاقت_زیادہ_سے_زیادہ_طاقت_الف} کو درج ذیل لکھا جا سکتا ہے
\begin{align*}
\hat{I}_{\text{بوجھ}}&=\frac{V_{\text{کھلا}} \phase{\phi_{\text{کھلا}}}}{R_{\text{تھونن}}+jX_{\text{تھونن}}+R_{\text{بوجھ}}+jX_{\text{بوجھ}}}
\end{align*}
جس کی حتمی قیمت درج ذیل ہے۔
\begin{align*}
I_{\text{بوجھ}}&=\frac{V_{\text{کھلا}}}{\sqrt{(R_{\text{تھونن}}+R_{\text{بوجھ}})^2+(X_{\text{تھونن}}+X_{\text{بوجھ}})^2}}
\end{align*}

بوجھ کو منتقل اوسط طاقت مساوات \حوالہ{مساوات_طاقت_مزاحمتی_طاقت_ب} کی مدد سے لکھتے ہیں۔
\begin{gather}
\begin{aligned}\label{مساوات_طاقت_زیادہ_سے_زیادہ_مساوات_الف}
P_{\text{بوجھ}}&=\frac{1}{2}I^2_{\text{بوجھ}} R_{\text{بوجھ}}\\
&=\frac{\frac{1}{2}V^2_{\text{کھلا}}\, R_{\text{بوجھ}}}{(R_{\text{تھونن}}+R_{\text{بوجھ}})^2+(X_{\text{تھونن}}+X_{\text{بوجھ}})^2}
\end{aligned}
\end{gather} 
ہم جانتے ہیں کہ \عددی{X} میں طاقت ضائع نہیں ہوتا لہٰذا اس کو اوسطاً صفر طاقت منتقل ہوتا ہے۔درج بالا مساوات میں کسر کے نچلے حصے
 میں \عددی{X_{\text{بوجھ}}+X_{\text{تھونن}}}  کی قیمت کم سے کم کرتے ہوئے طاقت بڑھائی جا سکتی ہے۔درج ذیل صورت میں اس قیمت کو صفر بنایا جا سکتا ہے۔
\begin{align}\label{مساوات_طاقت_زیادہ_سے_زیادہ_مساوات_ب}
X_{\text{بوجھ}}=-X_{\text{تھونن}}  \quad \quad \text{\RL{بوجھ کو زیادہ سے زیادہ طاقت کی منتقلی کا پہلا شرط}}
\end{align}
مساوات \حوالہ{مساوات_طاقت_زیادہ_سے_زیادہ_مساوات_ب} کے شرط پر پورا اترتے ہوئے مساوات \حوالہ{مساوات_طاقت_زیادہ_سے_زیادہ_مساوات_الف} کو درج ذیل لکھا جا سکتا ہے۔
\begin{align}\label{مساوات_طاقت_زیادہ_سے_زیادہ_مساوات_پ}
P_{\text{بوجھ}}&=\frac{V^2_{\text{کھلا}}\, R_{\text{بوجھ}}}{2(R_{\text{تھونن}}+R_{\text{بوجھ}})^2}
\end{align}
آئیں جانتے ہیں کہ کس قیمت کے \عددی{R_{\text{بوجھ}}} کو زیادہ سے زیادہ طاقت  منتقل ہو گی۔یہ جاننے کے لئے درج بالا مساوات کے تفرق کو صفر کے برابر پُر کرتے ہوئے  \عددی{R_{\text{بوجھ}}} کی درکار قیمت حاصل کرتے ہیں۔
\begin{align*}
\frac{\dif P_{\text{بوجھ}}}{\dif R_{\text{بوجھ}}} = \frac{V^2_{\text{بوجھ}}\left(R_{\text{تھونن}}+R_{\text{بوجھ}}\right)^2-2V_{\text{بوجھ}}^2 R_{\text{بوجھ}} \left(R_{\text{تھونن}}+R_{\text{بوجھ}}\right)}{2\left(R_{\text{تھونن}}+R_{\text{بوجھ}}\right)^4} =0
\end{align*}
اس سے
\begin{align}\label{مساوات_طاقت_زیادہ_سے_زیادہ_مساوات_ت}
R_{\text{بوجھ}}=R_{\text{تھونن}} \quad \quad \text{\RL{بوجھ کو زیادہ سے زیادہ طاقت کی منتقلی کا دوسرا شرط}}
\end{align}
حاصل ہوتا ہے۔اس نتیجے کے تحت بوجھ کو اس صورت زیادہ سے زیادہ طاقت منتقل ہو گی جب بوجھ کی مزاحمت دور کے تھونن مزاحمت کے برابر ہو۔مساوات \حوالہ{مساوات_طاقت_زیادہ_سے_زیادہ_مساوات_ب} اور مساوات \حوالہ{مساوات_طاقت_زیادہ_سے_زیادہ_مساوات_ت} کو استعمال کرتے ہوئے، بوجھ کو زیادہ سے زیادہ طاقت منتقل ہونے کی شرط کو درج ذیل لکھا جا سکتا ہے۔
\begin{gather}
\begin{aligned}
R_{\text{بوجھ}}+jX_{\text{بوجھ}} &=R_{\text{تھونن}}-jX_{\text{تھونن}}\\
\bZ_{\text{بوجھ}}=\bZ^*_{\text{تھونن}}
\end{aligned}
\end{gather}
آخر میں یہ بھی بتلاتا چلوں کہ مزاحمتی بوجھ \عددی{(X_L=0)} کی صورت میں مساوات \حوالہ{مساوات_طاقت_زیادہ_سے_زیادہ_مساوات_الف} کے تفرق کو صفر
\begin{align*}
\frac{\dif P_{\text{بوجھ}}}{\dif R_{\text{بوجھ}}}=0
\end{align*}
 کے برابر پر کرنے سے درج ذیل ملتا ہے۔
\begin{align}
R_{\text{بوجھ}}=\sqrt{R^2_{\text{تھونن}}+X^2_{\text{تھونن}}}
\end{align}

%=======================
\ابتدا{مثال}\شناخت{مثال_طاقت_زیادہ_سے-زیادہ_مثال_الف}
شکل \حوالہ{شکل_طاقت_زیادہ_سے-زیادہ_مثال_الف} میں بوجھ کے رکاوٹ کی وہ قیمت دریافت کریں جس پر بوجھ کو زیادہ سے زیادہ طاقت منتقل ہو گا۔اس طاقت کی قیمت بھی دریافت کریں۔
\begin{figure}
\centering
\begin{subfigure}{1\textwidth}
\centering
\begin{tikzpicture}
\draw(0,0) to [american voltage source,l={$20\phase{0^{\circ}}\,\si{\volt}$}]++(0,\y) to [inductor,l={$j2\,\si{\ohm}$}]++(\x,0) to [resistor,l={$\SI{6}{\ohm}$}]++(0,-\y) to [short] (0,0);
\draw(\x,0) to [short,*-]++(\x,0) to [european resistor,l_={$\bZ_{\text{بوجھ}}$}]++(0,\y) to [capacitor,-*,l_={$-j4\,\si{\ohm}$}]++(-\x,0);
\end{tikzpicture}
\caption*{(الف)}
\end{subfigure}
\begin{subfigure}{0.5\textwidth}
\centering
\begin{tikzpicture}
\draw(0,0) to [short]++(0,\y) to [inductor,l={$j2\,\si{\ohm}$}]++(\x,0) to [resistor,l={$\SI{6}{\ohm}$}]++(0,-\y) to [short] (0,0);
\draw(\x,0) to [short,*-o]++(\x,0)++(0,\y) to [capacitor,o-*,l_={$-j4\,\si{\ohm}$}]++(-\x,0);
\draw[stealth-] (2*\x,\y/2)--++(\x/8,0)--++(0,-\y/8)node[below]{$\bZ_{\text{تھونن}}$};
\end{tikzpicture}
\caption*{(ب)}
\end{subfigure}%
\begin{subfigure}{0.5\textwidth}
\centering
\begin{tikzpicture}
\draw(0,0) to [american voltage source,l={$20\phase{0^{\circ}}\,\si{\volt}$}]++(0,\y) to [inductor,l={$j2\,\si{\ohm}$}]++(\x,0) to [resistor,l={$\SI{6}{\ohm}$}]++(0,-\y) to [short] (0,0);
\draw(\x,0) to [short,*-o]++(\x,0) ++(0,\y) to [capacitor,o-*,l_={$-j4\,\si{\ohm}$}]++(-\x,0);
\draw(2*\x,\y/2)node{$\begin{aligned} &+ \\ &\hat{V}_{\text{کھلا}} \\ &- \end{aligned}$};
\end{tikzpicture}
\caption*{(پ)}
\end{subfigure}
\begin{subfigure}{1\textwidth}
\centering
\begin{tikzpicture}
\draw(0,0) to [american voltage source,l={$18.97\phase{-18.43^{\circ}}\,\si{\volt}$}]++(0,\y) to [european resistor,l={${0.6-j2.2\,\si{\ohm}}$}]++(\x,0) to [european resistor,i>^={$\hat{I}_{\text{بوجھ}}$},l={${0.6+j2.2\,\si{\ohm}}$}]++(0,-\y) to [short] (0,0);
\end{tikzpicture}
\caption*{(ت)}
\end{subfigure}
\caption{مثال \حوالہ{مثال_طاقت_زیادہ_سے-زیادہ_مثال_الف} کا دور۔}
\label{شکل_طاقت_زیادہ_سے-زیادہ_مثال_الف}
\end{figure}

حل:سب سے پہلے بوجھ کو ہٹاتے ہوئے بقایا دور کا تھونن مساوی حاصل کرنا ہو گا۔شکل-ب میں منبع دباو کو قصر دور کیا گیا ہے تا کہ تھونن مزاحمت حاصل کی جا سکے۔اسی طرح شکل-پ میں کھلے دور دباو کی نشاندہی کی گئی ہے۔ شکل-ب تھونن رکاوٹ لکھتے ہیں۔
\begin{align*}
\bZ_{\text{تھونن}}&=-j4+\frac{(6)(j2)}{6+j2}=\frac{3}{5}-j\frac{11}{5} \, \si{\ohm}
\end{align*}
یوں بوجھ کو زیادہ سے زیادہ طاقت کی منتقلی کے لئے ضروری ہے کہ بوجھ کی رکاوٹ درج ذیل ہو۔
\begin{align*}
\bZ_{\text{بوجھ}}=\frac{3}{5}+j\frac{11}{5} \, \si{\ohm}
\end{align*} 
شکل-پ میں برق گیر میں صفر رو ہے لہٰذا اس پر دباو بھی صفر ہو گا۔اس طرح مزاحمت پر دباو ہی تھونن دباو ہے جسے تقسیم دباو کے کلیے سے لکھتے ہیں۔
\begin{align*}
\hat{V}_{\text{کھلا}}&=\left(\frac{6}{6+j2}\right) (20\phase{0^{\circ}})=18.97\phase{-18.43^{\circ}}\,\si{\volt}
\end{align*}
شکل-ت میں تھونن مساوی دور کو بوجھ کے ساتھ جوڑ کر دکھایا گیا ہے جہاں سے رو حاصل کرتے ہیں۔
\begin{align*}
\hat{I}_{\text{بوجھ}}&=\frac{18.97\phase{-18.43^{\circ}}}{\frac{3}{5}-j\frac{11}{5}+\frac{3}{5}+j\frac{11}{5}}\\
&=15.81\phase{-18.43^{\circ}}\,\si{\ampere}
\end{align*}
یوں بوجھ کو منتقل طاقت درج ذیل ہو گا۔
\begin{align*}
P_{\text{بوجھ}}=\frac{(15.81^2)(0.6)}{2}=\SI{74.99}{\watt}
\end{align*}

\انتہا{مثال}
%=================

\باب{لاپلاس بدل}

\حصہ{تعریف}
کسی تفاعل \عددی{f(t)} کا \اصطلاح{لاپلاس بدل}\فرہنگ{لاپلاس بدل}\حاشیہب{Laplace transform}\فرہنگ{Laplace transform} درج ذیل مساوات دیتا ہے
\begin{align}
\Laplace[f(t)]=\bF(s)=\int_0^{\infty} f(t) e^{-st}\dif t
\end{align}
جہاں \عددی{s} \اصطلاح{مخلوط تعدد}\فرہنگ{مخلوط تعدد}\فرہنگ{تعدد!مخلوط}\حاشیہب{complex frequency}\فرہنگ{complex!frequency}\فرہنگ{frequency!complex} ہے
\begin{align}
s=\sigma+j\omega
\end{align}
اور تفاعل \عددی{f(t)} کی قیمت \عددی{t<0} پر صفر کے برابر ہے۔
\begin{align}
f(t)=0\quad t<0
\end{align}
لاپلاس بدل سے ادوار کا حل \عددی{t\ge0} کے لئے حاصل کیا جاتا ہے جبکہ \عددی{t<0} کو ابتدائی حالت میں سمایا جاتا ہے۔  

کسی تفاعل کا لاپلاس بدل اس صورت پایا جاتا ہے جب تفاعل درج ذیل شرط پر پورا اترتا ہو جہاں \عددی{\sigma} کوئی مثبت قیمت ہے۔
\begin{align}\label{مساوات_لاپلاس_شرط_بدل_پایا_جاتا_ہے}
\int_0^{\infty} e^{-\sigma t}\abs{f(t)}\dif t<\infty
\end{align}
لاپلاس بدل کے حصول میں \عددی{e^{-\sigma t}} کے ارتکازی جزو  کی بنا کئی ایسے کئی اہم تفاعل کے لاپلاس بدل پائے جاتے ہیں جن کے \اصطلاح{فوریئر بدل}\فرہنگ{فوریئر بدل}\حاشیہب{Fourier transform}\فرہنگ{Fourier transform} نہیں پائے جاتے۔برقی ادوار میں ایسے تفاعل استعمال کئے جاتے ہیں جن کے لاپلاس بدل پائے جاتے ہوں۔

\اصطلاح{الٹ لاپلاس بدل}\فرہنگ{الٹ لاپلاس بدل}\فرہنگ{لاپلاس بدل!الٹ}\حاشیہب{inverse Laplace transform}\فرہنگ{inverse Laplace transform}\فرہنگ{Laplace!inverse transform} درج ذیل مساوات دیتی ہے
\begin{align}
\Laplace^{-1}\left[\bF(s)\right]=f(t)=\frac{1}{2\pi j}\int_{\sigma_1-j\omega}^{\sigma+j\omega} \bF(s) e^{st} \dif s
\end{align}
 جہاں \عددی{\sigma_1} حقیقی ہے اور اس کی قیمت مساوات \حوالہ{مساوات_لاپلاس_شرط_بدل_پایا_جاتا_ہے} کے \عددی{\sigma} سے زیادہ ہے یعنی \عددی{\sigma_1>\sigma} ہے۔

لاپلاس بدل آسانی سے حاصل ہوتا ہے جبکہ الٹ لاپلاس بدل مشکل سے حاصل ہوتا ہے۔ہم کئی تفاعل کے لاپلاس بدل حاصل کرتے ہوئے انہیں جدول میں جوڑیوں کی صورت میں لکھیں گے اور الٹ بدل کو اسی جدول سے دیکھ کر حاصل کریں گے۔کسی بھی وقتی تفاعل \عددی{f(t)} کا منفرد لاپلاس بدل \عددی{\bF(s)} پایا جاتا ہے لہٰذا دو مختلف وقتی تفاعل \عددی{f_1(t)} اور \عددی{f_2(t)} کے لاپلاس بدل کسی بھی صورت میں یکساں نہیں ہو سکتے ہیں۔یوں کسی بھی لاپلاس بدل \عددی{\bF(s)} کو سادہ ترین اجزاء میں تقسیم کرتے ہوئے ان کے الٹ بدل کو جدول سے پڑھا جاتا ہے۔تمام اجزاء کے الٹ لاپلاس بدل کا مجموعہ درکار وقتی تفاعل ہو گا۔ہم لاپلاس بدل کو \اصطلاح{جزوی کسری پھیلاو}\فرہنگ{جزوی کسری پھیلاو}\حاشیہب{partial fraction expansion}\فرہنگ{partial fraction expansion} کے ذریعہ اجزاء میں تقسیم کریں گے۔

\حصہ{تفاعل یکتائی}
برقی ادوار میں \اصطلاح{اکائی سیڑھی تفاعل}\فرہنگ{اکائی سیڑھی تفاعل}\فرہنگ{سیڑھی!تفاعل}\حاشیہب{unit step function}\فرہنگ{unit step function} \عددی{u(t)} اور  \اصطلاح{اکائی جھٹکا تفاعل}\فرہنگ{اکائی جھٹکا تفاعل}\حاشیہب{unit impulse function}\فرہنگ{impulse!function}  \عددی{\sigma(t)} نہایت اہم ہیں۔ایسے تفاعل جو یا تو خود کہیں غیر متناہی ہوں اور یا ان کا تفرق  کہیں غیر متناہی ہو کو \اصطلاح{یکتائی تفاعل}\فرہنگ{یکتائی تفاعل}\حاشیہب{singularity function}\فرہنگ{singularity function} کہتا ہے۔ اکائی سیڑھی تفاعل اور اکائی جھٹکا تفاعل یکتائی تفاعل ہیں۔اکائی سیڑھی تفاعل پر صفحہ \حوالہصفحہ{حصہ_عارضی_دھڑکن} پر حصہ \حوالہ{حصہ_عارضی_دھڑکن} میں ہم غور کر چکے ہیں۔

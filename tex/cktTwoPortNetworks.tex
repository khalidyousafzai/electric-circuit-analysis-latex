\باب{چار سر ادوار کے ریاضی نمونے}

حصہ{فراوانی نمونہ}
شکل \حوالہ{شکل_نمونہ_ڈبہ_دور} میں چار سروں والا \اصطلاح{ڈبہ دور}\فرہنگ{ڈبہ دور}\حاشیہب{block diagram}\فرہنگ{block diagram} دکھایا گیا ہے جس کی اندرونی ساخت کے بار میں ہم کچھ نہیں جانتے۔دور کے داخلی سروں کو بائیں ہاتھ اور خارجی سروں کو دائیں ہاتھ دکھایا جاتا ہے لہٰذا \عددی{AB} داخلی اور \عددی{CD} خارجی سرے ہیں۔داخلی اور خارجی سروں پر دباو کے قطب اور رو کی سمتیں دکھائی گئی ہیں۔یوں نچلے سروں کو حوالہ سرا لیا جاتا ہے اور دونوں اطراف سے دور میں رو داخل ہوتی ہے۔
\begin{figure}
\centering
\begin{tikzpicture}
\draw(0,0) rectangle ++(\xx,\yy);
\draw(0,\yy-\yy/8) to [short,-o,i<_={$\bI_1$}]++(-\xx/2,0)node[left]{$A$};
\draw(0,\yy/8) to [short,-o]++(-\xx/2,0)node[left]{$B$};
\draw(\xx,\yy-\yy/8) to [short,-o,i<^={$\bI_2$}]++(\xx/2,0)node[right]{$C$};
\draw(\xx,\yy/8) to [short,-o]++(\xx/2,0)node[right]{$D$};
\draw(\xx/2,\yy/2) node{\RL{خطی دور}};
\draw(-\xx/2,\yy/2)node{$\begin{aligned} &+ \\ &\bV_1 \\ &- \end{aligned}$};
\draw(\xx+\xx/2,\yy/2)node{$\begin{aligned} &+ \\ &\bV_2 \\ &- \end{aligned}$};
\end{tikzpicture}
\caption{چار سروں والا دور۔}
\label{شکل_نمونہ_ڈبہ_دور}
\end{figure}

 داخلی متغیرات مثلاً \عددی{\bI_1} اور \عددی{\bV_1} کو زیر نوشت میں \عددی{1} سے ظاہر کیا جاتا ہے  جبکہ خارجی متغیرات کو زیر نوشت میں \عددی{2} سے ظاہر کیا جاتا ہے۔ڈبہ دور خطی دور ہے جس میں غیر تابع منبع نہیں پایا جاتا لہٰذا \عددی{\bI_1} اور \عددی{\bI_2} حاصل کرتے ہوئے مسئلہ خطی میل استعمال کیا جا سکتا ہے۔ یوں \عددی{\bV_1} اور \عددی{\bV_2} سے پیدا داخلی جانب رو کا مجموعہ \عددی{\bI_1} ہو گا اور اسی طرح خارجی جانب دونوں اطراف کے دباو سے پیدا رو کا مجموعہ \عددی{\bI_2} ہو گا یعنی
\begin{gather}
\begin{aligned}\label{مساوات_نمونہ_الف}
\bI_1&=y_{11}\bV_1+y_{12}\bV_2\\
\bI_2&=y_{21}\bV_1+y_{22}\bV_2
\end{aligned}
\end{gather}  
جہاں \عددی{y_{11}}، \عددی{y_{12}} وغیرہ فراوانی مستقل ہیں جنہیں سیمنز \عددی{\si{\siemens}} میں ناپا جاتا ہے۔ان مساوات کو قالب کی صورت میں لکھتے ہیں۔\عددی{y_{11}}، \عددی{y_{12}}، \عددی{y_{21}} اور \عددی{y_{22}} کو فراوانی مقدار \عددی{Y} کہتے ہیں۔اگر \عددی{Y} کی قیمتیں معلوم ہوں تب ڈبہ دور کی خارجی بالمقابل داخلی تعلقات مکمل طور پر تعین کی جا سکتی ہیں۔ 
\begin{align}\label{مساوات_نمونہ_ب}
\begin{bmatrix}
\bI_1 \\
\bI_2
\end{bmatrix}
=
\begin{bmatrix}
y_{11} & y_{12}\\
y_{21} & y_{22}
\end{bmatrix}
\begin{bmatrix}
\bV_1\\
\bV_2
\end{bmatrix}
\end{align}

مساوات \حوالہ{مساوات_نمونہ_الف} میں خارجی سروں کو قصر دور کرنے سے \عددی{\bV_2=0} ہو گا اور یوں \عددی{y_{11}} کو درج ذیل لکھا جا سکتا ہے۔
\begin{align}
y_{11}=\left. \frac{\bI_1}{\bV_1} \right|_{\bV_2=0}
\end{align}
\عددی{y_{11}} کو \اصطلاح{قصر دور داخلی فراوانی}\فرہنگ{قصر دور!داخلی فراوانی}\فرہنگ{داخلی فراوانی!قصر دور}\حاشیہب{short-circuit input admittance}\فرہنگ{admittance,short-circuit input} کہتے ہیں۔بقایا مقدار بھی اسی طرح حاصل کیے جا سکتے ہیں۔
\begin{gather}
\begin{aligned}
y_{12}&=\left. \frac{\bI_1}{\bV_2} \right|_{\bV_1=0}\\
y_{21}&=\left. \frac{\bI_2}{\bV_1} \right|_{\bV_2=0}\\
y_{22}&=\left. \frac{\bI_2}{\bV_2} \right|_{\bV_1=0}
\end{aligned}
\end{gather}
\عددی{y_{12}} اور \عددی{y_{21}} کو \اصطلاح{قصر دور فراوانی نما}\فرہنگ{قصر دور!فراوانی نما}\حاشیہب{short-circuit transadmittance}\فرہنگ{transadmittance!short-circuit} کہا جاتا ہے جبکہ \عددی{y_{22}} کو \اصطلاح{قصر دور خارجی فراوانی}\حاشیہب{short-circuit output admittance}\فرہنگ{output admittance!short-circuit} کہتے ہیں۔درج بالا مساوات کو استعمال کرتے ہوئے کسی بھی نا معلوم دور کے \عددی{Y} مقدار تجرباتی طور ناپے جا سکتے ہیں۔ مساوات \حوالہ{مساوات_نمونہ_ب} ڈبہ دور کا \اصطلاح{فراوانی نمونہ}\فرہنگ{فراوانی نمونہ}\فرہنگ{نمونہ!فراوانی}\حاشیہب{admittance model}\فرہنگ{admittance!model}\فرہنگ{model!admittance} ہے۔

 مساوات \حوالہ{مساوات_نمونہ_الف} کو شکل \حوالہ{شکل_چار_سر_فراوانی_مساوی_دور} ظاہر کرتی ہے۔یوں کسی بھی دور کے فراوانی مستقل حاصل کرنے کے بعد اس کو شکل \حوالہ{شکل_چار_سر_فراوانی_مساوی_دور} سے ظاہر کیا جا سکتا ہے۔یاد رہے کہ اس شکل میں \عددی{y_{11}} اور \عددی{y_{22}} فراوانی ہے نا کہ مزاحمت لہٰذا اوہم کا قانون لکھتے ہوئے خیال رکھیں۔
\begin{figure}
\centering
\begin{tikzpicture}[american voltages]
\draw(0,0) to [short,o-,i={$\bI_1$}]++(\x/2,0) to [short]++(\x+\x/4,0) to [american controlled current source,l_={$y_{12} \bV_2$}]++(0,-\y) to [short,-o]++(-\x-3/4*\x,0);
\draw(\x/2,0) to [european resistor,*-*,l_={$y_{11}$}]++(0,-\y);
\draw(4*\x+\x/2,0) to [short,o-,i_>={$\bI_2$}]++(-\x/2,0) to [short]++(-\x,0) to [american controlled current source,l={$y_{21}\bV_1$}]++(0,-\y) to [short,-o]++(\x+\x/2,0);
\draw(4*\x,0) to [european resistor,*-*,l={$y_{22}$}]++(0,-\y); 
\draw(0,0) to [open,v={$\bV_1$}]++(0,-\y);
\draw(4*\x+\x/2,0) to [open,v^<={$\bV_2$}]++(0,-\y);
\end{tikzpicture}
\caption{چار سر فراوانی مساوی دور۔}
\label{شکل_چار_سر_فراوانی_مساوی_دور}
\end{figure}
%============================
\ابتدا{مشق}
شکل \حوالہ{شکل_چار_سر_فراوانی_مساوی_دور} کے داخلی اور خارجی مساوات لکھ کر  مساوات \حوالہ{مساوات_نمونہ_الف} حاصل کریں۔
\انتہا{مشق}
%================
\ابتدا{مثال}\شناخت{مثال_نمونہ_مزاحمتی_دور_الف}
شکل \حوالہ{شکل_نمونہ_مزاحمتی_دور_الف} میں دور دکھایا گیا ہے۔اس کے \عددی{Y} مقدار دریافت کریں۔
\begin{figure}
\centering
\begin{subfigure}{1\textwidth}
\centering
\begin{tikzpicture}
\draw(0,0) to [short,o-,i={$\bI_1$}]++(\x,0) to [resistor,l={$\SI{4}{\ohm}$}]++(\x,0) to [short,,i<={$\bI_2$},-o]++(\x/2,0);
\draw(0,-\y) to [short,o-o]++(2*\x+\x/2,0);
\draw(\x,0) to [resistor,*-*,l={$\SI{2}{\ohm}$}]++(0,-\y);
\draw(0,-\y/2)node{$\begin{aligned}  &+ \\ &\bV_1 \\ &- \end{aligned}$};
\draw(2*\x+\x/2,-\y/2)node{$\begin{aligned}  &+ \\ &\bV_2 \\ &- \end{aligned}$};
\end{tikzpicture}
\caption*{(الف)}
\end{subfigure}
\begin{subfigure}{0.5\textwidth}
\centering
\begin{tikzpicture}
\draw(0,0) to [short,o-,i={$\bI_1$}]++(\x,0) to [resistor,l={$\SI{4}{\ohm}$}]++(\x,0) to[short,i<={$\bI_2$}]++(\x/4,0)to  [short]++(0,-\y);
\draw(0,-\y) to [short,o-]++(2*\x+\x/4,0);
\draw(\x,0) to [resistor,*-*,l={$\SI{2}{\ohm}$}]++(0,-\y);
\draw(0,-\y/2)node{$\begin{aligned}  &+ \\ &\bV_1 \\ &- \end{aligned}$};
\end{tikzpicture}
\caption*{(ب)}
\end{subfigure}%
\begin{subfigure}{0.5\textwidth}
\centering
\begin{tikzpicture}
\draw(0,0) to [short,i={$\bI_1$}]++(\x,0) to [resistor,l={$\SI{4}{\ohm}$}]++(\x,0) to [short,,i<={$\bI_2$},-o]++(\x/2,0);
\draw(0,-\y) to [short,-o]++(2*\x+\x/2,0);
\draw(\x,0) to [resistor,*-*,l={$\SI{2}{\ohm}$}]++(0,-\y);
\draw(0,0) --++(0,-\y);
\draw(2*\x+\x/2,-\y/2)node{$\begin{aligned}  &+ \\ &\bV_2 \\ &- \end{aligned}$};
\end{tikzpicture}
\caption*{(پ)}
\end{subfigure}
\begin{subfigure}{1\textwidth}
\centering
\begin{tikzpicture}
\draw(0,0) to [short,o-,i={$\bI_1$}]++(\x,0) to [resistor,l={$\SI{4}{\ohm}$}]++(\x,0) to [short,,i<={$\bI_2$},-o]++(\x/2,0);
\draw(0,-\y) to [short,o-o]++(2*\x+\x/2,0);
\draw(\x,0) to [resistor,*-*,l={$\SI{2}{\ohm}$}]++(0,-\y);
\draw(0,-\y) to [short,o-]++(-\x/2,0) to [american current source,l={$\SI{2}{\ampere}$}]++(0,\y) to [short,-o]++(\x/2,0);
\draw(2*\x+\x/2,0) to [short,o-]++(\x/2,0)  to [resistor,l={$\SI{3}{\ohm}$}]++(0,-\y) to [short,-o]++(-\x/2,0);
\draw(0,-\y/2)node{$\begin{aligned}  &+ \\ &\bV_1 \\ &- \end{aligned}$};
\draw(2*\x+\x/2,-\y/2)node{$\begin{aligned}  &+ \\ &\bV_2 \\ &- \end{aligned}$};
\end{tikzpicture}
\caption*{(ت)}
\end{subfigure}
\caption{مثال \حوالہ{مثال_نمونہ_مزاحمتی_دور_الف} کا دور۔}
\label{شکل_نمونہ_مزاحمتی_دور_الف}
\end{figure}

حل:\عددی{y_{11}} حاصل کرنے کی خاطر خارجی سروں کو قصر دور کرتے ہوئے داخلی جانب \عددی{\bV_1} مسلط کرتے ہیں۔شکل-ب میں ایسا دکھایا گیا ہے جہاں سے 
\begin{align*}
\bI_1&=\frac{\bV_1}{\frac{2\times 4}{2+4}}=\frac{3}{4} \bV_1
\end{align*}
لکھتے ہوئے
\begin{align*}
y_{11}&=\left. \frac{\bI_1}{\bV_1}\right|_{\bV_2=0}=\frac{3}{4}\, \si{\siemens}
\end{align*}
حاصل ہوتا ہے۔چونکہ \عددی{y_{11}} اور \عددی{y_{21}} کے حصول میں \عددی{\bV_2} کو قصر دور کیا جاتا ہے لہٰذا یہ دونوں شکل-ب سے حاصل ہوں گے۔دور کو دیکھ کر درج ذیل لکھا جا سکتا ہے
\begin{align*}
\bI_2=-\frac{\bV_1}{4}
\end{align*}
لہٰذا
\begin{align*}
y_{21}=\left. \frac{\bI_2}{\bV_1}\right|_{\bV_2=0}=-\frac{1}{4}\,\si{\siemens}
\end{align*}
ہو گا۔


\عددی{y_{12}} اور \عددی{y_{22}} کے حصول میں \عددی{\bV_1=0} کرنا ہو گا لہٰذا داخلی سروں کو قصر دور کرتے ہوئے شکل-پ حاصل کیا گیا ہے۔ اس میں \عددی{\SI{2}{\ohm}} کے مزاحمت کو ہٹایا جا سکتا ہے البتہ میں نے اس کو شکل میں دکھایا ہے۔اس دور سے درج ذیل لکھا جا سکتا ہے
\begin{align*}
\bI_1=-\frac{\bV_2}{4}
\end{align*} 
لہٰذا
\begin{align*}
y_{12}=\left.\frac{\bV_2}{\bI_1} \right|_{\bV_1=0}=-\frac{1}{4}\,\si{\siemens}
\end{align*}
ہو گا۔شکل-پ سے درج ذیل
\begin{align*}
\bI_2=\frac{\bV_2}{4}
\end{align*}
 لکھتے ہوئے
\begin{align*}
y_{22}=\left.\frac{\bI_2}{\bV_2} \right|_{\bV_1=0}=\frac{1}{4} \, \si{\siemens}
\end{align*}
حاصل ہوتا ہے۔

ان معلومات کو استعمال کرتے ہوئے مساوات \حوالہ{مساوات_نمونہ_الف} لکھتے ہیں
\begin{gather}
\begin{aligned}\label{مساوات_نمونہ_مثال_الف}
\bI_1&=\frac{3}{4}\bV_1-\frac{1}{4}\bV_2\\
\bI_2&=-\frac{1}{4}\bV_1+\frac{1}{4}\bV_2
\end{aligned}
\end{gather}
جنہیں قالب کی شکل میں لکھتے ہیں جو اس دور کو مکمل طور ظاہر کرتی ہے۔
\begin{align*}
\begin{bmatrix}
\bI_1\\
\bI_2
\end{bmatrix}
=
\begin{bmatrix}
\frac{3}{4} & -\frac{1}{4}\\
-\frac{1}{4} & \frac{1}{4}
\end{bmatrix}
\begin{bmatrix}
\bV_1\\
\bV_2
\end{bmatrix}
\end{align*}

اس مثال کو مکمل کرنے کی غرض سے  شکل \حوالہ{شکل_نمونہ_مزاحمتی_دور_الف}-الف کے داخلی جانب منبع رو اور خارجی جانب \عددی{\SI{3}{\ohm}} نسب کرتے ہوئے حل کرتے ہیں۔شکل-ت میں اسے دکھایا گیا ہے جہاں
\begin{align*}
\bI_1&=\SI{2}{\ampere}\\
\bV_2&=-3\bI_2
\end{align*}
ہیں۔انہیں مساوات \حوالہ{مساوات_نمونہ_مثال_الف} میں پر کرتے ہوئے 
\begin{align*}
\begin{bmatrix}
\frac{3}{4} & -\frac{1}{4}\\
-\frac{1}{4}& \frac{1}{3}+\frac{1}{4}
\end{bmatrix}
\begin{bmatrix}
\bV_1\\
\bV_2
\end{bmatrix}
=
\begin{bmatrix}
2\\
0
\end{bmatrix}
\end{align*}
ملتا ہے جو عین کرخوف مساوات جوڑ ہیں۔ان سے 
\begin{align*}
\bV_1&=\frac{28}{9}\, \si{\volt}\\
\bV_2&=\frac{4}{3} \, \si{\volt}
\end{align*}
حاصل ہوتا ہے۔ 
\انتہا{مثال}
%===================
\ابتدا{مشق}\شناخت{مشق_نمونہ_مشق_الف}
شکل \حوالہ{شکل_نمونہ_مشق_الف} میں دیے دور کے \عددی{Y} مقدار دریافت کریں۔
\begin{figure}
\centering
\begin{tikzpicture}
\draw(0,0) to [short,o-]++(\x/2,0) to [resistor,l={$\SI{40}{\ohm}$}]++(\x,0) to [short,-o]++(\x/2,0);
\draw(0,-\y) to [short,o-o]++(2*\x,0);
\draw(\x/2,0) to [resistor,*-*,l={$\SI{20}{\ohm}$}]++(0,-\y);
\draw(\x+\x/2,0) to [resistor,*-*,l={$\SI{10}{\ohm}$}]++(0,-\y);
\end{tikzpicture}
\caption{مشق \حوالہ{مشق_نمونہ_مشق_الف} کا دور۔}
\label{شکل_نمونہ_مشق_الف}
\end{figure}

جوابات:\عددی{y_{11}=\tfrac{3}{40}}، \عددی{y_{12}=-\tfrac{1}{40}}، \عددی{y_{21}=-\tfrac{1}{40}} اور \عددی{y_{22}=\tfrac{1}{8}}
\انتہا{مشق}
%==================
\ابتدا{مشق}
شکل \حوالہ{شکل_نمونہ_مشق_الف} میں داخلی جانب \عددی{\SI{3}{\ampere}} کا منبع رو نسب کیا جاتا ہے جبکہ خارجی جانب \عددی{\SI{30}{\ohm}} کا مزاحمت نسب کیا جاتا ہے۔گزشتہ مشق کے \عددی{Y} مقدار استعمال کرتے ہوئے \عددی{\bI_2} دریافت کریں۔

جواب:\عددی{\bI_2=-\tfrac{2}{9}\,\si{\ampere}}
\انتہا{مشق}
%===================

\حصہ{رکاوٹی نمونہ}
گزشتہ حصے میں ہم نے بے منبع دور  کو فراوانی نمونے سے ظاہر کیا۔اس حصے میں دور کے داخلی دباو  \عددی{\bV_1} اور خارجی دباو  \عددی{\bV_2} کو داخلی رو \عددی{\bI_1} اور خارجی رو \عددی{\bI_2} کا پیدا کردہ دباو تصور کرتے ہیں۔یوں دور کا \اصطلاح{رکاوٹی نمونہ}\فرہنگ{رکاوٹی نمونہ}\فرہنگ{نمونہ!رکاوٹی}\حاشیہب{impedance model}\فرہنگ{impedance!model}\فرہنگ{model!impedance} حاصل ہوتا ہے یعنی  
\begin{gather}
\begin{aligned}\label{مساوات_چار_سر_رکاوٹی_الف}
\bV_1&=z_{11}\bI_1+z_{12}\bI_2\\
\bV_2&=z_{21} \bI_1+z_{22} \bI_2
\end{aligned}
\end{gather}
یا
\begin{align}\label{مساوات_نمونہ_رکاوٹی}
\begin{bmatrix}
\bV_1\\
\bV_2
\end{bmatrix}
=
\begin{bmatrix}
z_{11} & z_{12}\\
z_{21}& z_{22}
\end{bmatrix}
\begin{bmatrix}
\bI_1\\
\bI_2
\end{bmatrix}
\end{align}
بالکل \عددی{Y} کی طرح \عددی{Z} مقدار تجرباتی طور حاصل کئے جا سکتے ہیں یعنی
\begin{gather}
\begin{aligned} 
z_{11}&=\left. \frac{\bV_1}{\bI_1}\right|_{\bI_2=0}\\
z_{12}&=\left. \frac{\bV_1}{\bI_2}\right|_{\bI_1=0}\\
z_{21}&=\left. \frac{\bV_2}{\bI_1}\right|_{\bI_2=0}\\
z_{22}&=\left. \frac{\bV_2}{\bI_2}\right|_{\bI_1=0}
\end{aligned}
\end{gather}
یاد رہے کہ رو کو صفر کرنے کی خاطر دور کو کھلے سر کیا جاتا ہے۔اس طرح \عددی{z_{11}} کو \اصطلاح{کھلے سر داخلی رکاوٹ}\فرہنگ{کھلے سر!داخلی رکاوٹ}\فرہنگ{داخلی رکاوٹ!کھلے سر}\حاشیہب{open-circuit input impedance}\فرہنگ{open circuit!input impedance}، \عددی{z_{12}} اور \عددی{z_{21}} کو \اصطلاح{کھلے سر رکاوٹ نما}\فرہنگ{کھلے سر!رکاوٹ نما}\فرہنگ{رکاوٹ نما!کھلے سر}\حاشیہب{open-circuit transimpedance}\فرہنگ{transimpedance!open-circuit} اور \عددی{z_{22}} کو \اصطلاح{کھلے سر خارجی رکاوٹ}\فرہنگ{خارجی رکاوٹ!کھلے سر}\فرہنگ{کھلے سر!خارجی رکاوٹ}\حاشیہب{open-circuit output impedance}\فرہنگ{output impedance!open-circuit} کہتے ہیں۔

مساوات \حوالہ{مساوات_چار_سر_رکاوٹی_الف} کو شکل \حوالہ{شکل_چار_سر_رکاوٹی_مساوی_دور} ظاہر کرتی ہے لہٰذا کسی بھی دور کے رکاوٹی مستقل کے حصول کے بعد اس کو شکل سے ظاہر کیا جا سکتا ہے۔
\begin{figure}
\centering
\begin{tikzpicture}[american voltages]
\draw(0,0) to [short,o-,i={$\bI_1$}]++(\x/4,0) to [european resistor,l={$z_{11}$}]++(\x,0);
\draw(0,-\y) to [short,o-]++(\x+\x/4,0) to [american controlled voltage source,l={$z_{12} \bI_2$}]++(0,\y);
\draw(3*\x+\x/2,0) to [short,o-,i_>={$\bI_2$}]++(-\x/4,0) to [european resistor,l_={$z_{22}$}]++(-\x,0);
\draw(3*\x+\x/2,-\y) to [short,o-]++(-\x-\x/4,0) to [american controlled voltage source,l_={$z_{21}\bI_1$}]++(0,\y);
\draw(0,0) to [open,v={$\bV_1$}]++(0,-\y);
\draw(3*\x+\x/2,0) to [open,v^<={$\bV_2$}]++(0,-\y);
\end{tikzpicture}
\caption{چار سر رکاوٹی مساوی دور۔}
\label{شکل_چار_سر_رکاوٹی_مساوی_دور}
\end{figure}




%======================
\ابتدا{مشق}
شکل \حوالہ{شکل_چار_سر_رکاوٹی_مساوی_دور} سے مساوات \حوالہ{مساوات_چار_سر_رکاوٹی_الف} حاصل کریں۔
\انتہا{مشق}
%======================
\ابتدا{مثال}\شناخت{مثال_نمونہ_پائے}
شکل \حوالہ{شکل_نمونہ_پائے}-الف کے دور کے \عددی{Z} مقدار معلوم کریں۔
\begin{figure}
\centering
\begin{subfigure}{1\textwidth}
\centering
\begin{tikzpicture}
\draw(0,0) to [short,o-,i={$\bI_1$}] ++(\x,0) to [inductor,l={$j2\,\si{\ohm}$}]++(\x,0) to [short,-o,i<_={$\bI_2$}]++(\x,0);
\draw(0,-\y) to [short,o-o]++(3*\x,0);
\draw(\x,0) to [capacitor,*-*,l_={$-j4\,\si{\ohm}$}]++(0,-\y);
\draw(2*\x,0) to [capacitor,*-*,l={$-j4\,\si{\ohm}$}]++(0,-\y);
\draw(0,-\y/2)node{$\begin{aligned} &+ \\ &\bV_1 \\ &-  \end{aligned}$};
\draw(3*\x,-\y/2)node{$\begin{aligned} &+ \\ &\bV_2 \\ &-  \end{aligned}$};
\end{tikzpicture}
\caption*{(الف)}
\end{subfigure}
\begin{subfigure}{1\textwidth}
\centering
\begin{tikzpicture}
\draw(0,0) to [short,o-,i={$\bI_1$}] ++(\x,0) to [inductor,l={$j2\,\si{\ohm}$}]++(\x,0) to [short,-o,i<_={${\bI_2=0}$}]++(\x,0);
\draw(0,-\y) to [short,o-o]++(3*\x,0);
\draw(\x,0) to [capacitor,*-*,l_={$-j4\,\si{\ohm}$}]++(0,-\y);
\draw(2*\x,0) to [capacitor,*-*,l={$-j4\,\si{\ohm}$}]++(0,-\y);
\draw(3*\x,-\y/2)node{$\begin{aligned} &+ \\ &\bV_2 \\ &-  \end{aligned}$};
\draw(0,-\y) to [short,o-]++(-\x/4,0) to [american voltage source,l={$\bV_1$}]++(0,\y) to [short,-o] ++(\x/4,0);
\end{tikzpicture}
\caption*{(ب)}
\end{subfigure}
\begin{subfigure}{1\textwidth}
\centering
\begin{tikzpicture}
\draw(0,0) to [short,o-,i={${\bI_1=0}$}] ++(\x,0) to [inductor,l={$j2\,\si{\ohm}$}]++(\x,0) to [short,-o,i<_={$\bI_2$}]++(\x,0);
\draw(0,-\y) to [short,o-o]++(3*\x,0);
\draw(\x,0) to [capacitor,*-*,l_={$-j4\,\si{\ohm}$}]++(0,-\y);
\draw(2*\x,0) to [capacitor,*-*,l={$-j4\,\si{\ohm}$}]++(0,-\y);
\draw(0,-\y/2)node{$\begin{aligned} &+ \\ &\bV_1 \\ &-  \end{aligned}$};
\draw(3*\x,-\y) to [short,o-]++(\x/4,0) to [american voltage source,l_={$\bV_2$}]++(0,\y) to [short,-o] ++(-\x/4,0);
\end{tikzpicture}
\caption*{(پ)}
\end{subfigure}
\caption{مثال \حوالہ{مثال_نمونہ_پائے} کا دور۔}
\label{شکل_نمونہ_پائے}
\end{figure}

حل:شکل \حوالہ{شکل_نمونہ_پائے}-ب میں داخلی سروں پر \عددی{\bV_1} مسلط کی گئی ہے۔ خارجی سروں کو کھلے دور رکھ کر \عددی{\bI_2=0} کیا گیا ہے۔یوں درج ذیل لکھا جا سکتا ہے
\begin{align*}
\bV_1&=\bI_1\left[\frac{-j4(j2-j4)}{-j4+j2-j4}\right]=-j\frac{4}{3}\bI_1
\end{align*}
جس سے
\begin{align*}
z_{11}&=\left. \frac{\bV_1}{\bI_1}\right|_{\bI_2=0}=\frac{-j4(j2-j4)}{-j4+j2-j4}=-j\frac{4}{3}
\end{align*}
اور
\begin{align*}
\bI_1=j\frac{3}{4}\bV_1
\end{align*}
حاصل ہوتے ہیں۔شکل-ب سے کھلے دور خارجی دباو کو تقسیم دباو کے کلیے سے حاصل کرتے ہیں۔
\begin{align*}
\bV_2&=\frac{-j4}{j2-j4}\bV_1=2\bV_1
\end{align*}
یوں درج ذیل حاصل ہوتا ہے۔
\begin{align*}
z_{21}&=\left. \frac{\bV_2}{\bI_1}\right|_{\bI_2=0}=\frac{2\bV_1}{j\frac{3}{4}\bV_1}=-j\frac{8}{3}
\end{align*}

شکل \حوالہ{شکل_نمونہ_پائے}-پ میں خارجی سروں پر دباو مسلط کرتے ہوئے داخلی سروں کو کھلے سر رکھا گیا ہے جس سے \عددی{\bI_1=0} رکھا گیا ہے۔تقسیم دباو کے کلیے سے \عددی{\bV_1} حاصل کرتے ہیں۔
\begin{align*}
\bV_1=\frac{-j4}{j2-j4} \bV_2=2\bV_2
\end{align*}
خارجی رو درج ذیل ہے۔
\begin{align*}
\bI_2&=\frac{\bV_2}{-j4}+\frac{\bV_2}{j2-j4}=j\frac{3}{4}\bV_2
\end{align*}
اس طرح
\begin{align*}
z_{12}&=\left.\frac{\bV_1}{\bI_2} \right|_{\bI_1=0}=\frac{2\bV_2}{j\frac{3}{4}\bV_2}=-j\frac{8}{3}
\end{align*}
ہو گا۔شکل-پ سے \عددی{z_{22}} لکھتے ہیں۔
\begin{align*}م
z_{22}=\left.\frac{\bV_2}{\bI_2}\right|_{\bI_1=0}=\frac{-j4(j2-j4)}{-j4+j2-j4}=-j\frac{4}{3}
\end{align*}
ان معلومات کو استعمال کرتے ہوئے شکل-الف کے دور کو درج ذیل مساوات سے ظاہر کیا جا سکتا ہے۔
\begin{align}\label{مساوات_مثال_نمونہ_رکاوٹی_الف}
\begin{bmatrix}
\bV_1 \\[2ex]
\bV_2
\end{bmatrix}
=
\begin{bmatrix}
-j\frac{4}{3} & -j\frac{8}{3}\\[2ex]
-j\frac{8}{3} & -j\frac{4}{3}
\end{bmatrix}
\begin{bmatrix}
\bI_1\\[2ex]
\bI_2
\end{bmatrix}
\end{align}
یہاں تسلی کر لیں کہ دور کے خارجی سروں کو کھلے دور رکھتے ہوئے داخلی رکاوٹ \عددی{z_{11}} ہے۔اسی طرح داخلی سروں کو کھلے سر رکھتے ہوئے خارجی سروں پر
 رکاوٹ \عددی{z_{22}} ہے۔
\انتہا{مثال}
%=========================
\ابتدا{مثال}\شناخت{مثال_نمونہ_پائے_ب}
شکل \حوالہ{شکل_نمونہ_پائے}-الف کے داخلی جانب \عددی{\bV_1=10\phase{0^{\circ}}\,\si{\volt}} نسب کرتے ہوئے خارجی جانب \عددی{\SI{6}{\ohm}} بوجھ ڈالا جاتا ہے۔اس دور کو شکل \حوالہ{شکل_نمونہ_پائے_ب} میں دکھایا گیا ہے۔دور کو حل کریں۔
\begin{figure}
\centering
\begin{tikzpicture}[american voltages]
\draw(0,0) to [short,o-,i={$\bI_1$}] ++(\x,0) to [inductor,l={$j2\,\si{\ohm}$}]++(\x,0) to [short,-o,i<_={$\bI_2$}]++(\x,0);
\draw(0,-\y) to [short,o-o]++(3*\x,0);
\draw(\x,0) to [capacitor,*-*,l_={$-j4\,\si{\ohm}$}]++(0,-\y);
\draw(2*\x,0) to [capacitor,*-*,l={$-j4\,\si{\ohm}$}]++(0,-\y);
\draw(0,-\y) to [short,o-]++(-\x/4,0) to [american voltage source,l={$10\phase{0^{\circ}}\,\si{\volt}$}]++(0,\y) to [short,-o]++(\x/4,0);
\draw(3*\x,0) to [short,o-]++(\x/4,0) to [resistor,l={$\SI{6}{\ohm}$},v={$\bV_2$}]++(0,-\y) to [short,-o]++(-\x/4,0);
\end{tikzpicture}
\caption{مثال \حوالہ{مثال_نمونہ_پائے_ب} کا دور۔}
\label{شکل_نمونہ_پائے_ب}
\end{figure}

حل:گزشتہ مثال میں دور کے \عددی{Z} مقدار حاصل کرتے ہوئے مساوات  \حوالہ{مساوات_مثال_نمونہ_رکاوٹی_الف} حاصل کی گئی۔شکل \حوالہ{شکل_نمونہ_پائے_ب} میں 
\begin{align*}
\bV_1&=10\phase{0^{\circ}}\\
\bV_2&=-6\bI_2
\end{align*}
ہیں جنہیں مساوات   \حوالہ{مساوات_مثال_نمونہ_رکاوٹی_الف} میں پر کرتے ہوئے درج ذیل ملتا ہے۔
\begin{align*}
\begin{bmatrix}
-j\frac{4}{3} & -j\frac{8}{3}\\[2ex]
-j\frac{8}{3} & 6-j\frac{4}{3}
\end{bmatrix}
\begin{bmatrix}
\bI_1\\[2ex]
\bI_2
\end{bmatrix}
=
\begin{bmatrix}
10\phase{0^{\circ}}\\[2ex]
0
\end{bmatrix}
\end{align*}
یہ مساوات کرخوف دائری مساوات ہیں جن سے درج ذیل رو حاصل ہوتی ہیں۔
\begin{align*}
\bI_1&=6.39\phase{43.8^{\circ}} \, \si{\ampere}\\
\bI_2&=2.77\phase{146.3^{\circ}}\, \si{\ampere}
\end{align*}
\انتہا{مثال}
%==========================
\ابتدا{مشق}\شناخت{مشق_نمونہ_پائے_پ}
شکل \حوالہ{شکل_نمونہ_پائے_پ} کے  رکاوٹی مقدار \عددی{Z} حاصل کریں۔
\begin{figure}
\centering
\begin{tikzpicture}
\draw(0,0) to [resistor,l={$\SI{6}{\ohm}$},o-]++(\x,0) to [resistor,l={$\SI{12}{\ohm}$},-o]++(\x,0);
\draw(0,-\y) to [short,o-o]++(2*\x,0);
\draw(\x,0) to [resistor,*-*,l={$\SI{4}{\ohm}$}]++(0,-\y);
\end{tikzpicture}
\caption{مشق \حوالہ{مشق_نمونہ_پائے_پ} کا دور۔}
\label{شکل_نمونہ_پائے_پ}
\end{figure}

جوابات:\عددی{z_{11}=10}، \عددی{z_{12}=4}، \عددی{z_{21}=4} اور \عددی{z_{22}=12}
\انتہا{مشق}
%===========================
\حصہ{دوغلائی نمونہ}
چار سر دور میں کل چار متغیرات پائے جاتے ہیں یعنی \عددی{\bV_1}، \عددی{\bV_2}، \عددی{\bI_1} اور \عددی{\bI_2} جن میں سے کسی دو کو غیر تابع اور بقایا دو کو تابع متغیرات تصور کیا جا سکتا ہے۔فراوانی نمونے  میں دباو کو غیر تابع متغیرات تصور کیا جاتا ہے جبکہ رکاوٹی  نمونے میں رو کو غیر تابع متغیرات تصور کیا جاتا ہے۔دوغلائی نمونے میں \عددی{\bI_1} اور \عددی{\bV_2} کو غیر تابع متغیرات تصور کیا جاتا ہے جبکہ \عددی{\bI_2} اور \عددی{\bV_1} کو تابع متغیرات تصور کیا جاتا ہے۔یوں \اصطلاح{دوغلائی نمونے}\فرہنگ{دوغلائی نمونہ}\فرہنگ{نمونہ!دوغلائی}\حاشیہب{hybrid model}\فرہنگ{model!hybrid}\فرہنگ{hybrid model} کو
\begin{gather}
\begin{aligned}\label{مساوات_چار_سر_دوغلائی_مساوی_الف}
\bV_1&=h_{11} \bI_1+h_{12}\bV_2\\
\bI_2&=h_{21}\bI_1+h_{22}\bV_2
\end{aligned}
\end{gather}
یا
\begin{align}
\begin{bmatrix}
\bV_1\\
\bI_2
\end{bmatrix}
=
\begin{bmatrix}
h_{11} & h_{12}\\
h_{21} & h_{22}
\end{bmatrix}
\begin{bmatrix}
\bI_1\\
\bV_2
\end{bmatrix}
\end{align}
لکھا جا سکتا ہے جہاں
\begin{gather}
\begin{aligned}
h_{11}&=\left. \frac{\bV_1}{\bI_1}\right|_{\bV_2=0}\\
h_{12}&=\left. \frac{\bV_1}{\bV_2}\right|_{\bI_1=0}\\
h_{21}&=\left. \frac{\bI_2}{\bI_1}\right|_{\bV_2=0}\\
h_{22}&=\left. \frac{\bI_2}{\bV_2}\right|_{\bI_1=0}
\end{aligned}
\end{gather}
ہیں۔\عددی{h_{11}}، \عددی{h_{12}}، \عددی{h_{21}} اور \عددی{h_{22}} بالترتیب \اصطلاح{قصر دور داخلی رکاوٹ}\فرہنگ{دوغلائی!قصر دور داخلی رکاوٹ}\حاشیہب{short-circuit input impedance}\فرہنگ{hybrid!$h_{11}$}، \اصطلاح{کھلے سر الٹ افزائش دباو}\فرہنگ{دوغلائی!کھلے سر الٹ افزائش دباو}\حاشیہب{open-circuit reverse voltage gain}\فرہنگ{hybrid!$h_{12}$}، \اصطلاح{قصر دور افزائش رو}\فرہنگ{دوغلائی!قصر دور افزائش رو}\حاشیہب{short-circuit current gain}\فرہنگ{hybrid!$h_{21}$} اور \اصطلاح{کھلے سر خارجی فراوانی}\فرہنگ{دوغلائی!کھلے سر خارجی فراوانی}\حاشیہب{open-circuit output admittance}\فرہنگ{hybrid!$h_{22}$} ہیں۔ 

مساوات \حوالہ{مساوات_چار_سر_دوغلائی_مساوی_الف} کو شکل \حوالہ{شکل_چار_سر_دوغلائی_مساوی_دور} ظاہر کرتی ہے۔چونکہ \عددی{h_{11}} فراوانی ہے لہٰذا اس شکل میں \عددی{\tfrac{1}{h_{22}}} استعمال کیا گیا ہے جو کہ رکاوٹ ہو گا۔
\begin{figure}
\centering
\begin{tikzpicture}[american voltages]
\draw(0,0) to [short,o-,i={$\bI_1$}]++(\x/4,0) to [european resistor,l={$h_{11}$}]++(\x,0);
\draw(0,-\y) to [short,o-]++(\x+\x/4,0) to [american controlled voltage source,l={$h_{12} \bV_2$}]++(0,\y);
\draw(4*\x,0) to [short,o-,i_>={$\bI_2$}]++(-\x/2,0) to [short]++(-\x,0) to [american controlled current source,l={$h_{21}\bI_1$}]++(0,-\y) to [short,-o]++(\x+\x/2,0);
\draw(3*\x+\x/2,0) to [european resistor,*-*,l={$\frac{1}{h_{22}}$}]++(0,-\y);
\draw(0,0) to [open,v={$\bV_1$}]++(0,-\y);
\draw(4*\x,0) to [open,v^<={$\bV_2$}]++(0,-\y);
\end{tikzpicture}
\caption{چار سر دوغلائی مساوی دور۔}
\label{شکل_چار_سر_دوغلائی_مساوی_دور}
\end{figure}

%=============================
\ابتدا{مشق}
شکل \حوالہ{شکل_چار_سر_دوغلائی_مساوی_دور} سے مساوات \حوالہ{مساوات_چار_سر_دوغلائی_مساوی_الف} حاصل کریں۔
\انتہا{مشق}
%===================
\ابتدا{مشق}
شکل \حوالہ{شکل_نمونہ_پائے_پ} کے دوغلائی مقدار حاصل کریں۔

جوابات:\عددی{h_{11}=9}، \عددی{h_{12}=\tfrac{1}{4}}، \عددی{h_{21}=-\tfrac{1}{4}} اور \عددی{h_{22}=\tfrac{1}{16}}
\انتہا{مشق}
%==============
\حصہ{ترسیلی نمونہ}
\اصطلاح{ترسیلی نمونے}\فرہنگ{نمونہ!ترسیلی}\فرہنگ{ترسیلی نمونہ}\حاشیہب{transmission model}\فرہنگ{model!transmission} کے مساوات درج ذیل ہیں
 \begin{gather}
\begin{aligned}
\bV_1&=A\bV_2-B\bI_2\\
\bI_1&=C\bV_2-D\bI_2
\end{aligned}
\end{gather}
جن کو قالبی صورت میں لکھتے ہیں۔
\begin{align}
\begin{bmatrix}
\bV_1\\
\bI_1
\end{bmatrix}
=
\begin{bmatrix}
A & B\\
C & D
\end{bmatrix}
\begin{bmatrix}
\bV_2\\
-\bI_2
\end{bmatrix}
\end{align}
\عددی{ABCD} کو تجرباتی طور حاصل کرنے کی ترکیب لکھتے ہیں۔
\begin{gather}
\begin{aligned}
A&=\left. \frac{\bV_1}{\bV_2}\right|_{\bI_2=0}\\
B&=\left. \frac{\bV_1}{-\bI_1}\right|_{\bV_2=0}\\
C&=\left. \frac{\bI_1}{\bV_2}\right|_{\bI_2=0}\\
D&=\left. \frac{\bI_1}{-\bI_2}\right|_{\bV_2=0}
\end{aligned}
\end{gather}
\عددی{A}، \عددی{B}، \عددی{C} اور \عددی{D} بالترتیب \اصطلاح{کھلے سر تناسب دباو}\فرہنگ{کھلے سر تناسب دباو}\فرہنگ{ABCD parameters}، \اصطلاح{منفی قصر دور رکاوٹ نما}\فرہنگ{منفی قصر دور رکاوٹ نما}، \اصطلاح{کھلے سر فراوانی نما}\فرہنگ{کھلے سر فراوانی نما} اور \اصطلاح{منفی تناسب رو}\فرہنگ{منفی تناسب رو} ہیں۔ 
%=============
\ابتدا{مثال}\شناخت{مثال_نمونہ_الف_ب_پ_ت}
شکل \حوالہ{شکل_نمونہ_الف_ب_پ_ت} میں دیے دور کے \عددی{ABCD} معلوم کریں۔
\begin{figure}
\centering
\begin{tikzpicture}
\draw(0,0) to [capacitor,o-,l={$-j2\,\si{\ohm}$},i>^={$\bI_1$}]++(\x,0) to [inductor,-o,l={$j4\,\si{\ohm}$},i^<={$\bI_2$}]++(\x,0);
\draw(0,-\y) to [short,o-o]++(2*\x,0);
\draw(\x,0)node[above]{$\bV_x$} to [resistor,*-*,l={$\SI{6}{\ohm}$}]++(0,-\y)node[ground]{};
\draw(0,-\y/2)node{$\begin{aligned} &+ \\ &\bV_1 \\ &- \end{aligned}$};
\draw(2*\x,-\y/2)node{$\begin{aligned} &+ \\ &\bV_2 \\ &- \end{aligned}$};
\end{tikzpicture}
\caption{مثال \حوالہ{مثال_نمونہ_الف_ب_پ_ت} کا دور۔}
\label{شکل_نمونہ_الف_ب_پ_ت}
\end{figure}

حل:خارجی سروں کو کھلے سر کرتے ہوئے \عددی{A} حاصل کرتے ہیں۔تقسیم دباو کے کلیے سے
\begin{align*}
\bV_2&=\frac{6}{6-j2}\bV_1
\end{align*}
لکھے ہوئے 
\begin{align*}
A&=\left.\frac{\bV_1}{\bV_2} \right|_{\bI_2=0}=1-\frac{j}{3}
\end{align*}
حاصل ہوتا ہے۔اسی طرح  خارجی سروں کو کھلے سر رکھتے ہوئے \عددی{\bI_1} کی مساوات لکھتے
\begin{align*}
\bI_1=\frac{\bV_1}{6-j2}
\end{align*} 
ہوئے
\begin{align*}
C&=\left.\frac{\bI_1}{\bV_2} \right|_{\bI_2=0}=\frac{\frac{\bV_1}{6-j2}}{\frac{6}{6-j2}\bV_1}=\frac{1}{6}
\end{align*}
حاصل ہوتا ہے۔خارجی سروں کو قصر دور کرتے ہوئے \عددی{B} اور \عددی{D} حاصل ہوں گے۔جوڑ \عددی{\bV_x} پر کرخوف مساوات رو لکھتے
\begin{align*}
\frac{\bV_x-\bV_1}{-j2}+\frac{\bV_x}{6}+\frac{\bV_x}{j4}=0
\end{align*}
ہوئے
\begin{align*}
\bV_x=\frac{6j}{2+3j}\bV_1
\end{align*}
حاصل ہوتا ہے جس سے 
\begin{align*}
\bI_2=-\frac{\bV_x}{j4}=-\frac{\frac{6j}{2+3j}\bV_1}{j4}=(-\frac{3}{13}+j\frac{9}{26})\bV_1
\end{align*}
لکھا جا سکتا ہے۔یوں
\begin{align*}
B=\left. -\frac{\bV_1}{\bI_2}\right|_{\bV_2=0}=-\frac{1}{-\frac{3}{13}+j\frac{9}{26}}=-\frac{4}{3}-j3
\end{align*}
ہو گا۔تقسیم رو سے \عددی{\bI_2} حاصل کرتے ہیں
\begin{align*}
\bI_2=-\frac{6}{6+j4}\bI_1
\end{align*}
جس سے 
\begin{align*}
D=\left. -\frac{\bI_1}{\bI_2}\right|_{\bV_2=0}=1+j\frac{2}{3}
\end{align*}
حاصل ہوتا ہے۔
\انتہا{مثال}
%=============
\حصہ{چار سر ادوار  کے باہمی جوڑ}
عموماً بڑا نظام متعدد چھوٹے حصوں پر مشتمل ہوتا ہے۔چھوٹے حصے پر مکمل توجہ دینا زیادہ آسان ہوتا ہے۔انفرادی چھوٹے حصوں کو مختلف طریقوں سے جوڑ کر مکمل نظام تخلیق دیا جاتا ہے۔ہر حصے کو چار سر دور تصور  کرتے ہوئے ہم دیکھتے ہیں کہ انہیں کس طرح آپس میں جوڑا جاتا ہے۔اس حصے میں متوازی، سلسلہ وار اور زنجیری جوڑ پر غور کیا جائے گا۔

شکل \حوالہ{شکل_نمونہ_متوازی_ادوار} میں دو عدد چار سر ادوار کو متوازی جوڑا گیا ہے۔ہم فرض کرتے ہیں کہ انہیں متوازی جوڑنے سے انفرادی حصے کی کارکردگی تبدیل نہیں ہوتی۔اس مکمل نظام کی \عددی{Y} قالب حاصل کرتے ہیں۔چونکہ
\begin{align*}
\bI_1&=\bI_{1a}+\bI_{1b}\\
\bI_2&=\bI_{2a}+\bI_{2b}
\end{align*}
ہے لہٰذا مساوات \حوالہ{مساوات_نمونہ_الف} استعمال کرتے ہوئے
\begin{align*}
\bI_1&=(y_{11a}\bV_{1a}+y_{12a}\bV_{2a})+(y_{11b}\bV_{1b}+y_{12b}\bV_{2b})\\
\bI_2&=(y_{21a}\bV_{1b}+y_{22a}\bV_{2b})+(y_{21b}\bV_{1b}+y_{22b}\bV_{2b})
\end{align*}

لکھا جا سکتا ہے۔اس میں \عددی{\bV_{1a}=\bV_{2a}=\bV_1} پر کرتے ہوئے
\begin{align*}
\bI_1&=(y_{11a}+y_{11b})\bV_{1}+(y_{12a}+y_{12b})\bV_{2}\\
\bI_2&=(y_{21a}+y_{21b})\bV_1+(y_{22a}+y_{22b})\bV_2
\end{align*}
ملتا ہے جس سے مکمل نظام کی \عددی{Y} قالب حاصل ہوتی ہے۔
\begin{align}
\begin{bmatrix}
y_{11}&y_{12}\\
y_{21}&y_{22}
\end{bmatrix}
=
\begin{bmatrix}
y_{11a}+y_{11b}& y_{12a}+y_{12b}\\
y_{21a}+y_{21b}& y_{22a}+y_{22b}
\end{bmatrix}
\end{align}
%
\begin{figure}
\centering
\begin{tikzpicture}[american voltages]
\draw(0,0) rectangle ++(\xx,\yy);
\draw(0,\yy/8) --++(-\xx/4,0)--++(-\xx/2,-\yy/2-\yy/4) to [short,*-o]++(-\xx,0)coordinate(b);
\draw(0,\yy-\yy/8)to [short,i<_={$\bI_{1a}$}]++(-\xx/4,0)--++(-\xx/2,-\yy/2-\yy/4) to [short,*-o,i<_={$\bI_1$}]++(-\xx,0)coordinate(a);
\draw(\xx,\yy/8)--++(\xx/4,0)--++(\xx/2,-\yy/2-\yy/4) to [short,*-]++(\xx,0);
\draw(\xx,\yy-\yy/8)to [short,i<={$\bI_{2a}$}]++(\xx/4,0)--++(\xx/2,-\yy/2-\yy/4) to [short,*-o,i<={$\bI_2$}]++(\xx,0)coordinate(c);
\draw(0,\yy-\yy/8) to [open,v={$\bV_{1a}$}]++(0,-3/4*\yy);
\draw(\xx,\yy-\yy/8) to [open,v^<={$\bV_{2a}$}]++(0,-3/4*\yy);
\draw(a) to [open,v={$\bV_1$}](b);
\draw(\xx/2,\yy/2)node{$\begin{aligned} & y_{11a} \quad y_{12a}\\ &y_{21a} \quad y_{22a} \end{aligned}$};
%
\draw(0,-\yy-\yy/2)coordinate(LLB) rectangle ++(\xx,\yy);
\draw(LLB)++(0,\yy/8)--++(-\xx/4,0)--++(-\xx/2,\yy/2+\yy/4);
\draw(LLB)++(0,\yy-\yy/8)to [short,i<_={$\bI_{1b}$}]++(-\xx/4,0)--++(-\xx/2,\yy/2+\yy/4);
\draw(LLB)++(\xx,\yy/8)--++(\xx/4,0)--++(\xx/2,\yy/2+\yy/4) to [short,*-o]++(\xx,0)coordinate(d);
\draw(LLB)++(\xx,\yy-\yy/8)to [short,i<={$\bI_{2b}$}]++(\xx/4,0)--++(\xx/2,\yy/2+\yy/4) to [short,*-o]++(\xx,0);
\draw(LLB)++(0,\yy-\yy/8) to [open,v={$\bV_{1b}$}]++(0,-3/4*\yy);
\draw(LLB)++(\xx,\yy-\yy/8) to [open,v^<={$\bV_{2b}$}]++(0,-3/4*\yy);
\draw(c) to [open,v={$\bV_2$}](d);
\draw(LLB)++(\xx/2,\yy/2)node{$\begin{aligned} & y_{11b} \quad y_{12b}\\ &y_{21b} \quad y_{22b} \end{aligned}$};
\end{tikzpicture}
\caption{چار سر ادوار متوازی جڑے ہیں۔}
\label{شکل_نمونہ_متوازی_ادوار}
\end{figure}

شکل \حوالہ{شکل_نمونہ_سلسلہ_وار_ادوار} میں چار سر ادوار کو سلسلہ وار جوڑا گیا ہے۔مکمل دور کی \عددی{Z} قالب درج ذیل ہے۔ 
\begin{align}
\begin{bmatrix}
z_{11}&z_{12}\\
z_{21}&z_{22}
\end{bmatrix}
=
\begin{bmatrix}
z_{11a}+z_{11b}&z_{12a}+z_{12b}\\
z_{21a}+z_{21b} & z_{22a}+z_{22b}
\end{bmatrix}
\end{align}
%=============================
\begin{figure}
\centering
\begin{tikzpicture}[american voltages]
\draw(0,0) rectangle ++(\xx,\yy);
\draw(\xx/2,\yy/2)node{$\begin{aligned} &z_{11a} \quad z_{12a}\\ & z_{21a} \quad z_{22a} \end{aligned}$};
\draw(0,\yy-\yy/8) to [short,-o,i<_={$\bI_1$}]++(-\xx,0)coordinate(a);
\draw(\xx,\yy-\yy/8) to [short,-o,i<={$\bI_2$}]++(\xx,0)coordinate(c);
\draw(0,\yy-\yy/8) to [open,v={$\bV_{1a}$}]++(0,-3/4*\yy);
\draw(\xx,\yy-\yy/8) to [open,v^<={$\bV_{2a}$}]++(0,-3/4*\yy);
%
\draw(0,-\yy-\yy/2)coordinate(LLL) rectangle ++(\xx,\yy);
\draw(LLL)++(\xx/2,\yy/2)node{$\begin{aligned} &z_{11a} \quad z_{12b}\\ & z_{21b} \quad z_{22b} \end{aligned}$};
\draw(LLL)++(0,\yy-\yy/8) to [short,i<_={$\bI_1$}]++(-\xx/2,0)--++(0,\yy/2+\yy/4)--++(\xx/2,0);
\draw(LLL)++(\xx,\yy-\yy/8) to [short,i<={$\bI_2$}]++(\xx/2,0)--++(0,\yy/2+\yy/4)--++(-\xx/2,0);
\draw(LLL)++(0,\yy/8) to [short,-o]++(-\xx,0)coordinate(b);
\draw(LLL)++(\xx,\yy/8) to [short,-o]++(\xx,0)coordinate(d);
\draw(LLL)++(0,\yy-\yy/8) to [open,v={$\bV_{1b}$}]++(0,-3/4*\yy);
\draw(LLL)++(\xx,\yy-\yy/8) to [open,v^<={$\bV_{2b}$}]++(0,-3/4*\yy);
\draw(a) to [open,v={$\bV_1$}] (b);
\draw(c) to [open,v={$\bV_2$}] (d);
\end{tikzpicture}
\caption{چار سر ادوار سلسلہ وار جڑے ہیں۔}
\label{شکل_نمونہ_سلسلہ_وار_ادوار}
\end{figure}

شکل \حوالہ{شکل_نمونہ_زنجیری_ادوار} میں چار سر ادوار زنجیری جڑے ہیں۔مکمل نظام کی مساوات  درج ذیل ہے۔
\begin{align}
\begin{bmatrix}
\bV_1\\
\bI_1
\end{bmatrix}
=
\begin{bmatrix}
A_a & B_a\\
C_a & D_a
\end{bmatrix}
\begin{bmatrix}
A_b & B_b\\
C_b & D_b
\end{bmatrix}
\begin{bmatrix}
\bV_2\\
-\bI_2
\end{bmatrix}
\end{align}
\begin{figure}
\centering
\begin{tikzpicture}[american voltages]
\draw(0,0) rectangle ++(\xx,\yy);
\draw(0,\yy-\yy/8) to [short,i_<={$\bI_{1a}$}]++(-\xx/4,0)to [short,-o,i_<={$\bI_1$}]++(-\xx/4,0);
\draw(0,\yy/8) to [short,-o]++(-\xx/2,0);
\draw(\xx,\yy-\yy/8) to [short,i^<={$\bI_{2a}$}]++(\xx/2,0) to [short,i={$\bI_{1b}$}]++(\xx/2,0);
\draw(\xx,\yy/8) to [short]++(\xx,0);
\draw(2*\xx,0) rectangle ++(\xx,\yy);
\draw(3*\xx,\yy-\yy/8) to [short,i<={$\bI_{2a}$}]++(\xx/4,0) to [short,-o,i<={$\bI_2$}]++(\xx/4,0);
\draw(3*\xx,\yy/8) to [short,-o]++(\xx/2,0);
%
\draw(-\xx/2,\yy-\yy/8) to [open,v={$\bV_1$}]++(0,-\yy+\yy/4);
\draw(0,\yy-\yy/8) to [open,v={$\bV_{1a}$}]++(0,-\yy+\yy/4);
\draw(\xx,\yy-\yy/8) to [open,v^<={$\bV_{2a}$}]++(0,-\yy+\yy/4);
\draw(2*\xx,\yy-\yy/8) to [open,v={$\bV_{1b}$}]++(0,-\yy+\yy/4);
\draw(3*\xx,\yy-\yy/8) to [open,v^<={$\bV_{2b}$}]++(0,-\yy+\yy/4);
\draw(3*\xx+\xx/2,\yy-\yy/8) to [open,v^<={$\bV_{2}$}]++(0,-\yy+\yy/4);
%
\draw(\xx/2,\yy/2)node{$\begin{aligned} &A_a \quad B_a\\ &C_a \quad D_a \end{aligned}$};
\draw(2*\xx+\xx/2,\yy/2)node{$\begin{aligned} &A_b \quad B_b\\ &C_b \quad D_b \end{aligned}$};
\end{tikzpicture}
\caption{چار سر ادوار زنجیری جڑے ہیں۔}
\label{شکل_نمونہ_زنجیری_ادوار}
\end{figure}
%=====================
\ابتدا{مثال}\شناخت{مثال_نمونہ_وائے}
شکل \حوالہ{شکل_نمونہ_وائے}-الف کی \عددی{Y} قالب حاصل کریں۔
\begin{figure}
\centering
\begin{subfigure}{1\textwidth}
\centering
\begin{tikzpicture}
\draw(0,0) to [short,o-]++(\x/4,0) to [resistor,l={$\SI{1}{\ohm}$}]++(\x,0) to [resistor,l={$\SI{2}{\ohm}$}]++(\x,0) to [short,-o]++(\x/4,0);
\draw(0,-\y) to [short,o-o]++(2*\x+\x/2,0);
\draw(\x+\x/4,0) to [resistor,*-*,l={$\SI{2}{\ohm}$}]++(0,-\y);
\draw(\x/4,0) to [short,*-]++(0,\y/2) to [inductor,l={$j4$}]++(2*\x,0) to [short,-*]++(0,-\y/2);
\end{tikzpicture}
\caption*{(الف)}
\end{subfigure}
\begin{subfigure}{0.5\textwidth}
\centering
\begin{tikzpicture}
\draw(0,0) to [short,o-]++(\x/4,0) to [resistor,l={$\SI{1}{\ohm}$}]++(\x,0) to [resistor,l={$\SI{2}{\ohm}$}]++(\x,0) to [short,-o]++(\x/4,0);
\draw(0,-\y) to [short,o-o]++(2*\x+\x/2,0);
\draw(\x+\x/4,0) to [resistor,*-*,l={$\SI{2}{\ohm}$}]++(0,-\y);
\end{tikzpicture}
\caption*{(ب)}
\end{subfigure}%
\begin{subfigure}{0.5\textwidth}
\centering
\begin{tikzpicture}
\draw(0,-\y) to [short,o-o]++(2*\x+\x/2,0);
\draw(0,0) to [inductor,l={$j4$},o-o]++(2*\x+\x/2,0);
\end{tikzpicture}
\caption*{(پ)}
\end{subfigure}
\caption{مثال \حوالہ{مثال_نمونہ_وائے} کا دور۔}
\label{شکل_نمونہ_وائے}
\end{figure}

حل:شکل-ب اور شکل-پ کو متوازی جوڑنے سے شکل-الف حاصل ہوتی ہے۔آئیں شکل-ب اور شکل-پ کے قالب حاصل کرتے ہوئے شکل-الف کی قالب دریافت کرتے ہیں۔شکل-ب کی قالب درج ذیل ہے جو با آسانی حاصل ہوتی ہے۔
\begin{align*}
\begin{bmatrix}
\frac{1}{2} & -1\\
-\frac{1}{4}& \frac{3}{2}
\end{bmatrix}
\end{align*}
 شکل-پ کی قالب درج ذیل ہے۔
\begin{align*}
\begin{bmatrix}
\frac{1}{j4}& -\frac{1}{j4}\\
-\frac{1}{j4}& \frac{1}{j4}
\end{bmatrix}
\end{align*}
یوں مکمل نظام کی قالب درج ذیل حاصل ہوتی ہے۔
\begin{align*}
\begin{bmatrix}
\frac{1}{2} +\frac{1}{j4}& -1-\frac{1}{j4}\\
-\frac{1}{4}-\frac{1}{j4}& \frac{3}{2}+\frac{1}{j4}
\end{bmatrix}
\end{align*}
آپ سے گزارش ہے شکل-الف سے یہی جواب حاصل کر کے دیکھیں۔آپ یقیناً ایسا کرنا زیادہ مشکل پائیں گے۔
\انتہا{مثال}
%=======================
\ابتدا{مثال}\شناخت{مثال_نمونہ_چھلنی_زنجیری}
شکل \حوالہ{شکل_نمونہ_چھلنی_زنجیری}-الف میں \عددی{T} دور دکھایا گیا ہے۔اس کی تین کڑیاں زنجیری جوڑنے سے شکل-ب حاصل ہوتا ہے۔شکل-ب کی \عددی{ABCD} قالب حاصل کریں۔
\begin{figure}
\centering
\begin{subfigure}{1\textwidth}
\centering
\begin{tikzpicture}
\draw(0,0) to [resistor,o-,l={$\SI{1}{\ohm}$}]++(\x,0) to [resistor,-o,l={$\SI{1}{\ohm}$}]++(\x,0);
\draw(0,-\y) to [short,o-o]++(2*\x,0);
\draw(\x,-\y) to [capacitor,*-*,l={$\SI{1}{\farad}$}]++(0,\y);
\end{tikzpicture}
\caption*{(الف)}
\end{subfigure}
\begin{subfigure}{1\textwidth}
\centering
\begin{tikzpicture}
\draw(0,0) to [resistor,o-,l={$\SI{1}{\ohm}$}]++(\x,0) to [resistor,l={$\SI{2}{\ohm}$}]++(\x,0)to [resistor,l={$\SI{2}{\ohm}$}]++(\x,0) to [resistor,-o,l={$\SI{1}{\ohm}$}]++(\x,0);
\draw(0,-\y) to [short,o-o]++(4*\x,0);
\draw(\x,-\y) to [capacitor,*-*,l={$\SI{1}{\farad}$}]++(0,\y);
\draw(2*\x,-\y) to [capacitor,*-*,l={$\SI{1}{\farad}$}]++(0,\y);
\draw(3*\x,-\y) to [capacitor,*-*,l={$\SI{1}{\farad}$}]++(0,\y);
\end{tikzpicture}
\caption*{(ب)}
\end{subfigure}
\caption{مثال \حوالہ{مثال_نمونہ_چھلنی_زنجیری} کا دور۔}
\label{شکل_نمونہ_چھلنی_زنجیری}
\end{figure}

حل:متعدد \عددی{T} ادوار کو زنجیری جوڑنے سے بہتر چھلنی کا حصول ممکن بنایا جاتا ہے۔شکل-الف کی قالب با آسانی حاصل ہوتی ہے۔اس کو پیش کرتے ہیں۔
\begin{align*}
\begin{bmatrix}
1+j\omega& 2+j\omega\\
j\omega&1+j\omega
\end{bmatrix}
\end{align*}
یوں مکمل دور کی قالب درج ذیل ہو گی۔
\begin{align*}
\begin{bmatrix}
A& B\\
C&D
\end{bmatrix}
=
\begin{bmatrix}
1+j\omega& 2+j\omega\\
j\omega&1+j\omega
\end{bmatrix}
\begin{bmatrix}
1+j\omega& 2+j\omega\\
j\omega&1+j\omega
\end{bmatrix}
\begin{bmatrix}
1+j\omega& 2+j\omega\\
j\omega&1+j\omega
\end{bmatrix}
\end{align*}
اس کو حل کرتے ہوئے درج ذیل ملتا ہے۔
\begin{align*}
\begin{bmatrix}
A& B\\
C&D
\end{bmatrix}
=
\begin{bmatrix}
1-12\omega^2+j\omega(9-4\omega^2) & 6-16\omega^2+j\omega(19-4\omega^2)\\
-8\omega^2+j\omega(3-4\omega^2)&1-12\omega^2+j\omega(9-4\omega^2)
\end{bmatrix}
\end{align*}
\انتہا{مثال}
%====================

\حصہء{سوالات}

%===============================
\ابتدا{سوال}\شناخت{سوال_چار_سر_الف}
شکل \حوالہ{شکل_سوال_چار_سر_الف}-الف کے \عددی{Y} مقدار حاصل کریں۔
\begin{figure}
\centering
\begin{subfigure}{0.5\textwidth}
\centering
\begin{tikzpicture}[american voltages]
\draw(0,0) to [short,o-,i>^={$\bI_1$}]++(\x/4,0) to [european resistor,l={$\bZ_L$}]++(\x,0) to [short,-o,i<^={$\bI_2$}]++(\x/4,0);
\draw(0,-\y) to [short,o-o]++(\x+\x/2,0);
\draw(0,0) to [open,v={$\bV_1$}]++(0,-\y);
\draw(\x+\x/2,0) to [open,v^<={$\bV_2$}]++(0,-\y);
\end{tikzpicture}
\caption*{(الف)}
\end{subfigure}%
\begin{subfigure}{0.5\textwidth}
\centering
\begin{tikzpicture}[american voltages]
\draw(0,0) to [short,o-,i>^={$\bI_1$}]++(\x/4,0) to [short]++(\x,0) to [short,-o,i<^={$\bI_2$}]++(\x/4,0);
\draw(0,-\y) to [short,o-o]++(\x+\x/2,0);
\draw(\x/2+\x/4,0) to [european resistor,*-*,l={$\bZ_L$}]++(0,-\y);
\draw(0,0) to [open,v={$\bV_1$}]++(0,-\y);
\draw(\x+\x/2,0) to [open,v^<={$\bV_2$}]++(0,-\y);
\end{tikzpicture}
\caption*{(ب)}
\end{subfigure}%
\caption{سوال \حوالہ{سوال_چار_سر_الف}  اور سوال \حوالہ{سوال_چار_سر_ب} کے ادوار۔}
\label{شکل_سوال_چار_سر_الف}
\end{figure}

جوابات: \عددی{y_{11}=\tfrac{1}{\bZ_L}}، \عددی{y_{12}=-\tfrac{1}{\bZ_L}}، \عددی{y_{21}=-\tfrac{1}{\bZ_L}}، \عددی{y_{22}=\tfrac{1}{\bZ_L}}
\انتہا{سوال}
%======= ======================
\ابتدا{سوال}\شناخت{سوال_چار_سر_ب}
شکل \حوالہ{شکل_سوال_چار_سر_الف}-ب کے \عددی{Z} مقدار حاصل کریں۔

جوابات:\عددی{z_{11}=\bZ_L}،  \عددی{z_{12}=\bZ_L}، \عددی{z_{21}=\bZ_L}، \عددی{z_{22}=\bZ_L}
\انتہا{سوال}
%==============================
\ابتدا{سوال}\شناخت{سوال_چار_سر_پ}
شکل \حوالہ{شکل_سوال_چار_سر_پ}-الف کے \عددی{Z} مقدار حاصل کریں۔
\begin{figure}
\centering
\begin{subfigure}{0.5\textwidth}
\centering
\begin{tikzpicture}[american voltages]
\draw(0,0) to [short,o-]++(\x/2,0) to [resistor,l={$\SI{9}{\ohm}$}]++(\x,0) to [short,-o]++(\x/2,0);  
\draw(0,-\y) to [short,o-o]++(\x+\x,0);
\draw(\x/2,0) to [resistor,*-*,l_={$\SI{9}{\ohm}$}]++(0,-\y);
\draw(\x+\x/2,0) to [resistor,*-*,l={$\SI{9}{\ohm}$}]++(0,-\y);
\end{tikzpicture}
\caption*{(الف)}
\end{subfigure}%
\begin{subfigure}{0.5\textwidth}
\centering
\begin{tikzpicture}[american voltages]
\draw(0,0) to [resistor,o-,l={$\SI{6}{\ohm}$}]++(\x,0) to [resistor,-o,l={$\SI{3}{\ohm}$}]++(\x,0);
\draw(0,-\y) to [short,o-o]++(2*\x,0);
\draw(\x,0) to [resistor,*-*,l={$\SI{9}{\ohm}$}]++(0,-\y);
\end{tikzpicture}
\caption*{(ب)}
\end{subfigure}%
\caption{سوال \حوالہ{سوال_چار_سر_پ}  اور سوال \حوالہ{سوال_چار_سر_ت} کے ادوار۔}
\label{شکل_سوال_چار_سر_پ}
\end{figure}

جوابات: \عددی{z_{11}=\SI{6}{\ohm}}، \عددی{z_{12}=\SI{3}{\ohm}}، \عددی{z_{21}=\SI{3}{\ohm}}، \عددی{z_{22}=\SI{6}{\ohm}}
\انتہا{سوال}
%=============================
\ابتدا{سوال}\شناخت{سوال_چار_سر_ت}
شکل \حوالہ{شکل_سوال_چار_سر_پ}-الف کے \عددی{Z} مقدار حاصل کریں۔

جوابات:\عددی{z_{11}=\SI{15}{\ohm}}، \عددی{z_{12}=\SI{9}{\ohm}}، \عددی{z_{21}=\SI{9}{\ohm}}، \عددی{z_{22}=\SI{12}{\ohm}}
\انتہا{سوال}
%==============================
\ابتدا{سوال}\شناخت{سوال_چار_سر_ٹ}
شکل \حوالہ{شکل_سوال_چار_سر_پ}-الف کے \عددی{Y} مقدار حاصل کریں۔

جوابات:\عددی{y_{11}=\tfrac{2}{9}\,\si{\siemens}}، \عددی{y_{12}=-\tfrac{1}{9}\,\si{\siemens}}، \عددی{y_{21}=-\tfrac{1}{9}\,\si{\siemens}}، \عددی{y_{22}=\tfrac{2}{9}\,\si{\siemens}}
\انتہا{سوال}
%==============================
\ابتدا{سوال}\شناخت{سوال_چار_سر_ث}
شکل \حوالہ{شکل_سوال_چار_سر_پ}-ب کے \عددی{Y} مقدار حاصل کریں۔

جوابات:\عددی{y_{11}=\tfrac{4}{33}\,\si{\siemens}}، \عددی{y_{12}=-\tfrac{1}{11}\,\si{\siemens}}، \عددی{y_{21}=-\tfrac{1}{11}\,\si{\siemens}}، \عددی{y_{22}=\tfrac{5}{33}\,\si{\siemens}}
\انتہا{سوال}
%==============================
\ابتدا{سوال}\شناخت{سوال_چار_سر_ج}
شکل \حوالہ{شکل_سوال_چار_سر_ج} میں ایمپلیفائر کا بلند تعددی مساوی دور دکھایا گیا ہے۔ اس کے \عددی{Y(s)} مقدار حاصل کریں۔
\begin{figure}
\centering
\begin{tikzpicture}[american voltages]
\draw(0,0) to [short,o-,i^>={$\bI_1$}]++(\x/2,0) to [capacitor,l={$\tfrac{1}{sC_2}$}]++(\x,0) to [short]++(2*\x,0) to [short,-o,i^<={$\bI_2$}]++(\x/2,0);
\draw(0,-\y) to [short,o-o]++(4*\x,0);
\draw(\x/2,0) to [capacitor,*-*,l={$\tfrac{1}{sC_1}$}]++(0,-\y);
\draw(\x+\x/2,0) to [american current source,*-*,l={$g\bV_1$}]++(0,-\y);
\draw(2*\x+\x/2,0) to [capacitor,*-*,l={$\tfrac{1}{sC_3}$}]++(0,-\y);
\draw(3*\x+\x/2,0) to [resistor,*-*,l={$R$}]++(0,-\y);
\draw(0,0) to [open,v={$\bV_1$}]++(0,-\y);
\draw(4*\x,0) to [open,v^<={$\bV_2$}]++(0,-\y);
\end{tikzpicture}
\caption{سوال \حوالہ{سوال_چار_سر_ج} کا دور۔}
\label{شکل_سوال_چار_سر_ج}
\end{figure}

جوابات:\عددی{y_{11}=s(C_1+C_2)}،  \عددی{y_{12}=-sC_2}، \عددی{y_{21}=g-sC_2}، \عددی{y_{11}=\tfrac{1}{R}+s(C_2+C_3)}
\انتہا{سوال}
%================================
\ابتدا{سوال}\شناخت{سوال_چار_سر_چ}
شکل \حوالہ{شکل_سوال_چار_سر_چ} میں دباو ایمپلیفائر کا پست تعددی مساوی دور دکھایا گیا ہے۔ اس کے \عددی{\bY} مقدار حاصل کریں۔
\begin{figure}
\centering
\begin{tikzpicture}[american voltages]
\draw(0,0) to [short,o-]++(\x/2,0) to [european resistor,l={$\bZ_1$}]++(0,-\y) to [short,-o]++(-\x/2,0);
\draw(\x/2,-\y) to [short,o-*]++(2*\x,0);
\draw(2*\x+\x/2,0) to [european resistor,o-,l_={$\bZ_2$}]++(-\x,0) to [american controlled voltage source,-*,l={$\bA_v \bV_1$}]++(0,-\y);
\draw(0,0) to [open,v={$\bV_1$}]++(0,-\y);
\draw(2*\x+\x/2,0) to [open,v^<={$\bV_2$}]++(0,-\y);
\end{tikzpicture}
\caption{سوال \حوالہ{سوال_چار_سر_چ} کا دور۔}
\label{شکل_سوال_چار_سر_چ}
\end{figure}

جوابات:\عددی{y_{11}=\tfrac{1}{\bZ_1}}، \عددی{y_{12}=0}، \عددی{y_{21}=\tfrac{\bA_v}{\bZ_2}}، \عددی{y_{22}=\tfrac{1}{\bZ_2}}
\انتہا{سوال}
%================================
\ابتدا{سوال}\شناخت{سوال_چار_سر_ح}
شکل \حوالہ{شکل_سوال_چار_سر_چ}  کے \عددی{\bZ} مقدار حاصل کریں۔

جوابات:\عددی{z_{11}=\bZ_1}، \عددی{z_{12}=0}، \عددی{z_{21}=-\bA_v\bZ_1}، \عددی{z_{22}=\bZ_2}
\انتہا{سوال}
%==============================
\ابتدا{سوال}\شناخت{سوال_چار_سر_خ}
شکل \حوالہ{شکل_سوال_چار_سر_خ}  کے \عددی{\bY} مقدار حاصل کریں۔
\begin{figure}
\centering
\begin{tikzpicture}[american voltages]
\draw(0,0) to [resistor,o-,l={$\SI{1}{\kilo\ohm}$},i>^={$\bI_1$}]++(\x,0) to [resistor,l_={$\SI{10}{\kilo\ohm}$}]++(\x,0) to [short,-o,i^<={$\bI_2$}]++(\x/2,0);
\draw(0,-\y) to [short,o-o]++(2*\x+\x/2,0);
\draw(\x,0) to [resistor,*-*,l_={$\SI{50}{\ohm}$}]++(0,-\y);
\draw(2*\x,0) to [short,*-]++(0,3/4*\y) to [american current source,l={$40\bI_1$}]++(-\x,0) to [short]++(0,-3/4*\y);
\draw(0,0) to [open,v={$\bV_1$}]++(0,-\y);
\draw(2*\x+\x/2,0) to [open,v^<={$\bV_2$}]++(0,-\y);
\end{tikzpicture}
\caption{سوال \حوالہ{سوال_چار_سر_خ} کا دور۔}
\label{شکل_سوال_چار_سر_خ}
\end{figure}


جوابات:\عددی{z_{11}=1050}، \عددی{z_{12}=50}، \عددی{z_{21}=\num{-399950}}، \عددی{z_{22}=\num{10050}}
\انتہا{سوال}
%==============================
\ابتدا{سوال}\شناخت{سوال_چار_سر_د}
شکل \حوالہ{شکل_سوال_چار_سر_د}-الف کے \عددی{h} مقدار حاصل کریں۔
\begin{figure}
\centering
\begin{subfigure}{0.5\textwidth}
\centering
\begin{tikzpicture}[american voltages]
\draw(0,0) to [short,o-]++(\x/2,0) to [resistor,l={$\SI{9}{\ohm}$}]++(\x,0) to [short,-o]++(\x/2,0);  
\draw(0,-\y) to [short,o-o]++(\x+\x,0);
\draw(\x/2,0) to [resistor,*-*,l_={$\SI{6}{\ohm}$}]++(0,-\y);
\draw(\x+\x/2,0) to [resistor,*-*,l={$\SI{12}{\ohm}$}]++(0,-\y);
\end{tikzpicture}
\caption*{(الف)}
\end{subfigure}%
\begin{subfigure}{0.5\textwidth}
\centering
\begin{tikzpicture}[american voltages]
\draw(0,0) to [resistor,o-,l={$\SI{6}{\ohm}$}]++(\x,0) to [resistor,-o,l={$\SI{12}{\ohm}$}]++(\x,0);
\draw(0,-\y) to [short,o-o]++(2*\x,0);
\draw(\x,0) to [resistor,*-*,l={$\SI{24}{\ohm}$}]++(0,-\y);
\end{tikzpicture}
\caption*{(ب)}
\end{subfigure}%
\caption{سوال \حوالہ{سوال_چار_سر_د}  اور سوال \حوالہ{سوال_چار_سر_ڈ} کے ادوار۔}
\label{شکل_سوال_چار_سر_د}
\end{figure}

جوابات:  \عددی{h_{11}=\tfrac{18}{5}}،  \عددی{h_{12}=\tfrac{2}{5}}،  \عددی{h_{21}=-\tfrac{2}{5}}،  \عددی{h_{22}=\tfrac{3}{20}}
\انتہا{سوال}
%===============================
\ابتدا{سوال}\شناخت{سوال_چار_سر_ڈ}
شکل \حوالہ{شکل_سوال_چار_سر_د}-ب کے \عددی{h} مقدار حاصل کریں۔

جوابات:\عددی{h_{11}=14}، \عددی{h_{12}=\tfrac{2}{3}}، \عددی{h_{21}=-\tfrac{2}{3}}، \عددی{h_{22}=\tfrac{1}{36}}
\انتہا{سوال}
%==============================
\ابتدا{سوال}\شناخت{سوال_چار_سر_ذ}
شکل \حوالہ{شکل_سوال_چار_سر_ذ}-الف کے \عددی{h} مقدار حاصل کریں۔
\begin{figure}
\centering
\begin{subfigure}{1\textwidth}
\centering
\begin{tikzpicture}[american voltages]
\draw(0,0) to [short,o-]++(\x/2,0) to [capacitor,l={$-j\,\si{\ohm}$}]++(\x,0) to [resistor,l={$\SI{1}{\ohm}$}]++(\x,0) to [short,-o]++(\x/2,0);  
\draw(0,-\y) to [short,o-o]++(3*\x,0);
\draw(\x/2,0) to [resistor,*-*,l_={$\SI{1}{\ohm}$}]++(0,-\y);
\draw(\x+\x/2,0) to [resistor,*-*,l_={$\SI{1}{\ohm}$}]++(0,-\y);
\draw(2*\x+\x/2,0) to [inductor,*-*,l_={$j1\,\si{\ohm}$}]++(0,-\y);
\end{tikzpicture}
\caption*{(الف)}
\end{subfigure}
\begin{subfigure}{1\textwidth}
\centering
\begin{tikzpicture}[american voltages]
\draw(0,0) to [short,o-]++(\x/4,0) to [resistor,l={$\SI{1}{\ohm}$}]++(\x,0) to [resistor,l={$\SI{1}{\ohm}$}]++(\x,0) to [short,-o]++(\x/4,0);
\draw(0,-\y) to [short,o-o]++(2*\x+\x/2,0);
\draw(\x+\x/4,0) to [resistor,*-*,l={$\SI{2}{\ohm}$}]++(0,-\y);
\draw(\x/4,0) to [short,*-]++(0,3/4*\y) to [capacitor,l_={$-j1$}]++(2*\x,0) to [short,-*]++(0,-3/4*\y);
\end{tikzpicture}
\caption*{(ب)}
\end{subfigure}%
\caption{سوال \حوالہ{سوال_چار_سر_ذ} اور سوال \حوالہ{سوال_چار_سر_ر} کے ادوار}
\label{شکل_سوال_چار_سر_ذ}
\end{figure}

جوابات:\عددی{h_{11}=\tfrac{7}{13}-j\tfrac{4}{13}}، \عددی{h_{12}=\tfrac{3}{13}+j\tfrac{2}{13}}، \عددی{h_{21}=-\tfrac{3}{13}-j\tfrac{2}{13}}، \عددی{h_{22}=\tfrac{8}{13}-j\tfrac{12}{13}}
\انتہا{سوال}
%===============================
\ابتدا{سوال}\شناخت{سوال_چار_سر_ر}
شکل \حوالہ{شکل_سوال_چار_سر_ذ}-ب کے \عددی{h} مقدار حاصل کریں۔

جوابات:\عددی{h_{11}=\tfrac{15}{34}-j\tfrac{25}{34}}، \عددی{h_{12}=\tfrac{31}{34}+j\tfrac{5}{34}}،
 \عددی{h_{21}=-\tfrac{31}{34}-j\tfrac{5}{34}}، \عددی{h_{22}=\tfrac{13}{34}+j\tfrac{1}{34}}
\انتہا{سوال}
%============================
\ابتدا{سوال}
شکل \حوالہ{شکل_سوال_چار_سر_الف}-الف کے \عددی{ABCD} معلوم کریں۔

جوابات:\عددی{A=1}، \عددی{B=\bZ_L}، \عددی{C=0}، \عددی{D=1}
\انتہا{سوال}
%=============================
\ابتدا{سوال}\شناخت{سوال_چار_سر_مساوی_ادوار_الف}
شکل \حوالہ{شکل_سوال_چار_سر_مساوی_ادوار_الف}-الف  کے \عددی{ABCD} معلوم کریں۔
\begin{figure}
\centering
\begin{subfigure}{0.4\textwidth}
\centering
\begin{tikzpicture}
\draw(0,0) to [capacitor,o-,l={$-j1$}]++(\x,0) to [short,-o]++(\x/2,0);
\draw(0,-\y) to [short,o-o]++(\x+\x/2,0);
\draw(\x,0) to [resistor,*-*,l_={$\SI{2}{\ohm}$}]++(0,-\y);
\end{tikzpicture}
\caption*{(الف)}
\end{subfigure}%
\begin{subfigure}{0.6\textwidth}
\centering
\begin{tikzpicture}
\draw(0,0) to [capacitor,*-*,l={$\SI{1}{\farad}$}]++(0,\y);
\draw(\x,0) to [resistor,*-*,l={$\SI{1}{\ohm}$}]++(0,\y);
\draw(2*\x,0) to [capacitor,*-*,l={$\SI{1}{\farad}$}]++(0,\y);
\draw(-\x/4,0) to [short,o-o]++(2*\x+\x/2,0);
\draw(-\x/4,\y) to [short,o-]++(\x/4,0) to [resistor,l={$\SI{1}{\ohm}$}]++(\x,0) to [resistor,l={$\SI{1}{\ohm}$}]++(\x,0) to [short,-o]++(\x/4,0);
\draw(0,\y) to [short]++(0,\y/2) to [capacitor,l={$\SI{1}{\farad}$}]++(2*\x,0) to [short]++(0,-\y/2);
\end{tikzpicture}
\caption*{(ب)}
\end{subfigure}%
\caption{سوال \حوالہ{سوال_چار_سر_مساوی_ادوار_الف} اور سوال \حوالہ{سوال_چار_سر_مساوی_ادوار_ب} کے ادوار۔}
\label{شکل_سوال_چار_سر_مساوی_ادوار_الف}
\end{figure}
جوابات:\عددی{A=1-j0.5}، \عددی{B=-j}، \عددی{C=0.5}، \عددی{D=1}
\انتہا{سوال}
%=========================
\ابتدا{سوال}\شناخت{سوال_چار_سر_مساوی_ادوار_ب}
شکل \حوالہ{شکل_سوال_چار_سر_مساوی_ادوار_الف}-ب  کے \عددی{Y} معلوم کریں۔

جوابات:\عددی{y_{11}=2s+\tfrac{2}{3}}، \عددی{y_{12}=-s-\tfrac{1}{3}}، \عددی{y_{21}=-s-\tfrac{1}{3}}، \عددی{y_{22}=2s+\tfrac{2}{3}}
\انتہا{سوال}
%===============================
\ابتدا{سوال}\شناخت{سوال_چار_سر_مساوی_ادوار_پ}
شکل \حوالہ{شکل_سوال_چار_سر_مساوی_ادوار_پ}-الف  کے \عددی{Y} حاصل کریں۔
\begin{figure}
\centering
\begin{subfigure}{0.5\textwidth}
\centering
\begin{tikzpicture}
\draw(0,0) to [capacitor,*-*,l={$\SI{1}{\farad}$}]++(0,\y);
\draw(\x,0) to [resistor,*-*,l={$\SI{1}{\ohm}$}]++(0,\y);
\draw(2*\x,0) to [capacitor,*-*,l={$\SI{1}{\farad}$}]++(0,\y);
\draw(-\x/4,0) to [short,o-o]++(2*\x+\x/2,0);
\draw(-\x/4,\y) to [short,o-]++(\x/4,0) to [resistor,l={$\SI{1}{\ohm}$}]++(\x,0) to [resistor,l={$\SI{1}{\ohm}$}]++(\x,0) to [short,-o]++(\x/4,0);
\draw(0,\y) to [short]++(0,\y/2) to [resistor,l_={$\SI{1}{\ohm}$}]++(2*\x,0) to [short]++(0,-\y/2);
\draw(0,\y+\y/2) to [short,*-]++(0,\y/2) to [capacitor,l={$\SI{1}{\farad}$}]++(2*\x,0) to [short,-*]++(0,-\y/2);
\end{tikzpicture}
\caption*{(الف)}
\end{subfigure}%
\begin{subfigure}{0.5\textwidth}
\centering
\begin{tikzpicture}
\draw(0,0) to [resistor,*-*,l={$\SI{1}{\ohm}$}]++(0,\y);
\draw(\x,0) to [resistor,*-*,l_={$\SI{1}{\ohm}$}]++(0,\y);
\draw(-\x/2,0) to [short,o-o]++(2*\x,0);
\draw(-\x/2,\y) to [short,o-]++(\x/2,0) to [inductor,l={$\SI{1}{\henry}$}]++(\x,0)to [short,-o]++(\x/2,0);
\draw(0,\y) to [short]++(0,\y/2) to [resistor,l={$\SI{1}{\ohm}$}]++(\x,0) to [short]++(0,-\y/2);
\end{tikzpicture}
\caption*{(ب)}
\end{subfigure}%
\caption{سوال \حوالہ{سوال_چار_سر_مساوی_ادوار_پ} اور سوال \حوالہ{سوال_چار_سر_مساوی_ادوار_ت} کے ادوار۔}
\label{شکل_سوال_چار_سر_مساوی_ادوار_پ}
\end{figure}

جوابات:\عددی{y_{11}=2s+\tfrac{5}{3}}، \عددی{y_{12}=-s-\tfrac{4}{3}}، \عددی{y_{21}=-s-\tfrac{4}{3}}، \عددی{y_{22}=2s+\tfrac{5}{3}}
\انتہا{سوال}
%========================
\ابتدا{سوال}\شناخت{سوال_چار_سر_مساوی_ادوار_ت}
شکل \حوالہ{شکل_سوال_چار_سر_مساوی_ادوار_پ}-ب  کے \عددی{Y} حاصل کریں۔

جوابات:\عددی{y_{11}=\tfrac{1}{s}+2}، \عددی{y_{12}=-\tfrac{1}{s}-1}، \عددی{y_{21}=-1\tfrac{1}{s}-1}، \عددی{y_{22}=\tfrac{1}{s}+2}
\انتہا{سوال}
%==========================

\باب{جوڑ اور دائری تجزیہ}
گزشتہ باب میں سادہ ترین ادوار کو کرچاف قوانین سے حل کرنا دکھایا گیا۔اس باب میں متعدد جوڑ اور متعدد دائروں والے ادوار کو کرچاف قوانین سے حل کرنا دکھایا جائے گا۔کرچاف قانون رو سے ہر جوڑ پر داخلی اور خارجی رو کے مجموعوں کو برابر پر کرتے ہوئے دور کے تمام جوڑوں پر دباو حاصل کیا جاتا ہے۔اس کے برعکس کرچاف قانون دباو کی مدد سے دور کے ہر دائرے میں دباو کے گھٹاو کے مجموعے کو دائرے میں دباو کے  بڑھاو کے مجموعے کے برابر پر کرتے ہوئے تمام دائروں کی رو حاصل کی جاتی ہے۔عموماً  دور یا تو کرچاف قانون دباو اور یا کرچاف قانون رو سے زیادہ آسانی سے حل ہوتا ہے۔آسان طریقہ چننا اس باب میں سکھایا جائے گا۔

\حصہ{تجزیہ جوڑ}

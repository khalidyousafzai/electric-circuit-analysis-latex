\باب{ترکیب جوڑ اور دائری ترکیب}
گزشتہ باب میں سادہ ترین ادوار کو کرخوف قوانین سے حل کرنا دکھایا گیا۔اس باب میں متعدد جوڑ اور متعدد دائروں والے ادوار کو کرخوف قوانین سے حل کرنا دکھایا جائے گا۔کرخوف قانون رو سے ہر جوڑ پر داخلی اور خارجی رو کے مجموعہ کو برابر پر کرتے ہوئے دور کے تمام جوڑوں پر دباو حاصل کیا جاتا ہے۔اس کے برعکس کرخوف قانون دباو کی مدد سے دور کے ہر دائرے میں دباو کے گھٹاو کے مجموعے کو دائرے میں دباو کے  بڑھنے کے مجموعے کے برابر پر کرتے ہوئے تمام دائروں کی رو حاصل کی جاتی ہے۔عموماً  دور یا تو کرخوف قانون دباو اور یا کرخوف قانون رو سے زیادہ آسانی سے حل ہوتا ہے۔آسان طریقہ چننا اس باب میں سکھایا جائے گا۔

\حصہ{تجزیہ جوڑ}
دور کو \اصطلاح{ترکیب جوڑ}\فرہنگ{جوڑ!ترکیب}\حاشیہب{nodal analysis}\فرہنگ{nodal analysis} سے حل کرتے ہوئے  جوڑ کے دباو کو  نا معلوم متغیرات چنا جاتا ہے۔کسی ایک جوڑ کو حوالہ چنتے ہوئے بقایا جوڑ کے دباو اس جوڑ سے ناپے جاتے ہیں۔یوں جس جوڑ کو حوالہ چنا گیا ہو، اس کے دباو کو صفر وولٹ تصور کیا جاتا ہے اور اس جوڑ کو  \اصطلاح{برقی زمین} کہا جاتا ہے۔عموماً اس جوڑ کو برقی زمین چنا جاتا ہے جس کے ساتھ سب سے زیادہ پرزے جڑے ہوں۔عموماً آلات کو موصل ڈبوں میں بند رکھا جاتا ہے اور عام طور دور کے برقی زمین کو ڈبے کے ساتھ جوڑا جاتا ہے۔ایسی صورت میں ڈبے کی سطح  بھی \عددی{\SI{0}{\volt}} پر ہوتی ہے۔

ہم دباو جوڑ کے متغیرات کو مثبت تصور کریں گے۔حقیقی دباو کی قیمت زمین کی نسبت سے منفی ہونے کی صورت میں تجزیے سے منفی قیمت حاصل ہو گی۔ 

\begin{figure}
\centering
\includegraphics{figNodalCurrentsFromNodeVoltages}
\caption{دباو جوڑ سے بازو کی رو حاصل کی جا سکتی ہے۔}
\label{شکل_جوڑ_دباو__جوڑ_سے_رو_کا_حصول}
\end{figure}%

آئیں دباو جوڑ جاننے کی افادیت کو  شکل \حوالہ{شکل_جوڑ_دباو__جوڑ_سے_رو_کا_حصول} کی مدد سے جانیں۔اس دور میں \عددی{a}، \عددی{b}، \عددی{c} اور \عددی{z} جوڑ پائے جاتے ہیں۔ہم نے جوڑ \عددی{z} کو برقی زمین چنا ہے لہٰذا اس کا دباو \عددی{\SI{0}{\volt}} ہے۔بقایا تین جوڑ کے دباو کو شکل میں دکھایا گیا ہے۔برقی زمین کو علامت سے ظاہر کیا گیا ہے۔

بالائی بائیں مزاحمت پر دباو درج ذیل پایا جاتا ہے
\begin{align*}
V_{ab}&=V_a-V_b\\
&=8-5\\
&=\SI{3}{\volt}
\end{align*}
لہٰذا قانون اوہم سے مزاحمت میں رو درج ذیل حاصل کی جاتی ہے۔
\begin{align*}
i_1&=\frac{V_{ab}}{\SI{2}{\kilo\ohm}}\\
&=\frac{3}{2000}\\
&=\SI{1.5}{\milli\ampere}
\end{align*}
اسی طرح بالائی دائیں مزاحمت پر دباو  درج ذیل ہو گا
\begin{align*}
V_{bc}&=V_b-V_c\\
&=5-4\\
&=\SI{1}{\volt}
\end{align*}
جس سے رو
\begin{align*}
i_2&=\frac{V_{bc}}{\SI{1}{\kilo\ohm}}\\
&=\frac{1}{1000}\\
&=\SI{1}{\milli\ampere}
\end{align*}
حاصل ہوتی ہے۔درمیانے مزاحمت پر دباو اور اس کی رو درج ذیل ہیں۔
\begin{align*}
V_{bz}&=V_b-V_z\\
&=5-0\\
&={\SI{5}{\volt}}\\
i_4&=\frac{V_{bz}}{\SI{10}{\kilo\ohm}}\\
&=\frac{5}{10000}\\
&=\SI{0.5}{\milli\ampere}
\end{align*}
چونکہ \عددی{\SI{1}{\kilo\ohm}} اور \عددی{\SI{4}{\kilo\ohm}} سلسلہ وار جڑے ہیں لہٰذا \عددی{\SI{4}{\kilo\ohm}} میں بھی \عددی{\SI{1}{\milli\ampere}} رو پائی جائے گی۔آپ اسی قیمت کو دباو جوڑ سے بھی حاصل کر سکتے ہیں یعنی
\begin{align*}
V_{cz}&=V_c-V_z\\
&=4-0\\
&=\SI{4}{\volt}\\
i_3&=\frac{V_{cz}}{\SI{4}{\kilo\ohm}}\\
&=\frac{4}{4000}\\
&=\SI{1}{\milli\ampere}
\end{align*}

یہاں اتمنان کر لیں کہ تمام جوڑوں پر آمدی رو اور خارجی رو برابر ہوں۔جوڑ \عددی{b} پر آمدی رو \عددی{\SI{1.5}{\milli\ampere}} ہے جو خارجی رو کے مجموعے \عددی{\SI{1}{\milli\ampere}+\SI{0.5}{\milli\ampere}} کے عین برابر ہے۔اسی طرح جوڑ \عددی{c} پر آمدی اور خارجی رو \عددی{\SI{1}{\milli\ampere}} ہیں۔جوڑ \عددی{a} پر کرخوف قانون رو سے منبع دباو کے مثبت سرے سے خارجی رو \عددی{\SI{1.5}{\milli\ampere}} حاصل ہوتی ہے۔

کسی بھی دو جوڑ \عددی{m} اور \عددی{n} کے مابین جڑی مزاحمت \عددی{R_{mn}} کی رو \عددی{i_R} قانون اوہم
\begin{align}\label{مساوات-جوڑ_قانون_اوہم}
i_R=\frac{v_m-v_n}{R_{mn}}
\end{align}
سے حاصل کی جاتی ہے۔

اب جب ہم دباو جوڑ کی افادیت جان چکے ہیں آئیں ترکیب جوڑ پر غور کریں۔اگر دور میں \عددی{J} جوڑ پائے جاتے ہوں تب ہمیں \عددی{J} دباو دریافت کرنے ہوں گے۔کسی ایک جوڑ کو زمین چنتے ہوئے اس کا دباو \عددی{\SI{0}{\volt}} تصور کیا جاتا ہے۔یوں بقایا \عددی{J-1} جوڑ کے دباو کو نا معلوم متغیرات تصور کیا جاتا ہے۔ان \عددی{J-1} جوڑ پر کرخوف قانون رو کا اطلاق کرتے ہوئے \عددی{J-1} مساوات لکھے جاتے ہیں۔آپ جانتے ہیں ہیں کہ \عددی{J-1} متغیرات معلوم کرنے کی خاطر \عددی{J-1} ہمزاد مساوات درکار ہیں۔یوں ان \عددی{J-1} ہمزاد مساوات کے حل سے تمام نا معلوم دباو جوڑ حاصل ہوتے ہیں۔کسی بھی جوڑ پر کروخوف کی مساوات لکھتے ہوئے جوڑ سے منسلک تمام بازو کی رو کو مساوات \حوالہ{مساوات-جوڑ_قانون_اوہم} کی طرز پر لکھا جاتا ہے۔یوں مزاحمت جانتے ہوئے، رو کو نا معلوم دباو کی صورت میں لکھا جاتا ہے۔اس طرح کرخوف قانون رو کی مساوات میں صرف نا معلوم دباو بطور متغیرات پائے جائیں گے۔

 یاد رہے کہ برقی دباو دو نقطوں کے مابین ہوتا ہے۔کسی نقطے کی حتمی دباو کوئی معنی نہیں رکھتی۔جوڑ پر کرخوف قانون رو کی مساوات لکھتے ہوئے جوڑ کا دباو زمین کے حوالے سے ناپا جاتا ہے۔ یوں شکل \حوالہ{شکل_جوڑ_دباو__جوڑ_سے_رو_کا_حصول} میں جوڑ \عددی{a} کا دباو جوڑ \عددی{z} کے حوالے سے \عددی{\SI{8}{\volt}} ہے اور جوڑ \عددی{b} کا دباو جوڑ \عددی{z} کے حوالے سے \عددی{\SI{5}{\volt}} ہے۔اس کے برعکس جوڑ \عددی{b} کے حوالے سے جوڑ \عددی{a} کا دباو \عددی{\SI{3}{\volt}} ہے جبکہ جوڑ \عددی{a} کے حوالے سے جوڑ \عددی{c} کا دباو \عددیء{\SI{-4}{\volt}} اور جوڑ \عددی{z} کا دباو \عددیء{\SI{-8}{\volt}} ہے۔ 


آئیں ترکیب جوڑ کو چند مثالوں کی مدد سے سیکھیں۔ہم آسان ترین مثال سے شروع کرتے ہوئے بتدریج مشکل مثال پیش کریں گے۔

\حصہ{غیر تابع منبع رو استعمال کرنے والے ادوار}
شکل \حوالہ{شکل_جوڑ_تین_جوڑ} میں تین جوڑ والا دور دکھایا گیا ہے جن میں نچلے جوڑ کو زمین چنا گیا ہے۔بقایا دو جوڑ کے نا معلوم برقی دباو کو متغیرات \عددی{v_1} اور \عددی{v_2} ظاہر کرتے ہیں۔ہم تمام شاخوں میں رو کی سمت چنتے ہیں۔یوں \عددی{i_1} کو بالائی بائیں جوڑ سے زمین کی جانب رواں چنا گیا ہے۔اسی طرح \عددی{i_2} کو بالائی بائیں جوڑ سے بالائی دائیں جوڑ کی جانب رواں چنا گیا ہے جبکہ \عددی{i_3} کو بالائی دائیں جوڑ سے زمین کی طرف رواں چنا گیا ہے۔
\begin{figure}
\centering
\includegraphics{figNodalIndependantSourcesNodalEquationsA}
\caption{تین جوڑ والا دور۔}
\label{شکل_جوڑ_تین_جوڑ}
\end{figure}%

بالائی بائیں جوڑ پر کرخوف قانون رو کی مساوات لکھتے ہیں۔جوڑ سے خارجی رو کو مثبت اور داخلی رو کو منفی لکھتے ہوئے درج ذیل لکھا جا سکتا ہے۔
\begin{align}\label{مساوات_جوڑ_پہلی_مثال_الف}
i_1-i_A+i_2&=0
\end{align}
قانون اوہم استعمال کرتے ہوئے اسے یوں
\begin{align*}
\frac{v_1}{R_1}-i_A+\frac{v_1-v_2}{R_2}=0
\end{align*}
یا 
\begin{align}\label{مساوات_جوڑ_پہلی_مثال_ب}
\left( \frac{1}{R_1}+\frac{1}{R_2}\right) v_1 - \frac{v_2}{R_2}=i_A
\end{align}
لکھا جا سکتا ہے۔بالائی دائیں جوڑ کے لئے
\begin{align}\label{مساوات_جوڑ_پہلی_مثال_پ}
-i_2+i_3+i_B&=0
\end{align}
اور
\begin{align*}
-\left(\frac{v_1-v_2}{R_2}\right)+\frac{v_2}{R_3}+i_B&=0
\end{align*}
یعنی
\begin{align}\label{مساوات_جوڑ_پہلی_مثال_ت}
-\frac{v_1}{R_2}+\left(\frac{1}{R_2}+\frac{1}{R_3}\right)v_2=-i_B
\end{align}
لکھا جائے گا۔نچلے جوڑ یعنی برقی زمین پر  کرخوف قانون رو کی مساوات لکھتے ہیں۔
\begin{align}\label{مساوات_جوڑ_پہلی_مثال_ٹ}
-i_1+i_A-i_3-i_B=0
\end{align}
مساوات \حوالہ{مساوات_جوڑ_پہلی_مثال_الف} اور مساوات \حوالہ{مساوات_جوڑ_پہلی_مثال_پ} کے مجموعے کو منفی ایک سے ضرب دینے سے مساوات \حوالہ{مساوات_جوڑ_پہلی_مثال_ٹ} حاصل ہوتا ہے۔مساوات \حوالہ{مساوات_جوڑ_پہلی_مثال_الف}، مساوات \حوالہ{مساوات_جوڑ_پہلی_مثال_پ} اور  مساوات \حوالہ{مساوات_جوڑ_پہلی_مثال_ٹ} میں کسی بھی دو مساواتوں سے تیسری مساوات حاصل کی جا سکتی ہے۔یوں ان میں صرف دو عدد مساوات آزاد مساوات ہیں جبکہ تیسری مساوات تابع مساوات ہے۔شکل \حوالہ{شکل_جوڑ_تین_جوڑ} کے دور میں کل تین عدد جوڑ ہیں۔آپ نے دیکھا کہ اس دور سے صرف دو عدد آزاد مساوات حاصل ہوتے ہیں یعنی \عددی{J=3} کی صورت میں \عددی{J-1=2} آزاد مساوات حاصل ہوتے ہیں۔


مساوات \حوالہ{مساوات_جوڑ_پہلی_مثال_ب} اور مساوات \حوالہ{مساوات_جوڑ_پہلی_مثال_ت} کو ایک ساتھ لکھتے ہیں۔
\begin{gather}
\begin{aligned}\label{مساوات_جوڑ_آزاد_مساوات_دو_جوڑ}
\left( \frac{1}{R_1}+\frac{1}{R_2}\right) v_1 - \frac{v_2}{R_2}&=i_A\\
-\frac{v_1}{R_2}+\left(\frac{1}{R_2}+\frac{1}{R_3}\right)v_2&=-i_B
\end{aligned}
\end{gather}
%================
\ابتدا{مثال}
شکل \حوالہ{شکل_جوڑ_تین_جوڑ} میں \عددی{i_A=\SI{2}{\milli\ampere}}، \عددی{i_B=\SI{5}{\milli\ampere}}، \عددی{R_1=\SI{4}{\kilo\ohm}}، \عددی{R_2=\SI{6}{\kilo\ohm}} اور \عددی{R_3=\SI{2}{\kilo\ohm}} ہیں۔تمام جوڑ پر دباو اور تمام شاخوں میں رو حاصل کریں۔

حل:مساوات \حوالہ{مساوات_جوڑ_آزاد_مساوات_دو_جوڑ} میں قیمتیں پُر کرتے ہیں۔
\begin{align*}
\left( \frac{1}{4000}+\frac{1}{6000}\right) v_1 - \frac{v_2}{6000}&=0.002\\
-\frac{v_1}{6000}+\left(\frac{1}{6000}+\frac{1}{2000}\right)v_2&=-0.005
\end{align*}
ان کی سادہ ترین صورت حاصل کرتے اور ترتیب دیتے ہوئے دوبارہ لکھتے ہیں۔
\begin{align*}
5v_1-2v_2&=24\\
-v_1+4v_2&=-30
\end{align*}
ان ہمزاد مساوات کو حل کرتے ہیں۔ایسا کرنے کی خاطر پہلی مساوات سے  \عددی{v_1=\tfrac{24+2v_2}{5}} لکھتے ہوئے دوسری مساوات میں پُر کر کے
\begin{align*}
-\left(\frac{24+2v_2}{5}\right)+4v_2=-30
\end{align*}
حل کرتے ہیں۔
\begin{align*}
v_2=\SI{-7}{\volt}
\end{align*} 
اس قیمت کو \عددی{v_1=\tfrac{24+2v_2}{5}} میں پُر کرنے سے درج ذیل ملتا ہے۔
\begin{align*}
v_1&=\SI{2}{\volt}
\end{align*}
دباو جوڑ جانتے ہوئے شاخوں کی رو قانون اوہم سے حاصل کرتے ہیں۔
\begin{align*}
i_1&=\frac{v_1}{R_1}=\frac{2}{4000}=\SI{0.5}{\milli\ampere}\\
i_2&=\frac{v_1-v_2}{R_2}=\frac{2-(-7)}{6000}=\SI{1.5}{\milli\ampere}\\
i_3&=\frac{v_2}{R_3}=\frac{-7}{2000}=\SI{-3.5}{\milli\ampere}
\end{align*}
\انتہا{مثال}
%=================


مساوات \حوالہ{مساوات_جوڑ_آزاد_مساوات_دو_جوڑ} کو \اصطلاح{قالبی مساوات}\فرہنگ{قالبی مساوات}\حاشیہب{matrix equation}\فرہنگ{matrix equation} کی صورت میں لکھتے ہیں۔
\begin{equation}\label{مساوات_جوڑ_دو_جوڑ_تشاکل_الف}
\begin{bmatrix}
\frac{1}{R_1}+\frac{1}{R_2} & - \frac{1}{R_2}\\[6pt]
-\frac{1}{R_2}& \frac{1}{R_2}+\frac{1}{R_3}
\end{bmatrix}
\begin{bmatrix}
v_1 \\[6pt]
v_2
\end{bmatrix}
=
\begin{bmatrix}
i_A\\[6pt]
-i_B
\end{bmatrix}
\end{equation}
قالبی مساوات میں
\begin{align*}
{\bf{G}}&=
\begin{bmatrix}
\frac{1}{R_1}+\frac{1}{R_2} & - \frac{1}{R_2}\\[6pt]
-\frac{1}{R_2}& \frac{1}{R_2}+\frac{1}{R_3}
\end{bmatrix}\\
{\bf{V}}&=
\begin{bmatrix}
v_1 \\
v_2
\end{bmatrix}\\
{\bf{I}}&=\begin{bmatrix}
i_A\\
-i_B
\end{bmatrix}
\end{align*}
لیتے ہوئے اسے یوں لکھا جا سکتا ہے
\begin{align*}
{\bf{G V=I}}
\end{align*}
جس سے
\begin{align*}
{\bf{V=G^{-1}I}}
\end{align*}
حاصل ہوتا ہے لہٰذا
\begin{equation}\label{مساوات_جوڑ_قالبی_حل_دو_جوڑ}
\begin{bmatrix}
v_1 \\[6pt]
v_2
\end{bmatrix}
=
\begin{bmatrix}
\frac{1}{R_1}+\frac{1}{R_2} & - \frac{1}{R_2}\\[6pt]
-\frac{1}{R_2}& \frac{1}{R_2}+\frac{1}{R_3}
\end{bmatrix}^{-1}
\begin{bmatrix}
i_A\\[6pt]
-i_B
\end{bmatrix}
\end{equation}
لکھا جائے گا۔

آج کل کمپیوٹر کا زمانہ ہے۔کمپیوٹر کی مدد سے قالبی مساوات نہایت آسانی سے حل کئے جا سکتے ہیں۔آپ سے التماس ہے کہ کمپیوٹر پر قالبی مساوات حل کرنا سیکھیں۔
%============
\ابتدا{مثال}
درج بالا مثال میں تمام دباو جوڑ کو مساوات \حوالہ{مساوات_جوڑ_قالبی_حل_دو_جوڑ} کی مدد سے حل کریں۔

حل:مساوات \حوالہ{مساوات_جوڑ_قالبی_حل_دو_جوڑ} میں دی معلومات پر کرتے ہوئے لکھتے ہیں۔
\begin{equation*}
\begin{bmatrix}
v_1\\[6pt]
v_2
\end{bmatrix}
=
\begin{bmatrix}
\frac{1}{2400} & -\frac{1}{6000}\\[6pt]
-\frac{1}{6000}& \frac{1}{1500}
\end{bmatrix}^{-1}
\begin{bmatrix}
0.002\\[6pt]
-0.005
\end{bmatrix}
\end{equation*}
قالب \عددی{{\bf{G}}} کا \اصطلاح{معکوس}\فرہنگ{معکوس}\فرہنگ{معکوس}\حاشیہب{inverse}\فرہنگ{inverse} \عددی{{\bf{G}}^{-1}} حاصل کرنے کی خاطر \عددی{{\bf{G}}} کا \اصطلاح{تبدیل محل قالب}\فرہنگ{تبدیل محل قالب}\حاشیہب{transpose matrix}\فرہنگ{transpose matrix} \عددی{\bf{G}^T} 
\begin{align*}
{\bf{G}}^T=
\begin{bmatrix}   
 \frac{1}{1500}& \frac{1}{6000}\\[6pt]
\frac{1}{6000}& \frac{1}{2400}
\end{bmatrix}
\end{align*}
اور  \اصطلاح{مقطع قالب}\فرہنگ{مقطع قالب}\حاشیہب{determinant}\فرہنگ{determinant}
\begin{align*}
\begin{vmatrix}
\frac{1}{2400} & -\frac{1}{6000}\\[6pt]
-\frac{1}{6000}& \frac{1}{1500}
\end{vmatrix}
&=\left(\frac{1}{2400} \right)\left( \frac{1}{1500}\right) -\left(-\frac{1}{6000} \right) \left(- \frac{1}{6000}\right)\\
&=\frac{1}{4\times 10^{6}}
\end{align*}
درکار ہوں گے۔یوں
\begin{align*}
\begin{bmatrix}
v_1\\
v_2
\end{bmatrix}
&=
4\times 10^{6}
\begin{bmatrix}
\frac{1}{1500}& \frac{1}{6000}\\[6pt]
\frac{1}{6000}& \frac{1}{2400}
\end{bmatrix}
\begin{bmatrix}
0.002\\[6pt]
-0.005
\end{bmatrix}\\
&=
4 \times 10^{6}
\begin{bmatrix}
0.5\times 10^{-6}\\[6pt]
-1.75\times 10^{-6}
\end{bmatrix}\\
&=
\begin{bmatrix}
2 \\
-7
\end{bmatrix}
\end{align*}
حاصل ہوتے ہیں یعنی \عددی{v_1=\SI{2}{\volt}} اور \عددی{v_2=\SI{-7}{\volt}} ہیں۔
\انتہا{مثال}
%==========================

آئیں شکل \حوالہ{شکل_جوڑ_چار_جوڑ_تین_آزاد_مساوات_الف} کے کرخوف قانون رو  کے مساوات لکھیں۔دور کے تمام شاخوں میں رو کی سمتیں چنی گئی ہیں۔نچلے جوڑ کو زمین چنا گیا ہے اور یہی حقیقت زمین کی علامت سے ظاہر کی گئی ہے۔دور میں کل چار \عددی{(J=4)} عدد جوڑ ہیں لہٰذا اس سے تین \عددی{(J-1=3)} عدد آزاد مساوات حاصل کئے جائیں گے۔پہلی جوڑ پر کرخوف قانون رو استعمال کرتے ہوئے
\begin{align*}
i_1+i_2+i_3+i_A=0
\end{align*}
لکھا جائے گا جہاں جوڑ سے خارج رو کو مثبت لکھا گیا ہے۔انفرادی شاخ کی رو کو قانون اوہم سے پُر کرتے ہوئے
\begin{align*}
\frac{v_1}{R_1}+\frac{v_1-v_2}{R_2}+\frac{v_1-v_3}{R_3}+i_A=0
\end{align*}
یعنی
\begin{align}\label{مساوات_جوڑ_چار_جوڑ_تین_آزاد_الف}
\left(\frac{1}{R_1}+\frac{1}{R_2}+\frac{1}{R_3}\right)v_1-\frac{v_2}{R_2}-\frac{v_3}{R_3}=-i_A
\end{align}
حاصل ہوتا ہے۔
\begin{figure}
\centering
\includegraphics{figNodalEquationNotDependentOnChosenCurrentDirectionsA}
\caption{چار جوڑ کے دور سے تین عدد آزاد مساوات حاصل ہوتے ہیں۔}
\label{شکل_جوڑ_چار_جوڑ_تین_آزاد_مساوات_الف}
\end{figure}%
دوسرے جوڑ سے
\begin{align*}
-i_2+i_4+i_5=0
\end{align*}
یعنی
\begin{align*}
-\left(\frac{v_1-v_2}{R_2}\right)+\frac{v_2}{R_4}+\frac{v_2-v_3}{R_5}=0
\end{align*}
یا
\begin{align}\label{مساوات_جوڑ_چار_جوڑ_تین_آزاد_ب}
-\frac{v_1}{R_2}+\left(\frac{1}{R_2}+\frac{1}{R_4}+\frac{1}{R_5}\right)v_2-\frac{v_3}{R_5}=0
\end{align}
حاصل ہوتا ہے۔تیسری جوڑ سے
\begin{align*}
-i_3-i_5-i_B=0
\end{align*}
یعنی
\begin{align*}
-\left(\frac{v_1-v_3}{R_3}\right)-\left(\frac{v_2-v_3}{R_5}\right)-i_B=0
\end{align*}
یا
\begin{align}\label{مساوات_جوڑ_چار_جوڑ_تین_آزاد_پ}
-\frac{v_1}{R_3}-\frac{v_2}{R_5}+\left(\frac{1}{R_3}+\frac{1}{R_5}\right)v_3=i_B
\end{align}
حاصل ہوتا ہے۔

مساوات \حوالہ{مساوات_جوڑ_چار_جوڑ_تین_آزاد_الف}، مساوات \حوالہ{مساوات_جوڑ_چار_جوڑ_تین_آزاد_ب} اور مساوات \حوالہ{مساوات_جوڑ_چار_جوڑ_تین_آزاد_پ} کو اکٹھے لکھتے ہوئے
\begin{gather}
\begin{aligned}
\left(\frac{1}{R_1}+\frac{1}{R_2}+\frac{1}{R_3}\right)v_1-\frac{v_2}{R_2}- \frac{v_3}{R_3}=-i_A\\
-\frac{v_1}{R_2}+\left(\frac{1}{R_2}+\frac{1}{R_4}+\frac{1}{R_5}\right)v_2-\frac{v_3}{R_5}=0\\
-\frac{v_1}{R_3}-\frac{v_2}{R_5}+\left(\frac{1}{R_3}+\frac{1}{R_5}\right)v_3=i_B
\end{aligned}
\end{gather}
قالبی مساوات کی صورت میں لکھتے ہیں۔
\begin{align}\label{مساوات_جوڑ_آزاد_مساوات_دو_جوڑ_ب}
\begin{bmatrix}
\frac{1}{R_1}+\frac{1}{R_2}+\frac{1}{R_3} & -\frac{1}{R_2} &- \frac{1}{R_3}\\[6pt]
-\frac{1}{R_2}&\frac{1}{R_2}+\frac{1}{R_4}+\frac{1}{R_5}& -\frac{1}{R_5}\\[6pt]
-\frac{1}{R_3} & -\frac{1}{R_5}&\frac{1}{R_3}+\frac{1}{R_5}
\end{bmatrix}
\begin{bmatrix}
v_1\\[6pt]
v_2\\[6pt]
v_3
\end{bmatrix}
=
\begin{bmatrix}
-i_A\\[6pt]
0\\[6pt]
i_B
\end{bmatrix}
\end{align}

مندرجہ بالا مساوات کا دایاں بازو منبع رو سے جوڑ میں داخل رو دیتی ہے جبکہ اس کا بایاں بازو جوڑ سے خارجی رو دیتی ہے۔

شکل \حوالہ{شکل_جوڑ_چار_جوڑ_تین_آزاد_مساوات_الف} کو دوبارہ شکل \حوالہ{شکل_جوڑ_چار_جوڑ_تین_آزاد_مساوات_ب} میں پیش کیا گیا ہے جہاں \عددی{i_1}، \عددی{i_3} اور \عددی{i_5} کی سمتیں گزشتہ سمتوں کے الٹ چنی گئی ہیں۔
\begin{figure}
\centering
\includegraphics{figNodalEquationNotDependentOnChosenCurrentDirectionsB}
\caption{مزاحمتوں اور آزاد منبع رو کی قالبی مساوات رو کی چنی سمتوں پر منحصر نہیں۔}
\label{شکل_جوڑ_چار_جوڑ_تین_آزاد_مساوات_ب}
\end{figure}%
تین جوڑ کے مساوات درج ذیل لکھے جائیں گے۔
\begin{align*}
i_A-i_1+i_2-i_3&=0\\
-i_2+i_4-i_5&=0\\
i_3+i_5-i_B&=0
\end{align*}
شاخوں کی رو قانون اوہم سے پُر کرتے ہوئے درج بالا کو یوں لکھا جا سکتا ہے
\begin{align*}
i_A-\left(\frac{0-v_1}{R_1}\right)+\frac{v_1-v_2}{R_2}-\left(\frac{v_3-v_1}{R_3}\right)&=0\\
-\left(\frac{v_1-v_2}{R_2}\right)+\frac{v_2}{R_4}-\left(\frac{v_3-v_2}{R_5}\right)&=0\\
\frac{v_3-v_1}{R_3}+\frac{v_3-v_2}{R_5}-i_B&=0
\end{align*}
جنہیں ترتیب دینے سے درج ذیل حاصل ہوتے ہیں۔
\begin{align}
\left(\frac{1}{R_1}+\frac{1}{R_2}+\frac{1}{R_3}\right) v_1-\frac{v_2}{R_2}-\frac{v_3}{R_3}&=-i_A \label{مساوات_جوڑ_پہلا_جوڑ}\\
-\frac{v_1}{R_2}+\left(\frac{1}{R_2}+\frac{1}{R_4}+\frac{1}{R_5}\right)v_2-\frac{v_3}{R_5}&=0 \label{مساوات_جوڑ_دوسرا_جوڑ}\\
-\frac{v_1}{R_3}-\frac{v_2}{R_5}+\left(\frac{1}{R_3}+\frac{1}{R_5}\right)v_3&=i_B \label{مساوات_جوڑ_تیسرا_جوڑ}
\end{align}
اس کو قالبی مساوات کی صورت میں لکھتے ہیں۔
\begin{align}\label{مساوات_جوڑ_آزاد_مساوات_دو_جوڑ_پ}
\begin{bmatrix}
\frac{1}{R_1}+\frac{1}{R_2}+\frac{1}{R_3} & -\frac{1}{R_2} &- \frac{1}{R_3}\\[6pt]
-\frac{1}{R_2}&\frac{1}{R_2}+\frac{1}{R_4}+\frac{1}{R_5}& -\frac{1}{R_5}\\[6pt]
-\frac{1}{R_3} & -\frac{1}{R_5}&\frac{1}{R_3}+\frac{1}{R_5}
\end{bmatrix}
\begin{bmatrix}
v_1\\[6pt]
v_2\\[6pt]
v_3
\end{bmatrix}
=
\begin{bmatrix}
-i_A\\[6pt]
0\\[6pt]
i_B
\end{bmatrix}
\end{align}

مساوات \حوالہ{مساوات_جوڑ_آزاد_مساوات_دو_جوڑ_ب} اور مساوات \حوالہ{مساوات_جوڑ_آزاد_مساوات_دو_جوڑ_پ} بالکل یکساں ہیں۔یوں آپ دیکھ سکتے ہیں کہ قالبی مساوات کا دارومدار شاخوں میں رو کی چنی گئی سمتوں پر منحصر نہیں ہوتا۔اس کتاب میں اس حقیقت کو استعمال کرتے ہوئے ہم جوڑ پر کرخوف قانون رو کی مساوات لکھتے ہوئے مزاحمتی شاخوں میں رو کی سمت جوڑ سے خارج ہوتی تصور کریں گے۔آئیں اس ترکیب کو شکل \حوالہ{شکل_جوڑ_شاخوں_کی_رو_خارجی} کی مدد سے سمجھیں۔
\begin{figure}
\centering
\begin{subfigure}{\textwidth}
\centering
\includegraphics{figNodalAssumeCurrentsLeavingEveryNodeA}
\caption*{(الف)}
\end{subfigure}
\begin{subfigure}{\textwidth}
\centering
\includegraphics{figNodalAssumeCurrentsLeavingEveryNodeB}
\caption*{(ب)}
\end{subfigure}
\caption{تمام جوڑ پر مزاحمتی شاخوں میں رو کی سمت جوڑ سے خارج ہوتی تصور کر سکتے ہیں۔}
\label{شکل_جوڑ_شاخوں_کی_رو_خارجی}
\end{figure}%

شکل \حوالہ{شکل_جوڑ_شاخوں_کی_رو_خارجی}-الف میں پہلے جوڑ پر تمام مزاحمتی شاخوں کی رو خارجی تصور کرتے ہوئے کرخوف قانون رو کے تحت خارجی رو کا مجموعہ داخلی رو کے مجموعے کے برابر پُر کرنے سے
\begin{align}\label{مساوات_جوڑ_رو_خارجی_الف}
i_1+i_2=i_A
\end{align}
یعنی
\begin{align}\label{مساوات_جوڑ_رو_خارجی_ب}
\frac{v_1}{R_a}+\frac{v_a-v_b}{R_d}=i_A
\end{align}
حاصل ہوتا ہے۔شکل \حوالہ{شکل_جوڑ_شاخوں_کی_رو_خارجی}-ب میں دوسرے جوڑ پر تمام مزاحمتی رو کی سمت خارجی تصور کی گئی ہیں یوں
\begin{align}\label{مساوات_جوڑ_رو_خارجی_پ}
i'_1+i'_2+i'_3=0
\end{align}
یعنی
\begin{align}\label{مساوات_جوڑ_رو_خارجی_ت}
\frac{v_2-v_1}{R_d}+\frac{v_2}{R_b}+\frac{v_2-v_3}{R_e}=0
\end{align}
لکھا جا سکتا ہے۔تیسرے جوڑ پر یہی ترکیب استعمال کرتے ہیں۔ہر جوڑ پر رو کی سمت شکل پر دکھانا ضروری نہیں ہے لہٰذا تیسرے جوڑ پر \عددی{i''_1} اور \عددی{i''_2} دکھانا ضروری نہیں ہے۔ساتھ ہی ساتھ ہر مرتبہ مساوات \حوالہ{مساوات_جوڑ_رو_خارجی_الف} اور مساوات \حوالہ{مساوات_جوڑ_رو_خارجی_پ} کے طرز پر مساوات لکھنے کی بھی ضرورت نہیں ہے بلکہ دل ہی دل میں جوڑ پر تمام مزاحمتی شاخوں کی رو خارجی تصور کرتے ہوئے سیدھ و سیدھ  مساوات \حوالہ{مساوات_جوڑ_رو_خارجی_ب} اور مساوات \حوالہ{مساوات_جوڑ_رو_خارجی_ت} کے طرز پر مساوات لکھے جا سکتے ہیں۔تیسرے جوڑ پر ایسا ہی کرتے ہوئے درج ذیل مساوات لکھی جا سکتی ہے۔
\begin{align}\label{مساوات_جوڑ_رو_خارجی_ٹ}
\frac{v_3-v_2}{R_e}+\frac{v_3}{R_c}+i_B=0
\end{align}

اس کتاب میں ہم مساوات \حوالہ{مساوات_جوڑ_رو_خارجی_ٹ} کی طرح جوڑ پر کرخوف قانون رو کے مساوات لکھیں گے۔

مساوات \حوالہ{مساوات_جوڑ_آزاد_مساوات_دو_جوڑ_پ} اور مساوات \حوالہ{مساوات_جوڑ_آزاد_مساوات_دو_جوڑ_ب} میں \اصطلاح{قالبِ موصلیت}\فرہنگ{قالب!موصلیت}\فرہنگ{موصلیت!قالب}\حاشیہب{conductance matrix}\فرہنگ{matrix!conductance}  \عددی{{\bf{G}}} کے بالائی بائیں کونے سے نچلے دائیں کونے تک ترچھی لکیر کے بالائی اور نچلی اطراف پر یکساں رکن پائے جاتے ہیں۔ایسا اتفاقی طور پر نہیں ہے  بلکہ مزاحمتوں اور آزاد منبع رو پر مبنی کسی بھی دور کے \عددی{{\bf{G}}} قالب کو تشاکل صورت میں لکھا جا سکتا ہے۔آئیں ان قالبوں پر مزید غور کریں۔

شکل \حوالہ{شکل_جوڑ_چار_جوڑ_تین_آزاد_مساوات_ب} میں پہلے جوڑ کا دباو \عددی{v_1}، دوسرے  جوڑ کا دباو \عددی{v_2} اور تیسرے جوڑ کا دباو \عددی{v_3} ہے۔قالب میں بالائی یعنی پہلے صف کے رکن مساوات \حوالہ{مساوات_جوڑ_پہلا_جوڑ} سے حاصل کئے گئے۔یہ مساوات پہلی جوڑ سے حاصل کی گئی ہے۔اس جوڑ پر مزاحمت \عددی{R_1}، \عددی{R_2} اور \عددی{R_3} جڑے ہیں۔ان مزاحمتوں کو متوازی جڑا تصور کرتے ہوئے مساوی مزاحمت \عددی{R_{m1}}
\begin{align*}
\frac{1}{R_{m1}}=\frac{1}{R_1}+\frac{1}{R_2}+\frac{1}{R_3}
\end{align*}
سے حاصل کیا جا سکتا ہے جہاں \عددی{\tfrac{1}{R_{m1}}} کو مساوی متوازی موصلیت \عددی{G_{m1}} کہا جاتا ہے۔یوں  قالب کے پہلے صف کا پہلا (بایاں) رکن  پہلے جوڑ سے  جڑے تمام مزاحمتوں کا مساوی متوازی موصلیت \عددی{\tfrac{1}{R_{m1}}}  ہے۔اسی صف کا دوسرا رکن  پہلے جوڑ اور دوسرے جوڑ کے مابین جڑے مزاحمت کی موصلیت کا منفی \عددی{-\tfrac{1}{R_2}} کے برابر ہے۔اسی طرح پہلے صف کا تیسرا رکن، پہلے جوڑ اور تیسرے جوڑ کے مابین جڑے موصلیت کے منفی \عددی{-\tfrac{1}{R_3}} کے برابر ہے۔قالب کے دوسرے صف کے ارکان مساوات \حوالہ{مساوات_جوڑ_دوسرا_جوڑ} سے حاصل کئے گئے۔اس صف کا پہلا رکن پہلے اور دوسرے جوڑ کے مابین مساوی متوازی موصلیت کے منفی \عددی{-\tfrac{1}{R_2}} کے برابر ہے۔صف کا دوسرا رکن دوسرے جوڑ پر تمام مزاحمتوں کا مساوی متوازی موصلیت \عددی{\tfrac{1}{R_{m2}}}
\begin{align*}
\frac{1}{R_{m2}}=\frac{1}{R_2}+\frac{1}{R_4}+\frac{1}{R_5}
\end{align*}
ہے جبکہ صف کا تیسرا رکن دوسرے اور تیسرے جوڑ کے مابین موصلیت کے منفی \عددی{-\tfrac{1}{R_3}} کے برابر ہے۔قالب کا تیسرا صف بھی اسی طرح حاصل کیا جا سکتا ہے۔قالبی مساوات میں دائیں ہاتھ \اصطلاح{قالبِ رو}\فرہنگ{رو!قالب}\فرہنگ{قالب!رو}\حاشیہب{current matrix}\فرہنگ{matrix!current} کے ارکان بالترتیب پہلے، دوسرے اور تیسرے جوڑ پر جڑے منبع رو سے جوڑ میں داخل ہوتی رو ہے۔منبع رو کی غیر موجودگی میں قالب کے رکن کو صفر لکھا جاتا ہے۔کسی بھی جوڑ پر ایک سے زیادہ منبع رو کی صورت میں جوڑ پر مجموعی داخلی رو، قالب کی رکن ہو گی۔پہلی جوڑ پر منبع کی رو \عددی{i_A} ہے جو جوڑ سے خارجی جانب ہے لہٰذا اسے قالب رو میں \عددی{-i_A} لکھا گیا ہے۔دوسرے جوڑ پر کوئی منبع رو نسب نہیں لہٰذا قالب کا دوسرا رکن صفر ہے۔تیسرے جوڑ پر منبع \عددی{i_B} کی رو جوڑ میں داخل ہوتی ہے لہٰذا قالب رو کا تیسرا رکن \عددی{i_B} ہے۔ 

ان معلومات کی مدد سے  مزاحمت اور منبع رو پر مبنی \عددی{J+1} جوڑ کے دور کی قالبی مساوات  دور کو دیکھ کر درج ذیل صورت میں لکھی جا سکتی ہے
\begin{align}\label{مساوات_جوڑ_عمومی_قالبی_مساوات_الف}
\begin{bmatrix}
+G_{11} & -G_{12} & -G_{13} & \cdots & -G_{1J}\\
-G_{21} & +G_{22} & -G_{23} & \cdots & -G_{2J}\\
-G_{31} & -G_{32} & +G_{33} & \cdots & -G_{3J}\\
\vdots\\
-G_{J1} & -G_{J2} & -G_{J3} & \cdots & +G_{JJ}
\end{bmatrix}
\begin{bmatrix}
v_1\\
v_2\\
v_3\\
\vdots\\
v_J
\end{bmatrix}
=
\begin{bmatrix}
I_1\\
I_2\\
I_3\\
\vdots\\
I_J
\end{bmatrix}
\end{align} 
جہاں \عددی{G_{nn}} سے مراد جوڑ \عددی{n} کے ساتھ منسلک تمام مزاحمتوں کی مساوی متوازی موصلیت جبکہ \عددی{G_{nm}} سے مراد جوڑ \عددی{n} اور \عددی{m} کے مابین مزاحمت کی موصلیت ہے۔یہ مساوات لکھتے ہوئے جوڑ \عددی{J+1} کو زمین چنا گیا ہے۔اگر جوڑ \عددی{n} اور جوڑ \عددی{m} کے مابین مزاحمت \عددی{R_{nm}} جڑی ہو تب جوڑ \عددی{m} اور جوڑ \عددی{n} کے مابین بھی یہی مزاحمت جڑی ہو گی لہٰذا آپ دیکھ سکتے ہیں کہ 
\begin{align}
G_{nm}=G_{mn}
\end{align}
ہو گا اور یوں مساوات \حوالہ{مساوات_جوڑ_عمومی_قالبی_مساوات_الف} کو درج ذیل صورت میں لکھا جا سکتا ہے
\begin{align}\label{مساوات_جوڑ_عمومی_قالبی_مساوات_ب}
\begin{bmatrix}
+G_{11} & -G_{12} & -G_{13} & \cdots & -G_{1J}\\
-G_{12} & +G_{22} & -G_{23} & \cdots & -G_{2J}\\
-G_{13} & -G_{23} & +G_{33} & \cdots & -G_{3J}\\
\vdots\\
-G_{1J} & -G_{2J} & -G_{3J} & \cdots & +G_{JJ}
\end{bmatrix}
\begin{bmatrix}
v_1\\
v_2\\
v_3\\
\vdots\\
v_J
\end{bmatrix}
=
\begin{bmatrix}
I_1\\
I_2\\
I_3\\
\vdots\\
I_J
\end{bmatrix}
\end{align}
جس میں \عددی{{\bf{G}}} کا قالب تشاکل ہے۔

\FloatBarrier
%==========================
\ابتدا{مشق}\شناخت{مشق_جوڑ_تیسرا_الف}
شکل \حوالہ{شکل_جوڑ_تیسرا_الف} میں \عددی{v_1} اور \عددی{v_2} پر کرخوف قانون رو کے مساوات لکھتے ہوئے دور کی قالبی مساوات حاصل کریں۔قالبی مساوات حل کرتے ہوئے نا معلوم دباو دریافت کریں۔
\begin{figure}[!h]
\centering
\includegraphics{figNodalQuizThree}
\caption{مشق \حوالہ{مشق_جوڑ_تیسرا_الف} کا دور۔}
\label{شکل_جوڑ_تیسرا_الف}
\end{figure}%

جوابات:\عددی{v_1=\SI{1}{\volt}}، \عددی{v_2=\SI{-20}{\volt}}
\انتہا{مشق}

%==========================
\ابتدا{مشق}\شناخت{مشق_جوڑ_تیسرا_ب}
شکل \حوالہ{شکل_جوڑ_تیسرا_ب} کی قالبی مساوات لکھتے ہوئے نا معلوم دباو حاصل کریں۔
\begin{figure}[!h]
\centering
\includegraphics{figNodalQuizFour}
\caption{مشق \حوالہ{مشق_جوڑ_تیسرا_ب} کا دور۔}
\label{شکل_جوڑ_تیسرا_ب}
\end{figure}%

جوابات:\عددی{v_1=\SI{-6}{\volt}}، \عددی{v_2=\SI{-2}{\volt}}،\عددی{v_3=\SI{4}{\volt}}
\انتہا{مشق}
%=========================

%==========================
\ابتدا{مشق}\شناخت{مشق_جوڑ_تیسرا_پ}
شکل \حوالہ{شکل_جوڑ_تیسرا_پ} کی قالبی مساوات لکھتے ہوئے نا معلوم دباو حاصل کریں۔
\begin{figure}[!h]
\centering
\includegraphics{figNodalQuizFive}
\caption{مشق \حوالہ{مشق_جوڑ_تیسرا_پ} کا دور۔}
\label{شکل_جوڑ_تیسرا_پ}
\end{figure}%

جوابات:\عددی{v_1=\SI{13.5}{\volt}}، \عددی{v_2=\SI{14}{\volt}}،\عددی{v_3=\SI{22}{\volt}}
\انتہا{مشق}
%=========================
\FloatBarrier

\حصہ{تابع منبع رو استعمال کرنے والے ادوار}
گزشتہ حصے میں ہم نے دیکھا کہ غیر تابع منبع رو اور مزاحمتوں کے ادوار سے تشاکل قالب موصلیت حاصل ہوتے ہے۔شکل \حوالہ{شکل_جوڑ_تابع_منبع_رو_غیر_تشاکل_قالب} میں تباع منبع رو استعمال کی گئی ہے۔ہم دیکھیں گے کہ اس کا \عددی{{\bf{G}}} قالب غیر تشاکل ہو گا۔اس دور کے تین جوڑوں سے درج ذیل لکھا جا سکتا ہے
\begin{gather}
\begin{aligned}\label{مساوات_جوڑ_تابع_منبع_رو_غیر_تشاکل_قالب_الف}
-\beta i_0+\frac{v_1}{R_1}+\frac{v_1-v_2}{R_2}&=0\\
\frac{v_2-v_1}{R_2}-i_A+\frac{v_2-v_3}{R_4}&=0\\
\frac{v_3}{R_3}+\beta i_0+\frac{v_3-v_2}{R_4}&=0
\end{aligned}
\end{gather}
جہاں
\begin{align}\label{مساوات_جوڑ_تابع_منبع_رو_غیر_تشاکل_قالب_ب}
i_0=\frac{v_1}{R_1}
\end{align}
کے برابر ہے۔مساوات \حوالہ{مساوات_جوڑ_تابع_منبع_رو_غیر_تشاکل_قالب_الف} میں مساوات \حوالہ{مساوات_جوڑ_تابع_منبع_رو_غیر_تشاکل_قالب_ب} پُر کرتے اور ترتیب دیتے ہوئے درج ذیل حاصل ہوتا ہے
\begin{gather}
\begin{aligned}\label{مساوات_جوڑ_تابع_منبع_رو_غیر_تشاکل_قالب_پ}
\left(\frac{1}{R_1}+\frac{1}{R_2}-\frac{\beta}{R_1}\right) v_1-\frac{v_2}{R_2}&=0\\
-\frac{v_1}{R_2}+\left(\frac{1}{R_2}+\frac{1}{R_4}\right)v_2-\frac{v_3}{R_4}&=i_A\\
\frac{\beta}{R_1} v_1 -\frac{v_2}{R_4}+\left(\frac{1}{R_3}+\frac{1}{R_4}\right)v_3&=0
\end{aligned}
\end{gather}
جسے قالبی صورت میں لکھتے ہیں۔
\begin{align}\label{مساوات_جوڑ_تابع_منبع_مثال_الف}
\begin{bmatrix}
\frac{1}{R_1}+\frac{1}{R_2}-\frac{\beta}{R_1}&-\frac{1}{R_2}&0\\
-\frac{1}{R_2}&\frac{1}{R_2}+\frac{1}{R_4}&-\frac{1}{R_4}\\
\frac{\beta}{R_1}& -\frac{1}{R_4}&\frac{1}{R_3}+\frac{1}{R_4}
\end{bmatrix}
\begin{bmatrix}
v_1\\
v_2\\
v_3
\end{bmatrix}
=
\begin{bmatrix}
0\\
i_A\\
0
\end{bmatrix}
\end{align}
آپ دیکھ سکتے ہیں کہ \عددی{{\bf{G}}} قالب غیر تشاکل ہے۔
\begin{figure}
\centering
\includegraphics{figNodalDependantCurrentSourceDestroysSymmetry}
\caption{تابع منبع رو سے غیر تشاکل قالب موصلیت حاصل ہوتا ہے۔}
\label{شکل_جوڑ_تابع_منبع_رو_غیر_تشاکل_قالب}
\end{figure}%
%=======================
\ابتدا{مثال}
شکل \حوالہ{شکل_جوڑ_تابع_منبع_رو_غیر_تشاکل_قالب} میں تمام جوڑ پر برقی دباو حاصل کریں۔معلومات درج ذیل ہیں۔
\begin{align*}
R_1=\SI{2}{\kilo\ohm}, \quad R_2=\SI{4}{\kilo\ohm}, \quad R_3=\SI{1}{\kilo\ohm}, \quad R_4=\SI{2}{\kilo\ohm}, \quad i_A=\SI{10}{\milli\ampere}, \quad \beta=4
\end{align*}

حل:درج بالا معلومات کو مساوات \حوالہ{مساوات_جوڑ_تابع_منبع_مثال_الف} میں پُر کرتے ہیں۔
\begin{align*}
\begin{bmatrix}
\frac{1}{2000}+\frac{1}{4000}-\frac{4}{2000}&-\frac{1}{4000}&0\\
-\frac{1}{4000}&\frac{1}{4000}+\frac{1}{2000}&-\frac{1}{2000}\\
\frac{\beta}{2000}& -\frac{1}{2000}&\frac{1}{1000}+\frac{1}{2000}
\end{bmatrix}
\begin{bmatrix}
v_1\\
v_2\\
v_3
\end{bmatrix}
=
\begin{bmatrix}
0\\
0.01\\
0
\end{bmatrix}
\end{align*}
اس قالبی مساوات کو حل کرتے ہوئے اور یا تینوں ہمزاد مساوات کو کسی بھی طریقے سے حل کرتے ہوئے  درج ذیل حاصل ہوتے ہیں۔
\begin{align*}
v_1&=\SI{-4}{\volt}\\
v_2&=\SI{20}{\volt}\\
v_3&=\SI{12}{\volt}
\end{align*}
\انتہا{مثال}
%====================
\ابتدا{مثال}\شناخت{مثال_جوڑ_تابع_منبع_مزید_مثال}
شکل \حوالہ{شکل_جوڑ_مثال_مزید} میں تمام نا معلوم دباو حاصل کریں۔دیگر معلومات درج ذیل ہیں۔
\begin{align*}
R_1=\SI{4}{\kilo\ohm}, \quad R_2=\SI{8}{\kilo\ohm}, \quad R_3=\SI{12}{\kilo\ohm},\quad R_4=\SI{6}{\kilo\ohm},\quad R_5=\SI{2}{\kilo\ohm}\\
 i_A=\SI{1}{\milli\ampere},\quad \alpha=0.002
\end{align*}
%===
\begin{figure}[!h]
\centering
\includegraphics{figNodalExampleControlledCurrentSourcesA}
\caption{مثال \حوالہ{مثال_جوڑ_تابع_منبع_مزید_مثال} کا دور۔}
\label{شکل_جوڑ_مثال_مزید}
\end{figure}%

حل:تمام جوڑ پر خارجی رو تصور کرتے ہوئے مساوات لکھتے ہیں۔
\begin{align*}
\frac{v_1}{R_1}+\frac{v_1-v_2}{R_2}+\frac{v_1-v_3}{R_3}&=i_A\\
\frac{v_2-v_1}{R_2}+\alpha v_x +\frac{v_2-v_3}{R_4}&=0\\
\frac{v_3-v_1}{R_3}+\frac{v_3-v_2}{R_4}+\frac{v_3}{R_5}&=0
\end{align*}
اس میں \عددی{v_x=v_2-v_3} پُر کرتے اور مساوات کے اجزاء کو ترتیب دیتے ہیں۔
\begin{align*}
\left(\frac{1}{R_1}+\frac{1}{R_2}+\frac{1}{R_3}\right)v_1-\frac{v_2}{R_2}-\frac{v_3}{R_3}&=i_A\\
-\frac{v_1}{R_2}+\left(\frac{1}{R_2}+\alpha+\frac{1}{R_4}\right)v_2-(\alpha+\frac{1}{R_4})v_3&=0\\
-\frac{v_1}{R_3}-\frac{v_2}{R_4}+\left(\frac{1}{R_3}+\frac{1}{R_4}+\frac{1}{R_5}\right)v_3&=0
\end{align*}
دی گئی معلومات پُر کرتے ہیں
\begin{align*}
\left(\frac{1}{4000}+\frac{1}{8000}+\frac{1}{12000}\right)v_1-\frac{v_2}{8000}-\frac{v_3}{12000}&=0.001\\
-\frac{v_1}{8000}+\left(\frac{1}{8000}+0.002+\frac{1}{6000}\right)v_2-(0.002+\frac{1}{6000})v_3&=0\\
-\frac{v_1}{12000}-\frac{v_2}{6000}+\left(\frac{1}{12000}+\frac{1}{6000}+\frac{1}{2000}\right)v_3&=0
\end{align*}
تینوں ہمزاد مساواتوں کو \عددی{1000} سے ضرب دیتے ہوئے درج ذیل لکھا جا سکتا ہے۔
\begin{align*}
\frac{11v_1}{24}-\frac{v_2}{8}-\frac{v_3}{12}&=1\\
-\frac{v_1}{8}+\frac{55 v_2}{24}-\frac{13 v_3}{6}&=0\\
-\frac{v_1}{12}-\frac{v_2}{6}+\frac{3v_3}{4}&=0
\end{align*}
انہیں حل کرتے ہوئے درج ذیل حاصل ہوتے ہیں۔
\begin{align*}
v_1&=\SI{2.38}{\volt}\\
v_2&=\SI{0.48}{\volt}\\
v_3&=\SI{0.37}{\volt}
\end{align*}
\انتہا{مثال}
%================
\FloatBarrier

\ابتدا{مشق}\شناخت{مشق_جوڑ_تابع_منبع_رو_الف}
شکل \حوالہ{شکل_مشق_جوڑ_تابع_منبع_رو_الف} میں نا معلوم دباو جوڑ \عددی{V_1} اور \عددی{V_2} دریافت کریں۔ 
\begin{figure}
\centering
\includegraphics{figNodalQuizControlledCurrentSourcesA}
\caption{مشق \حوالہ{مشق_جوڑ_تابع_منبع_رو_الف} کا دور۔}
\label{شکل_مشق_جوڑ_تابع_منبع_رو_الف}
\end{figure}%

جوابات:\عددی{V_1=\tfrac{120}{37}\,\si{\volt}}، \عددی{V_2=\tfrac{720}{37}\,\si{\volt}}
\انتہا{مشق}
%=================


\ابتدا{مشق}\شناخت{مشق_جوڑ_تابع_منبع_رو_ب}
شکل \حوالہ{شکل_مشق_جوڑ_تابع_منبع_رو_ب} میں نا معلوم دباو جوڑ \عددی{V_0} دریافت کریں۔ 
\begin{figure}[!h]
\centering
\includegraphics{figNodalQuizControlledCurrentSourcesB}
\caption{مشق \حوالہ{مشق_جوڑ_تابع_منبع_رو_ب} کا دور۔}
\label{شکل_مشق_جوڑ_تابع_منبع_رو_ب}
\end{figure}%

جواب:\عددی{V_0=\tfrac{16}{5}\,\si{\volt}}
\انتہا{مشق}
%======================

\ابتدا{مشق}\شناخت{مشق_جوڑ_تابع_منبع_رو_پ}
شکل \حوالہ{شکل_مشق_جوڑ_تابع_منبع_رو_پ} میں نا معلوم دباو جوڑ \عددی{V_0} دریافت کریں۔ 
\begin{figure}[!h]
\centering
\includegraphics{figNodalQuizControlledCurrentSourcesC}
\caption{مشق \حوالہ{مشق_جوڑ_تابع_منبع_رو_پ} کا دور۔}
\label{شکل_مشق_جوڑ_تابع_منبع_رو_پ}
\end{figure}%

جواب:\عددی{V_0=\tfrac{14}{11}\,\si{\volt}}
\انتہا{مشق}
%========================

\حصہ{غیر تابع منبع دباو استعمال کرنے والے ادوار}
گزشتہ حصوں کی طرح اس حصے کو بھی سادہ ترین مثال سے شروع کرتے ہیں۔بعد میں بتدریج مشکل مثال پیش کئے جائیں گے۔سب سے پہلے ایک مثال کی مدد سے ایسے دور پر غور کرتے ہیں جس میں غیر تابع منبع دباو کا ایک سرا  برقی زمین کے ساتھ جڑا ہو۔ایسے ادوار نسبتاً آسانی سے حل ہوتے ہیں۔
%==========

\ابتدا{مثال}\شناخت{مثال_جوڑ_غیر_تابع_منبع_دباو_الف}
شکل \حوالہ{شکل_جوڑ_غیر_تابع_منبع_دباو_الف}-الف کے دور میں دو عدد غیر تابع منبع دباو استعمال کئے گئے ہیں۔دونوں منبع  زمین کے ساتھ جڑے ہیں۔بالائی بایاں جوڑ \عددی{\SI{10}{\volt}} منبع دباو کے مثبت سرے کے ساتھ جڑا ہے  جبکہ بالائی دایاں جوڑ \عددی{\SI{20}{\volt}} منبع دباو کے منفی سرے کے ساتھ جڑا ہے لہٰذا
\begin{align*}
V_1&=\SI{10}{\volt}\\
V_2&=\SI{-20}{\volt}
\end{align*} 
ہیں۔بالائی درمیانے جوڑ پر کرخوف قانون رو کی مدد سے
\begin{align*}
\frac{V_2-10}{5000}+\frac{V_2}{10000}+\frac{V_2-(-20)}{20000}=0
\end{align*}
لکھا جا سکتا ہے جس سے
\begin{align*}
V_2=\frac{20}{7} \, \si{\volt}
\end{align*}
حاصل ہوتا ہے۔
\begin{figure}
\centering
\begin{subfigure}{1\textwidth}
\centering
\includegraphics{figNodalExampleIndependentVoltageSourceA}
\caption*{(الف)}
\end{subfigure}
\begin{subfigure}{1\textwidth}
\centering
\includegraphics{figNodalExampleIndependentVoltageSourceAA}
\caption*{(ب)}
\end{subfigure}
\caption{مثال \حوالہ{مثال_جوڑ_غیر_تابع_منبع_دباو_الف} کا دور۔}
\label{شکل_جوڑ_غیر_تابع_منبع_دباو_الف}
\end{figure}%

دباو جوڑ جاننے کے بعد تمام پرزوں میں رو دریافت کی جا سکتی ہے۔یوں بالترتیب  \عددی{\SI{5}{\kilo\ohm}}، \عددی{\SI{10}{\kilo\ohm}}، \عددی{\SI{20}{\kilo\ohm}} اور \عددی{\SI{30}{\kilo\ohm}} کے مزاحمتوں میں اوہم کے قانون سے درج ذیل رو حاصل ہوتے ہیں
\begin{align*}
\frac{V_1-V_2}{5000}&=\frac{10-\frac{20}{7}}{5000}=\frac{10}{7} \,\si{\milli\ampere}\\
\frac{V_2}{10000}&=\frac{\frac{20}{7}}{10000}=\frac{2}{7} \,\si{\milli\ampere}\\
\frac{V_2-V3}{20000}&=\frac{\frac{20}{7}-(-20)}{20000}=\frac{8}{7} \,\si{\milli\ampere}\\
\frac{V_1-V_3}{30000}&=\frac{10-(-20)}{30000}=\SI{1}{\milli\ampere}\\
\end{align*}
جہاں تمام رو کی سمتیں بائیں سے دائیں جانب ہیں۔جوڑ \عددی{V_1} پر کرخوف قانون رو سے \عددی{\SI{10}{\volt}} منبع کی خارجی رو درج ذیل حاصل ہوتی ہے۔
\begin{align*}
I_{\SI{10}{\volt}}=\frac{10}{7} \,\si{\milli\ampere}+\SI{1}{\milli\ampere}=\frac{17}{7} \,\si{\milli\ampere}
\end{align*}
اسی طرح دائیں منبع دباو کے منفی سرے پر داخلی رو یا مثبت سرے سے خارجی رو درج ذیل حاصل ہوتی ہے۔
\begin{align*}
I_{\SI{20}{\volt}}=\frac{8}{7} \,\si{\milli\ampere}+\SI{1}{\milli\ampere}=\frac{15}{7} \,\si{\milli\ampere}
\end{align*}
حاصل کردہ تمام رو کو شکل \حوالہ{شکل_جوڑ_غیر_تابع_منبع_دباو_الف}-ب میں دکھایا گیا ہے۔
\انتہا{مثال}
%===================
\ابتدا{مشق}\شناخت{مشق_جوڑ_غیر_تابع_منبع_دباو_الف}
شکل \حوالہ{شکل_جوڑ_مشق_غیر_تابع_منبع_دباو_الف} میں \عددی{I_0} حاصل کریں۔
\begin{figure}
\centering
\includegraphics{figNodalQuizIndependentVoltageSourceA}
\caption{مشق \حوالہ{مشق_جوڑ_غیر_تابع_منبع_دباو_الف} کا دور۔}
\label{شکل_جوڑ_مشق_غیر_تابع_منبع_دباو_الف}
\end{figure}%

جواب:\عددی{I_0=\SI{0.282}{\micro\ampere}}
\انتہا{مشق}

%===================

\ابتدا{مشق}\شناخت{مشق_جوڑ_غیر_تابع_منبع_دباو_ب}
شکل \حوالہ{شکل_جوڑ_مشق_غیر_تابع_منبع_دباو_ب} میں \عددی{V_0} دریافت کریں۔یاد رہے کہ آپ کسی بھی جوڑ کو برقی زمین چن سکتے ہیں۔
\begin{figure}
\centering
\includegraphics{figNodalQuizIndependentVoltageSourceB}
\caption{مشق \حوالہ{مشق_جوڑ_غیر_تابع_منبع_دباو_ب} کا دور۔}
\label{شکل_جوڑ_مشق_غیر_تابع_منبع_دباو_ب}
\end{figure}%

جواب:\عددی{V_0=\tfrac{396}{23}\,\si{\volt}}
\انتہا{مشق}

%===================

آئیں اب ایسے دور کو حل کریں جس میں منبع دباو برقی زمین سے ہٹ کر دو جوڑوں کے درمیان جڑا ہو۔

%==========================
\ابتدا{مثال}\شناخت{مثال_جوڑ_غیر_تابع_منبع_دباو_دو_جوڑ_مابین}
شکل \حوالہ{شکل_مثال_جوڑ_غیر_تابع_منبع_دباو_دو_جوڑ_مابین} میں \عددی{\SI{10}{\volt}} کا منبع دباو زمین سے ہٹ کر دو جوڑوں کے درمیان نسب ہے۔گزشتہ تمام مثالوں میں جوڑ پر رو یا تو منبع رو سے اخذ کی جا سکتیں تھیں اور یا انہیں مزاحمتی شاخ پر قانون اوہم لاگو کرتے ہوئے اخذ کیا جا سکتا تھا۔موجودہ شکل میں جوڑ \عددی{V_1} اور \عددی{V_2} کے درمیان نہ تو منبع رو نسب ہے اور نہ ہی مزاحمت لہٰذا گزشتہ ترکیب یہاں قابل استعمال نہیں ہیں۔آپ سے گزارش ہے کہ یہاں رک کر تسلی کر لیں کہ دس وولٹ منبع دباو کی رو گزشتہ ترکیبوں سے دریافت نہیں کی جا سکتی۔ 
\begin{figure}
\centering
\begin{subfigure}{1\textwidth}
\centering
\includegraphics{figNodalSuperNodeA}
\caption*{(الف)}
\end{subfigure}
\begin{subfigure}{1\textwidth}
\centering
\includegraphics{figNodalSuperNodeB}
\caption*{(ب)}
\end{subfigure}
\caption{مثال \حوالہ{مثال_جوڑ_غیر_تابع_منبع_دباو_دو_جوڑ_مابین} کا دور۔}
\label{شکل_مثال_جوڑ_غیر_تابع_منبع_دباو_دو_جوڑ_مابین}
\end{figure}%

اب منبع دباو کی رو ہم نہ تو جانتے ہیں اور نہ ہی اسے کسی مساوات سے ظاہر کر سکتے ہیں لہٰذا جوڑ \عددی{V_1} اور \عددی{V_2} پر کرخوف قانون رو کے مساوات لکھنا ممکن نہیں ہے۔آپ جانتے ہیں کہ \عددی{J} متغیرات معلوم کرنے کی خاطر \عددی{J} ہمزاد مساوات درکار ہیں۔آئیں دیکھیں کہ  جوڑ \عددی{V_1} اور \عددی{V_2} پر کرخوف قانون رو نہ لکھ پانے کے باوجود ہم اتنی تعداد میں مساوات کس طرح  لکھ پائیں گے۔

شکل \حوالہ{شکل_مثال_جوڑ_غیر_تابع_منبع_دباو_دو_جوڑ_مابین}-الف کو دیکھ کر
\begin{align}\label{مساوات_جوڑ_منبع_دباو_درمیان_جوڑ_الف}
V_2-V_1=10
\end{align}
لکھا جا سکتا ہے۔اس کے علاوہ اسی شکل میں دکھائے شاخوں کے برقی رو استعمال کرتے ہوئے ہم درج ذیل لکھ سکتے ہیں۔
\begin{align}
-\SI{8}{\milli\ampere}+I_1+I_m&=0 \label{مساوات_جوڑ_منبع_دباو_درمیان_جوڑ_ب}\\
-I_m+I_2+\SI{3}{\milli\ampere}&=0 \label{مساوات_جوڑ_منبع_دباو_درمیان_جوڑ_پ}
\end{align}
مساوات \حوالہ{مساوات_جوڑ_منبع_دباو_درمیان_جوڑ_ب} اور مساوات \حوالہ{مساوات_جوڑ_منبع_دباو_درمیان_جوڑ_پ} کے مجموعہ
\begin{align}\label{مساوات_جوڑ_منبع_دباو_درمیان_جوڑ_ت}
-\SI{5}{\milli\ampere}+I_1+I_2&=0
\end{align}
میں قانون اوہم کے استعمال سے درج ذیل حاصل ہوتا ہے۔
\begin{align}\label{مساوات_جوڑ_منبع_دباو_درمیان_جوڑ_ٹ}
-\SI{8}{\milli\ampere}+\frac{V_1}{\SI{8}{\kilo\ohm}}+\frac{V_2}{\SI{10}{\kilo\ohm}}+\SI{3}{\milli\ampere}&=0
\end{align}
مساوات \حوالہ{مساوات_جوڑ_منبع_دباو_درمیان_جوڑ_الف} اور مساوات \حوالہ{مساوات_جوڑ_منبع_دباو_درمیان_جوڑ_ٹ} درکار ہمزاد مساوات ہیں جنہیں حل کرنے سے
\begin{align*}
V_1&=\SI{240}{\volt}\\
V_2&=\SI{250}{\volt}
\end{align*}
حاصل ہوتے ہیں۔

ہم پہلے دیکھ چکے ہیں کہ کسی بھی جوڑ پر کرخوف قانون رو لکھتے ہوئے تمام مزاحمتی شاخوں میں رو کی سمت خارجی تصور کی جا سکتی ہے۔یہاں اس بات کا خیال رکھنا ضروری ہے کہ دو جوڑوں کے مابین نسب منبع دباو کی رو کو دونوں جوڑوں پر خارجی تصور نہیں کیا جا سکتا۔منبع دباو کے رو کی کوئی بھی سمت چننے کے بعد دونوں جوڑوں پر کرخوف قانون رو لکھتے ہوئے  منبع دباو کے رو کی  سمت چنی گئی سمت ہی تصور کی جائے گی۔

 مساوات \حوالہ{مساوات_جوڑ_منبع_دباو_درمیان_جوڑ_ٹ} کے حصول میں ہمیں مساوات \حوالہ{مساوات_جوڑ_منبع_دباو_درمیان_جوڑ_ب}، مساوات \حوالہ{مساوات_جوڑ_منبع_دباو_درمیان_جوڑ_پ} اور مساوات \حوالہ{مساوات_جوڑ_منبع_دباو_درمیان_جوڑ_ت} بھی لکھنے پڑھ گئے۔آئیں ان اضافی مساوات کے لکھنے سے چھٹکارا حاصل کریں۔

شکل  \حوالہ{شکل_مثال_جوڑ_غیر_تابع_منبع_دباو_دو_جوڑ_مابین}-ب میں زمین سے ہٹ کر دو جوڑوں کے مابین نسب منبع دباو کے گرد نقطہ دار دائرہ کھینچا گیا ہے۔اس بند دائرے کو \اصطلاح{مخلوط جوڑ}\فرہنگ{جوڑ!مخلوط}\حاشیہب{super node}\فرہنگ{node!super} کہا جاتا ہے۔آپ جانتے ہیں کہ کرخوف قانون رو بند دائرے پر بھی لاگو ہوتا ہے لہٰذا اس بند دائرے میں مجموعی داخلی رو اور مجموعی خارجی رو برابر ہوں گے۔شکل-ب میں مخلوط جوڑ سے تمام مزاحمتی شاخوں کے رو کی سمت خارجی تصور کرتے ہوئے
\begin{align}
-\SI{8}{\milli\ampere}+\frac{V_1}{\SI{8}{\kilo\ohm}}+\frac{V_2}{\SI{10}{\kilo\ohm}}+\SI{3}{\milli\ampere}=0
\end{align}
لکھا جا سکتا ہے جو مساوات  \حوالہ{مساوات_جوڑ_منبع_دباو_درمیان_جوڑ_ٹ} ہی ہے۔یاد رہے کہ دور حل کرنے کی خاطر مخلوط جوڑ کے دونوں جانب دباو کا تعلق
\begin{align}
V_2-V_1=10
\end{align}
بھی درکار ہے۔
\انتہا{مثال}
%==============

\ابتدا{مثال}\شناخت{مثال_جوڑ_غیر_تابع_منبع_دباو_دو_جوڑ_مابین_ب}
شکل \حوالہ{شکل_جوڑ_مثال_مخلوط_جوڑ_ب}-الف میں \عددی{I_1} اور \عددی{I_2} دریافت کریں۔
\begin{figure}
\centering
\begin{subfigure}{1\textwidth}
\centering
\includegraphics{figNodalSuperNodeExampleA}
\caption*{(الف)}
\end{subfigure}
\begin{subfigure}{1\textwidth}
\centering
\includegraphics{figNodalSuperNodeExampleB}
\caption*{(ب)}
\end{subfigure}
\caption{مثال \حوالہ{مثال_جوڑ_غیر_تابع_منبع_دباو_دو_جوڑ_مابین_ب} کا دور۔}
\label{شکل_جوڑ_مثال_مخلوط_جوڑ_ب}
\end{figure}%

حل:شکل \حوالہ{شکل_جوڑ_مثال_مخلوط_جوڑ_ب}-ب میں مخلوط جوڑ کو نقطہ دار دائرے میں گھیرا گیا ہے۔ساتھ ہی ساتھ مخلوط جوڑ کے سروں کے مابین دباو کے تعلق
\begin{align*}
V_3-V_2=16
\end{align*}
 کو استعمال کرتے ہوئے بالائی جوڑ کے دباو کو نچلے جوڑ کے دباو کی صورت میں
\begin{align*}
V_2=V_3-16
\end{align*}
 لکھا گیا ہے۔ہم بالائی جوڑ کے دباو کو \عددی{V_2} ہی لکھتے ہوئے نچلے جوڑ کے دباو کو \عددی{V_3=V_2+16} لکھ سکتے تھے۔ شکل \حوالہ{شکل_جوڑ_مثال_مخلوط_جوڑ_ب}-ب  کو دیکھ کر درج ذیل بھی لکھا جا سکتا ہے۔
\begin{align*}
V_1&=\SI{20}{\volt}\\
V_4&=\SI{10}{\volt}
\end{align*}
یوں صرف \عددی{V_3} نا معلوم دباو ہے۔کرخوف قانون رو استعمال کرتے ہوئے مخلوط جوڑ یعنی نقطہ دار دائرے کے لئے 
\begin{align*}
\frac{(V_3-16)-20}{\SI{8}{\kilo\ohm}}+\frac{(V_3-16)-10}{\SI{2}{\kilo\ohm}}+\frac{V_3-20}{\SI{1}{\kilo\ohm}}+\frac{V_3-10}{\SI{6}{\kilo\ohm}}+\frac{V_3}{\SI{4}{\kilo\ohm}}=0
\end{align*}
لکھا جا سکتا ہے جہاں تمام رو کی سمت خارجی چنی گئی ہے۔ اس سے درج ذیل حاصل ہوتا ہے۔
\begin{align*}
V_3=\SI{20}{\volt}
\end{align*}
یوں درکار رو درج ذیل ہیں۔
\begin{align*}
I_1&=\frac{V_1-V_3}{\SI{1}{\kilo\ohm}}=\frac{20-20}{\SI{1}{\kilo\ohm}}=\SI{0}{\ampere}\\
I_2&=\frac{V_3-V_4}{\SI{6}{\kilo\ohm}}=\frac{20-10}{\SI{6}{\kilo\ohm}}=\frac{5}{3}\,\si{\milli\ampere}
\end{align*}
\انتہا{مثال}
%===================
\ابتدا{مشق}\شناخت{مشق_جوڑ_مخلوط_الف}
شکل \حوالہ{شکل_جوڑ_مشق_مخلوط_الف} میں \عددی{I_0} دریافت کریں۔
\begin{figure}
\centering
\includegraphics{figNodalSuperNodeQuizA}
\caption{مشق \حوالہ{مشق_جوڑ_مخلوط_الف} کا دور۔}
\label{شکل_جوڑ_مشق_مخلوط_الف}
\end{figure}%

جواب:\عددی{\tfrac{49}{18} \, \si{\milli\ampere}}
\انتہا{مشق}
%=========================

\ابتدا{مشق}\شناخت{مشق_جوڑ_مخلوط_ب}
شکل \حوالہ{شکل_جوڑ_مشق_مخلوط_ب} میں \عددی{I_0} دریافت کریں۔
\begin{figure}
\centering
\includegraphics{figNodalSuperNodeQuizB}
\caption{مشق \حوالہ{مشق_جوڑ_مخلوط_ب} کا دور۔}
\label{شکل_جوڑ_مشق_مخلوط_ب}
\end{figure}%

جواب:\عددی{\tfrac{5}{11} \, \si{\milli\ampere}}
\انتہا{مشق}
%====================

\حصہ{تابع منبع دباو استعمال کرنے والے ادوار}
تابع  منبع استعمال کرنے والے ادوار کو بھی مندرجہ بالا طریقوں سے حل کیا جاتا ہے۔آئیں چند مثال دیکھیں۔


%====================
\ابتدا{مثال}\شناخت{مثال_جوڑ_تابع_منبع_دباو_الف}
شکل \حوالہ{شکل_جوڑ_مثال_تابع_منبع_دباو_الف} میں \عددی{I_0} حاصل کریں۔
\begin{figure}
\centering
\includegraphics{figNodalDependentVoltageExampleA}
\caption{مثال \حوالہ{مثال_جوڑ_تابع_منبع_دباو_الف} کا دور۔}
\label{شکل_جوڑ_مثال_تابع_منبع_دباو_الف}
\end{figure}%

حل:چونکہ جوڑ \عددی{V_1} تابع منبع دباو سے جڑا ہے لہٰذا
\begin{align*}
V_1=2000 I_a
\end{align*}
ہو گا جہاں
\begin{align*}
I_a=\frac{V_2}{\SI{1}{\kilo\ohm}}
\end{align*}
ہے۔جوڑ \عددی{V_2} پر کرخوف قانون رو سے درج ذیل لکھتے ہیں۔
\begin{align*}
\SI{2}{\milli\ampere}+\frac{V_2-V_1}{\SI{2}{\kilo\ohm}}+I_a=0
\end{align*}
انہیں حل کرنے سے
\begin{align*}
V_2&=\SI{-4}{\volt}\\
V_1&=\SI{-8}{\volt}\\
I_a&=\SI{-4}{\milli\ampere}
\end{align*}
حاصل ہوتے ہیں لہٰذا
\begin{align*}
I_0=\frac{(-4)-(-8)}{\SI{2}{\kilo\ohm}}=\SI{2}{\milli\ampere}
\end{align*}
ہو گی۔
\انتہا{مثال}

\FloatBarrier
%===============
\ابتدا{مثال}\شناخت{مثال_جوڑ_تابع_منبع_دباو_ب}
شکل \حوالہ{شکل_جوڑ_تابع_منبع_مخلوط_جوڑ_الف} میں تابع منبع مخلوط جوڑ کے مابین نسب ہے۔اس دور میں \عددی{V_0} حاصل کریں۔
\begin{figure}
\centering
\begin{subfigure}{1\textwidth}
\centering
\noindent
\begin{tikzpicture}
\pgfmathsetmacro{\y}{2}
\pgfmathsetmacro{\x}{2}
\draw(0,0) to [short]++(3*\x,0) to [american voltage source,l_={$\SI{5}{\volt}$}]++(0,\y) to [resistor,l_={$\SI{8}{\kilo\ohm}$}]++(0,\y) to [short]++(-\x,0)  to [american current source,l_={$\SI{2}{\milli\ampere}$}]++(-\x,0) to [short]++(-\x,0) to [resistor,l_={$\SI{4}{\kilo\ohm}$}]++(0,-2*\y);
\draw(\x,0) to [resistor,*-,l={$\SI{2}{\kilo\ohm}$}]++(0,\y) to [american controlled voltage source,-*,l={$3 V_a$}] ++(0,\y);
\draw(2*\x,2*\y) to [resistor,*-,l={$\SI{1}{\kilo\ohm}$}]++(0,-\y) to [american current source,-*,l={$\SI{1}{\milli\ampere}$}]++(0,-\y);
\draw(\x,\y) to [resistor,*-*,l={$\SI{6}{\kilo\ohm}$}]++(\x,0);
\draw(\x+\dx,\y/2)node[right]{$\begin{aligned} &+ \\ &V_a \\  &- \end{aligned}$};
\draw(3*\x-\dx,\y+\y/2)node[left]{$\begin{aligned} &+ \\ &V_0 \\  &- \end{aligned}$};
\draw(\x,0) node[ground]{};
\end{tikzpicture}%
\caption*{(الف)}
\end{subfigure}
%=====
\begin{subfigure}{1\textwidth}
\centering
\noindent
\begin{tikzpicture}
\draw(0,0) to [short]++(3*\x,0) to [american voltage source,l_={$\SI{5}{\volt}$}]++(0,\y)node[right]{$V_4$} to [resistor,*-,l_={$\SI{8}{\kilo\ohm}$}]++(0,\y) to [short]++(-\x,0)  to [american current source,l_={$\SI{2}{\milli\ampere}$}]++(-\x,0)to [short]++(-\x,0)node[above right]{$V_1+3V_a$}  to [resistor,l_={$\SI{4}{\kilo\ohm}$}]++(0,-2*\y);
\draw(\x,0) to [resistor,*-,l={$\SI{2}{\kilo\ohm}$}]++(0,\y) to [american controlled voltage source,-*,l={$3 V_a$}] ++(0,\y);
\draw(2*\x,2*\y)node[above]{$V_3$} to [resistor,*-,l={$\SI{1}{\kilo\ohm}$}]++(0,-\y) to [american current source,-*,l={$\SI{1}{\milli\ampere}$}]++(0,-\y);
\draw(\x,\y)node[left]{$V_1$} to [resistor,*-*,l={$\SI{6}{\kilo\ohm}$}]++(\x,0)node[right]{$V_2$};
\draw(\x+\dx,\y/2)node[right]{$\begin{aligned} &+ \\ &V_a \\  &- \end{aligned}$};
\draw(3*\x-\dx,\y+\y/2)node[left]{$\begin{aligned} &+ \\ &V_0 \\  &- \end{aligned}$};
\draw(\x,0) node[ground]{};
%supernode
\draw[dashed](\x,1.5*\y) circle (0.6cm and 1.4cm);
\end{tikzpicture}%
\caption*{(ب)}
\end{subfigure}
\caption{مثال \حوالہ{مثال_جوڑ_تابع_منبع_دباو_ب} کا دور۔}
\label{شکل_جوڑ_تابع_منبع_مخلوط_جوڑ_الف}
\end{figure}%

حل:شکل \حوالہ{شکل_جوڑ_تابع_منبع_مخلوط_جوڑ_الف}-ب میں جوڑ \عددی{V_1}، \عددی{V_2}،  \عددی{V_3} اور \عددی{V_4} کی نشاندہی کی گئی ہے۔مخلوط جوڑ کے نچلے سرے پر \عددی{V_1} دباو کی بدولت اس کے بالائی سرے پر \عددی{V_1+3V_a} دباو لکھا گیا ہے۔مخلوط جوڑ پر کرخوف قانون رو سے درج ذیل لکھا جائے گا۔
\begin{align*}
\frac{V_1+3V_a}{4000}-0.002+\frac{V_1}{2000}+\frac{V_1-V_2}{6000}=0
\end{align*}
اسی طرح \عددی{V_4=\SI{5}{\volt}} لیتے ہوئے  بالترتیب \عددی{V_2} اور \عددی{V_3}  جوڑ کے لئے کرخوف مساوات رو لکھتے ہیں۔
\begin{align*}
\frac{V_2-V_1}{6000}+0.001+\frac{V_2-V_3}{1000}&=0\\
0.002+\frac{V_3-V_2}{1000}+\frac{V_3-5}{8000}&=0
\end{align*}
مخلوط جوڑ کے مساوات میں \عددی{V_a=V_1} پر کرتے ہوئے مندرجہ بالا تین مساوات کو ایک ساتھ لکھتے ہیں۔
\begin{align*}
10 V_1-V_2&=12\\
-V_1+7V_2-6V_3&=-6\\
-8V_2+9V_3&=-11
\end{align*}
ان تین ہمزاد مساوات کو حل کرنے سے
\begin{align*}
V_1&=\frac{20}{47} \, \si{\volt}\\
V_2&=-\frac{364}{47} \, \si{\volt}\\
V_3&=-\frac{381}{47} \, \si{\volt}
\end{align*}
حاصل ہوتا ہے۔یوں
\begin{align*}
V_0&=V_3-V_4=-\frac{616}{47} \, \si{\volt}
\end{align*}
ہو گا۔
\انتہا{مثال}
%====================
\ابتدا{مثال}\شناخت{مثال_جوڑ_تابع_منبع_دباو_پ}
شکل \حوالہ{شکل_جوڑ_تابع_منبع_مخلوط_جوڑ_ب}-الف میں \عددی{I_0} دریافت کریں۔

\begin{figure}
\centering
\begin{subfigure}{1\textwidth}
\centering
\begin{tikzpicture}
\draw(0,0) to [american voltage source,l={$\SI{6}{\volt}$}]++(0,\y) to [resistor,l={$\SI{8}{\kilo\ohm}$}]++(0,\y) to [short]++(3*\x,0) to [american controlled current source,l={$2 I_a$}]++(0,-\y) to [resistor,l={$\SI{10}{\kilo\ohm}$}]++(0,-\y) to [short]++(-3*\x,0); 
\draw(0,\y) to [resistor,*-,l={$\SI{10}{\kilo\ohm}$}]++(\x,0) to [resistor,i>_={$I_0$},*-*,l={$\SI{2}{\kilo\ohm}$}]++(\x,0) to [resistor,-*,l={$\SI{8}{\kilo\ohm}$}]++(\x,0);
\draw(\x,2*\y) to [american controlled voltage source,*-,l={$3 V_b$}]++(0,-\y) to [resistor,i={$I_a$},-*,l={$\SI{4}{\kilo\ohm}$}]++(0,-\y);
\draw(2*\x,2*\y) to [american  voltage source,*-,l={$\SI{4}{\volt}$}]++(0,-\y);
\draw(3*\x-\dx,\y/2)node[left]{$\begin{aligned} &+ \\ &V_b \\ &-  \end{aligned}$};
\end{tikzpicture}
\caption*{(الف)}
\end{subfigure}
%====
\begin{subfigure}{1\textwidth}
\centering
\begin{tikzpicture}
\draw(0,0) to [american voltage source,l={$\SI{6}{\volt}$}]++(0,\y)node[left]{$V_1$} to [resistor,l={$\SI{8}{\kilo\ohm}$}]++(0,\y) to [short]++(1.5*\x,0)node[above]{$V_2$} to [short]++(1.5*\x,0) to [american controlled current source,l={$2 I_a$}]++(0,-\y)node[right]{$V_3$} to [resistor,l={$\SI{10}{\kilo\ohm}$}]++(0,-\y) to [short]++(-3*\x,0); 
\draw(0,\y) to [resistor,*-,l={$\SI{10}{\kilo\ohm}$}]++(\x,0)node[below right]{$V_4$} to [resistor,*-*,l={$\SI{2}{\kilo\ohm}$}]++(\x,0)node[below]{$V_5$} to [resistor,-*,l={$\SI{8}{\kilo\ohm}$}]++(\x,0);
\draw(\x,2*\y) to [american controlled voltage source,*-,l={$3 V_b$}]++(0,-\y) to [resistor,i={$I_a$},-*,l={$\SI{4}{\kilo\ohm}$}]++(0,-\y);
\draw(2*\x,2*\y) to [american  voltage source,*-,l={$\SI{4}{\volt}$}]++(0,-\y);
\draw(3*\x-\dx,\y/2)node[left]{$\begin{aligned} &+ \\ &V_b \\ &-  \end{aligned}$};
\draw(\x,0) node[ground]{};
%supernode
\draw[dashed](\x-\dx,\y-\dy) --(2*\x+\dx,\y-\dy)--(2*\x+\dx,2*\y+\dy)--(\x-\dx,2*\y+\dy)--cycle;

\end{tikzpicture}
\caption*{(ب)}
\end{subfigure}
\caption{مثال \حوالہ{مثال_جوڑ_تابع_منبع_دباو_پ} کا دور۔}
\label{شکل_جوڑ_تابع_منبع_مخلوط_جوڑ_ب}
\end{figure}%

حل: شکل \حوالہ{شکل_جوڑ_تابع_منبع_مخلوط_جوڑ_ب}-ب میں  نچلے جوڑ کو زمین چنتے ہوئے  بقایا جوڑوں کی نشاندہی کی گئی ہے۔مخلوط جوڑوں کو نقطہ دار چکور سے ظاہر کیا گیا ہے۔یہاں رک کر تسلی کر لیں کہ آپ مخلوط جوڑ کو پہچان سکتے ہیں۔مخلوط جوڑ کے نچلے بائیں اور دائیں سروں کے لئے درج ذیل لکھا جا سکتا ہے
\begin{align*}
V_4-V_2&=3V_b\\
V_5-V_2&=4
\end{align*}
جن  سے
\begin{align*}
V_4&=V_2+3V_b\\
V_5&=V_2+4
\end{align*}
حاصل ہوتا ہے۔شکل کو دیکھتے ہوئے
\begin{align*}
V_1&=6
\end{align*}
بھی لکھا جا سکتا ہے۔جوڑ \عددی{V_2} اور \عددی{V_3} کے کرخوف مساوات رو بالترتیب لکھتے ہیں جہاں \عددی{V_2} کی مساوات درحقیقت مخلوط جوڑ کی مساوات رو ہے۔
\begin{align*}
\frac{V_2-6}{8000}+\frac{(V_2+3V_b)-6}{10000}+\frac{(V_2+3V_b)}{4000}+\frac{(V_2+4)-V_3}{8000}+2I_a&=0\\
-2I_a+\frac{V_3-(V_2+4)}{8000}+\frac{V_3}{10000}&=0
\end{align*}
مندرجہ بالا دو مساوات میں درج ذیل پر کرتے
\begin{align*}
V_b&=V_3\\
I_a&=\frac{V_2+3V_b}{4000}=\frac{V_2+3V_3}{4000}\\
\end{align*}
ہوئے
\begin{align*}
\frac{V_2-6}{8000}+\frac{(V_2+3V_3)-6}{10000}+\frac{(V_2+3V_3)}{4000}+\frac{(V_2+4)-V_3}{8000}+2\left(\frac{V_2+3V_3}{4000}\right)&=0\\
-2\left(\frac{V_2+3V_3}{4000}\right)+\frac{V_3-(V_2+4)}{8000}+\frac{V_3}{10000}&=0
\end{align*}
یعنی
\begin{align*}
44V_2+97V_3&=34\\
50V_2+102V_3&=-40
\end{align*}
حاصل ہوتے ہیں جنہیں حل کرنے سے درج ذیل ملتے ہیں۔
\begin{align*}
V_2&=-\frac{3674}{181} \, \si{\volt}\\
V_3&=\frac{1730}{181} \, \si{\volt}
\end{align*}
یوں
\begin{align*}
I_0=\frac{V_4-V_5}{2000}=\frac{149}{12}\,\si{\milli\ampere}
\end{align*}
حاصل ہوتی ہے۔
\انتہا{مثال}

\FloatBarrier
%=====================
\ابتدا{مشق}\شناخت{مثال_جوڑ_تابع_منبع_مخلوط_جوڑ_مشق_الف}
شکل \حوالہ{شکل_جوڑ_تابع_منبع_مخلوط_جوڑ_مشق_الف} میں \عددی{I_0} حاصل کریں۔
\begin{figure}
\centering
\noindent
\begin{tikzpicture}
\draw(0,0) to [american current source,l={$\SI{4}{\milli\ampere}$}]++(0,\y) to [short]++(2*\x,0) to [american controlled voltage source,l={$2000 I_y$}]++(\x,0) to [short]++(\x,0) to [resistor,i={$I_y$},l={$\SI{4}{\kilo\ohm}$}]++(0,-\y) to [short]++(-4*\x,0);
\draw(\x,\y) to [american current source,*-*,l={$\SI{2}{\milli\ampere}$}]++(0,-\y);
\draw(2*\x,0) to [resistor,*-*,i<_={$I_0$},l_={$\SI{2}{\kilo\ohm}$}]++(0,\y);
\draw(3*\x,0) to [american current source,*-*,l_={$\SI{6}{\milli\ampere}$}]++(0,\y);
\draw(\x,0) node[ground]{};
\end{tikzpicture}%
\caption{مشق \حوالہ{مثال_جوڑ_تابع_منبع_مخلوط_جوڑ_مشق_الف} کا دور۔}
\label{شکل_جوڑ_تابع_منبع_مخلوط_جوڑ_مشق_الف}
\end{figure}%

جواب:\عددی{\SI{4}{\milli\ampere}}
\انتہا{مشق}
%=====================

%=====================
\ابتدا{مشق}\شناخت{مثال_جوڑ_تابع_منبع_مخلوط_جوڑ_مشق_ب}
شکل \حوالہ{شکل_جوڑ_تابع_منبع_مخلوط_جوڑ_مشق_ب} میں \عددی{V_0} حاصل کریں۔
\begin{figure}
\centering
\noindent
\begin{tikzpicture}
\draw(0,0) to [american current source,l={$\SI{6}{\milli\ampere}$}]++(0,\y) to [short]++(\x,0) to [american controlled voltage source,l={$2000 I_a$}]++(\x,0) to [resistor,l={$\SI{4}{\kilo\ohm}$}]++(\x,0) to [resistor,l={$\SI{8}{\kilo\ohm}$}]++(0,-\y) to [short]++(-3*\x,0);
\draw(3*\x,\y) to [short,*-]++(0,\y) to [american current source,l={$\SI{4}{\milli\ampere}$}]++(-2*\x,0) to [short,-*]++(0,-\y);
\draw(\x,0) to [resistor,i<_={$I_a$},*-*,l={$\SI{6}{\kilo\ohm}$}]++(0,\y);
\draw(2*\x,0) to [resistor,*-*,l={$\SI{2}{\kilo\ohm}$}]++(0,\y);
\draw(2*\x+\dx,\y/2)node[right]{$\begin{aligned} &+\\ &V_0 \\ &- \end{aligned}$};
\end{tikzpicture}%
\caption{مشق \حوالہ{مثال_جوڑ_تابع_منبع_مخلوط_جوڑ_مشق_ب} کا دور۔}
\label{شکل_جوڑ_تابع_منبع_مخلوط_جوڑ_مشق_ب}
\end{figure}%

جواب:\عددی{\tfrac{176}{17} \, \si{\volt}}
\انتہا{مشق}
%=====================
\FloatBarrier

\حصہ{دائری تجزیہ}
تجزیہ جوڑ میں نا معلوم متغیرات دباو جوڑ تھے جنہیں کرخوف قانون رو کی مدد سے حاصل کیا گیا۔جوڑ کے دباو جاننے کے بعد شاخوں کی رو کو قانون اوہم سے حاصل کیا گیا۔اس کے برعکس \اصطلاح{دائری ترکیب}\فرہنگ{دائری ترکیب}\حاشیہب{loop analysis}\فرہنگ{loop analysis} میں کرخوف قانون دباو کی مدد سے \اصطلاح{دائری رو}\فرہنگ{دائری رو}\فرہنگ{رو!دائری}\حاشیہب{loop current}\فرہنگ{loop current} دریافت کئے جاتے ہیں۔دائری رو جانتے ہوئے کسی بھی شاخ کا دباو قانون اوہم سے حاصل کیا جا سکتا ہے۔ایسا دور جس میں \عددی{J} جوڑ اور \عددی{S} شاخ  پائے جاتے ہوں سے \عددی{S-J+1} آزاد مساوات  بذریعہ کرخوف قانون دباو حاصل کئے جا سکتے ہیں۔شکل \حوالہ{شکل_جوڑ_دائری_ترکیب_الف} کو مثال بناتے ہوئے ہم دیکھتے ہیں کہ اس میں \عددی{J=5} اور \عددی{S=8} ہیں لہٰذا اس سے \عددی{8-5+1=4} آزاد مساوات حاصل کئے جا سکتے ہیں جن سے دائری رو \عددی{i_A}، \عددی{i_B}، \عددی{i_C} اور \عددی{i_D} حاصل ہوں گے۔دائری رو جانتے ہوئے شاخوں کی رو درج ذیل لکھی جا سکتی ہیں۔
\begin{align*}
i_1&=-i_A\\
i_2&=i_B-i_A\\
i_3&=i_B-i_C\\
i_4&=i_C\\
i_5&=i_D\\
i_6&=i_B\\
i_7&=i_D-i_A
\end{align*}
%===
\begin{figure}
\centering
\begin{tikzpicture}
\draw(0,0) to [resistor,i<_={$i_1$},l={$R_1$}]++(0,\yy)node[left]{$v_A$} to [american current source,l={$i_6$}]++(0,\yy) to [short]++(2*\xx,0) to [resistor,i>_={$i_4$},l={$R_4$}]++(0,-\yy) to [resistor,i={$i_5$},l={$R_5$}]++(0,-\yy) to [short]++(-2*\xx,0);
\draw(2*\xx,\yy)node[right]{$v_D$} to [american voltage source,i_={$i_8$},*-*]++(-\xx,0)node[above left]{$v_C$}  to [resistor,i={$i_2$},-*,l_={$R_2$}]++(-\xx,0);
\draw(\xx,0)node[below]{$v_E$} to [american voltage source,i>^={$i_7$},*-]++(0,\yy) to [resistor,-*,i<={$i_3$},l={$R_3$}]++(0,\yy)node[above]{$v_B$};
\draw(\xx,\yy/2)node[shift={(-45:0.75)}]{$v_1$};
\draw(\xx+\xx/2,\yy)node[shift={(-45:0.75)}]{$v_2$};
\draw[stealth-]([shift={(-150:\xx/5.5)}]\xx/2,\yy/2) arc (-150:150:\xx/5.5);
\draw(\xx/2,\yy/2)node{$i_A$};
\draw[stealth-]([shift={(-150:\xx/5.5)}]\xx/2,\yy+\yy/2) arc  (-150:150:\xx/5.5);
\draw(\xx/2,\yy+\yy/2)node{$i_B$};
\draw[stealth-]([shift={(-150:\xx/5.5)}]\xx+\xx/2,\yy+\yy/2) arc (-150:150:\xx/5.5);
\draw(\xx+\xx/2,\yy+\yy/2)node{$i_C$};
\draw[stealth-]([shift={(-150:\xx/5.5)}]\xx+\xx/2,\yy/2) arc (-150:150:\xx/5.5);
\draw(\xx+\xx/2,\yy/2)node{$i_D$};
\end{tikzpicture}%
\caption{دائری ترکیب کی مثال۔}
\label{شکل_جوڑ_دائری_ترکیب_الف}
\end{figure}%

اس کتاب میں صرف \اصطلاح{سطحی ادوار}\فرہنگ{دور!سطحی}\فرہنگ{سطحی!دور}\حاشیہب{planar circuits}\فرہنگ{planar} پر غور کیا جائے گا۔سطحی دور سے مراد ایسا دور ہے جسے کاغذ پر یوں بنایا جا سکتا ہو کہ کوئی تار دوسری تار کو نہ کاٹے۔سطحی ادوار میں دائروں کی نشاندہی نسبتاً آسان ہوتی ہے۔دائری ترکیب میں دائری رو یوں چنی جاتی ہیں کہ تمام شاخوں سے کم از کم ایک دائری رو گزرے، مزید یہ کہ ہر دائری رو کم از کم ایک ایسے شاخ سے گزرے جس سے کوئی دوسری دائری رو نہ گزرتی ہو۔

آئیں دائری ترکیب کو چند مثالوں کی مدد سے سمجھیں۔

\حصہ{غیر تابع منبع استعمال کرنے والے ادوار}
 شکل \حوالہ{شکل_جوڑ_آزاد_منبع_دباو_دائری_مثال} میں دو عدد غیر تابع منبع دباو استعمال کئے گئے ہیں۔اس دور میں  سات شاخ اور چھ جوڑ ہیں لہٰذا دور میں تمام نا معلوم دائری رو دریافت کرنے کی خاطر \عددی{7-6+1=2} غیر تابع مساوات درکار ہیں جنہیں کرخوف قانون دباو سے حاصل کیا جائے گا۔چونکہ دو عدد دائری رو درکار ہیں لہٰذا ہم دو عدد دائرے چنتے ہیں۔ہم ان دائروں کو مختلف انداز میں چن سکتے ہیں۔یوں ہم پہلا دائرہ \عددی{abcfa} اور دوسرا دائرہ \عددی{cdefc} چن سکتے ہیں۔ایسا کرتے ہوئے دائری رو \عددی{i_1} اور \عددی{i_2} ہوں گے جنہیں  شکل \حوالہ{شکل_جوڑ_آزاد_منبع_دباو_دائری_مثال}-الف میں دکھایا گیا ہے۔ یہاں ہم نے دونوں دائری رو کو گھڑی کی سمت تصور کیا ہے۔حقیقت میں آپ دونوں رو کو گھڑی کے الٹ بھی تصور کر سکتے ہیں اور ایسا بھی کر سکتے ہیں کہ ایک دائری رو گھڑی کی سمت اور دوسری رو گھڑی کی الٹ تصور کی جائے۔ترکیب جوڑ کی طرح یہاں بھی اگر حقیقت میں کسی رو کی سمت تصور کردہ سمت کے الٹ ہو تو ایسی صورت میں رو کی قیمت منفی حاصل ہو گی۔اس کتاب میں ہم دائری رو کو گھڑی کی سمت ہی تصور کریں گے۔ اسی طرح ہم دو عدد دائرے یوں بھی چن سکتے ہیں کہ پہلا دائرہ \عددی{abcfa} اور دوسرا دائرہ \عددی{abdea} لیں جن سے شکل \حوالہ{شکل_جوڑ_آزاد_منبع_دباو_دائری_مثال}-ب میں دکھائے دائری رو ملتے ہیں۔ہم باری باری شکل \حوالہ{شکل_جوڑ_آزاد_منبع_دباو_دائری_مثال}-الف اور شکل \حوالہ{شکل_جوڑ_آزاد_منبع_دباو_دائری_مثال}-ب کو حل کرتے ہیں۔

\begin{figure}
\centering
\begin{subfigure}{1\textwidth}
\centering
\begin{tikzpicture}
\draw(0,0)node[below]{$a$} to [american voltage source,l={$v_{M1}$}]++(0,\yy)node[above]{$b$} to [resistor,l_={$R_1$}]++(\xx,0)node[above]{$c$} to [resistor,l_={$R_4$}]++(0,-\yy) to [resistor,l_={$R_3$}]++(-\xx,0);
\draw(\xx,0)node[below]{$f$} to [american voltage source,*-,l_={$v_{M2}$}]++(\xx,0)node[below]{$e$} to [resistor,l={$R_5$}]++(0,\yy)node[above]{$d$} to [resistor,-*,l={$R_2$}]++(-\xx,0);
%loop currents
\draw[stealth-]([shift={(-150:\xx/5.5)}]\xx/2,\yy/2) arc (-150:150:\xx/5.5);
\draw(\xx/2,\yy/2)node{$i_1$};
\draw[stealth-]([shift={(-150:\xx/5.5)}]\xx+\xx/2,\yy/2) arc (-150:150:\xx/5.5);
\draw(\xx+\xx/2,\yy/2)node{$i_2$};
%branch voltages
\draw(\xx/2,\yy+\dy)node[above]{$+ \, v_1 \, -$};
\draw(\xx+\xx/2,\yy+\dy)node[above]{$+ \, v_2 \, -$};
\draw(\xx/2,-\dx)node[below]{$- \, v_3 \, +$};
\draw(\xx+\dx,\yy/2)node[right]{$\begin{aligned} &+ \\& v_4 \\ &- \end{aligned}$};
\draw(2*\xx+\dx,\yy/2)node[right]{$\begin{aligned} &+ \\& v_5 \\ &- \end{aligned}$};
\end{tikzpicture}%
\caption*{(الف)}
\end{subfigure}
%=====
\begin{subfigure}{1\textwidth}
\centering
\begin{tikzpicture}
\draw(0,0)node[below]{$a$} to [american voltage source,l={$v_{M1}$}]++(0,\yy)node[above]{$b$} to [resistor]++(\xx,0)node[above]{$c$} to [resistor,l_={$R_4$}]++(0,-\yy) to [resistor]++(-\xx,0);
\draw(\xx,0)node[below]{$f$} to [american voltage source,*-,l_={$v_{M2}$}]++(\xx,0)node[below]{$e$} to [resistor]++(0,\yy)node[above]{$d$} to [resistor,-*]++(-\xx,0);
%shifted labels
\draw(1/4*\xx,\yy)node[below]{$R_1$};
\draw(2*\xx-1/4*\xx+\dx,\yy)node[below]{$R_2$};
\draw(1/4*\xx,\dy)node[above]{$R_3$};
\draw(2*\xx-2*\dx,\dy)node[above]{$R_5$};
%loop currents
\draw[stealth-]([shift={(-150:\xx/6)}]\xx/2,\yy/2) arc (-150:150:\xx/6);
\draw(\xx/2,\yy/2)node{$i_1$};
\draw[-stealth]([shift={(150:7/24*\xx)}]\xx/2,\yy/2) arc (150:90:7/24*\xx)--++(\xx,0) arc (90:-90:7/24*\xx)--++(-\xx,0) arc (-90:-150:7/24*\xx);
\draw(\xx+\xx/2,\yy/2)node{$i_2$};
%branch voltages
\draw(\xx/2,\yy+\dy)node[above]{$+ \, v_1 \, -$};
\draw(\xx+\xx/2,\yy+\dy)node[above]{$+ \, v_2 \, -$};
\draw(\xx/2,-\dx)node[below]{$- \, v_3 \, +$};
\draw(\xx+\dx,\yy/2)node[right]{$\begin{aligned} &+ \\& v_4 \\ &- \end{aligned}$};
\draw(2*\xx+\dx,\yy/2)node[right]{$\begin{aligned} &+ \\& v_5 \\ &- \end{aligned}$};
\end{tikzpicture}%
\caption*{(ب)}
\end{subfigure}%
\caption{غیر تابع منبع دباو استعمال کرنے والا دور۔}
\label{شکل_جوڑ_آزاد_منبع_دباو_دائری_مثال}
\end{figure}%

شکل \حوالہ{شکل_جوڑ_آزاد_منبع_دباو_دائری_مثال}-الف میں دونوں دائروں پر کرخوف قانون دباو سے درج ذیل لکھا جاتا ہے۔
\begin{gather}
\begin{aligned}\label{مساوات_جوڑ_دائری_آزاد_منبع_الف}
-v_{M1}+v_1+v_4+v_3&=0\\
-v_4+v_2+v_5+v_{M2}&=0
\end{aligned}
\end{gather}
قانون اوہم سے دباو شاخ درج ذیل لکھے جا سکتے ہیں
\begin{align*}
v_1&=i_1 R_1\\
v_2&=i_2 R_2\\
v_3&=i_1 R_3\\
v_4&=(i_1-i_2)R_4\\
v_5&=i_2 R_5
\end{align*}
جنہیں مساوات \حوالہ{مساوات_جوڑ_دائری_آزاد_منبع_الف} میں پر کرنے سے درج ذیل حاصل ہوتا ہے۔
\begin{align*}
i_1(R_1+R_3+R_4)-i_2 R_4&=v_{M1}\\
-i_1 R_4+i_2(R_2+R_4+R_5)&=-v_{M2}
\end{align*}
اس کو قالبی صورت میں لکھتے ہیں۔
\begin{align}
\begin{bmatrix}
R_1+R_3+R_4 & -R_4\\
-R_4& R_2+R_4+R_5
\end{bmatrix}
\begin{bmatrix}
i_1\\
i_2
\end{bmatrix}
=
\begin{bmatrix}
v_{M1}\\-v_{M2}
\end{bmatrix}
\end{align}
%===============

شکل \حوالہ{شکل_جوڑ_دائری_ترکیب_الف} یا شکل \حوالہ{شکل_جوڑ_آزاد_منبع_دباو_دائری_مثال}-الف بالکل ماہی گیر کے جال کی مانند ہیں جسے مچھلیاں پکڑنے کے لئے استعمال کیا جاتا ہے۔ان اشکال میں ہر خانے میں دائری رو چنی گئی ہے۔اس کے برعکس شکل \حوالہ{شکل_جوڑ_آزاد_منبع_دباو_دائری_مثال}-ب میں \عددی{i_3} کو یوں چنا گیا ہے کہ یہ \عددی{i_1} کو بھی لپیٹتی ہے۔اس کتاب میں انفرادی خانے کی رو ہی چنتے ہوئے ادوار حل کئے جائیں گے۔ 

%===============

%=========================

\ابتدا{مثال}
شکل \حوالہ{شکل_جوڑ_آزاد_منبع_دباو_دائری_مثال_الف}-الف میں دائری رو دریافت کرتے ہوئے تمام شاخوں کی رو اور دباو حاصل کریں۔
\begin{figure}
\centering
\begin{subfigure}{1\textwidth}
\centering
\begin{tikzpicture}
\draw(0,0) to [american voltage source,l={$\SI{18}{\volt}$}]++(0,\yyy) to [resistor,l={$\SI{4}{\kilo\ohm}$}]++(\xxx,0) to [resistor]++(0,-\yyy) to [resistor,l={$\SI{5}{\kilo\ohm}$}]++(-\xxx,0);
\draw(\xxx,0) to [american voltage source,*-,l_={$\SI{15}{\volt}$}]++(\xxx,0) to [resistor,l_={$\SI{1}{\kilo\ohm}$}]++(0,\yyy) to [resistor,-*,l_={$\SI{2}{\kilo\ohm}$}]++(-\xxx,0);
\draw(\xxx,2/3*\yyy+\dy)node[above, left]{$\SI{3}{\kilo\ohm}$};
%loop currents
\draw[stealth-]([shift={(-150:\xxx/5.5)}]\xxx/2,\yyy/2) arc (-150:150:\xxx/5.5);
\draw(\xxx/2,\yyy/2)node{$i_1$};
\draw[stealth-]([shift={(-150:\xxx/5.5)}]\xxx+\xxx/2,\yyy/2) arc (-150:150:\xxx/5.5);
\draw(\xxx+\xxx/2,\yyy/2)node{$i_2$};
\draw(\xxx,0)node[ground]{};
\end{tikzpicture}%
\caption*{(الف)}
\end{subfigure}
%=====
\begin{subfigure}{1\textwidth}
\centering
\begin{tikzpicture}
\draw(0,0) to [american voltage source,l={$\SI{18}{\volt}$}]++(0,\yyy) to [resistor,i>^={$\SI{1}{\milli\ampere}$},l_={$\SI{4}{\kilo\ohm}$}]++(\xxx,0) to [resistor,i={$\SI{3}{\milli\ampere}$},l={$\SI{3}{\kilo\ohm}$}]++(0,-\yyy) to [resistor,i^>={$\SI{1}{\milli\ampere}$},l_={$\SI{5}{\kilo\ohm}$}]++(-\xxx,0);
\draw(\xxx,0) to [american voltage source,*-,l_={$\SI{15}{\volt}$}]++(\xxx,0) to [resistor,i_={$\SI{2}{\milli\ampere}$},l_={$\SI{1}{\kilo\ohm}$}]++(0,\yyy) to [resistor,i>_={$\SI{2}{\milli\ampere}$},-*,l={$\SI{2}{\kilo\ohm}$}]++(-\xxx,0);
\draw(\xxx,0)node[ground]{};
%branch voltages
\draw(\xxx-\dx,\yyy/2)node[left]{$\begin{aligned} &+ \\ &\SI{9}{\volt} \\ &- \end{aligned}$};
\draw(2*\xxx-\dx,\yyy/2)node[left]{$\begin{aligned} &- \\ &\SI{2}{\volt} \\ &+ \end{aligned}$};
\draw(\xxx/2,\yyy+\dy)node[above]{$+ \, \SI{4}{\volt} \, -$};
\draw(\xxx+\xxx/2,\yyy+\dy)node[above]{$- \, \SI{4}{\volt} \, +$};
\draw(\xxx/2,-\dy)node[below]{$- \, \SI{5}{\volt} \, +$};
\end{tikzpicture}%
\caption*{(ب)}
\end{subfigure}
\caption{غیر تابع منبع دباو استعمال کرنے والا دور کی مثال۔}
\label{شکل_جوڑ_آزاد_منبع_دباو_دائری_مثال_الف}
\end{figure}%

حل: کرخوف قانون دباو کی مدد سے بالترتیب بائیں اور دائیں خانوں کے لئے درج ذیل لکھتے ہیں۔
\begin{align*}
-18+4000 i_1 +3000(i_1-i_2)+5000 i_1&=\\
3000(i_2-i_1)+2000 i_2+1000 i_2+15&=0
\end{align*} 
انہیں ترتیب دیتے ہوئے یوں لکھا جا سکتا ہے۔
\begin{align*}
12000 i_1-3000 i_2&=18\\
-3000 i_1+6000 i_2&=-15
\end{align*}
کسی بھی ترکیب سے ان ہمزاد مساوات کو حل کیا جا سکتا ہے۔حاصل جوابات درج ذیل ہیں۔
\begin{align*}
i_1&=\SI{1}{\milli\ampere}\\
i_2&=\SI{-2}{\milli\ampere}
\end{align*}
چونکہ \عددی{i_2} کی قیمت منفی ہے لہٰذا حقیقت میں دائیں خانے میں رو کی سمت چنی گئی سمت کے الٹ ہے۔ان قیمتوں کو شکل-ب میں دکھایا گیا ہے۔کسی بھی مزاحمت کا دباو قانون اوہم سے حاصل کیا جا سکتا ہے۔تمام مزاحمتوں کے دباو شکل-ب میں دکھائے گئے ہیں۔امید کی جاتی ہے کہ آپ انہیں حاصل کر پائیں گے۔
\انتہا{مثال}
%===================
\ابتدا{مثال}
شکل \حوالہ{شکل_جوڑ_رو_دوسری_کو_لپیٹتی_ہے} کو حل کرتے ہوئے  تمام شاخوں کی رو دریافت کریں۔

\begin{figure}
\centering
\begin{tikzpicture}
\draw(0,0) to [american voltage source,l={$\SI{18}{\volt}$}]++(0,\yy) to [resistor,l={$\SI{4}{\kilo\ohm}$}]++(\xx,0) to [resistor,l={$\SI{3}{\kilo\ohm}$}]++(0,-\yy) to [resistor,l={$\SI{5}{\kilo\ohm}$}]++(-\xx,0);
\draw(\xx,0) to [american voltage source,*-,l_={$\SI{15}{\volt}$}]++(\xx,0) to [resistor,l_={$\SI{1}{\kilo\ohm}$}]++(0,\yy) to [resistor,-*,l_={$\SI{2}{\kilo\ohm}$}]++(-\xx,0);
\draw(\xx,0)node[ground]{};
%loop currents
\draw[stealth-]([shift={(-150:\xx/6)}]\xx/2,\yy/2) arc (-150:150:\xx/6);
\draw(\xx/2,\yy/2)node{$i_1$};
\draw[-stealth]([shift={(150:7/24*\xx)}]\xx/2,\yy/2) arc (150:90:7/24*\xx)--++(\xx,0) arc (90:-90:7/24*\xx)--++(-\xx,0) arc (-90:-150:7/24*\xx);
\draw(\xx+2/3*\xx,\yy/2)node{$i_3$};
\end{tikzpicture}%
\caption{ہر خانے کی علیحدہ رو تصور نہیں کرتے ہوئے حل کرتے ہیں۔}
\label{شکل_جوڑ_رو_دوسری_کو_لپیٹتی_ہے}
\end{figure}%%
حل:بائیں خانے کے لئے کرخوف قانون دباو سے
\begin{align*}
-18+4000(i_1+i_2)+3000 i_1+5000(i_1+i_2)&=0
\end{align*}
لکھا جا سکتا ہے۔بیرونی دائرے کے لئے درج ذیل لکھا جائے گا۔
\begin{align*}
-18+4000(i_1+i_2)+2000i_2 +1000i_2 +15+5000(i_1+i_2)&=0
\end{align*}
ان میں رو کو ترتیب سے لکھتے ہیں۔
\begin{align*}
12000 i_1+9000 i_2&=18\\
9000i_1+12000i_2&=3
\end{align*}
ان ہمزاد مساوات کو حل کرنے سے درج ذیل حاصل ہوتی ہیں۔
\begin{align*}
i_1&=\SI{3}{\milli\ampere}\\
i_2&=\SI{-2}{\milli\ampere}
\end{align*}
آپ گزشتہ مثال کے جوابات کے ساتھ موازنہ کر سکتے ہیں مثلاً بالائی \عددی{\SI{4}{\kilo\ohm}} میں \عددی{\SI{3}{\milli\ampere}-\SI{2}{\milli\ampere}=\SI{1}{\milli\ampere}} اور درمیانے \عددی{\SI{3}{\kilo\ohm}} میں \عددی{\SI{3}{\milli\ampere}} رو پائے جاتے ہیں۔
\انتہا{مثال}
%====================

مندرجہ بالا دو مثالوں میں ایک ہی دور میں دو مختلف طرز کے رو چنتے ہوئے حل کیا گیا۔دونوں حاصل جواب یکساں حاصل ہوئے۔آپ دیکھ سکتے ہیں کہ حاصل جواب چنی گئی رو پر منحصر نہیں ہے۔
%==================

\ابتدا{مثال}
شکل \حوالہ{شکل_جوڑ_دائری_ترکیب_عمومی} کے کرخوف مساوات دباو کو قالبی صورت میں لکھیں۔
\begin{figure}
\centering
\begin{tikzpicture}
\draw(0,0) to [resistor,l={$R_1$}]++(0,\yy) to [american voltage source,l={$v_A$}]++(0,\yy) to [short]++(2*\xx,0) to [resistor,l={$R_4$}]++(0,-\yy) to [resistor,l={$R_5$}]++(0,-\yy) to [short]++(-2*\xx,0);
\draw(2*\xx,\yy) to [american voltage source,*-*]++(-\xx,0)  to [resistor,-*,l_={$R_2$}]++(-\xx,0);
\draw(\xx,0) to [american voltage source,*-]++(0,\yy) to [resistor,-*,l={$R_3$}]++(0,\yy);
\draw(\xx,\yy/2)node[shift={(-45:0.75)}]{$v_B$};
\draw(\xx+\xx/2,\yy)node[shift={(-45:0.75)}]{$v_C$};
\draw[stealth-]([shift={(-150:\xx/5.5)}]\xx/2,\yy/2) arc (-150:150:\xx/5.5);
\draw(\xx/2,\yy/2)node{$i_A$};
\draw[stealth-]([shift={(-150:\xx/5.5)}]\xx/2,\yy+\yy/2) arc  (-150:150:\xx/5.5);
\draw(\xx/2,\yy+\yy/2)node{$i_B$};
\draw[stealth-]([shift={(-150:\xx/5.5)}]\xx+\xx/2,\yy+\yy/2) arc (-150:150:\xx/5.5);
\draw(\xx+\xx/2,\yy+\yy/2)node{$i_C$};
\draw[stealth-]([shift={(-150:\xx/5.5)}]\xx+\xx/2,\yy/2) arc (-150:150:\xx/5.5);
\draw(\xx+\xx/2,\yy/2)node{$i_D$};
\end{tikzpicture}%
\caption{دائری ترکیب کی مثال۔}
\label{شکل_جوڑ_دائری_ترکیب_عمومی}
\end{figure}%%

حل:ہم بالترتیب \عددیء{i_A}، \عددیء{i_B} \عددیء{i_C} اور \عددی{i_D}  کو استعمال کرتے ہوئے  درج ذیل مساوات لکھ سکتے ہیں۔
\begin{align*}
i_A R_1+(i_A-i_B)R_2+v_B&=0\\
-v_A+(i_B-i_C)R_3+(i_B-i_A)R_2&=0\\
(i_C-i_B)R_3+i_C R_4-v_C&=0\\
-v_B+v_C+i_D R_5&=0
\end{align*}
انہیں ترتیب دیتے ہوئے دوبارہ لکھتے ہوئے
\begin{align*}
i_A(R_1+R_2)-i_B R_2&=-v_B\\
-i_A R_2+i_B(R_2+R_3)-i_C R_3&=v_A\\
-i_B R_3 +i_C(R_3+R_4)&=v_C\\
i_D R_5&=v_B-v_C
\end{align*}
قالبی صورت میں لکھا جا سکتا ہے۔
\begin{align*}
\begin{bmatrix}
R_1+R_2& -R_2 & 0&0\\
-R_2 & R_2+R_3&-R_3&0\\
0&-R_3&R_3+R_4&0\\
0&0&0&R_5
\end{bmatrix}
\begin{bmatrix}
i_A\\
i_B\\
i_C\\
i_D
\end{bmatrix}
=
\begin{bmatrix}
-v_B\\
v_A\\
v_C\\
v_B-v_C
\end{bmatrix}
\end{align*}
\انتہا{مثال}
%=============================


مندرجہ بالا قالبی مساوات میں پہلی صف  (یعنی بالائی صف) کا پہلا جزو (یعنی بایاں جزو) ان مزاحمتوں کا مجموعہ ہے جن سے \عددی{i_A} گزرتی ہے یعنی \عددی{R_1+R_2} جبکہ اسی پہلی صف کا دوسرا جزو ان مزاحمتوں کے مجموعے کا منفی ہے جن سے \عددی{i_A} اور \عددی{i_B} دونوں گزرتی ہیں۔اسی طرح تیسرا جزو \عددی{i_A} اور \عددی{i_C} کا مشترکہ مزاحمتوں کا منفی ہے۔چونکہ موجودہ دور میں ایسا کوئی مزاحمت نہیں پایا جاتا جن سے \عددی{i_A} اور \عددی{i_C} دونوں گزرتی ہوں لہٰذا یہ جزو صفر کے برابر ہے۔اسی طرح پہلی صف کا چوتھا  جزو \عددی{i_A} اور \عددی{i_D} کے مشترک مزاحمتوں کے مجموعے کے منفی کے برابر ہے جو موجودہ دور میں صفر کے برابر ہے۔قالب کے دوسرے صف کا پہلا جزو \عددی{i_B} اور \عددی{i_A} کے مشترکہ مزاحمتوں کے مجموعے کا منفی ہے۔دوسرے صف کا دوسرا جزو ان مزاحمتوں کا مجموعہ ہے جن سے \عددی{i_B} گزرتی ہے جبکہ صف کا تیسرا جزو \عددی{i_B} اور \عددی{i_C} کے مشترک مزاحمتوں کے مجموعے کا منفی ہے۔اسی ترکیب سے تیسرا صف \عددی{i_C} کے مطابق اور چوتھا صف \عددی{i_D} کے مطابق لکھا جاتا ہے۔قالبی مساوات کا دایاں ہاتھ بالترتیب \عددی{i_A}، \عددی{i_B}، \عددی{i_C} اور \عددی{i_D} کی سمت میں گھومتے ہوئے منبع دباو کے مجموعی دباو میں اضافے کے برابر ہے۔چونکہ \عددی{i_A} کی سمت میں گھومتے ہوئے صرف منبع \عددی{v_B} سے گزرا جاتا ہے اور گھومنے کی سمت میں منبع کا دباو گھٹتا ہے لہٰذا قالبی مساوات کے بائیں ہاتھ پہلا جزو \عددی{-v_B} لکھا جائے گا۔آپ سے گزارش ہے کہ یہاں رک کر تسلی کر لیں کہ آپ قالبی مساوات کے تمام اجزاء یوں لکھ سکتے ہیں۔ 

اگر تمام خانوں میں، ایک ہی سمت میں گھومتی انفرادی دائری رو  تصور کی جائے تب عمومی قالبی مساوات درج ذیل لکھی جا سکتی ہے۔
\begin{align}\label{مساوات_جوڑ_عمومی_مساوات_متعدد_منبع}
\begin{bmatrix}
R_{11} & -R_{12}& -R_{13}& \cdots -R_{1m}\\
-R_{21} & R_{22}& -R_{23}& \cdots -R_{2m}\\
-R_{31} & -R_{32}& R_{33}& \cdots -R_{3m}\\
\vdots\\
-R_{m1}&-R_{m2}&-R_{m3}&\cdots R_{mm}
\end{bmatrix}
\begin{bmatrix}
i_1\\
i_2\\
i_3\\
\vdots\\
i_m
\end{bmatrix}
=
\begin{bmatrix}
v_{1}\\
v_{2}\\
v_{3}\\
\vdots\\
v_{m}
\end{bmatrix}
\end{align}
اس عمومی قالبی مساوات میں بائیں ہاتھ مزاحمتی قالب کے بالائی بائیں کونے سے نچلی دائیں کونے  تک ترچھی لکیر پر پائے جانے والے اجزاء مثبت ہیں جبکہ بقایا تمام منفی ہیں۔اس مساوات میں \عددی{R_{11}} سے مراد ان مزاحمتوں کا مجموعہ ہے جن سے \عددی{i_1} گزرتی ہے جبکہ \عددی{R_{12}} ان مزاحمتوں کا مجموعہ ہے جن سے \عددی{i_1} اور \عددی{i_2} دونوں گزرتی ہیں۔تمام خانوں میں رو کی سمت یکساں ہونے کی صورت میں دو پڑوسی خانوں کے مشترک شاخ میں پڑوسی رو الٹ سمت میں پائی جاتی ہے۔
%=================
\ابتدا{مثال}\شناخت{مثال_جوڑ_تین_دائری_رو_الف}
شکل \حوالہ{شکل_جوڑ_تین_دائری_رو_مثال_الف} میں نا معلوم رو حاصل کریں۔
\begin{figure}
\centering
\begin{tikzpicture}
\draw(0,0) to [resistor,l={$\SI{1}{\kilo\ohm}$}]++(0,\yy) to [american voltage source,l={$\SI{12}{\volt}$}]++(\xx,0) to [resistor]++(0,-\yy) to [resistor,l={$\SI{3}{\kilo\ohm}$}]++(-\xx,0);
\draw(\xx,0) to [american voltage source,*-,l_={$\SI{6}{\volt}$}]++(\xx,0) to [resistor,l_={$\SI{2}{\kilo\ohm}$}]++(0,\yy) to [resistor,-*,l_={$\SI{1}{\kilo\ohm}$}]++(-\xx,0);
\draw(0,\yy) to [short,*-]++(0,\yy) to [resistor,l={$\SI{5}{\kilo\ohm}$}]++(2*\xx,0) to [short,-*]++(0,-\yy);
\draw(\xx+\dx,3/4*\yy)node[right]{$\SI{4}{\kilo\ohm}$};
%loop currents
\draw[stealth-]([shift={(-150:\xx/6)}]\xx,\yy+\yy/2) arc (-150:150:\xx/6);
\draw(\xx,\yy+\yy/2)node{$i_1$};
\draw[stealth-]([shift={(-150:\xx/6)}]\xx/2,\yy/2) arc (-150:150:\xx/6);
\draw(\xx/2,\yy/2)node{$i_2$};
\draw[stealth-]([shift={(-150:\xx/6)}]\xx+\xx/2,\yy/2) arc (-150:150:\xx/6);
\draw(\xx+\xx/2,\yy/2)node{$i_3$};
\end{tikzpicture}
\caption{مثال \حوالہ{مثال_جوڑ_تین_دائری_رو_الف} کا دور۔}
\label{شکل_جوڑ_تین_دائری_رو_مثال_الف}
\end{figure}%%

حل:آئیں شکل کو دیکھ کر قالبی مساوات لکھیں۔
\begin{align*}
\begin{bmatrix}
\SI{5}{\kilo\ohm}+\SI{1}{\kilo\ohm} & 0 & -\SI{1}{\kilo\ohm}\\
0 & \SI{3}{\kilo\ohm}+\SI{1}{\kilo\ohm}+\SI{4}{\kilo\ohm}&-\SI{4}{\kilo\ohm}\\
-\SI{4}{\kilo\ohm}&-\SI{1}{\kilo\ohm}&\SI{4}{\kilo\ohm}+\SI{1}{\kilo\ohm}+\SI{2}{\kilo\ohm}
\end{bmatrix}
\begin{bmatrix}
i_1\\
i_2\\
i_3
\end{bmatrix}
=
\begin{bmatrix}
-\SI{12}{\volt}\\
\SI{12}{\volt}\\
-\SI{6}{\volt}\\
\end{bmatrix}
\end{align*}
اسے یوں لکھتے ہوئے
\begin{align*}
\begin{bmatrix}
6000& 0 & -1000\\
0 &8000&-4000\\
-4000&-1000&7000
\end{bmatrix}
\begin{bmatrix}
i_1\\
i_2\\
i_3
\end{bmatrix}
=
\begin{bmatrix}
-12\\
12\\
-6\\
\end{bmatrix}
\end{align*}
یہ عمومی قالبی مساوات
\begin{align*}
{\bf{R I}}={\bf{V}}
\end{align*}
ہے جس کا حل
\begin{align*}
{\bf{I}}={\bf{R^{-1}V}}
\end{align*}
ہے۔قالبی مساوات کو حل کرنے سے دائری رو درج ذیل حاصل ہوتی ہیں۔
\begin{align*}
i_1&=-\frac{33}{14} \, \si{\milli\ampere}\\
i_2&=\frac{3}{7} \, \si{\milli\ampere}\\
i_3&=\frac{15}{7} \, \si{\milli\ampere}
\end{align*}
\انتہا{مثال}
%==================

\حصہ{غیر تابع منبع رو استعمال کرنے والے ادوار}
منبع دباو کی موجودگی سے ترکیب جوڑ کا استعمال نسبتاً آسان ہو جاتا ہے۔بالکل اسی طرح منبع رو کی موجودگی سے دائری ترکیب کا استعمال نسبتاً آسان ہو جاتا ہے۔آئیں یہ حقیقت چند مثال حل کرتے ہوئے دیکھیں۔

%=======================
\ابتدا{مثال}
شکل \حوالہ{شکل_جوڑ_منبع_رو_آسان_ترکیب_دائری_الف} میں \عددی{V_1} اور \عددی{V_2} دائری ترکیب استعمال کرتے ہوئے حاصل کریں۔

\begin{figure}
\centering
\begin{tikzpicture}
\draw(0,0) to [american voltage source,l={$\SI{6}{\volt}$}]++(0,\yyy) to [resistor,l={$\SI{8}{\kilo\ohm}$}]++(\xxx,0) to [resistor,l={$\SI{2}{\kilo\ohm}$}]++(\xxx,0) to [american current source,l={$\SI{4}{\milli\ampere}$}]++(0,-\yyy) to [short]++(-2*\xxx,0);
\draw(\xxx,0) to [resistor,*-*]++(0,\yyy);
\draw(\xxx,3/4*\yyy)node[left]{$\SI{4}{\kilo\ohm}$};
%loop currents
\draw[stealth-]([shift={(-150:\xxx/5.5)}]\xxx/2,\yyy/2) arc (-150:150:\xxx/5.5);
\draw(\xxx/2,\yyy/2)node{$i_1$};
\draw[stealth-]([shift={(-150:\xxx/5.5)}]\xxx+\xxx/2,\yyy/2) arc (-150:150:\xxx/5.5);
\draw(\xxx+\xxx/2,\yyy/2)node{$i_2$};
%required voltages
\draw(\xxx+\dx,\yyy/2)node[right]{$\begin{aligned} &+ \\& V_1\\ &- \end{aligned}$};
\draw(1.5*\xxx,\yyy-\dy)node[below]{$+ \, V_2 \, -$};
\end{tikzpicture}
\caption{منبع رو سے دائری ترکیب نسبتاً آسان ہو جاتی ہے۔}
\label{شکل_جوڑ_منبع_رو_آسان_ترکیب_دائری_الف}
\end{figure}%%

حل:ایسا معلوم ہوتا ہے کہ دو عدد نا معلوم دائری رو \عددی{i_1} اور \عددی{i_2} پائے جاتے ہیں۔حقیقت میں \عددی{i_2} منبع رو سے گزرتی ہے لہٰذا اس کی قیمت کا تعین منبع رو ہی کرتی ہے یعنی
\begin{align*}
i_2=\SI{4}{\milli\ampere}
\end{align*}
ہے۔اس طرح بغیر حل کئے قانون اوہم کی مدد سے
\begin{align*}
V_2=2000 i_2=\SI{8}{\volt} 
\end{align*}
لکھا جا سکتا ہے۔بائیں خانے سے درج ذیل لکھا جاتا ہے
\begin{align*}
-6+8000i_1 +4000(i_1-i_2)=0
\end{align*}
جس میں \عددی{i_2=\SI{4}{\milli\ampere}} پُر کرتے ہوئے
\begin{align*}
i_1=\frac{11}{6} \, \si{\milli\ampere}
\end{align*}
اور
\begin{align*}
V_1=4000(i_1-i_2)=-\frac{26}{3} \,\si{\volt}
\end{align*}
حاصل ہوتے ہیں۔

آپ نے دیکھا کہ ایک عدد منبع رو کی وجہ سے نا معلوم رو کی تعداد دو عدد سے کم ہر کر ایک عدد رہ گئی۔
\انتہا{مثال}
%=================== 
\ابتدا{مثال}
شکل \حوالہ{شکل_جوڑ_منبع_رو_آسان_ترکیب_دائری_ب} میں \عددی{V_0} دریافت کریں۔

\begin{figure}
\centering
\begin{tikzpicture}
\draw(0,0) to [american current source,l={$\SI{2}{\milli\ampere}$}]++(0,\yy) to [resistor,l={$\SI{1}{\kilo\ohm}$}]++(\xx,0) to [resistor,l_={$\SI{2}{\kilo\ohm}$}]++(0,\yy) to [american current source,l_={$\SI{1}{\milli\ampere}$}]++(-\xx,0) to [short,-*]++(0,-\yy);
\draw(0,0) to [short]++(\xx,0) to [american voltage source,*-*,l_={$\SI{4}{\volt}$}]++(0,\yy);
\draw(\xx,2*\yy) to [short,*-]++(\xx,0) to [resistor,l={$\SI{4}{\kilo\ohm}$}]++(0,-\yy) to [american voltage source,l={$\SI{6}{\volt}$}]++(0,-\yy) to [short,-*]++(-\xx,0);
\draw(2*\xx,0) to [short,*-o]++(\xx,0);
\draw(2*\xx,2*\yy) to [short,*-o]++(\xx,0);
\draw(3*\xx,\yy)node{$\begin{aligned} &+ \\ \\  \\&V_0 \\ \\  \\ &- \end{aligned}$};
%loop currents
\draw[stealth-]([shift={(-150:\xx/5.5)}]\xx/2,\yy/2) arc (-150:150:\xx/5.5);
\draw(\xx/2,\yy/2)node{$i_1$};
\draw[stealth-]([shift={(-150:\xx/5.5)}]\xx/2,\yy+\yy/2) arc (-150:150:\xx/5.5);
\draw(\xx/2,\yy+\yy/2)node{$i_2$};
\draw[stealth-]([shift={(-150:\xx/5.5)}]\xx+\xx/2,\yy) arc (-150:150:\xx/5.5);
\draw(\xx+\xx/2,\yy)node{$i_3$};
\end{tikzpicture}
\caption{زیادہ منبع رو سے دائری ترکیب زیادہ آسان ہو سکتی ہے۔}
\label{شکل_جوڑ_منبع_رو_آسان_ترکیب_دائری_ب}
\end{figure}%%

حل:چونکہ \عددی{i_1} اور \عددی{i_2} منبع رو سے گزرتی ہیں لہٰذا ان کی قیمت لازمی طور پر انہیں منبع کی رو کے برابر ہوں گی۔یاد رہے کہ منبع رو سے کسی اور قیمت کی رو نہیں گزر سکتی۔یہی منبع رو کی تعریف ہے۔یوں
\begin{align*}
i_1&=\SI{2}{\milli\ampere}\\
i_2&=\SI{-1}{\milli\ampere}
\end{align*}
ہوں گے۔ یوں دور کو حل کرنے کی خاطر صرف ایک عدد مساوات دباو درکار ہے جسے \عددی{i_3} کی مدد سے لکھتے ہیں۔
\begin{align*}
-4 +2000(i_3-i_2)+4000i_3-6=0
\end{align*} 
اس میں \عددی{i_2=\SI{-1}{\milli\ampere}} پُر کرتے ہوئے حل کرنے سے درج ذیل حاصل ہوتا ہے۔
\begin{align*}
i_3=\frac{4}{3}\, \si{\milli\ampere}
\end{align*}
یوں شکل کو دیکھ کر درکار دباو لکھا جا سکتا ہے۔
\begin{align*}
V_0=4000i_3-6=-\frac{2}{3}\, \si{\volt}
\end{align*}
\انتہا{مثال}
%=====================
\ابتدا{مثال}\شناخت{مثال_جوڑ_زیادہ_منبع_رو_آسان_حل}
شکل \حوالہ{شکل_جوڑ_منبع_رو_آسان_ترکیب_دائری_پ} میں \عددی{I_0} حاصل کریں۔
\begin{figure}
\centering
\begin{subfigure}{1\textwidth}
\centering
\begin{tikzpicture}
\draw(0,0)node[left]{$a$} to [short]++(\xx,0) to [american current source,l_={$\SI{6}{\milli\ampere}$}]++(0,\yy)node[right]{$c$} to [resistor,l={$\SI{4}{\kilo\ohm}$}]++(-\xx,0)node[left]{$b$} to [american voltage source,l_={$\SI{10}{\volt}$}]++(0,-\yy);
\draw(0,\yy) to [short,*-]++(0,\yy) to [american current source,l={$\SI{4}{\milli\ampere}$}]++(\xx,0)node[above]{$d$} to [resistor,-*,i>_={$I_0$},l={$\SI{5}{\kilo\ohm}$}]++(0,-\yy);
\draw(\xx,0) to [short,*-]++(\xx,0)node[right]{$f$} to [american voltage source,l_={$\SI{8}{\volt}$}]++(0,\yy) to [resistor,l_={$\SI{10}{\kilo\ohm}$}]++(0,\yy)node[above]{$e$} to [short,-*]++(-\xx,0);
%loop currents
\draw[stealth-]([shift={(-150:\xx/5.5)}]\xx/2,\yy/2) arc (-150:150:\xx/5.5);
\draw(\xx/2,\yy/2)node{$i_1$};
\draw[stealth-]([shift={(-150:\xx/5.5)}]\xx/2,\yy+\yy/2) arc (-150:150:\xx/5.5);
\draw(\xx/2,\yy+\yy/2)node{$i_2$};
\draw[stealth-]([shift={(-150:\xx/5.5)}]\xx+\xx/2,\yy) arc (-150:150:\xx/5.5);
\draw(\xx+\xx/2,\yy)node{$i_3$};
\end{tikzpicture}
\caption{}
\end{subfigure}
\begin{subfigure}{1\textwidth}
\centering
\begin{tikzpicture}
\draw(0,0)node[left]{$a$} to [short]++(\xx,0) to [american current source,l_={$\begin{aligned} &+ \\ &v_x \\ &- \end{aligned}$}]++(0,\yy)node[right]{$c$} to [resistor,l={$\SI{4}{\kilo\ohm}$}]++(-\xx,0)node[left]{$b$} to [american voltage source,l_={$\SI{10}{\volt}$}]++(0,-\yy);
\draw(0,\yy) to [short,*-]++(0,\yy) to [american current source,l={$\SI{4}{\milli\ampere}$}]++(\xx,0)node[above]{$d$} to [resistor,-*,i>_={$I_0$},l={$\SI{5}{\kilo\ohm}$}]++(0,-\yy);
\draw(\xx,0) to [short,*-]++(\xx,0)node[right]{$f$} to [american voltage source,l_={$\SI{8}{\volt}$}]++(0,\yy) to [resistor,l_={$\SI{10}{\kilo\ohm}$}]++(0,\yy)node[above]{$e$} to [short,-*]++(-\xx,0);
\draw(\xx-\dx,1/4*\yy)node[left]{$\SI{6}{\milli\ampere}$};
%loop currents
\draw[stealth-]([shift={(-150:\xx/5.5)}]\xx/2,\yy/2) arc (-150:150:\xx/5.5);
\draw(\xx/2,\yy/2)node{$i_1$};
\draw[stealth-]([shift={(-150:\xx/5.5)}]\xx/2,\yy+\yy/2) arc (-150:150:\xx/5.5);
\draw(\xx/2,\yy+\yy/2)node{$i_2$};
\draw[stealth-]([shift={(-150:\xx/5.5)}]\xx+\xx/2,\yy) arc (-150:150:\xx/5.5);
\draw(\xx+\xx/2,\yy)node{$i_3$};
\end{tikzpicture}
\caption{}
\end{subfigure}%
\caption{مثال \حوالہ{مثال_جوڑ_زیادہ_منبع_رو_آسان_حل} کا دور۔}
\label{شکل_جوڑ_منبع_رو_آسان_ترکیب_دائری_پ}
\end{figure}%%

حل:یہاں \عددی{i_2} منبع رو سے گزرتی ہے لہٰذا
\begin{align*}
i_2=\SI{4}{\milli\ampere}
\end{align*}
ہو گی۔ہم اگر \عددی{i_2} کو استعمال کرتے ہوئے کرخوف قانون دباو لکھنا چاہیں تو \عددی{\SI{6}{\milli\ampere}} منبع سے گزرتے ہوئے دباو کی قیمت جاننے کا ہمارے پاس کوئی طریقہ موجود نہیں ہے۔یہ مسئلہ \عددی{i_3} کی صورت میں بھی درپیش ہے۔ہاں ہم دیکھتے ہیں کہ اس منبع رو سے \عددی{\SI{6}{\milli\ampere}} رو ہی گزر سکتی ہے لہٰذا
\begin{align}\label{مساوات_جوڑ_پہلی_ہمزاد_مساوات}
i_3-i_1=\SI{6}{\milli\ampere}
\end{align}
ہو گا۔چونکہ \عددی{i_2} ہم پہلے ہی حاصل کر چکے ہیں لہٰذا \عددی{i_1} اور \عددی{i_3} جاننے کے لئے دو عدد ہمزاد مساوات درکار ہیں۔مساوات \حوالہ{مساوات_جوڑ_پہلی_ہمزاد_مساوات} پہلی مساوات  ہے۔دوسری مساوات راہ \عددی{abcdefa} پر کرخوف قانون دباو سے لکھتے ہیں۔
\begin{align}\label{مساوات_جوڑ_مخلوط_خانہ_الف}
10+4000(i_1-\SI{4}{\milli\ampere})+5000(i_3-\SI{4}{\milli\ampere})+10000i_3+8=0
\end{align} 
مندرجہ بالا دو ہمزاد مساوات حل کرتے ہوئے درج ذیل حاصل ہوتا ہے۔
\begin{align*}
i_1&=-\frac{72}{19} \, \si{\milli\ampere}\\
i_3&=\frac{42}{19} \, \si{\milli\ampere}
\end{align*}
درکار رو حاصل کرتے ہیں۔
\begin{align*}
I_0=i_2-i_3=\frac{34}{19} \, \si{\milli\ampere}
\end{align*}

آئیں مساوات \حوالہ{مساوات_جوڑ_مخلوط_خانہ_الف} کو اس طرح حاصل کرنا سیکھیں کہ راہ \عددی{abcdefa} چننے کی ضرورت نہ ہو۔چونکہ \عددی{\SI{6}{\milli\ampere}} منبع رو کا دباو نا معلوم ہے لہٰذا اسے \عددی{v_x} متغیرہ سے ظاہر کیا جا سکتا ہے۔شکل \حوالہ{شکل_جوڑ_منبع_رو_آسان_ترکیب_دائری_پ}-ب میں ایسا دکھایا گیا ہے۔اسی شکل سے \عددی{i_1} اور \عددی{i_3} خانوں کے کرخوف قانون دباو سے درج ذیل حاصل ہوتے ہیں۔
\begin{align*}
10+4000(i_1-\SI{4}{\milli\ampere})+v_x&=0\\
-v_x+5000(i_3-\SI{4}{\milli\ampere})+10000i_3+8&=0
\end{align*}
ان مساوات کا مجموعہ لینے سے درج ذیل حاصل ہوتا ہے۔
\begin{align}\label{مساوات_جوڑ_مخلوط_خانہ_ب}
10+4000(i_1-\SI{4}{\milli\ampere})+5000(i_3-\SI{4}{\milli\ampere})+10000i_3+8&=0
\end{align}
مساوات \حوالہ{مساوات_جوڑ_مخلوط_خانہ_ب} کا مساوات \حوالہ{مساوات_جوڑ_مخلوط_خانہ_الف} کے ساتھ موازنہ کریں۔دونوں بالکل یکساں ہیں۔
\انتہا{مثال}

\FloatBarrier
%=====================
\ابتدا{مشق}\شناخت{مشق_جوڑ_دائری_ترکیب_الف}
شکل \حوالہ{شکل_جوڑ_مشق_دائری_ترکیب_الف} میں \عددی{V_0} کو دائری ترکیب سے حاصل کریں۔
\begin{figure}
\centering
\begin{tikzpicture}
\draw(0,0) to[american voltage source,l={$\SI{12}{\volt}$}]++(0,\y) to [resistor,l={$\SI{8}{\kilo\ohm}$}]++(\x,0) to [american current source,l={$\SI{2}{\milli\ampere}$}]++(\x,0) to [resistor,l={$\SI{6}{\kilo\ohm}$}]++(0,-\y) to [short]++(-2*\x,0);
\draw(\x,0) to [resistor,*-*,l={$\SI{4}{\kilo\ohm}$}]++(0,\y);
\draw(\x+\dx,\y/2)node[right]{$\begin{aligned} &+ \\ &V_0 \\ &- \end{aligned}$};
\end{tikzpicture}
\caption{مشق \حوالہ{مشق_جوڑ_دائری_ترکیب_الف} کا دور۔}
\label{شکل_جوڑ_مشق_دائری_ترکیب_الف}
\end{figure}%%

جواب:\عددی{V_0=-\tfrac{4}{3}\,\si{\volt}}
\انتہا{مشق}
%================

\ابتدا{مشق}\شناخت{مشق_جوڑ_دائری_ترکیب_ب}
شکل \حوالہ{شکل_جوڑ_مشق_دائری_ترکیب_ب} میں \عددی{V_0}، \عددی{V_1} اور \عددی{V_2} حاصل کریں۔
\begin{figure}
\centering
\begin{tikzpicture}
\draw(0,0) to[american current source,l={$\SI{5}{\milli\ampere}$}]++(0,\y) to [resistor,l={$\SI{2}{\kilo\ohm}$}]++(\x,0) to [resistor,l_={$\SI{8}{\kilo\ohm}$}]++(0,-\y) to [short]++(-\x,0);
\draw(\x,0) to [short,*-]++(\x,0) to [american voltage source,l_={$\SI{16}{\volt}$}]++(0,\y) to [resistor,-*,l_={$\SI{4}{\kilo\ohm}$}]++(-\x,0);
\draw(2*\x,\y) to [short,*-]++(0,\y) to [american current source,l={$\SI{8}{\milli\ampere}$}]++(-2*\x,0) to [short,-*]++(0,-\y);
\draw(\x+\dx,\y/2)node[right]{$\begin{aligned} &+ \\ &V_1 \\ &- \end{aligned}$};
\draw(\x/2,\y-\dy)node[below]{$+\, V_0 \, -$};
\draw(\x,2*\y+2*\dy)node[above]{$+\, V_2 \, -$};
\end{tikzpicture}
\caption{مشق \حوالہ{مشق_جوڑ_دائری_ترکیب_ب} کا دور۔}
\label{شکل_جوڑ_مشق_دائری_ترکیب_ب}
\end{figure}%%

جوابات:\عددی{\SI{26}{\volt}}، \عددی{\tfrac{136}{3} \, \si{\volt}}، \عددی{\tfrac{166}{3} \, \si{\volt}}
\انتہا{مشق}
%================

\حصہ{تابع منبع استعمال کرنے والے ادوار}
کرخوف کے مساوات کے نقطہ نظر سے تابع منبع اور آزاد منبع میں کوئی فرق نہیں پایا جاتا۔البتہ تابع منبع استعمال کرنے والے ادوار کو حل کرتے ہوئے تابع منبع کی \اصطلاح{قابو مساوات}\فرہنگ{قابو مساوات}\حاشیہب{control equation}\فرہنگ{control equation} بھی درکار ہوتی ہے۔آئیں چند مثالوں کی مدد سے ایسے ادوار حل کرنا دیکھیں۔آسان ترین مثال سے شروع کرتے ہوئے بتدریج مشکل مثال حل کرتے ہیں۔

%=============
\ابتدا{مثال}\شناخت{مثال_جوڑ_تابع_منبع_دائری_ترکیب_الف}
شکل \حوالہ{شکل_جوڑ_تابع_منبع_دائری_ترکیب_مثال_الف} میں \عددی{V_a} دریافت کریں۔
\begin{figure}
\centering
\begin{tikzpicture}
\draw(0,0) to [american voltage source,l={$\SI{6}{\volt}$}]++(0,\yy) to [resistor,l={$\SI{4}{\kilo\ohm}$}]++(\xx,0) to [resistor,l={$\SI{2}{\kilo\ohm}$}]++(\xx,0) to [american controlled voltage source,l={$2V_a$}]++(0,-\yy) to [short]++(-2*\xx,0);
\draw(\xx,\yy) to [resistor,*-*,l={$\SI{6}{\kilo\ohm}$}]++(0,-\yy);
\draw(\xx-\dx,\yy/2)node[left]{$\begin{aligned} &- \\ &V_a \\ &+ \end{aligned}$};
%currents
\draw[gray,stealth-]([shift={(-150:\xx/5.5)}]\xx/2,\yy/2) arc (-150:150:\xx/5.5);
\draw[gray](\xx/2,\yy/2)node{$i_1$};
\draw[gray,-stealth] (1/3*\xx,3/4*\yy) --++(4/3*\xx,0)--++(0,-1/2*\yy)node[above left]{$i_2$}--++(-4/3*\xx,0)--++(0,\dy);
\end{tikzpicture}
\caption{مثال \حوالہ{مثال_جوڑ_تابع_منبع_دائری_ترکیب_الف} کا دور۔}
\label{شکل_جوڑ_تابع_منبع_دائری_ترکیب_مثال_الف}
\end{figure}%%

حل: کرخوف مساوات لکھتے ہیں۔
\begin{align*}
-6+4000(i_1+i_2)-V_a&=0\\
-6+4000(i_1+i_2)+2000i_2-2V_a&=0
\end{align*}
ان میں تابع منبع دباو کی قابو مساوات
\begin{align*}
V_a=-6000i_1
\end{align*} 
پُر کرتے  ہوئے ترتیب دیتے ہیں۔
\begin{align*}
10000i_1+4000i_2&=6\\
16000i_1+6000i_2&=6
\end{align*}
ان ہمزاد مساوات کا حل درج ذیل ہے۔
\begin{align*}
i_1&=\SI{-3}{\milli\ampere}\\
i_2&=\SI{9}{\milli\ampere}
\end{align*}
یوں درکار دباو درج ذیل حاصل ہوتا ہے۔
\begin{align*}
V_a=-6000(-0.003)=\SI{18}{\volt}
\end{align*}
\انتہا{مثال}
%===================
\ابتدا{مثال}\شناخت{مثال_جوڑ_تابع_منبع_دائری_ترکیب_ب}
شکل \حوالہ{شکل_جوڑ_تابع_منبع_دائری_ترکیب_مثال_ب} میں \عددی{V_0} دریافت کریں۔

\begin{figure}
\centering
\begin{tikzpicture}
\draw(0,0) to [american current source,l={$\SI{4}{\milli\ampere}$}]++(0,\yy) to [american controlled current source,l={$\tfrac{V_x}{2000}$}]++(0,\yy) to [short]++(\xx,0) to [resistor,l={$\SI{8}{\kilo\ohm}$}]++(0,-\yy) to [resistor,-*,l={$\SI{2}{\kilo\ohm}$}]++(-\xx,0);
\draw(0,0) to [short]++(\xx,0) to [american voltage source,-*,l_={$\SI{2}{\volt}$}]++(0,\yy);
\draw(\xx,0) to [short,*-] ++(\xx,0) to [american voltage source,l_={$\SI{4}{\volt}$}]++(0,\yy) to [resistor,l_={$\SI{4}{\kilo\ohm}$}]++(0,\yy) to [short,-*]++(-\xx,0);
\draw(2*\xx,0) to [short,*-o]++(\xx,0);
\draw(2*\xx,2*\yy) to [short,*-o]++(\xx,0);
\draw(3*\xx,\yy)node{$\begin{aligned} &+ \\ \\ \\ &V_0 \\ \\ \\ &-  \end{aligned}$};
\draw(2*\xx-\dx,\yy+\yy/2)node[left]{$\begin{aligned} &+ \\ &V_x \\ &-  \end{aligned}$};
%currents
\draw[stealth-]([shift={(-150:\xx/5.5)}]\xx/2,\yy/2) arc (-150:150:\xx/5.5);
\draw[](\xx/2,\yy/2)node{$i_1$};
\draw[stealth-]([shift={(-150:\xx/5.5)}]\xx/2,\yy+\yy/2) arc (-150:150:\xx/5.5);
\draw[](\xx/2,\yy+\yy/2)node{$i_2$};
\draw[stealth-]([shift={(-150:\xx/5.5)}]\xx+\xx/2,\yy) arc (-150:150:\xx/5.5);
\draw[](\xx+\xx/2,\yy)node{$i_3$};
\end{tikzpicture}
\caption{مثال \حوالہ{مثال_جوڑ_تابع_منبع_دائری_ترکیب_ب} کا دور۔}
\label{شکل_جوڑ_تابع_منبع_دائری_ترکیب_مثال_ب}
\end{figure}%%

حل:چونکہ \عددی{i_1} اور \عددی{i_2} منبع رو سے گزرتی ہیں لہٰذا ان کی قیمت منبع رو کے برابر ہی ہو گی۔
\begin{align*}
i_1&=\SI{4}{\milli\ampere}\\
i_2&=\frac{V_x}{2000}
\end{align*}
دائیں خانے کی مساوات لکھتے ہیں۔
\begin{align*}
-2+8000\left(i_3-\frac{V_x}{2000}\right)+4000i_3+4=0
\end{align*}
اس میں تابع منبع کی قابو مساوات
\begin{align*}
V_x=4000i_3
\end{align*}
پُر کرتے
\begin{align*}
-2+8000\left(i_3-\frac{4000i_3}{2000}\right)+4000i_3+4=0
\end{align*}
ہوئے حل کرنے سے
\begin{align*}
i_3=\SI{0.5}{\milli\ampere}
\end{align*}
حاصل ہوتا ہے۔یوں درکار دباو درج ذیل حاصل ہوتا ہے۔
\begin{align*}
V_0=4000i_3+4=\SI{6}{\volt}
\end{align*}
\انتہا{مثال}
%==================
\ابتدا{مثال}\شناخت{مثال_جوڑ_تابع_منبع_دائری_ترکیب_پ}
شکل \حوالہ{شکل_جوڑ_تابع_منبع_دائری_ترکیب_مثال_پ} میں تمام خانوں کی رو دریافت کریں۔اس شکل میں رو  قابو منبع دباو اور دباو قابو منبع رو استعمال کئے گئے ہیں۔
\begin{figure}
\centering
\begin{tikzpicture}
\draw(0,0) to [american controlled voltage source,l={$2\si{\kilo}I_x$}]++(0,\yy) to [american current source,l={$\SI{2}{\milli\ampere}$}]++(0,\yy) to [short]++(2*\xxx,0) to [american controlled current source,l={$\tfrac{V_a}{1\si{\kilo}}$}]++(0,-\yy) to [resistor,i<={$I_x$},l={$\SI{6}{\kilo\ohm}$}]++(-\xxx,0) to [resistor,-*,l={$\SI{4}{\kilo\ohm}$}]++(-\xxx,0);
\draw(0,0) to [short]++(2*\xxx,0) to [american voltage source,-*,l_={$\SI{10}{\volt}$}]++(0,\yy);
\draw(\xxx,0) to [resistor,*-*]++(0,\yy) to [resistor,-*]++(0,\yy);
\draw(\xxx-\dx,\yy+3/4*\yy)node[left]{$\SI{2}{\kilo\ohm}$};
\draw(\xxx-\dx,1/4*\yy)node[left]{$\SI{4}{\kilo\ohm}$};
\draw(\xxx/2,\yy+\dy)node[above]{$+ \, V_a \, -$};
%currents
\draw[stealth-]([shift={(-150:\xx/5.5)}]\xxx/2,\yy/2) arc (-150:150:\xx/5.5);
\draw[](\xxx/2,\yy/2)node{$i_1$};
\draw[stealth-]([shift={(-150:\xx/5.5)}]\xxx/2,\yy+\yy/2) arc (-150:150:\xx/5.5);
\draw[](\xxx/2,\yy+\yy/2)node{$i_2$};
\draw[stealth-]([shift={(-150:\xx/5.5)}]\xxx+\xxx/2,\yy+\yy/2) arc (-150:150:\xx/5.5);
\draw[](\xxx+\xxx/2,\yy+\yy/2)node{$i_3$};
\draw[stealth-]([shift={(-150:\xx/5.5)}]\xxx+\xxx/2,\yy/2) arc (-150:150:\xx/5.5);
\draw[](\xxx+\xxx/2,\yy/2)node{$i_4$};
\end{tikzpicture}
\caption{مثال \حوالہ{مثال_جوڑ_تابع_منبع_دائری_ترکیب_پ} کا دور۔}
\label{شکل_جوڑ_تابع_منبع_دائری_ترکیب_مثال_پ}
\end{figure}%%

حل:چار خانوں کے مساوات درج ذیل ہیں
\begin{align*}
-2\si{\kilo} I_x+4\si{\kilo}(i_1-i_2)+4\si{\kilo}(i_1-i_4)&=0\\
i_2&=\frac{2}{1\si{\kilo}}\\
i_3&=\frac{V_a}{1\si{\kilo}}\\
4\si{\kilo}(i_4-i_1)+6\si{\kilo}(i_4-i_3)+10&=0
\end{align*}
جن میں
\begin{align*}
V_a&=4\si{\kilo}(i_1-i_2)\\
I_x&=i_4-i_3
\end{align*}
پُر کرتے اور ترتیب  دیتے ہوئے
\begin{align*}
4i_1-2i_2+i_3-3i_4&=0\\
i_2&=0.002\\
-4i_1+4i_2+i_3&=0\\
-2i_1-3i_3+5i_4&=-0.005
\end{align*}
حاصل ہوتے ہیں۔انہیں قالبی مساوات کی صورت میں لکھتے ہیں۔
\begin{align*}
\begin{bmatrix*}
4 & -2 & 1 & -3\\
0& 1 & 0 & 0\\
-4 & 4 & 1 &0\\
-2&0&-3&5
\end{bmatrix*}
\begin{bmatrix}
i_1\\
i_2\\
i_3\\
i_4
\end{bmatrix}
=
\begin{bmatrix}
0\\
0.002\\
0\\
-0.005 
\end{bmatrix}
\end{align*}
یہ قالبی مساوات \عددی{{\bf{RI=V}}} کی طرز کی ہے جس کا حل \عددی{{\bf{I=R^{-1}V}}} ہے۔یوں خانوں کی رو درج ذیل حاصل ہوتی ہیں۔
\begin{align*}
i_1&=\SI{13.5}{\milli\ampere}\\
i_1&=\SI{2}{\milli\ampere}\\
i_1&=\SI{46}{\milli\ampere}\\
i_1&=\SI{32}{\milli\ampere}
\end{align*}
\انتہا{مثال}
%================

\حصہ{دائری ترکیب اور ترکیب جوڑ کا موازنہ}
عموماً ترکیب جوڑ اور دائری ترکیب سے حاصل مساواتوں کی تعداد برابر نہیں ہوتی۔ کم تعداد کے ہمزاد مساوات حل کرنا نسبتاً آسان ہوتا ہے۔کسی بھی دور کو حل کرنے سے پہلے دیکھیں کہ کس ترکیب سے کم تعداد کے مساوات حاصل ہوتے ہیں۔

\begin{figure}
\centering
\begin{tikzpicture}
\draw(0,0) to [american voltage source,l={$v_a$}]++(0,2*\y) to [short]++(2*\x,0);
\draw(0,0) to  [short] ++(3*\x,0) to [american voltage source,l_={$v_b$}] ++(0,\y) to [resistor,l_={$R_6$}]++(-\x,0);
\draw(\x,0) to [resistor,*-,l={$R_3$}]++(0,\y) to [resistor,-*,l={$R_1$}]++(0,\y);
\draw(2*\x,0) to [resistor,*-,l_={$R_5$}]++(0,\y) to [resistor,l_={$R_2$}]++(0,\y);
\draw(\x,\y)node[left]{$v_1$} to [resistor,*-*,l={$R_4$}]++(\x,0)node[above right]{$v_2$};
\draw(\x,0)node[ground]{};
%currents
\draw[stealth-]([shift={(-150:\x/5.5)}]\x/2,\y) arc (-150:150:\x/5.5);
\draw[](\x/2,\y)node{$i_1$};
\draw[stealth-]([shift={(-150:\x/5.5)}]\x+\x/2,\y+\y/2+\dy) arc (-150:150:\x/5.5);
\draw[](\x+\x/2,\y+\y/2+\dy)node{$i_2$};
\draw[stealth-]([shift={(-150:\x/5.5)}]\x+\x/2,\y/2) arc (-150:150:\x/5.5);
\draw[](\x+\x/2,\y/2)node{$i_3$};
\draw[stealth-]([shift={(-150:\x/5.5)}]2*\x+\x/2+\dx/2,\y/2) arc (-150:150:\x/5.5);
\draw[](2*\x+\x/2+\dx/2,\y/2)node{$i_4$};
\end{tikzpicture}
\caption{اس دور میں ترکیب جوڑ کے مساواتوں کی تعداد کم ہے۔}
\label{شکل_جوڑ_کم_ترکیب_جوڑ_مساوات}
\end{figure}%%


شکل \حوالہ{شکل_جوڑ_کم_ترکیب_جوڑ_مساوات} میں چار خانے پائے جاتے ہیں لہٰذا ان خانوں کی رو حاصل کرنے کی خاطر چار عدد ہمزاد مساوات درکار ہوں گے۔ان مساوات کو یہاں پیش کرتے ہیں۔
\begin{align*}
-v_a+(i_1-i_2)R_1+(i_1-i_3)R_3&=0\\
(i_2-i_1)R_1+i_2R_2+(i_2-i_3)R_4&=0\\
(i_3-i_1)R_3+(i_3-i_2)R_4+(i_3-i_4)R_5&=0\\
(i_4-i_3)R_5+i_4 R_6+v_b&=0
\end{align*}
اس کے برعکس اس دور میں نچلا جوڑ برقی زمین  اور بالائی جوڑ \عددی{v_a} دباو پر ہے لہٰذا اس میں دو عدد نا معلوم جوڑ \عددی{v_1} اور \عددی{v_2} پائے جاتے  ہیں جن کے مساوات درج ذیل ہیں۔
\begin{align*}
\frac{v_1-v_a}{R_1}+\frac{v_1}{R_3}+\frac{v_1-v_2}{R_4}&=0\\
\frac{v_2-v_a}{R_2}+\frac{v_2}{R_5}+\frac{v_2-v_b}{R_6}&=0
\end{align*}
صاف ظاہر ہے کہ شکل \حوالہ{شکل_جوڑ_کم_ترکیب_جوڑ_مساوات} کو ترکیب جوڑ سے حل کرنا زیادہ آسان ہے۔ 

\begin{figure}
\centering
\begin{tikzpicture}
\draw(0,0)node[left]{$v_1$} to [american current source,l={$i_a$}]++(0,\y)node[left]{$v_2$} to [american  current source,l={$i_b$}]++(0,\y) to [short]++(\x,0)node[above]{$v_3$} to [resistor,l_={$R_2$}]++(0,-\y)node[above left]{$v_4$} to [resistor,l={$R_5$}]++(0,-\y) to [resistor,l={$R_4$}]++(-\x,0);
\draw(\x,0) to [resistor,*-,l_={$R_6$}]++(\x,0)node[right]{$v_6$} to [american current source,l_={$i_c$}]++(0,\y)node[right]{$v_5$} to [resistor,l_={$R_4$}]++(0,\y) to [short,-*]++(-\x,0);
\draw(0,\y) to [resistor,*-*,l={$R_1$}]++(\x,0) to [resistor,-*,l_={$R_3$}]++(\x,0);
%currents
\draw[stealth-]([shift={(-150:\x/5.5)}]\x+\x/2,\y+\y/2) arc (-150:150:\x/5.5);
\draw[](\x+\x/2,\y+\y/2)node{$i_1$};
\draw(\x,0)node[ground]{};
\end{tikzpicture}
\caption{اس دور میں دائری ترکیب کے مساواتوں کی تعداد کم ہے۔}
\label{شکل_جوڑ_کم_دائری_ترکیب_مساوات}
\end{figure}%%

آئیں اب شکل \حوالہ{شکل_جوڑ_کم_دائری_ترکیب_مساوات} کو دیکھیں۔یہاں تین خانوں کی رو، ان خانوں میں موجود منبع رو تعین کرتے ہیں لہٰذا ہمیں صرف ایک عدد خانے کی رو \عددی{i_1} درکار ہے۔دائری ترکیب کی مساوات درج ذیل ہے۔
\begin{align*}
(i_1-i_b)R_2+i_1R_4+(i_1+i_c)R_3&=0
\end{align*} 
اس کے برعکس درج ذیل چھ جوڑ کے مساوات لکھے جائیں گے۔
\begin{align*}
\frac{v_1}{R_4}+i_a&=0\\
-i_a+i_b+\frac{v_2-v_4}{R_1}&=0\\
-i_b+\frac{v_3-v_4}{R_2}+\frac{v_3-v_5}{R_4}&=0\\
\frac{v_4-v_2}{R_1}+\frac{v_4-v_3}{R_2}+\frac{v_4-v_5}{R_3}+\frac{v_4}{R_5}&=0\\
\frac{v_5-v_4}{R_3}+\frac{v_5-v_3}{R_4}-i_c&=0\\
\frac{v_6}{R_6}+i_c&=0
\end{align*}
آپ دیکھ سکتے ہیں کہ کرخوف قانون دباو سے اس دور کو حل کرنے زیادہ آسان ثابت ہوتا ہے۔

%============================
%=============================
\حصہء{سوالات}
%==================
\ابتدا{سوال}\شناخت{سوال_دائری_رو_الف}
شکل \حوالہ{شکل_سوال_دائری_رو_الف} میں \عددی{I_0} دریافت کریں۔
\begin{figure}
\centering
\begin{tikzpicture}
\draw(0,0) to [american current source,l_={$\SI{4}{\milli\ampere}$}]++(0,-\yyy) to [short]++(\xxx,0) to [short,i<={$I_0$}]++(\xxx,0) to [short]++(\xxx,0) to [american current source,l_={$\SI{6}{\milli\ampere}$}]++(0,\yyy) to [short]++(-\xxx,0) to [resistor,*-*,l={$\SI{2}{\kilo\ohm}$}]++(0,-\yyy);
\draw(0,0) to [short]++(\xxx,0) to [american current source,l={$\SI{2}{\milli\ampere}$}]++(\xxx,0);
\draw(\xxx,-\yyy) to [resistor,*-*,l={$\SI{2}{\kilo\ohm}$}]++(0,\yyy);
\end{tikzpicture}
\caption{سوال \حوالہ{سوال_دائری_رو_الف} کا دور۔}
\label{شکل_سوال_دائری_رو_الف}
\end{figure}%%

جواب:\عددی{I_0=\SI{2}{\milli\ampere}}
\انتہا{سوال}
%==================
\ابتدا{سوال}\شناخت{سوال_دائری_رو_ب}
شکل \حوالہ{شکل_سوال_دائری_رو_ب} میں \عددی{I_0} دریافت کریں۔
\begin{figure}
\centering
\begin{tikzpicture}
\draw(0,0) to [resistor,l_={$\SI{4}{\kilo\ohm}$},i={$I_0$}]++(0,-\yyy) to [short]++(\xxx,0) to [short]++(\xxx,0) to [short]++(\xxx,0) to [resistor,l_={$\SI{8}{\kilo\ohm}$}]++(0,\yyy) to [short]++(-\xxx,0) to [resistor,*-*,l={$\SI{2}{\kilo\ohm}$}]++(0,-\yyy);
\draw(0,0) to [short]++(\xxx,0) to [american current source,l={$\SI{5}{\milli\ampere}$}]++(\xxx,0);
\draw(\xxx,-\yyy) to [resistor,*-*,l={$\SI{2}{\kilo\ohm}$}]++(0,\yyy);
\end{tikzpicture}
\caption{سوال \حوالہ{سوال_دائری_رو_ب} کا دور۔}
\label{شکل_سوال_دائری_رو_ب}
\end{figure}%%

جواب:\عددی{I_0=-\tfrac{5}{3}\,\si{\milli\ampere}}
\انتہا{سوال}
%==================
\ابتدا{سوال}\شناخت{سوال_دائری_رو_پ}
شکل \حوالہ{شکل_سوال_دائری_رو_پ} میں \عددی{I_0} دریافت کریں۔
\begin{figure}
\centering
\begin{tikzpicture}
\draw(0,0) to [american current source,l={$\SI{4}{\milli\ampere}$}]++(0,\yyy) to [resistor,l={$\SI{4}{\kilo\ohm}$}]++(\xxx,0) to [resistor,l={$\SI{6}{\kilo\ohm}$}]++(\xxx,0) to [american current source,l={$\SI{6}{\milli\ampere}$}]++(0,-\yyy) to [short](0,0);
\draw(\xxx,\yyy) to [resistor,*-*,l={$\SI{2}{\kilo\ohm}$},i={$I_0$}]++(0,-\yyy);
\draw(0,\yyy) to [short,*-]++(0,3/4*\yyy) to [american current source,l={$\SI{2}{\milli\ampere}$}]++(2*\xxx,0) to [short,-*]++(0,-3/4*\yyy);
\end{tikzpicture}
\caption{سوال \حوالہ{سوال_دائری_رو_پ} کا دور۔}
\label{شکل_سوال_دائری_رو_پ}
\end{figure}%%

جواب:\عددی{I_0=\SI{-2}{\milli\ampere}}
\انتہا{سوال}
%==================
\ابتدا{سوال}\شناخت{سوال_دائری_رو_ت}
شکل \حوالہ{شکل_سوال_دائری_رو_ت} میں \عددی{I_0} دریافت کریں۔
\begin{figure}
\centering
\begin{tikzpicture}[american voltages]
\draw(0,0) to [resistor,l_={$\SI{4}{\kilo\ohm}$}]++(0,-\yyy) to [short]++(\xxx,0) to [short]++(\xxx,0) to [short]++(\xxx,0) to [resistor,l_={$\SI{8}{\kilo\ohm}$}]++(0,\yyy) to [short]++(-\xxx,0) to [resistor,*-*,l={$\SI{2}{\kilo\ohm}$}]++(0,-\yyy);
\draw(0,0) to [short]++(\xxx,0) to [american current source,l={$\SI{5}{\milli\ampere}$}]++(\xxx,0);
\draw(\xxx,-\yyy) to [resistor,*-*,l_={$\SI{2}{\kilo\ohm}$},v^>={$V_1$}]++(0,\yyy);
\draw(\xxx,0) to [short,*-]++(0,3/4*\yyy) to [resistor,l={$\SI{5}{\kilo\ohm}$}]++(\xxx,0) to [short,-*]++(0,-3/4*\yyy);
\end{tikzpicture}
\caption{سوال \حوالہ{سوال_دائری_رو_ت} کا دور۔}
\label{شکل_سوال_دائری_رو_ت}
\end{figure}%%

جواب:\عددی{V_0=-\tfrac{500}{119}\,\si{\volt}}
\انتہا{سوال}
%==================
\ابتدا{سوال}\شناخت{سوال_دائری_رو_ٹ}
شکل \حوالہ{شکل_سوال_دائری_رو_ٹ} میں \عددی{V_0} دریافت کریں۔
\begin{figure}
\centering
\begin{tikzpicture}[american voltages]
\draw(0,0) to [american voltage source,l={$\SI{6}{\volt}$}]++(0,\yyy) to [resistor,l={$\SI{2}{\kilo\ohm}$}]++(\xxx,0) to [resistor,l={$\SI{6}{\kilo\ohm}$}]++(\xxx,0) to [american voltage source,l={$\SI{12}{\volt}$}]++(0,-\yyy) to [short](0,0);
\draw(\xxx,\yyy) to [resistor,*-*,l={$\SI{8}{\kilo\ohm}$},v={$V_0$}]++(0,-\yyy);
\end{tikzpicture}
\caption{سوال \حوالہ{سوال_دائری_رو_ٹ} کا دور۔}
\label{شکل_سوال_دائری_رو_ٹ}
\end{figure}%%


جواب:\عددی{V_0=\tfrac{24}{19}\,\si{\volt}}
\انتہا{سوال}
%=================
\ابتدا{سوال}\شناخت{سوال_دائری_رو_ث}
شکل \حوالہ{شکل_سوال_دائری_رو_ث} میں \عددی{V_0} دریافت کریں۔

\begin{figure}
\centering
\begin{tikzpicture}[american voltages]
\draw(0,0) to [resistor,l={$\SI{1}{\kilo\ohm}$}]++(0,\yyy) to [resistor,l={$\SI{4}{\kilo\ohm}$}]++(\xxx,0) to [resistor,l={$\SI{6}{\kilo\ohm}$}]++(\xxx,0) to [resistor,l={$\SI{2}{\kilo\ohm}$}]++(0,-\yyy) to [short](0,0);
\draw(\xxx,0) to [american voltage source,*-*,l={$\SI{10}{\volt}$}]++(0,\yyy);
\draw(0,\yyy) to [short,*-]++(0,3/4*\yyy) to [resistor,l={$\SI{4}{\kilo\ohm}$},v={$V_0$}]++(2*\xxx,0) to [short,-*]++(0,-3/4*\yyy);
\end{tikzpicture}
\caption{سوال \حوالہ{سوال_دائری_رو_ث} کا دور۔}
\label{شکل_سوال_دائری_رو_ث}
\end{figure}%%

جواب:\عددی{V_0=-\tfrac{20}{63}\,\si{\volt}}
\انتہا{سوال}
%==================
\ابتدا{سوال}\شناخت{سوال_دائری_رو_ج}
شکل \حوالہ{شکل_سوال_دائری_رو_ج} میں \عددی{V_0} دریافت کریں۔
\begin{figure}
\centering
\begin{tikzpicture}[american voltages]
\draw(0,0) to [american voltage source,l={$\SI{10}{\volt}$}]++(0,\yyy) to [resistor,l={$\SI{1}{\kilo\ohm}$}]++(\xxx,0) to [resistor,l={$\SI{3}{\kilo\ohm}$}]++(\xxx,0) to [resistor,l={$\SI{2}{\kilo\ohm}$},v={$V_0$}]++(0,-\yyy) to [short](0,0);
\draw(\xxx,0) to [american current source,*-*,l={$\SI{5}{\milli\ampere}$}]++(0,\yyy);
\end{tikzpicture}
\caption{سوال \حوالہ{سوال_دائری_رو_ج} کا دور۔}
\label{شکل_سوال_دائری_رو_ج}
\end{figure}%%

جواب:\عددی{V_0=\SI{5}{\volt}}
\انتہا{سوال}
%====================
\ابتدا{سوال}\شناخت{سوال_دائری_رو_چ}
شکل \حوالہ{شکل_سوال_دائری_رو_چ} میں \عددی{V_0} دریافت کریں۔

\begin{figure}
\centering
\begin{tikzpicture}[american voltages]
\draw(0,0) to [american current source,l={$\SI{6}{\milli\ampere}$}]++(0,\yyy) to [resistor,l={$\SI{2}{\kilo\ohm}$},v={$V_0$}]++(\xxx,0) to [resistor,l={$\SI{4}{\kilo\ohm}$}]++(\xxx,0);
\draw (0,0) to [short] ++(2*\xxx,0) to [american current source,l={$\SI{2}{\milli\ampere}$}]++(0,\yyy);
\draw(\xxx,0) to [american voltage source,*-*,l={$\SI{3}{\volt}$}]++(0,\yyy);
\draw(0,\yyy) to [short,*-]++(0,3/4*\yyy) to [resistor,l={$\SI{6}{\kilo\ohm}$}]++(2*\xxx,0) to [short,-*]++(0,-3/4*\yyy);
\end{tikzpicture}
\caption{سوال \حوالہ{سوال_دائری_رو_چ} کا دور۔}
\label{شکل_سوال_دائری_رو_چ}
\end{figure}%%


جواب:\عددی{V_0=\tfrac{34}{3}\,\si{\volt}}
\انتہا{سوال}
%=========================
\ابتدا{سوال}\شناخت{سوال_دائری_جوڑ_حل_الف}
شکل \حوالہ{شکل_سوال_دائری_جوڑ_حل_الف}  میں \عددی{I_0} کو ترکیب جوڑ سے حاصل کریں۔
\begin{figure}
\centering
\begin{tikzpicture}
\draw(0,0) to [short]++(2*\xxx,0) to [american voltage source,l={$\SI{8}{\volt}$}]++(0,2*\yyy) to [short,i={$I_0$}]++(-\xxx,0) to [short]++(-\xxx,0) to [resistor,l={$\SI{6}{\kilo\ohm}$}]++(0,-\yyy) to [resistor,l={$\SI{8}{\kilo\ohm}$}]++(0,-\yyy);
\draw(\xxx,0) to [resistor,*-*,l_={$\SI{4}{\kilo\ohm}$}]++(0,\yyy) to [resistor,-*,l_={$\SI{2}{\kilo\ohm}$}]++(0,\yyy);
\draw(\xxx,\yyy) to [american voltage source,*-*,l={$\SI{4}{\volt}$}]++(-\xxx,0);
\end{tikzpicture}
\caption{سوال \حوالہ{سوال_دائری_جوڑ_حل_الف} کا دور۔}
\label{شکل_سوال_دائری_جوڑ_حل_الف}
\end{figure}%%


جواب:\عددی{I_0=\SI{2}{\milli\ampere}}
\انتہا{سوال}
%===========================
\ابتدا{سوال}\شناخت{سوال_دائری_جوڑ_حل_ب}
شکل \حوالہ{شکل_سوال_دائری_جوڑ_حل_ب} میں ترکیب جوڑ سے \عددی{V_1}، \عددی{V_2}، \عددی{V_3} اور \عددی{V_4} دریافت کریں۔

\begin{figure}
\centering
\begin{tikzpicture}
\draw(0,0) to [short]++(\xxx,0)node[ground]{} to [short]++(\xxx,0) to [american voltage source,l_={$\SI{5}{\volt}$}]++(0,\yyy)node[right]{$V_4$} to [american current source,l_={$\SI{2}{\milli\ampere}$}] ++(0,\yyy) to [short]++(-\xxx,0)node[above]{$V_2$} to [short]++(-\xxx,0) to [resistor,l_={$\SI{0.4}{\siemens}$}]++(0,-\yyy)node[left]{$V_1$} to [resistor,l_={$\SI{0.2}{\siemens}$}]++(0,-\yyy);
\draw(\xxx,0) to [american current source,*-*,l_={$\SI{3}{\milli\ampere}$}]++(0,\yyy) to [american voltage source,-*,l_={$\SI{6}{\volt}$}]++(0,\yyy);
\draw(0,\yyy) to [resistor,*-*,l={$\SI{2}{\siemens}$}]++(\xxx,0) node [above left]{$V_3$}to [resistor,-*,l={$\SI{1}{\siemens}$}]++(\xxx,0);
\end{tikzpicture}
\caption{سوال \حوالہ{سوال_دائری_جوڑ_حل_ب} کا دور۔}
\label{شکل_سوال_دائری_جوڑ_حل_ب}
\end{figure}%

جوابات:\عددی{V_1=\SI{4.68}{\volt}}، \عددی{V_2=\SI{10.07}{\volt}}، \عددی{V_3=\SI{4.07}{\volt}}، \عددی{V_4=\SI{5}{\volt}}
\انتہا{سوال}
%==========================
\ابتدا{سوال}\شناخت{سوال_دائری_جوڑ_حل_پ}
شکل \حوالہ{شکل_سوال_دائری_جوڑ_حل_پ} میں ترکیب جوڑ سے \عددی{I_0} حاصل کریں۔
\begin{figure}
\centering
\begin{tikzpicture}[american voltages]
\draw(0,0) to [american voltage source,l={$\SI{6}{\volt}$}]++(0,\yyy) to [resistor,l={$\SI{2}{\kilo\ohm}$},v_<={$V_x$}]++(\xxx,0) to [resistor,l={$\SI{8}{\kilo\ohm}$},i={$I_0$}]++(0,-\yyy) to [short](0,0);
\draw(\xxx,0) to [short,*-]++(\xxx,0) to [american controlled voltage source,l_={$2V_x$}]++(0,\yyy) to [short,-*]++(-\xxx,0);
\end{tikzpicture}
\caption{سوال \حوالہ{سوال_دائری_جوڑ_حل_پ} کا دور۔}
\label{شکل_سوال_دائری_جوڑ_حل_پ}
\end{figure}%

جواب:\عددی{I_0=\tfrac{6}{11}\,\si{\milli\ampere}}
\انتہا{سوال}
%========================
\ابتدا{سوال}\شناخت{سوال_دائری_جوڑ_حل_ت}
شکل \حوالہ{شکل_سوال_دائری_جوڑ_حل_ت} میں ترکیب جوڑ سے \عددی{V_0} حاصل کریں۔
\begin{figure}
\centering
\begin{tikzpicture}[american voltages]
\draw(0,0) to [american voltage source,*-,l={$\SI{6}{\volt}$}]++(0,\yyy) to [american voltage source,-*,l={$\SI{4}{\volt}$}]++(0,\yyy);
\draw(-\xxx,0) to [resistor,l={$\SI{4}{\kilo\ohm}$}]++(0,\yyy) to [resistor,l={$\SI{2}{\kilo\ohm}$}]++(0,\yyy);
\draw(\xxx,0) to [resistor,*-,l={$\SI{8}{\kilo\ohm}$}]++(0,\yyy) to [american voltage source,-*,l={$\SI{2}{\volt}$}]++(0,\yyy);
\draw(2*\xxx,0) to [resistor,l={$\SI{4}{\kilo\ohm}$},v_>={$V_0$}]++(0,\yyy) to [resistor,l_={$\SI{6}{\kilo\ohm}$}]++(0,\yyy);
\draw(-\xxx,0) to [short]++(3*\xxx,0);
\draw(-\xxx,2*\yyy) to [short]++(3*\xxx,0);
\draw(0,\yyy) to [resistor,*-*,l={$\SI{4}{\kilo\ohm}$}]++(-\xxx,0);
\draw(2*\xxx,\yyy) to [american voltage source,*-*,l={$\SI{4}{\volt}$}]++(-\xxx,0);
\end{tikzpicture}
\caption{سوال \حوالہ{سوال_دائری_جوڑ_حل_ت} کا دور۔}
\label{شکل_سوال_دائری_جوڑ_حل_ت}
\end{figure}%%

جواب:\عددی{V_0=\SI{4}{\volt}}
\انتہا{سوال}
%==============================
\ابتدا{سوال}\شناخت{سوال_دائری_جوڑ_حل_ٹ}
شکل \حوالہ{شکل_سوال_دائری_جوڑ_حل_ٹ} میں ترکیب جوڑ سے \عددی{V_0} حاصل کریں۔
\begin{figure}
\centering
\begin{tikzpicture}[american voltages]
\draw(0,0) to [resistor,l={$\SI{8}{\kilo\ohm}$}]++(45:\xxx)coordinate(kT) to [resistor,l={$\SI{8}{\kilo\ohm}$}]++(-45:\xxx)coordinate(kR) to [resistor,l={$\SI{8}{\kilo\ohm}$}]++(-135:\xxx)coordinate(kB) to [resistor,l={$\SI{8}{\kilo\ohm}$}](0,0);
\draw(0,0) to [american voltage source,*-*,l={$\SI{12}{\volt}$}](kR);
\draw(kB)to[short,*-]++(0,-\yyy/8) to [short]++(-\xxx-\xxx/2,0)coordinate(kBL);
\draw(kT) to [short,*-]++(0,\yyy/8) to [short]++(-\xxx-\xxx/2,0) to [resistor,l={$\SI{8}{\kilo\ohm}$},v={$V_0$}](kBL);
\end{tikzpicture}
\caption{سوال \حوالہ{سوال_دائری_جوڑ_حل_ٹ} کا دور۔}
\label{شکل_سوال_دائری_جوڑ_حل_ٹ}
\end{figure}%%

جواب:\عددی{V_0=\SI{0}{\volt}}
\انتہا{سوال}
%======================
\ابتدا{سوال}\شناخت{سوال_دائری_جوڑ_حل_ث}
شکل \حوالہ{شکل_سوال_دائری_جوڑ_حل_ث} میں ترکیب جوڑ سے \عددی{V_0} حاصل کریں۔
\begin{figure}
\centering
\begin{tikzpicture}[american voltages]
\draw(0,0) to [american controlled voltage source,l={$3V_0$}]++(0,\yyy);
\draw(\xxx,0) to [american current source,*-*,l={$\SI{4}{\milli\ampere}$}]++(0,\yyy);
\draw(2*\xxx,0) to [resistor,-*,l={$\SI{2}{\kilo\ohm}$},v_>={$V_0$}]++(0,\yyy);
\draw(0,0) to [short]++(2*\xxx,0);
\draw(0,\yyy) to [resistor,l={$\SI{4}{\kilo\ohm}$}]++(\xxx,0) to [resistor,l={$\SI{6}{\kilo\ohm}$}]++(\xxx,0);
\draw(0,\yyy) to [short,*-]++(0,3/4*\yyy) to [resistor,l={$\SI{8}{\kilo\ohm}$}]++(2*\xxx,0) to [short,-*]++(0,-3/4*\yyy);
\end{tikzpicture}
\caption{سوال \حوالہ{سوال_دائری_جوڑ_حل_ث} کا دور۔}
\label{شکل_سوال_دائری_جوڑ_حل_ث}
\end{figure}%%

جواب:\عددی{V_0=\SI{32}{\volt}}
\انتہا{سوال}
%=====================
\ابتدا{سوال}\شناخت{سوال_دائری_جوڑ_حل_ج}
شکل \حوالہ{شکل_سوال_دائری_جوڑ_حل_ج} میں ترکیب جوڑ سے \عددی{V_0} حاصل کریں۔
\begin{figure}
\centering
\begin{tikzpicture}[american voltages]
\draw(0,0) to [resistor,*-,l={$\SI{2}{\kilo\ohm}$}]++(0,\yyy) to [american voltage source,-*,l={$\SI{4}{\volt}$}]++(0,\yyy);
\draw(-\xxx,0) to [resistor,l={$\SI{4}{\kilo\ohm}$}]++(0,\yyy) to [resistor,l={$\SI{6}{\kilo\ohm}$}]++(0,\yyy);
\draw(\xxx,0) to [resistor,*-,l={$\SI{2}{\kilo\ohm}$}]++(0,\yyy) to [resistor,-*,l={$\SI{4}{\kilo\ohm}$}]++(0,\yyy);
\draw(2*\xxx,0) to [resistor,l={$\SI{8}{\kilo\ohm}$},v_>={$V_0$}]++(0,\yyy) to [american current source,l_={$\SI{4}{\milli\ampere}$}]++(0,\yyy);
\draw(-\xxx,0) to [short]++(3*\xxx,0);
\draw(-\xxx,2*\yyy) to [short]++(3*\xxx,0);
\draw(0,\yyy) to [american current source,*-*,l={$\SI{2}{\milli\ampere}$}]++(-\xxx,0);
\draw(2*\xxx,\yyy) to [american voltage source,*-*,l_={$\SI{10}{\volt}$}]++(-\xxx,0);
\end{tikzpicture}
\caption{سوال \حوالہ{سوال_دائری_جوڑ_حل_ج} کا دور۔}
\label{شکل_سوال_دائری_جوڑ_حل_ج}
\end{figure}%%

جواب:\عددی{V_0=\SI{-11.67}{\volt}}
\انتہا{سوال}
%==============================
\ابتدا{سوال}\شناخت{سوال_دائری_جوڑ_حل_چ}
شکل \حوالہ{شکل_سوال_دائری_جوڑ_حل_چ} میں ترکیب جوڑ سے \عددی{V_0} حاصل کریں۔
\begin{figure}
\centering
\begin{tikzpicture}[american voltages]
\draw(0,0) to [american voltage source,l={$\SI{8}{\volt}$}]++(0,\yyy) to [resistor,l={$\SI{1}{\kilo\ohm}$}]++(0,\yyy);
\draw(\xxx,0) to [american controlled current source,*-,l_={$3I_0$}]++(0,\yyy) to [resistor,-*,l_={$\SI{2}{\kilo\ohm}$}]++(0,\yyy);
\draw(2*\xxx,0) to [resistor,l={$\SI{4}{\kilo\ohm}$},i<={$I_0$},v_>={$V_0$}]++(0,2*\yyy);
\draw(0,\yyy) to [resistor,*-*,l={$\SI{2}{\kilo\ohm}$}]++(\xxx,0);
\draw(0,0) to [short]++(2*\xxx,0);
\draw(0,2*\yyy) to [short]++(2*\xxx,0);
\end{tikzpicture}
\caption{سوال \حوالہ{سوال_دائری_جوڑ_حل_چ} کا دور۔}
\label{شکل_سوال_دائری_جوڑ_حل_چ}
\end{figure}%%

جواب:\عددی{V_0=\SI{4.57}{\volt}}
\انتہا{سوال}
%==============================
\ابتدا{سوال}\شناخت{سوال_دائری_جوڑ_حل_ح}
شکل \حوالہ{شکل_سوال_دائری_جوڑ_حل_ح} میں ترکیب جوڑ سے \عددی{V_0} حاصل کریں۔
\begin{figure}
\centering
\begin{tikzpicture}[american voltages]
\draw(0,0) to [american controlled voltage source,-*,l={$4V_x$}]++(0,\yyy);
\draw(\xxx,0) to [resistor,*-*,l={$\SI{6}{\kilo\ohm}$}]++(0,\yyy);
\draw(2*\xxx,0) to [american current source,*-*,l={$\SI{4}{\milli\ampere}$}]++(0,\yyy);
\draw(3*\xxx,0) to [resistor,*-*,l={$\SI{4}{\kilo\ohm}$},v_>={$V_x$}]++(0,\yyy);
\draw(4*\xxx,0) to [resistor,l={$\SI{5}{\kilo\ohm}$},v_>={$V_0$}]++(0,\yyy) to [resistor,l_={$\SI{1}{\kilo\ohm}$}]++(-\xxx,0);
\draw(0,0) to [short]++(4*\xxx,0);
\draw(0,\yyy) to [resistor,l={$\SI{4}{\kilo\ohm}$}]++(\xxx,0) to [american voltage source,l={$\SI{8}{\volt}$}]++(\xxx,0) to [resistor,l={$\SI{2}{\kilo\ohm}$}]++(\xxx,0);
\draw(0,\yyy) to [short]++(0,3/4*\yyy) to [american current source,l={$\SI{2}{\milli\ampere}$}]++(3*\xxx,0) to [short]++(0,-3/4*\yyy);
\end{tikzpicture}
\caption{سوال \حوالہ{سوال_دائری_جوڑ_حل_ح} کا دور۔}
\label{شکل_سوال_دائری_جوڑ_حل_ح}
\end{figure}%%

جواب:\عددی{V_0=\tfrac{660}{13}\,\si{\volt}}
\انتہا{سوال}
%==============================
\ابتدا{سوال}\شناخت{سوال_دائری_جوڑ_حل_خ}
شکل \حوالہ{شکل_سوال_دائری_جوڑ_حل_خ} میں ترکیب جوڑ سے \عددی{I_0} حاصل کریں۔
\begin{figure}
\centering
\begin{tikzpicture}[american voltages]
\draw(0,0) to [american voltage source,-*,l={$\SI{8}{\volt}$}]++(0,\yyy) to [american controlled current source,l={$3I_x$}]++(0,\yyy);
\draw(\xxx,0) to [resistor,*-*,l={$\SI{2}{\kilo\ohm}$},i<_={$I_0$}]++(0,\yyy) to [resistor,-*,l={$\SI{2}{\kilo\ohm}$}]++(0,\yyy);
\draw(2*\xxx,\yyy) to [american current source,*-,l={$\SI{4}{\milli\ampere}$}]++(0,-\yyy);
\draw(2*\xxx,\yyy) to [american controlled voltage source,l_={$3V_x$}]++(0,\yyy);
\draw(0,0) to [short]++(2*\xxx,0);
\draw(0,2*\yyy) to [short]++(2*\xxx,0);
\draw(0,\yyy) to [resistor,l={$\SI{4}{\kilo\ohm}$},v={$V_x$}]++(\xxx,0) to [resistor,l={$\SI{4}{\kilo\ohm}$},i<={$I_x$}]++(\xxx,0);
\end{tikzpicture}%
\caption{سوال \حوالہ{سوال_دائری_جوڑ_حل_خ} کا دور۔}
\label{شکل_سوال_دائری_جوڑ_حل_خ}
\end{figure}%%

جواب:\عددی{I_0=\tfrac{16}{3}\,\si{\milli\ampere}}
\انتہا{سوال}
%==============================
\ابتدا{سوال}\شناخت{سوال_دائری_جوڑ_حل_د}
شکل \حوالہ{شکل_سوال_دائری_جوڑ_حل_د} میں تمام جوڑ کے دباو حاصل کریں۔

\begin{figure}
\centering
\begin{tikzpicture}[american voltages]
\draw(0,0) to [american voltage source,-*,l={$\SI{10}{\volt}$}]++(0,\yyy) to [resistor,l={$\SI{10}{\kilo\ohm}$},v={$V_x$}]++(0,\yyy)node[left]{$V_1$};
\draw(\xxx,0)node[ground]{} to [resistor,*-*,l={$\SI{10}{\kilo\ohm}$}]++(0,\yyy)node[above left]{$V_3$} to [american controlled voltage source,-*,l={$2V_x$}]++(0,\yyy);
\draw(2*\xxx,2*\yyy)node[above]{$V_2$} to [american current source,l={$\SI{2}{\milli\ampere}$}]++(0,-\yyy) to [american controlled current source,*-,l={$3I_x$}]++(0,-\yyy);
\draw(0,0) to [short]++(2*\xxx,0);
\draw(0,2*\yyy) to [resistor,l={$\SI{6}{\kilo\ohm}$},i={$I_x$}]++(\xxx,0) to [short]++(\xxx,0);
\draw(0,\yyy) to [resistor,l={$\SI{8}{\kilo\ohm}$}]++(\xxx,0) to [resistor,l={$\SI{8}{\kilo\ohm}$}]++(\xxx,0)node[right]{$V_4$};
\end{tikzpicture}
\caption{سوال \حوالہ{سوال_دائری_جوڑ_حل_د} کا دور۔}
\label{شکل_سوال_دائری_جوڑ_حل_د}
\end{figure}%%

جوابات:\عددی{V_1=\tfrac{510}{61}\,\si{\volt}}، \عددی{V_2=\tfrac{450}{61}\,\si{\volt}}، \عددی{V_3=\tfrac{250}{61}\,\si{\volt}}، \عددی{V_4=\tfrac{986}{61}\,\si{\volt}}
\انتہا{سوال}
%=============================
\ابتدا{سوال}\شناخت{سوال_دائری_دائری_حل_الف}
شکل \حوالہ{شکل_سوال_دائری_دائری_حل_الف} کو دائری ترکیب سے حل کرتے ہوئے \عددی{I_0} اور \عددی{I_1} دریافت کریں۔
\begin{figure}
\centering
\begin{tikzpicture}
\draw(0,0) to [american voltage source,l={$\SI{8}{\volt}$}]++(0,\yy) to [resistor,l={$\SI{2}{\kilo\ohm}$},i<={$I_0$}]++(\xx,0) to [resistor,l={$\SI{6}{\kilo\ohm}$}]++(0,-\yy) to [short](0,0);
\draw(\xx,0) to [short,*-]++(\xx,0) to [american voltage source,l_={$\SI{6}{\volt}$}]++(0,\yy) to [resistor,-*,l_={$\SI{4}{\kilo\ohm}$},i>_={$I_1$}]++(-\xx,0);
\end{tikzpicture}
\caption{سوال \حوالہ{سوال_دائری_دائری_حل_الف} کا دور۔}
\label{شکل_سوال_دائری_دائری_حل_الف}
\end{figure}%%

جواب:\عددی{I_0=\SI{-1}{\milli\ampere}}، \عددی{I_1=0}
\انتہا{سوال}
%=============================
\ابتدا{سوال}\شناخت{سوال_دائری_دائری_حل_ب}
شکل \حوالہ{شکل_سوال_دائری_دائری_حل_ب} کو دائری ترکیب سے حل کرتے ہوئے \عددی{I_0} دریافت کریں۔

\begin{figure}
\centering
\begin{tikzpicture}
\draw(0,0) to [american voltage source,l={$\SI{12}{\volt}$}]++(0,2*\yy) to [resistor,l={$\SI{2}{\kilo\ohm}$}]++(\xx,0) to [resistor,l={$\SI{6}{\kilo\ohm}$}]++(\xx,0) to [resistor,l_={$\SI{2}{\kilo\ohm}$}]++(0,-2*\yy) to [short](0,0);
\draw(\xx,0) to [american voltage source,*-,l_={$\SI{8}{\volt}$}]++(0,\yy) to [resistor,-*,l_={$\SI{4}{\kilo\ohm}$},i<={$I_0$}]++(0,\yy);
\end{tikzpicture}
\caption{سوال \حوالہ{سوال_دائری_دائری_حل_ب} کا دور۔}
\label{شکل_سوال_دائری_دائری_حل_ب}
\end{figure}%%

جواب:\عددی{I_0=\tfrac{2}{7}\,\si{\milli\ampere}}
\انتہا{سوال}
%===============================

\ابتدا{سوال}\شناخت{سوال_دائری_دائری_حل_پ}
شکل \حوالہ{شکل_سوال_دائری_دائری_حل_پ}-الف کو دائری ترکیب سے حل کرتے ہوئے \عددی{V_0} دریافت کریں۔
\begin{figure}
\centering
\begin{tikzpicture}[american voltages]
\draw(0,0) to [resistor,l_={$\SI{2}{\kilo\ohm}$}]++(0,\yy) to [american voltage source,l={$\SI{4}{\volt}$}]++(\xx,0) to [american voltage source,l={$\SI{6}{\volt}$}]++(\xx,0) to [resistor,l={$\SI{6}{\kilo\ohm}$}]++(0,-\yy) to [short] (0,0);
\draw(\xx,0) to [resistor,*-*,l_={$\SI{4}{\kilo\ohm}$},v^<={$V_0$}]++(0,\yy);
\end{tikzpicture}
\caption{سوال \حوالہ{سوال_دائری_دائری_حل_پ} کا دور۔}
\label{شکل_سوال_دائری_دائری_حل_پ}
\end{figure}%%

جواب:\عددی{V_0=-\tfrac{12}{11}\,\si{\volt}}
\انتہا{سوال}
%===============================
\ابتدا{سوال}\شناخت{سوال_دائری_دائری_حل_ت}
دائری ترکیب سے حل کرتے ہوئے شکل \حوالہ{شکل_سوال_دائری_دائری_حل_ت} میں \عددی{V_0} دریافت کریں۔

\begin{figure}
\centering
\begin{tikzpicture}[american voltages]
\draw(0,0) to [american voltage source,l={$\SI{10}{\volt}$}]++(0,\yy) to [resistor,l={$\SI{4}{\kilo\ohm}$}]++(\xx,0) to [resistor,l={$\SI{2}{\kilo\ohm}$}]++(\xx,0) to [american current source,l_={$\SI{2}{\milli\ampere}$},v^<={$V_0$}]++(0,-\yy) to [short] (0,0);
\draw(\xx,0) to [resistor,*-*,l={$\SI{6}{\kilo\ohm}$}]++(0,\yy);
\end{tikzpicture}
\caption{سوال \حوالہ{سوال_دائری_دائری_حل_ت} کا دور۔}
\label{شکل_سوال_دائری_دائری_حل_ت}
\end{figure}%%

جواب:\عددی{V_0=\SI{-2.8}{\volt}}
\انتہا{سوال}
%=======================
\ابتدا{سوال}\شناخت{سوال_دائری_دائری_حل_ٹ}
دائری ترکیب سے حل کرتے ہوئے شکل \حوالہ{شکل_سوال_دائری_دائری_حل_ٹ} کے \عددی{\SI{8}{\kilo\ohm}} مزاحمت میں طاقتی ضیاع حاصل کریں۔
\begin{figure}
\centering
\begin{tikzpicture}
\draw(0,0) to [american voltage source,l={$\SI{12}{\volt}$}]++(0,\yy) to [short]++(0,\yy);
\draw(\xx,0) to [resistor,*-*,l={$\SI{4}{\kilo\ohm}$}]++(0,\yy) to [resistor,-*,l={$\SI{4}{\kilo\ohm}$}]++(0,\yy);
\draw(2*\xx,0) to [american voltage source,l={$\SI{6}{\volt}$}]++(0,\yy) to [american voltage source,l={$\SI{8}{\volt}$}]++(0,\yy);
\draw(0,0) to [short]++(2*\xx,0);
\draw(0,\yy) to [resistor,*-*,l={$\SI{6}{\kilo\ohm}$}]++(\xx,0) to [resistor,-*,l={$\SI{8}{\kilo\ohm}$}]++(\xx,0);
\draw(0,2*\yy) to [resistor,l={$\SI{2}{\kilo\ohm}$}]++(\xx,0) to [resistor,l={$\SI{2}{\kilo\ohm}$}]++(\xx,0);
\end{tikzpicture}
\caption{سوال \حوالہ{سوال_دائری_دائری_حل_ٹ} کا دور۔}
\label{شکل_سوال_دائری_دائری_حل_ٹ}
\end{figure}%

جواب:\عددی{\SI{106.5}{\micro\watt}}
\انتہا{سوال}
%=======================
\ابتدا{سوال}\شناخت{سوال_دائری_دائری_حل_ث}
دائری ترکیب سے حل کرتے ہوئے شکل \حوالہ{شکل_سوال_دائری_دائری_حل_ث} میں \عددی{V_0} دریافت کریں۔

\begin{figure}
\centering
\begin{tikzpicture}[american voltages]
\draw(0,0) to [american voltage source,l={$\SI{6}{\volt}$}]++(0,\yy) to [short]++(0,\yy);
\draw(\xx,0) to [resistor,*-*,l={$\SI{2}{\kilo\ohm}$}]++(0,\yy) to [american current source,-*,l={$\SI{2}{\milli\ampere}$}]++(0,\yy);
\draw(2*\xx,0) to [resistor,l={$\SI{3}{\kilo\ohm}$}]++(0,\yy) to [american current source,l_={$\SI{4}{\milli\ampere}$}]++(0,\yy);
\draw(0,0) to [short]++(2*\xx,0);
\draw(0,\yy) to [resistor,*-*,l={$\SI{6}{\kilo\ohm}$}]++(\xx,0) to [resistor,-*,l={$\SI{1}{\kilo\ohm}$}]++(\xx,0);
\draw(0,2*\yy) to [resistor,l={$\SI{2}{\kilo\ohm}$},v={$V_0$}]++(\xx,0) to [resistor,l={$\SI{4}{\kilo\ohm}$}]++(\xx,0);
\end{tikzpicture}
\caption{سوال \حوالہ{سوال_دائری_دائری_حل_ث} کے ادوار۔}
\label{شکل_سوال_دائری_دائری_حل_ث}
\end{figure}%

جواب:\عددی{V_0=\SI{-12}{\volt}}
\انتہا{سوال}
%=====================
\ابتدا{سوال}\شناخت{سوال_دائری_دائری_حل_ج}
دائری ترکیب سے حل کرتے ہوئے شکل \حوالہ{شکل_سوال_دائری_دائری_حل_ج} کو دائری ترکیب سے حل کرتے ہوئے \عددی{I_0} حاصل کریں۔
\begin{figure}
\centering
\begin{tikzpicture}
\draw(0,0) to [resistor,l_={$\SI{1}{\kilo\ohm}$}]++(0,\yy) to [resistor,l={$\SI{1}{\kilo\ohm}$}]++(\xx,0) to [american current source,l={$\SI{2}{\milli\ampere}$}]++(0,-\yy) to [short,i={$I_0$}](0,0);
\draw(\xx,\yy) to [resistor,*-*,l={$\SI{2}{\kilo\ohm}$}]++(\xx,0) to [resistor,l={$\SI{4}{\kilo\ohm}$}]++(0,-\yy) to [short,-*]++(-\xx,0);
\draw(0,\yy) to [short,*-]++(0,3/4*\yy) to [american current source,l={$\SI{4}{\milli\ampere}$}]++(2*\xx,0) to [short]++(0,-3/4*\yy);
\end{tikzpicture}
\caption{سوال \حوالہ{سوال_دائری_دائری_حل_ج} کا دور۔}
\label{شکل_سوال_دائری_دائری_حل_ج}
\end{figure}%

جواب:\عددی{I_0=\SI{3}{\milli\ampere}}
\انتہا{سوال}
%=========================
\ابتدا{سوال}\شناخت{سوال_دائری_دائری_حل_چ}
دائری ترکیب سے حل کرتے ہوئے شکل \حوالہ{شکل_سوال_دائری_دائری_حل_چ} کو دائری ترکیب سے حل کرتے ہوئے \عددی{I_0} حاصل کریں۔

\begin{figure}
\centering
\begin{tikzpicture}
\draw(0,0) to [resistor,l={$\SI{2}{\kilo\ohm}$}]++(0,\yy) to [resistor,l={$\SI{4}{\kilo\ohm}$}]++(\xx,0) to [american current source,l_={$\SI{10}{\milli\ampere}$}]++(0,-\yy) to [short](0,0);
\draw(\xx,\yy) to [resistor,*-*,l={$\SI{6}{\kilo\ohm}$}]++(\xx,0) to [resistor,l_={$\SI{4}{\kilo\ohm}$}]++(0,-\yy) to [short,-*]++(-\xx,0);
\draw(0,\yy) to [short,*-]++(0,3/4*\yy) to [resistor,l={$\SI{8}{\kilo\ohm}$},i={$I_0$}]++(2*\xx,0) to [short]++(0,-3/4*\yy);
\end{tikzpicture}
\caption{سوال \حوالہ{سوال_دائری_دائری_حل_چ} کے ادوار۔}
\label{شکل_سوال_دائری_دائری_حل_چ}
\end{figure}%

جواب:\عددی{I_0=\SI{212.8}{\micro\ampere}}
\انتہا{سوال}
%===============
\ابتدا{سوال}\شناخت{سوال_دائری_دائری_حل_ح}
دائری ترکیب سے حل کرتے ہوئے شکل \حوالہ{شکل_سوال_دائری_دائری_حل_ح} کو دائری ترکیب سے حل کرتے ہوئے \عددی{I_0} حاصل کریں۔
\begin{figure}
\centering
\begin{tikzpicture}
\draw(0,\yy) to [american current source,*-,l_={$\SI{4}{\milli\ampere}$}]++(0,-\yy);
\draw(0,\yy) to [american current source,l={$\SI{6}{\milli\ampere}$}]++(0,\yy);
\draw(\xx,0) to [resistor,*-*,l_={$\SI{6}{\kilo\ohm}$},i<={$I_0$}]++(0,\yy) to [resistor,-*,l_={$\SI{2}{\kilo\ohm}$}]++(0,\yy);
\draw(2*\xx,0) to [american voltage source,*-*,l_={$\SI{4}{\volt}$}]++(0,\yy) to [resistor,-*,l_={$\SI{4}{\kilo\ohm}$}]++(0,\yy);
\draw(3*\xx,0) to [american current source,l_={$\SI{2}{\milli\ampere}$}]++(0,2*\yy);
\draw(0,\yy) to [resistor,l={$\SI{1}{\kilo\ohm}$}]++(\xx,0) to [resistor,l={$\SI{2}{\kilo\ohm}$}]++(\xx,0);
\draw(0,0) to [short]++(3*\xx,0); 
\draw(0,2*\yy) to [short]++(3*\xx,0); 
\end{tikzpicture}
\caption{سوال \حوالہ{سوال_دائری_دائری_حل_ح} کا دور۔}
\label{شکل_سوال_دائری_دائری_حل_ح}
\end{figure}%
جواب:\عددی{I_0=\SI{-0.4}{\milli\ampere}}
\انتہا{سوال}
%=================
\ابتدا{سوال}\شناخت{سوال_دائری_دائری_حل_خ}
دائری ترکیب سے حل کرتے ہوئے شکل \حوالہ{شکل_سوال_دائری_دائری_حل_خ} کو دائری ترکیب سے حل کرتے ہوئے  تمام دائری رو دریافت کریں۔
\begin{figure}
\centering
\begin{tikzpicture}[american voltages]
\draw(0,0) to [american voltage source,l={$\SI{6}{\volt}$}]++(0,\yy) to [resistor,l={$\SI{1}{\ohm}$}]++(0,\yy);
\draw(\xx,0) to [american controlled current source,*-*,l={$4I_x$}]++(0,\yy) to [american controlled voltage source,-*,l={$2V_x$},i={$I_x$}]++(0,\yy);
\draw(2*\xx,0) to [resistor,l_={$\SI{2}{\ohm}$}]++(0,\yy) to [american current source,l_={$\SI{1}{\ampere}$}]++(0,\yy);
\draw(0,0) to [short]++(2*\xx,0);
\draw(0,2*\yy) to [resistor,l={$\SI{1}{\ohm}$}]++(\xx,0) to [resistor,l={$\SI{2}{\ohm}$}]++(\xx,0);
\draw(0,\yy) to [resistor,*-,l={$\SI{4}{\ohm}$},v_>={$V_x$}]++(\xx,0) to [resistor,-*,l={$\SI{6}{\ohm}$}]++(\xx,0);
%currents
\draw[stealth-] ([shift={(-150:\x/5)}]\xx/2-0.4,\yy/2-0.4) arc (-150:150:\x/5);
\draw(\xx/2-0.4,\yy/2-0.4)node{$I_1$};
\draw[stealth-] ([shift={(-150:\x/5)}]\xx/2-0.4,\yy+\yy/2+0.4) arc (-150:150:\x/5);
\draw(\xx/2-0.4,\yy+\yy/2+0.4)node{$I_2$};
\draw[stealth-] ([shift={(-150:\x/5)}]\xx+\xx/2,\yy+\yy/2) arc (-150:150:\x/5);
\draw(\xx+\xx/2,\yy+\yy/2)node{$I_3$};
\draw[stealth-] ([shift={(-150:\x/5)}]\xx+\xx/2,\yy/2) arc (-150:150:\x/5);
\draw(\xx+\xx/2,\yy/2)node{$I_4$};
\end{tikzpicture}
\caption{سوال \حوالہ{سوال_دائری_دائری_حل_خ} کا دور۔}
\label{شکل_سوال_دائری_دائری_حل_خ}
\end{figure}%

جوابات:\عددی{I_1=-\tfrac{12}{7}\,\si{\ampere}}، \عددی{I_2=-\tfrac{72}{49}\,\si{\ampere}}، \عددی{I_3=-1\,\si{\ampere}}، 
\عددی{I_4=\tfrac{8}{49}\,\si{\ampere}}
\انتہا{سوال}
%=======================
\ابتدا{سوال}\شناخت{سوال_دائری_دائری_حل_د}
دائری ترکیب سے حل کرتے ہوئے شکل \حوالہ{شکل_سوال_دائری_دائری_حل_د} کو دائری ترکیب سے حل کرتے ہوئے  تمام دائری رو دریافت کریں۔
\begin{figure}
\centering
\begin{tikzpicture}[american voltages]
\draw(0,0) to [american voltage source,l={$\SI{8}{\volt}$}]++(0,\yy);
\draw(\xx,0) to [american current source,*-*,l={$\SI{2}{\ampere}$}]++(0,\yy);
\draw(2*\xx,\yy) to [american controlled current source,*-*,l={$3V_x$}]++(0,-\yy);
\draw(3*\xx,0) to [resistor,l_={$\SI{6}{\ohm}$}]++(0,\yy);
\draw(0,0) to [short]++(3*\xx,0);
\draw(0,\yy) to [resistor,l={$\SI{2}{\ohm}$}]++(\xx,0) to [resistor,l={$\SI{2}{\ohm}$}]++(\xx,0) to [resistor,l={$\SI{4}{\ohm}$}]++(\xx,0);
\draw(3*\xx,\yy) to [short,*-]++(0,3/4*\yy) to [american voltage source,l_={$\SI{6}{\volt}$}]++(-\xx,0) to [resistor,l_={$\SI{2}{\ohm}$},v^<={$V_x$}]++(-\xx,0) to [short,-*]++(0,-3/4*\yy);
%currents
\draw[stealth-] ([shift={(-150:\xx/5)}]\xx/2,\yy/2-0.5) arc (-150:150:\x/5);
\draw(\xx/2,\yy/2-0.5)node{$I_1$};
\draw[stealth-] ([shift={(-150:\x/5)}]\xx+\xx/2,\yy/2) arc (-150:150:\x/5);
\draw(\xx+\xx/2,\yy/2)node{$I_2$};
\draw[stealth-] ([shift={(-150:\x/5)}]2*\xx+\xx/2+0.4,\yy/2-0.5) arc (-150:150:\x/5);
\draw(2*\xx+\xx/2+0.4,\yy/2-0.5)node{$I_3$};
\draw[stealth-] ([shift={(-150:\x/5)}]2*\xx,\yy+3/8*\yy) arc (-150:150:\x/5);
\draw(2*\xx,\yy+3/8*\yy)node{$I_4$};
\end{tikzpicture}
\caption{سوال \حوالہ{سوال_دائری_دائری_حل_د} کا دور۔}
\label{شکل_سوال_دائری_دائری_حل_د}
\end{figure}%

جوابات:\عددی{I_1=-\tfrac{2}{3}\,\si{\ampere}}، \عددی{I_2=\tfrac{4}{3}\,\si{\ampere}}، \عددی{I_3=-\tfrac{2}{3}\,\si{\ampere}}،
 \عددی{I_4=-\tfrac{1}{3}\,\si{\ampere}}
\انتہا{سوال}
%======================


\باب{جوڑ اور دائری تجزیہ}
گزشتہ باب میں سادہ ترین ادوار کو کرخوف قوانین سے حل کرنا دکھایا گیا۔اس باب میں متعدد جوڑ اور متعدد دائروں والے ادوار کو کرخوف قوانین سے حل کرنا دکھایا جائے گا۔کرخوف قانون رو سے ہر جوڑ پر داخلی اور خارجی رو کے مجموعوں کو برابر پر کرتے ہوئے دور کے تمام جوڑوں پر دباو حاصل کیا جاتا ہے۔اس کے برعکس کرخوف قانون دباو کی مدد سے دور کے ہر دائرے میں دباو کے گھٹاو کے مجموعے کو دائرے میں دباو کے  بڑھاو کے مجموعے کے برابر پر کرتے ہوئے تمام دائروں کی رو حاصل کی جاتی ہے۔عموماً  دور یا تو کرخوف قانون دباو اور یا کرخوف قانون رو سے زیادہ آسانی سے حل ہوتا ہے۔آسان طریقہ چننا اس باب میں سکھایا جائے گا۔

\حصہ{تجزیہ جوڑ}
دور کو \اصطلاح{ترکیب جوڑ}\فرہنگ{جوڑ!ترکیب}\حاشیہب{nodal analysis}\فرہنگ{nodal analysis} سے حل کرتے ہوئے  جوڑ کے دباو کو  نا معلوم متغیرات چننا جاتا ہے۔کسی ایک جوڑ کو حوالہ چنتے ہوئے بقایا جوڑ کے دباو اس جوڑ سے ناپے جاتے ہیں۔یوں جس جوڑ کو حوالہ چننا گیا ہو، اس کی دباو کو صفر وولٹ تصور کیا جاتا ہے اور اس جوڑ کو  \اصطلاح{برقی زمین} کہا جاتا ہے۔عموماً اس جوڑ کو برقی زمین چننا جاتا ہے جس کے ساتھ سب سے زیادہ پرزے جڑے ہوں۔عموماً آلات کو موصل ڈبوں میں بند رکھا جاتا ہے اور عام طور دور کے برقی زمین کو ڈبے کے ساتھ جوڑا جاتا ہے۔ایسی صورت میں ڈبے کی سطح  بھی \عددی{\SI{0}{\volt}} پر ہوتی ہے۔

ہم دباو جوڑ کے متغیرات کو مثبت تصور کریں گے۔حقیقی دباو کی قیمت زمین کی نسبت سے منفی ہونے کی صورت میں تجزیے سے منفی قیمت حاصل ہو گی۔ 

\begin{figure}
\centering
\includegraphics{figNodalCurrentsFromNodeVoltages}
\caption{دباو جوڑ سے بازو کی رو حاصل کی جا سکتی ہے۔}
\label{شکل_جوڑ_دباو__جوڑ_سے_رو_کا_حصول}
\end{figure}

آئیں دباو جوڑ جاننے کی افادیت کو  شکل \حوالہ{شکل_جوڑ_دباو__جوڑ_سے_رو_کا_حصول} کی مدد سے جانیں۔اس دور میں \عددی{a}، \عددی{b}، \عددی{c} اور \عددی{z} جوڑ پائے جاتے ہیں۔ہم نے جوڑ \عددی{z} کو برقی زمین چننا ہے لہٰذا اس کی دباو \عددی{\SI{0}{\volt}} ہے۔بقایا تین جوڑ کی دباو کو شکل میں دکھایا گیا ہے۔برقی زمین کو علامت سے ظاہر کیا گیا ہے۔

بالائی بائیں مزاحمت پر دباو درج ذیل پایا جاتا ہے
\begin{align*}
V_{ab}&=V_a-V_b\\
&=8-5\\
&=\SI{3}{\volt}
\end{align*}
لہٰذا قانون اوہم سے مزاحمت میں رو درج ذیل حاصل کی جاتی ہے۔
\begin{align*}
i_1&=\frac{V_{ab}}{\SI{2}{\kilo\ohm}}\\
&=\frac{3}{2000}\\
&=\SI{1.5}{\milli\ampere}
\end{align*}
اسی طرح بالائی دائیں مزاحمت پر دباو  درج ذیل ہو گا
\begin{align*}
V_{bc}&=V_b-V_c\\
&=5-4\\
&=\SI{1}{\volt}
\end{align*}
جس سے رو
\begin{align*}
i_2&=\frac{V_{bc}}{\SI{1}{\kilo\ohm}}\\
&=\frac{1}{1000}\\
&=\SI{1}{\milli\ampere}
\end{align*}
حاصل ہوتی ہے۔درمیانے مزاحمت پر دباو اور اس کی رو درج ذیل ہیں۔
\begin{align*}
V_{bz}&=V_b-V_z\\
&=5-0\\
&={\SI{5}{\volt}}\\
i_4&=\frac{V_{bz}}{\SI{10}{\kilo\ohm}}\\
&=\frac{5}{10000}\\
&=\SI{0.5}{\milli\ampere}
\end{align*}
چونکہ \عددی{\SI{1}{\kilo\ohm}} اور \عددی{\SI{4}{\kilo\ohm}} سلسلہ وار جڑے ہیں لہٰذا \عددی{\SI{4}{\kilo\ohm}} میں بھی \عددی{\SI{1}{\milli\ampere}} رو پائی جائے گی۔آپ اسی قیمت کو دباو جوڑ سے بھی حاصل کر سکتے ہیں یعنی
\begin{align*}
V_{cz}&=V_c-V_z\\
&=4-0\\
&=\SI{4}{\volt}\\
i_3&=\frac{V_{cz}}{\SI{4}{\kilo\ohm}}\\
&=\frac{4}{4000}\\
&=\SI{1}{\milli\ampere}
\end{align*}

یہاں اتمنان کر لیں کہ تمام جوڑوں پر آمدی رو اور خارجی رو برابر ہوں۔جوڑ \عددی{b} پر آمدی رو \عددی{\SI{1.5}{\milli\ampere}} ہے جو خارجی رو کے مجموعے \عددی{\SI{1}{\milli\ampere}+\SI{0.5}{\milli\ampere}} کے عین برابر ہے۔اسی طرح جوڑ \عددی{c} پر آمدی اور خارجی رو \عددی{\SI{1}{\milli\ampere}} ہیں۔جوڑ \عددی{a} پر کرخوف قانون رو سے منبع دباو کے مثبت سرے سے خارجی رو \عددی{\SI{1.5}{\milli\ampere}} حاصل ہوتی ہے۔

کسی بھی دو جوڑ \عددی{m} اور \عددی{n} کے مابین جڑی مزاحمت \عددی{R_{mn}} کی رو \عددی{i_R} قانون اوہم
\begin{align}\label{مساوات-جوڑ_قانون_اوہم}
i_R=\frac{v_m-v_n}{R_{mn}}
\end{align}
سے حاصل کی جاتی ہے۔

اب جب ہم دباو جوڑ کی افادیت جان چکے ہیں آئیں ترکیب جوڑ پر غور کریں۔اگر دور میں \عددی{J} جوڑ پائے جاتے ہوں تب ہمیں \عددی{J} دباو دریافت کرنے ہوں گے۔کسی ایک جوڑ کو زمین چنتے ہوئے اس کی دباو \عددی{\SI{0}{\volt}} تصور کی جاتی ہے۔یوں بقایا \عددی{J-1} جوڑ کی دباو کو نا معلوم متغیرات تصور کیا جاتا ہے۔ان \عددی{J-1} جوڑ پر کرخوف قانون رو کا اطلاق کرتے ہوئے \عددی{J-1} مساوات لکھے جاتے ہیں۔آپ جانتے ہیں ہیں کہ \عددی{J-1} متغیرات معلوم کرنے کی خاطر \عددی{J-1} ہمزاد مساوات درکار ہیں۔یوں ان \عددی{J-1} ہمزاد مساوات کے حل سے تمام نا معلوم دباو جوڑ حاصل ہوتے ہیں۔کسی بھی جوڑ پر کروخوف کی مساوات لکھتے ہوئے جوڑ سے منسلک تمام بازو کی رو کو مساوات \حوالہ{مساوات-جوڑ_قانون_اوہم} کی طرز پر لکھا جاتا ہے۔یوں مزاحمت جانتے ہوئے، رو کو نا معلوم دباو کی صورت میں لکھا جاتا ہے۔اس طرح کرخوف قانون رو کی مساوات میں صرف نا معلوم دباو بطور متغیرات پائے جائیں گے۔

 یاد رہے کہ برقی دباو دو نقطوں کے مابین ہوتا ہے۔کسی نقطے کی حتمی دباو کوئی معنی نہیں رکھتی۔جوڑ پر کرخوف قانون رو کی مساوات لکھتے ہوئے جوڑ کا دباو زمین کے حوالے سے ناپا جاتا ہے۔ یوں شکل \حوالہ{شکل_جوڑ_دباو__جوڑ_سے_رو_کا_حصول} میں جوڑ \عددی{a} کا دباو جوڑ \عددی{z} کے حوالے سے \عددی{\SI{8}{\volt}} ہے اور جوڑ \عددی{b} کا دباو جوڑ \عددی{z} کے حوالے سے \عددی{\SI{5}{\volt}} ہے۔اس کے برعکس جوڑ \عددی{b} کے حوالے سے جوڑ \عددی{a} کا دباو \عددی{\SI{3}{\volt}} ہے جبکہ جوڑ \عددی{a} کے حوالے سے جوڑ \عددی{c} کا دباو \عددیء{\SI{-4}{\volt}} اور جوڑ \عددی{z} کا دباو \عددیء{\SI{-8}{\volt}} ہے۔ 


آئیں ترکیب جوڑ کو چند مثالوں کی مدد سے سیکھیں۔ہم آسان ترین مثال سے شروع کرتے ہوئے بتدریج مشکل مثال پیش کریں گے۔

\حصہ{غیر تابع منبع رو استعمال کرنے والے ادوار}
شکل \حوالہ{شکل_جوڑ_تین_جوڑ} میں تین جوڑ والا دور دکھایا گیا ہے جن میں نچلے جوڑ کو زمین چننا گیا ہے۔بقایا دو جوڑ کے نا معلوم برقی دباو کو متغیرات \عددی{v_1} اور \عددی{v_2} ظاہر کرتے ہیں۔ہم تمام شاخوں میں رو کی سمت چنتے ہیں۔یوں \عددی{i_1} کو بالائی بائیں جوڑ سے زمین کی جانب رواں چننا گیا ہے۔اسی طرح \عددی{i_2} کو بالائی بائیں جوڑ سے بالائی دائیں جوڑ کی جانب رواں چننا گیا ہے جبکہ \عددی{i_3} کو بالائی دائیں جوڑ سے زمین کی طرف رواں چننا گیا ہے۔
\begin{figure}
\centering
\includegraphics{figNodalIndependantSourcesNodalEquationsA}
\caption{تین جوڑ والا دور۔}
\label{شکل_جوڑ_تین_جوڑ}
\end{figure}

بالائی بائیں جوڑ پر کرخوف قانون رو کی مساوات لکھتے ہیں۔جوڑ سے خارجی رو کو مثبت اور داخلی رو کو منفی لکھتے ہوئے درج ذیل لکھا جا سکتا ہے۔
\begin{align}\label{مساوات_جوڑ_پہلی_مثال_الف}
i_1-i_A+i_2&=0
\end{align}
قانون اوہم استعمال کرتے ہوئے اسے یوں
\begin{align*}
\frac{v_1}{R_1}-i_A+\frac{v_1-v_2}{R_2}=0
\end{align*}
یا 
\begin{align}\label{مساوات_جوڑ_پہلی_مثال_ب}
\left( \frac{1}{R_1}+\frac{1}{R_2}\right) v_1 - \frac{v_2}{R_2}=i_A
\end{align}
لکھا جا سکتا ہے۔بالائی دائیں جوڑ کے لئے
\begin{align}\label{مساوات_جوڑ_پہلی_مثال_پ}
-i_2+i_3+i_B&=0
\end{align}
اور
\begin{align*}
-\left(\frac{v_1-v_2}{R_2}\right)+\frac{v_2}{R_3}+i_B&=0
\end{align*}
یعنی
\begin{align}\label{مساوات_جوڑ_پہلی_مثال_ت}
-\frac{v_1}{R_2}+\left(\frac{1}{R_2}+\frac{1}{R_3}\right)v_2=-i_B
\end{align}
لکھا جائے گا۔نچلے جوڑ یعنی برقی زمین پر  کرخوف قانون رو کی مساوات لکھتے ہیں۔
\begin{align}\label{مساوات_جوڑ_پہلی_مثال_ٹ}
-i_1+i_A-i_3-i_B=0
\end{align}
مساوات \حوالہ{مساوات_جوڑ_پہلی_مثال_الف} اور مساوات \حوالہ{مساوات_جوڑ_پہلی_مثال_پ} کے مجموعے کو منفی ایک سے ضرب دینے سے مساوات \حوالہ{مساوات_جوڑ_پہلی_مثال_ٹ} حاصل ہوتا ہے۔مساوات \حوالہ{مساوات_جوڑ_پہلی_مثال_الف}، مساوات \حوالہ{مساوات_جوڑ_پہلی_مثال_پ} اور  مساوات \حوالہ{مساوات_جوڑ_پہلی_مثال_ٹ} میں کسی بھی دو مساواتوں سے تیسری مساوات حاصل کی جا سکتی ہے۔یوں ان میں صرف دو عدد مساوات آزاد مساوات ہیں جبکہ تیسری مساوات تابع مساوات ہے۔شکل \حوالہ{شکل_جوڑ_تین_جوڑ} کے دور میں کل تین عدد جوڑ ہیں۔آپ نے دیکھا کہ اس دور سے صرف دو عدد آزاد مساوات حاصل ہوتے ہیں یعنی \عددی{J=3} کی صورت میں \عددی{J-1=2} آزاد مساوات حاصل ہوتے ہیں۔


مساوات \حوالہ{مساوات_جوڑ_پہلی_مثال_ب} اور مساوات \حوالہ{مساوات_جوڑ_پہلی_مثال_ت} کو ایک ساتھ لکھتے ہیں۔
\begin{gather}
\begin{aligned}\label{مساوات_جوڑ_آزاد_مساوات_دو_جوڑ}
\left( \frac{1}{R_1}+\frac{1}{R_2}\right) v_1 - \frac{v_2}{R_2}&=i_A\\
-\frac{v_1}{R_2}+\left(\frac{1}{R_2}+\frac{1}{R_3}\right)v_2&=-i_B
\end{aligned}
\end{gather}
%================
\ابتدا{مثال}
شکل \حوالہ{شکل_جوڑ_تین_جوڑ} میں \عددی{i_A=\SI{2}{\milli\ampere}}، \عددی{i_B=\SI{5}{\milli\ampere}}، \عددی{R_1=\SI{4}{\kilo\ohm}}، \عددی{R_2=\SI{6}{\kilo\ohm}} اور \عددی{R_3=\SI{2}{\kilo\ohm}} ہیں۔تمام جوڑ پر دباو اور تمام شاخوں میں رو حاصل کریں۔

حل:مساوات \حوالہ{مساوات_جوڑ_آزاد_مساوات_دو_جوڑ} میں قیمتیں پُر کرتے ہیں
\begin{gather}
\begin{aligned}
\left( \frac{1}{4000}+\frac{1}{6000}\right) v_1 - \frac{v_2}{6000}&=0.002\\
-\frac{v_1}{6000}+\left(\frac{1}{6000}+\frac{1}{2000}\right)v_2&=-0.005
\end{aligned}
\end{gather}
ان ہمزاد مساوات کو حل کرنے سے
\begin{align*}
v_1&=\SI{2}{\volt}\\
v_2&=\SI{-7}{\volt}
\end{align*}
حاصل ہوتا ہے۔دباو جوڑ جانتے ہوئے شاخوں کی رو قانون اوہم سے حاصل کرتے ہیں۔
\begin{align*}
i_1&=\frac{v_1}{R_1}=\frac{2}{4000}=\SI{0.5}{\milli\ampere}\\
i_2&=\frac{v_1-v_2}{R_2}=\frac{2-(-7)}{6000}=\SI{1.5}{\milli\ampere}\\
i_3&=\frac{v_2}{R_3}=\frac{-7}{2000}=\SI{-3.5}{\milli\ampere}
\end{align*}
\انتہا{مثال}
%=================


مساوات \حوالہ{مساوات_جوڑ_آزاد_مساوات_دو_جوڑ} کو \اصطلاح{قالبی مساوات}\فرہنگ{قالبی مساوات}\حاشیہب{matrix equation}\فرہنگ{matrix equation} کی صورت میں لکھتے ہیں۔
\begin{equation}\label{مساوات_جوڑ_دو_جوڑ_تشاکل_الف}
\begin{bmatrix}
\frac{1}{R_1}+\frac{1}{R_2} & - \frac{1}{R_2}\\[6pt]
-\frac{1}{R_2}& \frac{1}{R_2}+\frac{1}{R_3}
\end{bmatrix}
\begin{bmatrix}
v_1 \\[6pt]
v_2
\end{bmatrix}
=
\begin{bmatrix}
i_A\\[6pt]
-i_B
\end{bmatrix}
\end{equation}
قالبی مساوات میں
\begin{align*}
{\bf{G}}&=
\begin{bmatrix}
\frac{1}{R_1}+\frac{1}{R_2} & - \frac{1}{R_2}\\[6pt]
-\frac{1}{R_2}& \frac{1}{R_2}+\frac{1}{R_3}
\end{bmatrix}\\
{\bf{V}}&=
\begin{bmatrix}
v_1 \\
v_2
\end{bmatrix}\\
{\bf{I}}&=\begin{bmatrix}
i_A\\
-i_B
\end{bmatrix}
\end{align*}
لیتے ہوئے اسے یوں لکھا جا سکتا ہے
\begin{align*}
{\bf{G V=I}}
\end{align*}
جس سے
\begin{align*}
{\bf{V=G^{-1}I}}
\end{align*}
حاصل ہوتا ہے لہٰذا
\begin{equation}\label{مساوات_جوڑ_قالبی_حل_دو_جوڑ}
\begin{bmatrix}
v_1 \\[6pt]
v_2
\end{bmatrix}
=
\begin{bmatrix}
\frac{1}{R_1}+\frac{1}{R_2} & - \frac{1}{R_2}\\[6pt]
-\frac{1}{R_2}& \frac{1}{R_2}+\frac{1}{R_3}
\end{bmatrix}^{-1}
\begin{bmatrix}
i_A\\[6pt]
-i_B
\end{bmatrix}
\end{equation}
لکھا جائے گا۔

آج کل کمپیوٹر کا زمانہ ہے۔کمپیوٹر کی مدد سے قالبی مساوات نہایت آسانی سے حل کئے جا سکتے ہیں۔آپ سے التماس ہے کہ کمپیوٹر پر قالبی مساوات حل کرنا سیکھیں۔
%============
\ابتدا{مثال}
درج بالا مثال میں تمام دباو جوڑ کو مساوات \حوالہ{مساوات_جوڑ_قالبی_حل_دو_جوڑ} کی مدد سے حل کریں۔

حل:مساوات \حوالہ{مساوات_جوڑ_قالبی_حل_دو_جوڑ} میں دی معلومات پر کرتے ہوئے لکھتے ہیں۔
\begin{equation*}
\begin{bmatrix}
v_1\\[6pt]
v_2
\end{bmatrix}
=
\begin{bmatrix}
\frac{1}{2400} & -\frac{1}{6000}\\[6pt]
-\frac{1}{6000}& \frac{1}{1500}
\end{bmatrix}^{-1}
\begin{bmatrix}
0.002\\[6pt]
-0.005
\end{bmatrix}
\end{equation*}
قالب \عددی{{\bf{G}}} کا ریاضی معکوس \عددی{{\bf{G}}^{-1}} حاصل کرنے کی خاطر \عددی{{\bf{G}}} کا شریک قالب \عددی{{\bf{G}}_{\text{شریک}}} 
\begin{align*}
{\bf{G}}_{\text{شریک}}=
\begin{bmatrix}   
 \frac{1}{1500}& \frac{1}{6000}\\[6pt]
\frac{1}{6000}& \frac{1}{2400}
\end{bmatrix}
\end{align*}
اور قالب کی حتمی قیمت
\begin{align*}
\begin{vmatrix}
\frac{1}{2400} & -\frac{1}{6000}\\[6pt]
-\frac{1}{6000}& \frac{1}{1500}
\end{vmatrix}
&=\left(\frac{1}{2400} \right)\left( \frac{1}{1500}\right) -\left(-\frac{1}{6000} \right) \left(- \frac{1}{6000}\right)\\
&=\frac{1}{4\times 10^{6}}
\end{align*}
درکار ہوں گے۔یوں
\begin{align*}
\begin{bmatrix}
v_1\\
v_2
\end{bmatrix}
&=
4\times 10^{6}
\begin{bmatrix}
\frac{1}{1500}& \frac{1}{6000}\\[6pt]
\frac{1}{6000}& \frac{1}{2400}
\end{bmatrix}
\begin{bmatrix}
0.002\\[6pt]
-0.005
\end{bmatrix}\\
&=
4 \times 10^{6}
\begin{bmatrix}
0.5\times 10^{-6}\\[6pt]
-1.75\times 10^{-6}
\end{bmatrix}\\
&=
\begin{bmatrix}
2 \\
-7
\end{bmatrix}
\end{align*}
حاصل ہوتے ہیں یعنی \عددی{v_1=\SI{2}{\volt}} اور \عددی{v_2=\SI{-7}{\volt}} ہیں۔
\انتہا{مثال}
%==========================

آئیں شکل \حوالہ{شکل_جوڑ_چار_جوڑ_تین_آزاد_مساوات_الف} کے کرخوف قانون رو  کے مساوات لکھیں۔دور کے تمام شاخوں میں رو کی سمتیں چننی گئی ہیں۔نچلے جوڑ کو زمین چننا گیا ہے اور یہی حقیقت زمین کی علامت سے ظاہر کی گئی ہے۔دور میں کل چار \عددی{(J=4)} عدد جوڑ ہیں لہٰذا اس سے تین \عددی{(J-1=3)} عدد آزاد مساوات حاصل کئے جائیں گے۔پہلی جوڑ پر کرخوف قانون رو استعمال کرتے ہوئے
\begin{align*}
i_1+i_2+i_3+i_A=0
\end{align*}
لکھا جائے گا جہاں جوڑ سے خارج رو کو مثبت لکھا گیا ہے۔انفرادی شاخ کی رو کو قانون اوہم سے پُر کرتے ہوئے
\begin{align*}
\frac{v_1}{R_1}+\frac{v_1-v_2}{R_2}+\frac{v_1-v_3}{R_3}+i_A=0
\end{align*}
یعنی
\begin{align}\label{مساوات_جوڑ_چار_جوڑ_تین_آزاد_الف}
\left(\frac{1}{R_1}+\frac{1}{R_2}+\frac{1}{R_3}\right)v_1-\frac{v_2}{R_2}-\frac{v_3}{R_3}=-i_A
\end{align}
حاصل ہوتا ہے۔
\begin{figure}
\centering
\includegraphics{figNodalEquationNotDependentOnChosenCurrentDirectionsA}
\caption{چار جوڑ کے دور سے تین عدد آزاد مساوات حاصل ہوتے ہیں۔}
\label{شکل_جوڑ_چار_جوڑ_تین_آزاد_مساوات_الف}
\end{figure}
دوسرے جوڑ سے
\begin{align*}
-i_2+i_4+i_5=0
\end{align*}
یعنی
\begin{align*}
-\left(\frac{v_1-v_2}{R_2}\right)+\frac{v_2}{R_4}+\frac{v_2-v_3}{R_5}=0
\end{align*}
یا
\begin{align}\label{مساوات_جوڑ_چار_جوڑ_تین_آزاد_ب}
-\frac{v_1}{R_2}+\left(\frac{1}{R_2}+\frac{1}{R_4}+\frac{1}{R_5}\right)v_2-\frac{v_3}{R_5}=0
\end{align}
حاصل ہوتا ہے۔تیسری جوڑ سے
\begin{align*}
-i_3-i_5-i_B=0
\end{align*}
یعنی
\begin{align*}
-\left(\frac{v_1-v_3}{R_3}\right)-\left(\frac{v_2-v_3}{R_5}\right)-i_B=0
\end{align*}
یا
\begin{align}\label{مساوات_جوڑ_چار_جوڑ_تین_آزاد_پ}
-\frac{v_1}{R_3}-\frac{v_2}{R_5}+\left(\frac{1}{R_3}+\frac{1}{R_5}\right)v_3=i_B
\end{align}
حاصل ہوتا ہے۔

مساوات \حوالہ{مساوات_جوڑ_چار_جوڑ_تین_آزاد_الف}، مساوات \حوالہ{مساوات_جوڑ_چار_جوڑ_تین_آزاد_ب} اور مساوات \حوالہ{مساوات_جوڑ_چار_جوڑ_تین_آزاد_پ} کو اکٹھے لکھتے ہوئے
\begin{gather}
\begin{aligned}
\left(\frac{1}{R_1}+\frac{1}{R_2}+\frac{1}{R_3}\right)v_1-\frac{v_2}{R_2}- \frac{v_3}{R_3}=-i_A\\
-\frac{v_1}{R_2}+\left(\frac{1}{R_2}+\frac{1}{R_4}+\frac{1}{R_5}\right)v_2-\frac{v_3}{R_5}=0\\
-\frac{v_1}{R_3}-\frac{v_2}{R_5}+\left(\frac{1}{R_3}+\frac{1}{R_5}\right)v_3=i_B
\end{aligned}
\end{gather}
قالبی مساوات کی صورت میں لکھتے ہیں۔
\begin{align}\label{مساوات_جوڑ_آزاد_مساوات_دو_جوڑ_ب}
\begin{bmatrix}
\frac{1}{R_1}+\frac{1}{R_2}+\frac{1}{R_3} & -\frac{1}{R_2} &- \frac{1}{R_3}\\[6pt]
-\frac{1}{R_2}&\frac{1}{R_2}+\frac{1}{R_4}+\frac{1}{R_5}& -\frac{1}{R_5}\\[6pt]
-\frac{1}{R_3} & -\frac{1}{R_5}&\frac{1}{R_3}+\frac{1}{R_5}
\end{bmatrix}
\begin{bmatrix}
v_1\\[6pt]
v_2\\[6pt]
v_3
\end{bmatrix}
=
\begin{bmatrix}
-i_A\\[6pt]
0\\[6pt]
i_B
\end{bmatrix}
\end{align}

مندرجہ بالا مساوات کا دایاں بازو منبع رو سے جوڑ میں داخل رو دیتی ہے جبکہ اس کا بایاں بازو جوڑ سے خارجی رو دیتی ہے۔

شکل \حوالہ{شکل_جوڑ_چار_جوڑ_تین_آزاد_مساوات_الف} کو دوبارہ شکل \حوالہ{شکل_جوڑ_چار_جوڑ_تین_آزاد_مساوات_ب} میں پیش کیا گیا ہے جہاں \عددی{i_1}، \عددی{i_3} اور \عددی{i_5} کی سمتیں گزشتہ سمتوں کے الٹ چننی گئی ہیں۔
\begin{figure}
\centering
\includegraphics{figNodalEquationNotDependentOnChosenCurrentDirectionsB}
\caption{مزاحمتوں اور آزاد منبع رو کی قالبی مساوات رو کی چننی سمتوں پر منحصر نہیں۔}
\label{شکل_جوڑ_چار_جوڑ_تین_آزاد_مساوات_ب}
\end{figure}
تین جوڑ کے مساوات درج ذیل لکھے جائیں گے۔
\begin{align*}
i_A-i_1+i_2-i_3&=0\\
-i_2+i_4-i_5&=0\\
i_3+i_5-i_B&=0
\end{align*}
شاخوں کی رو قانون اوہم سے پُر کرتے ہوئے درج بالا کو یوں لکھا جا سکتا ہے
\begin{align*}
i_A-\left(\frac{0-v_1}{R_1}\right)+\frac{v_1-v_2}{R_2}-\left(\frac{v_3-v_1}{R_3}\right)&=0\\
-\left(\frac{v_1-v_2}{R_2}\right)+\frac{v_2}{R_4}-\left(\frac{v_3-v_2}{R_5}\right)&=0\\
\frac{v_3-v_1}{R_3}+\frac{v_3-v_2}{R_5}-i_B&=0
\end{align*}
جنہیں ترتیب دینے سے درج ذیل حاصل ہوتے ہیں۔
\begin{align}
\left(\frac{1}{R_1}+\frac{1}{R_2}+\frac{1}{R_3}\right) v_1-\frac{v_2}{R_2}-\frac{v_3}{R_3}&=-i_A \label{مساوات_جوڑ_پہلا_جوڑ}\\
-\frac{v_1}{R_2}+\left(\frac{1}{R_2}+\frac{1}{R_4}+\frac{1}{R_5}\right)v_2-\frac{v_3}{R_5}&=0 \label{مساوات_جوڑ_دوسرا_جوڑ}\\
-\frac{v_1}{R_3}-\frac{v_2}{R_5}+\left(\frac{1}{R_3}+\frac{1}{R_5}\right)v_3&=i_B \label{مساوات_جوڑ_تیسرا_جوڑ}
\end{align}
اس کو قالبی مساوات کی صورت میں لکھتے ہیں۔
\begin{align}\label{مساوات_جوڑ_آزاد_مساوات_دو_جوڑ_پ}
\begin{bmatrix}
\frac{1}{R_1}+\frac{1}{R_2}+\frac{1}{R_3} & -\frac{1}{R_2} &- \frac{1}{R_3}\\[6pt]
-\frac{1}{R_2}&\frac{1}{R_2}+\frac{1}{R_4}+\frac{1}{R_5}& -\frac{1}{R_5}\\[6pt]
-\frac{1}{R_3} & -\frac{1}{R_5}&\frac{1}{R_3}+\frac{1}{R_5}
\end{bmatrix}
\begin{bmatrix}
v_1\\[6pt]
v_2\\[6pt]
v_3
\end{bmatrix}
=
\begin{bmatrix}
-i_A\\[6pt]
0\\[6pt]
i_B
\end{bmatrix}
\end{align}

مساوات \حوالہ{مساوات_جوڑ_آزاد_مساوات_دو_جوڑ_ب} اور مساوات \حوالہ{مساوات_جوڑ_آزاد_مساوات_دو_جوڑ_پ} بالکل یکساں ہیں۔یوں آپ دیکھ سکتے ہیں کہ قالبی مساوات کا دارومدار شاخوں میں رو کی چننی گئی سمتوں پر منحصر نہیں ہوتا۔اس کتاب میں اس حقیقت کو استعمال کرتے ہوئے ہم جوڑ پر کرخوف قانون رو کی مساوات لکھتے ہوئے مزاحمتی شاخوں میں رو کی سمت جوڑ سے خارج ہوتی تصور کریں گے۔آئیں اس ترکیب کو شکل \حوالہ{شکل_جوڑ_شاخوں_کی_رو_خارجی} کی مدد سے سمجھیں۔
\begin{figure}
\centering
\begin{subfigure}{\textwidth}
\centering
\includegraphics{figNodalAssumeCurrentsLeavingEveryNodeA}
\caption*{(الف)}
\end{subfigure}
\begin{subfigure}{\textwidth}
\centering
\includegraphics{figNodalAssumeCurrentsLeavingEveryNodeB}
\caption*{(ب)}
\end{subfigure}
\caption{تمام جوڑ پر مزاحمتی شاخوں میں رو کی سمت جوڑ سے خارج ہوتی تصور کر سکتے ہیں۔}
\label{شکل_جوڑ_شاخوں_کی_رو_خارجی}
\end{figure}

شکل \حوالہ{شکل_جوڑ_شاخوں_کی_رو_خارجی}-الف میں پہلے جوڑ پر تمام مزاحمتی شاخوں کی رو خارجی تصور کرتے ہوئے کرخوف قانون رو کے تحت خارجی رو کا مجموعہ داخلی رو کے مجموعے کے برابر پُر کرنے سے
\begin{align}\label{مساوات_جوڑ_رو_خارجی_الف}
i_1+i_2=i_A
\end{align}
یعنی
\begin{align}\label{مساوات_جوڑ_رو_خارجی_ب}
\frac{v_1}{R_a}+\frac{v_a-v_b}{R_d}=i_A
\end{align}
حاصل ہوتا ہے۔شکل \حوالہ{شکل_جوڑ_شاخوں_کی_رو_خارجی}-ب میں دوسرے جوڑ پر تمام مزاحمتی رو کی سمت خارجی تصور کی گئی ہیں یوں
\begin{align}\label{مساوات_جوڑ_رو_خارجی_پ}
i'_1+i'_2+i'_3=0
\end{align}
یعنی
\begin{align}\label{مساوات_جوڑ_رو_خارجی_ت}
\frac{v_2-v_1}{R_d}+\frac{v_2}{R_b}+\frac{v_2-v_3}{R_e}=0
\end{align}
لکھا جا سکتا ہے۔تیسرے جوڑ پر یہی ترکیب استعمال کرتے ہیں۔ہر جوڑ پر رو کی سمت شکل پر دکھانا ضروری نہیں ہے لہٰذا تیسرے جوڑ پر \عددی{i''_1} اور \عددی{i''_2} دکھانا ضروری نہیں ہے۔ساتھ ہی ساتھ ہر مرتبہ مساوات \حوالہ{مساوات_جوڑ_رو_خارجی_الف} اور مساوات \حوالہ{مساوات_جوڑ_رو_خارجی_پ} کے طرز پر مساوات لکھنے کی بھی ضرورت نہیں ہے بلکہ دل ہی دل میں جوڑ پر تمام مزاحمتی شاخوں کی رو خارجی تصور کرتے ہوئے سیدھ و سیدھ  مساوات \حوالہ{مساوات_جوڑ_رو_خارجی_ب} اور مساوات \حوالہ{مساوات_جوڑ_رو_خارجی_ت} کے طرز پر مساوات لکھے جا سکتے ہیں۔تیسرے جوڑ پر ایسا ہی کرتے ہوئے درج ذیل مساوات لکھی جا سکتی ہے۔
\begin{align}\label{مساوات_جوڑ_رو_خارجی_ٹ}
\frac{v_3-v_2}{R_e}+\frac{v_3}{R_c}+i_B=0
\end{align}

اس کتاب میں ہم مساوات \حوالہ{مساوات_جوڑ_رو_خارجی_ٹ} کی طرح جوڑ پر کرخوف قانون رو کے مساوات لکھیں گے۔

مساوات \حوالہ{مساوات_جوڑ_آزاد_مساوات_دو_جوڑ_پ} اور مساوات \حوالہ{مساوات_جوڑ_آزاد_مساوات_دو_جوڑ_ب} میں \اصطلاح{قالبِ موصلیت}\فرہنگ{قالب!موصلیت}\فرہنگ{موصلیت!قالب}\حاشیہب{conductance matrix}\فرہنگ{matrix!conductance}  \عددی{{\bf{G}}} کے بالائی بائیں کونے سے نچلے دائیں کونے تک ترچھی لکیر کے بالائی اور نچلی اطراف پر یکساں رکن پائے جاتے ہیں۔ایسا اتفاقی طور پر نہیں ہے  بلکہ مزاحمتوں اور آزاد منبع رو پر مبنی کسی بھی دور کے \عددی{{\bf{G}}} قالب کو تشاکل صورت میں لکھا جا سکتا ہے۔آئیں ان قالبوں پر مزید غور کریں۔

شکل \حوالہ{شکل_جوڑ_چار_جوڑ_تین_آزاد_مساوات_ب} میں پہلے جوڑ کی دباو \عددی{v_1}، دوسرے  جوڑ کی دباو \عددی{v_2} اور تیسرے جوڑ کی دباو \عددی{v_3} ہے۔قالب میں بالائی یعنی پہلے صف کے رکن مساوات \حوالہ{مساوات_جوڑ_پہلا_جوڑ} سے حاصل کئے گئے۔یہ مساوات پہلی جوڑ سے حاصل کی گئی ہے۔اس جوڑ پر مزاحمت \عددی{R_1}، \عددی{R_2} اور \عددی{R_3} جڑے ہیں۔ان مزاحمتوں کو متوازی جڑا تصور کرتے ہوئے مساوی مزاحمت \عددی{R_{m1}}
\begin{align*}
\frac{1}{R_{m1}}=\frac{1}{R_1}+\frac{1}{R_2}+\frac{1}{R_3}
\end{align*}
سے حاصل کیا جا سکتا ہے جہاں \عددی{\tfrac{1}{R_{m1}}} کو مساوی متوازی موصلیت \عددی{G_{m1}} کہا جاتا ہے۔یوں  قالب کے پہلے صف کا پہلا (بایاں) رکن  پہلے جوڑ سے  جڑے تمام مزاحمتوں کا مساوی متوازی موصلیت \عددی{\tfrac{1}{R_{m1}}}  ہے۔اسی صف کا دوسرا رکن  پہلے جوڑ اور دوسرے جوڑ کے مابین جڑے مزاحمت کی موصلیت کا منفی \عددی{-\tfrac{1}{R_2}} کے برابر ہے۔اسی طرح پہلے صف کا تیسرا رکن، پہلے جوڑ اور تیسرے جوڑ کے مابین جڑے موصلیت کے منفی \عددی{-\tfrac{1}{R_3}} کے برابر ہے۔قالب کے دوسرے صف کے ارکان مساوات \حوالہ{مساوات_جوڑ_دوسرا_جوڑ} سے حاصل کئے گئے۔اس صف کا پہلا رکن پہلے اور دوسرے جوڑ کے مابین مساوی متوازی موصلیت کے منفی \عددی{-\tfrac{1}{R_2}} کے برابر ہے۔صف کا دوسرا رکن دوسرے جوڑ پر تمام مزاحمتوں کا مساوی متوازی موصلیت \عددی{\tfrac{1}{R_{m2}}}
\begin{align*}
\frac{1}{R_{m2}}=\frac{1}{R_2}+\frac{1}{R_4}+\frac{1}{R_5}
\end{align*}
ہے جبکہ صف کا تیسرا رکن دوسرے اور تیسرے جوڑ کے مابین موصلیت کے منفی \عددی{-\tfrac{1}{R_3}} کے برابر ہے۔قالب کا تیسرا صف بھی اسی طرح حاصل کیا جا سکتا ہے۔قالبی مساوات میں دائیں ہاتھ \اصطلاح{قالبِ رو}\فرہنگ{رو!قالب}\فرہنگ{قالب!رو}\حاشیہب{current matrix}\فرہنگ{matrix!current} کے ارکان بالترتیب پہلے، دوسرے اور تیسرے جوڑ پر جڑے منبع رو سے جوڑ میں داخل ہوتی رو ہے۔منبع رو کی غیر موجودگی میں قالب کے رکن کو صفر لکھا جاتا ہے۔کسی بھی جوڑ پر ایک سے زیادہ منبع رو کی صورت میں جوڑ پر مجموعی داخلی رو، قالب کی رکن ہو گی۔پہلی جوڑ پر منبع کی رو \عددی{i_A} ہے جو جوڑ سے خارجی جانب ہے لہٰذا اسے قالب رو میں \عددی{-i_A} لکھا گیا ہے۔دوسرے جوڑ پر کوئی منبع رو نسب نہیں لہٰذا قالب کا دوسرا رکن صفر ہے۔تیسرے جوڑ پر منبع \عددی{i_B} کی رو جوڑ میں داخل ہوتی ہے لہٰذا قالب رو کا تیسرا رکن \عددی{i_B} ہے۔ 

ان معلومات کی مدد سے  مزاحمت اور منبع رو پر مبنی \عددی{J+1} جوڑ کے دور کی قالبی مساوات  دور کو دیکھ کر درج ذیل صورت میں لکھی جا سکتی ہے
\begin{align}\label{مساوات_جوڑ_عمومی_قالبی_مساوات_الف}
\begin{bmatrix}
+G_{11} & -G_{12} & -G_{13} & \cdots & -G_{1J}\\
-G_{21} & +G_{22} & -G_{23} & \cdots & -G_{2J}\\
-G_{31} & -G_{32} & +G_{33} & \cdots & -G_{3J}\\
\vdots\\
-G_{J1} & -G_{J2} & -G_{J3} & \cdots & +G_{JJ}
\end{bmatrix}
\begin{bmatrix}
v_1\\
v_2\\
v_3\\
\vdots\\
v_J
\end{bmatrix}
=
\begin{bmatrix}
I_1\\
I_2\\
I_3\\
\vdots\\
I_J
\end{bmatrix}
\end{align} 
جہاں \عددی{G_{nn}} سے مراد جوڑ \عددی{n} کے ساتھ منسلک تمام مزاحمتوں کی مساوی متوازی موصلیت جبکہ \عددی{G_{nm}} سے مراد جوڑ \عددی{n} اور \عددی{m} کے مابین مزاحمت کی موصلیت ہے۔یہ مساوات لکھتے ہوئے جوڑ \عددی{J+1} کو زمین چننا گیا ہے۔اگر جوڑ \عددی{n} اور جوڑ \عددی{m} کے مابین مزاحمت \عددی{R_{nm}} جڑی ہو تب جوڑ \عددی{m} اور جوڑ \عددی{n} کے مابین بھی یہی مزاحمت جڑی ہو گی لہٰذا آپ دیکھ سکتے ہیں کہ 
\begin{align}
G_{nm}=G_{mn}
\end{align}
ہو گا اور یوں مساوات \حوالہ{مساوات_جوڑ_عمومی_قالبی_مساوات_الف} کو درج ذیل صورت میں لکھا جا سکتا ہے
\begin{align}\label{مساوات_جوڑ_عمومی_قالبی_مساوات_ب}
\begin{bmatrix}
+G_{11} & -G_{12} & -G_{13} & \cdots & -G_{1J}\\
-G_{12} & +G_{22} & -G_{23} & \cdots & -G_{2J}\\
-G_{13} & -G_{23} & +G_{33} & \cdots & -G_{3J}\\
\vdots\\
-G_{1J} & -G_{2J} & -G_{3J} & \cdots & +G_{JJ}
\end{bmatrix}
\begin{bmatrix}
v_1\\
v_2\\
v_3\\
\vdots\\
v_J
\end{bmatrix}
=
\begin{bmatrix}
I_1\\
I_2\\
I_3\\
\vdots\\
I_J
\end{bmatrix}
\end{align}
جس میں \عددی{{\bf{G}}} کا قالب تشاکل ہے۔

\FloatBarrier
%==========================
\ابتدا{مشق}\شناخت{مشق_جوڑ_تیسرا_الف}
شکل \حوالہ{شکل_جوڑ_تیسرا_الف} میں \عددی{v_1} اور \عددی{v_2} پر کرخوف قانون رو کے مساوات لکھتے ہوئے دور کی قالبی مساوات حاصل کریں۔قالبی مساوات حل کرتے ہوئے نا معلوم دباو دریافت کریں۔
\begin{figure}[!h]
\centering
\includegraphics{figNodalQuizThree}
\caption{مشق \حوالہ{مشق_جوڑ_تیسرا_الف} کا دور۔}
\label{شکل_جوڑ_تیسرا_الف}
\end{figure}

جوابات:\عددی{v_1=\SI{1}{\volt}}، \عددی{v_2=\SI{-20}{\volt}}
\انتہا{مشق}

%==========================
\ابتدا{مشق}\شناخت{مشق_جوڑ_تیسرا_ب}
شکل \حوالہ{شکل_جوڑ_تیسرا_ب} کی قالبی مساوات لکھتے ہوئے نا معلوم دباو حاصل کریں۔
\begin{figure}[!h]
\centering
\includegraphics{figNodalQuizFour}
\caption{مشق \حوالہ{مشق_جوڑ_تیسرا_ب} کا دور۔}
\label{شکل_جوڑ_تیسرا_ب}
\end{figure}

جوابات:\عددی{v_1=\SI{-6}{\volt}}، \عددی{v_2=\SI{-2}{\volt}}،\عددی{v_3=\SI{4}{\volt}}
\انتہا{مشق}
%=========================

%==========================
\ابتدا{مشق}\شناخت{مشق_جوڑ_تیسرا_پ}
شکل \حوالہ{شکل_جوڑ_تیسرا_پ} کی قالبی مساوات لکھتے ہوئے نا معلوم دباو حاصل کریں۔
\begin{figure}[!h]
\centering
\includegraphics{figNodalQuizFive}
\caption{مشق \حوالہ{مشق_جوڑ_تیسرا_پ} کا دور۔}
\label{شکل_جوڑ_تیسرا_پ}
\end{figure}

جوابات:\عددی{v_1=\SI{13.5}{\volt}}، \عددی{v_2=\SI{14}{\volt}}،\عددی{v_3=\SI{22}{\volt}}
\انتہا{مشق}
%=========================
\FloatBarrier

\حصہ{تابع منبع رو استعمال کرنے والے ادوار}
گزشتہ حصے میں ہم نے دیکھا کہ غیر تابع منبع رو اور مزاحمتوں کے ادوار سے تشاکل قالب موصلیت حاصل ہوتے ہے۔شکل \حوالہ{شکل_جوڑ_تابع_منبع_رو_غیر_تشاکل_قالب} میں تباع منبع رو استعمال کی گئی ہے۔ہم دیکھیں گے کہ اس کا \عددی{{\bf{G}}} قالب غیر تشاکل ہو گا۔اس دور کے تین جوڑوں سے درج ذیل لکھا جا سکتا ہے
\begin{gather}
\begin{aligned}\label{مساوات_جوڑ_تابع_منبع_رو_غیر_تشاکل_قالب_الف}
-\beta i_0+\frac{v_1}{R_1}+\frac{v_1-v_2}{R_2}&=0\\
\frac{v_2-v_1}{R_2}-i_A+\frac{v_2-v_3}{R_4}&=0\\
\frac{v_3}{R_3}+\beta i_0+\frac{v_3-v_2}{R_4}&=0
\end{aligned}
\end{gather}
جہاں
\begin{align}\label{مساوات_جوڑ_تابع_منبع_رو_غیر_تشاکل_قالب_ب}
i_0=\frac{v_1}{R_1}
\end{align}
کے برابر ہے۔مساوات \حوالہ{مساوات_جوڑ_تابع_منبع_رو_غیر_تشاکل_قالب_الف} میں مساوات \حوالہ{مساوات_جوڑ_تابع_منبع_رو_غیر_تشاکل_قالب_ب} پُر کرتے اور ترتیب دیتے ہوئے درج ذیل حاصل ہوتا ہے
\begin{gather}
\begin{aligned}\label{مساوات_جوڑ_تابع_منبع_رو_غیر_تشاکل_قالب_پ}
\left(\frac{1}{R_1}+\frac{1}{R_2}-\frac{\beta}{R_1}\right) v_1-\frac{v_2}{R_2}&=0\\
-\frac{v_1}{R_2}+\left(\frac{1}{R_2}+\frac{1}{R_4}\right)v_2-\frac{v_3}{R_4}&=i_A\\
\frac{\beta}{R_1} v_1 -\frac{v_2}{R_4}+\left(\frac{1}{R_3}+\frac{1}{R_4}\right)v_3&=0
\end{aligned}
\end{gather}
جسے قالبی صورت میں لکھتے ہیں۔
\begin{align}\label{مساوات_جوڑ_تابع_منبع_مثال_الف}
\begin{bmatrix}
\frac{1}{R_1}+\frac{1}{R_2}-\frac{\beta}{R_1}&-\frac{1}{R_2}&0\\
-\frac{1}{R_2}&\frac{1}{R_2}+\frac{1}{R_4}&-\frac{1}{R_4}\\
\frac{\beta}{R_1}& -\frac{1}{R_4}&\frac{1}{R_3}+\frac{1}{R_4}
\end{bmatrix}
\begin{bmatrix}
v_1\\
v_2\\
v_3
\end{bmatrix}
=
\begin{bmatrix}
0\\
i_A\\
0
\end{bmatrix}
\end{align}
آپ دیکھ سکتے ہیں کہ \عددی{{\bf{G}}} قالب غیر تشاکل ہے۔
\begin{figure}
\centering
\includegraphics{figNodalDependantCurrentSourceDestroysSymmetry}
\caption{تابع منبع رو سے غیر تشاکل قالب موصلیت حاصل ہوتا ہے۔}
\label{شکل_جوڑ_تابع_منبع_رو_غیر_تشاکل_قالب}
\end{figure}
%=======================
\ابتدا{مثال}
شکل \حوالہ{شکل_جوڑ_تابع_منبع_رو_غیر_تشاکل_قالب} میں تمام جوڑ پر برقی دباو حاصل کریں۔معلومات درج ذیل ہیں۔
\begin{align*}
R_1=\SI{2}{\kilo\ohm}, \quad R_2=\SI{4}{\kilo\ohm}, \quad R_3=\SI{1}{\kilo\ohm}, \quad R_4=\SI{2}{\kilo\ohm}, \quad i_A=\SI{10}{\milli\ampere}, \quad \beta=4
\end{align*}

حل:درج بالا معلومات کو مساوات \حوالہ{مساوات_جوڑ_تابع_منبع_مثال_الف} میں پُر کرتے ہیں۔
\begin{align*}
\begin{bmatrix}
\frac{1}{2000}+\frac{1}{4000}-\frac{4}{2000}&-\frac{1}{4000}&0\\
-\frac{1}{4000}&\frac{1}{4000}+\frac{1}{2000}&-\frac{1}{2000}\\
\frac{\beta}{2000}& -\frac{1}{2000}&\frac{1}{1000}+\frac{1}{2000}
\end{bmatrix}
\begin{bmatrix}
v_1\\
v_2\\
v_3
\end{bmatrix}
=
\begin{bmatrix}
0\\
0.01\\
0
\end{bmatrix}
\end{align*}
اس قالبی مساوات کو حل کرتے ہوئے اور یا تینوں ہمزاد مساوات کو کسی بھی طریقے سے حل کرتے ہوئے  درج ذیل حاصل ہوتے ہیں۔
\begin{align*}
v_1&=\SI{-4}{\volt}\\
v_2&=\SI{20}{\volt}\\
v_3&=\SI{12}{\volt}
\end{align*}
\انتہا{مثال}
%====================
\ابتدا{مثال}\شناخت{مثال_جوڑ_تابع_منبع_مزید_مثال}
شکل \حوالہ{شکل_جوڑ_مثال_مزید} میں تمام نا معلوم دباو حاصل کریں۔دیگر معلومات درج ذیل ہیں۔
\begin{align*}
R_1=\SI{4}{\kilo\ohm}, \quad R_2=\SI{8}{\kilo\ohm}, \quad R_3=\SI{12}{\kilo\ohm},\quad R_4=\SI{6}{\kilo\ohm},\quad R_5=\SI{2}{\kilo\ohm}\\
 i_A=\SI{1}{\milli\ampere},\quad \alpha=0.002
\end{align*}
%===
\begin{figure}[!h]
\centering
\includegraphics{figNodalExampleControlledCurrentSourcesA}
\caption{مثال \حوالہ{مثال_جوڑ_تابع_منبع_مزید_مثال} کا دور۔}
\label{شکل_جوڑ_مثال_مزید}
\end{figure}

حل:تمام جوڑ پر خارجی رو تصور کرتے ہوئے مساوات لکھتے ہیں۔
\begin{align*}
\frac{v_1}{R_1}+\frac{v_1-v_2}{R_2}+\frac{v_1-v_3}{R_3}&=i_A\\
\frac{v_2-v_1}{R_2}+\alpha v_x +\frac{v_2-v_3}{R_4}&=0\\
\frac{v_3-v_1}{R_3}+\frac{v_3-v_2}{R_4}+\frac{v_3}{R_5}&=0
\end{align*}
اس میں \عددی{v_x=v_2-v_3} پُر کرتے اور مساوات کے اجزاء کو ترتیب دیتے ہیں۔
\begin{align*}
\left(\frac{1}{R_1}+\frac{1}{R_2}+\frac{1}{R_3}\right)v_1-\frac{v_2}{R_2}-\frac{v_3}{R_3}&=i_A\\
-\frac{v_1}{R_2}+\left(\frac{1}{R_2}+\alpha+\frac{1}{R_4}\right)v_2-(\alpha+\frac{1}{R_4})v_3&=0\\
-\frac{v_1}{R_3}-\frac{v_2}{R_4}+\left(\frac{1}{R_3}+\frac{1}{R_4}+\frac{1}{R_5}\right)v_3&=0
\end{align*}
دی گئی معلومات پُر کرتے ہیں
\begin{align*}
\left(\frac{1}{4000}+\frac{1}{8000}+\frac{1}{12000}\right)v_1-\frac{v_2}{8000}-\frac{v_3}{12000}&=0.001\\
-\frac{v_1}{8000}+\left(\frac{1}{8000}+0.002+\frac{1}{6000}\right)v_2-(0.002+\frac{1}{6000})v_3&=0\\
-\frac{v_1}{12000}-\frac{v_2}{6000}+\left(\frac{1}{12000}+\frac{1}{6000}+\frac{1}{2000}\right)v_3&=0
\end{align*}
تینوں ہمزاد مساواتوں کو \عددی{1000} سے ضرب دیتے ہوئے درج ذیل لکھا جا سکتا ہے۔
\begin{align*}
\frac{11v_1}{24}-\frac{v_2}{8}-\frac{v_3}{12}&=1\\
-\frac{v_1}{8}+\frac{55 v_2}{24}-\frac{13 v_3}{6}&=0\\
-\frac{v_1}{12}-\frac{v_2}{6}+\frac{3v_3}{4}&=0
\end{align*}
انہیں حل کرتے ہوئے درج ذیل حاصل ہوتے ہیں۔
\begin{align*}
v_1&=\SI{2.38}{\volt}\\
v_2&=\SI{0.48}{\volt}\\
v_3&=\SI{0.37}{\volt}
\end{align*}
\انتہا{مثال}
%================

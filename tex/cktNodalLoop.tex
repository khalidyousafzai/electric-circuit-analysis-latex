\باب{جوڑ اور دائری تجزیہ}
گزشتہ باب میں سادہ ترین ادوار کو کرخوف قوانین سے حل کرنا دکھایا گیا۔اس باب میں متعدد جوڑ اور متعدد دائروں والے ادوار کو کرخوف قوانین سے حل کرنا دکھایا جائے گا۔کرخوف قانون رو سے ہر جوڑ پر داخلی اور خارجی رو کے مجموعوں کو برابر پر کرتے ہوئے دور کے تمام جوڑوں پر دباو حاصل کیا جاتا ہے۔اس کے برعکس کرخوف قانون دباو کی مدد سے دور کے ہر دائرے میں دباو کے گھٹاو کے مجموعے کو دائرے میں دباو کے  بڑھاو کے مجموعے کے برابر پر کرتے ہوئے تمام دائروں کی رو حاصل کی جاتی ہے۔عموماً  دور یا تو کرخوف قانون دباو اور یا کرخوف قانون رو سے زیادہ آسانی سے حل ہوتا ہے۔آسان طریقہ چننا اس باب میں سکھایا جائے گا۔

\حصہ{تجزیہ جوڑ}
دور کو \اصطلاح{ترکیب جوڑ}\فرہنگ{جوڑ!ترکیب}\حاشیہب{nodal analysis}\فرہنگ{nodal analysis} سے حل کرتے ہوئے  جوڑ کے دباو کو  نا معلوم متغیرات چننا جاتا ہے۔کسی ایک جوڑ کو حوالہ چنتے ہوئے بقایا جوڑ کے دباو اس جوڑ سے ناپے جاتے ہیں۔یوں جس جوڑ کو حوالہ چننا گیا ہو، اس کی دباو کو صفر وولٹ تصور کیا جاتا ہے اور اس جوڑ کو  \اصطلاح{برقی زمین} کہا جاتا ہے۔عموماً اس جوڑ کو برقی زمین چننا جاتا ہے جس کے ساتھ سب سے زیادہ پرزے جڑے ہوں۔عموماً آلات کو موصل ڈبوں میں بند رکھا جاتا ہے اور عام طور دور کے برقی زمین کو ڈبے کے ساتھ جوڑا جاتا ہے۔ایسی صورت میں ڈبے کی سطح  بھی \عددی{\SI{0}{\volt}} پر ہوتی ہے۔

ہم دباو جوڑ کے متغیرات کو مثبت تصور کریں گے۔حقیقی دباو کی قیمت زمین کی نسبت سے منفی ہونے کی صورت میں تجزیے سے منفی قیمت حاصل ہو گی۔ 

\begin{figure}
\centering
\includegraphics{figNodalCurrentsFromNodeVoltages}
\caption{دباو جوڑ سے بازو کی رو حاصل کی جا سکتی ہے۔}
\label{شکل_جوڑ_دباو__جوڑ_سے_رو_کا_حصول}
\end{figure}

آئیں دباو جوڑ جاننے کی افادیت کو  شکل \حوالہ{شکل_جوڑ_دباو__جوڑ_سے_رو_کا_حصول} کی مدد سے جانیں۔اس دور میں \عددی{a}، \عددی{b}، \عددی{c} اور \عددی{z} جوڑ پائے جاتے ہیں۔ہم نے جوڑ \عددی{z} کو برقی زمین چننا ہے لہٰذا اس کی دباو \عددی{\SI{0}{\volt}} ہے۔بقایا تین جوڑ کی دباو کو شکل میں دکھایا گیا ہے۔برقی زمین کو علامت سے ظاہر کیا گیا ہے۔

بالائی بائیں مزاحمت پر دباو درج ذیل پایا جاتا ہے
\begin{align*}
V_{ab}&=V_a-V_b\\
&=8-5\\
&=\SI{3}{\volt}
\end{align*}
لہٰذا قانون اوہم سے مزاحمت میں رو درج ذیل حاصل کی جاتی ہے۔
\begin{align*}
i_1&=\frac{V_{ab}}{\SI{2}{\kilo\ohm}}\\
&=\frac{3}{2000}\\
&=\SI{1.5}{\milli\ampere}
\end{align*}
اسی طرح بالائی دائیں مزاحمت پر دباو  درج ذیل ہو گا
\begin{align*}
V_{bc}&=V_b-V_c\\
&=5-4\\
&=\SI{1}{\volt}
\end{align*}
جس سے رو
\begin{align*}
i_2&=\frac{V_{bc}}{\SI{1}{\kilo\ohm}}\\
&=\frac{1}{1000}\\
&=\SI{1}{\milli\ampere}
\end{align*}
حاصل ہوتی ہے۔درمیانے مزاحمت پر دباو اور اس کی رو درج ذیل ہیں۔
\begin{align*}
V_{bz}&=V_b-V_z\\
&=5-0\\
&={\SI{5}{\volt}}\\
i_4&=\frac{V_{bz}}{\SI{10}{\kilo\ohm}}\\
&=\frac{5}{10000}\\
&=\SI{0.5}{\milli\ampere}
\end{align*}
چونکہ \عددی{\SI{1}{\kilo\ohm}} اور \عددی{\SI{4}{\kilo\ohm}} سلسلہ وار جڑے ہیں لہٰذا \عددی{\SI{4}{\kilo\ohm}} میں بھی \عددی{\SI{1}{\milli\ampere}} رو پائی جائے گی۔آپ اسی قیمت کو دباو جوڑ سے بھی حاصل کر سکتے ہیں یعنی
\begin{align*}
V_{cz}&=V_c-V_z\\
&=4-0\\
&=\SI{4}{\volt}\\
i_3&=\frac{V_{cz}}{\SI{4}{\kilo\ohm}}\\
&=\frac{4}{4000}\\
&=\SI{1}{\milli\ampere}
\end{align*}

یہاں اتمنان کر لیں کہ تمام جوڑوں پر آمدی رو اور خارجی رو برابر ہوں۔جوڑ \عددی{b} پر آمدی رو \عددی{\SI{1.5}{\milli\ampere}} ہے جو خارجی رو کے مجموعے \عددی{\SI{1}{\milli\ampere}+\SI{0.5}{\milli\ampere}} کے عین برابر ہے۔اسی طرح جوڑ \عددی{c} پر آمدی اور خارجی رو \عددی{\SI{1}{\milli\ampere}} ہیں۔جوڑ \عددی{a} پر کرخوف قانون رو سے منبع دباو کے مثبت سرے سے خارجی رو \عددی{\SI{1.5}{\milli\ampere}} حاصل ہوتی ہے۔

کسی بھی دو جوڑ \عددی{m} اور \عددی{n} کے مابین جڑی مزاحمت \عددی{R_{mn}} کی رو \عددی{i_R} قانون اوہم
\begin{align}\label{مساوات-جوڑ_قانون_اوہم}
i_R=\frac{v_m-v_n}{R_{mn}}
\end{align}
سے حاصل کی جاتی ہے۔

اب جب ہم دباو جوڑ کی افادیت جان چکے ہیں آئیں ترکیب جوڑ پر غور کریں۔اگر دور میں \عددی{J} جوڑ پائے جاتے ہوں تب ہمیں \عددی{J} دباو دریافت کرنے ہوں گے۔کسی ایک جوڑ کو زمین چنتے ہوئے اس کی دباو \عددی{\SI{0}{\volt}} تصور کی جاتی ہے۔یوں بقایا \عددی{J-1} جوڑ کی دباو کو نا معلوم متغیرات تصور کیا جاتا ہے۔ان \عددی{J-1} جوڑ پر کرخوف قانون رو کا اطلاق کرتے ہوئے \عددی{J-1} مساوات لکھے جاتے ہیں۔آپ جانتے ہیں ہیں کہ \عددی{J-1} متغیرات معلوم کرنے کی خاطر \عددی{J-1} ہمزاد مساوات درکار ہیں۔یوں ان \عددی{J-1} ہمزاد مساوات کے حل سے تمام نا معلوم دباو جوڑ حاصل ہوتے ہیں۔کسی بھی جوڑ پر کروخوف کی مساوات لکھتے ہوئے جوڑ سے منسلک تمام بازو کی رو کو مساوات \حوالہ{مساوات-جوڑ_قانون_اوہم} کی طرز پر لکھا جاتا ہے۔یوں مزاحمت جانتے ہوئے، رو کو نا معلوم دباو کی صورت میں لکھا جاتا ہے۔اس طرح کرخوف قانون رو کی مساوات میں صرف نا معلوم دباو بطور متغیرات پائے جائیں گے۔

 یاد رہے کہ برقی دباو دو نقطوں کے مابین ہوتا ہے۔کسی نقطے کی حتمی دباو کوئی معنی نہیں رکھتی۔جوڑ پر کرخوف قانون رو کی مساوات لکھتے ہوئے جوڑ کا دباو زمین کے حوالے سے ناپا جاتا ہے۔ یوں شکل \حوالہ{شکل_جوڑ_دباو__جوڑ_سے_رو_کا_حصول} میں جوڑ \عددی{a} کا دباو جوڑ \عددی{z} کے حوالے سے \عددی{\SI{8}{\volt}} ہے اور جوڑ \عددی{b} کا دباو جوڑ \عددی{z} کے حوالے سے \عددی{\SI{5}{\volt}} ہے۔اس کے برعکس جوڑ \عددی{b} کے حوالے سے جوڑ \عددی{a} کا دباو \عددی{\SI{3}{\volt}} ہے جبکہ جوڑ \عددی{a} کے حوالے سے جوڑ \عددی{c} کا دباو \عددیء{\SI{-4}{\volt}} اور جوڑ \عددی{z} کا دباو \عددیء{\SI{-8}{\volt}} ہے۔ 


آئیں ترکیب جوڑ کو چند مثالوں کی مدد سے سیکھیں۔ہم آسان ترین مثال سے شروع کرتے ہوئے بتدریج مشکل مثال پیش کریں گے۔

\حصہ{غیر تابع منبع استعمال کرنے والے ادوار}
شکل \حوالہ{شکل_جوڑ_تین_جوڑ} میں تین جوڑ والا دور دکھایا گیا ہے جن میں نچلے جوڑ کو زمین چننا گیا ہے۔بقایا دو جوڑ کے نا معلوم برقی دباو کو متغیرات \عددی{v_1} اور \عددی{v_2} ظاہر کرتے ہیں۔ہم تمام شاخوں میں رو کی سمت چنتے ہیں۔یوں \عددی{i_1} کو بالائی بائیں جوڑ سے زمین کی جانب رواں چننا گیا ہے۔اسی طرح \عددی{i_2} کو بالائی بائیں جوڑ سے بالائی دائیں جوڑ کی جانب رواں چننا گیا ہے جبکہ \عددی{i_3} کو بالائی دائیں جوڑ سے زمین کی طرف رواں چننا گیا ہے۔
\begin{figure}
\centering
\includegraphics{figNodalIndependantSourcesNodalEquationsA}
\caption{تین جوڑ والا دور۔}
\label{شکل_جوڑ_تین_جوڑ}
\end{figure}

بالائی بائیں جوڑ پر کرخوف قانون رو کی مساوات لکھتے ہیں۔جوڑ سے خارجی رو کو مثبت اور داخلی رو کو منفی لکھتے ہوئے درج ذیل لکھا جا سکتا ہے۔
\begin{align}\label{مساوات_جوڑ_پہلی_مثال_الف}
i_1-i_A+i_2&=0
\end{align}
قانون اوہم استعمال کرتے ہوئے اسے یوں
\begin{align*}
\frac{v_1}{R_1}-i_A+\frac{v_1-v_2}{R_2}=0
\end{align*}
یا 
\begin{align}\label{مساوات_جوڑ_پہلی_مثال_ب}
\left( \frac{1}{R_1}+\frac{1}{R_2}\right) v_1 - \left(\frac{1}{R_2}\right) v_2=i_A
\end{align}
لکھا جا سکتا ہے۔بالائی دائیں جوڑ کے لئے
\begin{align}\label{مساوات_جوڑ_پہلی_مثال_پ}
-i_2+i_3+i_B&=0
\end{align}
اور
\begin{align*}
-\left(\frac{v_1-v_2}{R_1}\right)+\frac{v_2}{R_3}+i_B&=0
\end{align*}
یعنی
\begin{align}\label{مساوات_جوڑ_پہلی_مثال_ت}
-\left(\frac{1}{R_1}\right)v_1+\left(\frac{1}{R_1}+\frac{1}{R_3}\right)v_2=-i_B
\end{align}
لکھا جائے گا۔نچلے جوڑ یعنی برقی زمین پر  کرخوف قانون رو کی مساوات لکھتے ہیں۔
\begin{align}\label{مساوات_جوڑ_پہلی_مثال_ٹ}
-i_1+i_A-i_3-i_B=0
\end{align}
مساوات \حوالہ{مساوات_جوڑ_پہلی_مثال_الف} اور مساوات \حوالہ{مساوات_جوڑ_پہلی_مثال_پ} کے مجموعے کو منفی ایک سے ضرب دینے سے مساوات \حوالہ{مساوات_جوڑ_پہلی_مثال_ٹ} حاصل ہوتا ہے۔مساوات \حوالہ{مساوات_جوڑ_پہلی_مثال_الف}، مساوات \حوالہ{مساوات_جوڑ_پہلی_مثال_پ} اور  مساوات \حوالہ{مساوات_جوڑ_پہلی_مثال_ٹ} میں کسی بھی دو مساواتوں سے تیسری مساوات حاصل کی جا سکتی ہے۔یوں ان میں صرف دو عدد مساوات آزاد مساوات ہیں جبکہ تیسری مساوات تابع مساوات ہے۔شکل \حوالہ{شکل_جوڑ_تین_جوڑ} کے دور میں کل تین عدد جوڑ ہیں۔آپ نے دیکھا کہ اس دور سے صرف دو عدد آزاد مساوات حاصل ہوتے ہیں یعنی \عددی{J=3} کی صورت میں \عددی{J-1=2} آزاد مساوات حاصل ہوتے ہیں۔


مساوات \حوالہ{مساوات_جوڑ_پہلی_مثال_ب} اور مساوات \حوالہ{مساوات_جوڑ_پہلی_مثال_ت} کو ایک ساتھ لکھتے ہیں۔
\begin{gather}
\begin{aligned}\label{مساوات_جوڑ_آزاد_مساوات_دو_جوڑ}
\left( \frac{1}{R_1}+\frac{1}{R_2}\right) v_1 - \left(\frac{1}{R_2}\right) v_2&=i_A\\
-\left(\frac{1}{R_1}\right)v_1+\left(\frac{1}{R_1}+\frac{1}{R_3}\right)v_2&=-i_B
\end{aligned}
\end{gather}
%================
\ابتدا{مثال}
شکل \حوالہ{شکل_جوڑ_تین_جوڑ} میں \عددی{i_A=\SI{2}{\milli\ampere}}، \عددی{i_B=\SI{5}{\milli\ampere}}، \عددی{R_1=\SI{4}{\kilo\ohm}}، \عددی{R_2=\SI{6}{\kilo\ohm}} اور \عددی{R_3=\SI{2}{\kilo\ohm}} ہیں۔تمام جوڑ پر دباو حاصل کریں۔

حل:مساوات \حوالہ{مساوات_جوڑ_آزاد_مساوات_دو_جوڑ} میں قیمتیں پُر کرتے ہیں
\begin{gather}
\begin{aligned}
\left( \frac{1}{4000}+\frac{1}{6000}\right) v_1 - \frac{v_2}{6000}&=0.002\\
-\frac{v_1}{4000}+\left(\frac{1}{4000}+\frac{1}{2000}\right)v_2&=-0.005
\end{aligned}
\end{gather}
ان ہمزاد مساوات کو حل کرنے سے
\begin{align*}
v_1&=\frac{32}{13} \, \si{\volt}\\
v_2&=-\frac{76}{13} \, \si{\volt}
\end{align*}
حاصل ہوتا ہے۔
\انتہا{مثال}
%=================


مساوات \حوالہ{مساوات_جوڑ_آزاد_مساوات_دو_جوڑ} کو \اصطلاح{قالبی مساوات}\فرہنگ{قالبی مساوات}\حاشیہب{matrix equation}\فرہنگ{matrix equation} کی صورت میں لکھتے ہیں۔
\begin{equation}
\begin{bmatrix}
\frac{1}{R_1}+\frac{1}{R_2} & - \frac{1}{R_2}\\[6pt]
-\frac{1}{R_1}& \frac{1}{R_1}+\frac{1}{R_3}
\end{bmatrix}
\begin{bmatrix}
v_1 \\[6pt]
v_2
\end{bmatrix}
=
\begin{bmatrix}
i_A\\[6pt]
-i_B
\end{bmatrix}
\end{equation}
قالبی مساوات میں
\begin{align*}
{\bf{G}}&=
\begin{bmatrix}
\frac{1}{R_1}+\frac{1}{R_2} & - \frac{1}{R_2}\\[6pt]
-\frac{1}{R_1}& \frac{1}{R_1}+\frac{1}{R_3}
\end{bmatrix}\\
{\bf{V}}&=
\begin{bmatrix}
v_1 \\
v_2
\end{bmatrix}\\
{\bf{I}}&=\begin{bmatrix}
i_A\\
-i_B
\end{bmatrix}
\end{align*}
لیتے ہوئے اسے یوں لکھا جا سکتا ہے
\begin{align*}
{\bf{G V=I}}
\end{align*}
جس سے
\begin{align*}
{\bf{V=G^{-1}I}}
\end{align*}
حاصل ہوتا ہے لہٰذا
\begin{equation}\label{مساوات_جوڑ_قالبی_حل_دو_جوڑ}
\begin{bmatrix}
v_1 \\[6pt]
v_2
\end{bmatrix}
=
\begin{bmatrix}
\frac{1}{R_1}+\frac{1}{R_2} & - \frac{1}{R_2}\\[6pt]
-\frac{1}{R_1}& \frac{1}{R_1}+\frac{1}{R_3}
\end{bmatrix}^{-1}
\begin{bmatrix}
i_A\\[6pt]
-i_B
\end{bmatrix}
\end{equation}
لکھا جائے گا۔
%============
\ابتدا{مثال}
درج بالا مثال کو مساوات \حوالہ{مساوات_جوڑ_قالبی_حل_دو_جوڑ} کی مدد سے حل کریں۔

حل:مساوات \حوالہ{مساوات_جوڑ_قالبی_حل_دو_جوڑ} میں دی معلومات پر کرتے ہوئے لکھتے ہیں۔
\begin{equation*}
\begin{bmatrix}
v_1\\[6pt]
v_2
\end{bmatrix}
=
\begin{bmatrix}
\frac{1}{2400} & -\frac{1}{6000}\\[6pt]
-\frac{1}{4000}& \frac{3}{4000}
\end{bmatrix}^{-1}
\begin{bmatrix}
0.002\\[6pt]
-0.005
\end{bmatrix}
\end{equation*}
قالب \عددی{{\bf{G}}} کا ریاضی معکوس \عددی{{\bf{G}}^{-1}} حاصل کرنے کی خاطر \عددی{{\bf{G}}} کا شریک قالب \عددی{{\bf{G}}_{\text{شریک}}} 
\begin{align*}
{\bf{G}}_{\text{شریک}}=
\begin{bmatrix}   
 \frac{3}{4000}& \frac{1}{6000}\\[6pt]
\frac{1}{4000}& \frac{1}{2400}
\end{bmatrix}
\end{align*}
اور قالب کی حتمی قیمت
\begin{align*}
\begin{vmatrix}
\frac{1}{2400} & -\frac{1}{6000}\\
-\frac{1}{4000}& \frac{3}{4000}
\end{vmatrix}
&=\left(\frac{1}{2400} \right)\left( \frac{3}{4000}\right) -\left(-\frac{1}{6000} \right) \left(- \frac{1}{4000}\right)\\
&=\frac{13}{48\times 10^{6}}
\end{align*}
درکار ہوں گے۔یوں
\begin{align*}
\begin{bmatrix}
v_1\\
v_2
\end{bmatrix}
&=
\frac{48\times 10^{6}}{13} 
\begin{bmatrix}
\frac{3}{4000}& \frac{1}{6000}\\[6pt]
\frac{1}{4000}& \frac{1}{2400}
\end{bmatrix}
\begin{bmatrix}
0.002\\[6pt]
-0.005
\end{bmatrix}\\
&=
\frac{48 \times 10^{-6}}{13} 
\begin{bmatrix}
\frac{2\times 10^{-6}}{3}\\[6pt]
-\frac{19\times 10^{-3}}{12}
\end{bmatrix}\\
&=
\begin{bmatrix}
\frac{32}{13} \\[6pt]
-\frac{76}{13}
\end{bmatrix}
\end{align*}
حاصل ہوتے ہیں یعنی \عددی{v_1=\tfrac{32}{13} \, \si{\volt}} اور \عددی{v_2=-\tfrac{76}{13} \, \si{\volt}} ہیں۔
\انتہا{مثال}
%==========================

\باب{مسئلے}
گزشتہ بابوں میں ہم نے ادوار میں مختلف مقامات پر دباو اور رو حاصل کرنے کے چند ترکیب دیکھے۔ایسا کرتے ہوئے ہم نے چند حقائق کا استعمال کیا جنہیں یہاں دوبارہ پیش کرتے ہیں۔

\حصہ{مساوی دور}
آپ جانتے ہیں کہ سلسلہ وار مزاحمتوں  کی جگہ ان کا مساوی مزاحمت نسب کرتے ہوئے ان کی رو حاصل کی جا سکتی ہے۔اسی طرح متوازی مزاحمتوں کی جگہ ان کا مساوی مزاحمت نسب کرتے ہوئے ان  پر دباو حاصل کیا جا سکتا ہے۔یہ عمل شکل \حوالہ{شکل_مسئلہ_مساوی_ادوار} میں دکھائے گئے ہیں۔اسی طرح سلسلہ وار منبع دباو کا مساوی اور متوازی منبع رو کا مساوی بالترتیب شکل-ج اور شکل-د میں دکھائے گئے ہیں۔یاد رہے کہ دو یا دو سے زیادہ منبع رو کو صرف اور صرف اس صورت سلسلہ وار جوڑا جا سکتا ہے جب تمام کی رو برابر ہو اور تمام  ایک ہی سمت میں ہوں۔ اسی طرح دو یا دو سے زیادہ منبع دباو کو صرف اور صرف اس صورت متوازی جوڑا جا سکتا ہے جب تمام منبع کی دباو برابر اور سمت ایک ہو۔

\begin{figure}
\centering
\begin{subfigure}{0.5\textwidth}
\centering
\begin{tikzpicture}
\draw(0,0) to [resistor,l={$R_1$}]++(0,\y) to [resistor,l={$R_2$}]++(0,\y) to [short,-o]++(\x/4,0);
\draw(0,0) to [short,-o]++(\x/4,0);
\draw[thick,-stealth] (0.3,\y)--++(0.3,0);
\draw(\x,\y/2) to [short,o-]++(-\x/4,0) to [resistor,l_={$R_1+R_2$}]++(0,\y) to [short,-o]++(\x/4,0);
\end{tikzpicture}
\caption{سلسلہ وار مزاحمتوں کا مساوی مزاحمت}
\end{subfigure}%
\begin{subfigure}{0.5\textwidth}
\centering
\begin{tikzpicture}
\draw(0,0) to [resistor,l={$R_1$}]++(0,\y) ;
\draw(\x,0) to [resistor,*-*,l={$R_2$}]++(0,\y);
\draw(0,0) to [short,-o]++(\x+\x/4,0);
\draw(0,\y) to [short,-o]++(\x+\x/4,0);
\draw[thick,-stealth] (\x+0.3,\y/2)--++(0.3,0);
\draw(2*\x,0) to [short,o-]++(-\x/4,0) to [resistor,l_={$\frac{R_1 R_2}{R_1+R_2}$}]++(0,\y) to [short,-o]++(\x/4,0);
\end{tikzpicture}
\caption{متوازی مزاحمتوں کا مساوی مزاحمت۔}
\end{subfigure}
%
\begin{subfigure}{0.5\textwidth}
\centering
\begin{tikzpicture}
\draw(0,0) to [american voltage source,l={$V_1$}]++(0,\y) to [american voltage source,l={$V_2$}]++(0,\y) to [short,-o]++(\x/4,0);
\draw(0,0) to [short,-o]++(\x/4,0);
\draw[thick,-stealth] (0.3,\y)--++(0.3,0);
\draw(\x,\y/2) to [short,o-]++(-\x/4,0) to [american voltage source,l_={$V_1+V_2$}]++(0,\y) to [short,-o]++(\x/4,0);
\end{tikzpicture}
\caption{سلسلہ وار منبع دباو کا مساوی منبع۔}
\end{subfigure}%
\begin{subfigure}{0.5\textwidth}
\centering
\begin{tikzpicture}
\draw(0,0) to [american current source,l={$I_1$}]++(0,\y) ;
\draw(\x,0) to [american current source,*-*,l={$I_2$}]++(0,\y);
\draw(0,0) to [short,-o]++(\x+\x/4,0);
\draw(0,\y) to [short,-o]++(\x+\x/4,0);
\draw[thick,-stealth] (\x+\x/4,\y/2)--++(0.3,0);
\draw(2*\x,0) to [short,o-]++(-\x/4,0) to [american current source,l_={$I_1+I_2$}]++(0,\y) to [short,-o]++(\x/4,0);
\end{tikzpicture}
\caption{متوازی منبع رو کا مساوی منبع۔}
\end{subfigure}
\caption{مساوی ادوار کی مثال۔}
\label{شکل_مسئلہ_مساوی_ادوار}
\end{figure}

\حصہ{مسئلہ خطیّت}
برقی ادوار میں دباو اور رو درکار متغیرات ہیں۔ اس کتاب میں صرف ایسے ادوار پر غور کیا جائے گا جن میں دباو اور رو کا تعلق \اصطلاح{خطی}\فرہنگ{خطی}\حاشیہب{linear}\فرہنگ{linear} ہے۔انہیں خطی ادوار کہا جاتا ہے۔خطی ادوار میں ایک متغیرہ کو \عددی{n} گنا کرنے سے دوسرے متغیرات بھی \عددی{n} گنا ہو جاتے ہیں۔آئیں خطیت کی خاصیت سے دور حل کرنا دیکھیں۔

%==========================
\ابتدا{مثال}\شناخت{مثال_مسئلہ_خطیت_سے_دور_حل}
شکل \حوالہ{شکل_مسئلہ_خطیت_دور_حل} میں \عددی{\SI{60}{\ohm}} پر دباو معلوم کریں۔
\begin{figure}
\centering
\begin{tikzpicture}
\draw(0,0) to [american current source,l={$\SI{5}{\milli\ampere}$}]++(0,\yy)node[above]{$V_2$} to [resistor,i_={$I_3$},l={$\SI{100}{\ohm}$}]++(\xx,0)node[above]{$V_1$} to [resistor,i_={$I_1$},l={$\SI{40}{\ohm}$}]++(\xx,0)node[above]{$V_0$} to [resistor,i={$I_0$},l_={$\SI{60}{\ohm}$}]++(0,-\yy) to [short]++(-2*\xx,0);
\draw(\xx,0) to [resistor,i<_={$I_2$},*-*,l={$\SI{100}{\ohm}$}]++(0,\yy);
\draw(2*\xx+\dx,\yy/2)node[right]{$\begin{aligned}&+\\&V_R\\&-  \end{aligned}$};
\end{tikzpicture}
\caption{مثال \حوالہ{مثال_مسئلہ_خطیت_سے_دور_حل} کا دور۔}
\label{شکل_مسئلہ_خطیت_دور_حل}
\end{figure}

حل:ہم اس دور کو با آسانی قوانین کرخوف سے حل کر سکتے ہیں۔آئیں اس دور کو خطیت کی خاصیت  کی مدد سے حل کریں۔اس ترکیب میں ہم درکار دباو کو \عددی{\SI{1}{\volt}} تصور کرتے ہوئے منبع رو کی قیمت دریافت کریں گے۔اس کے بعد خطیت کو استعمال کرتے ہوئے منبع رو کی اصل قیمت کے مطابقت سے درکار دباو حاصل کی جائے گی۔

یوں \عددی{V_R=\SI{1}{\volt}} تصور کرتے ہوئے 
\begin{align*}
V_0&=\SI{1}{\volt}\\
I_0&=\frac{V_0}{60}=\frac{1}{60} \, \si{\ampere}\\
I_1&=I_0=\frac{1}{60} \,\si{\ampere}
\end{align*}
حاصل ہوتے ہیں۔قانون اوہم استعمال کرتے ہوئے
\begin{align*}
V_1-V_0=40 \times \frac{1}{60}=\frac{2}{3} \, \si{\volt}
\end{align*}
یعنی
\begin{align*}
V_1=1+\frac{2}{3}=\frac{5}{3}\,\si{\volt}
\end{align*}
حاصل ہوتا ہے۔قانون اوہم کا دوبارہ استعمال کرنے سے
\begin{align*}
I_2=\frac{\frac{5}{3}}{100}=\frac{1}{60} \, \si{\ampere}
\end{align*}
ملتا ہے لہٰذا
\begin{align*}
I_3=I_1+I_2=\frac{1}{60} +\frac{1}{60}=\frac{1}{30}\,\si{\ampere}
\end{align*}
ہو گا۔یوں \عددی{V_R=\SI{1}{\volt}} تصور کرتے ہوئے منبع کی رو \عددی{\tfrac{1}{30} \, \si{\ampere}} متوقع ہے۔

اب ہم کہہ سکتے ہیں کہ اگر منبع کی رو \عددی{\tfrac{1}{30} \, \si{\ampere}} ہو تب \عددی{V_R=\SI{1}{\volt}} ہو گا لہٰذا خطیت کے اصول کو استعمال کرتے ہوئے ہم کہہ سکتے ہیں کہ منبع کی رو \عددی{\SI{5}{\milli\ampere}} ہونے کی صورت میں \عددی{V_R} کی قیمت
\begin{align*}
\frac{0.005\times 1}{\frac{1}{30}}=\SI{0.15}{\volt}
\end{align*}
ہو گی۔
\انتہا{مثال}
%=========================
\ابتدا{مشق}\شناخت{مشق_مسئلہ_خطیت_الف}
شکل \حوالہ{شکل_مسئلہ_خطیت_الف} میں \عددی{I_0=\SI{10}{\milli\ampere}} تصور کرتے ہوئے \عددی{I_M} حاصل کریں۔اب \عددی{I_M=\SI{20}{\milli\ampere}} کی صورت میں خطیت کے استعمال سے \عددی{I_0} معلوم کریں۔
\begin{figure}
\centering
\begin{tikzpicture}
\draw(0,0) to [american current source,l={$I_M$}]++(0,\y) to [resistor,l_={$\SI{2}{\kilo\ohm}$}]++(-\x,0) to [resistor,l_={$\SI{6}{\kilo\ohm}$}]++(0,-\y) to [short]++(\x,0);
\draw(0,0) to [short,*-]++(2*\x,0) to [resistor,i<_={$I_0$},l_={$\SI{8}{\kilo\ohm}$}]++(0,\y) to [resistor,l_={$\SI{6}{\kilo\ohm}$}]++(-\x,0) to [resistor,-*,l_={$\SI{4}{\kilo\ohm}$}]++(-\x,0);
\draw(\x,0) to [resistor,*-*,l={$\SI{12}{\kilo\ohm}$}]++(\x,\y);
\draw(\x,0) to [resistor,-*,l={$\SI{10}{\kilo\ohm}$}]++(0,\y);
\end{tikzpicture}
\caption{مشق \حوالہ{مشق_مسئلہ_خطیت_الف} کا دور۔}
\label{شکل_مسئلہ_خطیت_الف}
\end{figure}
\انتہا{مشق}
%=====================
\ابتدا{مشق}\شناخت{مشق_مسئلہ_خطیت_ب}
شکل \حوالہ{شکل_مسئلہ_خطیت_ب} میں \عددی{V_R=\SI{2}{\volt}}  تصور کرتے ہوئے منبع دباو کی قیمت دریافت کریں۔خطیت کے استعمال سے منبع دباو کی اصل قیمت پر \عددی{V_R} دریافت کریں۔ 
\begin{figure}
\centering
\begin{tikzpicture}
\draw(0,0) to [american voltage source,l={$\SI{50}{\volt}$}]++(0,\y) to [resistor,l={$\SI{2}{\kilo\ohm}$}]++(\x,0) to [resistor,l={$\SI{4}{\kilo\ohm}$}]++(\x,0) to [resistor,l={$\SI{8}{\kilo\ohm}$}]++(\x,0) to [resistor,l_={$\SI{2}{\kilo\ohm}$}]++(0,-\y) to [resistor,l={$\SI{4}{\kilo\ohm}$}]++(-\x,0) to [resistor,l={$\SI{2}{\kilo\ohm}$}]++(-\x,0) to [short]++(-\x,0);
\draw(\x,0) to [resistor,*-*,l={$\SI{20}{\kilo\ohm}$}]++(0,\y);
\draw(2*\x,0) to [resistor,*-*,l={$\SI{10}{\kilo\ohm}$}]++(0,\y);
\draw(3*\x+\dx,\y/2)node[right]{$\begin{aligned} &+ \\ &V_R \\ &- \end{aligned}$};
\end{tikzpicture}
\caption{مشق \حوالہ{مشق_مسئلہ_خطیت_ب} کا دور۔}
\label{شکل_مسئلہ_خطیت_ب}
\end{figure}
\انتہا{مشق}
%========================
\حصہ{مسئلہ نفاذ}
متعدد منبع کی صورت میں ہر منبع کا انفرادی اثر دیکھنے کی خاطر شکل \حوالہ{شکل_مسئلہ_منبع_انفرادی_اثر}-الف کو مثال بناتے ہیں۔
\begin{figure}
\centering
\begin{subfigure}{1\textwidth}
\centering
\begin{tikzpicture}
\draw(0,0) to [american voltage source,l={$\SI{4}{\volt}$}]++(0,\yy) to [resistor,l={$\SI{2}{\kilo\ohm}$}]++(\xx,0) to [resistor]++(0,-\yy) to [resistor,l={$\SI{6}{\kilo\ohm}$}]++(-\xx,0);
\draw(\xx+\dx,1/4*\yy)node[right]{$\SI{4}{\kilo\ohm}$};
\draw(\xx,0) to [short,*-]++(\xx,0) to [american voltage source,l_={$\SI{6}{\volt}$}]++(0,\yy) to [resistor,-*,l_={$\SI{8}{\kilo\ohm}$}]++(-\xx,0);
%loop currents
\draw[stealth-]([shift={(-150:\xx/5.5)}]\xx/2,\yy/2) arc (-150:150:\xx/5.5);
\draw(\xx/2,\yy/2)node{$i_1$};
\draw[stealth-]([shift={(-150:\xx/5.5)}]\xx+\xx/2,\yy/2) arc (-150:150:\xx/5.5);
\draw(\xx+\xx/2,\yy/2)node{$i_2$};
\end{tikzpicture}
\caption*{(الف) دو عدد انفرادی منبع کا مجموعی اثر۔}
\end{subfigure}
\begin{subfigure}{0.5\textwidth}
\centering
\begin{tikzpicture}
\draw(0,0) to [american voltage source,l={$\SI{4}{\volt}$}]++(0,\y) to [resistor,l={$\SI{2}{\kilo\ohm}$}]++(\x,0) to [resistor]++(0,-\y) to [resistor,l={$\SI{6}{\kilo\ohm}$}]++(-\x,0);
\draw(\x+\dx,1/4*\y-\dy)node[right]{$\SI{4}{\kilo\ohm}$};
\draw(\x,0) to [short,*-]++(\x,0) to [short]++(0,\y) to [resistor,-*,l_={$\SI{8}{\kilo\ohm}$}]++(-\x,0);
%loop currents
\draw[stealth-]([shift={(-150:\x/5.5)}]\x/2,\y/2) arc (-150:150:\x/5.5);
\draw(\x/2,\y/2)node{$i'_1$};
\draw[stealth-]([shift={(-150:\x/5.5)}]\x+\x/2,\y/2) arc (-150:150:\x/5.5);
\draw(\x+\x/2,\y/2)node{$i'_2$};
\end{tikzpicture}
\caption*{(ب) بائیں منبع کا اثر دیکھتے وقت دائیں منبع کے اثر کو ختم کیا گیا ہے۔}
\end{subfigure}%
\begin{subfigure}{0.5\textwidth}
\centering
\begin{tikzpicture}
\draw(0,0) to [short]++(0,\y) to [resistor,l={$\SI{2}{\kilo\ohm}$}]++(\x,0) to [resistor]++(0,-\y) to [resistor,l={$\SI{6}{\kilo\ohm}$}]++(-\x,0);
\draw(\x+\dx,1/4*\y-\dy)node[right]{$\SI{4}{\kilo\ohm}$};
\draw(\x,0) to [short,*-]++(\x,0) to [american voltage source,l_={$\SI{6}{\volt}$}]++(0,\y) to [resistor,-*,l_={$\SI{8}{\kilo\ohm}$}]++(-\x,0);
%loop currents
\draw[stealth-]([shift={(-150:\xx/5.5)}]\x/2,\y/2) arc (-150:150:\x/5.5);
\draw(\x/2,\y/2)node{$i''_1$};
\draw[stealth-]([shift={(-150:\x/5.5)}]\x+\x/2,\y/2) arc (-150:150:\x/5.5);
\draw(\x+\x/2,\y/2)node{$i''_2$};
\end{tikzpicture}
\caption*{(پ) دائیں منبع کا اثر دیکھتے وقت بائیں منبع کے اثر کو ختم کیا گیا ہے۔}
\end{subfigure}%
\caption{مجموعی اثر انفرادی اثرات کا مجموعہ ہے۔}
\label{شکل_مسئلہ_منبع_انفرادی_اثر}
\end{figure}
دونوں منبع کا مجموعی اثر دیکھنے کی خاطر دونوں منبع کی موجودگی میں اس دور کو حل کرتے ہیں۔دو خانوں کے مساوات لکھتے ہیں۔
\begin{align*}
-4+2000i_1+4000(i_1-i_2)+6000i_1&=0\\
4000(i_2-i_1)+8000i_2+6&=0
\end{align*}
ان کا حل درج ذیل ہے۔
\begin{align*}
i_1&=\frac{3}{16}\, \si{\milli\ampere}\\
i_2&=-\frac{7}{16}\, \si{\milli\ampere}
\end{align*}
انفرادی منبع سے دور میں مختلف مقامات پر نافذ دباو اور رو دریافت کرنے کی خاطر باری باری ایک ایک منبع کے علاوہ بقایا تمام منبع کے اثر کو ختم کرتے ہوئے دور کو حل کیا جاتا ہے۔منبع دباو کا اثر ختم کرنے کی خاطر اس کو قصر دور کیا جاتا ہے جبکہ منبع رو کے اثر کو ختم کرنے کی خاطر اس کو کھلے دور کیا جاتا ہے۔

آئیں انفرادی منبع کی نافذ رو دریافت کریں۔یوں  \عددی{\SI{4}{\volt}} منبع کی نافذ رو حاصل کرتے وقت \عددی{\SI{6}{\volt}} کی منبع کو قصر دور کرتے ہیں۔ایسا کرنے سے شکل \حوالہ{شکل_مسئلہ_منبع_انفرادی_اثر}-ب حاصل ہوتا ہے جس کے مساوات
\begin{align*}
-4+2000i'_1+4000(i'_1-i'_2)+6000i'_1&=0\\
4000(i'_2-i'_1)+8000i'_2&=0
\end{align*}
اور حل درج ذیل ہیں۔
\begin{align*}
i'_1&=\frac{3}{8}\, \si{\milli\ampere}\\
i'_2&=\frac{1}{8}\, \si{\milli\ampere}
\end{align*}
اسی طرح \عددی{\SI{6}{\volt}} منبع کی نافذ رو حاصل کرنے کی خاطر \عددی{\SI{4}{\volt}} منبع کو قصر دور کیا جاتا ہے۔ایسا شکل \حوالہ{شکل_مسئلہ_منبع_انفرادی_اثر}-پ میں دکھایا گیا ہے جس کے مساوات
\begin{align*}
2000i''_1+4000(i''_1-i''_2)+6000i''_1&=0\\
4000(i''_2-i''_1)+8000i''_2+6&=0
\end{align*}
اور حل درج ذیل ہیں۔
\begin{align*}
i''_1&=-\frac{3}{16}\, \si{\milli\ampere}\\
i''_2&=-\frac{9}{16}\, \si{\milli\ampere}
\end{align*}
آپ دیکھ سکتے ہیں کہ انفرادی منبع کی نافذ رو  کا مجموعہ تمام منبع کی مجموعی نافذ رو کے برابر ہے۔
\begin{align*}
i_1&=i'_1+i''_1\\
i_2&=i'_2+i''_2
\end{align*}

اس حقیقت کو \اصطلاح{مسئلہ نفاذ}\فرہنگ{مسئلہ نفاذ}\حاشیہب{superposition}\فرہنگ{superposition} کہا جاتا ہے  جسے درج ذیل طریقے سے بیان کیا جا سکتا ہے۔

\ابتدا{قانون}
مسئلہ نفاذ کے تحت کسی بھی خطی دور، جس میں متعدد غیر تابع منبع دباو اور غیر تابع منبع رو پائے جاتے ہوں، میں  کسی بھی مقام پر نافذ دباو (رو)، تمام منبع کے انفرادی نافذ کردہ قیمتوں  کے مجموعے  کے برابر ہو گا۔
\انتہا{قانون}

آپ دیکھ سکتے ہیں کہ ہر منبع، دور میں یوں دباو اور رو نافذ کرتا ہے جیسے دور میں کوئی دوسرا منبع پایا ہی نا جاتا ہو۔

مسئلہ نفاذ کا عمومی ثبوت پیش کرتے ہیں۔صفحہ \حوالہصفحہ{مساوات_جوڑ_عمومی_مساوات_متعدد_منبع} پر مساوات \حوالہ{مساوات_جوڑ_عمومی_مساوات_متعدد_منبع} متعدد منبع دباو استعمال کرنے والے دور کی عمومی مساوات ہے جسے یہاں دوبارہ پیش کرتے ہیں۔
\begin{align}\label{مساوات_جوڑ_عمومی_مساوات_متعدد_منبع_دوبارہ}
\begin{bmatrix}
R_{11} & -R_{12}& -R_{13}& \cdots -R_{1m}\\
-R_{21} & R_{22}& -R_{23}& \cdots -R_{2m}\\
-R_{31} & -R_{32}& R_{33}& \cdots -R_{3m}\\
\vdots\\
-R_{m1}&-R_{m2}&-R_{m3}&\cdots R_{mm}
\end{bmatrix}
\begin{bmatrix}
i_1\\
i_2\\
i_3\\
\vdots\\
i_m
\end{bmatrix}
=
\begin{bmatrix}
v_{1}\\
v_{2}\\
v_{3}\\
\vdots\\
v_{m}
\end{bmatrix}
\end{align}
اس مساوات میں مزاحمتی قالب کا دارومدار صرف اور صرف مزاحمتوں پر ہے۔دور میں موجود منبع دباو کا اس قالب پر کوئی اثر نہیں ہے۔اس قالبی مساوات \عددی{\bf{R}  \bf{I} = \bf{V}} کا حل \عددی{\bf{I} = \bf{R^{-1}}  \bf{V}} ہے۔ چونکہ مزاحمتی قالب \عددی{\bf{R}} کے اجزاء صرف اور صرف دور کے مزاحمتوں پر مبنی ہے لہٰذا اس کے ریاضی معکوس \عددیء{\bf{R^{-1}}} کے اجزاء بھی صرف مزاحمتوں پر مبنی ہوں گے۔ریاضی معکوس کے قالب کو درج ذیل عمومی شکل میں لکھا جا سکتا ہے۔
\begin{align*}
\bf{R^{-1}}=
\begin{bmatrix}
g_{11} & -g_{12}& -g_{13}& \cdots -g_{1m}\\
-g_{21} & g_{22}& -g_{23}& \cdots -g_{2m}\\
-g_{31} & -g_{32}& g_{33}& \cdots -g_{3m}\\
\vdots\\
-g_{m1}&-g_{m2}&-g_{m3}&\cdots g_{mm}
\end{bmatrix}
\end{align*}
یوں حل درج ذیل ہو گا
\begin{align*}
\begin{bmatrix}
i_1\\
i_2\\
i_3\\
\vdots\\
i_m
\end{bmatrix}
=
\begin{bmatrix}
g_{11} & -g_{12}& -g_{13}& \cdots -g_{1m}\\
-g_{21} & g_{22}& -g_{23}& \cdots -g_{2m}\\
-g_{31} & -g_{32}& g_{33}& \cdots -g_{3m}\\
\vdots\\
-g_{m1}&-g_{m2}&-g_{m3}&\cdots g_{mm}
\end{bmatrix}
\begin{bmatrix}
v_{1}\\
v_{2}\\
v_{3}\\
\vdots\\
v_{m}
\end{bmatrix}
\end{align*}
جس سے \عددی{i_1} لکھتے ہیں۔
\begin{align}\label{مساوات_مسئلہ_عمومی_رو_حل}
i_1=g_{11} v_1-g_{12}v_2-g_{13}v_3 -\cdots -\g_{1m}v_m
\end{align}
اگر \عددی{v_1} کے علاوہ تمام منبع دباو کو قصر دور کیا جائے تب ان کی قیمت \عددی{\SI{0}{\volt}} پُر کرتے ہوئے مساوات \حوالہ{مساوات_مسئلہ_عمومی_رو_حل} سے 
\begin{align*}
i'_1=g_{11} v_1
\end{align*}
حاصل ہوتا ہے۔ یہ صرف اور صرف \عددی{v_1} کی نافذ رو ہے۔اسی طرح \عددی{v_2} کے علاوہ تمام منبع کو قصر دور کرنے سے \عددی{i''_1=-g_{12}v_2}  نافذ ہوتی ہے۔اسی طرح بقایا منبع دباو کی نافذ  رو بھی حاصل کی جا سکتی ہیں۔آپ دیکھ سکتے ہیں کہ تمام  منبع کی انفرادی نافذ رو کا مجموعہ مساوات \حوالہ{مساوات_مسئلہ_عمومی_رو_حل} دیتی ہے۔

مساوات \حوالہ{مساوات_جوڑ_عمومی_مساوات_متعدد_منبع_دوبارہ} ان ادوار کو ظاہر کرتی ہے جن میں صرف منبع دباو پائے جاتے ہوں۔آپ اسی ترکیب کو استعمال کرتے ہوئے منبع رو کے اثرات کو بھی شامل کر سکتے ہیں۔

مسئلہ نفاذ  ان ادوار پر بھی لاگو ہوتا  ہے جن میں تابع منبع  پائے جاتے ہوں البتہ تابع منبع دباو کو قصر دور اور تابع منبع رو کو کھلے دور نہیں کیا جاتا۔ آئیں مسئلہ نفاذ کا استعمال چند مثالوں کی مدد سے سیکھیں۔

%================
\ابتدا{مثال}\شناخت{مثال_مسئلہ_متعدد_منبع_انفرادی_اثر_الف}
شکل \حوالہ{شکل_مسئلہ_مثال_منبع_انفرادی_اثر_الف} میں منبع دباو اور منبع رو کے انفرادی نافذ دباو حاصل کرتے ہوئے کل \عددی{V_0} حاصل کریں۔
\begin{figure}
\centering
\begin{subfigure}{1\textwidth}
\centering
\begin{tikzpicture}
\draw(0,0) to [american voltage source,l={$\SI{10}{\volt}$}]++(0,\y) to [resistor,l={$\SI{1}{\kilo\ohm}$}]++(\x,0) to [american current source,l_={$\SI{5}{\milli\ampere}$}]++(0,-\y) to [short]++(-\x,0);
\draw(\x,0) to [short,*-]++(\x,0) to [resistor,l={$\SI{4}{\kilo\ohm}$}]++(0,\y) to [short,-*]++(-\x,0);
\draw(2*\x+\dx,\y/2)node[right]{$\begin{aligned} &+\\& V_0 \\ &- \end{aligned}$};
\end{tikzpicture}
\caption*{(الف)}
\end{subfigure}
\begin{subfigure}{0.5\textwidth}
\centering
\begin{tikzpicture}
\draw(0,0) to [american voltage source,l={$\SI{10}{\volt}$}]++(0,\y) to [resistor,l={$\SI{1}{\kilo\ohm}$}]++(\x,0) ++(0,-\y) to [short]++(-\x,0);
\draw(\x,0) to [short]++(\x/2,0) to [resistor,l={$\SI{4}{\kilo\ohm}$}]++(0,\y) to [short]++(-\x/2,0);
\draw(1.5*\x+\dx,\y/2)node[right]{$\begin{aligned} &+\\& V_0 \\ &- \end{aligned}$};
\end{tikzpicture}
\caption*{(ب)}
\end{subfigure}%
\begin{subfigure}{0.5\textwidth}
\centering
\begin{tikzpicture}
\draw(0,0) to [short]++(0,\y) to [resistor,l={$\SI{1}{\kilo\ohm}$}]++(\x,0) to [american current source,l_={$\SI{5}{\milli\ampere}$}]++(0,-\y) to [short]++(-\x,0);
\draw(\x,0) to [short,*-]++(\x,0) to [resistor,l={$\SI{4}{\kilo\ohm}$}]++(0,\y) to [short,-*]++(-\x,0);
\draw(2*\x+\dx,\y/2)node[right]{$\begin{aligned} &+\\& V_0 \\ &- \end{aligned}$};
\end{tikzpicture}
\caption*{(پ)}
\end{subfigure}
\caption{مثال \حوالہ{مثال_مسئلہ_متعدد_منبع_انفرادی_اثر_الف} کا دور۔}
\label{شکل_مسئلہ_مثال_منبع_انفرادی_اثر_الف}
\end{figure}
\انتہا{مثال}
%====================
\ابتدا{مثال}\شناخت{مثال_مسئلہ_منبع_دباو_منبع_رو_مجموعی_دباو}
شکل \حوالہ{شکل_مسئلہ_منبع_دباو_منبع_رو_مجموعی} میں منبع دباو اور منبع رو کو باری باری لیتے ہوئے \عددی{\SI{12}{\kilo\ohm}} پر نافذ دباو حاصل کرتے ہوئے دونوں منبع کی موجودگی میں کُل دباو حاصل کریں۔
\begin{figure}
\centering
\begin{tikzpicture}
\draw(0,0) to [american current source,l={$\SI{2}{\milli\ampere}$}]++(0,\y) to [american voltage source,l={$\SI{4}{\volt}$}]++(0,\y) to [short]++(\x,0) to [resistor,l={$\SI{10}{\kilo\ohm}$}]++(0,-\y) to [resistor,l={$\SI{8}{\kilo\ohm}$}]++(0,-\y) to [resistor,l={$\SI{4}{\kilo\ohm}$}]++(-\x,0);
\draw(0,\y) to [resistor,*-*,l={$\SI{1}{\kilo\ohm}$}]++(\x,0);
\draw(\x,0) to [short,*-] ++(\x,0) to [resistor,l={$\SI{12}{\kilo\ohm}$}]++(0,2*\y) to [short,-*]++(-\x,0);
\draw(2*\x+\dx,\y)node[right]{$\begin{aligned} &+ \\ &V_0 \\ &- \end{aligned}$};
\end{tikzpicture}
\caption{مثال \حوالہ{مثال_مسئلہ_منبع_دباو_منبع_رو_مجموعی_دباو} کا دور۔}
\label{شکل_مسئلہ_منبع_دباو_منبع_رو_مجموعی}
\end{figure}

\begin{figure}
\begin{subfigure}{0.5\textwidth}
\centering
\begin{tikzpicture}
\draw(0,0)++(0,\y) to [american voltage source]++(0,\y) to [short]++(\x,0) to [resistor,l={$\SI{10}{\kilo\ohm}$}]++(0,-\y) to [resistor,l={$\SI{8}{\kilo\ohm}$}]++(0,-\y) to [resistor,l={$\SI{4}{\kilo\ohm}$}]++(-\x,0);
\draw(-\dx,\y+3/4*\y)node[left]{$\SI{4}{\volt}$};
\draw(0,\y) to [resistor,-*,l={$\SI{1}{\kilo\ohm}$}]++(\x,0);
\draw(\x,0) to [short,*-] ++(\x,0) to [resistor,l={$\SI{12}{\kilo\ohm}$}]++(0,2*\y) to [short,-*]++(-\x,0);
\draw(2*\x+\dx,\y)node[right]{$\begin{aligned} &+ \\ &V'_0 \\ &- \end{aligned}$};
\end{tikzpicture}
\caption*{(الف)}
\end{subfigure}%
\begin{subfigure}{0.5\textwidth}
\centering
\begin{tikzpicture}
\draw(0,0) to [american voltage source,l={$\SI{4}{\volt}$}]++(0,2*\y) to [short]++(\x,0) to [resistor,l={$\SI{10}{\kilo\ohm}$}]++(0,-2*\y) to [resistor,l={$\SI{1}{\kilo\ohm}$}]++(-\x,0);
\draw(\x,0) to [short,*-]++(\x,0) to [resistor,l={$\SI{8}{\kilo\ohm}$}]++(0,\y) to [resistor,l={$\SI{12}{\kilo\ohm}$}]++(0,\y) to [short,-*]++(-\x,0);
\draw(2*\x+\dx,\y+\y/2)node[right]{$\begin{aligned} &+ \\ &V'_0 \\ &- \end{aligned}$};
\draw(\x-\dx,\y)node[left]{$\begin{aligned} &+ \\  \\ \\ &V'_1 \\ \\ \\ &- \end{aligned}$};
\end{tikzpicture}
\caption*{(ب)}
\end{subfigure}
\begin{subfigure}{0.5\textwidth}
\centering
\begin{tikzpicture}
\draw(0,0) to [american voltage source,l={$\SI{4}{\volt}$}]++(0,2*\y) to [short]++(\x,0) to [resistor,l={$\SI{10}{\kilo\ohm}$}]++(0,-2*\y) to [resistor,l={$\SI{1}{\kilo\ohm}$}]++(-\x,0);
\draw(\x,0) to [short,*-]++(\x,0) to [resistor,l_={$\SI{20}{\kilo\ohm}$}]++(0,2*\y) to [short,-*]++(-\x,0);
\draw(\x-\dx,\y)node[left]{$\begin{aligned} &+ \\  \\ \\ &V'_1 \\ \\ \\ &- \end{aligned}$};
\end{tikzpicture}
\caption*{(پ)}
\end{subfigure}%
\begin{subfigure}{0.5\textwidth}
\centering
\begin{tikzpicture}
\draw(0,0) to [american voltage source,l={$\SI{4}{\volt}$}]++(0,2*\y) to [short]++(\x,0) to [resistor,l={$\frac{20}{3}\,\si{\kilo\ohm}$}]++(0,-2*\y) to [resistor,l={$\SI{1}{\kilo\ohm}$}]++(-\x,0);
\draw(\x-\dx,\y)node[left]{$\begin{aligned} &+ \\  \\ \\ &V'_1 \\ \\ \\ &- \end{aligned}$};
\end{tikzpicture}
\caption*{(ت)}
\end{subfigure}%
\caption{منبع دباو کا حصہ معلوم کرتے ہیں۔ }
\label{شکل_مسئلہ_مثال_منبع_دباو_حصہ}
\end{figure}

حل:شکل \حوالہ{شکل_مسئلہ_مثال_منبع_دباو_حصہ}-الف میں منبع رو کو کھلے دور کیا گیا ہے تا کہ منبع دباو سے پیدا دباو کا حصہ دریافت کریں۔شکل \حوالہ{شکل_مسئلہ_مثال_منبع_دباو_حصہ}-ب میں شکل کو قدر مختلف صورت دی گئی ہے۔چونکہ \عددی{\SI{4}{\kilo\ohm}} کا ایک سرا کہیں نہیں جڑا لہٰذا اس کا بقایا دور پر کوئی اثر نہیں ہو گا اور اسی لئے اس کو شکل-ب میں نہیں دکھایا گیا ہے۔

شکل-ب میں \عددی{\SI{12}{\kilo\ohm}} اور \عددی{\SI{8}{\kilo\ohm}} سلسلہ وار جڑے ہیں لہٰذا ان کا مساوی مزاحمت \عددی{\SI{20}{\kilo\ohm}} ہو گا۔شکل-پ میں ایسا دکھایا گیا ہے۔شکل-پ میں \عددی{\SI{20}{\kilo\ohm}} اور \عددی{\SI{10}{\kilo\ohm}} متوازی جڑے ہیں لہٰذا ان کا مساوی مزاحمت
 \عددی{\tfrac{\SI{20}{\kilo\ohm} \times \SI{10}{\kilo\ohm}}{\SI{20}{\kilo\ohm} +\SI{10}{\kilo\ohm} }=\tfrac{20}{3}\,\si{\kilo\ohm}} ہو گا جسے شکل-ت میں دکھایا گیا ہے جہاں سے تقسیم دباو کے کلیے سے
\begin{align*}
V'_1=4\left(\frac{\frac{20}{3} \, \si{\kilo\ohm}}{\SI{1}{\kilo\ohm}+\frac{20}{3} \, \si{\kilo\ohm}}\right) =\frac{80}{23}\,\si{\volt}
\end{align*}
لکھا جا سکتا ہے۔شکل-ب کو دیکھتے ہوئے تقسیم دباو کے کلیے سے درج ذیل حاصل ہوتا ہے۔
\begin{align*}
V'_0=\frac{80}{23}\left(\frac{\SI{12}{\kilo\ohm}}{\SI{12}{\kilo\ohm}+\SI{8}{\kilo\ohm}}\right)=\frac{48}{23}\, \si{\volt}
\end{align*}
آئیں اب منبع دباو کو قصر دور کرتے ہوئے حل کریں ۔شکل \حوالہ{شکل_مسئلہ_منبع_دباو_قصر_دور_کیا_گیا_ہے}-الف  میں منبع دباو کو قصر دور کیا گیا ہے۔آپ دیکھ سکتے ہیں کہ \عددی{\SI{1}{\kilo\ohm}} اور \عددی{\SI{10}{\kilo\ohm}} متوازی جڑے ہیں لہٰذا ان کی جگہ \عددی{\tfrac{\SI{1}{\kilo\ohm} \times \SI{10}{\kilo\ohm}}{\SI{1}{\kilo\ohm}+\SI{10}{\kilo\ohm}}=\tfrac{10}{11}\,\si{\kilo\ohm}} نسب کیا جا سکتا ہے۔ایسا ہی شکل-ب میں کیا گیا ہے جہاں \عددی{\tfrac{10}{11}\,\si{\kilo\ohm}} اور \عددی{\SI{8}{\kilo\ohm}} سلسلہ وار جڑے ہیں لہٰذا ان کی جگہ شکل-پ میں \عددی{\tfrac{98}{11}\,\si{\kilo\ohm}} نسب کیا گیا ہے۔شکل-ت میں متوازی جڑے \عددی{\tfrac{98}{11}\,\si{\kilo\ohm}} اور \عددی{\SI{12}{\kilo\ohm}} کی جگہ \عددی{\tfrac{588}{115}\,\si{\kilo\ohm}} نسب کیا گیا ہے۔اس شکل سے درج ذیل لکھا جا سکتا ہے۔
\begin{align*}
V''_0=\frac{588}{115} \, \si{\kilo\ohm} \times \SI{2}{\milli\ampere}=\frac{1176}{115}\, \si{\volt}
\end{align*}
یوں دونوں منبع کی موجودگی میں جواب درج ذیل ہو گا۔
\begin{align*}
V_0=V'_0+V''_0=12\frac{36}{115}\,\si{\volt}
\end{align*}

\begin{figure}
\centering
\begin{subfigure}{0.5\textwidth}
\centering
\begin{tikzpicture}
\draw(0,0) to [american current source,l={$\SI{2}{\milli\ampere}$}]++(0,\y) to [short]++(0,\y) to [short]++(\x,0) to [resistor,l={$\SI{10}{\kilo\ohm}$}]++(0,-\y) to [resistor,l={$\SI{8}{\kilo\ohm}$}]++(0,-\y) to [resistor,l={$\SI{4}{\kilo\ohm}$}]++(-\x,0);
\draw(0,\y) to [resistor,*-*,l={$\SI{1}{\kilo\ohm}$}]++(\x,0);
\draw(\x,0) to [short,*-] ++(\x,0) to [resistor,l={$\SI{12}{\kilo\ohm}$}]++(0,2*\y) to [short,-*]++(-\x,0);
\draw(2*\x+\dx,\y)node[right]{$\begin{aligned} &+ \\ &V''_0 \\ &- \end{aligned}$};
\end{tikzpicture}
\caption*{(الف)}
\end{subfigure}%
\begin{subfigure}{0.5\textwidth}
\centering
\begin{tikzpicture}
\draw(0,0) to [american current source,l={$\SI{2}{\milli\ampere}$}]++(0,2*\y) to [short]++(\x,0) to [resistor,l={$\frac{10}{11}\,\si{\kilo\ohm}$}]++(0,-\y) to [resistor,l={$\SI{8}{\kilo\ohm}$}]++(0,-\y) to [resistor,l={$\SI{4}{\kilo\ohm}$}]++(-\x,0);
\draw(\x,0) to [short,*-] ++(\x,0) to [resistor,l={$\SI{12}{\kilo\ohm}$}]++(0,2*\y) to [short,-*]++(-\x,0);
\draw(2*\x+\dx,\y)node[right]{$\begin{aligned} &+ \\ &V''_0 \\ &- \end{aligned}$};
\end{tikzpicture}
\caption*{(ب)}
\end{subfigure}
\begin{subfigure}{0.5\textwidth}
\centering
\begin{tikzpicture}
\draw(0,0) to [american current source,l={$\SI{2}{\milli\ampere}$}]++(0,2*\y) to [short]++(\x,0) to [resistor,l_={$\frac{98}{11}\,\si{\kilo\ohm}$}]++(0,-2*\y)  to [resistor,l={$\SI{4}{\kilo\ohm}$}]++(-\x,0);
\draw(\x,0) to [short,*-] ++(\x,0) to [resistor,l={$\SI{12}{\kilo\ohm}$}]++(0,2*\y) to [short,-*]++(-\x,0);
\draw(2*\x+\dx,\y)node[right]{$\begin{aligned} &+ \\ &V''_0 \\ &- \end{aligned}$};
\end{tikzpicture}
\caption*{(پ)}
\end{subfigure}%
\begin{subfigure}{0.5\textwidth}
\centering
\begin{tikzpicture}
\draw(0,0) to [american current source,l={$\SI{2}{\milli\ampere}$}]++(0,2*\y) to [short]++(\x+\x/2,0) to [resistor,l_={$\frac{588}{115}\,\si{\kilo\ohm}$}]++(0,-2*\y)  to [resistor,l={$\SI{4}{\kilo\ohm}$}]++(-\x-\x/2,0);
\draw(\x+\x/2+\dx,\y)node[right]{$\begin{aligned} &+ \\ &V''_0 \\ &- \end{aligned}$};
\end{tikzpicture}
\caption*{(ت)}
\end{subfigure}%
\caption{منبع دباو کو قصر دور کیا گیا ہے۔}
\label{شکل_مسئلہ_منبع_دباو_قصر_دور_کیا_گیا_ہے}
\end{figure}
\انتہا{مثال}
%===================

مسئلہ نفاذ سے متعدد منبع استعمال کرنے والے ادوار حل کرتے ہوئے ضروری نہیں کہ تمام منبع کے انفرادی نافذ حصوں کو علیحدہ علیحدہ جانا جائے۔یوں بھی ممکن ہے کہ منبع کے گروہ بناتے ہوئے باری باری ایک ایک گروہ کے مجموعی نافذ دباو یا رو دیکھیں جائیں اور آخر میں تمام کا مجموعہ لیا جائے۔مسئلہ نفاذ سے دور میں کسی بھی مقام پر نافذ  دباو یا نافذ رو حاصل کیا جا سکتا ہے البتہ اس مسئلے کا اطلاق طاقت دریافت کرنے کے لئے نہیں کیا جا سکتا۔آپ جانتے ہیں کہ مزاحمت میں طاقت کو \عددی{\tfrac{V^2}{T}} یا \عددی{I^2 R} لکھا جا سکتا ہے جو غیر خطی تعلق ہیں لہٰذا طاقت کو مسئلہ نفاذ کی مدد سے حاصل نہیں کیا جا سکتا۔

%=====================
\ابتدا{مشق}\شناخت{مشق_مسئلہ_متعدد_منبع_باری_باری_الف}
شکل \حوالہ{شکل_مسئلہ_باری_باری_الف} میں باری باری ایک ایک منبع کا نافذ دباو معلوم کرتے ہوئے \عددی{V_0} دریافت کریں۔

\begin{figure}
\centering
\begin{tikzpicture}
\draw(0,0) to [american voltage source,l_={$\SI{6}{\volt}$}]++(0,-\y) to [short]++(2*\x,0) to [resistor,l={$\SI{6}{\kilo\ohm}$}]++(0,\y) to [resistor,l_={$\SI{4}{\kilo\ohm}$}]++(-\x,0) to [resistor,l_={$\SI{2}{\kilo\ohm}$}]++(-\x,0);
\draw(\x,-\y) to [american current source,*-*,l={$\SI{4}{\milli\ampere}$}]++(0,\y);
\draw(2*\x+\dx,-\y/2)node[right]{$\begin{aligned} &+ \\ &V_0 \\ &- \end{aligned}$};
\end{tikzpicture}
\caption{مشق \حوالہ{مشق_مسئلہ_متعدد_منبع_باری_باری_الف} کا دور۔}
\label{شکل_مسئلہ_باری_باری_الف}
\end{figure}
\انتہا{مشق}
%=====================

\ابتدا{مشق}\شناخت{مشق_مسئلہ_متعدد_منبع_باری_باری_ب}
شکل \حوالہ{شکل_مسئلہ_باری_باری_ب} میں مسئلہ نفاذ کی مدد سے \عددی{V_0} دریافت کریں۔

\begin{figure}
\centering
\begin{tikzpicture}
\draw(0,0) to [american voltage source,l={$\SI{12}{\volt}$}]++(0,\y) to [short]++(\x,0) to [american current source,l={$\SI{4}{\milli\ampere}$}]++(\x,0) to [resistor,l={$\SI{2}{\kilo\ohm}$}]++(\x,0) to [resistor,l_={$\SI{10}{\kilo\ohm}$}]++(0,-\y) to [short]++(-3*\x,0);
\draw(\x,\y) to [short,*-]++(0,\y) to [american voltage source,l={$\SI{6}{\volt}$}]++(\x,0) to [resistor,l={$\SI{1}{\kilo\ohm}$}]++(\x,0) to [short,-*]++(0,-\y);
\draw(\x,0) to [resistor,*-*,l={$\SI{4}{\kilo\ohm}$}]++(0,\y);
\draw(2*\x,0) to [resistor,*-*,l={$\SI{8}{\kilo\ohm}$}]++(0,\y);
\draw(3*\x+\dx,\y/2)node[right]{$\begin{aligned} &+ \\ &V_0 \\ &- \end{aligned}$};
\end{tikzpicture}
\caption{مشق \حوالہ{مشق_مسئلہ_متعدد_منبع_باری_باری_ب} کا دور۔}
\label{شکل_مسئلہ_باری_باری_ب}
\end{figure}
\انتہا{مشق}
%=====================


\ابتدا{مشق}\شناخت{مشق_مسئلہ_متعدد_منبع_باری_باری_پ}
شکل \حوالہ{شکل_مسئلہ_باری_باری_پ} کو مسئلہ نفاذ سے حل کرتے ہوئے  \عددی{I_0} دریافت کریں۔

\begin{figure}
\centering
\begin{tikzpicture}
\draw(0,0) to [short]++(3*\x,0) to [resistor,l_={$\SI{1}{\kilo\ohm}$}]++(0,\y) to [american voltage source,l_={$\SI{2}{\volt}$}]++(-\x,0) to [resistor,l_={$\SI{4}{\kilo\ohm}$}]++(-\x,0) to [short]++(-\x,0) to [short,*-]++(0,\y) to [american current source,l={$\SI{6}{\milli\ampere}$}]++(\x,0)  to [resistor,l={$\SI{1}{\kilo\ohm}$}]++(\x,0)to [short]++(\x,0) to [short,-*]++(0,-\y);
\draw(0,\y) to [resistor,-*,l={$\SI{12}{\kilo\ohm}$}]++(3*\x,\y);
\draw(0,\y) to [american current source,l_={$\SI{4}{\milli\ampere}$}]++(0,-\y);
\draw(\x,0) to [resistor,*-*,l={$\SI{2}{\kilo\ohm}$}]++(0,\y);
\draw(2*\x,0) to [resistor,i<_={$I_0$},*-*,l={$\SI{8}{\kilo\ohm}$}]++(0,\y);
\end{tikzpicture}
\caption{مشق \حوالہ{مشق_مسئلہ_متعدد_منبع_باری_باری_پ} کا دور۔}
\label{شکل_مسئلہ_باری_باری_پ}
\end{figure}
\انتہا{مشق}
%=====================
\ابتدا{مشق}\شناخت{مشق_مسئلہ_منبع_کے_گروہ_کی_رو}
شکل \حوالہ{شکل_مسئلہ_نفاذ_منبع_کے_گروہ} میں \عددی{\SI{6}{\volt}} منبع کے اثر کو ختم کرتے ہوئے \عددی{\SI{10}{\volt}} اور \عددی{\SI{3}{\milli\ampere}} منبع کا مجموعی نافذ رو  \عددی{i'} حاصل کریں۔اب اکیلے \عددی{\SI{6}{\volt}} منبع کا اسی مزاحمت میں نافذ رو \عددی{i''} دریافت کریں۔دونوں جوابات سے تینوں منبع سے پیدا مجموعی رو \عددی{i=i'+i''} دریافت کریں۔
\begin{figure}
\centering
\centering
\begin{subfigure}{1\textwidth}
\centering
\begin{tikzpicture}
\draw(0,0) to [american voltage source,l={$\SI{10}{\volt}$}]++(0,\y) to [resistor,l={$\SI{4}{\kilo\ohm}$}]++(\x,0)node[above]{$v_1$} to [resistor,l={$\SI{4}{\kilo\ohm}$}]++(\x,0)node[above]{$v_2$}  to [resistor,i={$i$},l={$\SI{2}{\kilo\ohm}$}]++(\x,0);
\draw(0,0) to [short]++(3*\x,0);
\draw(\x,0) to [resistor,*-*,l={$\SI{3}{\kilo\ohm}$}]++(0,\y);
\draw(2*\x,0) to [american current source,*-*,l={$\SI{3}{\milli\ampere}$}]++(0,\y);
\draw(\x,0) node[ground]{};
\draw(3*\x,0) to [american voltage source,l_={$\SI{6}{\volt}$}]++(0,\y);
\end{tikzpicture}
\caption*{(الف)}
\end{subfigure}
\begin{subfigure}{1\textwidth}
\centering
\begin{tikzpicture}
\draw(0,0) to [american voltage source,l={$\SI{10}{\volt}$}]++(0,\y) to [resistor,l={$\SI{4}{\kilo\ohm}$}]++(\x,0)node[above]{$v_1$} to [resistor,l={$\SI{4}{\kilo\ohm}$}]++(\x,0)node[above]{$v_2$}  to [resistor,i={$i'$},l={$\SI{2}{\kilo\ohm}$}]++(\x,0);
\draw(0,0) to [short]++(3*\x,0);
\draw(\x,0) to [resistor,*-*,l={$\SI{3}{\kilo\ohm}$}]++(0,\y);
\draw(2*\x,0) to [american current source,*-*,l={$\SI{3}{\milli\ampere}$}]++(0,\y);
\draw(\x,0) node[ground]{};
\draw(3*\x,0) to [short]++(0,\y);
\end{tikzpicture}
\caption*{(ب)}
\end{subfigure}
\begin{subfigure}{1\textwidth}
\centering
\begin{tikzpicture}
\draw(0,0) to [short]++(0,\y) to [resistor,l={$\SI{4}{\kilo\ohm}$}]++(\x,0)node[above]{$v_1$} to [resistor,l={$\SI{4}{\kilo\ohm}$}]++(\x,0)node[above]{$v_2$}  to [resistor,i={$i''$},l={$\SI{2}{\kilo\ohm}$}]++(\x,0);
\draw(0,0) to [short]++(3*\x,0);
\draw(\x,0) to [resistor,*-*,l={$\SI{3}{\kilo\ohm}$}]++(0,\y);
\draw(\x,0) node[ground]{};
\draw(3*\x,0) to [american voltage source,l_={$\SI{6}{\volt}$}]++(0,\y);
\end{tikzpicture}
\caption*{(پ)}
\end{subfigure}
\caption{مشق \حوالہ{مشق_مسئلہ_منبع_کے_گروہ_کی_رو} کا دور۔}
\label{شکل_مسئلہ_نفاذ_منبع_کے_گروہ}
\end{figure}

جوابات:شکل \حوالہ{شکل_مسئلہ_نفاذ_منبع_کے_گروہ}-ب سے \عددی{i'=\tfrac{25}{9} \, \si{\milli\ampere}} اور شکل \حوالہ{شکل_مسئلہ_نفاذ_منبع_کے_گروہ}-پ سے \عددی{i''=-\tfrac{7}{9}\, \si{\milli\ampere}} حاصل ہوتا ہے۔یوں شکل-الف میں \عددی{i=\SI{2}{\milli\ampere}} حاصل ہوتا ہے۔
\انتہا{مشق}
%=================
\حصہ{مساوی ادوار}
شکل \حوالہ{شکل_مسئلہ_مساوی_دور_الف} میں دو عدد ادوار نقطہ دار لکیر میں بند دکھائے گئے ہیں۔تصور کریں کہ نقطہ دار لکیر بند ڈبے کو ظاہر کرتی ہے جس کے اندر دیکھنا ممکن نہیں ہے۔ہم بند ڈبے سے باہر نکلتی برقی سروں پر مختلف مزاحمت یا ادوار نسب کرتے ہوئے معلوم کرنا چاہتے ہیں کہ ان کے اندر کیا نسب ہے۔
\begin{figure}
\centering
\begin{subfigure}{0.5\textwidth}
\centering
\begin{tikzpicture}
\draw(0,0) to [short,-o]++(\x+\x/2,0);
\draw(0,0) to [american voltage source,l={$\SI{8}{\volt}$}]++(0,\y) to [resistor,l={$\SI{4}{\kilo\ohm}$}]++(\x,0) to [short,-o]++(\x/2,0);
\draw(\x,0) to [resistor,*-*,l={$\SI{4}{\kilo\ohm}$}]++(0,\y);
\draw[dashed,gray] (-3/4*\x,-\y/4) rectangle (\x+\x/4,\y+\y/2);
\end{tikzpicture}
\caption*{(الف)}
\end{subfigure}%
\begin{subfigure}{0.5\textwidth}
\centering
\begin{tikzpicture}
\draw(0,0) to [short,-o]++(\x+\x/2,0);
\draw(0,0) to [american voltage source,l={$\SI{4}{\volt}$}]++(0,\y) to [resistor,l={$\SI{2}{\kilo\ohm}$}]++(\x,0) to [short,-o]++(\x/2,0);
\draw[dashed,gray] (-3/4*\x,-\y/4) rectangle (\x+\x/4,\y+\y/2);
\end{tikzpicture}
\caption*{(ب)}
\end{subfigure}%
\caption{مساوی ادوار۔}
\label{شکل_مسئلہ_مساوی_دور_الف}
\end{figure}
%========================
\ابتدا{مثال}\شناخت{مثال_مسئلہ_مساوی_دور_الف}
شکل \حوالہ{شکل_مسئلہ_مساوی_دور_الف}-الف اور شکل \حوالہ{شکل_مسئلہ_مساوی_دور_الف}-ب کے  برقی سروں پر \عددی{\SI{8}{\kilo\ohm}} مزاحمت نسب کرتے ہوئے برقی سروں پر دباو اور رو حاصل کریں۔بند ڈبے کو نہیں دکھایا گیا تا کہ شکل صاف ستھری نظر آئے۔

حل:شکل \حوالہ{شکل_مسئلہ_مساوی_دور_ب} میں صورت حال دکھایا گیا ہے جہاں \عددی{v_0} اور \عددی{i_0} مطلوب ہیں۔
\begin{figure}
\centering
\begin{subfigure}{0.5\textwidth}
\centering
\begin{tikzpicture}
\draw(0,0) to [short]++(\x,0) to [short,-o]++(\x/2,0);
\draw(0,0) to [american voltage source,l={$\SI{8}{\volt}$}]++(0,\y) to [resistor,l={$\SI{4}{\kilo\ohm}$}]++(\x,0) to [short,-o,i={$i_0$}]++(\x/2,0);
\draw(\x,0) to [resistor,*-*,l={$\SI{4}{\kilo\ohm}$}]++(0,\y);
\draw(\x+\x/2,0) to [short,o-]++(\x/2,0) to [resistor,l_={$\SI{8}{\kilo\ohm}$}]++(0,\y) to [short,-o]++(-\x/2,0);
\draw(\x+\x/2,\y/2)node{$\begin{aligned} &+ \\ &v_0 \\ &- \end{aligned}$};
%\draw[dashed,gray] (-3/4*\x,-\y/4) rectangle (\x+\x/4,\y+\y/2);
\end{tikzpicture}
\caption*{(الف)}
\end{subfigure}%
\begin{subfigure}{0.5\textwidth}
\centering
\begin{tikzpicture}
\draw(0,0) to [short]++(\x,0) to [short,-o]++(\x/4,0);
\draw(0,0) to [american voltage source,l={$\SI{4}{\volt}$}]++(0,\y) to [resistor,l={$\SI{2}{\kilo\ohm}$}]++(\x,0) to [short,-o,i={$i_0$}]++(\x/4,0);
\draw(\x+\x/4,0) to [short,o-]++(\x/2,0) to [resistor,l_={$\SI{8}{\kilo\ohm}$}]++(0,\y) to [short,-o]++(-\x/2,0);
\draw(\x+\x/4,\y/2)node{$\begin{aligned} &+ \\ &v_0 \\ &- \end{aligned}$};
%\draw[dashed,gray] (-3/4*\x,-\y/4) rectangle (\x+\x/4,\y+\y/2);
\end{tikzpicture}
\caption*{(ب)}
\end{subfigure}%
\caption{مثال \حوالہ{مثال_مسئلہ_مساوی_دور_الف} کے ادوار۔}
\label{شکل_مسئلہ_مساوی_دور_ب}
\end{figure}
شکل \حوالہ{شکل_مسئلہ_مساوی_دور_ب}-الف میں \عددی{\SI{4}{\kilo\ohm} \parallel \SI{8}{\kilo\ohm}=\tfrac{8}{3} \, \si{\kilo\ohm}} لیتے ہوئے تقسیم دباو کے کلیے سے 
\begin{align*}
v_0=8\left(\frac{\frac{8}{3} \, \si{\kilo\ohm}}{\frac{8}{3} \, \si{\kilo\ohm}+\SI{4}{\kilo\ohm}}\right)=\frac{16}{5}\,\si{\volt}
\end{align*}
لکھا جا سکتا ہے اور یوں
\begin{align*}
i_0=\frac{v_0}{\SI{8}{\kilo\ohm}}=\frac{\frac{16}{5}}{8000}=\frac{2}{5}\, \si{\milli\ampere}
\end{align*}
ہو گی۔شکل \حوالہ{شکل_مسئلہ_مساوی_دور_ب}-ب کو دیکھ کر درج ذیل لکھا جا سکتا ہے۔
\begin{align*}
v_0&=\frac{4\times 8000}{4000+8000}=\frac{16}{5}\,\si{\volt}\\
i_0&=\frac{4}{2000+8000}=\frac{2}{5}\, \si{\milli\ampere}
\end{align*}
ہم دیکھتے ہیں کہ شکل-الف اور شکل-ب دونوں سے یکساں جوابات حاصل ہوتے ہیں۔آئیں مزید تجربے کرتے ہوئے دیکھیں کہ بند ڈبوں میں کیا پایا جاتا ہے۔
\انتہا{مثال}
%========================
\ابتدا{مثال}\شناخت{مثال_مسئلہ_مساوی_دور_ب}
شکل \حوالہ{شکل_مسئلہ_مساوی_دور_الف} کے  برقی سروں پر سلسلہ وار جڑے منبع دباو اور مزاحمت نسب کرتے ہوئے شکل \حوالہ{شکل_مسئلہ_مساوی_دور_پ} میں دکھایا گیا ہے۔انہیں حل کریں۔
\begin{figure}
\centering
\begin{subfigure}{1\textwidth}
\centering
\begin{tikzpicture}
\draw(0,0) to [short]++(\x,0) to [short,-o]++(\x/2,0);
\draw(0,0) to [american voltage source,l={$\SI{8}{\volt}$}]++(0,\y) to [resistor,l={$\SI{4}{\kilo\ohm}$}]++(\x,0) to [short,-o,i={$i_0$}]++(\x/2,0);
\draw(\x,0) to [resistor,*-*,l={$\SI{4}{\kilo\ohm}$}]++(0,\y);
\draw(\x+\x/2,0) to [short,o-]++(\x,0) to [american voltage source,l_={$\SI{3}{\volt}$}]++(0,\y) to [resistor,l_={$\SI{6}{\kilo\ohm}$},-o]++(-\x,0);
\draw(\x+\x/2,\y/2)node{$\begin{aligned} &+ \\ &v_0 \\ &- \end{aligned}$};
\draw(\x,0)node[ground]{};
%\draw[dashed,gray] (-3/4*\x,-\y/4) rectangle (\x+\x/4,\y+\y/2);
\end{tikzpicture}
\caption*{(الف)}
\end{subfigure}
\begin{subfigure}{1\textwidth}
\centering
\begin{tikzpicture}
\draw(0,0) to [short,-*]++(\x,0) to [short,-o]++(\x/2,0);
\draw(0,0) to [american voltage source,l={$\SI{4}{\volt}$}]++(0,\y) to [resistor,l={$\SI{2}{\kilo\ohm}$}]++(\x,0) to [short,-o,i={$i_0$}]++(\x/2,0);
\draw(\x+\x/2,0) to [short,o-]++(\x,0) to [american voltage source,l_={$\SI{3}{\volt}$}]++(0,\y) to [resistor,l_={$\SI{6}{\kilo\ohm}$},-o]++(-\x,0);
\draw(\x+\x/2,\y/2)node{$\begin{aligned} &+ \\ &v_0 \\ &- \end{aligned}$};
\draw(\x,0)node[ground]{};
%\draw[dashed,gray] (-3/4*\x,-\y/4) rectangle (\x+\x/4,\y+\y/2);
\end{tikzpicture}
\caption*{(ب)}
\end{subfigure}
\caption{مثال \حوالہ{مثال_مسئلہ_مساوی_دور_ب} کے ادوار۔}
\label{شکل_مسئلہ_مساوی_دور_پ}
\end{figure}

حل:نچلی جوڑ کو زمین چنتے ہوئے بالائی جوڑ پر دباو \عددی{v_0} لکھی جائے گی۔یوں  شکل \حوالہ{شکل_مسئلہ_مساوی_دور_پ}-الف کے بالائی جوڑ پر درج ذیل کرخوف مساوات رو لکھی جا سکتی ہے
\begin{align*}
\frac{v_0-8}{4000}+\frac{v_0}{4000}+\frac{v_0-3}{6000}=0
\end{align*}
جسے حل کرنے سے
\begin{align*}
v_0=\frac{15}{4}\, \si{\volt}
\end{align*}
حاصل ہوتا ہے اور یوں
\begin{align*}
i_0=\frac{v_0-3}{6000}=\frac{\frac{15}{4}-3}{6000}=\frac{1}{8}\,\si{\milli\ampere}
\end{align*}
ہو گی۔آئیں اب شکل  \حوالہ{شکل_مسئلہ_مساوی_دور_پ}-ب کو حل کرتے ہیں۔بالائی جوڑ پر کرخوف مساوات رو
\begin{align*}
\frac{v_0-4}{2000}+\frac{v_0-3}{6000}=0
\end{align*}
سے
\begin{align*}
v_0=\frac{15}{4}\, \si{\volt}
\end{align*}
حاصل ہوتا ہے جبکہ قانون اوہم سے رو درج ذیل لکھی جا سکتی ہے۔
\begin{align*}
i_0=\frac{4-3}{2000+6000}=\frac{1}{8}\,\si{\milli\ampere}
\end{align*}
\انتہا{مثال}
%=============================

مندرجہ بالا دو مثالوں کے تجربات سے گمان ہوتا ہے کہ شکل \حوالہ{شکل_مسئلہ_مساوی_دور_الف} کے دونوں بند ڈبوں میں یکساں ادوار پائے جاتے ہیں۔ دیکھا یہ گیا ہے کہ دونوں بند ڈبوں کے بیرونی برقی سروں پر یکساں دور نسب کرنے سے بالکل یکساں جوابات حاصل ہوتے ہیں۔یہ ایک دلچسپ صورت حال ہے۔ایسی صورت میں ہم کہتے ہیں کہ شکل \حوالہ{شکل_مسئلہ_مساوی_دور_الف}-الف اور شکل \حوالہ{شکل_مسئلہ_مساوی_دور_الف}-ب \اصطلاح{مساوی ادوار}\فرہنگ{مساوی!دور}\حاشیہب{equivalent circuit}\فرہنگ{equivalent circuit} ہیں۔مزید یہ کہ شکل-الف کا دور، خطی ہونے کی صورت میں، جتنا بھی پیچیدہ کیوں نہ ہو، اس کا مساوی دور ایک عدد منبع اور ایک عدد مزاحمت سلسلہ وار جوڑنے سے حاصل کیا جا سکتا ہے۔

مساوی ادوار صرف اور صرف برقی سروں پر یکساں جوابات دیتے ہیں۔اس حقیقت کو سمجھنے کی خاطر شکل \حوالہ{شکل_مسئلہ_مساوی_دور_الف} میں برقی سرے کھلے رکھتے ہوئے دونوں ادوار میں طاقت کا ضیاع دریافت کرتے ہیں۔شکل-الف میں طاقت کا ضیاع
\begin{align*}
\frac{8^2}{4000+4000}=\SI{8}{\milli\watt}
\end{align*}
ہے جبکہ شکل-ب میں طاقت کا ضیاع \عددی{\SI{0}{\watt}} ہے۔مساوی ادوار کے اندرونی متغیرات عموماً یکساں نہیں ہوتے۔

اگلے حصے میں \اصطلاح{تھونن مساوی دور} اور \اصطلاح{نارٹن مساوی دور} پر غور کیا جائے گا۔ان پر غور کرتے ہوئے \اصطلاح{مسئلہ تبادلہ منبع} بھی اخذ کیا جائے گا۔

%=============================
\حصہ{مسئلہ تھونن، مسئلہ نارٹن اور مسئلہ تبادلہ منبع}
شکل \حوالہ{شکل_مسئلہ_تھونن_سمجھنا_دور}-الف  کے تین جوڑ پر کرخوف مساوات رو لکھتے
\begin{align*}
\frac{v_1-10}{4000}+\frac{v_1}{3000}+\frac{v_1-v_2}{4000}&=0\\
\frac{v_2-v_1}{4000}-0.003+\frac{v_2-v_3}{2000}&=0\\
\frac{v_3-v_2}{2000}+\frac{v_3}{6000}+\frac{v_3+2}{8000}&=0
\end{align*}
ہوئے حل کرنے سے درج ذیل حاصل ہوتے ہیں۔
\begin{align*}
v_1&=\SI{6}{\volt}\\
v_2&=\SI{10}{\volt}\\
v_3&=\SI{6}{\volt}
\end{align*}
دباو جوڑ جانتے ہوئے تمام شاخوں کی رو دریافت کی جا سکتی ہے۔آئیں اس دور کو نقطہ دار لکیر پر دو ٹکڑوں میں تقسیم کرتے ہیں۔شکل \حوالہ{شکل_مسئلہ_تھونن_سمجھنا_دور}-ب میں بائیں حصے کو دکھایا گیا ہے جہاں جوڑ \عددی{v_3} پر \عددی{\SI{6}{\volt}} منبع دباو نسب کیا گیا ہے۔ اس کو حل کرنے کی خاطر کرخوف قانون رو سے درج ذیل لکھتے ہیں
\begin{align*}
\frac{v_1-10}{4000}+\frac{v_1}{3000}+\frac{v_1-v_2}{4000}&=0\\
\frac{v_2-v_1}{4000}-0.003+\frac{v_2-6}{2000}&=0
\end{align*}
جنہیں حل کرتے ہوئے ایک بار دوبارہ
\begin{align*}
v_1&=\SI{6}{\volt}\\
v_2&=\SI{10}{\volt}
\end{align*}
حاصل ہوتے ہیں۔آپ نے دیکھا کہ شکل-ب کے  دباو جوڑ بالکل تبدیل نہیں ہوئے لہٰذا اس میں تمام مقامات پر رو بھی وہی ہو گی جو شکل-الف میں تھی۔

شکل \حوالہ{شکل_مسئلہ_تھونن_سمجھنا_دور}-الف میں نقطہ دار لکیر کے بائیں حصے پر لکیر کے دائیں جانب دور کا اثر صرف اور صرف جوڑ \عددی{v_3} کے ذریعہ ہوتا ہے۔یوں جیسا شکل-ب میں کیا گیا، اگر جوڑ \عددی{v_3} پر دباو اسی قیمت پر رکھا جائے جو لکیر کے دائیں جانب دور کے نسب کرنے سے حاصل ہوتا ہے، تب  لکیر کے بائیں جانب دور کے متغیرات جوں کے توں رہتے ہیں۔  

شکل \حوالہ{شکل_مسئلہ_تھونن_سمجھنا_دور}-ب میں رو  \عددی{i} کو  مسئلہ نفاذ سے حاصل کیا جا سکتا ہے۔آپ مشق \حوالہ{مشق_مسئلہ_منبع_کے_گروہ_کی_رو} میں اس دور کو مسئلہ نفاذ کی مدد سے حل کر چکے ہیں۔اسی مشق کے شکل \حوالہ{شکل_مسئلہ_نفاذ_منبع_کے_گروہ}-پ میں بقایا منبع کے اثر کو ختم کرتے ہوئے \عددی{\SI{6}{\volt}} کو صرف مزاحمت نظر آتے ہیں۔آئیں شکل-پ میں دیے دور کا مساوی مزاحمت حاصل کرتے ہیں۔منبع سے دور ترین نقطے سے شروع کرتے ہیں جہاں چار کلو اوہم اور تین کلو اوہم  متوازی \عددی{\SI{4}{\kilo\ohm} \parallel \SI{3}{\kilo\ohm}} جڑے ہیں۔متوازی جڑے مزاحمت اذ خود سلسلہ وار جڑے \عددی{\SI{2}{\kilo\ohm}} اور \عددی{\SI{4}{\kilo\ohm}} کے ساتھ سلسلہ وار پائے جاتے ہیں لہٰذا ان تمام کا مجموعی مساوی مزاحمت
\begin{align*}
R_{\text{تھونن}}=\left(\SI{4}{\kilo\ohm} \parallel \SI{3}{\kilo\ohm}\right)+\left(\SI{2}{\kilo\ohm}+\SI{4}{\kilo\ohm}\right)=\frac{54}{7}\, \si{\kilo\ohm}
\end{align*}
ہو گا جسے \اصطلاح{تھونن مزاحمت}\فرہنگ{تھونن!مزاحمت}\فرہنگ{مزاحمت!تھونن}\حاشیہب{Thevenin Resistance}\فرہنگ{Thevenin!Resistance} کہتے ہیں۔
\begin{figure}
\centering
\begin{subfigure}{1\textwidth}
\centering
\begin{tikzpicture}
\draw(0,0) to [american voltage source,l={$\SI{10}{\volt}$}]++(0,\y) to [resistor,l={$\SI{4}{\kilo\ohm}$}]++(\x,0)node[above]{$v_1$} to [resistor,l={$\SI{4}{\kilo\ohm}$}]++(\x,0)node[above]{$v_2$}  to [resistor,l={$\SI{2}{\kilo\ohm}$}]++(\x,0)node[above]{$v_3$} to [resistor,l={$\SI{8}{\kilo\ohm}$}]++(\x,0) to [american voltage source,l={$\SI{2}{\volt}$}]++(0,-\y) to [short]++(-4*\x,0);
\draw(\x,0) to [resistor,*-*,l={$\SI{3}{\kilo\ohm}$}]++(0,\y);
\draw(2*\x,0) to [american current source,*-*,l={$\SI{3}{\milli\ampere}$}]++(0,\y);
\draw(3*\x,0) to [resistor,*-*,l_={$\SI{6}{\kilo\ohm}$}]++(0,\y);
\draw(\x,0) node[ground]{};
\draw[gray,dashed] (3*\x-\x/6,-\y/4) --++(0,\y+\y/2);
\end{tikzpicture}
\caption*{(الف)}
\label{شکل_مسئلہ_تھونن_سمجھنا_دور_الف}
\end{subfigure}
\begin{subfigure}{1\textwidth}
\centering
\begin{tikzpicture}
\draw(0,0) to [american voltage source,l={$\SI{10}{\volt}$}]++(0,\y) to [resistor,l={$\SI{4}{\kilo\ohm}$}]++(\x,0)node[above]{$v_1$} to [resistor,l={$\SI{4}{\kilo\ohm}$}]++(\x,0)node[above]{$v_2$}  to [resistor,i={$i$},l={$\SI{2}{\kilo\ohm}$}]++(\x,0);
\draw(0,0) to [short]++(3*\x,0);
\draw(\x,0) to [resistor,*-*,l={$\SI{3}{\kilo\ohm}$}]++(0,\y);
\draw(2*\x,0) to [american current source,*-*,l={$\SI{3}{\milli\ampere}$}]++(0,\y);
\draw(\x,0) node[ground]{};
\draw(3*\x,0) to [american voltage source,l_={$\SI{6}{\volt}$}]++(0,\y)node[above]{$v_3$};
\end{tikzpicture}
\caption*{(ب)}
\end{subfigure}
\caption{مسئلہ تھونن سمجھنے کا دور۔}
\label{شکل_مسئلہ_تھونن_سمجھنا_دور}
\end{figure}

آئیں ان حقائق کو سامنے رکھتے ہوئے \اصطلاح{مسئلہ تھونن}\فرہنگ{مسئلہ تھونن}\حاشیہب{Thevenin theorem}\فرہنگ{Thevenin theorem} اور \اصطلاح{مسئلہ نارٹن}\فرہنگ{مسئلہ نارٹن}\حاشیہب{Norton theorem}\فرہنگ{Norton theorem} سیکھیں۔ساتھ ہی ساتھ \اصطلاح{مسئلہ تبادلہ منبع}\فرہنگ{مسئلہ!تبادلہ منبع}\حاشیہب{Source Transformation theorem}\فرہنگ{theorem!Source Transformation} پر بھی غور کیا جائے گا۔مسئلہ تھونن کہتا ہے کہ کسی بھی خطی دور کو سلسلہ وار جڑے ایک عدد منبع اور ایک عدد مزاحمت سے ظاہر کیا جا سکتا ہے۔اس دور کو مساوی تھونن دور کہا جائے گا۔اسی طرح مسئلہ نارٹن کہتا ہے کہ کسی بھی خطی دور کو متوازی جڑے ایک عدد منبع رو اور ایک عدد مزاحمت سے ظاہر  کیا جا سکتا ہے۔اس دور کو مساوی نارٹن دور کہا جائے گا۔

شکل \حوالہ{شکل_مسئلہ_تھونن_عمومی_دور}-الف میں عمومی ڈبہ دور دکھایا گیا ہے۔اس کو دو حصوں میں تقسیم کرتے ہوئے شکل-ب حاصل ہوتا ہے۔شکل-ب میں بائیں حصے کے مساوی تھونن دور اور مساوی نارٹن دور حاصل کیے جائیں گے۔بایاں حصہ خطی ہونا ضروری ہے۔دایاں حصہ خطی یا غیر خطی ہو سکتا ہے۔دائیں حصے کو برقی بوجھ تصور کیا جائے گا۔یہ حصے دو تاروں سے آپس میں جڑے ہیں۔ان تاروں کے مابین \عددی{v_0} دباو پایا جاتا ہے جبکہ بوجھ کو رو \عددی{i} مہیا کی جاتی ہے۔اگر شکل-ب میں بائیں ڈبے دور کی جگہ اس کا مساوی تھونن دور یا مساوی نارٹن دور نسب کرنے سے \عددی{v_0} اور \عددی{i} کی قیمتوں پر فرق نہیں پڑے تب دائیں ڈبے کی نقطہ نظر سے دور میں کوئی تبدیلی رو نما نہیں ہوئی ہے لہٰذا اس کے لئے بایاں ڈبے کا دور اور مساوی تھونن (یا مساوی نارٹن) دور یک برابر ہیں۔

شکل-الف میں تابع منبع کی موجودگی میں ڈبے دور کو اس طرح دو ٹکڑوں میں تقسیم کیا جائے گا کہ تابع منبع اور اسے قابو کرنے والا متغیر ایک ہی ڈبے کا حصہ بنیں۔تابع منبع استعمال کرنے والے ادوار کو حل کرنا اگلے حصے میں سکھایا جائے گا۔

شکل-پ میں دائیں حصے کی جگہ  منبع دباو نسب کیا گیا ہے جس کا دباو \عددی{v_0} ہے۔ 

\begin{figure}
\centering
\begin{subfigure}{1\textwidth}
\centering
\begin{tikzpicture}
\draw  plot [smooth cycle] coordinates {(0,0) (3,0) (3,2.5) (0,2.5)};
\draw(1.5,1.25)node{\RL{عمومی ڈبہ دور}};
\end{tikzpicture}
\caption*{(الف)}
\end{subfigure}
\begin{subfigure}{0.5\textwidth}
\centering
\begin{tikzpicture}
\draw[name path=leftCircuit]  plot [smooth cycle] coordinates {(0,0) (1,0) (1,2.5) (0,2.5)};
\draw(0.5,1.25)node[rotate=90]{\RL{بایاں خطی حصہ}};
\draw[name path=rightCircuit]  plot [smooth cycle] coordinates {(3,0) (4,0) (4,2.5) (3,2.5)};
\draw(3.5,1.25)node[rotate=90]{\RL{دایاں حصہ (بوجھ)}};
\path[name path=lineLower] (0.5,0.25)--++(3.5,0);
\path[name path=lineUpper] (0.5,2.25)--++(3.5,0);
\draw[name intersections={of=leftCircuit and lineLower,by=leftLEnd },name intersections={of=rightCircuit and lineLower,by=rightLEnd }]
(leftLEnd) to [short,-o] ++(1,0)coordinate(ka)node[below]{$B$} to [short](rightLEnd);
\draw[name intersections={of=leftCircuit and lineUpper,by=leftUEnd },name intersections={of=rightCircuit and lineUpper,by=rightUEnd }]
(leftUEnd) to [short,i={$i$},-o]++(1,0)coordinate(kb)node[above]{$A$} to [short](rightUEnd);
\draw($(ka)!0.5!(kb)$)node{$\begin{aligned} &+ \\ & v_0 \\ &- \end{aligned}$};
\end{tikzpicture}
\caption*{(ب)}
\end{subfigure}%
\begin{subfigure}{0.5\textwidth}
\centering
\begin{tikzpicture}
\draw[name path=leftC]  plot [smooth cycle] coordinates {(0,0) (1,0) (1,2.5) (0,2.5)};
\draw(0.5,1.25)node[rotate=90]{\RL{بایاں خطی حصہ}};
\path[name path=lineL] (0.5,0.25)--++(1,0);
\path[name path=lineU] (0.5,2.25)--++(1,0);
\draw[name intersections={of=leftC and lineL,by=leftL},name intersections={of=leftC and lineU,by=leftU}](leftL) to [short,-o] ++(1,0)coordinate(kka)node[below]{$B$} to [short]++(1,0)coordinate(kLowR)  (leftU) to [short,i={$i$},-o] ++(1,0)coordinate(kkb)node[above]{$A$} to [short]++(1,0)coordinate(kUpR);
\draw(kLowR) to [american voltage source,l_={$v_0$}](kUpR);
\end{tikzpicture}
\caption*{(پ)}
\end{subfigure}%
\caption{مسئلہ تھونن کا عمومی دور۔}
\label{شکل_مسئلہ_تھونن_عمومی_دور}
\end{figure}

شکل  \حوالہ{شکل_مسئلہ_تھونن_عمومی_دور}-پ میں \عددی{i} کو مسئلہ نفاذ کی مدد سے دو حصوں میں تقسیم  کیا جا سکتا ہے۔ پہلا حصہ \عددی{i'} کو ڈبہ دور کے اندرونی منبع نافذ کرتے ہیں جبکہ دوسرا حصہ \عددی{i''} کو بیرونی  منبع \عددی{v_0} نافذ کرتا ہے۔جیسا شکل \حوالہ{شکل_مسئلہ_ڈبہ_دور_دباو_اور_رو}-الف میں دکھایا گیا ہے،  \عددی{i'} حاصل کرتے وقت بیرونی منبع کو قصر دور کیا جاتا ہے لہٰذا اس رو کو \عددی{i_{\text{قصر}}} کہا  جائے گا۔
\begin{align}
i'=i_{\text{قصر}}
\end{align}

 اسی طرح جیسا شکل \حوالہ{شکل_مسئلہ_ڈبہ_دور_دباو_اور_رو}-ب میں دکھایا گیا ہے،   \عددی{i''} حاصل کرتے وقت ڈبہ دور کے تمام اندرونی منبع کے اثر کو ختم کیا جاتا ہے۔ڈبہ دور  کے تمام اندرونی منبع کو صفر کرنے سے  بیرونی منبع \عددی{v_0} کو ڈبہ دور کے اندرونی مزاحمتوں کا مساوی مزاحمت \عددی{R_{\text{تھونن}}} نظر آئے گا لہٰذا رو درج ذیل ہو گی۔
\begin{align}
i''=\frac{v_0}{R_{\text{تھونن}}}
\end{align}
شکل \حوالہ{شکل_مسئلہ_ڈبہ_دور_دباو_اور_رو}-الف اور شکل \حوالہ{شکل_مسئلہ_ڈبہ_دور_دباو_اور_رو}-ب میں رو کی سمتوں کو دیکھتے ہوئی \عددی{i=i'-i''} لکھا جا سکتا ہے۔
\begin{align}\label{مساوات_مسئلہ_تھونن_الف}
i=i_{\text{قصر}}-\frac{v_0}{R_{\text{تھونن}}}  \quad \quad \text{\RL{مسئلہ نارٹن}}
\end{align}
لکھا جا سکتا ہے۔

مساوات \حوالہ{مساوات_مسئلہ_تھونن_الف} عمومی مساوات ہے جس میں \عددی{i_{\text{قصر}}} اور \عددی{R_{\text{تھونن}}} صرف بائیں ڈبہ دور پر منحصر ہیں جبکہ \عددی{v_0} اور \عددی{i} پر دایاں ڈبہ دور بھی اثر انداز ہوتا ہے۔یوں اگر  شکل \حوالہ{شکل_مسئلہ_تھونن_عمومی_دور}-ب میں بائیں ڈبہ دور تبدیل نہ کیا جائے تب  \عددی{i_{\text{قصر}}} اور \عددی{R_{\text{تھونن}}} اٹل قیمتیں ہوں گی جبکہ \عددی{v_0} اور \عددی{i} متغیرات ہوں گے جو دائیں ڈبہ دور پر منحصر ہوں گے۔چونکہ مساوات \حوالہ{مساوات_مسئلہ_تھونن_الف} عمومی مساوات ہے لہٰذا یہ ہر ممکنہ صورت حال کے لئے درست ہو گی۔یوں دائیں ڈبہ دور کھلا دور ہونے کی صورت میں بھی یہی مساوات کارآمد ہو گی۔اگر دائیں ڈبہ دور کو کھلا دور تصور کیا جائے تب
\begin{gather}
\begin{aligned}\label{مساوات_مسئلہ_تھونن_ب}
i&=0\\
v_0&=v_{\text{کھلا}}
\end{aligned}
\end{gather}
ہوں گے۔شکل \حوالہ{شکل_مسئلہ_کھلا_دور_تھونن} میں کھلے دور کی صورت حال دکھائی گئی ہے۔اس طرح مساوات \حوالہ{مساوات_مسئلہ_تھونن_الف} میں مساوات \حوالہ{مساوات_مسئلہ_تھونن_ب} پُر کرتے ہوئے
\begin{align*}
0=i_{\text{قصر}}-\frac{v_{\text{کھلا}}}{R_{\text{تھونن}}}
\end{align*}
یعنی
\begin{align}\label{مساوات_مسئلہ_تھونن_پ}
i_{\text{قصر}}=\frac{v_{\text{کھلا}}}{R_{\text{تھونن}}} \quad \quad \text{\RL{مسئلہ تبادلہ منبع}}
\end{align}
یا
\begin{align}\label{مساوات_مسئلہ_تھونن_ت}
v_{\text{کھلا}}= i_{\text{قصر}} R_{\text{تھونن}}\quad \quad \text{\RL{مسئلہ تبادلہ منبع}}
\end{align}
حاصل ہوتا ہے۔مساوات \حوالہ{مساوات_مسئلہ_تھونن_پ} کو مساوات \حوالہ{مساوات_مسئلہ_تھونن_الف} میں پُر کرنے سے
\begin{align*}
i=\frac{v_{\text{کھلا}}}{R_{\text{تھونن}}}-\frac{v_0}{R_{\text{تھونن}}}
\end{align*}
یعنی
\begin{align}\label{مساوات_مسئلہ_تھونن_ٹ}
v_0=v_{\text{کھلا}} - i R_{\text{تھونن}} \quad \quad \text{\RL{مسئلہ تھونن}}
\end{align}
حاصل ہوتا ہے۔

\begin{figure}
\centering
\begin{subfigure}{0.5\textwidth}
\centering
\begin{tikzpicture}
\draw[name path=leftC]  plot [smooth cycle] coordinates {(0,0) (1,0) (1,2.5) (0,2.5)};
\draw(0.5,1.25)node[rotate=90]{\RL{بایاں خطی حصہ}};
\path[name path=lineL] (0.5,0.25)--++(1,0);
\path[name path=lineU] (0.5,2.25)--++(1,0);
\draw[name intersections={of=leftC and lineL,by=leftL},name intersections={of=leftC and lineU,by=leftU}](leftL) to [short,-o] ++(2,0)coordinate(kka)node[below]{$B$} to [short]++(1,0)coordinate(kLowR)  (leftU) to [short,i={${i'=i_{\text{قصر}}}$},-o] ++(2,0)coordinate(kkb)node[above]{$A$} to [short]++(1,0)coordinate(kUpR);
\draw(kLowR) to [short](kUpR);
\end{tikzpicture}
\caption*{(الف)}
\end{subfigure}%
\begin{subfigure}{0.5\textwidth}
\centering
\begin{tikzpicture}
\draw[name path=leftC]  plot [smooth cycle] coordinates {(0,0) (1,0) (1,2.5) (0,2.5)};
%\draw(0.5,1)node[rotate=90]{\RL{بایاں خطی حصہ}};
\path[name path=lineL] (0.5,0.25)--++(1,0);
\path[name path=lineU] (0.5,2.25)--++(1,0);
\draw[name intersections={of=leftC and lineL,by=leftL},name intersections={of=leftC and lineU,by=leftU}](leftL) to [short,-o] ++(2,0)coordinate(kka)node[below]{$B$} to [short]++(1,0)coordinate(kLowR)  (leftU) to [short,i<={${i'' =\frac{v_0}{R_{\text{تھونن}}}}$},-o] ++(2,0)coordinate(kkb)node[above]{$A$} to [short]++(1,0)coordinate(kUpR);
\draw(kLowR) to [american voltage source,l_={$v_0$}](kUpR);
\draw(leftL)--++(-0.5,0) to [resistor,l={$R_{\text{تھونن}}$}]  ++(0,\y) |- (leftU);
\end{tikzpicture} 
\caption*{(ب)}
\end{subfigure}%
\caption{رو کو مسئلہ نفاذ سے دو حصوں میں تقسیم کیا جا سکتا ہے۔}
\label{شکل_مسئلہ_ڈبہ_دور_دباو_اور_رو}
\end{figure}

\begin{figure}
\centering
\begin{tikzpicture}
\draw[name path=leftC]  plot [smooth cycle] coordinates {(0,0) (1,0) (1,2.5) (0,2.5)};
\draw(0.5,1.25)node[rotate=90]{\RL{بایاں خطی حصہ}};
\path[name path=lineL] (0.5,0.25)--++(1,0);
\path[name path=lineU] (0.5,2.25)--++(1,0);
\draw[name intersections={of=leftC and lineL,by=leftL},name intersections={of=leftC and lineU,by=leftU}](leftL) to [short,-o] ++(2,0)coordinate(kka)node[below]{$B$}  (leftU) to [short,i={${i=0}$},-o] ++(2,0)coordinate(kkb)node[above]{$A$};
\draw($(kka)!0.5!(kkb)$) node{$\begin{aligned} &+ \\ &v_{\text{کھلا}} \\ &- \end{aligned}$};
\end{tikzpicture}
\caption{کھلے دور سروں پر صفر رو اور تھونن دباو پائی جاتی ہے۔}
\label{شکل_مسئلہ_کھلا_دور_تھونن}
\end{figure}

مساوات \حوالہ{مساوات_مسئلہ_تھونن_الف} \اصطلاح{مسئلہ نارٹن}\فرہنگ{مسئلہ!نارٹن}\فرہنگ{نارٹن!مسئلہ}\حاشیہد{ایڈورڈ لوری نارٹن اور ہنس فرڈینانڈ میئر نے اس مسئلے کو علیحدہ علیحدہ 1926؁ میں اخذ کیا۔}\حاشیہب{Norton Theorem}\فرہنگ{theorem!Norton}\فرہنگ{Norton theorem} بیان کرتی ہے جسے شکل \حوالہ{شکل_مسئلہ_تھونن_نارٹن}-الف میں دکھایا گیا ہے جبکہ مساوات \حوالہ{مساوات_مسئلہ_تھونن_ٹ} \اصطلاح{مسئلہ تھونن}\فرہنگ{مسئلہ!تھونن}\فرہنگ{تھونن!مسئلہ}\حاشیہد{لیوں شارلس تھونن نے 1883؁ میں اور  ہرمن لڈوگ فرڈینانڈ ون ہلم ہولٹز نے 1853؁ میں اس مسئلے کو علیحدہ علیحدہ اخذ کیا۔}\حاشیہب{Thevenin Theorem}\فرہنگ{Thevenin theorem}\فرہنگ{theorem!Thevenin} بیان کرتی ہے جسے شکل \حوالہ{شکل_مسئلہ_تھونن_نارٹن}-ب میں دکھایا گیا ہے۔مساوات \حوالہ{مساوات_مسئلہ_تھونن_پ} \اصطلاح{مسئلہ تبادلہ منبع}\فرہنگ{مسئلہ!تبادلہ منبع}\فرہنگ{تبادلہ منبع!مسئلہ}\حاشیہب{Source Transformation Theorem}\فرہنگ{theorem!Source Transformation}\فرہنگ{Source Transformation theorem} بیان کرتی ہے۔
\begin{figure}
\centering
\begin{subfigure}{0.5\textwidth}
\centering
\begin{tikzpicture}
\draw(0,0) to [american current source,l={$i_{\text{قصر}}$}] ++(0,\y);
\draw(\x,0) to [resistor,*-*,l={$R_{\text{تھونن}}$}] ++(0,\y);
\draw(0,0) to [short,-o]++(2*\x,0);
\draw(0,\y) to [short]++(1.5*\x,0) to [short,i={$i$},-o]++(\x/2,0);
\draw(2*\x,\y/2)node{$\begin{aligned} &+ \\ &v_0 \\ &- \end{aligned}$};
\draw(-0.3,3/4*\y)node[left]{$i_{\text{نارٹن}}$};
\end{tikzpicture}
\caption*{(الف) نارٹن مساوی دور۔}
\end{subfigure}%
\begin{subfigure}{0.5\textwidth}
\centering
\begin{tikzpicture}
\draw(0,0) to [short,o-] ++(-\xx,0) to [american voltage source,l={$v_{\text{کھلا}}$}]++(0,\y) to [resistor,i={$i$},-o,l={$R_{\text{تھونن}}$}]++(\xx,0);
\draw(0,\y/2)node{$\begin{aligned} &+ \\ &v_0 \\ &- \end{aligned}$};
\draw(-\xx-0.25,3/4*\y)node[left]{$v_{\text{تھونن}}$};
\end{tikzpicture}
\caption*{(ب) تھونن مساوی دور۔}
\end{subfigure}%
\caption{تھونن اور نارٹن مساوی ادوار۔}
\label{شکل_مسئلہ_تھونن_نارٹن}
\end{figure}

شکل \حوالہ{شکل_مسئلہ_تھونن_نارٹن}-الف کی کرخوف مساوات دباو اور شکل \حوالہ{شکل_مسئلہ_تھونن_نارٹن}-ب کے بالائی جوڑ پر کرخوف مساوات رو درج ذیل ہیں۔
\begin{align*}
v_0&=v_{\text{کھلا}}-i R_{\text{تھونن}}\\
i&=i_{\text{قصر}}-\frac{v_0}{R_{\text{تھونن}}}
\end{align*}
ان کا مساوات \حوالہ{مساوات_مسئلہ_تھونن_الف} اور مساوات \حوالہ{مساوات_مسئلہ_تھونن_ب} سے موازنہ کرنے سے صاف ظاہر ہے کہ شکل \حوالہ{شکل_مسئلہ_تھونن_نارٹن}-الف اور شکل \حوالہ{شکل_مسئلہ_تھونن_نارٹن}-ب انہیں مساوات کو ظاہر کرتے ہیں۔

یوں کسی بھی دور کو شکل \حوالہ{شکل_مسئلہ_تھونن_نارٹن}-الف کا تھونن مساوی دور یا شکل \حوالہ{شکل_مسئلہ_تھونن_نارٹن}-ب کا نارٹن مساوی دور ظاہر کر سکتا ہے۔نارٹن مساوی دور میں منبع رو کو \عددی{i_{\text{نارٹن}}} یعنی \اصطلاح{نارٹن رو}\فرہنگ{نارٹن!رو}\فرہنگ{رو!نارٹن}\حاشیہب{norton current}\فرہنگ{norton!current} بھی پکارا جاتا ہے۔اسی طرح تھونن مساوی دور میں منبع دباو کو \عددی{v_{\text{تھونن}}} یعنی \اصطلاح{تھونن دباو}\فرہنگ{تھونن!دباو}\فرہنگ{دباو!تھونن}\حاشیہب{thevenin voltage}\فرہنگ{thevenin voltage} بھی پکارا جاتا ہے۔

مساوات \حوالہ{مساوات_مسئلہ_تھونن_پ} یا مساوات \حوالہ{مساوات_مسئلہ_تھونن_ت} یعنی مسئلہ تبادلہ منبع کی مدد سے تھونن دور سے نارٹن دور اور نارٹن دور سے تھونن دور حاصل ہوتا ہے۔

آئیں ان مسئلوں کا استعمال مثالوں کو حل کرتے ہوئے دیکھیں۔

%===============
\ابتدا{مثال}\شناخت{مثال_مسئلہ_تھونن_الف}
شکل \حوالہ{شکل_مسئلہ__مثال_تھونن_الف}-الف میں مسئلہ تھونن استعمال کرتے ہوئے \عددی{V_0} حاصل کریں۔
\begin{figure}
\centering
\begin{subfigure}{1\textwidth}
\centering
\begin{tikzpicture}
\draw(0,0) to [american current source,*-*,l={$\SI{2}{\milli\ampere}$}]++(0,\y) to [resistor,l_={$\SI{1}{\kilo\ohm}$}]++(-\x,0) to [resistor,l_={$\SI{3}{\kilo\ohm}$}]++(0,-\y) to [short]++(\x,0) to [short]++(\x,0) to [resistor,*-*,l={$\SI{6}{\kilo\ohm}$}]++(0,\y);
\draw(\x+\x/2,\y) to [short,o-] ++(-\x/2,0) to [american voltage source,l_={$\SI{3}{\volt}$}]++(-\x,0);
\draw(\x,0) to [short,-o]++(\x/2,0);
\draw(\x+\x/2,\y/2)node{$\begin{aligned} &+ \\& V_0 \\ &- \end{aligned}$};
\end{tikzpicture}
\caption*{(الف)}
\end{subfigure}
\begin{subfigure}{0.5\textwidth}
\centering
\begin{tikzpicture}
\draw(0,0)node[ground]{} to [american current source,*-*,l={$\SI{2}{\milli\ampere}$}]++(0,\y)node[above]{$V_1$} to [resistor,l_={$\SI{1}{\kilo\ohm}$}]++(-\x,0) to [resistor,l_={$\SI{3}{\kilo\ohm}$}]++(0,-\y) to [short]++(\x,0) to [short,-o]++(\x,0);
\draw(\x,\y)  to [american voltage source,o-,l_={$\SI{3}{\volt}$}]++(-\x,0);
\draw(\x,\y/2)node{$\begin{aligned} &+ \\& V_{\text{کھلا}} \\ &- \end{aligned}$};
\end{tikzpicture}%
\caption*{(ب) تھونن دباو۔}
\end{subfigure}%
\begin{subfigure}{0.5\textwidth}
\centering
\begin{tikzpicture}
\draw(0,0) to [short,o-]++(-\x,0) to [resistor,l={$\SI{3}{\kilo\ohm}$}]++(0,\y) to [resistor,-o,l={$\SI{1}{\kilo\ohm}$}]++(\x,0);
\draw[stealth-](0,\y/2)--++(\x/2,0)--++(0,-\y/4)node[below]{$R_{\text{تھونن}}$};
\end{tikzpicture}
\caption*{(پ) تھونن مزاحمت۔}
\end{subfigure}
\begin{subfigure}{1\textwidth}
\centering
\begin{tikzpicture}
\draw(0,0) to [short,o-]++(-\x-\x,0) to [american voltage source,l={$\SI{5}{\volt}$}]++(0,\y) to [resistor,l_={$\SI{4}{\kilo\ohm}$}]++(\x,0) to [short,-o]++(\x,0);
\draw(-\x/2,0) to [resistor,*-*,l={$\SI{6}{\kilo\ohm}$}]++(0,\y);
\draw(0,\y/2)node{$\begin{aligned} &+ \\ &V_0 \\ &- \end{aligned}$};
\draw(-\x-\x-0.5,3/4*\y)node[left]{$V_{\text{کھلا}}$};
\draw(-\x/2-\x,\y+0.3)node[above]{$R_{\text{تھونن}}$};
\end{tikzpicture}
\caption*{(ت) تھونن مساوی دور استعمال کیا گیا ہے۔}
\end{subfigure}
\caption{مثال \حوالہ{مثال_مسئلہ_تھونن_الف} کا دور۔}
\label{شکل_مسئلہ__مثال_تھونن_الف}
\end{figure}

حل:اس دور کو حل کرنے کی خاطر ہم  \عددی{\SI{6}{\kilo\ohm}} کے علاوہ بقایا دور کا تھونن مساوی دور حاصل کرتے ہیں۔یوں \عددی{\SI{6}{\kilo\ohm}} کو بوجھ تصور کیا جائے گا۔شکل-ب میں بوجھ کو ہٹاتے ہوئے بقایا دور دکھایا گیا ہے جس کا تھونن مساوی دور درکار ہے۔اس دور کے کھلے سروں پر \عددی{V_{\text{کھلا}}} پایا جاتا ہے۔نچلی جوڑ کو زمین تصور کرتے ہوئے بالائی جوڑ \عددی{V_1} پر دباو دریافت کرتے ہیں۔منبع رو کی پوری رو بائیں خانے میں گھڑی کی الٹ گھومتی ہے لہٰذا
\begin{align*}
V_1=\SI{2}{\milli\ampere} \left(\SI{3}{\kilo\ohm}+\SI{1}{\kilo\ohm}\right)=\SI{8}{\volt}
\end{align*}
لکھا جا سکتا ہے۔یوں
\begin{align*}
V_{\text{کھلا}}= V_1-\SI{3}{\volt}=\SI{5}{\volt}
\end{align*}
حاصل ہوتا ہے۔آئیں اب تھونن مزاحمت حاصل کریں۔

دور میں منبع دباو کو قصر دور اور منبع رو کو کھلے دور کرتے ہوئے شکل-پ حاصل ہوتا ہے جہاں سے
\begin{align*}
R_{\text{تھونن}}=\SI{4}{\kilo\ohm}
\end{align*}
لکھا جا سکتا ہے۔یوں شکل-ب کی جگہ اس کا مساوی تھونن دور نسب کرتے ہوئے شکل-الف کی جگہ شکل-ت حاصل ہوتا ہے جسے دیکھتے ہوئے  تقسیم دباو کے کلیے سے بوجھ پر دباو  درج ذیل لکھا جا سکتا ہے۔
\begin{align}
V_0=5\left(\frac{\SI{6}{\kilo\ohm}}{\SI{6}{\kilo\ohm}+\SI{4}{\kilo\ohm}}\right)=\SI{3}{\volt}
\end{align}
\انتہا{مثال}
%================
\ابتدا{مثال}\شناخت{مثال_مسئلہ_تھونن_ب}
شکل \حوالہ{شکل_مسئلہ__مثال_تھونن_الف}-الف میں مسئلہ نارٹن استعمال کرتے ہوئے \عددی{V_0} حاصل کریں۔

\begin{figure}
\centering
\begin{subfigure}{0.5\textwidth}
\centering
\begin{tikzpicture}
\draw(0,0)node[ground]{} to [american current source,*-*,l={$\SI{2}{\milli\ampere}$}]++(0,\y)node[above]{$V_1$} to [resistor,i_={$i_1$},l_={$\SI{1}{\kilo\ohm}$}]++(-\x,0) to [resistor,l_={$\SI{3}{\kilo\ohm}$}]++(0,-\y) to [short]++(\x,0) to [short]++(\x,0) to [short,i<_={$i_{\text{قصر}}$}]++(0,\y);
\draw(\x,\y)  to [american voltage source,l_={$\SI{3}{\volt}$}]++(-\x,0);
%\draw(\x,\y/2)node{$\begin{aligned} &+ \\& V_{\text{کھلا}} \\ &- \end{aligned}$};
\end{tikzpicture}%
\caption*{(الف) قصر دور رو۔}
\end{subfigure}
\begin{subfigure}{0.5\textwidth}
\centering
\begin{tikzpicture}
\draw(0,0) to [american current source,l={$\frac{5}{4}\, \si{\milli\ampere}$}]++(0,\y) to [short,-o]++(2*\x+\x/2,0);
\draw(0,0) to [short,-o]++(2*\x+\x/2,0);
\draw(\x,0) to [resistor,*-*,l={$\SI{4}{\kilo\ohm}$}]++(0,\y);
\draw(2*\x,0) to [resistor,*-*,l={$\SI{6}{\kilo\ohm}$}]++(0,\y);
\draw(2*\x+\x/2,\y/2)node{$\begin{aligned} &+ \\& V_0 \\ &- \end{aligned}$};
\draw(-0.5,3/4*\y)node[left]{$i_{\text{قصر}}$};
\draw(\x-0.3,3/4*\y)node[left]{$R_{\text{تھونن}}$};
\end{tikzpicture}%
\caption*{(ب) نارٹن مساوی دور استعمال کیا گیا ہے}
\end{subfigure}
\caption{مثال \حوالہ{مثال_مسئلہ_تھونن_ب} کا دور۔}
\label{شکل_مسئلہ__مثال_تھونن_ب}
\end{figure}

حل:گزشتہ مثال کی طرح دور کو دو ٹکڑوں میں تقسیم کیا جاتا ہے لہٰذا شکل \حوالہ{شکل_مسئلہ__مثال_تھونن_الف}-الف میں \عددی{\SI{6}{\kilo\ohm}} کو بوجھ سمجھتے ہوئے بقایا دور، جسے شکل \حوالہ{شکل_مسئلہ__مثال_تھونن_الف}-ب میں دکھایا گیا ہے،  کا نارٹن مساوی دور  حاصل کیا جائے گا۔

نارٹن مساوی دور میں \عددی{R_{\text{تھونن}}} کے ساتھ ساتھ \عددی{i_{\text{قصر}}} بھی درکار ہے۔تھونن مزاحمت کو گزشتہ مثال میں حاصل کیا گیا ہے لہٰذا صرف قصر دور رو معلوم کرنا باقی ہے۔شکل \حوالہ{شکل_مسئلہ__مثال_تھونن_الف}-ب کو قصر دور کرتے ہوئے شکل \حوالہ{شکل_مسئلہ__مثال_تھونن_ب}-الف میں دکھایا گیا ہے جس سے  \عددی{i_{\text{قصر}}} حاصل کرتے ہیں۔دور کو دیکھتے ہوئے 
\begin{align*}
V_1=\SI{3}{\volt}
\end{align*}
اور یوں
\begin{align*}
i_1=\frac{\SI{3}{\volt}}{\SI{1}{\kilo\ohm}+\SI{3}{\kilo\ohm}}=\frac{3}{4} \, \si{\milli\ampere}
\end{align*}
لکھا جا سکتا ہے۔بالائی جوڑ \عددی{V_1} پر کرخوف قانون رو سے درج ذیل لکھا جا سکتا ہے۔
\begin{align*}
i_{\text{قصر}} = \SI{2}{\milli\ampere}-\frac{3}{4} \, \si{\milli\ampere}=\frac{5}{4} \, \si{\milli\ampere}
\end{align*}
نارٹن دور کے متغیرات  استعمال کرتے ہوئے شکل \حوالہ{شکل_مسئلہ__مثال_تھونن_ب}-ب حاصل ہوتا ہے جہاں منبع رو کے متوازی مزاحمتوں کا مساوی
\begin{align*}
\SI{4}{\kilo\ohm} \parallel \SI{6}{\kilo\ohm}=\frac{12}{5} \, \si{\kilo\ohm}
\end{align*}
ہے  جس میں \عددی{\tfrac{5}{4} \, \si{\milli\ampere}} گزرنے سے دباو
\begin{align*}
V_0=\frac{5}{4} \, \si{\milli\ampere} \times \frac{12}{5} \, \si{\kilo\ohm}=\SI{3}{\volt}
\end{align*}
پیدا ہو گا۔

اس مثال میں \عددی{i_{قصر}} کو مساوات \حوالہ{مساوات_مسئلہ_تھونن_ت} یعنی مسئلہ تبادلہ منبع سے بھی حاصل کیا جا سکتا تھا یعنی
\begin{align*}
i_{\text{قصر}}=\frac{v_{\text{کھلا}}}{R_{\text{تھونن}}}=\frac{\SI{5}{\volt}}{\SI{4}{\kilo\ohm}}=\frac{5}{4} \, \si{\milli\ampere}
\end{align*}
\انتہا{مثال}
%==================
\ابتدا{مثال}\شناخت{مثال_مسئلہ_تھونن_پ}
شکل \حوالہ{شکل_مثال_تھونن_سے_نارٹن_الف}-الف میں ایک دور کا مساوی تھونن دور دیا گیا ہے۔اس دور کا مساوی نارٹن دور حاصل کریں۔
\begin{figure}
\centering
\begin{subfigure}{0.5\textwidth}
\centering
\begin{tikzpicture}
\draw(0,0) to [short,o-]++(-\x-\x/4,0) to [american voltage source,l={$\SI{12}{\volt}$}]++(0,\y) to [resistor,l={$\SI{24}{\kilo\ohm}$}]++(\x,0) to [short,-o]++(\x/4,0);
\draw(-\x-\x/4-0.5,3/4*\y)node[left]{$v_{\text{تھونن}}$};
\draw(-\x/2-\x/4,\y-0.3)node[below]{$R_{\text{تھونن}}$};
\end{tikzpicture}
\caption*{(الف) تھونن دور۔}
\end{subfigure}%
\begin{subfigure}{0.5\textwidth}
\centering
\begin{tikzpicture}
\draw(0,0) to [short,o-]++(-\x-\x/2,0) to [american current source,l={$\SI{0.5}{\milli\ampere}$}]++(0,\y) to [short,-o]++(\x+\x/2,0);
\draw(-\x/2,0) to [resistor,*-*,l={$\SI{24}{\kilo\ohm}$}]++(0,\y);
\draw(-\x-\x/4-1,3/4*\y)node[left]{$i_{\text{نارٹن}}$};
\draw(-\x/4-0.8,3/4*\y)node[left]{$R_{\text{تھونن}}$};
\end{tikzpicture}
\caption*{(ب) مساوی نارٹن دور۔}
\end{subfigure}%
\caption{مثال \حوالہ{مثال_مسئلہ_تھونن_پ} کا مساوی تھونن دور۔}
\label{شکل_مثال_تھونن_سے_نارٹن_الف}
\end{figure}

حل:تھونن دور سے نارٹن دور یا نارٹن دور سے تھونن دور کے حصول میں مساوات \حوالہ{مساوات_مسئلہ_تھونن_ت} اہم کردار ادا کرتی ہے۔اس مساوات کی مدد سے  تھونن دور کے متغیرات  \عددی{v_{\text{کھلا}}} اور \عددی{R_{\text{تھونن}}} سے نارٹن دور میں استعمال ہونے والا متغیر \عددی{i_{\text{قصر}}} حاصل کیا جا سکتا ہے۔اسی طرح اسی مساوات کی مدد سے نارٹن دور میں استعمال ہونے والے متغیرات \عددی{i_{\text{قصر}}} اور   \عددی{R_{\text{تھونن}}} سے تھونن دور کا متغیر  \عددی{v_{\text{کھلا}}} حاصل کیا جا سکتا ہے۔دونوں ادوار میں \عددی{R_{\text{تھونن}}} کی قیمت یکساں ہے۔

مساوات \حوالہ{مساوات_مسئلہ_تھونن_ت} استعمال کرتے ہوئے
\begin{align*}
i_{\text{قصر}} = \frac{v_{\text{کھلا}}}{R_{\text{تھونن}}}=\frac{\SI{12}{\volt}}{\SI{24}{\kilo\ohm}}=\SI{0.5}{\milli\ampere}
\end{align*}
حاصل ہوتا ہے جسے استعمال کرتے ہوئے  شکل \حوالہ{شکل_مثال_تھونن_سے_نارٹن_الف}-ب کا مساوی نارٹن دور حاصل ہوتا ہے۔ 
\انتہا{مثال}
%=====================
\ابتدا{مثال}\شناخت{مثال_مسئلہ_تھونن_ت}
شکل \حوالہ{شکل_مثال_تھونن_سے_نارٹن_ب}-الف میں \عددی{\SI{3}{\kilo\ohm}} کو بوجھ تصور کریں۔بار بار تھونن سے نارٹن اور نارٹن سے تھونن مساوی دور حاصل کرتے ہوئے بقایا دور کا  تھونن مساوی حاصل کرتے ہوئے بوجھ پر دباو حاصل کریں۔

\begin{figure}
\centering
\begin{subfigure}{1\textwidth}
\centering
\begin{tikzpicture}
\draw(0,0) to [short,-o]++(3*\x+\x/2,0)node[below]{$B$};
\draw(0,0) to [american voltage source,l={$\SI{8}{\volt}$}]++(0,\y) to [resistor,l={$\SI{2}{\kilo\ohm}$}]++(\x,0) to [resistor,l={$\SI{1}{\kilo\ohm}$}]++(\x,0) to [resistor,l={$\SI{4}{\kilo\ohm}$}]++(\x,0)to [short,-o]++(\x/2,0)node[above]{$A$};
\draw(\x,0)node[below]{$b$} to [resistor,*-*,l={$\SI{12}{\kilo\ohm}$}]++(0,\y)node[above]{$a$};
\draw(2*\x,0)node[below]{$d$} to [american current source,*-*,l={$\SI{4}{\milli\ampere}$}]++(0,\y)node[above]{$c$};
\draw(3*\x,0)node[below]{$f$} to [resistor,*-*,l={$\SI{3}{\kilo\ohm}$}]++(0,\y)node[above]{$e$};
\draw(3*\x+\x/2,\y/2)node{$\begin{aligned} &+ \\ &v_0 \\ &- \end{aligned}$};
\end{tikzpicture}
\caption*{(الف)}
\end{subfigure}
\begin{subfigure}{1\textwidth}
\centering
\begin{tikzpicture}
\draw(-\x,0) to [short,-o]++(4*\x+\x/2,0)node[below]{$B$};
\draw(-\x,0) to [american current source,l={$\SI{4}{\milli\ampere}$}]++(0,\y) to [short]++(2*\x,0) to [resistor,l={$\SI{1}{\kilo\ohm}$}]++(\x,0) to [resistor,l={$\SI{4}{\kilo\ohm}$}]++(\x,0)to [short,-o]++(\x/2,0)node[above]{$A$};
\draw(0,0) to [resistor,*-*,l={$\SI{2}{\kilo\ohm}$}]++(0,\y);
\draw(\x,0)node[below]{$b$} to [resistor,*-*,l={$\SI{12}{\kilo\ohm}$}]++(0,\y)node[above]{$a$};
\draw(2*\x,0)node[below]{$d$} to [american current source,*-*,l={$\SI{4}{\milli\ampere}$}]++(0,\y)node[above]{$c$};
\draw(3*\x,0)node[below]{$f$} to [resistor,*-*,l={$\SI{3}{\kilo\ohm}$}]++(0,\y)node[above]{$e$};
\draw(3*\x+\x/2,\y/2)node{$\begin{aligned} &+ \\ &v_0 \\ &- \end{aligned}$};
\end{tikzpicture}
\caption*{(ب)}
\end{subfigure}
\begin{subfigure}{1\textwidth}
\centering
\begin{tikzpicture}
\draw(0,0) to [short,-o]++(3*\x+\x/2,0)node[below]{$B$};
\draw(0,0) to [american current source,l={$\SI{4}{\milli\ampere}$}]++(0,\y) to [short]++(\x,0) to [resistor,l={$\SI{1}{\kilo\ohm}$}]++(\x,0) to [resistor,l={$\SI{4}{\kilo\ohm}$}]++(\x,0)to [short,-o]++(\x/2,0)node[above]{$A$};
\draw(\x,0)node[below]{$b$} to [resistor,*-*,l={$\frac{12}{7} \,\si{\kilo\ohm}$}]++(0,\y)node[above]{$a$};
\draw(2*\x,0)node[below]{$d$} to [american current source,*-*,l={$\SI{4}{\milli\ampere}$}]++(0,\y)node[above]{$c$};
\draw(3*\x,0)node[below]{$f$} to [resistor,*-*,l={$\SI{3}{\kilo\ohm}$}]++(0,\y)node[above]{$e$};
\draw(3*\x+\x/2,\y/2)node{$\begin{aligned} &+ \\ &v_0 \\ &- \end{aligned}$};
\end{tikzpicture}
\caption*{(پ)}
\end{subfigure}
\begin{subfigure}{1\textwidth}
\centering
\begin{tikzpicture}
\draw(0,0) to [short,-o]++(3*\x+\x/2,0)node[below]{$B$};
\draw(0,0) to [american voltage source,l={$\frac{48}{7} \, \si{\volt}$}]++(0,\y) to [resistor,l={$\frac{12}{7} \, \si{\kilo\ohm}$}]++(\x,0) to [resistor,l={$\SI{1}{\kilo\ohm}$}]++(\x,0) to [resistor,l={$\SI{4}{\kilo\ohm}$}]++(\x,0)to [short,-o]++(\x/2,0)node[above]{$A$};
\draw(2*\x,0)node[below]{$d$} to [american current source,*-*,l={$\SI{4}{\milli\ampere}$}]++(0,\y)node[above]{$c$};
\draw(3*\x,0)node[below]{$f$} to [resistor,*-*,l={$\SI{3}{\kilo\ohm}$}]++(0,\y)node[above]{$e$};
\draw(3*\x+\x/2,\y/2)node{$\begin{aligned} &+ \\ &v_0 \\ &- \end{aligned}$};
\end{tikzpicture}
\caption*{(ت)}
\end{subfigure}
\begin{subfigure}{1\textwidth}
\centering
\begin{tikzpicture}
\draw(\x,0) to [short,-o]++(2*\x+\x/2,0)node[below]{$B$};
\draw(\x,0) to [american voltage source,l={$\frac{48}{7} \, \si{\volt}$}]++(0,\y) to [resistor,l={$\frac{19}{7} \, \si{\kilo\ohm}$}]++(\x,0)  to [resistor,l={$\SI{4}{\kilo\ohm}$}]++(\x,0)to [short,-o]++(\x/2,0)node[above]{$A$};
\draw(2*\x,0)node[below]{$d$} to [american current source,*-*,l={$\SI{4}{\milli\ampere}$}]++(0,\y)node[above]{$c$};
\draw(3*\x,0)node[below]{$f$} to [resistor,*-*,l={$\SI{3}{\kilo\ohm}$}]++(0,\y)node[above]{$e$};
\draw(3*\x+\x/2,\y/2)node{$\begin{aligned} &+ \\ &v_0 \\ &- \end{aligned}$};
\end{tikzpicture}
\caption*{(ٹ)}
\end{subfigure}
\caption{مثال \حوالہ{مثال_مسئلہ_تھونن_ت} کا دور۔}
\label{شکل_مثال_تھونن_سے_نارٹن_ب}
\end{figure}

حل: شکل \حوالہ{شکل_مثال_تھونن_سے_نارٹن_ب} کے بائیں سر سے شروع کرتے ہیں جہاں \عددی{\SI{8}{\volt}} اور \عددی{\SI{2}{\kilo\ohm}} کو تھونن مساوی دور تصور کیا جا سکتا ہے۔اس دور کے سروں کو \عددی{a} اور \عددی{b} تصور کیا جا سکتا ہے۔یوں \عددی{v_{\text{تھونن}}=\SI{8}{\volt}} اور \عددی{R_{\text{تھونن}}=\SI{2}{\kilo\ohm}} لیتے ہوئے مساوات \حوالہ{مساوات_مسئلہ_تھونن_پ} کی مدد سے 
\begin{align*}
i_{\text{نارٹن}}=\frac{\SI{8}{\volt}}{\SI{2}{\kilo\ohm}}=\SI{4}{\milli\ampere}
\end{align*}
حاصل ہوتا ہے۔نقطہ \عددی{a} اور \عددی{b} کے بائیں جانب تھونن دور کی جگہ یوں مساوی نارٹن دور نسب کیا جا سکتا ہے۔شکل-ب میں ایسا ہی کیا ہوا دکھایا گیا ہے جہاں \عددی{\SI{2}{\kilo\ohm}} اور \عددی{\SI{12}{\kilo\ohm}} متوازی مزاحمتوں کا مساوی \عددی{\tfrac{\SI{2}{\kilo\ohm} \times \SI{12}{\kilo\ohm}}{\SI{2}{\kilo\ohm}+ \SI{12}{\kilo\ohm}}=\tfrac{12}{7}\,\si{\kilo\ohm}} ہو گا۔ شکل-پ میں متوازی مزاحمتوں کی جگہ \عددی{\tfrac{12}{7}\,\si{\kilo\ohm}} کو دکھایا گیا ہے۔

شکل-پ میں \عددی{\SI{4}{\milli\ampere}} کو \عددی{i_{\text{نارٹن}}} اور \عددی{\tfrac{12}{7}\,\si{\kilo\ohm}} کو \عددی{R_{\text{تھونن}}} تصور کیا جا سکتا ہے۔ان دو اجزاء کے نارٹن دور کا مساوی تھونن دور حاصل کرنے کی خاطر مساوات \حوالہ{مساوات_مسئلہ_تھونن_پ} کی مدد سے 
\begin{align*}
v_{\text{تھونن}}=i_{\text{نارٹن}} R_{\text{تھونن}}=\SI{4}{\milli\ampere} \times \frac{12}{7} \, \si{\kilo\ohm}=\frac{48}{7} \, \si{\volt}
\end{align*}
حاصل کیا جاتا ہے۔شکل-پ میں \عددی{\SI{4}{\milli\ampere}}  اور  \عددی{\tfrac{12}{7}\,\si{\kilo\ohm}} کے نارٹن دور کی جگہ \عددی{\tfrac{48}{7} \, \si{\volt}} اور \عددی{\tfrac{12}{7}\,\si{\kilo\ohm}} کا تھونن دور نسب کرنے سے شکل-ت حاصل ہوتا ہے۔شکل-ت میں سلسلہ وار جڑے \عددی{\tfrac{12}{7}\,\si{\kilo\ohm}} اور \عددی{\SI{1}{\kilo\ohm}} کی جگہ ان کا مساوی \عددی{\tfrac{19}{7}\,\si{\kilo\ohm}} نسب کرنے سے شکل-ٹ حاصل ہوتا ہے۔

شکل-ٹ میں \عددی{\tfrac{19}{7}\,\si{\kilo\ohm}} اور  \عددی{\tfrac{48}{7} \, \si{\volt}} مل کر تھونن دور بناتے ہیں جن کی جگہ نارٹن دور نسب کرنے کی غرض سے 
\begin{align*}
i_{\text{نارٹن}}=\frac{v_{\text{تھونن}}}{R_{\text{تھونن}}}=\frac{\frac{48}{7} \, \si{\volt}}{\frac{19}{7} \, \si{\kilo\ohm}}=\frac{48}{19}\,\si{\milli\ampere}
\end{align*}
حاصل کرتے ہیں۔شکل \حوالہ{شکل_مثال_تھونن_سے_نارٹن_پ}-الف میں حاصل دور دکھایا گیا ہے جہاں \عددی{\frac{48}{19}\,\si{\milli\ampere}} اور \عددی{\SI{4}{\milli\ampere}} متوازی جڑے منبع ہیں جن کا مجموعہ 
\begin{align*}
\frac{48}{19}\,\si{\milli\ampere}+\SI{4}{\milli\ampere}=\frac{124}{19}\,\si{\milli\ampere}
\end{align*}
کے برابر ہے۔شکل \حوالہ{شکل_مثال_تھونن_سے_نارٹن_پ}-ب میں متوازی منبع کی جگہ ان کی مجموعی قیمت کا منبع نسب کیا گیا ہے۔
\begin{figure}
\centering
\begin{subfigure}{1\textwidth}
\centering
\begin{tikzpicture}
\draw(0,0) to [short,-o]++(3*\x+\x/2,0)node[below]{$B$};
\draw(0,0) to [american current source,l={$\frac{48}{19} \, \si{\milli\ampere}$}]++(0,\y) to [short]++(2*\x,0)  to [resistor,l={$\SI{4}{\kilo\ohm}$}]++(\x,0)to [short,-o]++(\x/2,0)node[above]{$A$};
\draw(\x,0) to [resistor,*-*,l={$\frac{19}{7} \, \si{\kilo\ohm}$}]++(0,\y);
\draw(2*\x,0)node[below]{$d$} to [american current source,*-*,l={$\SI{4}{\milli\ampere}$}]++(0,\y)node[above]{$c$};
\draw(3*\x,0)node[below]{$f$} to [resistor,*-*,l={$\SI{3}{\kilo\ohm}$}]++(0,\y)node[above]{$e$};
\draw(3*\x+\x/2,\y/2)node{$\begin{aligned} &+ \\ &v_0 \\ &- \end{aligned}$};
\end{tikzpicture}
\caption*{(الف)}
\end{subfigure}
\begin{subfigure}{1\textwidth}
\centering
\begin{tikzpicture}
\draw(\x,0) to [short,-o]++(2*\x+\x/2,0)node[below]{$B$};
\draw(\x,0) to [american current source,l={$\frac{124}{19} \, \si{\milli\ampere}$}]++(0,\y) to [short]++(\x,0)  to [resistor,l={$\SI{4}{\kilo\ohm}$}]++(\x,0)to [short,-o]++(\x/2,0)node[above]{$A$};
\draw(2*\x,0)node[below]{$d$} to [resistor,*-*,l={$\frac{19}{7} \, \si{\kilo\ohm}$}]++(0,\y)node[above]{$c$};
\draw(3*\x,0)node[below]{$f$} to [resistor,*-*,l={$\SI{3}{\kilo\ohm}$}]++(0,\y)node[above]{$e$};
\draw(3*\x+\x/2,\y/2)node{$\begin{aligned} &+ \\ &v_0 \\ &- \end{aligned}$};
\end{tikzpicture}
\caption*{(ب)}
\end{subfigure}
\begin{subfigure}{0.5\textwidth}
\centering
\begin{tikzpicture}
\draw(\x,0) to [short,-o]++(2*\x+\x/2,0)node[below]{$B$};
\draw(\x,0) to [american voltage source,l={$\frac{124}{7} \, \si{\volt}$}]++(0,\y) to [resistor,l={$\frac{19}{7} \, \si{\kilo\ohm}$}]++(\x,0)  to [resistor,l={$\SI{4}{\kilo\ohm}$}]++(\x,0)to [short,-o]++(\x/2,0)node[above]{$A$};
\draw(3*\x,0)node[below]{$f$} to [resistor,*-*,l={$\SI{3}{\kilo\ohm}$}]++(0,\y)node[above]{$e$};
\draw(3*\x+\x/2,\y/2)node{$\begin{aligned} &+ \\ &v_0 \\ &- \end{aligned}$};
\end{tikzpicture}
\caption*{(پ)}
\end{subfigure}%
\begin{subfigure}{0.5\textwidth}
\centering
\begin{tikzpicture}
\draw(2*\x,0) to [short,-o]++(\x+\x/2,0)node[below]{$B$};
\draw(2*\x,0) to [american voltage source,l={$\frac{124}{7} \, \si{\volt}$}]++(0,\y) to [resistor,l={$\frac{47}{7} \, \si{\kilo\ohm}$}]++(\x,0) to [short,-o]++(\x/2,0)node[above]{$A$};
\draw(3*\x,0)node[below]{$f$} to [resistor,*-*,l={$\SI{3}{\kilo\ohm}$}]++(0,\y)node[above]{$e$};
\draw(3*\x+\x/2,\y/2)node{$\begin{aligned} &+ \\ &v_0 \\ &- \end{aligned}$};
\end{tikzpicture}
\end{subfigure}%
\caption*{(ت)}
\caption{مثال \حوالہ{مثال_مسئلہ_تھونن_ت} حل کرتے ہوئے حاصل کئے گئے ادوار۔}
\label{شکل_مثال_تھونن_سے_نارٹن_پ}
\end{figure}

شکل \حوالہ{شکل_مثال_تھونن_سے_نارٹن_پ}-ب میں \عددی{\frac{124}{19}\,\si{\milli\ampere}} اور \عددی{\tfrac{19}{7}\,\si{\kilo\ohm}} نارٹن دور  کی جگہ ان کا مساوی تھونن دور  نسب کرنے سے شکل-پ حاصل ہوتا ہے جس میں \عددی{\tfrac{19}{7}\,\si{\kilo\ohm}} اور \عددی{\SI{4}{\kilo\ohm}} سلسلہ وار جڑے ہیں جن کا مساوی \عددی{\tfrac{47}{7}\,\si{\kilo\ohm}} ہے۔شکل \حوالہ{شکل_مثال_تھونن_سے_نارٹن_پ}-ت میں یہی مساوی مزاحمت دکھایا گیا ہے۔

شکل-ت میں \عددی{\SI{3}{\kilo\ohm}} بوجھ ہے جبکہ بقایا تھونن مساوی ہے۔تقسیم دباو سے بوجھ پر دباو درج ذیل حاصل ہوتا ہے۔
\begin{align*}
v_0=\frac{124}{7}\left(\frac{\SI{3}{\kilo\ohm}}{\SI{3}{\kilo\ohm}+\frac{47}{7}\,\si{\kilo\ohm}} \right)=\frac{93}{17} \, \si{\volt}
\end{align*}
\انتہا{مثال}
%======================
\ابتدا{مثال}\شناخت{مثال_مسئلہ_تھونن_ٹ}
گزشتہ مثال کا تھونن دور دوبارہ حاصل کرتے ہیں۔اس مرتبہ دور کو ایسی جگہوں پر ٹکڑے کرتے ہوئے حل کرتے ہیں کہ جواب جلد حاصل ہو۔شکل \حوالہ{شکل_مثال_تھونن_سے_نارٹن_ٹ} میں دور کو دوبارہ پیش کیا گیا ہے۔

\begin{figure}
\centering
\begin{subfigure}{1\textwidth}
\centering
\begin{tikzpicture}
\draw(0,0) to [short,-o]++(3*\x+\x/2,0)node[below]{$B$};
\draw(0,0) to [american voltage source,l={$\SI{8}{\volt}$}]++(0,\y) to [resistor,l={$\SI{2}{\kilo\ohm}$}]++(\x,0) to [resistor,l={$\SI{1}{\kilo\ohm}$}]++(\x,0) to [resistor,l={$\SI{4}{\kilo\ohm}$}]++(\x,0)to [short,-o]++(\x/2,0)node[above]{$A$};
\draw(\x,0)node[below]{$b$} to [resistor,*-*,l={$\SI{12}{\kilo\ohm}$}]++(0,\y)node[above]{$a$};
\draw(2*\x,0)node[below]{$d$} to [american current source,*-*,l={$\SI{4}{\milli\ampere}$}]++(0,\y)node[above]{$c$};
\draw(3*\x,0)node[below]{$f$} to [resistor,*-*,l={$\SI{3}{\kilo\ohm}$}]++(0,\y)node[above]{$e$};
\draw(3*\x+\x/2,\y/2)node{$\begin{aligned} &+ \\ &v_0 \\ &- \end{aligned}$};
\end{tikzpicture}
\caption*{(الف)}
\end{subfigure}
\begin{subfigure}{0.5\textwidth}
\centering
\begin{tikzpicture}
\draw(0,0) to [short,-o]++(2*\x,0)node[below]{$d$};
\draw(0,0) to [american voltage source,l={$\SI{8}{\volt}$}]++(0,\y) to [resistor,l={$\SI{2}{\kilo\ohm}$}]++(\x,0) to [resistor,-o,l={$\SI{1}{\kilo\ohm}$}]++(\x,0) node[above]{$c$};
\draw(\x,0)node[below]{$b$} to [resistor,*-*,l={$\SI{12}{\kilo\ohm}$}]++(0,\y)node[above]{$a$};
\draw(2*\x,\y/2)node{$\begin{aligned} &+ \\ &v_{\text{کھلا}} \\ &- \end{aligned}$};
\end{tikzpicture}
\caption*{(ب)}
\end{subfigure}%
\begin{subfigure}{0.5\textwidth}
\centering
\begin{tikzpicture}
\draw(0,0) to [short,-o]++(1.5*\x,0)node[below]{$d$};
\draw(0,0) to [american current source,l={$\frac{48}{19}\,\si{\milli\ampere}$}]++(0,\y) to [short,-o]++(\x+\x/2,0) node[above]{$c$};
\draw(\x,0) to [resistor,*-*,l={$\frac{19}{7}\si{\kilo\ohm}$}]++(0,\y);
\end{tikzpicture}
\caption*{(پ)}
\end{subfigure}
\begin{subfigure}{1\textwidth}
\centering
\begin{tikzpicture}
\draw(0,0) to [short,-o]++(3*\x+\x/2,0)node[below]{$B$};
\draw(0,0) to [american current source,l={$\frac{48}{19}\,\si{\milli\ampere}$}]++(0,\y) to [short]++(2*\x,0) to [resistor,l={$\SI{4}{\kilo\ohm}$}]++(\x,0)to [short,-o]++(\x/2,0)node[above]{$A$};
\draw(\x,0) to [resistor,*-*,l={$\frac{19}{7}\,\si{\kilo\ohm}$}]++(0,\y);
\draw(2*\x,0)node[below]{$d$} to [american current source,*-*,l={$\SI{4}{\milli\ampere}$}]++(0,\y)node[above]{$c$};
\draw(3*\x,0)node[below]{$f$} to [resistor,*-*,l={$\SI{3}{\kilo\ohm}$}]++(0,\y)node[above]{$e$};
\draw(3*\x+\x/2,\y/2)node{$\begin{aligned} &+ \\ &v_0 \\ &- \end{aligned}$};
\end{tikzpicture}
\caption*{(ت)}
\end{subfigure}
\begin{subfigure}{0.5\textwidth}
\centering
\begin{tikzpicture}
\draw(0,0) to [short,-o]++(2*\x+\x/2,0)node[below]{$B$};
\draw(0,0) to [american current source,l={${\tfrac{124}{19}\,\si{\milli\ampere}}$}]++(0,\yy) to [short]++(\x,0) to [resistor,l={$\SI{4}{\kilo\ohm}$}]++(\x,0)to [short,-o]++(\x/2,0)node[above]{$A$};
\draw(\x,0) to [resistor,*-*,l={$\frac{19}{7}\,\si{\kilo\ohm}$}]++(0,\yy);
\draw(2*\x,0)node[below]{$f$} to [resistor,i<^={$\frac{31}{17}\,\si{\milli\ampere}$},*-*,l={$\SI{3}{\kilo\ohm}$}]++(0,\yy)node[above]{$e$};
\draw(2*\x+\x/2,\yy/2)node{$\begin{aligned} &+ \\ &v_0 \\ &- \end{aligned}$};
\end{tikzpicture}
\caption*{(ٹ)}
\end{subfigure}%
\caption{مثال \حوالہ{مثال_مسئلہ_تھونن_ٹ} حل کرتے ہوئے حاصل کئے گئے ادوار۔}
\label{شکل_مثال_تھونن_سے_نارٹن_ٹ}
\end{figure}

حل:دور کو \عددی{cd} پر توڑ  کر شکل-ب میں دکھایا گیا ہے۔یوں \عددی{cd}  پر مساوی دور حاصل کیا جائے گا۔شکل-ب میں \عددی{v_{ab}} اور \عددی{v_{cd}} برابر ہیں۔یوں
\begin{align*}
v_{\text{کھلا}} = v_{cd}= v_{ab} =\frac{8\times 12000}{12000+2000}=\frac{48}{7} \, \si{\volt}
\end{align*}
ہو گا اور  \عددی{cd} سے دیکھتے ہوئے تھونن مزاحمت
\begin{align*}
\frac{2000\times 12000}{2000+12000}+1000=\frac{19}{7}\, \si{\kilo\ohm}
\end{align*}
ہو گا۔ان قیمتوں کو استعمال کرتے ہوئے مساوات  \حوالہ{مساوات_مسئلہ_تھونن_پ} سے
\begin{align*}
i_{\text{قصر}} =\frac{v_{\text{کھلا}}}{R_{\text{تھونن}}} =\frac{\frac{48}{7}}{\frac{19}{7}}=\frac{48}{19}\, \si{\milli\ampere}
\end{align*}
ملتا ہے۔یوں شکل-ب کا مساوی نارٹن دور شکل-پ حاصل ہوتا ہے جسے شکل-الف میں \عددی{cd} کے بائیں جانب دور کی جگہ نسب کرنے سے شکل-ت ملتا ہے۔شکل-ت میں دو عدد منبع رو متوازی جڑی ہیں جن کی جگہ ایک عدد
\begin{align*}
\frac{48}{19} \, \si{\milli\ampere}+\SI{4}{\milli\ampere}=\frac{124}{19} \, \si{\milli\ampere}
\end{align*}
 \عددی{\SI{8}{\milli\ampere}} کی منبع نسب کی جا سکتی ہے  جس سے شکل-ٹ حاصل ہوتا ہے۔شکل-ٹ میں سلسلہ وار جڑے \عددی{\SI{4}{\kilo\ohm}} اور \عددی{\SI{3}{\kilo\ohm}} از خود \عددی{\tfrac{19}{7} \, \si{\kilo\ohm}} کے متوازی ہے۔یوں سلسلہ وار مزاحمتوں میں رو کو تقسیم رو کے کلیے سے درج ذیل لکھا جا سکتا ہے
\begin{align*}
\frac{124}{19}\,\si{\milli\ampere}\left(\frac{\frac{19}{7} \, \si{\kilo\ohm}}{\frac{19}{7} \, \si{\kilo\ohm}+\SI{4}{\kilo\ohm}+\SI{3}{\kilo\ohm}} \right)=\frac{31}{17} \, \si{\milli\ampere}
\end{align*}
جسے شکل \حوالہ{شکل_مثال_تھونن_سے_نارٹن_ٹ}-ٹ میں دکھایا گیا ہے۔تین کلو  بوجھ پر دباو درج ذیل ہے۔
\begin{align*}
v_{\text{کھلا}}=\frac{31}{17} \, \si{\milli\ampere} \times \SI{3}{\kilo\ohm}=\frac{93}{17} \, \si{\volt}
\end{align*}
آخر میں مسئلہ اتنا سادہ بن چکا تھا کہ تقسیم رو اور اوہم کے قانون سے دباو حاصل کیا گیا۔آپ دیکھ سکتے ہیں کہ بوجھ پر دباو  جلد حاصل ہوا لہٰذا مسئلے کو دیکھ کر فیصلہ کریں کہ کہاں سے دور کو ٹکڑے کرتے ہوئے حل کرنا ہے۔
\انتہا{مثال}
%=========================
\ابتدا{مثال}\شناخت{مثال_مسئلہ_تھونن_ث}
شکل \حوالہ{شکل_مثال_تھونن_سے_نارٹن_ث}-الف میں مسئلہ نارٹن کی مدد سے \عددی{V_0} حاصل کریں۔

\begin{figure}
\centering
\begin{subfigure}{1\textwidth}
\centering
\begin{tikzpicture}
\draw(0,0) to [american voltage source,l={$\SI{10}{\volt}$}]++(0,\y) to [resistor,l_={$\SI{4}{\kilo\ohm}$}]++(\x,0) to [resistor,l={$\SI{1}{\kilo\ohm}$}]++(0,-\y) to [short]++(-\x,0);
\draw(\x,\y) to [resistor,*-*,l={$\SI{6}{\kilo\ohm}$}]++(0,\y) to [short]++(-\x,0) to [american current source,-*,l_={$\SI{2}{\milli\ampere}$}]++(0,-\y);
\draw(\x,0) to [short,*-o]++(\x+\x/2,0);
\draw(2*\x,0) to [resistor,*-*,l={$\SI{8}{\kilo\ohm}$}]++(0,2*\y) to [resistor,l={$\SI{2}{\kilo\ohm}$}]++(-\x,0);
\draw(2*\x,2*\y) to [short,*-o]++(\x/2,0);
\draw(2*\x+\x/2,\y)node{$\begin{aligned}&+ \\ \\ &V_0 \\ \\ &-  \end{aligned}$};
\end{tikzpicture}
\caption*{(الف)}
\end{subfigure}
\begin{subfigure}{0.5\textwidth}
\centering
\begin{tikzpicture}
\draw(0,0) to [short]++(0,\y) to [resistor,l_={$\SI{4}{\kilo\ohm}$}]++(\x,0) to [resistor,l={$\SI{1}{\kilo\ohm}$}]++(0,-\y) to [short]++(-\x,0);
\draw(\x,\y) to [resistor,*-,l={$\SI{6}{\kilo\ohm}$}]++(0,\y);
\draw(\x,0) to [short,*-o]++(\x,0);
\draw(\x,2*\y) to [resistor,-o,l={$\SI{2}{\kilo\ohm}$}]++(\x,0);
\draw[stealth-] (2*\x,\y) --++(\x/4,0) --++(0,-\y/4)node[below]{$R_{\text{تھونن}}$};
\end{tikzpicture}
\caption*{(ب)}
\end{subfigure}%
\begin{subfigure}{0.5\textwidth}
\centering
\begin{tikzpicture}
\draw(0,0) to [american voltage source,l={$\SI{10}{\volt}$}]++(0,\y) to [resistor,l_={$\SI{4}{\kilo\ohm}$}]++(\x,0) to [resistor,-*,l={$\SI{1}{\kilo\ohm}$}]++(0,-\y) to [short]++(-\x,0);
\draw(\x,\y) to [resistor,*-*,l_={$\SI{6}{\kilo\ohm}$}]++(0,\y) to [short]++(-\x,0) to [american current source,-*,l_={$\SI{2}{\milli\ampere}$}]++(0,-\y);
\draw(\x,0) to [short]++(\x,0) to [short,i<_={$I_{\text{قصر}}$}]++(0,2*\y);
\draw(\x,2*\y)  to [resistor,l={$\SI{2}{\kilo\ohm}$}]++(\x,0);
%loop currents
\draw[stealth-]([shift={(-150:\x/5.5)}]\x/2,\y/2-\dy) arc (-150:150:\x/5.5);
\draw(\x/2,\y/2-\dy)node{$i_1$};
\draw[stealth-]([shift={(-150:\x/5.5)}]\x/2,\y+\y/2) arc (-150:150:\x/5.5);
\draw(\x/2,\y+\y/2)node{$i_2$};
\draw[stealth-]([shift={(-150:\x/5.5)}]\x+\x/2,\y) arc (-150:150:\x/5.5);
\draw(\x+\x/2,\y)node{$i_3$};
\end{tikzpicture}
\caption*{(پ)}
\end{subfigure}
\begin{subfigure}{0.5\textwidth}
\centering
\begin{tikzpicture}
\draw(0,0) to [short,-o]++(\x+\x/4,0);
\draw(0,\y) to [short,-o]++(\x+\x/4,0);
\draw(0,\y) to [american current source,l={$\frac{29}{22}\,\si{\milli\ampere}$}]++(0,-\y);
\draw(\x,0) to [resistor,*-*,l_={$\frac{44}{5}\, \si{\kilo\ohm}$}]++(0,\y);
\end{tikzpicture}%
\caption*{(ت)}
\end{subfigure}%
\begin{subfigure}{0.5\textwidth}
\centering
\begin{tikzpicture}
\draw(0,0) to [short,-o]++(2*\x+\x/2,0);
\draw(0,\y) to [short,-o]++(2*\x+\x/2,0);
\draw(0,\y) to [american current source,l_={$\frac{29}{22}\,\si{\milli\ampere}$}]++(0,-\y);
\draw(\x,0) to [resistor,*-*,l={$\frac{44}{5}\, \si{\kilo\ohm}$}]++(0,\y);
\draw(2*\x,0) to [resistor,*-*,l={$\SI{8}{\kilo\ohm}$}]++(0,\y);
\draw(2*\x+\x/2,\y/2) node{$\begin{aligned} &+ \\ &V_0 \\ &- \end{aligned}$};
\end{tikzpicture}%
\caption*{(ٹ)}
\end{subfigure}%
\caption{مثال \حوالہ{مثال_مسئلہ_تھونن_ث} کا دور۔}
\label{شکل_مثال_تھونن_سے_نارٹن_ث}
\end{figure}

حل:آٹھ کلو اوہم کی مزاحمت کو بوجھ تصور کرتے ہوئے بقایا دور کا نارٹن مساوی حاصل کرتے ہیں۔بوجھ کو بقایا دور سے علیحدہ کرتے  ہوئے تھونن مزاحمت حاصل کرنے کی خاطر منبع رو کو کھلے دور اور منبع دباو کو قصر دور کرتے ہوئے شکل \حوالہ{شکل_مثال_تھونن_سے_نارٹن_ث}-ب  حاصل ہوتا ہے۔اس کو دیکھ کر 
\begin{align*}
R_{\text{تھونن}}=\frac{\SI{4}{\kilo\ohm} \times \SI{1}{\kilo\ohm}}{\SI{4}{\kilo\ohm} +\SI{1}{\kilo\ohm}}+\SI{6}{\kilo\ohm}+\SI{2}{\kilo\ohm}=\frac{44}{5}\, \si{\kilo\ohm}
\end{align*}
لکھا جا سکتا ہے۔

قصر دور رو یعنی نارٹن رو حاصل کرنے کی خاطر \عددی{\SI{8}{\kilo\ohm}} بوجھ کو قصر دور کرتے ہوئے شکل  \حوالہ{شکل_مثال_تھونن_سے_نارٹن_ث}-پ  حاصل کرتے ہیں جس سے درج ذیل مساوات لکھے جا سکتے ہیں۔
\begin{align*}
-10+(4000+1000)i_1-4000 i_2 - 1000 i_3&=0\\
i_2&=-0.002\\
-1000 i_1-6000i_2+(1000+6000+2000)i_3&=0
\end{align*} 
درج بالا مساوات کو حل کرنے سے
\begin{align*}
I_{\text{قصر}}=i_3=-\frac{29}{22} \, \si{\milli\ampere}
\end{align*}
حاصل ہوتا ہے۔تھونن مزاحمت اور نارٹن رو جانتے ہوئے   \عددی{\SI{8}{\kilo\ohm}} بوجھ کے علاوہ \حوالہ{شکل_مثال_تھونن_سے_نارٹن_ث}-الف کے بقایا دور کا مساوی نارٹن  دور شکل \حوالہ{شکل_مثال_تھونن_سے_نارٹن_ث}-ت میں دکھایا گیا ہے جہاں نارٹن رو کی قیمت منفی ہونے کی بنا پر اسے الٹ سمت میں دکھایا گیا ہے۔نارٹن مساوی دور کے ساتھ \عددی{\SI{8}{\kilo\ohm}} بوجھ جوڑنے سے شکل-ٹ حاصل ہوتی ہے۔اس شکل کو دیکھ کر درکار دباو درج ذیل لکھی جا سکتی ہے۔
\begin{align*}
V_0=-\frac{29}{22}\, \si{\milli\ampere} \left(\frac{\frac{44}{5}\, \si{\kilo\ohm}\times \SI{8}{\kilo\ohm}}{\frac{44}{5}\, \si{\kilo\ohm} + \SI{8}{\kilo\ohm}} \right)=-\frac{116}{21} \, \si{\volt}
\end{align*}
\انتہا{مثال}
%========================
\ابتدا{مشق}\شناخت{مشق_مسئلہ_تھونن_الف}
شکل \حوالہ{شکل_مسئلہ_مشق_تھونن_الف} میں دور دکھایا گیا ہے جسے  مسئلہ تھونن سے حل کرتے ہوئے \عددی{V_0} حاصل کریں۔
\begin{figure}
\centering
\begin{tikzpicture}
\draw(0,0) to [american current source,l={$\SI{5}{\milli\ampere}$}]++(0,2*\y) to [short]++(\x,0) to [resistor,l={$\SI{6}{\kilo\ohm}$}]++(\x,0) to [resistor,l_={$\SI{2}{\kilo\ohm}$}]++(0,-2*\y) to [short]++(-2*\x,0);
\draw(\x,0) to [american voltage source,*-,l={$\SI{10}{\volt}$}]++(0,\y) to [resistor,-*,l={$\SI{4}{\kilo\ohm}$}]++(0,\y);
\draw(2*\x,0) to [short,*-o,]++(\x/2,0)node[right]{$A$};
\draw(2*\x,2*\y) to [short,*-o,]++(\x/2,0)node[right]{$B$};
\draw(2*\x+\x/2,\y)node{$\begin{aligned} &+ \\ \\ &V_0 \\ \\&-  \end{aligned}$};
\end{tikzpicture}
\caption{مشق \حوالہ{مشق_مسئلہ_تھونن_الف} کا دور۔}
\label{شکل_مسئلہ_مشق_تھونن_الف}
\end{figure}
\انتہا{مشق}
%==========================
\ابتدا{مشق}\شناخت{مشق_مسئلہ_تھونن_ب}
شکل \حوالہ{شکل_مسئلہ_مشق_تھونن_ب} کو تھونن مساوی دور سے حل کرتے ہوئے \عددی{V_0} حاصل کریں۔
\begin{figure}
\centering
\begin{tikzpicture}
\draw(0,0) to [short,-o]++(3*\x+\x/2,0)node[right]{$B$};
\draw(0,0) to [american current source,l={$\SI{6}{\milli\ampere}$}]++(0,\y) to [short]++(\x,0) to [american voltage source,l={$\SI{4}{\volt}$}] ++(\x,0) to [resistor,l={$\SI{2}{\kilo\ohm}$}]++(\x,0) to [short]++(0,\y) to [american voltage source,l={$\SI{8}{\volt}$}]++(-\x,0) to [resistor,l={$\SI{1}{\kilo\ohm}$}]++(-\x,0) to [short]++(0,-\y);
\draw(\x,0) to [resistor,*-*,l={$\SI{8}{\kilo\ohm}$}]++(0,\y);
\draw(2*\x,0) to [resistor,*-*,l={$\SI{6}{\kilo\ohm}$}]++(0,\y);
\draw(3*\x,0) to [resistor,*-*,l={$\SI{4}{\kilo\ohm}$}]++(0,\y);
\draw(3*\x,\y) to [short,-o]++(\x/2,0)node[right]{$A$};
\draw(3*\x+\x/2,\y/2)node{$\begin{aligned} &+ \\ &V_0 \\ &-  \end{aligned}$};
\end{tikzpicture}
\caption{مشق \حوالہ{مشق_مسئلہ_تھونن_ب} کا دور۔}
\label{شکل_مسئلہ_مشق_تھونن_ب}
\end{figure}
\انتہا{مشق}
%===================
\ابتدا{مشق}\شناخت{مشق_مسئلہ_تھونن_پ}
مسئلہ نارٹن کی مدد سے شکل \حوالہ{شکل_مسئلہ_مشق_تھونن_پ} میں \عددی{I_0} حاصل کریں۔
\begin{figure}
\centering
\begin{tikzpicture}
\draw(0,0) to [american voltage source,l={$\SI{10}{\volt}$}]++(0,\y) to [resistor,l={$\SI{4}{\kilo\ohm}$}]++(0,\y) to [resistor,l={$\SI{2}{\kilo\ohm}$}]++(\x,0) to [resistor,l={$\SI{6}{\kilo\ohm}$}]++(\x,0) to [resistor,l={$\SI{2}{\kilo\ohm}$}]++(0,-\y) to [resistor,i={$I_0$},l={$\SI{4}{\kilo\ohm}$}]++(0,-\y) to [short]++(-2*\x,0);
\draw(2*\x,\y) to [american current source,*-*,l={$\SI{1}{\milli\ampere}$}]++(-\x,0) to [american current source,-*,l={$\SI{2}{\milli\ampere}$}]++(0,\y);
\draw(\x,0) to [american voltage source,*-,l={$\SI{8}{\volt}$}]++(0,\y);
\end{tikzpicture}
\caption{مشق \حوالہ{مشق_مسئلہ_تھونن_پ} کا دور۔}
\label{شکل_مسئلہ_مشق_تھونن_پ}
\end{figure}

حل:
\انتہا{مشق}
%==================

\حصہ{تابع منبع استعمال کرنے والے ادوار}
صرف تابع منبع استعمال کرنے والے ادوار کا تھونن یا نارٹن مساوی دور صرف \عددی{R_{\text{تھونن}}} ہوتا ہے۔ایسے ادوار میں چونکہ غیر تابع منبع نہیں پایا جاتا لہٰذا یہ از خود طاقت مہیا نہیں کر سکتے اور یوں ان سے تھونن دباو اور نارٹن رو صفر حاصل ہوتی ہیں۔تابع منبع استعمال کرنے والے  ادوار کا تھونن مزاحمت حاصل کرتے ہوئے اندرونی تابع منبع دباو کو قصر دور اور اندرونی تابع منبع رو کو کھلے دور نہیں کیا جاتا۔ان ادوار کے برقی سروں پر پیمائشی دباو \عددی{v_p} مہیا کرتے ہوئے انہیں سروں پر رو \عددی{i_p} حاصل کی جاتی ہے۔مزاحمت کی تعریف سے تھونن مزاحمت درج ذیل لکھی جاتی ہے۔
\begin{align}
R_{\text{تھونن}}=\frac{v_p}{i_p}
\end{align}
آئیں چند مثال دیکھیں۔

%=============
\ابتدا{مثال}\شناخت{مثال_مسئلہ_تھونن_تابع_منبع_الف}
شکل \حوالہ{شکل_مسئلہ_مثال_تھونن_تابع-منبع_الف}-الف میں تابع منبع دباو پایا جاتا ہے۔اس دور کا مساوی تھونن دور حاصل کریں۔

\begin{figure}
\centering
\begin{subfigure}{0.5\textwidth}
\centering
\begin{tikzpicture}
\draw(0,0) to [short,-o]++(2*\x+\x/4,0)node[right]{$B$};
\draw(0,0) to [resistor,l_={$\SI{4}{\kilo\ohm}$}]++(0,\y) to [resistor,l={$\SI{8}{\kilo\ohm}$}]++(\x,0) to [resistor,l={$\SI{2}{\kilo\ohm}$}]++(\x,0) to [short,-o]++(\x/4,0)node[right]{$A$};
\draw(0,\y) to [short,*-]++(0,\y/2) to [resistor,l={$\SI{10}{\kilo\ohm}$}]++(2*\x,0) to [short,-*]++(0,-\y/2);
\draw(\x,0) to [american controlled voltage source,*-*,l_={$3 V_x$}]++(0,\y);
\draw(2*\x,0) to [resistor,*-*,l_={$\SI{6}{\kilo\ohm}$}]++(0,\y);
\draw(\x/2,\y-\dy)node[below]{$- \, V_x \, +$};
\end{tikzpicture}
\caption*{(الف)}
\end{subfigure}%
\begin{subfigure}{0.5\textwidth}
\centering
\begin{tikzpicture}
\draw(0,0) to [short]++(2*\x+\x/2,0) to [american voltage source]++(0,\y);
\draw(2*\x+\x/2,\y/4-\dy)node[right]{$\SI{1}{\volt}$};
\draw(0,0) to [resistor,l_={$\SI{4}{\kilo\ohm}$}]++(0,\y)node[left]{$V_1$} to [resistor,l={$\SI{8}{\kilo\ohm}$}]++(\x,0) to [resistor,i<={$i_b$},l={$\SI{2}{\kilo\ohm}$}]++(\x,0) to [short,i<={$i_p$}]++(\x/2,0);
\draw(0,\y) to [short,*-]++(0,\y/2) to [resistor,i<={$i_a$},l={$\SI{10}{\kilo\ohm}$}]++(2*\x,0) to [short,-*]++(0,-\y/2);
\draw(\x,0) to [american controlled voltage source,*-*]++(0,\y);
\draw(\x-\dx,\y/4)node[left]{$3 V_x$};
\draw(2*\x,0) to [resistor,i<_={$i_c$},*-*,l={$\SI{6}{\kilo\ohm}$}]++(0,\y);
\draw(\x/2,\y-\dy)node[below]{$- \, V_x \, +$};
\draw(\x,0)node[ground]{};
\end{tikzpicture}
\caption*{(ب)}
\end{subfigure}%
\caption{مثال \حوالہ{مثال_مسئلہ_تھونن_تابع_منبع_الف} کا دور۔}
\label{شکل_مسئلہ_مثال_تھونن_تابع-منبع_الف}
\end{figure}

حل:شکل \حوالہ{شکل_مسئلہ_مثال_تھونن_تابع-منبع_الف}-ب میں برقی سروں \عددی{AB} پر پیمائشی دباو لاگو کرتے ہوئے \عددی{i_p} حاصل کرتے ہیں۔پیمائشی دباو کی قیمت کچھ بھی چننی جا سکتی ہے۔ہم نے  \عددی{v_p=\SI{1}{\volt}} چننا ہے۔نچلی جوڑ کو زمین چنتے ہوئے  درج ذیل مساوات لکھے جا سکتے ہیں
\begin{align*}
\frac{V_1}{\SI{4}{\kilo\ohm}}+\frac{V_1-3V_x}{\SI{8}{\kilo\ohm}}+\frac{V_1-1}{\SI{10}{\kilo\ohm}}&=0\\
V_x&=3V_x-V_1
\end{align*}
جن سے
\begin{align*}
V_1&=\frac{8}{23}\,\si{\volt}\\
V_x&=\frac{4}{23}\, \si{\volt}
\end{align*}
حاصل ہوتے ہیں لہٰذا دور کو دیکھتے ہوئے کرخوف قانون رو سے درج ذیل لکھا جا سکتا ہے۔
\begin{align*}
i_p&=i_a+i_b+i_c\\
&=\frac{1-\frac{8}{23}}{10000}+\frac{1-3\times \frac{4}{23}}{2000}+\frac{1}{6000}\\
&=\frac{65}{138}\, \si{\milli\ampere}
\end{align*}
تھونن مزاحمت درج ذیل ہو گا۔
\begin{align*}
R_{\text{تھونن}}=\frac{v_p}{i_p}=\frac{138}{65} \, \si{\kilo\ohm}
\end{align*}
\انتہا{مثال}
%===================
\ابتدا{مثال}\شناخت{مثال_مسئلہ_تھونن_تابع_منبع_ب}
شکل \حوالہ{شکل_مسئلہ_مثال_تھونن_تابع-منبع_ب}-الف کا مساوی تھونن دور حاصل کریں۔

\begin{figure}
\centering
\begin{subfigure}{1\textwidth}
\centering
\begin{tikzpicture}
\draw(0,0) to [short,-o]++(2*\x+\x/2,0);
\draw(0,\y) to [american controlled current source,l_={$2 I_x$}]++(0,-\y);
\draw(\x,0)node[ground]{} to [resistor,i<_={$I_x$},*-*,l={$\SI{5}{\kilo\ohm}$}]++(0,\y)node[above]{$V_1$};
\draw(2*\x,0) to [resistor,*-*,l={$\SI{4}{\kilo\ohm}$}]++(0,\y);
\draw(0,\y) to [short]++(\x,0)  to [resistor,l={$\SI{1}{\kilo\ohm}$}]++(\x,0) to [short,-o]++(\x/2,0);
\draw(2*\x+\x/2,\y/2)node{$\begin{aligned} &+ \\ & V_{\text{تھونن}} \\ &-  \end{aligned}$};
\end{tikzpicture}
\caption*{(الف)}
\end{subfigure}
\begin{subfigure}{1\textwidth}
\centering
\begin{tikzpicture}
\draw(0,0) to [short,-o]++(2*\x+\x/2,0) to [short] ++ (\x/2,0) to [american current source,l_={$\SI{1}{\milli\ampere}$}]++(0,\y) to [short,-o]++(-\x/2,0);
\draw(0,\y) to [american controlled current source,l_={$2 I_x$}]++(0,-\y);
\draw(\x,0)node[ground]{} to [resistor,i<_={$I_x$},*-*,l={$\SI{5}{\kilo\ohm}$}]++(0,\y)node[above]{$V_1$};
\draw(2*\x,0) to [resistor,*-*,l={$\SI{4}{\kilo\ohm}$}]++(0,\y);
\draw(0,\y) to [short]++(\x,0)  to [resistor,l={$\SI{1}{\kilo\ohm}$}]++(\x,0)node[above]{$V_2$} to [short,-o]++(\x/2,0);
\draw(2*\x+\x/2,\y/2)node{$\begin{aligned} &+ \\ & v_p\\ &-  \end{aligned}$};
\end{tikzpicture}
\caption*{(ب)}
\end{subfigure}
\caption{مثال \حوالہ{مثال_مسئلہ_تھونن_تابع_منبع_ب} کا دور۔}
\label{شکل_مسئلہ_مثال_تھونن_تابع-منبع_ب}
\end{figure}

حل:اس دور میں صرف تابع منبع پایا جاتا ہے اور ہم توقع کرتے ہیں کہ نارٹن رو یا تھونن دباو صفر حاصل ہو گا۔ آئیں دیکھیں کہ آیا ہماری توقع درست ہے۔شکل \حوالہ{شکل_مسئلہ_مثال_تھونن_تابع-منبع_ب}-الف میں نچلے جوڑ کو زمین تصور کرتے  ہوئے جوڑ \عددی{V_1} پر کرخوف قانون رو سے درج ذیل لکھا جا سکتا ہے۔
\begin{align*}
2I_x+\frac{V_1}{5000}+\frac{V_1}{1000+4000}=0
\end{align*}
جس میں
\begin{align*}
I_x=\frac{V_1}{5000}
\end{align*}
پُر کرنے سے 
\begin{align*}
\frac{2 V_1}{5000}+\frac{V_1}{5000}+\frac{V_1}{1000+4000}=0
\end{align*}
یعنی
\begin{align*}
V_1=\SI{0}{\volt}
\end{align*}
حاصل ہوتا ہے۔تقسیم دباو کے کلیے سے 
\begin{align*}
V_{\text{تھونن}}=\left(\frac{1000}{1000+4000}\right) V_1=\SI{0}{\volt}
\end{align*}
حاصل ہوتا ہے۔چونکہ تھونن دباو صفر ہے لہٰذا مسئلہ تبادلہ منبع کے تحت نارٹن رو بھی صفر ہو گی۔


دور کی تھونن مزاحمت حاصل کرنے کی خاطر برقی سروں پر بیرونی منبع نسب کرنا ہو گا۔شکل \حوالہ{شکل_مسئلہ_مثال_تھونن_تابع-منبع_ب}-ب میں برقی سروں پر
 \عددی{i_p=\SI{1}{\milli\ampere}} کا پیمائشی رو نسب کیا گیا ہے۔برقی سروں پر پیمائشی دباو \عددی{v_p} جانتے ہوئے  تھونن مزاحمت حاصل کی جا سکتی ہے۔

شکل \حوالہ{شکل_مسئلہ_مثال_تھونن_تابع-منبع_ب}-ب کے بالائی دو جوڑ پر کرخوف مساوات رو لکھتے ہیں۔
\begin{align*}
2I_x +\frac{V_1}{5000}+\frac{V_1-V_2}{1000}&=0\\
\frac{V_2-V_1}{1000}+\frac{V_2}{4000}-0.001&=0
\end{align*}
ان میں \عددی{I_x=\tfrac{V_1}{5000}} پُر کرتے اور ترتیب دیتے ہوئے دوبارہ لکھتے ہیں
\begin{align*}
8V_1-5V_2&=0\\
4V_1-5V_2&=-4
\end{align*}
جس سے \عددی{V_2=\tfrac{8}{5} \, \si{\volt}} حاصل ہوتا ہے لہٰذا
\begin{align*}
v_p=\frac{8}{5} \, \si{\volt}
\end{align*}
ہو گا۔یوں تھونن مزاحمت درج ذیل ہو گا۔
\begin{align*}
R_{\text{تھونن}}=\frac{v_p}{i_p}=\frac{8}{5}\,\si{\kilo\ohm}
\end{align*}
\انتہا{مثال}
%======================
\ابتدا{مثال}\شناخت{مثال_مسئلہ_تھونن_تابع_منبع_پ}
گزشتہ مثال کے دور کو سلسلہ وار جڑے بیرونی منبع  اور مزاحمت سے طاقت مہیا کی جاتی ہے۔شکل \حوالہ{شکل_مسئلہ_مثال_تھونن_تابع-منبع_پ} میں اسے دکھایا گیا ہے۔برقی سروں پر دباو \عددی{v_0} اور رو \عددی{i_0} حاصل کریں۔اب گزشتہ مثال کے دور کی جگہ اس کا مساوی تھونن دور نسب کرتے ہوئے دوبارہ حل کریں۔
\begin{figure}
\centering
\begin{subfigure}{1\textwidth}
\centering
\begin{tikzpicture}
\draw(0,0) to [short,-o]++(2*\x+\x/2,0) to [short] ++ (\x,0) to [american voltage source,l_={$\SI{6}{\volt}$}]++(0,\y) to [resistor,l={$\SI{2}{\kilo\ohm}$},-o]++(-\x,0);
\draw(0,\y) to [american controlled current source,l_={$2 I_x$}]++(0,-\y);
\draw(\x,0)node[ground]{} to [resistor,i<_={$I_x$},*-*,l={$\SI{5}{\kilo\ohm}$}]++(0,\y)node[above]{$V_1$};
\draw(2*\x,0) to [resistor,*-*,l={$\SI{4}{\kilo\ohm}$}]++(0,\y);
\draw(0,\y) to [short]++(\x,0)  to [resistor,l={$\SI{1}{\kilo\ohm}$}]++(\x,0)node[above]{$V_2$} to [short,i<={$i_0$},-o]++(\x/2,0);
\draw(2*\x+\x/2,\y/2)node{$\begin{aligned} &+ \\ & v_0\\ &-  \end{aligned}$};
\end{tikzpicture}
\caption*{(الف)}
\end{subfigure}
\begin{subfigure}{1\textwidth}
\centering
\begin{tikzpicture}
\draw(0,0) to [short,-o]++(\x/2,0) to [short]++(\x,0) to [american voltage source,l_={$\SI{6}{\volt}$}]++(0,\y) to [resistor,-o,l={$\SI{2}{\kilo\ohm}$}]++(-\x,0) to [short,i_={$i_0$}]++(-\x/2,0) to [resistor,l_={$\frac{8}{5} \, \si{\kilo\ohm}$}]++(0,-\y);
\draw(\x/2,\y/2)node{$\begin{aligned} &+ \\ &v_0 \\ &- \end{aligned}$};
\end{tikzpicture}
\caption*{(ب)}
\end{subfigure}
\caption{مثال \حوالہ{مثال_مسئلہ_تھونن_تابع_منبع_پ} کا دور۔}
\label{شکل_مسئلہ_مثال_تھونن_تابع-منبع_پ}
\end{figure}

حل: بالائی جوڑوں پر کرخوف مساوات رو لکھتے ہیں
\begin{align*}
2I_x +\frac{V_1}{5000}+\frac{V_1-V_2}{1000}&=0\\
\frac{V_2-V_1}{1000}+\frac{V_2}{4000}+\frac{V_2-6}{2000}&=0
\end{align*}
جن میں \عددی{I_x=\tfrac{V_1}{5000}} پر کرتے ہوئے اور ترتیب دیتے ہوئے درج ذیل حاصل ہوتا ہے۔
\begin{align*}
8V_1-5V_2&=0\\
-4V_1+7V_2&=12
\end{align*}
انہیں حل کرتے ہوئے
\begin{align*}
V_1&=\frac{5}{3} \,\si{\volt}\\
V_2&=\frac{8}{3} \,\si{\volt}\\
\end{align*}
حاصل ہوتے  ہیں لہٰذا 
\begin{align*}
v_0&=V_2=\frac{8}{3} \,\si{\volt}\\
i_0&=\frac{6-\frac{8}{3}}{2000}=\frac{5}{3}\,\si{\milli\ampere}
\end{align*}
ہوں گے۔

آئیں اب تھونن مساوی دور کی مدد سے اسی کو دوبارہ حل کریں۔گزشتہ مثال میں \عددی{v_{\text{تھونن}}=\SI{0}{\volt}} اور \عددی{R_{\text{تھونن}}=\frac{8}{5} \, \si{\kilo\ohm}} حاصل کئے گئے۔تھونن مساوی دور استعمال کرتے ہوئے شکل \حوالہ{شکل_مسئلہ_مثال_تھونن_تابع-منبع_پ}-ب حاصل ہوتا ہے جہاں قانون اوہم کی مدد سے
\begin{align*}
i_0&=\frac{\SI{6}{\volt}}{\frac{8}{5} \, \si{\kilo\ohm}+\SI{2}{\kilo\ohm}}=\frac{5}{3}\,\si{\milli\ampere}
\end{align*}
اور تقسیم دباو کے کلیے سے
\begin{align*}
v_0=6\left(\frac{\frac{8}{5}\,\si{\kilo\ohm}}{\frac{8}{5}\,\si{\kilo\ohm}+\SI{2}{\kilo\ohm}}\right)=\frac{8}{3} \, \si{\volt}
\end{align*}
حاصل ہوتے ہیں۔آپ دیکھ سکتے ہیں کہ بیرونی برقی سروں پر اصل دور اور تھونن مساوی دور بالکل یکساں دکھائی دیتے ہیں۔آپ نے یہ بھی دیکھ لیا ہو گا کہ تھونن دور استعمال کرتے ہوئے جوابات نہایت آسانی سے حاصل ہوتے ہیں۔
\انتہا{مثال}
%========================

\حصہ{تابع منبع اور غیر تابع منبع دونوں استعمال کرنے والے ادوار}
ان ادوار میں \عددی{v_{\text{کھلا}}} اور \عددی{i_{\text{قصر}}} حاصل کرتے ہوئے \عددی{R_{\text{تھونن}}} حاصل کیا جاتا ہے۔یاد رہے کہ دور کو دو ٹکڑوں میں تقسیم کرتے ہوئے تابع منبع اور اس کا قابو متغیر علیحدہ نہیں کئے جا سکتے ہیں۔

آئیں چند مثال دیکھیں۔

%=======================
\ابتدا{مثال}\شناخت{مثال_مسئلہ_تابع_غیر_تابع_الف}
شکل \حوالہ{شکل_مسئلہ_تابع_غیر_تابع_مثال_الف} میں \عددی{V_0} کو  مسئلہ تھونن سے حاصل کریں۔

\begin{figure}
\centering
\begin{subfigure}{1\textwidth}
\centering
\begin{tikzpicture}
\draw(0,0) to [short]++(3*\x,0) to [short,*-o]++(\x/2,0);
\draw(0,\y) to [short]++(\x,0) to [american voltage source,l={$\SI{6}{\volt}$}]++(\x,0) to [resistor,l={$\SI{2}{\kilo\ohm}$}]++(\x,0) to [resistor,l_={$\SI{6}{\kilo\ohm}$}]++(0,-\y);
\draw(3*\x,\y) to [short,*-o]++(\x/2,0);
\draw(0,\y) to [american controlled current source,l_={$4I_x$}]++(0,-\y);
\draw(\x,0) to [resistor,*-*,l={$\SI{4}{\kilo\ohm}$}]++(0,\y);
\draw(2*\x,0) to [resistor,i<^={$I_x$},*-*,l={$\SI{8}{\kilo\ohm}$}]++(0,\y);
\draw(3*\x+\x/2,\y/2)node{$\begin{aligned}&+ \\ &V_0 \\ &-  \end{aligned}$};
\end{tikzpicture}
\caption*{(الف)}
\end{subfigure}
\begin{subfigure}{1\textwidth}
\centering
\begin{tikzpicture}
\draw(0,0) to [short]++(2*\x,0) to [short,*-o]++(\x/2,0);
\draw(0,\y) to [short]++(\x,0) to [american voltage source,l={$\SI{6}{\volt}$}]++(\x,0);
\draw(2*\x,\y) to [short,*-o]++(\x/2,0);
\draw(0,\y) to [american controlled current source,l_={$4I'_x$}]++(0,-\y);
\draw(\x,0) to [resistor,*-*,l={$\SI{4}{\kilo\ohm}$}]++(0,\y);
\draw(2*\x,0) to [resistor,i<^={$I'_x$},*-*,l={$\SI{8}{\kilo\ohm}$}]++(0,\y);
\draw(2*\x+\x/2,\y/2)node{$\begin{aligned}&+ \\ &V_{\text{کھلا}} \\ &-  \end{aligned}$};
\end{tikzpicture}
\caption*{(ب)}
\end{subfigure}
\begin{subfigure}{0.5\textwidth}
\centering
\begin{tikzpicture}
\draw(0,0) to [short]++(2*\x,0) to [short,*-o]++(\x/2,0) to [short]++(\x/4,0) to [short,i<={$I_{\text{قصر}}$}]++(0,\y) to [short,-o]++(-\x/4,0);
\draw(0,\y) to [short]++(\x,0) to [american voltage source,l={$\SI{6}{\volt}$}]++(\x,0);
\draw(2*\x,\y) to [short,*-o]++(\x/2,0);
\draw(0,\y) to [american controlled current source,l_={$4I''_x$}]++(0,-\y);
\draw(\x,0) to [resistor,*-*,l={$\SI{4}{\kilo\ohm}$}]++(0,\y);
\draw(2*\x,0) to [resistor,i<^={$I''_x$},*-*,l={$\SI{8}{\kilo\ohm}$}]++(0,\y);
\end{tikzpicture}
\caption*{(پ)}
\end{subfigure}%
\begin{subfigure}{0.5\textwidth}
\centering
\begin{tikzpicture}
\draw(\x,0) to [short]++(\x,0) to [short,-o]++(\x/2,0) to [short]++(\x/4,0) to [short,i<={$I_{\text{قصر}}$}]++(0,\y) to [short,-o]++(-\x/4,0);
\draw(\x,\y)to [american voltage source,l={$\SI{6}{\volt}$}]++(\x,0);
\draw(2*\x,\y) to [short,-o]++(\x/2,0);
\draw(\x,0) to [resistor,l={$\SI{4}{\kilo\ohm}$}]++(0,\y);
\end{tikzpicture}
\caption*{(ت)}
\end{subfigure}
\begin{subfigure}{0.5\textwidth}
\centering
\begin{tikzpicture}
\draw(0,0) to [short,-o]++(2*\x,0) to [short,*-o]++(\x/2,0);
\draw(0,0) to [american voltage source,l={$\frac{12}{7} \, \si{\volt}$}]++(0,\y) to [resistor,l={$\frac{8}{7} \, \si{\kilo\ohm}$}]++(\x,0) to [resistor,l={$\SI{2}{\kilo\ohm}$}]++(\x,0) to [resistor,l_={$\SI{6}{\kilo\ohm}$}]++(0,-\y);
\draw(2*\x,\y) to [short,*-o]++(\x/2,0);
\draw(2*\x+\x/2,\y/2)node{$\begin{aligned}&+ \\ &V_0 \\ &-  \end{aligned}$};
\end{tikzpicture}
\caption*{(ٹ)}
\end{subfigure}%
\caption{مثال \حوالہ{مثال_مسئلہ_تابع_غیر_تابع_الف} کا دور۔}
\label{شکل_مسئلہ_تابع_غیر_تابع_مثال_الف}
\end{figure}

حل:دور کے ٹکڑے کرتے ہوئے تھونن مساوی دور حاصل کرتے ہیں۔یاد رہے کہ تابع منبع  اور اس کا قابو متغیر کو علیحدہ نہیں کیا جا سکتا ہے لہٰذا قابو منبع رو اور \عددی{\SI{8}{\kilo\ohm}} مزاحمت کو علیحدہ نہیں کیا جا سکتا ہے۔یوں دور کو ٹکڑے کرتے ہوئے، برقی سروں سے دور ترین نقطے سے شروع کرتے ہوئے  کم از کم اتنے اجزاء شامل کئے جائیں گے کہ قابو منبع سے لے کر تابع متغیر تک تمام اس میں موجود ہوں۔شکل \حوالہ{شکل_مسئلہ_تابع_غیر_تابع_مثال_الف}-ب میں ایسا ٹکڑا دکھایا گیا ہے جہاں قابو متغیر کو \عددی{I'_x} کہا گیا ہے۔بالائی  مخلوط جوڑ پر کرخوف مساوات رو لکھتے ہیں
\begin{align*}
4 I'_x+\frac{V_{\text{کھلا}}-6}{4000}+\frac{V_{\text{کھلا}}}{8000}=0
\end{align*}
جس میں \عددی{I'_x=\tfrac{V_{\text{کھلا}}}{8000}} پُر کرتے ہوئے حل کرنے سے 
\begin{align*}
V_{\text{کھلا}}=\frac{12}{7} \, \si{\volt}
\end{align*}
حاصل ہوتا ہے۔

شکل \حوالہ{شکل_مسئلہ_تابع_غیر_تابع_مثال_الف}-ب میں قصر دور رو حاصل کرنے کی خاطر اس کے برقی سروں کو قصر دور کرتے ہوئے شکل-پ حاصل کیا جاتا ہے جس میں \عددی{I''_x=0} کی بنا پر قابو منبع کی رو بھی صفر ہو گی۔ان حقائق کو مد نظر رکھتے ہوئے شکل \حوالہ{شکل_مسئلہ_تابع_غیر_تابع_مثال_الف}-ت حاصل ہوتا ہے جسے دیکھ کر 
\begin{align*}
I_{\text{قصر}}=\frac{6}{4000}=\frac{3}{2}\, \si{\milli\ampere}
\end{align*}
لکھا جا سکتا ہے۔تھونن دباو اور نارٹن رو استعمال کرتے ہوئے  تھونن مزاحمت درج ذیل حاصل ہوتا ہے۔
\begin{align*}
R_{\text{تھونن}}=\frac{V_{\text{کھلا}}}{I_{\text{قصر}}}  = \frac{\frac{12}{7} \, \si{\volt}}{\frac{3}{2} \, \si{\milli\ampere}}=\frac{8}{7} \, \si{\kilo\ohm}
\end{align*}
شکل \حوالہ{شکل_مسئلہ_تابع_غیر_تابع_مثال_الف}-ب کی جگہ تھونن مساوی دور نسب کرتے ہوئے شکل-الف سے شکل-ٹ حاصل ہوتا ہے جس سے تقسیم دباو کے کلیے سے 
\begin{align*}
V_0=\frac{\frac{12}{7} \, \si{\volt}}{\frac{8}{7} \, \si{\kilo\ohm}+\SI{2}{\kilo\ohm}+\SI{6}{\kilo\ohm}}=\frac{3}{16} \, \si{\volt}
\end{align*}
لکھا جا سکتا ہے۔ 
\انتہا{مثال}
%========================
\ابتدا{مثال}\شناخت{مثال_مسئلہ_تابع_غیر_تابع_ب}
شکل \حوالہ{شکل_مسئلہ_تابع_غیر_تابع_مثال_ب} میں مسئلہ تھونن کی مدد سے \عددی{V_0} حاصل کریں۔
\begin{figure}
\centering
\begin{subfigure}{0.5\textwidth}
\centering
\begin{tikzpicture}
\draw(0,0) to [short,-o]++(2*\x+\x/2,0);
\draw(0,2*\y) to [short,-o]++(2*\x+\x/2,0);
\draw(0,0) to [american controlled current source,l={$\frac{V_x}{1000}$}]++(0,\y);
\draw(0,2*\y) to [american current source,l_={$\SI{4}{\milli\ampere}$}]++(0,-\y);
\draw(\x,0) to [american voltage source,*-,l_={$\SI{10}{\volt}$}]++(0,\y) to [resistor,-*,l_={$\SI{6}{\kilo\ohm}$}]++(0,\y);
\draw(2*\x,0) to [resistor,*-*,l={$\SI{8}{\kilo\ohm}$}]++(0,2*\y);
\draw(0,\y) to [resistor,*-*,l={$\SI{2}{\kilo\ohm}$}]++(\x,0);
\draw(\x/2,\y-\dy)node[below]{$+ \, V_x \, -$};
\draw(2*\x+\x/2,\y) node{$\begin{aligned}&+ \\ \\ \\  &V_0 \\ \\  \\ &-  \end{aligned}$};
\end{tikzpicture}
\caption*{(الف)}
\end{subfigure}%
\begin{subfigure}{0.5\textwidth}
\centering
\begin{tikzpicture}
\draw(0,0) to [short,-o]++(\x+\x/2,0);
\draw(0,2*\y) to [short,-o]++(\x+\x/2,0);
\draw(0,0) to [american controlled current source,l={$\frac{V'_x}{1000}$}]++(0,\y);
\draw(0,2*\y) to [american current source,l_={$\SI{4}{\milli\ampere}$}]++(0,-\y);
\draw(\x,0) to [american voltage source,*-,l_={$\SI{10}{\volt}$}]++(0,\y) to [resistor,-*,l_={$\SI{6}{\kilo\ohm}$}]++(0,\y);
\draw(0,\y) to [resistor,*-*,l={$\SI{2}{\kilo\ohm}$}]++(\x,0);
\draw(\x/2,\y-\dy)node[below]{$+ \, V'_x \, -$};
\draw(\x+\x/2,\y) node{$\begin{aligned}&+ \\ \\ \\ &V_{\text{کھلا}} \\ \\  \\ &-  \end{aligned}$};
%loop currents
\draw[stealth-]([shift={(-150:\x/5.5)}]\x/2,\y/2-\dy) arc (-150:150:\x/5.5);
\draw(\x/2,\y/2-\dy)node{$i_1$};
\draw[stealth-]([shift={(-150:\x/5.5)}]\x/2,\y+\y/2+\dy) arc (-150:150:\x/5.5);
\draw(\x/2,\y+\y/2+\dy)node{$i_2$};
\end{tikzpicture}
\caption*{(ب)}
\end{subfigure}
\begin{subfigure}{0.5\textwidth}
\centering
\begin{tikzpicture}
\draw(0,0) to [short,-o]++(\x+\x/2,0);
\draw(0,2*\y) to [short,-o]++(\x+\x/2,0) to [short]++(\x/4,0) to [short,i={$I_{\text{قصر}}$}]++(0,-2*\y) to [short,-o]++(-\x/4,0);
\draw(0,0) to [american controlled current source,l={$\frac{V''_x}{1000}$}]++(0,\y);
\draw(0,2*\y) to [american current source,l_={$\SI{4}{\milli\ampere}$}]++(0,-\y);
\draw(\x,0) to [american voltage source,*-,l_={$\SI{10}{\volt}$}]++(0,\y) to [resistor,i={$I_a$},-*,l_={$\SI{6}{\kilo\ohm}$}]++(0,\y);
\draw(0,\y) to [resistor,*-*,l={$\SI{2}{\kilo\ohm}$}]++(\x,0);
\draw(\x/2,\y-\dy)node[below]{$+ \, V''_x \, -$};
\end{tikzpicture}
\caption*{(پ)}
\end{subfigure}
\begin{subfigure}{0.5\textwidth}
\centering
\begin{tikzpicture}
\draw(0,0) to [short]++(3*\x,0);
\draw(0,\y) to [short]++(\x,0) to [resistor,l={$\SI{2}{\kilo\ohm}$}]++(\x,0) to [short]++(\x,0);
\draw(0,0) to [american current source,l={$\SI{4}{\milli\ampere}$}]++(0,\y);
\draw(\x,0) to [american controlled current source,*-*,l={$\frac{V''_x}{1000}$}]++(0,\y);
\draw(2*\x,0) to [american voltage source,*-*,l={$\SI{10}{\volt}$}]++(0,\y);
\draw(3*\x,0) to [resistor,i<={$I_a$},l={$\SI{6}{\kilo\ohm}$}]++(0,\y);
\draw(\x+\x/2,\y-\dy)node[below]{$+ \, V''_x \, -$};
\end{tikzpicture}
\caption*{(ت)}
\end{subfigure}%
\begin{subfigure}{0.5\textwidth}
\centering
\begin{tikzpicture}
\draw(0,0) to [short,-o]++(\x+\x/2,0);
\draw(0,\y) to [resistor,l={$\SI{6}{\kilo\ohm}$}]++(\x,0) to [short,-o]++(\x/2,0);
\draw(0,\y) to [american voltage source,l_={$\SI{14}{\volt}$}]++(0,-\y);
\draw(\x,0) to [resistor,*-*,l={$\SI{8}{\kilo\ohm}$}]++(0,\y);
\draw(\x+\x/2,\y/2) node{$\begin{aligned} &+ \\& V_0 \\ &- \end{aligned}$};
\end{tikzpicture}
\caption*{(ٹ)}
\end{subfigure}%
\caption{مثال \حوالہ{مثال_مسئلہ_تابع_غیر_تابع_ب} کا دور۔}
\label{شکل_مسئلہ_تابع_غیر_تابع_مثال_ب}
\end{figure}

حل:خارجی \عددی{\SI{8}{\kilo\ohm}} مزاحمت کو بوجھ تصور کرتے ہوئے بقایا دور جسے شکل \حوالہ{شکل_مسئلہ_تابع_غیر_تابع_مثال_ب}-ب میں دکھایا گیا ہے کا تھونن مساوی حاصل کرتے ہیں۔بالائی خانے کو دیکھ کر
\begin{align*}
i_2&=-0.004
\end{align*}
لہٰذا
\begin{align*}
V_{\text{کھلا}}=6000 i_2+10= 6000(-0.004)+10=\SI{-14}{\volt}
\end{align*}
لکھا جا سکتا ہے۔تابع منبع کی موجودگی کی بنا پر تھونن مزاحمت حاصل کرنے کی خاطر قصر دور رو درکار ہو گی۔شکل \حوالہ{شکل_مسئلہ_تابع_غیر_تابع_مثال_ب}-ب کے برقی سر قصر دور کرتے ہوئے  شکل-پ حاصل ہوتا ہے جسے شکل-ت کی طرز پر بنایا جا سکتا ہے۔شکل-ت میں \عددی{I_{\text{قصر}}} کی نشاندہی کرنا قدر مشکل کام ہے البتہ اس سے \عددی{I_a} نہایت آسانی سے
\begin{align*}
I_a=\frac{\SI{10}{\volt}}{\SI{6}{\kilo\ohm}}=\frac{5}{3}\, \si{\milli\ampere}
\end{align*}
 حاصل ہوتی ہے۔شکل \حوالہ{شکل_مسئلہ_تابع_غیر_تابع_مثال_ب}-پ سے درج ذیل لکھا جا سکتا ہے۔
\begin{align*}
I_{\text{قصر}}=I_a-\SI{4}{\milli\ampere}=\frac{5}{3}\, \si{\milli\ampere}-\SI{4}{\milli\ampere}=-\frac{7}{3}\, \si{\milli\ampere}
\end{align*}
ان معلومات کو استعمال کرتے ہوئے
\begin{align*}
R_{\text{تھونن}}=\frac{V_{\text{کھلا}}}{I_{\text{قصر}}}=\frac{\SI{-14}{\volt}}{-\frac{7}{3}\,\si{\milli\ampere}}=\SI{6}{\kilo\ohm}
\end{align*}
حاصل ہوتا ہے جن کی مدد سے شکل-ب کا تھونن مساوی دور حاصل کیا جا سکتا ہے جسے شکل-الف میں پُر کرنے سے شکل-ٹ حاصل ہوتا ہے۔شکل-ٹ سے تقسیم دباو کے کلیے سے درج ذیل لکھا جا سکتا ہے۔
\begin{align*}
V_0=-14 \left(\frac{\SI{8}{\kilo\ohm}}{\SI{8}{\kilo\ohm}+\SI{6}{\kilo\ohm}}\right)=\SI{-8}{\volt}
\end{align*}
\انتہا{مثال}
%=====================
\حصہء{تھونن ادوار حل کرنے کا قدم با قدم طریقہ}
برقی بوجھ کو ہٹا کر کھلے سروں کے مابین  دباو \عددی{V_{\text{کھلا}}} حاصل کریں۔تھونن دباو حاصل کرتے وقت ادوار حل کرنے کے تمام طریقے بروئے کار لائے جا سکتے ہیں۔

کھلے سروں کے مابین تھونن مزاحمت حاصل کریں۔یہ مزاحمت حاصل کرتے وقت تین اقسام کے ادوار کا سامنا کرنا پڑ سکتا ہے۔پہلی قسم  کے ادوار میں صرف غیر تابع منبع استعمال کیا جاتا ہے۔ان ادوار میں منبع دباو کو قصر دور اور منبع رو کو کھلے دور کرتے ہوئے  تھونن مزاحمت حاصل کی جاتی ہے۔ دوسری قسم کے ادوار میں صرف تابع منبع پائے جاتے ہیں۔ان ادوار کے برقی سروں پر پیمائشی منبع دباو یا پیمائشی منبع رو نسب کرتے ہوئے برقی سروں پر دباو اور رو حاصل کی جاتی ہے۔برقی سروں کی دباو تقسیم رو سے تھونن مزاحمت حاصل ہوتی ہے۔تیسری قسم کے ادوار میں تابع منبع اور غیر تابع منبع دونوں پائے جاتے ہیں۔ان اقسام کے ادوار میں برقی سروں کو آپس میں قصر دور کرتے ہوئے قصر دور رو حاصل کی جاتی ہے۔کھلے دور دباو تقسیم قصر دور رو کی شرح تھونن مزاحمت دیتی ہے۔

سلسلہ وار جڑے کھلے دور دباو \عددی{V_{\text{کھلا}}} اور تھونن مزاحمت \عددی{R_{\text{تھونن}}} کے ساتھ بوجھ جوڑتے ہوئے بوجھ پر دباو اور اس کی رو حاصل کی جاتی ہے۔ 

مسئلہ نارٹن کے استعمال میں بالکل اسی طرح چلتے ہوئے آخری قدم پر متوازی جڑے قصر دور رو \عددی{I_{\text{قصر}}} اور تھونن مزاحمت \عددی{R_{\text{تھونن}}} کے ساتھ بوجھ جوڑتے ہوئے بوجھ کی دباو اور رو حاصل کی جاتی ہیں۔ 

%==============
\ابتدا{مشق}\شناخت{مشق_مسئلہ_تابع_غیر_تابع_الف}
شکل \حوالہ{شکل_مسئلہ_تابع_غیر_تابع_الف} میں مسئلہ تھونن کی مدد سے \عددی{V_0} حاصل کریں۔
\begin{figure}
\centering
\begin{tikzpicture}
\draw(0,0) to [short,-o]++(2*\x+\x/2,0);
\draw(0,0) to [american current source,l={$\SI{2}{\milli\ampere}$}]++(0,2*\y) to [resistor,l={$\SI{4}{\kilo\ohm}$}]++(\x,0) to [short]++(\x,0) to [short,-o]++(\x/2,0);
\draw(\x,0) to [american controlled current source,*-,l={$\frac{V_x}{2000}$}]++(0,\y) to [resistor,-*,l={$\SI{2}{\kilo\ohm}$}]++(0,\y);
\draw(2*\x,0) to [resistor,*-*,l={$\SI{6}{\kilo\ohm}$}]++(0,2*\y);
\draw(\x/2,2*\y-\dy)node[below]{$+ \, V_x \, -$};
\draw(2*\x+\x/2,\y)node{$\begin{aligned} &+ \\ \\ &V_0 \\ \\ &- \end{aligned}$};
\end{tikzpicture}
\caption{مشق \حوالہ{مشق_مسئلہ_تابع_غیر_تابع_الف} کا دور۔}
\label{شکل_مسئلہ_تابع_غیر_تابع_الف}
\end{figure}
\انتہا{مشق}
%====================

\ابتدا{مشق}\شناخت{مشق_مسئلہ_تابع_غیر_تابع_ب}
شکل \حوالہ{شکل_مسئلہ_تابع_غیر_تابع_ب} میں مسئلہ تھونن کی مدد سے \عددی{V_0} حاصل کریں۔
\begin{figure}
\centering
\begin{tikzpicture}
\draw(0,0) to [short,-o]++(3*\x+\x/2,0);
\draw(0,0) to [american voltage source,l={$\SI{10}{\volt}$}]++(0,\y) to [short]++(\x,0) to [american controlled voltage source,l={$2000 I_x$}]++(\x,0) to [resistor,i<^={$I_x$},l_={$\SI{4}{\kilo\ohm}$}]++(\x,0) to [resistor,-*,l_={$\SI{6}{\kilo\ohm}$}]++(0,-\y);
\draw(3*\x,\y) to [short,*-o]++(\x/2,0);
\draw(3*\x,\y) to [short]++(0,\y) to [american current source,l={$\SI{4}{\milli\ampere}$}]++(-2*\x,0) to [short]++(0,-\y);
\draw(\x,0) to [resistor,*-*,l={$\SI{2}{\kilo\ohm}$}]++(0,\y);
\draw(2*\x,0) to [resistor,*-*,l={$\SI{8}{\kilo\ohm}$}]++(0,\y);
\draw(3*\x+\x/2,\y/2)node{$\begin{aligned} &+ \\ &V_0 \\ &- \end{aligned}$};
\end{tikzpicture}
\caption{مشق \حوالہ{مشق_مسئلہ_تابع_غیر_تابع_ب} کا دور۔}
\label{شکل_مسئلہ_تابع_غیر_تابع_ب}
\end{figure}
\انتہا{مشق}
%====================


\ابتدا{مشق}\شناخت{مشق_مسئلہ_تابع_غیر_تابع_پ}
شکل \حوالہ{شکل_مسئلہ_تابع_غیر_تابع_پ} میں مسئلہ تھونن کی مدد سے  منبع دباو کی فراہم کردہ طاقت حاصل کریں۔
\begin{figure}
\centering
\begin{tikzpicture}
\draw(0,0) to [short]++(3*\x,0);
\draw(0,0) to [american current source,l={$\SI{2}{\milli\ampere}$}]++(0,\y) to [short]++(\x,0) to [american  voltage source,l={$\SI{10}{\volt}$}]++(\x,0) to [resistor,l_={$\SI{6}{\kilo\ohm}$}]++(\x,0) to [resistor,*-,l_={$\SI{1}{\kilo\ohm}$}]++(0,-\y);
\draw(3*\x,\y) to [short]++(0,\y) to [american current source,l={$\SI{4}{\milli\ampere}$}]++(-2*\x,0) to [short]++(0,-\y);
\draw(\x,0) to [resistor,*-*,l={$\SI{4}{\kilo\ohm}$}]++(0,\y);
\draw(2*\x,0) to [resistor,*-*,l={$\SI{2}{\kilo\ohm}$}]++(0,\y);
\end{tikzpicture}
\caption{مشق \حوالہ{مشق_مسئلہ_تابع_غیر_تابع_پ} کا دور۔}
\label{شکل_مسئلہ_تابع_غیر_تابع_پ}
\end{figure}
\انتہا{مشق}
%====================

\حصہ{زیادہ سے زیادہ طاقت کی منتقل کرنے کا مسئلہ}

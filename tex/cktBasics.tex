\باب{بنیاد}
اس کتاب میں \اصطلاح{بین الاقوامی نظام اکائی}\فرہنگ{بین الاقوامی نظام اکائی}\فرہنگ{نظام اکائی!بین الاقوامی}\حاشیہب{SI system}\فرہنگ{SI system} استعمال کی جائے گی جس کے چند بنیادی اکایاں کلوگرام (\عددی{\si{\kilo\gram}})، میٹر (\عددی{\si{\meter}})، سیکنڈ (\عددی{\si{\second}}) اور کیلون (\عددی{\si{\kelvin}}) ہیں۔ان اکایوں کے ساتھ عموماً شکل \حوالہ{شکل_بنیادی_سابقہ} میں دکھائے گئے ضربیے استعمال کئے جاتے ہیں جن سے آپ بخوبی واقف ہیں۔
\begin{figure}
\centering
\includegraphics{figBasicSIprefix}
\caption{بین الاقوامی نظام اکائی کے ضربیے۔}
\label{شکل_بنیادی_سابقہ}
\end{figure}

اس کتاب میں \اصطلاح{برقی بار}\فرہنگ{برقی بار}\حاشیہب{electric charge}\فرہنگ{charge} اور \اصطلاح{برقی رو}\فرہنگ{برقی رو}\حاشیہب{electric current}\فرہنگ{current} کلیدی کردار ادا کریں گے۔ برقی بار کی اصطلاح کو چھوٹا کر کے صرف \اصطلاح{برق} یا صرف \اصطلاح{بار} کی اصطلاح استعمال کی جائے گی جبکہ برقی رو کی اصطلاح کو چھوٹا کر کے \اصطلاح{رو} کی اصطلاح استعمال کی جائے گی۔برقی بار کے حرکت کو برقی رو کہتے ہیں۔چونکہ بار کی حرکت سے توانائی ایک مقام سے دوسرے مقام منتقل ہوتی ہے لہٰذا ہماری دلچسپی کا مرکز برقی رو ہو گی۔

موصل تار  کی مدد سے برقی پرزہ جات کو مختلف انداز میں آپس میں جوڑنے سے \اصطلاح{برقی دور}\فرہنگ{برقی دور}\حاشیہب{electric circuit}\فرہنگ{circuit} حاصل ہوتا ہے۔جیسے پائپ سے پانی کو ایک مقام سے دوسرے مقام تک منتقل کیا جاتا ہے، بالکل اسی طرح برقی دور میں ایک نقطے سے دوسرے نقطے تک بار موصل تار کے ذریعہ پہنچایا جاتا ہے۔یوں اگر پانی کو بار تصور کیا جائے تو حرکت کرتے پانی کو برقی رو تصور کیا جائے گا جبکہ موصل تار کو پائپ تصور کیا جائے گا۔برقی ادوار سمجھنے میں یہ مشابہت مدد گار ثابت ہوتی ہے۔  

کسی بھی نقطے پر برقی رو سے مراد اس نقطے سے فی سیکنڈ گزرتا بار  ہے۔رو اور بار کے تعلق کو \اصطلاح{تفرقی}\فرہنگ{تفرقی صورت}\حاشیہب{differential form}\فرہنگ{differential form} صورت میں یوں
\begin{align}\label{مساوات_بنیادی_رو_تعریف}
i=\frac{\dif q}{\dif t}
\end{align}
اور \اصطلاح{تکملہ صورت}\فرہنگ{تکملہ صورت}\حاشیہب{integral form}\فرہنگ{integral form} میں یوں
\begin{align}
q=\int_{-\infty}^{t} i \dif t
\end{align}
لکھا جا سکتا ہے جہاں برقی بار کو \عددی{q} سے ظاہر کیا گیا ہے اور برقی رو کو \عددی{i} سے ظاہر کیا گیا ہے۔بدلتے متغیرات کو انگریزی کے چھوٹے حروف تہجی  مثلاً \عددی{i} یا \عددی{q} سے ظاہر کیا جاتا ہے جبکہ غیر متغیر مقدار کو انگریزی کے بڑے حروف تہجی سے ظاہر کیا جاتا ہے۔یوں غیر متغیر رو کو \عددی{I} اور غیر متغیر بار کو \عددی{Q} سے ظاہر کیا جائے گا۔

بار کی اکائی کو \اصطلاح{کولمب}\فرہنگ{کولمب}\حاشیہب{Coulomb}\فرہنگ{Coulomb} کہتے ہیں جسے \عددی{\si{\coulomb}} کی علامت سے ظاہر کیا جاتا ہے جبکہ رو کی اکائی کو \اصطلاح{ایمپیئر}\فرہنگ{ایمپیئر}\حاشیہب{Ampere}\فرہنگ{Ampere} کہتے ہیں۔ایمپیئر کی علامت \عددی{\si{\ampere}} ہے۔اگر تار سے ایک سیکنڈ دورانیے میں ایک کولمب کا بار گزر رہا ہو تب تار میں ایک ایمپیئر کی برقی رو پائی جائے گی۔

روایتی طور پر یہ تصور کیا جاتا تھا کہ مثبت بار کے حرکت سے برقی رو پیدا ہوتی ہے۔اب ہم جانتے ہیں کہ حقیقت میں موصل تار میں مثبت ایٹم ساکن ہوتے ہیں اور آزاد منفی الیکٹران کے  حرکت سے  رو پیدا ہوتی ہے۔اس حقیقت کے باوجود، تصور کیا جاتا ہے کہ مثبت بار کی حرکت برقی رو کو جنم دیتی ہے۔شکل-الف میں فی سیکنڈ \عددی{\SI{3}{\coulomb}} کا بار بائیں سے دائیں جانب منتقل ہو رہا ہے جبکہ شکل-ب میں فی سیکنڈ  \عددی{\SI{2}{\coulomb}} کا بار دائیں سے بائیں جانب منتقل ہو رہا ہے۔یوں آپ دیکھ سکتے ہیں کہ برقی رو  کی مقدار اور سمت دونوں بیان کرنا ضروری ہیں۔

غیر متغیر برقی رو کو \اصطلاح{یک سمتی رو}\فرہنگ{یک سمتی رو}\فرہنگ{رو!یک سمتی}\حاشیہب{direct current, DC}\فرہنگ{direct current}\فرہنگ{DC} کہتے ہیں۔یک سمتی رو کی مقدار وقت کے ساتھ تبدیل نہیں ہوتی۔وقت کے ساتھ تبدیل ہوتی برقی رو کو \اصطلاح{بدلتی رو}\فرہنگ{بدلتی رو}\فرہنگ{رو!بدلتی}\حاشیہب{alternating current, AC}\فرہنگ{alternating current}\فرہنگ{AC} کہتے ہیں۔ ان دونوں کو شکل میں دکھایا گیا ہے۔موبائل کی بیٹری یک سمتی رو پیدا کرتی ہے جبکہ گھریلو پنکھا بدلتی رو سے چلتا ہے۔

عام زندگی میں اونچائی کو زمین سے ناپا جاتا ہے جہاں زمین کی اونچائی صفر کے برابر لی جاتی ہے۔یوں اونچائی کے ناپ میں زمین کو نقطہ \اصطلاح{حوالہ}\فرہنگ{حوالہ}\حاشیہب{reference}\فرہنگ{reference} لیا جاتا ہے۔شکل \حوالہ{شکل_بنیادی_دباو_اور_اونچائی}-الف میں سات منزلہ عمارت دکھائی گئی ہے۔اگر زمین نقطہ ت پر ہو تب نقطہ ن مثبت تین پڑھا جا سکتا ہے۔اس کے برعکس اگر زمین نقطہ ٹ پر ہو تب نقطہ ن زمین یعنی صفر پر ہے جبکہ زمین نقطہ ث پر ہونے کی صورت میں نقطہ ن منفی چار پر ہو گا۔آپ دیکھ سکتے ہیں کہ نقطہ ن کی حتمی اونچائی کوئی معنی نہیں رکھتی۔اونچائی صرف اس صورت میں معنی خیز ہوتی ہے جب نقطہ حوالہ بھی بیان کیا جائے۔
\begin{figure}
\centering
\includegraphics{figBasicVoltageHeight}
\caption{برقی دباو میں نقطہ حوالہ کی اہمیت۔}
\label{شکل_بنیادی_دباو_اور_اونچائی}
\end{figure}
برقی دباو بھی بالکل اونچائی کی طرح ناپی جاتی ہے۔یوں شکل \حوالہ{شکل_بنیادی_دباو_اور_اونچائی}-ب میں نقطہ ت کے حوالے سے نقطہ ٹ مثبت دو وولٹ \عددی{\SI{2}{\volt}} پر ہے جبکہ نقطہ ث کے حوالے سے نقطہ ٹ منفی پانچ وولٹ \عددی{\SI{-5}{\volt}} پر ہے۔اسی طرح نقطہ ٹ کے حوالے سے نقطہ ت \عددی{\SI{-2}{\volt}} پر اور نقطہ ث \عددی{\SI{5}{\volt}} پر ہیں۔نقطہ ت کے حوالے سے نقطہ ث \عددی{\SI{7}{\volt}} پر ہے جبکہ نقطہ ث کے حوالے سے نقطہ ت \عددی{\SI{-7}{\volt}} پر ہے۔یاد رہے کہ نقطہ حوالہ کی برقی دباو صفر تصور کی جاتی ہے۔

برقی دباو کی قیمت بھی بیان کرتے ہوئے ضروری ہے کہ نقطہ حوالہ بیان کیا جائے۔برقی دور میں دباو کی نشاندہی کرتے ہوئے نقطہ حوالہ کو منفی کی علامت \عددی{(-)} سے ظاہر کیا جاتا ہے جبکہ مطلوبہ نقطے کو مثبت علامت \عددی{(+)} سے ظاہر کیا جاتا ہے۔شکل \حوالہ{شکل_بنیادی_دباو_کا_اظہار}-الف میں یوں نچلی تار نقطہ حوالہ ہے۔یوں اگر \عددی{V_1=\SI{4}{\volt}} ہو تب نچلی تار کی نسبت سے بالائی تار مثبت چار وولٹ پر ہو گا۔اسی طرح \عددی{V_1=\SI{-7}{\volt}} کی صورت میں نچلی تار کی نسبت سے بالائی تار منفی سات وولٹ پر ہو گا جس کا مطلب ہے کہ بالائی تار کو حوالہ لیتے ہوئے نچلی تار کی برقی دباو مثبت سات وولٹ ہو گی۔شکل  \حوالہ{شکل_بنیادی_دباو_کا_اظہار}-ب میں نچلی تار کو \عددی{a} نام دیا گیا ہے جبکہ بالائی تار کو \عددی{b} کہا گیا ہے۔اس صورت میں نچلی تار کے حوالے سے بالائی تار کی دباو کو \عددی{V_{ba}} لکھا جاتا ہے جہاں زیر نوشت میں پہلے درکار نقطے کا نام اور بعد میں نقطہ حوالہ کا نام بیان کیا جاتا ہے۔یوں اگر \عددی{V_{ba}} کی قیمت منفی ہو تب بالائی تار کے حوالے سے نچلی تار پر مثبت دباو ہو گا۔برقی دور میں عموماً کسی ایک نقطے کو \اصطلاح{برقی زمین}\فرہنگ{زمین!برقی}\حاشیہب{electrical ground}\فرہنگ{ground!electrical} چننا جاتا ہے۔یوں مختلف مقامات کے دباو بیان کرتے ہوئے ہر مرتبہ برقی زمین کی نشاندہی کرنا ضروری نہیں ہوتا۔شکل \حوالہ{شکل_بنیادی_اوہم_قانون} میں برقی زمین کی علامت استعمال کی گئی ہے۔برقی زمین کی برقی دباو صفر کے برابر لی جاتی ہے۔شکل \حوالہ{شکل_بنیادی_اوہم_قانون} میں بالائی تار کی برقی دباو \عددی{V_b=\SI{10}{\volt}} لکھی جائے گی جہاں زیر نوشت میں صرف بالائی تار کی نشاندہی \عددی{b} لکھ کر کی گئی جبکہ برقی زمین کا کوئی ذکر نہیں کیا گیا۔
\begin{figure}
\centering
\includegraphics{figBasicVoltagePositiveNegativeSides}
\caption{برقی دباو کا اظہار۔}
\label{شکل_بنیادی_دباو_کا_اظہار}
\end{figure}


\اصطلاح{ثقلی میدان}\فرہنگ{ثقلی میدان}\فرہنگ{میدان!ثقلی}\حاشیہب{gravitational field}\فرہنگ{gravitational field} میں میکانی بار \عددی{m} پر قوت \عددی{F=m g} عمل کرتا ہے جہاں \عددی{g=\SI{9.8}{\meter\per\second\squared}} کے برابر ہے۔یوں ثقلی میدان کے مخالف \عددی{m} کو \عددی{h} بلندی تک پہنچانے کی خاطر \عددی{w=Fh=mgh} توانائی درکار ہے۔بالکل اسی طرح \اصطلاح{برقی میدان}\فرہنگ{برقی میدان}\فرہنگ{میدان!ثقلی}\حاشیہب{electric field}\فرہنگ{electric field} \عددی{E} میں برقی بار \عددی{q} پر \عددی{F=qE} قوت عمل کرتا ہے اور برقی میدان کے مخالف \عددی{h} فاصلے تک بار کو منتقل کرنے کی خاطر 
\begin{align}\label{مساوات_بنیادی_توانائی_دباو}
w=q E h
\end{align}
توانائی درکار ہے۔ابتدائی نقطے سے اختتامی نقطے تک اکائی برقی بار منتقل کرنے کے لئے درکار توانائی کو ابتدائی نقطے کے حوالے سے اختتامی نقطے کی برقی دباو کہا جاتا ہے۔
%========================
\ابتدا{مثال}
برقی میدان \عددی{E=\SI{600}{\volt\per\meter}} میں \عددی{\SI{0.2}{\coulomb}} بار قوت کے مخالف \عددی{\SI{12}{\milli\meter}} فاصلہ دُور منتقل کیا جاتا ہے۔درکار توانائی حاصل کریں۔ابتدائی نقطہ \عددی{i} اور اختتامی نقطہ \عددی{k} کے مابین برقی دباو حاصل کریں۔

حل:درکار توانائی
\begin{align*}
w=0.2 \times 600 \times 0.012=\SI{1.44}{\joule}
\end{align*}
کے برابر ہے جبکہ برقی دباو
\begin{align*}
V_{ki}=\frac{1.44}{0.2}=\SI{7.2}{\volt}
\end{align*}
کے برابر ہے۔
\انتہا{مثال}
%=================

مساوات \حوالہ{مساوات_بنیادی_توانائی_دباو} کی تفرقی صورت 
\begin{align*}
\dif w= E h \dif q
\end{align*}
لکھی جا سکتی ہے جو چھوٹی برقی بار \عددی{\dif q} کو منتقل کرنے کے لئے درکار توانائی \عددی{\dif w} دیتی ہے۔یوں اکائی بار کو منتقل کرنے کی خاطر \عددی{\tfrac{\dif w}{\dif q}} توانائی درکار ہو گی جسے برقی دباو \عددی{v} کہتے ہیں یعنی
\begin{align}\label{مساوات_بنیادی_دباو_تعریف}
v=\frac{\dif w}{\dif q}
\end{align}
لکھی جا سکتی ہے۔

مساوات \حوالہ{مساوات_بنیادی_دباو_تعریف} کو مساوات \حوالہ{مساوات_بنیادی_رو_تعریف} سے ضرب دینے سے
\begin{align}
v \times i = \frac{\dif w}{\dif q} \times \frac{\dif q}{\dif t}=\frac{\dif w}{\dif t}=p
\end{align}
حاصل ہوتا ہے جو \اصطلاح{طاقت}\فرہنگ{طاقت}\حاشیہب{power}\فرہنگ{power} \عددی{p} کو ظاہر کرتا ہے۔فی سیکنڈ درکار توانائی کو طاقت کہتے ہیں۔طاقت کی اکائی \اصطلاح{واٹ}\فرہنگ{واٹ}\حاشیہب{watt}\فرہنگ{watt} \عددی{\si{\watt}} ہے۔مندرجہ بالا مساوات کی تکملہ صورت درج ذیل ہے۔
\begin{align}
w=\int_{t_1}^{t_2} p \dif t=\int_{t_1}^{t_2} v i\dif t
\end{align}

\حصہ{قانون اوہم}
آئیں ان معلومات کو مد نظر رکھتے ہوئے شکل \حوالہ{شکل_بنیادی_اوہم_قانون} پر غور کریں جہاں \عددی{\SI{10}{\volt}} کی \اصطلاح{منبع برقی دباو}\فرہنگ{منبع!برقی دباو}\حاشیہب{voltage source}\فرہنگ{source!voltage}\فرہنگ{voltage source} کے ساتھ \عددی{\SI{5}{\ohm}} کی \اصطلاح{برقی مزاحمت}\فرہنگ{برقی مزاحمت}\حاشیہب{electrical resistance}\فرہنگ{resistance} جوڑی گئی ہے۔\اصطلاح{قانون اوہم}\فرہنگ{قانون اوہم}\فرہنگ{اوہم!قانون}\حاشیہب{Ohm's law}\فرہنگ{Ohm!law} کے تحت اس دور میں \اصطلاح{سمتِ گھڑی}\فرہنگ{سمت گھڑی}\حاشیہب{clockwise}\فرہنگ{clockwise} \عددی{\SI{2}{\ampere}} کی برقی رو پائی جائی گی۔دور میں \عددی{\SI{2}{\ampere}} برقی رو سے مراد یہ ہے کہ دور میں کسی بھی نقطے پر اگر دیکھا جائے تو اس نقطے سے فی سیکنڈ \عددی{\SI{2}{\coulomb}} بار گزرے گا۔ اس دور میں نچلی تار کے حوالے سے بالائی تار پر مثبت دس وولٹ کی دباو ہے۔یوں مزاحمت کے بالائی یعنی مثبت سرے سے  مزاحمت کے نچلے یعنی منفی سرے کی جانب فی سیکنڈ دو کولمب بار منتقل ہوتا ہے۔یہ بالکل ایسا ہی ہے جیسے ثقلی میدان میں بلند مقام سے میکانی بار گِر رہا ہو۔دو کولمب کا بار دس وولٹ نیچے گرتے ہوئے \عددی{\SI{20}{\joule}} کی \اصطلاح{مخفی توانائی}\فرہنگ{مخفی توانائی}\فرہنگ{توانائی!مخفی}\حاشیہب{potential energy}\فرہنگ{potential energy} کھوئے\حاشیہد{مخفی توانائی کی اصطلاح خفیہ توانائی سے حاصل کی گئی ہے۔} گا جو \اصطلاح{حرارتی توانائی}\فرہنگ{حرارتی توانائی}\فرہنگ{توانائی!حرارتی}\حاشیہب{thermal energy}\فرہنگ{thermal energy} میں تبدیل ہو کر مزاحمت کو گرم کرے گی۔ہم کہتے ہیں کہ مزاحمت میں فی سیکنڈ توانائی کا \اصطلاح{ضیاع}\فرہنگ{ضیاع}\حاشیہب{loss}\فرہنگ{loss} \عددی{\SI{20}{\joule}} ہے یا کہ مزاحمت میں \اصطلاح{طاقتی ضیاع}\فرہنگ{طاقتی ضیاع}\حاشیہب{power loss}\فرہنگ{power loss} \عددی{\SI{20}{\watt}} ہے۔مزاحمت میں طاقت کے ضیاع کو \اصطلاح{حرارتی ضیاع}\فرہنگ{حرارتی ضیاع}\فرہنگ{ضیاع!حرارتی}\حاشیہب{thermal loss}\فرہنگ{thermal loss} اور \اصطلاح{مزاحمتی ضیاع}\فرہنگ{مزاحمتی ضیاع}\حاشیہب{resistive loss}\فرہنگ{resistive loss} بھی کہتے ہیں۔
\begin{figure}
\centering
\includegraphics{figBasicOhmsLaw}
\caption{اوہم کا قانون۔}
\label{شکل_بنیادی_اوہم_قانون}
\end{figure}

\باب{بنیاد}
اس کتاب میں \اصطلاح{بین الاقوامی نظام اکائی}\فرہنگ{بین الاقوامی نظام اکائی}\فرہنگ{نظام اکائی!بین الاقوامی}\حاشیہب{SI system}\فرہنگ{SI system} استعمال کی گئی ہے جس کے چند بنیادی اکایاں کلوگرام (\عددی{\si{\kilo\gram}})، میٹر (\عددی{\si{\meter}})، سیکنڈ (\عددی{\si{\second}}) اور کیلون (\عددی{\si{\kelvin}}) ہیں۔ان اکایوں کے ساتھ عموماً شکل \حوالہ{شکل_بنیادی_سابقہ} میں دکھائے گئے ضربیے استعمال کئے جاتے ہیں جن سے آپ بخوبی واقف ہیں۔
\begin{figure}
\centering
\includegraphics{figBasicSIprefix}
\caption{بین الاقوامی نظام اکائی کے ضربیے۔}
\label{شکل_بنیادی_سابقہ}
\end{figure}

\حصہ{برقی بار، برقی رو اور برقی دباو}
اس کتاب میں \اصطلاح{برقی بار}\فرہنگ{برقی بار}\حاشیہب{electric charge}\فرہنگ{charge}  اور
 \اصطلاح{برقی رو}\فرہنگ{برقی رو}\حاشیہب{electric current}\فرہنگ{current}  \,  کلیدی کردار ادا کریں گے۔ برقی بار کی اصطلاح کو چھوٹا کر کے صرف \اصطلاح{برق} یا صرف \اصطلاح{بار} کی اصطلاح استعمال کی جائے گی جبکہ برقی رو کی اصطلاح کو چھوٹا کر کے \اصطلاح{رو} کی اصطلاح استعمال کی جائے گی۔برقی بار کی حرکت کو برقی رو کہتے ہیں۔چونکہ بار کی حرکت سے توانائی ایک مقام سے دوسرے مقام منتقل ہوتی ہے لہٰذا ہماری دلچسپی کا مرکز برقی رو ہو گی۔

موصل تار  کی مدد سے برقی پرزہ جات کو مختلف انداز میں آپس میں جوڑنے سے \اصطلاح{برقی دور}\فرہنگ{برقی دور}\حاشیہب{electric circuit}\فرہنگ{circuit} حاصل ہوتا ہے۔جیسے پائپ سے پانی کو ایک مقام سے دوسرے مقام تک منتقل کیا جاتا ہے، بالکل اسی طرح برقی دور میں ایک نقطے سے دوسرے نقطے تک بار موصل تار کے ذریعہ پہنچایا جاتا ہے۔یوں اگر پانی کو بار تصور کیا جائے تو حرکت کرتے پانی کو برقی رو تصور کیا جائے گا جبکہ موصل تار کو پائپ تصور کیا جائے گا۔برقی ادوار سمجھنے میں یہ مشابہت مدد گار ثابت ہوتی ہے۔  

کسی بھی نقطے پر برقی رو سے مراد اس نقطے سے فی سیکنڈ گزرتا بار  ہے۔رو اور بار کے تعلق کو \اصطلاح{تفرقی}\فرہنگ{تفرقی صورت}\حاشیہب{differential form}\فرہنگ{differential form} صورت میں یوں
\begin{align}\label{مساوات_بنیادی_رو_تعریف}
i=\frac{\dif q}{\dif t}
\end{align}
اور \اصطلاح{تکملہ صورت}\فرہنگ{تکملہ صورت}\حاشیہب{integral form}\فرہنگ{integral form} میں یوں
\begin{align}
q=\int_{-\infty}^{t} i \dif t
\end{align}
لکھا جا سکتا ہے جہاں برقی بار کو \عددی{q} سے ظاہر کیا گیا ہے اور برقی رو کو \عددی{i} سے ظاہر کیا گیا ہے۔بدلتے متغیرات کو انگریزی کے چھوٹے حروف تہجی  مثلاً \عددی{i} یا \عددی{q} سے ظاہر کیا جاتا ہے جبکہ غیر متغیر مقدار کو انگریزی کے بڑے حروف تہجی سے ظاہر کیا جاتا ہے۔یوں غیر متغیر رو کو \عددی{I} اور غیر متغیر بار کو \عددی{Q} سے ظاہر کیا جائے گا۔

بار کی اکائی کو \اصطلاح{کولمب}\فرہنگ{کولمب}\حاشیہب{Coulomb}\فرہنگ{Coulomb} کہتے ہیں جسے \عددی{\si{\coulomb}} کی علامت سے ظاہر کیا جاتا ہے جبکہ رو کی اکائی کو \اصطلاح{ایمپیئر}\فرہنگ{ایمپیئر}\حاشیہب{Ampere}\فرہنگ{Ampere} کہتے ہیں۔ایمپیئر کی علامت \عددی{\si{\ampere}} ہے۔اگر تار سے ایک سیکنڈ دورانیے میں ایک کولمب کا بار گزر رہا ہو تب تار میں ایک ایمپیئر کی برقی رو پائی جائے گی۔

روایتی طور پر تصور کیا جاتا تھا کہ تار میں مثبت بار کی حرکت سے برقی رو پیدا ہوتی ہے۔اب ہم جانتے ہیں کہ حقیقت میں موصل تار میں مثبت ایٹم ساکن ہوتے ہیں اور آزاد منفی الیکٹران کی  حرکت سے  رو پیدا ہوتی ہے۔اس حقیقت کے باوجود، تصور کیا جاتا ہے کہ مثبت بار کی حرکت برقی رو کو جنم دیتی ہے۔شکل \حوالہ{شکل_بنیادی_مثبت_منفی_بار_رو}-الف میں فی سیکنڈ \عددی{\SI{3}{\coulomb}} کا بار بائیں سے دائیں جانب منتقل ہو رہا ہے جو بائیں سے دائیں جانب \عددی{\SI{3}{\ampere}} رو کو جنم دیتی ہے۔ شکل \حوالہ{شکل_بنیادی_مثبت_منفی_بار_رو}-ب میں فی سیکنڈ  \عددی{\SI{-2}{\coulomb}} کا بار دائیں سے بائیں جانب منتقل ہو رہا ہے جو بائیں سے دائیں جانب \عددی{\SI{2}{\ampere}} کی رو پیدا کرتی ہے۔بار کا قطب اور سمت بہاو جانتے ہوئے رو کی مقدار اور سمت کا تعین ممکن ہوتا ہے۔
\begin{figure}
\centering
\begin{subfigure}{0.5\textwidth}
\centering
\begin{tikzpicture}
\draw(0,0) rectangle ++(-0.5,\y+0.5);
\draw(\x,0) rectangle ++(0.5,\y+0.5);
\draw(-0.25,\y/2+0.25)node[rotate=90]{\RL{بایاں دور}};
\draw(\x+0.25,\y/2+0.25)node[rotate=90]{\RL{دایاں دور}};
\draw(0,0.25)--++(\x,0);
\draw(0,0.25+\y) to [short,i_={$\SI{3}{\ampere}$}]++(\x,0);
\draw[-latex](\x/4,\y+0.25+0.3)--++(\x/2,0)node[pos=0.5,above]{$\SI{3}{\coulomb\per\second}$};
\end{tikzpicture}
\caption*{(الف)}
\end{subfigure}%
\begin{subfigure}{0.5\textwidth}
\centering
\begin{tikzpicture}
\draw(0,0) rectangle ++(-0.5,\y+0.5);
\draw(\x,0) rectangle ++(0.5,\y+0.5);
\draw(-0.25,\y/2+0.25)node[rotate=90]{\RL{بایاں دور}};
\draw(\x+0.25,\y/2+0.25)node[rotate=90]{\RL{دایاں دور}};
\draw(0,0.25)--++(\x,0);
\draw(0,0.25+\y) to [short,i_={$\SI{2}{\ampere}$}]++(\x,0);
\draw[latex-](\x/4,\y+0.25+0.3)--++(\x/2,0)node[pos=0.5,above]{$\SI{-2}{\coulomb\per\second}$};
\end{tikzpicture}
\caption*{(ب)}
\end{subfigure}%
\caption{مثبت بار اور منفی بار کی حرکت سے پیدا رو۔}
\label{شکل_بنیادی_مثبت_منفی_بار_رو}
\end{figure}

غیر متغیر برقی رو کو \اصطلاح{یک سمتی رو}\فرہنگ{یک سمتی رو}\فرہنگ{رو!یک سمتی}\حاشیہب{direct current, DC}\فرہنگ{direct current}\فرہنگ{DC} کہتے ہیں۔یک سمتی رو کی مقدار وقت کے ساتھ تبدیل نہیں ہوتی۔وقت کے ساتھ تبدیل ہوتی برقی رو کو \اصطلاح{بدلتی رو}\فرہنگ{بدلتی رو}\فرہنگ{رو!بدلتی}\حاشیہب{alternating current, AC}\فرہنگ{alternating current}\فرہنگ{AC} کہتے ہیں۔ ان دونوں کو شکل میں دکھایا گیا ہے۔موبائل کی بیٹری یک سمتی رو پیدا کرتی ہے جبکہ گھریلو پنکھا بدلتی رو سے چلتا ہے۔
\begin{figure}
\centering
\includegraphics{figBasicCurrentAndDirection}
\caption{برقی رو کو بیان کرنے کے درست طریقے۔}
\label{شکل_بنیادی_رو_درست_بیان}
\end{figure}

شکل \حوالہ{شکل_بنیادی_رو_درست_بیان}-الف میں دور ت اور دور ٹ کو دو تاروں سے آپس میں جوڑا گیا ہے۔بالائی تار میں دور ت سے دور ٹ کی جانب تین ایمپیئر کی رو پائی جاتی ہے۔اس تار پر تیر کا نشان رو کی سمت کو ظاہر کرتا ہے جبکہ تار کے نیچے \عددی{\SI{3}{\ampere}} لکھ کر رو کی مقدار بیان کی گئی ہے۔اب تصور کریں کہ تار پر تیر کا نشان نہیں دیا گیا ہے۔ایسی صورت میں برقی رو \عددی{I} کو یا تو دور ت سے دور ٹ کی جانب تصور کیا جا سکتا ہے اور یا دور ٹ سے دور ت کی جانب۔پہلی صورت کو شکل-الف میں دکھایا گیا ہے جہاں تار سے ہٹ کر دور ت سے دور ٹ کی جانب تیر سے رو \عددی{I} کو دکھایا گیا ہے۔چونکہ اصل رو اسی سمت میں ہے لہٰذا \عددی{I=\SI{3}{\ampere}} لکھا جائے گا۔دوسری صورت کو شکل-ب میں دکھایا گیا ہے جہاں دور ٹ سے دور ت کی جانب تیر کھینچا گیا ہے۔یوں شکل-ب میں برقی رو کی سمت دور ٹ سے دور ت کی جانب لی گئی ہے۔چونکہ اصل رو کی سمت تصور کردہ سمت کے الٹ ہے لہٰذا یہاں \عددی{I=\SI{-3}{\ampere}} لکھا جائے گا۔شکل-الف اور شکل-ب میں دکھائے گئے دونوں طریقے درست ہیں۔
\begin{figure}
\centering
\includegraphics{figBasicVoltageCurrentRelation}
\caption{مزاحمت کی رو اور دباو لکھنے کے چار ممکنہ طریقے۔}
\label{شکل_بنیادی_غیر_عامل_ترکیب}
\end{figure}
%
\begin{figure}
\centering
\includegraphics{figBasicOhmLawPassiveConvention}
\caption{انفعالی سمت کے ترکیب کی پہچان۔}
\label{شکل_بنیادی_غیر_عامل_ترکیب_پہچان}
\end{figure}

شکل \حوالہ{شکل_بنیادی_غیر_عامل_ترکیب}-الف میں \عددی{\SI{5}{\ohm}} کی مزاحمت میں \عددی{\SI{4}{\ampere}} کی رو پائی جاتی ہے۔اس مزاحمت کے دونوں سرے مزید پرزہ جات سے جڑے ہیں جنہیں شکل میں نہیں دکھایا گیا ہے۔شکل-ب تا شکل-ٹ میں مزاحمت پر دباو اور مزاحمت میں رو کو مختلف طریقوں سے لکھا گیا ہے۔کسی بھی دو متغیرات کو کل چار انداز میں لکھا جا سکتا ہے۔یہی دو عدد متغیرات یعنی دباو اور رو کے لئے بھی درست ہے لہٰذا انہیں لکھنے کے کل چار طریقے ہیں۔شکل \حوالہ{شکل_بنیادی_غیر_عامل_ترکیب_پہچان} میں برقی دباو اور برقی رو کے مقدار لکھے بغیر یہی چار طریقے دوبارہ دکھائے گئے ہیں۔ان میں شکل-ب اور شکل-ٹ کے طرز کو \اصطلاح{انفعالی سمت کی ترکیب}\فرہنگ{انفعالی سمت کی ترکیب}\حاشیہب{passive sign convention}\فرہنگ{passive sign convention} کہتے ہیں۔انفعالی سمت کی ترکیب میں دباو \عددی{V} اور رو \عددی{I} کی سمتیں یوں چننی جاتی ہیں کہ برقی پرزے میں رو مثبت سرے سے داخل ہوتی ہے۔یوں شکل-ب میں مزاحمت کے بالائی سرے کو دباو کا مثبت سرا چنا گیا ہے لہٰذا انفعالی سمت کی ترکیب میں اسی سرے پر رو مزاحمت میں ہو گی۔اسی طرح شکل-ٹ میں مزاحمت کا نچلا سرا دباو کا مثبت سر ہے لہٰذا انفعالی سمت کی ترکیب میں اسی سر پر مزاحمت میں رو داخل ہو گی۔یاد رہے کہ انفعالی سمت کی ترکیب میں اصل برقی رو اور برقی دباو کی درست سمتوں کا کوئی کردار نہیں۔\اصطلاح{قانونِ اوہم}\فرہنگ{قانون اوہم}\فرہنگ{اوہم!قانون}\حاشیہب{Ohm's law}\فرہنگ{Ohm!law} اور طاقت کے حساب میں انفعالی سمت کی ترکیب استعمال کیا جاتا ہے۔

\ابتدا{قانون}
\اصطلاح{انفعالی سمت کی ترکیب} میں برقی پرزے پر دباو کی سمت چننے کے بعد رو کی سمت یوں چننی جاتی ہے کہ چنے گئے دباو کے مثبت سر سے پرزے میں  رو داخل ہو۔
\انتہا{قانون}

عام زندگی میں اونچائی کو زمین سے ناپا جاتا ہے جہاں زمین کی اونچائی صفر کے برابر لی جاتی ہے۔یوں اونچائی کے ناپ میں زمین کو نقطہ \اصطلاح{حوالہ}\فرہنگ{حوالہ}\حاشیہب{reference}\فرہنگ{reference} لیا جاتا ہے۔شکل \حوالہ{شکل_بنیادی_دباو_اور_اونچائی}-الف میں سات منزلہ عمارت دکھائی گئی ہے۔اگر زمین نقطہ ت پر ہو تب نقطہ ن مثبت تین پڑھا جا سکتا ہے۔اس کے برعکس اگر زمین نقطہ ٹ پر ہو تب نقطہ ن زمین یعنی صفر پر ہے جبکہ زمین نقطہ ث پر ہونے کی صورت میں نقطہ ن منفی چار پر ہو گا۔آپ دیکھ سکتے ہیں کہ نقطہ ن کی حتمی اونچائی کوئی معنی نہیں رکھتی۔اونچائی صرف اس صورت میں معنی خیز ہوتی ہے جب نقطہ حوالہ بھی بیان کیا جائے۔
\begin{figure}
\centering
\includegraphics{figBasicVoltageHeight}
\caption{برقی دباو میں نقطہ حوالہ کی اہمیت۔}
\label{شکل_بنیادی_دباو_اور_اونچائی}
\end{figure}
برقی دباو بھی بالکل اونچائی کی طرح ناپی جاتی ہے۔یوں شکل \حوالہ{شکل_بنیادی_دباو_اور_اونچائی}-ب میں نقطہ ت کے حوالے سے نقطہ ٹ مثبت دو وولٹ \عددی{\SI{2}{\volt}} پر ہے جبکہ نقطہ ث کے حوالے سے نقطہ ٹ منفی پانچ وولٹ \عددی{\SI{-5}{\volt}} پر ہے۔اسی طرح نقطہ ٹ کے حوالے سے نقطہ ت \عددی{\SI{-2}{\volt}} پر اور نقطہ ث \عددی{\SI{5}{\volt}} پر ہیں۔نقطہ ت کے حوالے سے نقطہ ث \عددی{\SI{7}{\volt}} پر ہے جبکہ نقطہ ث کے حوالے سے نقطہ ت \عددی{\SI{-7}{\volt}} پر ہے۔یاد رہے کہ نقطہ حوالہ کا برقی دباو صفر تصور کیا جاتا ہے۔

برقی دباو کی قیمت بھی بیان کرتے ہوئے ضروری ہے کہ نقطہ حوالہ بیان کیا جائے۔برقی دور میں دباو کی نشاندہی کرتے ہوئے نقطہ حوالہ کو منفی کی علامت \عددی{(-)} سے ظاہر کیا جاتا ہے جبکہ مطلوبہ نقطے کو مثبت علامت \عددی{(+)} سے ظاہر کیا جاتا ہے۔شکل \حوالہ{شکل_بنیادی_دباو_کا_اظہار}-الف میں یوں نچلی تار نقطہ حوالہ ہے۔یوں اگر \عددی{V_1=\SI{4}{\volt}} ہو تب نچلی تار کی نسبت سے بالائی تار مثبت چار وولٹ پر ہو گا۔اسی طرح \عددی{V_1=\SI{-7}{\volt}} کی صورت میں نچلی تار کی نسبت سے بالائی تار منفی سات وولٹ پر ہو گا جس کا مطلب ہے کہ بالائی تار کو حوالہ لیتے ہوئے نچلی تار کا برقی دباو مثبت سات وولٹ ہو گی۔شکل  \حوالہ{شکل_بنیادی_دباو_کا_اظہار}-ب میں نچلی تار کو \عددی{a} نام دیا گیا ہے جبکہ بالائی تار کو \عددی{b} کہا گیا ہے۔اس صورت میں نچلی تار کے حوالے سے بالائی تار کے دباو کو \عددی{V_{ba}} لکھا جاتا ہے۔یوں اگر \عددی{V_{ba}} کی قیمت منفی ہو تب بالائی تار کے حوالے سے نچلی تار پر مثبت دباو ہو گا۔برقی دور میں عموماً کسی ایک نقطے کو \اصطلاح{برقی زمین}\فرہنگ{زمین!برقی}\حاشیہب{electrical ground}\فرہنگ{ground!electrical} چننا جاتا ہے۔یوں مختلف مقامات کے دباو بیان کرتے ہوئے ہر مرتبہ برقی زمین کی نشاندہی کرنا ضروری نہیں ہوتا۔شکل \حوالہ{شکل_بنیادی_دباو_کا_اظہار}-پ میں برقی زمین کی علامت استعمال کی گئی ہے۔برقی زمین کا برقی دباو صفر کے برابر لی جاتی ہے۔جب کسی نقطے کے دباو کو برقی زمین کی نسبت سے ناپا جائے تب نقطہ حوالے کا ذکر کرنا ضروری نہیں ہوتا۔یوں اس شکل میں بالائی تار کا برقی دباو \عددی{V_b=\SI{10}{\volt}} لکھی جا سکتی ہے  جہاں زیر نوشت میں نقطہ حوالہ کا ذکر نہیں کیا گیا۔شکل-پ میں اب بھی \عددی{V_{ba}=\SI{10}{\volt}} یا \عددی{V_{ab}=\SI{-10}{\volt}} لکھا جا سکتا ہے۔
\begin{figure}
\centering
\includegraphics{figBasicVoltagePositiveNegativeSides}
\caption{برقی دباو کا اظہار۔}
\label{شکل_بنیادی_دباو_کا_اظہار}
\end{figure}
%
%=========================
\حصہ{قانونِ اوہم}
\اصطلاح{قانونِ اوہم}\فرہنگ{قانون!اوہم}\فرہنگ{اوہم!قانون}\حاشیہب{Ohm's law}\فرہنگ{Ohm's law} سے آپ بخوبی واقف ہیں
\begin{align}
V=I R
\end{align}
جو مزاحمت کی برقی رو اور مزاحمت کا برقی دباو کا تعلق بیان کرتا ہے۔اس قانون\حاشیہد{یہ قانون جرمنی کے جارج سائمن اوہم نے پیش کیا۔} کے استعمال میں دباو \عددی{V} اور رو \عددی{I} کو انفعالی سمت کی ترکیب سے چننا جاتا ہے۔شکل \حوالہ{شکل_بنیادی_قانون_اوہم_اور_غیر_عامل_ترکیب} میں ایک عدد مزاحمت اور دو عدد منبع دباو کا دور دکھایا گیا ہے۔برقی زمین کے حوالے سے مزاحمت کے بائیں سرے پر \عددی{\SI{5}{\volt}} اور دائیں سرے پر \عددی{\SI{9}{\volt}} دباو پایا جاتا ہے۔قانون اوہم میں مزاحمت کے دو سروں کے مابین برقی دباو استعمال کیا جاتا ہے۔یوں مزاحمت کے ایک سرے کو \اصطلاح{حوالہ} لیتے ہوئے مزاحمت کے دوسرے سرے پر برقی دباو لی جاتی ہے۔شکل-الف میں مزاحمت کا بایاں سرا بطور حوالہ چننا گیا ہے جبکہ مزاحمت کے دائیں سرے پر برقی دباو استعمال کی جائے گی۔یہ حقیقت مزاحمت کے قریب \عددی{V_R} کے بائیں جانب \عددی{(-)} کی علامت اور دائیں جانب \عددی{(+)}  کی علامت سے ظاہر کی جاتی ہے۔یوں انفعالی سمت کی ترکیب کے تحت برقی رو کی سمت دائیں سے بائیں جانب چننی جائے گی۔شکل-الف میں یوں
\begin{align*}
V_R=9-5=\SI{4}{\volt}
\end{align*}  
ہو گا جسے اوہم کے قانون میں استعمال کرتے ہوئے
\begin{align*}
I_R=\frac{V_R}{R}=\frac{4}{8}=\SI{0.5}{\ampere}
\end{align*}
حاصل ہوتا ہے۔حاصل برقی رو کی قیمت مثبت مقدار ہے جس کا مطلب ہے  کہ رو کی سمت وہی ہے جو شکل-الف میں چننی گئی ہے۔


\begin{figure}
\centering
\includegraphics{figBasicOhmLawExamples}
\caption{قانونِ اوہم اور انفعالی سمت کی ترکیب۔}
\label{شکل_بنیادی_قانون_اوہم_اور_غیر_عامل_ترکیب}
\end{figure}

شکل \حوالہ{شکل_بنیادی_قانون_اوہم_اور_غیر_عامل_ترکیب}-ب میں مزاحمت کا دایاں سرا بطور نقطہ حوالہ چننا گیا ہے۔یوں \عددی{V_R} کے دائیں جانب \عددی{(-)} کی علامت لگائی گئی ہے۔انفعالی سمت کی ترکیب کے تحت رو کی سمت بائیں سے دائیں کو چننی گئی ہے۔یہاں
\begin{align*}
V_R=5-9=\SI{-4}{\volt}
\end{align*}
کے برابر ہے جسے اوہم کے قانون میں استعمال کرتے ہوئے
\begin{align*}
I_R=\frac{-4}{8}=\SI{-0.5}{\ampere}
\end{align*}
حاصل ہوتا ہے۔شکل-ب میں \عددی{V_R} کی قیمت منفی حاصل ہوئی جس کا مطلب ہے کہ حقیقت میں مزاحمت پر برقی دباو چننی گئی سمت کے الٹ ہے۔اسی طرح رو \عددی{I_R} کی قیمت بھی منفی حاصل ہوئی ہے جس کا مطلب ہے کہ حقیقت میں رو چننی گئی سمت کے الٹ ہے یعنی برقی رو حقیقت میں دائیں سے بائیں جانب کو ہے۔

شکل \حوالہ{شکل_بنیادی_قانون_اوہم_صحیح_استعمال} میں قانون اوہم کا صحیح استعمال دکھایا گیا ہے۔



\begin{figure}
\centering
\includegraphics{figBasicOhmLawGeneral}
\caption{قانونِ اوہم کا صحیح استعمال۔ }
\label{شکل_بنیادی_قانون_اوہم_صحیح_استعمال}
\end{figure}

%===================

\حصہ{توانائی اور طاقت}
\اصطلاح{ثقلی میدان}\فرہنگ{ثقلی میدان}\فرہنگ{میدان!ثقلی}\حاشیہب{gravitational field}\فرہنگ{gravitational field} میں میکانی بار \عددی{m} پر قوت \عددی{F=m g} عمل کرتا ہے جہاں \عددی{g=\SI{9.8}{\meter\per\second\squared}} کے برابر ہے۔یوں ثقلی میدان کے مخالف \عددی{m} کو \عددی{h} بلندی تک پہنچانے کی خاطر \عددی{w=Fh=mgh} توانائی درکار ہے۔بالکل اسی طرح \اصطلاح{برقی میدان}\فرہنگ{برقی میدان}\فرہنگ{میدان!ثقلی}\حاشیہب{electric field}\فرہنگ{electric field} \عددی{E} میں برقی بار \عددی{q} پر \عددی{F=qE} قوت عمل کرتی ہے اور برقی میدان کے مخالف \عددی{h} فاصلے تک بار کو منتقل کرنے کی خاطر 
\begin{align}\label{مساوات_بنیادی_توانائی_دباو}
w=q E h
\end{align}
توانائی درکار ہے۔برقی میدان میں ابتدائی نقطے سے اختتامی نقطے تک اکائی برقی بار منتقل کرنے کے لئے درکار توانائی کو ابتدائی نقطے کے حوالے سے اختتامی نقطے کا برقی دباو کہا جاتا ہے۔
%========================
\ابتدا{مثال}
برقی میدان \عددی{E=\SI{600}{\volt\per\meter}} میں \عددی{\SI{0.2}{\coulomb}} بار قوت کے مخالف \عددی{\SI{12}{\milli\meter}} فاصلہ دُور منتقل کیا جاتا ہے۔درکار توانائی حاصل کریں۔ابتدائی نقطہ \عددی{i} اور اختتامی نقطہ \عددی{k} کے مابین برقی دباو حاصل کریں۔

حل:درکار توانائی
\begin{align*}
w=0.2 \times 600 \times 0.012=\SI{1.44}{\joule}
\end{align*}
کے برابر ہے جبکہ برقی دباو
\begin{align*}
V_{ki}=\frac{1.44}{0.2}=\SI{7.2}{\volt}
\end{align*}
کے برابر ہے۔
\انتہا{مثال}
%=================

مساوات \حوالہ{مساوات_بنیادی_توانائی_دباو} کی تفرقی صورت 
\begin{align*}
\dif w= E h \dif q
\end{align*}
لکھی جا سکتی ہے جو چھوٹے برقی بار \عددی{\dif q} کو منتقل کرنے کے لئے درکار توانائی \عددی{\dif w} دیتی ہے۔یوں اکائی بار کو منتقل کرنے کی خاطر \عددی{\tfrac{\dif w}{\dif q}} توانائی درکار ہو گی جسے برقی دباو \عددی{v} کہتے ہیں یعنی
\begin{align}\label{مساوات_بنیادی_دباو_تعریف}
v=\frac{\dif w}{\dif q}
\end{align}
لکھی جا سکتی ہے۔

مساوات \حوالہ{مساوات_بنیادی_دباو_تعریف} کو مساوات \حوالہ{مساوات_بنیادی_رو_تعریف} سے ضرب دینے سے
\begin{align}\label{مساوات_بنیادی_طاقت_مساوی_دباو_ضرب_رو}
v \times i = \frac{\dif w}{\dif q} \times \frac{\dif q}{\dif t}=\frac{\dif w}{\dif t}=p
\end{align}
حاصل ہوتا ہے جو \اصطلاح{طاقت}\فرہنگ{طاقت}\حاشیہب{power}\فرہنگ{power} \عددی{p} کو ظاہر کرتا ہے۔فی سیکنڈ درکار توانائی کو طاقت کہتے ہیں۔طاقت کی اکائی \اصطلاح{واٹ}\فرہنگ{واٹ}\حاشیہب{watt}\فرہنگ{watt} \عددی{\si{\watt}} ہے۔مندرجہ بالا مساوات کی تکملہ صورت درج ذیل ہے۔
\begin{align}
w=\int_{t_1}^{t_2} p \dif t=\int_{t_1}^{t_2} v i\dif t
\end{align}

آئیں ان معلومات کو مد نظر رکھتے ہوئے شکل \حوالہ{شکل_بنیادی_اوہم_قانون} پر غور کریں جہاں \عددی{\SI{10}{\volt}} کی \اصطلاح{منبع برقی دباو}\فرہنگ{منبع!برقی دباو}\حاشیہب{voltage source}\فرہنگ{source!voltage}\فرہنگ{voltage source} کے ساتھ \عددی{\SI{5}{\ohm}} کی \اصطلاح{برقی مزاحمت}\فرہنگ{برقی مزاحمت}\حاشیہب{electrical resistance}\فرہنگ{resistance} جوڑی گئی ہے۔اس دور میں برقی رو کو منبع پیدا کرتی ہے لہٰذا منبع کو \اصطلاح{فعال پرزہ}\فرہنگ{فعال پرزہ}\فرہنگ{پرزہ!عامل}\حاشیہب{active component}\فرہنگ{active component} جبکہ مزاحمت کو \اصطلاح{انفعال پرزہ}\فرہنگ{انفعال پرزہ}\فرہنگ{پرزہ!انفعال}\حاشیہب{passive component}\فرہنگ{passive component} کہا جاتا ہے۔\اصطلاح{انفعالی سمت کی ترکیب} کا نام اسی حقیقت سے نکلا ہے کہ اس ترکیب کے استعمال سے انفعالی پرزہ جات پر مثبت طاقت حاصل ہوتی ہے۔

\اصطلاح{قانون اوہم}\فرہنگ{قانون اوہم}\فرہنگ{اوہم!قانون}\حاشیہب{Ohm's law}\فرہنگ{Ohm!law} کے تحت شکل \حوالہ{شکل_بنیادی_اوہم_قانون}  کے دور میں \اصطلاح{سمتِ گھڑی}\فرہنگ{سمت گھڑی}\حاشیہب{clockwise}\فرہنگ{clockwise} \عددی{\SI{2}{\ampere}} کی برقی رو پائی جائے گی جسے دور میں بالائی تار پر تیر کے نشان سے دکھایا گیا ہے۔دور میں \عددی{\SI{2}{\ampere}} برقی رو سے مراد یہ ہے کہ دور میں کسی بھی نقطے پر اگر دیکھا جائے تو اس نقطے سے فی سیکنڈ \عددی{\SI{2}{\coulomb}} بار گزرے گا۔ اس دور میں نچلی تار کے حوالے سے بالائی تار پر مثبت دس وولٹ کا دباو ہے۔یوں مزاحمت کے بالائی یعنی مثبت سرے سے  مزاحمت کے نچلے یعنی منفی سرے کی جانب فی سیکنڈ دو کولمب بار منتقل ہوتا ہے۔یہ بالکل ایسا ہی ہے جیسے ثقلی میدان میں بلند مقام سے میکانی بار گِر رہا ہو۔دو کولمب کا بار دس وولٹ نیچے گرتے ہوئے \عددی{\SI{20}{\joule}} کی \اصطلاح{مخفی توانائی}\فرہنگ{مخفی توانائی}\فرہنگ{توانائی!مخفی}\حاشیہب{potential energy}\فرہنگ{potential energy} کھوئے\حاشیہد{مخفی توانائی کی اصطلاح خفیہ توانائی سے حاصل کی گئی ہے۔} گا جو \اصطلاح{حرارتی توانائی}\فرہنگ{حرارتی توانائی}\فرہنگ{توانائی!حرارتی}\حاشیہب{thermal energy}\فرہنگ{thermal energy} میں تبدیل ہو کر مزاحمت کو گرم کرے گی۔ہم کہتے ہیں کہ مزاحمت میں فی سیکنڈ توانائی کا \اصطلاح{ضیاع}\فرہنگ{ضیاع}\حاشیہب{loss}\فرہنگ{loss} \عددی{\SI{20}{\joule}} ہے یا کہ مزاحمت میں \اصطلاح{طاقتی ضیاع}\فرہنگ{طاقتی ضیاع}\حاشیہب{power loss}\فرہنگ{power loss} \عددی{\SI{20}{\watt}} ہے۔مزاحمت میں طاقت کے ضیاع کو \اصطلاح{حرارتی ضیاع}\فرہنگ{حرارتی ضیاع}\فرہنگ{ضیاع!حرارتی}\حاشیہب{thermal loss}\فرہنگ{thermal loss} اور \اصطلاح{مزاحمتی ضیاع}\فرہنگ{مزاحمتی ضیاع}\حاشیہب{resistive loss}\فرہنگ{resistive loss} بھی کہتے ہیں۔
\begin{figure}
\centering
\includegraphics[width=\textwidth]{figBasicOhmsLaw}
\caption{طاقت کی پیداوار اور طاقت کا ضیاع۔}
\label{شکل_بنیادی_اوہم_قانون}
\end{figure}

انفعالی سمت کی ترکیب استعمال کرتے ہوئے ہم شکل \حوالہ{شکل_بنیادی_اوہم_قانون}-الف میں منبع کے دباو کو \عددی{V_M} اور مزاحمت کے دباو کو \عددی{V_الفR} چننے کے بعد ان دباو کے مثبت سر سے منفی سر کی جانب رو کی سمت چنتے ہیں۔یوں حاصل منبع کی برقی رو \عددی{I_M} اور مزاحمت کی برقی رو \عددی{I_R} کو شکل-الف میں دکھایا گیا ہے۔شکل- کو دیکھتے ہوئے درج ذیل لکھا جا سکتا ہے۔
\begin{align*}
V_M&=\SI{10}{\volt}\\
V_R&=\SI{10}{\volt}\\
I_M&=\SI{-2}{\ampere}\\
I_R&=\SI{2}{\ampere}
\end{align*} 
ان قیمتوں کو مساوات \حوالہ{مساوات_بنیادی_طاقت_مساوی_دباو_ضرب_رو} میں پر کرتے ہوئے منبع اور مزاحمت کی طاقت حاصل کرتے ہیں۔
\begin{align*}
P_M&=10 \times (-2)=\SI{-20}{\watt} \quad \quad \text{\RL{طاقت کی منفی قیمت، طاقت کی پیداوار کو ظاہر کرتی ہے}}\\
P_R&=10 \times 2 =\SI{20}{\watt}\quad \quad\quad\quad \text{\RL{طاقت کی مثبت قیمت، طاقت کی ضیاع کو ظاہر کرتی ہے}}
\end{align*}
یہاں غیر متغیر طاقت کو بڑھے حروف تہجی میں \عددی{P_M} اور \عددی{P_R} لکھا گیا۔مزاحمت کی طاقت مثبت مقدار حاصل ہوئی ہے جبکہ منبع کی طاقت منفی مقدار ہے۔یوں مساوات \حوالہ{مساوات_بنیادی_طاقت_مساوی_دباو_ضرب_رو} سے حاصل مثبت مقدار  طاقت کے ضیاع کو ظاہر کرتی ہے جبکہ منفی مقدار طاقت کی پیدا وار کو ظاہر کرتی ہے۔

شکل \حوالہ{شکل_بنیادی_اوہم_قانون}-ب میں برقی دباو کے سمت الٹ چننے گئے جس کی وجہ سے رو کی سمتیں بھی الٹ کر دی گئی ہیں۔یوں
\begin{align*}
V_M&=\SI{-10}{\volt}\\
V_R&=\SI{-10}{\volt}\\
I_M&=\SI{2}{\ampere}\\
I_R&=\SI{-2}{\ampere}
\end{align*} 
لکھے جائیں گے جن سے  دوبارہ
\begin{align*}
P_M&=(-10) \times 2=\SI{-20}{\watt} \quad \quad\quad \text{\RL{طاقت کی منفی قیمت، طاقت کی پیداوار کو ظاہر کرتی ہے}}\\
P_R&=(-10) \times (-2) =\SI{20}{\watt}\quad \quad \text{\RL{طاقت کی مثبت قیمت، طاقت کی ضیاع کو ظاہر کرتی ہے}}
\end{align*}
حاصل ہوتے ہیں۔
%======================

\ابتدا{مثال}
شکل \حوالہ{شکل_بنیادی_فعال_انفعال_مثال} میں دو ادوار دکھائے گئے ہیں۔دریافت کریں کہ آیا بیرونی پرزہ بقایا دور کو طاقت فراہم کرتا ہے یا کہ اس سے طاقت حاصل کرتا ہے۔طاقت کی قیمت بھی دریافت کریں۔

\begin{figure}
\centering
\includegraphics{figBasicPassiveSignExample}
\caption{فعال اور انفعال پرزے کی مثال۔}
\label{شکل_بنیادی_فعال_انفعال_مثال}
\end{figure}

حل:شکل-الف میں برقی رو کی قیمت منفی لکھی گئی ہے جس کا مطلب ہے کہ حقیقت میں رو تیر کے نشان کے الٹ سمت میں ہے۔رو کی سمت الٹ تصور کرتے ہوئے ہم دیکھتے ہیں کہ بقایا دور کے مثبت سرے پر رو اندر داخل ہوتی ہے۔یوں بقایا دور انفعال ہے۔بیرونی پرزے کے مثبت سرے سے حقیقی رو خارج ہوتی ہے لہٰذا یہ فعال پرزہ ہے۔یوں بیرونی پرزہ طاقت فراہم کرتا ہے جبکہ بقایا دور میں طاقت خرچ ہوتا ہے۔یہی نتائج انفعال سمت کے ترکیب سے یوں حاصل ہوتی ہے۔بیرونی پرزے کے برقی دباو کو دیکھتے ہوئے رو کی دکھائی گئی سمت ہی استعمال کی جائے گی۔یوں بیرونی پرزے کی طاقت \عددی{P=5\times (-6)=\SI{-30}{\watt}} ہے جو طاقت کی پیداوار ہے۔بقایا دور میں رو کی انفعال سمت دکھائے گئے سمت کے الٹ ہے لہٰذا طاقت \عددی{P=5\times 6=\SI{30}{\watt}} حاصل ہوتا ہے جو طاقت کی ضیاع کو ظاہر کرتا ہے۔آپ نے دیکھا کہ بیرونی پرزہ \عددی{\SI{30}{\watt}} طاقت پیدا کرتا ہے جبکہ بقایا دور اتنی ہی طاقت استعمال کرتا ہے۔ آپ دیکھ سکتے ہیں \اصطلاح{قانونِ  بقا}\فرہنگ{قانون! بقا}\حاشیہب{law of conservation of energy}\فرہنگ{law!conservation of energy} کارآمد ہے۔کسی بھی دور میں توانائی کی پیداوار اور خرچ برابر ہوتے ہیں۔

شکل-ب میں رو نچلی تار میں دائیں سے بائیں طرف رواں ہے۔یوں بیرونی پرزے کے مثبت سرے سے رو خارج ہوتی ہے جبکہ بقایا دور کے مثبت سرے میں رو داخل ہوتی ہے۔یوں بیرونی پرزہ فعال اور بقایا دور انفعال ہے۔بیرونی پرزے کی طاقت \عددی{P=7\times (-3)=\SI{-21}{\watt}} ہے جو طاقت کی پیداوار ہے جبکہ بقایا دور کی طاقت \عددی{P=7\times 3=\SI{21}{\watt}} ہے جو طاقت کی ضیاع کو ظاہر کرتی ہے۔
\انتہا{مثال}
%===========================

\ابتدا{مشق}
شکل \حوالہ{شکل_بنیادی_فعال_انفعال_مشق} میں بیرونی پرزے کی طاقت حاصل کریں۔

\begin{figure}
\centering
\includegraphics{figBasicPassiveSignQuiz}
\caption{فعال اور انفعال پرزے کی مشق۔}
\label{شکل_بنیادی_فعال_انفعال_مشق}
\end{figure}

جوابات: (الف) \عددی{\SI{8}{\watt}}؛ (ب) \عددی{\SI{27}{\watt}}
\انتہا{مشق}
%==========================
\ابتدا{مثال}
شکل \حوالہ{شکل_بنیادی_طاقت_مثال}-الف میں برقی رو کی مقدار اور سمت حاصل کریں جبکہ شکل-ب میں برقی دباو اور اس کا مثبت سرا دریافت کریں۔
\begin{figure}
\centering
\includegraphics{figBasicPowerExample}
\caption{طاقت اور ایک متغیرہ دیا گیا ہے۔دوسرا دریافت کرنا ہے۔}
\label{شکل_بنیادی_طاقت_مثال}
\end{figure}

حل: شکل-الف میں بیرونی پرزے کی طاقت منفی ہے۔یوں بیرونی پرزہ طاقت پیدا کرتا ہے لہٰذا اس کے مثبت سرے سے رو خارج ہو گی یعنی دور میں گھڑی کے الٹ سمت میں رو پائی جائے گی۔رو کی قیمت \عددی{\SI{4}{\ampere}} ہو گی۔

شکل-ب میں بیرونی پرزے کی طاقت مثبت ہے لہٰذا اس میں طاقت کا ضیاع ہو گا اور برقی رو مثبت سرے سے پرزے میں داخل ہو گی۔دور میں گھڑی کی سمت میں منفی رو دکھائی گئی ہے لہٰذا حقیقت میں رو گھڑی کی الٹ سمت ہے۔حقیقی رو کو گھڑی کے الٹ سمت تصور کرتے ہوئے  بیرونی پرزے کا نچلا سرا مثبت ہو گا اور برقی دباو کی قیمت \عددی{\SI{2}{\volt}} ہو گی۔
\انتہا{مثال}
%==========================

\ابتدا{مشق}
شکل \حوالہ{شکل_بنیادی_طاقت_مشق} میں نا معلوم متغیرہ دریافت کریں۔ 
\begin{figure}
\centering
\includegraphics{figBasicPowerQuiz}
\caption{طاقت اور ایک متغیرہ دیا گیا ہے۔دوسرا دریافت کریں۔}
\label{شکل_بنیادی_طاقت_مشق}
\end{figure}

جوابات: (الف) گھڑی کے الٹ \عددی{\SI{3}{\ampere}}؛ (ب) بالائی تار مثبت ہے جبکہ دباو \عددی{\SI{3}{\volt}} ہے۔  
\انتہا{مشق}
%====================

آخر میں دوبارہ اس حقیقت کی نشاندہی کرتے ہیں کہ کسی بھی برقی دور میں پیداوار طاقت اور طاقت کا ضیاع برابر ہوں گے۔

\حصہ{برقی پرزے}
برقی پرزوں کو دو اقسام میں تقسیم کیا جا سکتا ہے۔وہ پرزے جو طاقت پیدا کرتے ہیں \اصطلاح{فعال پرزے}\فرہنگ{فعال پرزہ}\حاشیہب{active components}\فرہنگ{active component} کہلاتے ہیں جبکہ طاقت ضائع کرنے والے پرزوں کو \اصطلاح{انفعال پرزے}\فرہنگ{انفعال پرزہ}\حاشیہب{passive components}\فرہنگ{passive component} کہتے ہیں۔ جنریٹر اور بیٹری فعال پرزوں کی مثال ہے جبکہ مزاحمت، امالہ گیر\حاشیہب{inductor} اور برق گیر\حاشیہب{capacitor} انفعال پرزے ہیں۔

فعال پرزوں پر اس باب میں غور کیا جائے گا جبکہ انفعال پرزوں پر اگلے باب میں تفصیلاً غور کیا جائے گا۔ 

\جزوحصہ{غیر تابع منبع}
\اصطلاح{غیر تابع منبع دباو}\فرہنگ{غیر تابع منبع دباو}\فرہنگ{منبع دباو!غیر تابع}\حاشیہب{independent voltage source}\فرہنگ{independent voltage source} سے مراد ایسی منبع ہے جو، منبع میں سے گزرتی رو کے قطع نظر، اپنے دو سروں کے درمیان مخصوص برقی دباو برقرار رکھتا ہے۔ غیر تابع منبع دباو کی علامت کو شکل \حوالہ{شکل_بنیادی_آزاد_منبع_دباو} میں دکھایا گیا ہے جہاں نقطہ \عددی{A} کے حوالے سے نقطہ \عددی{B} پر \عددی{v(t)} برقی دباو برقرار رہتا ہے۔شکل میں غیر تابع منبع دباو کا دباو بالمقابل رو  \عددی{v-i} خط بھی دکھایا گیا ہے۔اس خط کے مطابق برقی دباو کی قیمت پر برقی رو کا کوئی اثر نہیں پایا جاتا۔

\begin{figure}
\centering
\includegraphics{figBasicIndependentVoltageSource}
\caption{غیر تابع منبع دباو اور اس کا \عددی{v-i} خط۔}
\label{شکل_بنیادی_آزاد_منبع_دباو}
\end{figure}

شکل \حوالہ{شکل_بنیادی_آزاد_منبع_رو} میں \اصطلاح{غیر تابع منبع رو}\فرہنگ{آزاد!منبع رو}\فرہنگ{منبع رو!غیر تابع}\حاشیہب{independent current source}\فرہنگ{independent!current source} کی علامت اور رو بالمقابل دباو \عددی{v-i} خط دکھایا گیا ہے۔غیر تابع منبع رو سے مراد ایسی منبع ہے جو، منبع پر دباو کے قطع نظر،  مخصوص برقی رو برقرار رکھتا ہے۔غیر تابع منبع رو کے دباو بالمقابل رو خط کے تحت منبع پر برقی دباو کے تبدیلی کا منبع کی رو پر کوئی اثر نہیں پایا جاتا۔منبع رو میں مثبت رو کی سمت کو تیر کے نشان سے دکھایا جاتا ہے۔
\begin{figure}
\centering
\includegraphics{figBasicIndependentCurrentSource}
\caption{غیر تابع منبع رو اور اس کا \عددی{v-i} خط۔}
\label{شکل_بنیادی_آزاد_منبع_رو}
\end{figure}

عام استعمال میں منبع بقایا دور کو طاقت فراہم کرتی ہے۔شکل \حوالہ{شکل_بنیادی_طاقت_مشق}-ب میں اگر بیرونی پرزہ منبع ہو تب آپ دیکھ سکتے ہیں کہ منبع کو بھی طاقت فراہم کی جا سکتی ہے۔

منبع محدود صلاحیت کا حامل ہے۔اگرچہ ہم توقع کرتے ہیں کہ منبع دباو کسی بھی قیمت کی برقی رو فراہم کرتے ہوئے پیدا کردہ  برقی دباو برقرار رکھے گا، حقیقت میں کوئی بھی منبع کسی محدود رو کی حد تک ایسا کر پاتا ہے۔

%==============================
\ابتدا{مثال}
شکل \حوالہ{شکل_بنیادی_طاقت_حساب}-الف میں تینوں پرزوں کی طاقت دریافت کریں۔ (اشارہ: سلسلہ وار جڑے پرزوں میں یکساں رو پائی جاتی ہے۔)
\begin{figure}
\centering
\begin{subfigure}{0.5\textwidth}
\centering
\includegraphics{figBasicInternalResistanceExample}
\caption{}
\end{subfigure}%
%
\begin{subfigure}{0.5\textwidth}
\centering
\includegraphics{figBasicInternalResistanceQuiz}
\caption{}
\end{subfigure}%
\caption{طاقت کا حساب۔}
\label{شکل_بنیادی_طاقت_حساب}
\end{figure}

حل:منبع کے مثبت سر سے رو خارج ہو رہی ہے لہٰذا یہ پرزہ طاقت فراہم کر رہا ہے جبکہ بقایا دو پرزوں کے مثبت سر سے رو پرزے میں داخل ہوتی ہے لہٰذا ان دونوں پرزوں میں طاقت ضائع ہوتا ہے۔منبع کی طاقت \عددی{{12\times(-4)=\SI{-48}{\watt}}} ہے جبکہ پرزہ-1 کی طاقت \عددی{5\times 4 =\SI{20}{\watt}} اور پرزہ-2 کی طاقت \عددی{7\times 4=\SI{28}{\watt}} ہے۔آپ دیکھ سکتے ہیں کہ طاقت کی ضیاع \عددی{\SI{20}{\watt}+\SI{28}{\watt}=\SI{48}{\watt}} عین طاقت کی پیداوار کے برابر ہے۔
\انتہا{مثال}
%==========================
\ابتدا{مشق}
شکل \حوالہ{شکل_بنیادی_طاقت_حساب}-ب میں تینوں پرزوں کی طاقت حاصل کریں۔

جوابات:منبع رو کی طاقت \عددی{\SI{-16}{\watt}} ہے۔پرزہ-1 کی طاقت \عددی{\SI{20}{\watt}} ہے۔پرزہ-2 بھی منبع ہے اور اس کی طاقت \عددی{\SI{-4}{\watt}} ہے۔
\انتہا{مشق}
%==========================

\جزوحصہ{تابع منبع}
غیر تابع منبع دباو کی پیدا کردہ دباو کا انحصار منبع سے گزرتی رو پر بالکل نہیں ہوتا۔ اسی طرح غیر تابع منبع رو کی پیدا کردہ رو کا انحصار منبع پر دباو پر بالکل نہیں ہوتا۔اس کے برعکس \اصطلاح{تابع منبع دباو}\فرہنگ{تابع!منبع دباو}\فرہنگ{منبع دباو!تابع}\فرہنگ{دباو!تابع منبع}\حاشیہب{dependent voltage source}\فرہنگ{dependent voltage source} کی پیدا کردہ دباو،  دور میں کسی مخصوص مقام کی رو یا دباو پر منحصر ہوتا ہے۔اسی طرح \اصطلاح{تابع منبع رو}\فرہنگ{تابع!منبع رو}\فرہنگ{منبع رو!تابع}\فرہنگ{رو!تابع منبع}\حاشیہب{dependent current source}\فرہنگ{dependent current source} کی پیدا کردہ رو،  دور میں کسی مخصوص مقام کی رو یا دباو پر منحصر ہوتا ہے۔تابع منبع برقیات کی میدان میں کلیدی کردار ادا کرتے ہیں جہاں برقیاتی پرزہ جات مثلاً \اصطلاح{دو جوڑ ٹرانزسٹر}\فرہنگ{ٹرانزسٹر!دو جوڑ}\حاشیہب{bipolar transistor, BJT}\فرہنگ{transistor, BJT} یا \اصطلاح{میدانی ٹرانزسٹر}\فرہنگ{ٹرانزسٹر!میدانی}\حاشیہب{MOSFET}\فرہنگ{MOSFET} کے  \اصطلاح{ریاضی نمونے}\فرہنگ{ریاضی نمومے}\فرہنگ{نمونہ!ریاضی}\حاشیہب{mathematical model}\فرہنگ{model} تابع منبع سے بنائے جاتے ہیں۔متعدد ٹرانزسٹر پر مبنی برقیاتی ادوار کا حسابی حل انہیں ریاضی نمونوں کی مدد سے حاصل کیا جاتا ہے۔

\begin{figure}
\centering
\begin{subfigure}{0.5\textwidth}
\centering
\includegraphics[scale=0.9]{figBasicDependentVoltageSource}
\caption*{(الف) تابع منبع دباو}
\end{subfigure}%
\begin{subfigure}{0.5\textwidth}
\centering
\includegraphics[scale=0.9]{figBasicDependentCurrentSource}
\caption*{(ب) تابع منبع رو}
\end{subfigure}
%
\begin{subfigure}{0.5\textwidth}
\centering
\includegraphics[scale=0.9]{figBasicDependentResistanceSource}
\caption*{(پ) تابع منبع مزاحمت-نما}
\end{subfigure}%
\begin{subfigure}{0.5\textwidth}
\centering
\includegraphics[scale=0.9]{figBasicDependentConductanceSource}
\caption*{(ت) تابع منبع موصلیت-نما}
\end{subfigure}%
\caption{تابع منبع کے چار اقسام۔}
\label{شکل_بنیادی_تابع_منبع_اقسام}
\end{figure}

غیر تابع منبع کو گول دائرے سے ظاہر کیا جاتا ہے جبکہ تابع منبع کو ہیرا شکل سے ظاہر کیا جاتا ہے۔شکل \حوالہ{شکل_بنیادی_تابع_منبع_اقسام} میں چار اقسام کے تابع منبع دکھائے گئے ہیں۔شکل-الف میں \اصطلاح{تابع منبع دباو}\فرہنگ{تابع!منبع دباو}\حاشیہب{dependent voltage source}\فرہنگ{dependent!voltage source} کی پیدا کردہ دباو کا انحصار بائیں جانب  کے دباو \عددی{v_S} پر ہے۔ یوں \عددی{v_S} \اصطلاح{ضابط دباو}\فرہنگ{ضابط!دباو}\فرہنگ{دباو!ضابط}\حاشیہب{control voltage}\فرہنگ{control voltage} کہلاتا ہے۔یہ منبع \عددی{\mu v_S} دباو پیدا کرتا ہے۔ شکل-ب میں \اصطلاح{تابع منبع رو}\فرہنگ{تابع!منبع رو}\حاشیہب{depended current source}\فرہنگ{depended!current source} کو \عددی{i_S} قابو کرتا ہے۔ان دو اقسام کے منبع کے مستقل \عددی{\mu} اور \عددی{\beta} \اصطلاح{بے بُعد}\فرہنگ{بے بعد}\حاشیہب{dimensionless}\فرہنگ{dimensionless} مقدار ہیں۔شکل-پ میں \عددی{i_s} رو پیدا کردہ دباو کو قابو کرتی ہے۔اس منبع کے مستقل \عددی{r} کا \اصطلاح{بُعد}\فرہنگ{بُعد}\حاشیہب{dimension}\فرہنگ{dimension} \عددی{\si{\volt\per\ampere}} ہے جو عین مزاحمت کی بُعد ہے۔اسی لئے اس منبع کو \اصطلاح{تابع منبع مزاحمت-نما}\فرہنگ{تابع منبع مزاحمت-نما}\حاشیہب{dependent transresistance source}\فرہنگ{dependent!transresistance source} کہا جاتا ہے۔شکل-ت میں \اصطلاح{تابع منبع موصلیت-نما}\فرہنگ{تابع منبع موصلیت-نما}\حاشیہب{dependent transconductance source}\فرہنگ{dependent!transconductance source} کی پیدا کردہ رو کا انحصار \عددی{v_S} پر ہے۔اس منبع کے مستقل \عددی{g} کا بُعد \عددی{\si{\ampere\per\volt}} ہے جو موصلیت کی بھی بُعد ہے۔
%=======================
\ابتدا{مثال}
شکل \حوالہ{شکل_تابع_منبع_دباو_اور_رو_استعمال}-الف میں خارجی دباو اور شکل-ب میں خارجی رو دریافت کریں۔

\begin{figure}
\centering
\begin{subfigure}{0.5\textwidth}
\centering
\includegraphics[scale=0.85]{figBasicDependentVoltageSourceExample}
\caption*{(الف) تابع منبع دباو کی مثال}
\end{subfigure}%
%
\begin{subfigure}{0.5\textwidth}
\centering
\includegraphics[scale=0.9]{figBasicDependentCurrentSourceExample}
\caption*{(ب) تابع منبع رو کی مثال}
\end{subfigure}%
\caption{تابع منبع دباو اور تابع منبع رو کے استعمال کی مثال۔}
\label{شکل_تابع_منبع_دباو_اور_رو_استعمال}
\end{figure} 

حل:شکل-الف میں ضابط دباو \عددی{\SI{0.2}{\volt}} اور منبع کا مستقل \عددی{7} ہے۔یوں پیدا کردہ دباو \عددی{0.2\times 7=\SI{1.4}{\volt}} ہو گا۔شکل-ب میں ضابط رو \عددی{\SI{3}{\milli\ampere}} اور منبع کا مستقل \عددی{12} ہے۔یوں پیدا کردہ رو \عددی{0.003\times 12=\SI{36}{\milli\ampere}} ہو گی۔
\انتہا{مثال}
%====================


اس مثال میں تابع منبع دباو داخلی دباو کو \عددی{7} گنا بڑھاتا ہے گویا منبع بطور \اصطلاح{ایمپلیفائر دباو}\فرہنگ{ایمپلیفائر!دباو}\حاشیہب{voltage amplifier}\فرہنگ{amplifier!voltage} کردار ادا کرتا ہے اور اس ایمپلیفائر کی \اصطلاح{افزائش دباو}\فرہنگ{افزائش!دباو}\حاشیہب{voltage gain}\فرہنگ{voltage gain} \عددی{7} ہے۔اسی طرح شکل-ب میں تابع منبع رو نے داخلی رو کو \عددی{12} گنا بڑھا کر خارج کیا، گویا یہ منبع بطور \اصطلاح{ایمپلیفائر رو}\فرہنگ{ایمپلیفائر!رو}\حاشیہب{current amplifier}\فرہنگ{amplifier!current} کردار ادا کرتا ہے اور اس ایمپلیفائر کی \اصطلاح{افزائش رو}\فرہنگ{افزائش!رو}\حاشیہب{current gain}\فرہنگ{current gain} کی قیمت \عددی{12} ہے۔

شکل \حوالہ{شکل_بنیادی_تابع_منبع_اقسام}-پ بالکل اسی طرح داخلی ضابط رو کی نسبت سے برقی دباو خارج کرتے ہوئے بطور \اصطلاح{ایمپلیفائر مزاحمت-نما}\فرہنگ{ایمپلیفائر!مزاحمت-نما}\حاشیہب{transresistance amplifier}\فرہنگ{amplifier!transresistance} کردار ادا کرتا ہے جہاں منبع کا مستقل \اصطلاح{افزائش مزاحمت-نما}\فرہنگ{افزائش!مزاحمت-نما}\حاشیہب{transresistance gain}\فرہنگ{transresistance gian}  کہلاتا ہے۔شکل \حوالہ{شکل_بنیادی_تابع_منبع_اقسام}-ت بطور \اصطلاح{ایمپلیفائر موصلیت-نما}\فرہنگ{ایمپلیفائر!موصلیت-نما}\حاشیہب{transconductance amplifier}\فرہنگ{amplifier!transconductance} کام کرتا ہے اور اس کے مستقل کو \اصطلاح{افزائش موصلیت-نما}\فرہنگ{افزائش!موصلیت-نما}\حاشیہب{transconductance gain}\فرہنگ{transconductance gain} کہتے ہیں۔

%=================

\ابتدا{مشق}
شکل \حوالہ{شکل_تابع_منبع_دباو_اور_رو_مشق} میں برقی بوجھ کی طاقت دریافت کریں۔
\begin{figure}
\centering
\begin{subfigure}{0.5\textwidth}
\centering
\includegraphics[scale=0.8]{figBasicDependentVoltageSourceExamplePower}
\caption*{(الف) تابع منبع دباو کی مشق}
\end{subfigure}%
%
\begin{subfigure}{0.5\textwidth}
\centering
\includegraphics[scale=0.8]{figBasicDependentCurrentSourceExamplePower}
\caption*{(ب) تابع منبع رو کی مشق}
\end{subfigure}%
\caption{تابع منبع دباو اور تابع منبع رو کے استعمال کی مشق۔}
\label{شکل_تابع_منبع_دباو_اور_رو_مشق}
\end{figure} 

جوابات: (الف): \عددی{\SI{69.3}{\watt}}، (ب) \عددی{\SI{120}{\watt}} 

\انتہا{مشق}
%====================
\ابتدا{مثال}\شناخت{مثال_بنیادی_طاقت_کا_حساب}
شکل \حوالہ{شکل_بنیادی_طاقت_حساب_مثال} میں تمام پرزہ جات کی طاقت دریافت کریں۔

\begin{figure}
\centering
\includegraphics{figBasicExampleTwoSourceThreeResistorPower}
\caption{مثال \حوالہ{مثال_بنیادی_طاقت_کا_حساب} کا دور۔}
\label{شکل_بنیادی_طاقت_حساب_مثال}
\end{figure}

حل: بوجھ-الف میں برقی رو صفر ہے اور اس کے دونوں سروں کے مابین دباو بھی صفر ہے لہٰذا اس کی طاقت \عددی{0\times 0 =\SI{0}{\watt}} ہے۔ بوجھ-ب کی  طاقت \عددی{5\times 1.25=\SI{6.25}{\watt}} ہے۔ بوجھ-پ کی طاقت \عددی{5\times 1=\SI{5}{\watt}} اور بوجھ-ت کی طاقت \عددی{5\times 1.25=\SI{6.25}{\watt}} ہے۔بائیں منبع کی طاقت \عددی{5\times1=\SI{5}{\watt}} جبکہ دائیں منبع کی طاقت \عددی{10\times(-2.25)=\SI{-22.5}{\watt}} ہے۔

کل طاقت کا ضیاع  \عددی{0+6.25+5+6.25+5=\SI{22.5}{\watt}} ہے۔دایاں منبع تمام طاقت پیدا کرتا ہے جبکہ بائیں منبع  کو ازخود طاقت درکار ہے۔ 
\انتہا{مثال}
%======================
\ابتدا{مشق}
شکل \حوالہ{شکل_بنیادی_طاقت_حصول_مشق} کے تمام پرزوں میں طاقت حاصل کریں۔کیا طاقت کی پیدا وار اور اس کا ضیاع برابر ہیں۔

\begin{figure}
\centering
\includegraphics{figBasicPowerExampleMultiSupplies}
\caption{طاقت کے حصول کی مشق۔}
\label{شکل_بنیادی_طاقت_حصول_مشق}
\end{figure}

جوابات:بالترتیب الف تا ٹ: \عددی{\SI{1.5125}{\watt}}، \عددی{\SI{4.5375}{\watt}}، \عددی{\SI{4.05}{\watt}}، \عددی{\SI{3.6}{\watt}}، \عددی{\SI{1.6}{\watt}}؛ منبع دباو کی طاقت \عددی{\SI{-0.3}{\watt}} اور منبع رو کی طاقت \عددی{\SI{-15}{\watt}} ہے۔دور میں کل طاقت کی پیداوار \عددی{\SI{15.3}{\watt}} ہے۔اتنی ہی طاقت پیدا بھی ہوتی ہے لہٰذا دونوں برابر ہیں۔
\انتہا{مشق}
%=======================
\ابتدا{مثال}\شناخت{مثال_بنیادی_بار_اور_رو}
شکل \حوالہ{شکل_بنیادی_بار_بالمقابل_وقت}-الف میں ڈبہ دور دکھایا گیا ہے جس میں برقی بار بھری جا رہی ہے۔برقی بار بالمقابل وقت کا خط شکل-ب میں دیا گیا ہے۔اس خط سے برقی رو بالمقابل وقت کا خط حاصل کریں۔

\begin{figure}
\centering
\begin{subfigure}{1\textwidth}
\centering
\includegraphics{figBasicChargeSuppliedToBox}
\caption{ڈبہ دور}
\end{subfigure}
%
\begin{subfigure}{1\textwidth}
\centering
\includegraphics{figBasicChargeVersusCurrent}
\caption{بار بالمقابل وقت کا خط۔}
\end{subfigure}
\caption{مثال \حوالہ{مثال_بنیادی_بار_اور_رو} کا شکل۔}
\label{شکل_بنیادی_بار_بالمقابل_وقت}
\end{figure}

حل:وقت \عددی{t=0} تا \عددی{t=\SI{0.5}{\micro\second}} تک برقی بار بلا تبدیل ہوئے \عددی{\SI{0.5}{\milli\coulomb}} رہتا ہے لہٰذا  \عددی{\Delta q=0} ہے اور یوں اس دورانیے میں
\begin{align*}
i&=\frac{\Delta q}{\Delta t}=\frac{\SI{0}{\coulomb}}{\SI{0.5}{\micro\second}}=\SI{0}{\ampere} \quad \quad (0<t<\SI{0.5}{\micro\second})
\end{align*} 
ہو گا۔وقت \عددی{t=\SI{0.5}{\micro\second}} تا \عددی{t=\SI{2}{\micro\second}} کے دوران برقی بار \عددی{\SI{0.5}{\milli\coulomb}} سے تبدیل ہو کر \عددی{\SI{2}{\milli\coulomb}} ہو جاتا ہے لہٰذا اس دورانیے کے لئے
\begin{align*}
i&=\frac{\SI{2}{\milli\coulomb}-\SI{0.5}{\milli \coulomb}}{\SI{2}{\micro\second}-\SI{0.5}{\micro\second}}=\SI{1000}{\ampere}\quad \quad (\SI{0.5}{\micro\second}<t<\SI{2}{\micro\second})
\end{align*} 
ہو گا۔اسی طرح بقایا دورانیوں میں
\begin{align*}
i&=\frac{\SI{2.5}{\milli\coulomb}-\SI{2}{\milli \coulomb}}{\SI{3.5}{\micro\second}-\SI{2}{\micro\second}}=\SI{333.33}{\ampere}\quad \quad (\SI{2}{\micro\second}<t<\SI{3.5}{\micro\second})\\
i&=\frac{\SI{2.5}{\milli\coulomb}-\SI{2.5}{\milli \coulomb}}{\SI{4}{\micro\second}-\SI{3.5}{\micro\second}}=\SI{0}{\ampere}\quad \quad (\SI{3.5}{\micro\second}<t<\SI{4}{\micro\second})\\
i&=\frac{\SI{-1}{\milli\coulomb}-\SI{2.5}{\milli \coulomb}}{\SI{5}{\micro\second}-\SI{4}{\micro\second}}=\SI{-3500}{\ampere}\quad \quad (\SI{4}{\micro\second}<t<\SI{5}{\micro\second})\\
i&\frac{\SI{-1}{\milli\coulomb}-(\SI{-1}{\milli\coulomb})}{\SI{6}{\micro\second}-\SI{5}{\micro\second}}=\SI{0}{\ampere}\quad \quad\quad \quad (\SI{5}{\micro\second}<t<\SI{6}{\micro\second})\\
i&=\frac{\SI{-0.5}{\milli\coulomb}-(\SI{-1}{\milli \coulomb})}{\SI{7}{\micro\second}-\SI{6}{\micro\second}}=\SI{500}{\ampere}\quad \quad (\SI{6}{\micro\second}<t<\SI{7}{\micro\second})\\
i&=\SI{0}{\ampere}\quad \quad\quad \quad\quad \quad\quad \quad\quad \quad (\SI{7}{\micro\second}<t)
\end{align*} 
اور اس کے بعد \عددی{i=\SI{0}{\ampere}} ہے۔ان نتائج کو شکل \حوالہ{شکل_بنیادی_برقی_رو_مثال} میں دکھایا گیا ہے۔آپ دیکھ سکتے ہیں کہ بار نہ بدلنے کی صورت میں رو صفر ہوتی ہے۔بڑھتے بار کی صورت میں مثبت رو اور گھٹتے بار کی صورت میں منفی رو پائی جاتی ہے۔

\begin{figure}
\centering
\includegraphics{figBasicChargeVersusCurrentAnswer}
\caption{برقی رو مثال \حوالہ{مثال_بنیادی_بار_اور_رو}}
\label{شکل_بنیادی_برقی_رو_مثال}
\end{figure}
\انتہا{مثال}
%======================
\ابتدا{مثال}
مندرجہ بالا مثال میں طاقت بالمقابل وقت حاصل کریں۔

حل:طاقت \عددی{p=v i} ہوتا ہے۔شکل \حوالہ{شکل_بنیادی_بار_بالمقابل_وقت}-الف سے دباو کی قیمت \عددی{\SI{15}{\volt}} ملتی ہے جبکہ شکل \حوالہ{شکل_بنیادی_برقی_رو_مثال} سے رو کی قیمت مختلف دورانیے کے لئے حاصل کی جا سکتی ہے۔یوں مختلف دورانیے کے طاقت درج ذیل حاصل ہوتے ہیں۔
\begin{align*}
p&=15 \times 0=\SI{0}{\watt} \quad \quad \quad (0<t<\SI{0.5}{\micro\second}) \\
p&=15 \times 1000=\SI{15}{\kilo\watt} \quad \quad \quad (\SI{0.5}{\micro\second}<t<\SI{2}{\micro\second}) \\
p&=15 \times 333.33=\SI{5}{\kilo\watt} \quad \quad \quad (\SI{2}{\micro\second}<t<\SI{3.5}{\micro\second}) \\
p&=15 \times 0=\SI{0}{\watt} \quad \quad \quad \quad\quad (\SI{3.5}{\micro\second}<t<\SI{4}{\micro\second}) \\
p&=15 \times (-3500)=\SI{-52.5}{\kilo\watt} \quad \quad \quad (\SI{4}{\micro\second}<t<\SI{5}{\micro\second}) \\
p&=15 \times 0=\SI{0}{\watt} \quad \quad \quad \quad\quad (\SI{5}{\micro\second}<t<\SI{6}{\micro\second}) \\
p&=15 \times 500=\SI{7.5}{\kilo\watt} \quad \quad \quad (\SI{6}{\micro\second}<t<\SI{7}{\micro\second}) \\
p&=15 \times 0=\SI{0}{\watt} \quad \quad \quad (\SI{7}{\micro\second}<t) 
\end{align*}

\begin{figure}
\centering
\includegraphics[width=\textwidth]{figBasicPowerSuppliedToBox}
\caption{طاقت بالمقابل وقت}
\label{شکل_بنیادی_طاقت_بالمقابل_وقت}
\end{figure}

ان جوابات کو شکل \حوالہ{شکل_بنیادی_طاقت_بالمقابل_وقت} میں دکھایا گیا ہے۔
\انتہا{مثال}
%===================
\ابتدا{مثال}
آج کل \اصطلاح{کمپیوٹر}\فرہنگ{کمپیوٹر}\حاشیہب{computer}\فرہنگ{computer} کا زمانہ ہے اور  یو-ایس-بی\فرہنگ{یو-ایس-بی}\حاشیہب{USB Universal Serial Port}\فرہنگ{USB, universal serial bus} یعنی \اصطلاح{عمومی سلسلہ وار پھاٹک}\فرہنگ{پھاٹک!عمومی سلسلہ وار} کا استعمال عام ہے۔کسی بھی کمپیوٹر یا \اصطلاح{عددی دور}\فرہنگ{عددی دور}\حاشیہب{digital circuit}\فرہنگ{digital circuit} کو \اصطلاح{عددی مواد}\فرہنگ{عددی مواد}\فرہنگ{مواد!عددی}\حاشیہب{digital data}\فرہنگ{digital data}\فرہنگ{data!digital} جن برقی تاروں کے ذریعہ فراہم کیا جاتا ہے وہ کمپیوٹر یا عددی دور کے \اصطلاح{داخلی پھاٹک}\فرہنگ{داخلی پھاٹک}\فرہنگ{پھاٹک!داخلی}\حاشیہب{input port}\فرہنگ{input port} کہلاتے ہیں اور جن تاروں کے ذریعہ کمپیوٹر یا عددی دور سے عددی مواد حاصل کیا جاتا ہے، کمپیوٹر یا عددی دور کے  \اصطلاح{خارجی پھاٹک}\فرہنگ{خارجی پھاٹک}\فرہنگ{پھاٹک!خارجی}\حاشیہب{output port}\فرہنگ{output port} کہلاتے ہیں۔عمومی سلسلہ وار پھاٹک (یو-ایس-بی) پر کمپیوٹر عددی مواد حاصل بھی کر سکتا ہے اور خارج بھی  کر سکتا ہے۔یوں یہ \اصطلاح{داخلی-خارجی پھاٹک}\فرہنگ{داخلی-خارجی پھاٹک}\حاشیہب{input-output port}\فرہنگ{port!input-output} ہے۔اس پھاٹک کی مدد سے کمپیوٹر کے ساتھ بیرونی آلات مثلاً موبائل فون، عددی کیمرہ وغیرہ جوڑے جا سکتے ہیں۔یہ پھاٹک بیرونی آلات کو برقی طاقت فراہم کرنے کی صلاحیت بھی رکھتا ہے۔یہ پھاٹک چار عدد برقی تاروں پر مشتمل ہے جن میں دو تار عددی مواد کے ترسیل اور دو تار برقی طاقت کی فراہمی کے لئے استعمال ہوتے ہیں۔یہ پھاٹک عام حالت میں \عددی{\SI{100}{\milli\ampere}} برقی رو فراہم کر سکتا ہے جبکہ سافٹ وئیر کے ذریعہ پھاٹک سے برقی رو کی فراہمی \عددی{\SI{500}{\milli\ampere}} تک بڑھائی جا سکتی ہے۔

یو-ایس-بی پھاٹک استعمال کرتے ہوئے موبائل کی \اصطلاح{بے بار}\فرہنگ{بے بار}\حاشیہب{discharged}\فرہنگ{discharged} بیٹری میں بار بھرا جاتا ہے۔بیٹری کی استعداد \عددی{\SI{1700}{\milli\ampere\hour}} ہے۔الف) بیٹری کی استعداد کولمب \عددی{\si{\coulomb}} میں حاصل کریں۔ ب) اگر پھاٹک \عددی{\SI{100}{\milli\ampere}} رو فراہم کر رہا ہو تب بیٹری کو مکمل بھرنے میں کتنی دیر لگے گی۔ 

حل:الف) مکمل بھری بیٹری میں کل بار ہی بیٹری کی استعداد ہوتی ہے۔بیٹری کی استعداد کو کولمب \عددی{\si{\coulomb}} کی بجائے \عددی{\si{\ampere\hour}} میں بیان کیا جاتا ہے۔دی گئی بیٹری کی استعداد
\begin{align*}
Q=I\times t=1700 \times 10^{-3} \times 3600=\SI{6120}{\coulomb}
\end{align*}
ہے جہاں ایک گھنٹہ  \عددی{3600} سیکنڈ کے برابر ہے۔ 

ب) یوں \عددی{\SI{100}{\milli\ampere}} کی رو سے بیٹری بھرنے میں
\begin{align*}
t=\frac{6120}{100\times 10^{-3}}=\SI{61200}{\second} =\SI{17}{\hour}
\end{align*}
سترہ گھنٹے درکار ہوں گے۔
\انتہا{مثال}
%=====================
%==========================
\حصہء{سوالات}
\ابتدا{سوال}
ایک تار میں \عددی{\SI{1.5}{\ampere}} رو پائی جاتی ہے۔اس تار پر کسی نقطے سے \عددی{\SI{45}{\second}} میں کتنا بار گزرتا ہے۔

جواب:\عددی{\SI{67.5}{\coulomb}}
\انتہا{سوال}
%======================
\ابتدا{سوال}
ایک تار سے \عددی{\SI{22}{\second}} میں کل \عددی{10^{21}} الیکٹران گزرتے ہیں۔تار میں اوسط رو دریافت کریں۔
 
جواب:\عددی{\SI{7.27}{\ampere}}
\انتہا{سوال}
%======================
\ابتدا{سوال}
\عددی{\SI{20}{\ampere}} بیٹری چارجر کتنی دیر میں \عددی{\SI{4000}{\coulomb}} بار فراہم کرے گا۔

جواب:\عددی{\SI{200}{\second}} 
\انتہا{سوال}
%======================
\ابتدا{سوال}
آسمانی بجلی \عددی{\SI{60}{\micro\second}} کے لئے \عددی{\SI{20}{\kilo\ampere}} فراہم کرتی ہے۔آسمانی بجلی میں کل بار دریافت کریں۔

جواب:\عددی{\SI{1.2}{\coulomb}}
\انتہا{سوال}
%======================
\ابتدا{سوال}
ایک تار میں \عددی{\SI{32}{\second}} میں \عددی{\SI{88}{\coulomb}} بار گزرتا ہے۔ تار میں رو دریافت کریں۔

جواب:\عددی{\SI{2.75}{\ampere}}
\انتہا{سوال}
%====================
\ابتدا{سوال}\شناخت{سوال_مزاحمتی_ضاع_الٹ_دباو}
شکل \حوالہ{شکل_سوال_مزاحمتی_ضاع_الٹ_دباو} میں نقطہ-الف سے نقطہ-ب تک بیرونی پرزے میں دس کولومب کا بار گزرتا ہے جبکہ پرزے میں توانائی کا ضیاع پچاس جاول ہے۔دباو \عددی{V_1} حاصل کریں۔
\begin{figure}
\centering
\begin{tikzpicture}[american voltages]
\draw(0,0) rectangle ++(-0.5,\yy+0.5);
\draw(0,0.25) to [short,-o]++(\xx/2,0)node[below]{\text{الف}}  to [short]++(\xx/2,0)to [european resistor]++(0,\yy) to [short,-o]++(-\xx/2,0)node[above]{\text{ب}} to [short]++(-\xx/2,0);
\draw (\xx/2,\yy+0.25) to [open,v={$V_1$}]++(0,-\yy);
\end{tikzpicture}
\caption{سوال \حوالہ{سوال_مزاحمتی_ضاع_الٹ_دباو} کا دور۔}
\label{شکل_سوال_مزاحمتی_ضاع_الٹ_دباو}
\end{figure}

جواب:\عددی{V_1=\SI{-5}{\volt}}
\انتہا{سوال}
%=============
\ابتدا{سوال}\شناخت{سوال_بنیادی_ترسیم_رو_بار}
ایک پرزے کا رو بالمقابل وقت ترسیم شکل \حوالہ{شکل_سوال_بنیادی_ترسیم_رو_بار} میں دیا گیا ہے۔ان بیس سیکنڈ دورانیے میں کتنا بار پرزے میں داخل ہو گا۔
\begin{figure}
\centering
\begin{tikzpicture}
\centering
\begin{axis}[small,axis lines*=middle,xlabel={$t\,(\si{\second})$},ylabel={$i(t)\, (\si{\milli\ampere})$},ylabel style={rotate=-90},ytick={2},yticklabels={$20$},xtick={2,4,6},xticklabels={$2$,$4$,$6$}]
\addplot[] plot coordinates{(0,2) (4,2) (6,0)};
\end{axis}
\end{tikzpicture}
\caption{سوال \حوالہ{سوال_بنیادی_ترسیم_رو_بار} کا ترسیم۔}
\label{شکل_سوال_بنیادی_ترسیم_رو_بار}
\end{figure}

جواب:\عددی{\SI{100}{\milli\coulomb}}
\انتہا{سوال}
%==============
\ابتدا{سوال}
ایک پرزے کے مثبت سر سے \عددی{q(t)=22e^{-5t}\,\si{\milli\ampere}} بار داخل ہوتا ہے جبکہ پرزے پر دباو \عددی{v(t)=15e^{-2t}\,\si{\volt}} ہے۔دورانیے \عددی{0 \le t \le \SI{3}{\second}} میں پرزے کو کتنی توانائی منتقل ہوتی ہے۔

جواب:\عددی{\SI{47.14}{\milli\joule}}
\انتہا{سوال}
%=================
\ابتدا{سوال}\شناخت{سوال_بنیادی_طاقت_وقت}
ایک پرزے کو منتقل توانائی بالمقابل وقت ترسیم شکل \حوالہ{شکل_سوال_بنیادی_طاقت_وقت} میں دیا گیا ہے جبکہ پرزے پر \عددی{\SI{2}{\volt}} دباو پایا جاتا ہے۔ان تیس سیکنڈ دورانیے میں پرزے کو کتنا بار فراہم کیا گیا ہے۔
\begin{figure}
\centering
\begin{tikzpicture}
\centering
\begin{axis}[small,axis lines*=middle,xlabel={$t\,(\si{\second})$},ylabel={$w(t)\, (\si{\milli\joule})$},ylabel style={rotate=-90},ytick={20,30},yticklabels={$20$,$30$},xtick={8,12,15,20,30},xticklabels={$8$,$12$,$15$,$20$,$30$}]
\addplot[] plot coordinates{(0,0) (8,20) (12,20) (15,30) (20,30) (30,0)};
\end{axis}
\end{tikzpicture}
\caption{سوال \حوالہ{سوال_بنیادی_طاقت_وقت} کا ترسیم۔}
\label{شکل_سوال_بنیادی_طاقت_وقت}
\end{figure}

جواب:\عددی{\SI{107}{\milli\coulomb}}
\انتہا{سوال}
%==============
\ابتدا{سوال}\شناخت{سوال_بنیادی_بار_وقت}
ایک پرزے کے مثبت سر پر داخلی بار بالمقابل وقت ترسیم شکل \حوالہ{شکل_سوال_بنیادی_بار_وقت}-الف میں دیا گیا ہے جبکہ پرزے پر \عددی{\SI{6}{\volt}} دباو پایا جاتا ہے۔ان بیس سیکنڈ دورانیے میں پرزے کو کتنی توانائی فراہم کی جاتی ہے۔دور کو شکل-ب میں دکھایا گیا ہے۔
\begin{figure}
\centering
\begin{subfigure}{0.7\textwidth}
\centering
\begin{tikzpicture}
\centering
\begin{axis}[small,xlabel={$t\,(\si{\second})$},ylabel={$q(t)\, (\si{\milli\coulomb})$},ylabel style={rotate=-90},ytick={-10,0,4,6},yticklabels={$-10$,$0$,$4$,$6$},xtick={0,5,7,9,12,18,20},xticklabels={$0$,$5$,$7$,$9$,$12$,$18$,$20$}]
\addplot[] plot coordinates{(0,4) (5,4) (7,0) (9,0) (12,-10) (18,6) (20,6)};
\end{axis}
\end{tikzpicture}
\caption*{(الف)}
\end{subfigure}%
\begin{subfigure}{0.3\textwidth}
\centering
\begin{tikzpicture}
\draw(0,0) rectangle ++(\xx/2,\yy+0.5);
\draw(0,0.25) to [short]++(-\xx/2,0) to [american voltage source,l={$\SI{6}{\volt}$}]++(0,\yy) to [short,i={$i(t)$}]++(\xx/2,0);
\draw(\xx/4,\yy/2+0.25)node{ڈبہ};
\end{tikzpicture}
\caption*{(ب)}
\end{subfigure}
\caption{سوال \حوالہ{سوال_بنیادی_بار_وقت} کا ترسیم۔}
\label{شکل_سوال_بنیادی_بار_وقت}
\end{figure}

جواب:\عددی{\SI{159}{\milli\joule}}
\انتہا{سوال}
%=======================
\ابتدا{سوال}\شناخت{سوال_بنیادی_توانائی_وقت}
ایک ڈبے کو فراہم طاقت بالمقابل وقت ترسیم شکل \حوالہ{شکل_سوال_بنیادی_توانائی_وقت}-الف میں دیا گیا ہے۔ان تیس سیکنڈ دورانیے میں ڈبے کو کتنی توانائی فراہم کی گئی؟۔دور کو شکل-ب میں دکھایا گیا ہے۔
\begin{figure}
\centering
\begin{subfigure}{0.7\textwidth}
\centering
\begin{tikzpicture}
\centering
\begin{axis}[small,xlabel={$t\,(\si{\second})$},ylabel={$p(t)\, (\si{\watt})$},ylabel style={rotate=-90},ytick={-6,0,10},
yticklabels={$-6$,$0$,$10$},xtick={0,5,12,20,25,30},xticklabels={$0$,$5$,$12$,$20$,$25$,$30$}]
\addplot[] plot coordinates{(-1,0) (0,0) (5,10) (12,10) (20,0) (25,0) (25,-6) (30,0)(32,0)};
\end{axis}
\end{tikzpicture}
\caption*{(الف)}
\end{subfigure}%
\begin{subfigure}{0.3\textwidth}
\centering
\begin{tikzpicture}
\draw(0,0) rectangle ++(\xx/2,\yy+0.5);
\draw(0,0.25) to [short]++(-\xx/2,0) to [american voltage source,l={$\SI{8}{\volt}$}]++(0,\yy) to [short,i={$i(t)$}]++(\xx/2,0);
\draw(\xx/4,\yy/2+0.25)node{ڈبہ};
\end{tikzpicture}
\caption*{(ب)}
\end{subfigure}
\caption{سوال \حوالہ{سوال_بنیادی_توانائی_وقت} کا ترسیم۔}
\label{شکل_سوال_بنیادی_توانائی_وقت}
\end{figure}

جواب:\عددی{\SI{120}{\joule}}
\انتہا{سوال}
%===============
\ابتدا{سوال}\شناخت{سوال_بنیادی_طاقت_مہیا_حاصل_الف}
شکل \حوالہ{شکل_بنیادی_طاقت_مہیا_حاصل_الف} میں پرزہ الف کو مہیا یا اس سے حاصل طاقت درج ذیل صورتوں میں دریافت کریں۔
\begin{itemize}
\item
$V_1=\SI{6}{\volt}$ اور $I=\SI{2}{\ampere}$ ہیں۔
\item
$V_1=\SI{-3}{\volt}$ اور $I=\SI{7}{\ampere}$ ہیں۔
\item
$V_1=\SI{5}{\volt}$ اور $I=\SI{-4}{\ampere}$ ہیں۔
\item
$V_1=\SI{-4}{\volt}$ اور $I=\SI{-2}{\ampere}$ ہیں۔
\end{itemize}
%
\begin{figure}
\centering
\begin{tikzpicture}[american voltages]
\draw(0,-0.25) rectangle ++(-0.5,\yy+0.5);
\draw(0,\yy) to [short,-o]++(\xx/2,0) to [short,o-,i={$I$}]++(\xx/2,0) to [european resistor,l={الف}]++(0,-\yy) to [short,-o]++(-\xx/2,0) to [short,o-]++(-\xx/2,0);
\draw(\xx/2,\yy) to [open,v={$V_1$}]++(0,-\yy);
\end{tikzpicture}
\caption{سوال \حوالہ{سوال_بنیادی_طاقت_مہیا_حاصل_الف} کا دور۔}
\label{شکل_بنیادی_طاقت_مہیا_حاصل_الف}
\end{figure}

جوابات:\عددی{\SI{12}{\watt}}، \عددی{\SI{-21}{\watt}}، \عددی{\SI{-20}{\watt}}، \عددی{\SI{8}{\watt}} 
\انتہا{سوال}
%===============
\ابتدا{سوال}\شناخت{سوال_بنیادی_طاقت_مہیا_حاصل_ب}
شکل \حوالہ{شکل_سوال_بنیادی_طاقت_مہیا_حاصل_ب} میں پرزہ الف اور ب کو مہیا یا حاصل طاقت  دریافت کریں۔
%
\begin{figure}
\centering
\begin{subfigure}{0.5\textwidth}
\centering
\begin{tikzpicture}[american voltages]
\draw(0,-0.25) rectangle ++(-0.5,\yy+0.5);
\draw(0,\yy) to [short,-o]++(\xx/2,0) to [short,o-,i={$\SI{2}{\ampere}$}]++(\xx/2,0) to [european resistor,v_>={$\SI{6}{\volt}$},l={الف}]++(0,-\yy) to [short,-o]++(-\xx/2,0) to [short,o-]++(-\xx/2,0);
\end{tikzpicture}
\caption*{(الف)}
\end{subfigure}%
\begin{subfigure}{0.5\textwidth}
\centering
\begin{tikzpicture}[american voltages]
\draw(0,-0.25) rectangle ++(-0.5,\yy+0.5);
\draw(0,\yy) to [short,-o]++(\xx/2,0) to [short,o-,i<={$\SI{3}{\ampere}$}]++(\xx/2,0) to [european resistor,v_>={$\SI{2}{\volt}$},l={ب}]++(0,-\yy) to [short,-o]++(-\xx/2,0) to [short,o-]++(-\xx/2,0);
\end{tikzpicture}
\caption*{(ب)}
\end{subfigure}%
\caption{سوال \حوالہ{سوال_بنیادی_طاقت_مہیا_حاصل_ب} کا دور۔}
\label{شکل_سوال_بنیادی_طاقت_مہیا_حاصل_ب}
\end{figure}

جوابات:پرزہ الف سے \عددی{\SI{12}{\watt}} طاقت حاصل کی جا رہی ہے۔ پرزہ ب کو \عددی{\SI{6}{\watt}} فراہم کی جا رہی ہے۔
\انتہا{سوال}
%===============
\ابتدا{سوال}\شناخت{سوال_بنیادی_طاقت_مہیا_حاصل_پ}
شکل \حوالہ{شکل_سوال_بنیادی_طاقت_مہیا_حاصل_پ} میں پرزہ الف کو \عددی{\SI{20}{\watt}} فراہم کی جا رہی ہے جبکہ پرزہ ب سے \عددی{\SI{12}{\watt}} حاصل کیا جا رہا ہے۔دباو \عددی{V_x} اور \عددی{V_y} دریافت کریں۔
%
\begin{figure}
\centering
\begin{subfigure}{0.5\textwidth}
\centering
\begin{tikzpicture}[american voltages]
\draw(0,-0.25) rectangle ++(-0.5,\yy+0.5);
\draw(0,\yy) to [short,-o]++(\xx/2,0) to [short,o-,i={$\SI{5}{\ampere}$}]++(\xx/2,0) to [european resistor,v_>={$V_x$},l={الف}]++(0,-\yy) to [short,-o]++(-\xx/2,0) to [short,o-]++(-\xx/2,0);
\end{tikzpicture}
\caption*{(الف)}
\end{subfigure}%
\begin{subfigure}{0.5\textwidth}
\centering
\begin{tikzpicture}[american voltages]
\draw(0,-0.25) rectangle ++(-0.5,\yy+0.5);
\draw(0,\yy) to [short,-o]++(\xx/2,0) to [short,o-,i<={$\SI{3}{\ampere}$}]++(\xx/2,0) to [european resistor,v_>={$V_y$},l={ب}]++(0,-\yy) to [short,-o]++(-\xx/2,0) to [short,o-]++(-\xx/2,0);
\end{tikzpicture}
\caption*{(ب)}
\end{subfigure}%
\caption{سوال \حوالہ{سوال_بنیادی_طاقت_مہیا_حاصل_پ} کا دور۔}
\label{شکل_سوال_بنیادی_طاقت_مہیا_حاصل_پ}
\end{figure}
س
جوابات:\عددی{V_x=\SI{-4}{\volt}}، \عددی{V_y=\SI{-4}{\volt}}
\انتہا{سوال}
%===============
\ابتدا{سوال}\شناخت{سوال_بنیادی_طاقت_مہیا_حاصل_ت}
شکل \حوالہ{شکل_سوال_بنیادی_طاقت_مہیا_حاصل_ت} میں پرزہ الف کو \عددی{\SI{48}{\watt}} فراہم کی جا رہی ہے جبکہ پرزہ ب سے \عددی{\SI{36}{\watt}} حاصل کی جا رہی ہے۔رو \عددی{I_x} اور \عددی{I_y} دریافت کریں۔
%
\begin{figure}
\centering
\begin{subfigure}{0.5\textwidth}
\centering
\begin{tikzpicture}[american voltages]
\draw(0,-0.25) rectangle ++(-0.5,\yy+0.5);
\draw(0,\yy) to [short,-o]++(\xx/2,0) to [short,o-]++(\xx/2,0) to [european resistor,v_>={$\SI{12}{\volt}$},l={الف}]++(0,-\yy) to [short,-o,i={$I_x$}]++(-\xx/2,0) to [short,o-]++(-\xx/2,0);
\end{tikzpicture}
\caption*{(الف)}
\end{subfigure}%
\begin{subfigure}{0.5\textwidth}
\centering
\begin{tikzpicture}[american voltages]
\draw(0,-0.25) rectangle ++(-0.5,\yy+0.5);
\draw(0,\yy) to [short,-o]++(\xx/2,0) to [short,o-]++(\xx/2,0) to [european resistor,v_>={$\SI{6}{\volt}$},l={ب}]++(0,-\yy) to [short,-o,i<={$I_y$}]++(-\xx/2,0) to [short,o-]++(-\xx/2,0);
\end{tikzpicture}
\caption*{(ب)}
\end{subfigure}%
\caption{سوال \حوالہ{سوال_بنیادی_طاقت_مہیا_حاصل_ت} کا دور۔}
\label{شکل_سوال_بنیادی_طاقت_مہیا_حاصل_ت}
\end{figure}

جوابات:\عددی{I_x=\SI{-4}{\ampere}}، \عددی{I_y=\SI{-6}{\ampere}}
\انتہا{سوال}
%===============
\ابتدا{سوال}\شناخت{سوال_بنیادی_طاقت_مہیا_حاصل_ٹ}
شکل \حوالہ{شکل_سوال_بنیادی_طاقت_مہیا_حاصل_ٹ}-الف میں ایک پرزے کو \عددی{\SI{36}{\watt}} فراہم کئے جاتے ہیں جبکہ پرزے پر دباو \عددی{\SI{12}{\volt}} ہے۔پرزہ الف کی طاقت دریافت کریں۔کیا اس پرزے کو طاقت فراہم کی جا رہی ہے؟ شکل-ب میں پرزہ ب کے لئے بھی حل کریں۔
%
\begin{figure}
\centering
\begin{subfigure}{0.5\textwidth}
\centering
\begin{tikzpicture}[american voltages]
\draw(0,-0.25) rectangle ++(-0.5,\yy+0.5);
\draw(0,\yy) to [short,-o]++(\xx/2,0) to [european resistor,o-,v^<={$\substack{\SI{36}{\watt} \\ \SI{12}{\volt}}$}]++(\xx,0) to
 [european resistor,v^<={$\SI{6}{\volt}$},l_={الف}]++(0,-\yy) to [short,-o]++(-\xx,0) to [short,o-]++(-\xx/2,0);
\end{tikzpicture}
\caption*{(الف)}
\end{subfigure}%
\begin{subfigure}{0.5\textwidth}
\centering
\begin{tikzpicture}[american voltages]
\draw(0,-0.25) rectangle ++(-0.5,\yy+0.5);
\draw(0,\yy) to [short,-o]++(\xx/2,0) to [european resistor,o-,v^<={$\substack{\SI{24}{\watt} \\ \SI{6}{\volt}}$}]++(\xx,0) to
 [european resistor,v^>={$\SI{5}{\volt}$},l_={ب}]++(0,-\yy) to [short,-o]++(-\xx,0) to [short,o-]++(-\xx/2,0);
\end{tikzpicture}
\caption*{(ب)}
\end{subfigure}%
\caption{سوال \حوالہ{سوال_بنیادی_طاقت_مہیا_حاصل_ٹ} کا دور۔}
\label{شکل_سوال_بنیادی_طاقت_مہیا_حاصل_ٹ}
\end{figure}

جوابات:پرزہ الف کو \عددی{\SI{18}{\watt}} فراہم کی جاتی ہے جبکہ پرزہ ب سے \عددی{\SI{20}{\watt}} حاصل کیا جاتا ہے۔
\انتہا{سوال}
%===============
\ابتدا{سوال}\شناخت{سوال_بنیادی_طاقت_مہیا_حاصل_ث}
شکل \حوالہ{شکل_سوال_بنیادی_طاقت_مہیا_حاصل_ث}-الف میں  ایک پرزے کو \عددی{\SI{10}{\watt}} فراہم کئے جاتے ہیں جبکہ پرزے پر دباو \عددی{\SI{2}{\volt}} ہے۔پرزہ الف کی طاقت دریافت کریں۔کیا اس پرزے کو طاقت فراہم کی جا رہی ہے؟ شکل-ب کو بھی حل کریں۔
%
\begin{figure}
\centering
\begin{subfigure}{0.5\textwidth}
\centering
\begin{tikzpicture}[american voltages]
\draw(0,0) to [american current source,l={$\SI{8}{\ampere}$}]++(0,\yy) to [short]++(1.5*\xx,0) to [european resistor,l={$\substack{\SI{10}{\watt} \\ \SI{2}{\volt}}$}]++(0,-\yy) to [short]++(-1.5*\xx,0);
\draw(3/4*\xx,0) to [european resistor,*-*,l={الف}]++(0,\yy);
\end{tikzpicture}
\caption*{(الف)}
\end{subfigure}%
\begin{subfigure}{0.5\textwidth}
\centering
\begin{tikzpicture}[american voltages]
\draw(0,0) to [american current source,l={$\SI{6}{\ampere}$}]++(0,\yy) to [short]++(1.5*\xx,0) to [european resistor,i={$\SI{2}{\ampere}$},l={$\SI{12}{\watt}$}]++(0,-\yy) to [short]++(-1.5*\xx,0);
\draw(3/4*\xx,0) to [european resistor,*-*,l={ب}]++(0,\yy);
\end{tikzpicture}
\caption*{(ب)}
\end{subfigure}%
\caption{سوال \حوالہ{سوال_بنیادی_طاقت_مہیا_حاصل_ث} کا دور۔}
\label{شکل_سوال_بنیادی_طاقت_مہیا_حاصل_ث}
\end{figure}

جوابات:پرزہ الف کو \عددی{\SI{2}{\watt}} فراہم کئے جاتے ہیں۔ پرزہ ب کو \عددی{\SI{24}{\watt}} فراہم کئے جاتے ہیں۔
\انتہا{سوال}
%===============
\ابتدا{سوال}\شناخت{سوال_بنیادی_طاقت_مہیا_حاصل_ج}
شکل \حوالہ{شکل_بنیادی_سوال_طاقت_مہیا_حاصل_ج} میں پرزہ الف اور ب کی طاقر دریافت کریں۔ 
%
\begin{figure}
\centering
\begin{subfigure}{0.5\textwidth}
\centering
\begin{tikzpicture}[american voltages]
\draw (0,0) to [american current source,i={$\SI{4}{\ampere}$},v_>={$\SI{9}{\volt}$}]++(0,\yy) to [european resistor,l={الف}] ++(\xx,0);
\draw(0,0) to [short]++(\xx,0) to [american voltage source,l_={$\SI{12}{\volt}$}]++(0,\yy);
\end{tikzpicture}
\caption*{(الف)}
\end{subfigure}%
\begin{subfigure}{0.5\textwidth}
\centering
\begin{tikzpicture}[american voltages]
\draw (0,0) to [american current source,i={$\SI{4}{\ampere}$},v_<={$\SI{9}{\volt}$}]++(0,\yy) to [european resistor,l={ب}] ++(\xx,0);
\draw(0,0) to [short]++(\xx,0) to [american voltage source,l_={$\SI{12}{\volt}$}]++(0,\yy);
\end{tikzpicture}
\caption*{(ب)}
\end{subfigure}%
\caption{سوال \حوالہ{سوال_بنیادی_طاقت_مہیا_حاصل_ج} کا دور۔}
\label{شکل_بنیادی_سوال_طاقت_مہیا_حاصل_ج}
\end{figure}

جوابات:پرزہ الف \عددی{\SI{12}{\volt}} فراہم کرتا ہے۔ پرزہ ب سے \عددی{\SI{84}{\watt}} حاصل کیا جاتا  ہے۔
\انتہا{سوال}
%==============
\ابتدا{سوال}\شناخت{سوال_بنیادی_طاقت_مہیا_حاصل_چ}
شکل \حوالہ{شکل_بنیادی_سوال_طاقت_مہیا_حاصل_چ}-الف میں پرزہ الف \عددی{\SI{6}{\watt}} فراہم کرتا ہے۔اس میں رو کی مقدار اور سمت دریافت کریں۔شکل-ب میں پرزہ ب کو \عددی{\SI{12}{\watt}} طاقت فرہم کی جاتی ہے۔پرزہ ب میں رو دریافت کریں۔
%
\begin{figure}
\centering
\begin{subfigure}{0.5\textwidth}
\centering
\begin{tikzpicture}[american voltages]
\draw (0,0) to [american voltage source,l={$\SI{7}{\volt}$}]++(0,\yy) to [european resistor,l={الف}] ++(\xx,0);
\draw(0,0) to [short]++(\xx,0) to [american voltage source,l_={$\SI{10}{\volt}$}]++(0,\yy);
\end{tikzpicture}
\caption*{(الف)}
\end{subfigure}%
\begin{subfigure}{0.5\textwidth}
\centering
\begin{tikzpicture}[american voltages]
\draw (0,0) to [american voltage source,l={$\SI{7}{\volt}$}]++(0,\yy) to [european resistor,l={ب}] ++(\xx,0);
\draw(0,0) to [short]++(\xx,0) to [american voltage source,l_={$\SI{10}{\volt}$}]++(0,\yy);
\end{tikzpicture}
\caption*{(ب)}
\end{subfigure}%
\caption{سوال \حوالہ{سوال_بنیادی_طاقت_مہیا_حاصل_چ} کا دور۔}
\label{شکل_بنیادی_سوال_طاقت_مہیا_حاصل_چ}
\end{figure}

جوابات:پرزہ الف کے دائیں سر سے \عددی{\SI{2}{\ampere}} نکل کر \عددی{\SI{10}{\volt}} کے منبع میں داخل ہوتی ہے۔ پرزہ ب میں \عددی{\SI{4}{\ampere}} پائی جاتی ہے جو پرزے میں دائیں سے بائیں جانب رواں ہے۔ 
\انتہا{سوال}
%===============
\ابتدا{سوال}\شناخت{سوال_بنیادی_طاقت_مہیا_حاصل_ح}
شکل \حوالہ{شکل_بنیادی_سوال_طاقت_مہیا_حاصل_ح} میں پرزہ الف اور ب کی طاقت دریافت کریں۔
%
\begin{figure}
\centering
\begin{tikzpicture}[american voltages]
\draw (0,0) to [american controlled voltage source,l={$4 I_x $}]++(0,\yy) to [american voltage source,l={$\SI{20}{\volt}$}]++(\xx,0)
 to [european resistor,v^<={$\SI{18}{\volt}$},l_={الف}]++(\xx,0) to [european resistor,v^<={$\SI{22}{\volt}$},l_={ب}]++(0,-\yy) to [short,i={$\SI{5}{\ampere}$}] (0,0);
\end{tikzpicture}
\caption{سوال \حوالہ{سوال_بنیادی_طاقت_مہیا_حاصل_ح} کا دور۔}
\label{شکل_بنیادی_سوال_طاقت_مہیا_حاصل_ح}
\end{figure}

جوابات: پرزہ الف کو \عددی{\SI{90}{\watt}} مہیا کیا جاتا ہے۔ پرزہ ب کو \عددی{\SI{110}{\watt}} مہیا کیا جاتا ہے۔
\انتہا{سوال}
%===============
\ابتدا{سوال}\شناخت{سوال_بنیادی_طاقت_مہیا_حاصل_خ}
شکل \حوالہ{شکل_بنیادی_سوال_طاقت_مہیا_حاصل_خ} میں پرزہ الف، ب اور پ کی طاقت دریافت کریں۔
%
\begin{figure}
\centering
\begin{tikzpicture}[american voltages]
\draw(0,0) to [american voltage source,l={$\SI{36}{\volt}$},i={$\SI{5}{\ampere}$}]++(0,\yy) to [european resistor,v^<={$\SI{16}{\volt}$},l_={الف}]++(\xx,0) to [american voltage source,v^>={$\SI{6}{\volt}$} ,i={$\SI{2}{\ampere}$}]++(\xx,0) to [european resistor,v={$\SI{26}{\volt}$},l={پ}]++(0,-\yy) to [short] (0,0);
\draw(\xx,\yy) to [european resistor,*-*,v={$\SI{20}{\volt}$},i={$\SI{3}{\ampere}$},l={ب}] ++(0,-\yy);
\end{tikzpicture}
\caption{سوال \حوالہ{سوال_بنیادی_طاقت_مہیا_حاصل_خ} کا دور۔}
\label{شکل_بنیادی_سوال_طاقت_مہیا_حاصل_خ}
\end{figure}

جوابات: تینوں پرزوں کو طاقت فراہم کی جاتی ہے۔الف کو \عددی{\SI{80}{\watt}}، ب  کو \عددی{\SI{60}{\watt}} اور  پ کو \عددی{\SI{52}{\watt}} طاقت فراہم کی جاتی ہے۔ 
\انتہا{سوال}
%===============
\ابتدا{سوال}\شناخت{سوال_بنیادی_طاقت_مہیا_حاصل_د}
شکل \حوالہ{شکل_بنیادی_سوال_طاقت_مہیا_حاصل_د} میں رو \عددی{I_x} دریافت کریں۔
%
\begin{figure}
\centering
\begin{tikzpicture}[american voltages]
\draw(0,0) to [american voltage source,l={$\SI{36}{\volt}$},i={$I_x$}]++(0,\yy) to [european resistor,v^<={$\SI{16}{\volt}$},l_={الف}]++(\xx,0) to [american voltage source,v^>={$\SI{6}{\volt}$} ,i={$\SI{6}{\ampere}$}]++(\xx,0) to [european resistor,v={$\SI{26}{\volt}$},l={پ}]++(0,-\yy) to [short] (0,0);
\draw(\xx,\yy) to [european resistor,*-*,v={$\SI{20}{\volt}$},i={$\SI{4}{\ampere}$},l={ب}] ++(0,-\yy);
\end{tikzpicture}
\caption{سوال \حوالہ{سوال_بنیادی_طاقت_مہیا_حاصل_د} کا دور۔}
\label{شکل_بنیادی_سوال_طاقت_مہیا_حاصل_د}
\end{figure}

جوابات: \عددی{I_x=\SI{10}{\ampere}}
\انتہا{سوال}
%===============
\ابتدا{سوال}\شناخت{سوال_بنیادی_طاقت_مہیا_حاصل_ڈ}
شکل \حوالہ{شکل_بنیادی_سوال_طاقت_مہیا_حاصل_ڈ} میں پرزہ الف اور ب کا طاقت دریافت کریں۔
%
\begin{figure}
\centering
\begin{tikzpicture}[american voltages]
\draw(0,0) to [american voltage source,l={$\SI{36}{\volt}$},i={$I_x$}]++(0,\yy) to [european resistor,v^<={$\SI{16}{\volt}$},l_={الف}]++(\xx,0) to [american voltage source,v^>={$\SI{6}{\volt}$} ,i<={$\SI{6}{\ampere}$}]++(\xx,0) to [european resistor,v={$\SI{26}{\volt}$},l={پ}]++(0,-\yy) to [short] (0,0);
\draw(\xx,\yy) to [american controlled current source,*-*,l={$4I_x$}] ++(0,-\yy);
\end{tikzpicture}
\caption{سوال \حوالہ{سوال_بنیادی_طاقت_مہیا_حاصل_ڈ} کا دور۔}
\label{شکل_بنیادی_سوال_طاقت_مہیا_حاصل_ڈ}
\end{figure}

جوابات: پرزہ الف کو \عددی{\SI{32}{\watt}} فراہم کیا جاتا ہے جبکہ پرزہ ب سے \عددی{\SI{156}{\watt}} حاصل ہوتا ہے۔
\انتہا{سوال}
%===============
\ابتدا{سوال}\شناخت{سوال_بنیادی_طاقت_مہیا_حاصل_ذ}
شکل \حوالہ{شکل_بنیادی_سوال_طاقت_مہیا_حاصل_ذ}-الف میں پرزہ الف کا طاقت دریافت کریں۔شکل-ب میں پرزہ ب کی رو، دباو اور طاقت دریافت کریں۔
%
\begin{figure}
\centering
\begin{subfigure}{0.5\textwidth}
\centering
\begin{tikzpicture}[american voltages]
\draw(0,0) to [european resistor,l={$\SI{36}{\watt}$}]++(0,\y) to [european resistor,l={$\SI{12}{\watt}$}]++(\x,0) to [european resistor,l={الف}] ++(0,-\y) to [short] (0,0);
\end{tikzpicture}
\caption*{الف}
\end{subfigure}%
\begin{subfigure}{0.5\textwidth}
\centering
\begin{tikzpicture}[american voltages]
\draw(0,0) to [european resistor,l_={$\SI{24}{\watt}$},v^>={$\SI{6}{\volt}$}]++(0,\y) to [european resistor,l_={$\SI{12}{\watt}$},v^<={$\SI{2}{\volt}$}]++(\x,0) to [european resistor,l={ب}] ++(0,-\y) to [short] (0,0);
\end{tikzpicture}
\caption*{ب}
\end{subfigure}
\caption{سوال \حوالہ{سوال_بنیادی_طاقت_مہیا_حاصل_ذ} کا دور۔}
\label{شکل_بنیادی_سوال_طاقت_مہیا_حاصل_ذ}
\end{figure}

جوابات: پرزہ الف کو \عددی{\SI{24}{\watt}} فراہم کیا جاتا ہے۔پرزہ ب کی رو \عددی{\SI{4}{\ampere}}، دباو \عددی{\SI{4}{\volt}} اور اس کو فراہم کردہ طاقت \عددی{\SI{16}{\watt}} ہے۔
\انتہا{سوال}
%===============
\ابتدا{سوال}\شناخت{سوال_بنیادی_طاقت_مہیا_حاصل_ر}
شکل \حوالہ{شکل_بنیادی_سوال_طاقت_مہیا_حاصل_ر}-الف میں پرزہ الف کا طاقت دریافت کریں۔شکل-ب میں پرزہ ب کی رو، دباو اور طاقت دریافت کریں۔
%
\begin{figure}
\centering
\begin{subfigure}{0.5\textwidth}
\centering
\begin{tikzpicture}[american voltages]
\draw(0,0) to [european resistor,l_={$\SI{3}{\watt}$}]++(0,\y) to [european resistor,l={$\SI{-12}{\watt}$}]++(\x,0) to [european resistor,l={الف}] ++(0,-\y) to [short] (0,0);
\draw(\x,\y) to [european resistor,*-,l={$\SI{5}{\watt}$}]++(\x,0) to [european resistor,l_={$\SI{13}{\watt}$}]++(0,-\y) to [short,-*]++(-\x,0);
\end{tikzpicture}
\caption*{الف}
\end{subfigure}%
\begin{subfigure}{0.5\textwidth}
\centering
\begin{tikzpicture}[american voltages]
\draw(0,0) to [european resistor,l_={$\SI{6}{\watt}$},v^>={$\SI{2}{\volt}$}]++(0,\y) to [european resistor]++(\x,0) to [european resistor,v={$\SI{6}{\volt}$},l={ب}] ++(0,-\y) to [short] (0,0);
\draw(\x,\y) to [european resistor,*-,v^<={$\SI{3}{\volt}$},l_={$\SI{6}{\watt}$}]++(\x,0) to [european resistor]++(0,-\y) to [short,-*]++(-\x,0);
\end{tikzpicture}
\caption*{ب}
\end{subfigure}
\caption{سوال \حوالہ{سوال_بنیادی_طاقت_مہیا_حاصل_ر} کا دور۔}
\label{شکل_بنیادی_سوال_طاقت_مہیا_حاصل_ر}
\end{figure}

جوابات: پرزہ الف سے \عددی{\SI{9}{\watt}} حاصل کیا جاتا ہے۔پرزہ ب سے \عددی{\SI{30}{\watt}} حاصل کیا جاتا ہے۔
\انتہا{سوال}
%===============
\ابتدا{سوال}\شناخت{سوال_بنیادی_طاقت_مہیا_حاصل_ڑ}
شکل \حوالہ{شکل_بنیادی_سوال_طاقت_مہیا_حاصل_ڑ}-الف میں پرزہ الف کا طاقت دریافت کریں۔شکل-ب میں پرزہ ب کی رو، دباو اور طاقت دریافت کریں۔
%
\begin{figure}
\centering
\begin{subfigure}{0.5\textwidth}
\centering
\begin{tikzpicture}[american voltages]
\draw(0,0) to [european resistor,l_={$\SI{10}{\watt}$},v^>={$\SI{2}{\volt}$}]++(0,\y) to [european resistor,v^>={$\SI{4}{\volt}$}]++(\x,0) to [european resistor,l={الف}] ++(0,-\y) to [short] (0,0);
\draw(\x,\y) to [european resistor,*-]++(\x,0) to [european resistor,l={$\SI{24}{\watt}$},v={$\SI{4}{\volt}$}]++(0,-\y) to [short,-*]++(-\x,0);
\end{tikzpicture}
\caption*{الف}
\end{subfigure}%
\begin{subfigure}{0.5\textwidth}
\centering
\begin{tikzpicture}[american voltages]
\draw(0,0) to [european resistor,l_={$\SI{-20}{\watt}$},v^>={$\SI{4}{\volt}$}]++(0,\y) to [european resistor,v^<={$\SI{6}{\volt}$}]++(\x,0) to [european resistor,l={ب}] ++(0,-\y) to [short] (0,0);
\draw(\x,\y) to [european resistor,*-]++(\x,0) to [european resistor,l={$\SI{20}{\watt}$},v={$\SI{2}{\volt}$}]++(0,-\y) to [short,-*]++(-\x,0);
\end{tikzpicture}
\caption*{ب}
\end{subfigure}
\caption{سوال \حوالہ{سوال_بنیادی_طاقت_مہیا_حاصل_ڑ} کا دور۔}
\label{شکل_بنیادی_سوال_طاقت_مہیا_حاصل_ڑ}
\end{figure}

جوابات: پرزہ الف سے \عددی{\SI{66}{\watt}} حاصل کیا جاتا ہے۔ پرزہ ب کو \عددی{\SI{10}{\watt}} فراہم کیا جاتا ہے۔
\انتہا{سوال}
%===============



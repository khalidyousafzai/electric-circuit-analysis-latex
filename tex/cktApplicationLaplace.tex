\باب{ادوار کا حل بذریعہ لاپلاس بدل}
\حصہ{ادوار کا حل}
لاپلاس بدل کا استعمال دیکھنے کی خاطر شکل \حوالہ{شکل_لاپالس_حل_امالہ_مزاحمت} میں \عددی{RL} دور کو حل کرتے ہوئے \عددی{i(t)} دریافت کرتے ہیں۔دور کی کرخوف مساوات لکھتے ہیں۔
\begin{align*}
v_d(t)=i(t)R+L\frac{\dif i(t)}{\dif t}
\end{align*}
اس دور کے  فطری حل اور جبری حل  کا مجموعہ درکار عمومی حل ہو گا۔ عمومی حل میں ابتدائی معلومات سے حاصل کردہ مستقل پر کرنے سے مخصوص حل حاصل ہو گا۔لاپلاس بدل سے دور حل کرتے ہوئے مخصوص حل ایک ہی بار میں حاصل ہوتا ہے۔درج بالا مساوات کے دونوں اطراف کا لاپلاس بدل لیتے ہیں۔
\begin{align*}
\Laplace\left[10 u(t)\right]& =R \Laplace[i(t)]+L\Laplace\left[\frac{\dif i(t)}{\dif t}\right]
\end{align*}   
صفحہ \حوالہصفحہ{جدول_لاپلاس_بدل_جوڑیاں} پر جدول \حوالہ{جدول_لاپلاس_بدل_جوڑیاں} اور صفحہ \حوالہصفحہ{جدول_لاپلاس_مسئلے} پر جدول \حوالہ{جدول_لاپلاس_مسئلے} کی مدد لیتے ہیں۔
\begin{align*}
\frac{10}{s}& =R \bI(s)+L[s\bI(s) -i(0)]
\end{align*} 
چونکہ \عددی{i(0)=\SI{0}{\ampere}} ہے لہٰذا
\begin{align*}
\frac{10}{s}& =R \bI(s)+sL\bI(s)
\end{align*} 
یعنی
\begin{align*}
\bI(s)=\frac{10}{s(sL+R)}
\end{align*}
یا
\begin{align*}
\bI(s)=\frac{10}{R}\left(\frac{1}{s}-\frac{1}{s+\frac{R}{L}}\right)
\end{align*}
حاصل ہوتا ہے جہاں جزوی کسری پھیلاو لکھی گئی ہے۔درج بالا سے وقتی تفاعل  لکھتے ہیں۔
\begin{align*}
i(t)=\frac{10}{R}\left(1-e^{-\frac{R}{L}t}\right)u(t)
\end{align*}
آپ نے دیکھا کہ مخصوص حل یک وقت حاصل ہوتا ہے۔دور کی ابتدائی معلومات لاپلاس بدل لیتے وقت استعمال کی جاتی ہے۔

جیسا آپ نے دیکھا، لاپلاس بدل سے تفرقی و تکملی مساوات الجبرائی مساوات میں تبدیل ہو جاتی ہے جس سے درکار تفاعل کا لاپلاس بدل نہایت آسانی سے حاصل ہوتا ہے۔حاصل تفاعل کا الٹ لاپلاس بدل وقتی تفاعل دیتا ہے۔الٹ لاپالاس بدل جدول کی مدد سے حاصل کیا جاتا ہے۔
\begin{figure}
\centering
\begin{circuitikz}
\draw(0,0) to [american voltage source,l={${ v_d(t)=10u(t)}$}]++(0,\y) to [resistor,i^>={$i(t)$},l={$R$}]++(\x,0) to [inductor,l={$L$}]++(0,-\y) to [short] (0,0);
\end{circuitikz}
\caption{سلسلہ وار \عددی{RL} دور۔}
\label{شکل_لاپالس_حل_امالہ_مزاحمت}
\end{figure}

%=================
\حصہ{پرزوں کے مساوی لاپلاسی ادوار}
برقی پرزوں کی خصوصیات سے ان کے مساوی لاپلاسی ادوار حاصل کئے جا سکتے ہیں۔تمام پرزوں کے دباو بالمقابل رو تعلق لکھتے ہوئے انفعالی رائج سمت استعمال کئے گئے ہیں۔مزاحمت کے دباو اور رو کا تعلق
\begin{align}
v(t)=R i(t)
\end{align}
ہے۔دونوں اطراف کا لاپلاس بدل لیتے ہوئے اس تعلق کو درج ذیل لکھا جا سکتا ہے۔
\begin{align}
\bV(s)=R \bI(s)
\end{align}
شکل \حوالہ{شکل_لاپلاس_دور_مزاحمت_اظہار} میں مزاحمت کے دباو بالمقابل کا تعلق وقتی دائرہ کار اور مخلوط تعددی دائرہ کار میں دکھائے گئے ہیں۔
\begin{figure}
\centering
\begin{subfigure}{0.5\textwidth}
\centering
\begin{tikzpicture}
\draw(0,0) to [short,o-]++(\x,0) to [resistor,l_={$R$}]++(0,\y) to [short,-o,i<_={$i(t)$}]++(-\x,0);
\draw(0,0) to [short,o-]++(-\x/4,0)coordinate(kBL);
\draw(0,\y) to [short,o-]++(-\x/4,0);
\draw(kBL)++(0,-0.25) rectangle ++(-0.5,\y+0.5);
\draw(0,\y/2)node{$\begin{aligned}&+ \\&  v(t) \\ &- \end{aligned}$};
\end{tikzpicture}
\caption*{(الف)}
\end{subfigure}%
\begin{subfigure}{0.5\textwidth}
\centering
\begin{tikzpicture}
\draw(0,0) to [short,o-]++(\x,0) to [resistor,l_={$R$}]++(0,\y) to [short,-o,i<_={$\bI(s)$}]++(-\x,0);
\draw(0,0) to [short,o-]++(-\x/4,0)coordinate(kBL);
\draw(0,\y) to [short,o-]++(-\x/4,0);
\draw(kBL)++(0,-0.25) rectangle ++(-0.5,\y+0.5);
\draw(0,\y/2)node{$\begin{aligned}&+ \\&  \bV(s) \\ &- \end{aligned}$};
\end{tikzpicture}
\caption*{(ب)}
\end{subfigure}%
\caption{وقتی اور مخلوط تعددی دائرہ کار میں مزاحمت کا اظہار۔}
\label{شکل_لاپلاس_دور_مزاحمت_اظہار}
\end{figure}

برق گیر کے تعلقات
\begin{align}
v(t)&=\frac{1}{C}\int_0^t i(t) \dif t +v(0)\\
i(t)&=C\frac{\dif v(t)}{\dif t}
\end{align}
ہیں۔دونوں اطراف کا لاپلاس بدل لیتے ہوئے مخلوط تعددی دائرہ کار میں تعلقات حاصل ہوتے ہیں جنہیں شکل \حوالہ{شکل_لاپلاس_دور_برق_گیر_اظہار} میں دکھایا گیا ہے۔ابتدائی معومات سے پیدا منبع رو کی سمت اور منبع دباو کے قطب پر غور کریں۔ابتدائی رو کی سمت الٹ کرنے یا ابتدائی دباو کے قطب الٹ کرنے سے پیدا منبع رو کی سمت اور منبع دباو کے قطب الٹ ہوں گے۔
\begin{align}
\bV(s)&=\frac{\bI(s)}{sC}+\frac{v(0)}{s}\\
\bI(s)&=sC\bV(s)-Cv(0)
\end{align}
%
\begin{figure}
\centering
\begin{subfigure}{1\textwidth}
\centering
\begin{tikzpicture}[american voltages]
\draw(0,0) to [short,o-]++(\x,0) to [capacitor,l={$C$},v_>={$v(0)$}]++(0,\y) to [short,-o,i<_={$i(t)$}]++(-\x,0);
\draw(0,0) to [short,o-]++(-\x/4,0)coordinate(kBL);
\draw(0,\y) to [short,o-]++(-\x/4,0);
\draw(kBL)++(0,-0.25) rectangle ++(-0.5,\y+0.5);
\draw(0,\y/2)node{$\begin{aligned}&+ \\&  v(t) \\ &- \end{aligned}$};
\end{tikzpicture}
\caption*{(الف)}
\end{subfigure}
\begin{subfigure}{0.4\textwidth}
\centering
\begin{tikzpicture}
\draw(0,0) to [short,o-]++(\x,0) to [american voltage source,l_={$\frac{v(0)}{s}$}]++(0,3/4*\y) to [capacitor,l_={$\frac{1}{sC}$}]++(0,3/4*\y) to [short,-o,i<_={$\bI(s)$}]++(-\x,0);
\draw(0,0) to [short,o-]++(-\x/4,0)coordinate(kBL);
\draw(0,\y+\y/2) to [short,o-]++(-\x/4,0);
\draw(kBL)++(0,-0.25) rectangle ++(-0.5,\y+\y/2+0.5);
\draw(0,\y/2+\y/4)node{$\begin{aligned}&+ \\ \\ &  \bV(s) \\  \\&- \end{aligned}$};
\end{tikzpicture}
\caption*{(ب)}
\end{subfigure}%
\begin{subfigure}{0.6\textwidth}
\centering
\begin{tikzpicture}
\draw(0,0) to [short,o-]++(\x,0) to [capacitor,l_={$\frac{1}{sC}$}]++(0,\y) to [short,-o,i<_={$\bI(s)$}]++(-\x,0);
\draw(0,0) to [short,o-]++(-\x/4,0)coordinate(kBL);
\draw(\x,0) to [short,*-]++(\x,0) to [american current source,l_={$Cv(0)$}]++(0,\y) to [short,-*]++(-\x,0);
\draw(0,\y) to [short,o-]++(-\x/4,0);
\draw(kBL)++(0,-0.25) rectangle ++(-0.5,\y+0.5);
\draw(0,\y/2)node{$\begin{aligned}&+ \\&  \bV(s) \\ &- \end{aligned}$};
\end{tikzpicture}
\caption*{(پ)}
\end{subfigure}%
\caption{وقتی اور مخلوط تعددی دائرہ کار میں برق گیر کا اظہار۔}
\label{شکل_لاپلاس_دور_برق_گیر_اظہار}
\end{figure}

امالہ گیر کے تعلقات
\begin{align}
v(t)&=L\frac{\dif i(t)}{\dif t}\\
i(t)&=\frac{1}{L}\int_0^t v(t) \dif t+i(0)
\end{align}
ہیں جن سے
\begin{align}
\bV(s)&=sL\bI(s)-Li(0)\\
\bI(s)&=\frac{\bV(s)}{sL}+\frac{i(0)}{s}
\end{align}
حاصل ہوتے ہیں۔انہیں شکل \حوالہ{شکل_لاپلاس_دور_امالہ_گیر_اظہار} میں دکھایا گیا ہے۔یہاں بھی ابتدائی معلومات سے پیدا منبع کا دارومدار ابتدائی رو کی سمت اور ابتدائی دباو کے قطب  پر ہے۔ 
\begin{figure}
\centering
\begin{subfigure}{1\textwidth}
\centering
\begin{tikzpicture}
\draw(0,0) to [short,o-]++(\x,0) to [inductor,l_={$L$}]++(0,\y) to [short,-o,i<_={$i(t)$}]++(-\x,0);
\draw(0,0) to [short,o-]++(-\x/4,0)coordinate(kBL);
\draw(0,\y) to [short,o-]++(-\x/4,0);
\draw(kBL)++(0,-0.25) rectangle ++(-0.5,\y+0.5);
\draw(0,\y/2)node{$\begin{aligned}&+ \\&  v(t) \\ &- \end{aligned}$};
\draw[latex-](\x,0)++(-0.4,\y/4)--++(0,\y/2)node[pos=0.5,left]{$i(0)$};
\end{tikzpicture}
\caption*{(الف)}
\end{subfigure}
\begin{subfigure}{0.4\textwidth}
\centering
\begin{tikzpicture}
\draw(0,\y+\y/2) to [short,o-,i>^={$\bI(s)$}]++(\x,0) to [inductor,l={$sL$}]++(0,-3/4*\y) to [american voltage source,l={$L i(0)$}]++(0,-3/4*\y)  to [short,-o,]++(-\x,0);
\draw(0,0) to [short,o-]++(-\x/4,0)coordinate(kBL);
\draw(0,\y+\y/2) to [short,o-]++(-\x/4,0);
\draw(kBL)++(0,-0.25) rectangle ++(-0.5,\y+\y/2+0.5);
\draw(0,\y/2+\y/4)node{$\begin{aligned}&+ \\ \\ &  \bV(s) \\  \\&- \end{aligned}$};
\end{tikzpicture}
\caption*{(ب)}
\end{subfigure}%
\begin{subfigure}{0.6\textwidth}
\centering
\begin{tikzpicture}
\draw(0,0) to [short,o-]++(\x,0) to [inductor,l_={$sL$}]++(0,\y) to [short,-o,i<_={$\bI(s)$}]++(-\x,0);
\draw(0,0) to [short,o-]++(-\x/4,0)coordinate(kBL);
\draw(\x,\y) to [short,*-]++(\x,0) to [american current source,l={$\frac{i(0)}{s}$}]++(0,-\y) to [short,-*]++(-\x,0);
\draw(0,\y) to [short,o-]++(-\x/4,0);
\draw(kBL)++(0,-0.25) rectangle ++(-0.5,\y+0.5);
\draw(0,\y/2)node{$\begin{aligned}&+ \\&  \bV(s) \\ &- \end{aligned}$};
\end{tikzpicture}
\caption*{(پ)}
\end{subfigure}%
\caption{وقتی اور مخلوط تعددی دائرہ کار میں امالہ گیر کا اظہار۔}
\label{شکل_لاپلاس_دور_امالہ_گیر_اظہار}
\end{figure}

شکل \حوالہ{شکل_لاپلاس_استعمال_مشترک_امالہ_دباو_الف} میں دکھائے گئے مربوط لچھوں کے تعلق درج ذیل ہیں۔
\begin{align}
v_1(t)&=L_1 \frac{\dif i_1(t)}{\dif t}+M\frac{\dif i_2(t)}{\dif t}\\
v_2(t)&=L_2 \frac{\dif i_2(t)}{\dif t}+M\frac{i_1(t)}{\dif t}
\end{align} 
یہی مساوات \عددی{s} دائرہ کار میں درج ذیل لکھے جائیں گے۔
\begin{align}
\bV_1(s)&=sL_1 \bI_1(s)-L_1 i_1(0)+sM\bI_2(s)-M i_2(0)\\
\bV_2(s)&=sL_2 \bI_2(s)-L_2 i_2(0)+sM\bI_1(s)-M i_1(0)
\end{align}
%
\begin{figure}
\centering
\begin{subfigure}{1\textwidth}
\centering
\begin{circuitikz}
\draw(0,0) rectangle ++(-\boxW,\boxH);
\draw(0,0.25) to [short,-o]++(\x/4,0) to [short,o-]++(3/4*\x,0)coordinate(BL) to [inductor,l={$L_1$}]++(0,\y)coordinate(TL) to [short,-o,i<_={$i_1$}]++(-3/4*\x,0) to [short,o-]++(-\x/4,0);
\draw(\x+\x/3+\x,0.25) to [short,-o]++(-\x/4,0) to [short,o-]++(-3/4*\x,0)coordinate(BR) to [inductor,l_={$L_2$}]++(0,\y)coordinate(TR) to [short,-o,i<^={$i_2$}]++(3/4*\x,0) to [short,o-]++(\x/4,0);
\draw($(TL)!0.5!(TR)$)node[above]{$M$};
\draw(TL)++(-0.5,-0.5) node[circ]{}; 
\draw(TR)++(0.5,-0.5) node[circ]{}; 
\draw(2*\x+\x/3,0) rectangle ++(\boxW,\boxH);
\draw(\x/4,\boxH/2) node[]{$\begin{aligned} &+ \\ &v_1 \\ &-  \end{aligned}$};
\draw(2*\x+\x/3-\x/4,\boxH/2) node[]{$\begin{aligned} &+ \\ &v_2 \\ &-  \end{aligned}$};
\end{circuitikz}
\caption*{(الف)}
\end{subfigure}
\begin{subfigure}{1\textwidth}
\centering
\begin{circuitikz}
\draw(0,0) rectangle ++(-\boxW,\boxH);
\draw(0,0.25) to [short,-o]++(\x/2,0) to [short,o-]++(\x/2+\x,0)coordinate(BL) to [inductor,l={$sL_1$}]++(0,\y)coordinate(TL);
\draw(0,0.25+\y) to [short,-o]++(\x/2,0) to [short,i>^={$\bI_1(s)$}]++(\x/2,0) to [american voltage source]++(\x,0)coordinate(upL);
\draw(\x+\x/3+3*\x,0.25) to [short,-o]++(-\x/2,0) to [short,o-]++(-\x/2-\x,0)coordinate(BR) to [inductor,l_={$sL_2$}]++(0,\y)coordinate(TR);
 \draw(\x+\x/3+2*\x+\x,0.25+\y)to [short,-o]++(-\x/2,0) to [short,i>_={$\bI_2(s)$}]++(-\x/2,0) to [american voltage source]++(-\x,0)coordinate(upR);
\draw($(TL)!0.5!(TR)$)node[above]{$sM$};
\draw(TL)++(-0.5,-0.5) node[circ]{}; 
\draw(TR)++(0.5,-0.5) node[circ]{}; 
\draw(2*\x+\x/3+2*\x,0) rectangle ++(\boxW,\boxH);
\draw(\x/2+0.3,\boxH/2) node[]{$\begin{aligned} &+ \\ &\bV_1(s) \\ &-  \end{aligned}$};
\draw(2*\x+\x/3-\x/2+2*\x+0.3,\boxH/2) node[]{$\begin{aligned} &+ \\ &\bV_2(s) \\ &-  \end{aligned}$};
\draw(upL)++(-3/4*\x,0.9)node{$L_1i_1(0)+Mi_2(0)$};
\draw(upR)++(3/4*\x,0.9)node{$L_2i_2(0)+Mi_1(0)$};
\end{circuitikz}
\caption*{(ب)}
\end{subfigure}
\caption{مشترکہ امالہ کا لاپلاسی بدل۔}
\label{شکل_لاپلاس_استعمال_مشترک_امالہ_دباو_الف}
\end{figure}

تابع اور غیر تابع منبع دباو اور منبع رو کو بھی \عددی{s} دائرہ کار میں ظاہر کیا جا سکتا ہے
\begin{align}
\bV_1(s)&=\Laplace[v_1(t)]\\
\bI_2(s)&=\Laplace[i_2(t)]
\end{align}
اور اگر \عددی{v_1(t)=A_r i_2(t)} ہو جہاں \عددی{A_r} افزائش مزاحمت نما ہے تب 
 \begin{align}
\bV_1(s)=A_r \bI_2(s)
\end{align}
لکھا جا سکتا ہے۔

\حصہ{تجزیاتی تراکیب}
درج بالا حصے میں ہم نے برقی پرزوں کے \عددی{s} دائرہ کار میں مساوی ادوار حاصل کئے۔انہیں استعمال کرتے ہوئے ادوار حل کئے جا سکتے ہیں۔ایسا کرنے کی خاطر درج ذیل کرنا ہو گا۔

\begin{itemize}
\item
ابتدائی حالت جاننے کے لئے \عددی{t<\SI{0}{\second}} کے لئے دور حل کریں۔اگر \عددی{t<\SI{0}{\second}} میں دور برقرار حالت میں ہو تب برق گیر کو کھلے سر اور امالہ گیر کو قصر دور تصور کرتے ہوئے ابتدائی رو اور ابتدائی دباو حاصل کئے جا سکتے ہیں۔
\item
ابتدائی معلومات شامل کرتے ہوئے تمام پرزوں کی جگہ ان کے مساوی مخلوط تعددی دائرہ کار کے ادوار نسب کریں۔
\item
کسی بھی ترکیب کو استعمال کرتے ہوئے دور کو حل کریں۔جوابات \عددی{s} دائرہ کار میں ہوں گے۔
\item
الٹ لاپلاس بدل لیتے ہوئے وقتی دائرہ کار میں جوابات حاصل کریں۔
\end{itemize}
%==================
\ابتدا{مثال}\شناخت{مثال_لاپلاس_استعمال_مزاحمت_برق_گیر_الف}
لاپلاس بدل کی مدد سے شکل \حوالہ{شکل_لاپلاس_استعمال_مزاحمت_برق_گیر_الف}-الف میں \عددی{v_C(t)} حاصل کریں۔
\begin{figure}
\centering
\begin{subfigure}{0.5\textwidth}
\centering
\begin{tikzpicture}[american voltages]
\draw(0,0) to [american voltage source,l={$20u(t)$}]++(0,\y) to [resistor,l={$\SI{4}{\ohm}$}]++(\x,0) to [capacitor,l_={$\SI{2}{\farad}$},v^<={$v_C(t)$}]++(0,-\y) to [short](0,0);
\end{tikzpicture}
\caption*{(الف)}
\end{subfigure}%
\begin{subfigure}{0.5\textwidth}
\centering
\begin{tikzpicture}[american voltages]
\draw(0,0) to [american voltage source,l={$\frac{20}{s}$}]++(0,\y) to [resistor,l={$\SI{4}{\ohm}$}]++(\x,0) to [capacitor,l_={$\frac{1}{2s}$},v^<={$\bV_C(s)$}]++(0,-\y) to [short](0,0);
\end{tikzpicture}
\caption*{(ب)}
\end{subfigure}%
\caption{مثال \حوالہ{مثال_لاپلاس_استعمال_مزاحمت_برق_گیر_الف} کا دور۔}
\label{شکل_لاپلاس_استعمال_مزاحمت_برق_گیر_الف}
\end{figure}

حل:ابتدائی دباو \عددی{v_C(0)=\SI{0}{\volt}} ہے۔تمام پرزوں کی جگہ \عددی{s} دائرہ کار کے مساوی دور پر کرتے ہوئے شکل-ب حاصل ہوتا ہے۔شکل-ب میں تقسیم دباو کے کلیے سے برق گیر کا دباو لکھتے ہیں۔
\begin{align*}
\bV_C(s)&=\left(\frac{\frac{1}{2s}}{4+\frac{1}{2s}}\right)\frac{20}{s}\\
&=20\left(\frac{1}{s}-\frac{1}{s+\frac{1}{8}}\right)
\end{align*}
الٹ لاپلاس بدل لیتے ہوئے \عددی{v_C(t)} حاصل کرتے ہیں۔
\begin{align*}
v_C(t)=20\left(1-e^{-\frac{t}{8}}\right)u(t)
\end{align*}
\انتہا{مثال}
%====================
\ابتدا{مثال}\شناخت{مثال_لاپلاس_استعمال_دائری_مساوات}
شکل \حوالہ{شکل_لاپلاس_استعمال_دائری_مساوات} کے دائری مساوات اور مساوات جوڑ لکھیں۔
\begin{figure}
\centering
\begin{subfigure}{1\textwidth}
\centering
\begin{tikzpicture}[american voltages]
\draw(0,0) to [american voltage source,l={$v_A(t)$}]++(0,2*\y) to [capacitor,l={$C_1$},v={$v_1(0)$}]++(\x,0) to  [resistor,l={$R_1$}]++(\x,0) to [inductor,l={$L_2$}]++(\x,0)coordinate(ka) to [resistor,l={$R_2$}]++(\x,0) to [american voltage source,l={$v_B(t)$}]++(0,-2*\y) to [short](0,0);
\draw(2*\x,0)node[ground]{}coordinate(kb) to [inductor,*-,l={$L_1$}]++(0,\y) to [capacitor,-*,l={$C_2$},v={$v_2(0)$}]++(0,\y)node[above]{$v_0(t)$};
%initial currents
\draw[-latex] (ka)++(-\x/4,-0.3)node[below]{$i_2(0)$}--++(-\x/4,0);
\draw[-latex] (kb)++(0.3,\y/4)node[right]{$i_1(0)$}--++(0,\y/4);
\end{tikzpicture}
\caption*{(الف)}
\end{subfigure}
\begin{subfigure}{1\textwidth}
\centering
\begin{tikzpicture}[]
\draw(0,0) to [american voltage source,l={$\bV_A(s)$}]++(0,3.5*\y) to [capacitor,l={$\frac{1}{sC_1}$}]++(\x,0) ++(3/4*\x,0) to [american voltage source,l_={$\frac{v_1(0)}{s}$}]++(-3/4*\x,0)++(3/4*\x,0)to  [resistor,l={$R_1$}]++(\x,0) to [inductor,l={$sL_2$}]++(\x,0)++(3/4*\x,0) to [american voltage source,l_={$L_2 i_2(0)$}]++(-3/4*\x,0)++(3/4*\x,0) to [resistor,l={$R_2$}]++(\x,0) to [american voltage source,l={$\bV_B(s)$}]++(0,-3.5*\y) to [short](0,0);
\draw(2*\x+3/4*\x,0)node[ground]{} to [inductor,*-,l={$s L_1$}]++(0,\y) to [american voltage source,l={$L_1 i_1(0)$}]++(0,3/4*\y)++(0,3/4*\y) to [american voltage source,l_={$\frac{v_2(0)}{s}$}]++(0,-3/4*\y)++(0,3/4*\y)to [capacitor,-*,l={$\frac{1}{sC_2}$}]++(0,\y)node[above]{$\bV_0(s)$};
%currents
\draw[stealth-] ([shift={(-150:\x/2)}]1.375*\x,1.75*\y) arc (-150:150:\x/2);
\draw[] (1.375*\x,1.75*\y) node{$\bI_1(s)$};
\draw[stealth-] ([shift={(-150:\x/2)}]2.75*\x+1.375*\x,1.75*\y) arc (-150:150:\x/2);
\draw[] (2.75*\x+1.375*\x,1.75*\y) node{$\bI_2(s)$};
\end{tikzpicture}
\caption*{(ب)}
\end{subfigure}
\caption{مثال \حوالہ{مثال_لاپلاس_استعمال_دائری_مساوات} کا دور۔}
\label{شکل_لاپلاس_استعمال_دائری_مساوات}
\end{figure}

حل:لاپلاس بدل شکل \حوالہ{شکل_لاپلاس_استعمال_دائری_مساوات}-ب میں دکھایا گیا ہے جہاں سے کرخوف دائری مساوات لکھتے ہیں۔
\begin{align*}
\bI_1(s)\left[\frac{1}{sC_1}+R_1+\frac{1}{sC_2}+sL_1\right]-\bI_2(s)\left[\frac{1}{sC_2}+sL_1\right]&=\bV_A(s)-\frac{v_1(0)}{s}+\frac{v_2(0)}{s}-L_1 i_1(0)\\
-\bI_1(s)\left[sL_1+\frac{1}{sC_2}\right]+\bI_2(s)\left[sL_1+\frac{1}{sC_2}+sL_2+R_2\right]&=\bV_B(s)+L_1i_1(0)-\frac{v_2(0)}{s}-L_2i_2(0)
\end{align*} 
مساوات جوڑ لکھتے ہیں۔
\begin{align*}
\frac{\bV_0(s)-\bV_A(s)+\frac{v_1(0)}{s}}{R_1+\frac{1}{sC_1}}+\frac{\bV_0(s)+\frac{v_2(0)}{s}-L_1 i_1(0)}{\frac{1}{sC_2}+sL_1}+\frac{\bV_0(s)-L_2i_2(0)+\bV_B(s)}{sL_2+R_2}=0
\end{align*}
\انتہا{مثال}
%===================
\ابتدا{مثال}\شناخت{مثال_لاپلاس_استعمال_متعدد_طریقے_الف}
شکل \حوالہ{شکل_لاپلاس_استعمال_متعدد_طریقے_الف}-الف میں دور دیا گیا ہے۔اس کو ہم دائری ترکیب، ترکیب جوڑ، مسئلہ خطی میل، تبادلہ منبع اور مسئلہ تھونن کی مدد سے حل کرتے ہیں۔
\begin{figure}
\centering
\begin{subfigure}{0.5\textwidth}
\centering
\begin{tikzpicture}[american voltages]
\draw(0,0) to [american voltage source,l={$8u(t)\,\si{\volt}$}]++(0,2*\y) to [inductor,l={$\SI{2}{\henry}$}]++(\x,0) to [capacitor,l={$\SI{0.25}{\farad}$}]++(\x,0) to [resistor,l_={$\SI{6}{\ohm}$},v^<={$v_0(t)$}]++(0,-2*\y) to [short](0,0); 
\draw(\x,0)node[ground]{} to [american voltage source,*-,l={$2u(t)\,\si{\volt}$}]++(0,\y) to [resistor,-*,l={$\SI{2}{\ohm}$}]++(0,\y);
\end{tikzpicture}
\caption*{(الف)}
\end{subfigure}%
\begin{subfigure}{0.5\textwidth}
\centering
\begin{tikzpicture}[american voltages]
\draw(0,0) to [american voltage source,l={${\frac{8}{s}}$}]++(0,2*\y) to [inductor,l={$2s$}]++(\x,0) to [capacitor,l={$\frac{4}{s}$}]++(\x,0) to [resistor,l_={$6$},v^<={$\bV_0(s)$}]++(0,-2*\y) to [short](0,0); 
\draw(\x,0)node[ground]{} to [american voltage source,*-,l={$\frac{2}{s}$}]++(0,\y) to [resistor,-*,l={$2$}]++(0,\y)node[above]{$\bV_1(s)$};
%currents
\draw[stealth-]([shift={(-150:\x/4)}]\x/2,\y) arc (-150:150:\x/4);
\draw(\x/2,\y)node{$\bI_1(s)$};
\draw[stealth-]([shift={(-150:\x/4)}]\x+\x/3,\y) arc (-150:150:\x/4);
\draw(\x+\x/3,\y)node{$\bI_2(s)$};
\end{tikzpicture}
\caption*{(ب)}
\end{subfigure}
\caption{مثال \حوالہ{مثال_لاپلاس_استعمال_متعدد_طریقے_الف} کا دور۔}
\label{شکل_لاپلاس_استعمال_متعدد_طریقے_الف}
\end{figure}

حل: لاپلاس مساوی شکل-ب میں دکھایا گیا ہے۔ ہم جوڑ \عددی{\bV_1(s)} کو حاصل کرتے ہوئے \عددی{\bV_0(s)} کو تقسیم دباو کے کلیے سے حاصل کریں گے۔ مساوات جوڑ لکھتے ہیں
\begin{align*}
\frac{\bV_1(s)-\frac{8}{s}}{2s}+\frac{\bV_1(s)-\frac{2}{s}}{2}+\frac{\bV_1(s)}{6+\frac{4}{s}}=0
\end{align*}
جس سے
\begin{align*}
\bV_1(s)\left(\frac{1}{2s}+\frac{1}{2}+\frac{1}{6+\frac{4}{s}}\right)&=\frac{4}{s^2}+\frac{1}{s}
\end{align*}
یعنی
\begin{align*}
\bV_1(s)=\frac{2(s+4)(3s+2)}{s(4s^2+5s+2)}
\end{align*}
حاصل ہوتا ہے۔تقسیم دباو کے کلیے سے \عددی{\bV_0(s)} لکھتے ہیں۔
\begin{align*}
\bV_0(s)&=\left(\frac{6}{6+\frac{4}{s}}\right)\bV_1(s)\\
&=\left(\frac{6s}{6s+4}\right)\left[\frac{2(s+4)(3s+2)}{s(4s^2+5s+2)}\right]\\
&=\frac{6(s+4)}{4s^2+5s+2}
\end{align*}

اس دباو کا جزوی کسری پھیلاو لکھتے ہوئے وقتی تفاعل حاصل کرنا ہو گا۔ میں یہاں گزارش کروں گا ہوں کہ آپ صفحہ \حوالہصفحہ{مثال_تعددی_درست_تجزی} پر مثال \حوالہ{مثال_تعددی_درست_تجزی} کو ضرور دیکھیں۔
\begin{align*}
\bV_0(s)&=\frac{6(s+4)}{4(s^2+\frac{5}{4}s+\frac{1}{2})}\\
&=\frac{6(s+4)}{4(s+\frac{5}{8}+j\frac{\sqrt{7}}{8})(s+\frac{5}{8}-j\frac{\sqrt{7}}{8})}\\
&=\frac{K}{s+\frac{5}{8}+j\frac{\sqrt{7}}{8}}+\frac{K^*}{s+\frac{5}{8}-j\frac{\sqrt{7}}{8}}
\end{align*}
مستقل \عددی{K} اور \عددی{K^*} حاصل کرتے ہیں۔
\begin{align*}
K&=\left. \frac{6(s+4)}{4(s+\frac{5}{8}-j\frac{\sqrt{7}}{8})} \right|_{s=-\frac{5}{8}-j\frac{\sqrt{7}}{8}}\\
&=\frac{3}{4}+j\frac{81}{4\sqrt{7}}\\
K^*&=\left. \frac{6(s+4)}{4(s+\frac{5}{8}+j\frac{\sqrt{7}}{8})} \right|_{s=-\frac{5}{8}+j\frac{\sqrt{7}}{8}}\\
&=\frac{3}{4}-j\frac{81}{4\sqrt{7}}
\end{align*}
یوں درج ذیل لکھا جائے گا۔
\begin{align*}
\bV_0(s)&=\frac{\frac{3}{4}+j\frac{81}{4\sqrt{7}}}{s+\frac{5}{8}+j\frac{\sqrt{7}}{8}}+\frac{\frac{3}{4}-j\frac{81}{4\sqrt{7}}}{s+\frac{5}{8}-j\frac{\sqrt{7}}{8}}
\end{align*}
الٹ لاپلاس بدل لیتے ہیں۔
\begin{align*}
v_0(t)&=\left(\frac{3}{4}+j\frac{81}{4\sqrt{7}}\right)e^{-(\frac{5}{8}+j\frac{\sqrt{7}}{8})t}+\left(\frac{3}{4}-j\frac{81}{4\sqrt{7}}\right)e^{-(\frac{5}{8}-j\frac{\sqrt{7}}{8})t}\\
&=e^{-\frac{5}{8}t}\left[\frac{3}{4}\left(e^{-j\frac{\sqrt{7}}{8}t}+e^{j\frac{\sqrt{7}}{8}t}\right)+j\frac{81}{4\sqrt{7}}\left(e^{-j\frac{\sqrt{7}}{8}t}-e^{j\frac{\sqrt{7}}{8}t}\right)\right]\\
&=\frac{1}{4} e^{-\frac{5}{8}t}\left[6\cos\left(\frac{\sqrt{7} t}{8}\right)+\frac{162}{\sqrt{7}} \sin \left(\frac{\sqrt{7}t}{8}\right)\right] \,\si{\volt}
\end{align*}

آئیں یہی جواب دائری ترکیب سے حاصل کریں۔دائری مساوات لکھتے ہیں۔
\begin{align*}
\bI_1(s)\left(2s+2\right)-2\bI_2(s)&=\frac{8}{s}-\frac{2}{s}\\
-2\bI_1(s)+\bI_2(s)\left(2+\frac{4}{s}+6\right)&=\frac{2}{s}
\end{align*}
ان ہمزاد مساوات کا حل درج ذیل ہے
\begin{align*}
\bI_1(s)&=\frac{13s+6}{4s^3+5s^2+2s}\\
\bI_2(s)&=\frac{s+4}{4s^2+5s+2}
\end{align*}
جس سے خارجی دباو حاصل ہوتا ہے۔
\begin{align*}
\bV_0(s)=6\bI_2(s)=\frac{6(s+4)}{4s^2+5s+2}
\end{align*}
%
\begin{figure}
\centering
\begin{subfigure}{0.5\textwidth}
\centering
\begin{tikzpicture}[american voltages]
\draw(0,0) to [american voltage source,l={${\frac{8}{s}}$}]++(0,2*\y) to [inductor,l={$2s$}]++(\x,0) to [capacitor,l_={$\frac{4}{s}$}]++(\x,0) to [resistor,l_={$6$},v^<={$\bV'_0(s)$}]++(0,-2*\y) to [short](0,0); 
\draw(\x,0)node[ground]{}  to [resistor,*-*,l_={$2$}]++(0,2*\y)node[above]{$\bV'_1(s)$};
\draw [decorate,decoration={brace,amplitude=10pt,raise=4pt},yshift=0pt](\x-\x/8,2*\y+0.5) --++ (\x+\x/4,0) node [black,midway,yshift={0.8cm}] {\footnotesize $\bZ_1(s)$};
\end{tikzpicture}
\caption*{(الف)}
\end{subfigure}%
\begin{subfigure}{0.5\textwidth}
\centering
\begin{tikzpicture}[american voltages]
\draw(0,0) to [short]++(0,2*\y) to [inductor,l={$2s$}]++(\x,0) to [capacitor,l_={$\frac{4}{s}$}]++(\x,0) to [resistor,l_={$6$},v^<={$\bV''_0(s)$}]++(0,-2*\y) to [short](0,0); 
\draw(\x,0)node[ground]{} to [american voltage source,*-,l={$\frac{2}{s}$}]++(0,\y) to [resistor,-*,l={$2$}]++(0,\y)node[above]{$\bV''_1(s)$};
\end{tikzpicture}
\caption*{(ب)}
\end{subfigure}
\caption{مسئلہ خطی میل سے حل کرتے ہوئے باری باری ایک ایک منبع کو نافذ کیا گیا ہے}
\label{شکل_لاپلاس_استعمال_متعدد_طریقے_میل}
\end{figure}

مسئلہ خطی میل سے اب اسی دور کو حل کرتے ہیں۔شکل \حوالہ{شکل_لاپلاس_استعمال_متعدد_طریقے_میل} میں باری باری ایک ایک منبع کو لاگو کیا گیا ہے۔شکل \حوالہ{شکل_لاپلاس_استعمال_متعدد_طریقے_میل}-الف کو دیکھ کر \عددی{\bZ_1(s} لکھتے ہیں۔
\begin{align*}
\bZ_1(s)=\frac{2(6+\frac{4}{s})}{2+6+\frac{4}{s}}=\frac{3s+2}{2s+1}
\end{align*}
یوں تقسیم دباو کے کلیے سے \عددی{\bV'_1(s)}  لکھا جا سکتا ہے۔
\begin{align*}
\bV'_1(s)&=\left(\frac{\bZ_1(s)}{2s+\bZ_1(s)}\right)\frac{8}{s}\\
&=\left(\frac{\frac{3s+2}{2s+1}}{2s+\frac{3s+2}{2s+1}}\right)\frac{8}{s}\\
&=\frac{\frac{8}{s}(3s+2)}{4s^2+5s+2}
\end{align*}
تقسیم دباو کے کلیے کو دوبارہ استعمال کرتے ہوئے \عددی{\bV'_1(s)} سے \عددی{\bV''_0(s)} لکھتے ہیں۔
\begin{align*}
\bV'_0(s)&=\left(\frac{6}{6+\frac{4}{s}}\right)\bV'_1(s)\\
&=\left(\frac{3s}{3s+2}\right)\frac{\frac{8}{s}(3s+2)}{4s^2+5s+2}\\
&=\frac{24}{4s^2+5s+2}
\end{align*}
اب شکل \حوالہ{شکل_لاپلاس_استعمال_متعدد_طریقے_میل}-ب سے دوسرے منبع سے پیدا \عددی{\bV''_0(s)} حاصل کرتے ہیں۔یہاں \عددی{2s} اور \عددی{(6+\tfrac{4}{s})} متوازی جڑے ہیں جن کے مساوی کو \عددی{\bZ_2(s)} کہہ کر حاصل کرتے ہیں۔
\begin{align*}
\bZ_2(s)&=\frac{2s(6+\frac{4}{s})}{2s+6+\frac{4}{s}}\\
&=\frac{2s(3s+2)}{s^2+3s+2}
\end{align*}
یوں تقسیم دباو کے کلیے سے درج ذیل لکھا جا سکتا ہے
\begin{align*}
\bV''_1(s)&=\left(\frac{\bZ_2(s)}{2+\bZ_2(s)}\right)\frac{2}{s}\\
&=\left(\frac{\frac{2s(3s+2)}{s^2+3s+2}}{2+\frac{2s(3s+2)}{s^2+3s+2}}\right)\frac{2}{s}\\
&=\frac{2(3s+2)}{4s^2+5s+2}
\end{align*}
اور ایک مرتبہ دوبارہ تقسیم دباو سے 
\begin{align*}
\bV''_0(s)&=\left(\frac{6}{6+\frac{4}{s}}\right)\bV''_1(s)\\
&=\left(\frac{3s}{3s+2}\right)\frac{2(3s+2)}{4s^2+5s+2}\\
&=\frac{6s}{4s^2+5s+2}
\end{align*}
حاصل ہوتا ہے۔یوں دونوں منبع کی موجودگی میں \عددی{\bV_0(s)=\bV'_0(s)+\bV''_0(s)} ہو گا۔
\begin{align*}
\bV_0(s)&=\frac{24}{4s^2+5s+2}+\frac{6s}{4s^2+5s+2}\\
&=\frac{6(s+4)}{4s^2+5s+2}
\end{align*}

%  

آئیں اب شکل \حوالہ{شکل_لاپلاس_استعمال_متعدد_طریقے_الف}-الف کو تبادلہ منبع سے حل کریں۔دونوں منبع دباو کے مساوی منبع رو نسب کرتے ہوئے شکل \حوالہ{شکل_لاپلاس_استعمال_تبادلہ_منبع_الف}-الف ملتا ہے جہاں منبع دباو \عددی{\tfrac{8}{s}} اور اس کے سلسلہ وار \عددی{2s} کو منبع رو \عددی{\tfrac{8/s}{2s}=\tfrac{4}{s^2}} جس کے متوازی \عددی{2s} جڑا ہے میں تبدیل کیا گیا ہے۔اسی طرح منبع دباو \عددی{\tfrac{2}{s}} اور سلسلہ وار \عددی{2} کو منبع رو \عددی{\tfrac{2/s}{2}=\tfrac{1}{s}} میں تبدیل کیا گیا ہے جس کے متوازی \عددی{2} نسب ہے۔

شکل \حوالہ{شکل_لاپلاس_استعمال_تبادلہ_منبع_الف}-الف میں متوازی جڑے منبع رو کا مساوی منبع رو \عددی{\tfrac{4}{s^2}+\tfrac{1}{s}=\tfrac{s+4}{s^2}} ہے۔اسی طرح منبع کے متوازی \عددی{2} اور \عددی{2s} مل کر \عددی{\tfrac{2(2s)}{2+2s}=\tfrac{2s}{s+1}} دیتے ہیں۔یوں شکل-ب حاصل ہوتا ہے۔

شکل \حوالہ{شکل_لاپلاس_استعمال_تبادلہ_منبع_الف}-ب میں منبع رو \عددی{\tfrac{s+4}{s^2}} اور متوازی رکاوٹ \عددی{\tfrac{2s}{s+1}} کو سلسلہ وار جڑے منبع دباو \عددی{(\tfrac{s+4}{s^2})(\tfrac{2s}{s+1})=\tfrac{2(s+4)}{s(s+1)}} اور رکاوٹ \عددی{\tfrac{2s}{s+1}} میں تبدیل کرتے ہوئے شکل-پ حاصل ہوتی ہے جس سے تقسیم دباو کے کلیے سے \عددی{\bV_0(s)} لکھتے ہیں۔
\begin{align*}
\bV_0(s)&=\left(\frac{6}{\frac{2s}{s+1}+\frac{4}{s}+6}\right)\frac{2(s+4)}{s(s+1)}\\
&=\frac{6(s+4)}{4s^2+5s+2}
\end{align*}
%
\begin{figure}
\centering
\begin{subfigure}{1\textwidth}
\centering
\begin{tikzpicture}[american voltages]
\draw(0,0) to [american current source,l={$\frac{4}{s^2}$}]++(0,\y) to [short]++(3*\x,0) to [capacitor,l={$\frac{4}{s}$}]++(\x,0) to [resistor,l={$6$},v_<={$\bV_0(s)$}]++(0,-\y) to [short](0,0);
\draw(\x,0) to [inductor,*-*,l={$2s$}]++(0,\y);
\draw(2*\x,0) to [american current source,*-*,l={$\frac{1}{s}$}]++(0,\y);
\draw(3*\x,0) to [resistor,*-*,l={$2$}]++(0,\y);
\end{tikzpicture}
\caption*{(الف)}
\end{subfigure}
\begin{subfigure}{0.5\textwidth}
\centering
\begin{tikzpicture}[american voltages]
\draw(0,0) to [american current source,l={$\frac{s+4}{s^2}$}]++(0,\y) to [short]++(1*\x,0) to [capacitor,l={$\frac{4}{s}$}]++(\x,0) to [resistor,l={$6$},v_<={$\bV_0(s)$}]++(0,-\y) to [short](0,0);
\draw(\x,0) to [european resistor,*-*,l={$\frac{2s}{s+1}$}]++(0,\y);
\end{tikzpicture}
\caption*{(ب)}
\end{subfigure}%
\begin{subfigure}{0.5\textwidth}
\centering
\begin{tikzpicture}[american voltages]
\draw(0,0) to [american voltage source,l={$\frac{2(s+4)}{s(s+1)}$}]++(0,\y) to [european resistor,l={$\frac{2s}{s+1}$}]++(1*\x,0) to [capacitor,l={$\frac{4}{s}$}]++(\x,0) to [resistor,l={$6$},v_<={$\bV_0(s)$}]++(0,-\y) to [short](0,0);
\end{tikzpicture}
\caption*{(پ)}
\end{subfigure}
\caption{منبع دباو کی جگہ منبع رو نسب کیا گیا ہے۔}
\label{شکل_لاپلاس_استعمال_تبادلہ_منبع_الف}
\end{figure}

مسئلہ تھونن سے حل کرنے کی خاطر شکل \حوالہ{شکل_لاپلاس_استعمال_متعدد_طریقے_الف}-الف میں سلسلہ وار جڑے \عددی{\SI{6}{\ohm}} اور \عددی{\SI{0.25}{\farad}} کو بوجھ تصور کرتے ہوئے بقایا دور کا تھونن مساوی حاصل کرتے ہیں۔تھونن دباو شکل \حوالہ{شکل_لاپلاس_استعمال_متعدد_تھونن}-الف اور تھونن رکاوٹ شکل-ب سے حاصل کی جائے گی۔شکل-الف سے درج ذیل لکھتے
\begin{align*}
\bI(s)&=\frac{\frac{8}{s}-\frac{2}{s}}{2s+2}\\
&=\frac{3}{s(s+1)}
\end{align*}
ہوئے تھونن دباو حاصل کی جا سکتی ہے۔
\begin{align*}
\bV_{\text{تھونن}}&=\frac{2}{s}+2\bI(s)\\
&=\frac{2}{s}+\frac{6}{s(s+1)}\\
&=\frac{2(s+4)}{s+1}
\end{align*}
شکل-ب سے تھونن رکاوٹ حاصل کرتے ہیں۔
 \begin{align*}
\bZ_{\text{تھونن}}&=\frac{(2)(2s)}{2+2s}\\
&=\frac{2s}{s+1}
\end{align*}
تھونن دباو اور تھونن رکاوٹ استعمال کرتے ہوئے تھونن دور حاصل ہوتا ہے جس کے ساتھ بوجھ جوڑتے ہوئے  شکل \حوالہ{شکل_لاپلاس_استعمال_متعدد_تھونن}-پ حاصل ہوتی ہے جہاں سے تقسیم دباو کے کلیے سے \عددی{\bV_0(s)} حاصل ہو گا۔
\begin{align*}
\bV_0(s)&=\left(\frac{6}{\frac{2s}{s+1}+\frac{4}{s}+6}\right)\frac{2(s+4)}{s(s+1)}\\
&=\frac{6(s+4)}{4s^2+5s+2}
\end{align*}
%
\begin{figure}
\centering
\begin{subfigure}{0.5\textwidth}
\centering
\begin{tikzpicture}[american voltages]
\draw(0,0) to [american voltage source,l={${\frac{8}{s}}$}]++(0,2*\y) to [inductor,l={$2s$}]++(\x,0) to [short,-o]++(\x/2,0);
\draw(\x,0)node[circ]{} to [american voltage source,*-,l={$\frac{2}{s}$}]++(0,\y) to [resistor,-*,l={$2$}]++(0,\y);
\draw(0,0) to [short,-o]++(\x+\x/2,0);
\draw[stealth-]([shift={(-150:\x/4)}]\x/2,\y) arc (-150:150:\x/4);
\draw[](\x/2,\y) node{$\bI (s)$};
\draw(\x+\x/2,\y)node[]{$\begin{aligned} &+ \\ \\ \\ &\bV_{\text{تھونن}} \\ \\ \\ &- \end{aligned}$};
\end{tikzpicture}
\caption*{(الف)}
\end{subfigure}%
\begin{subfigure}{0.5\textwidth}
\centering
\begin{tikzpicture}[american voltages]
\draw(0,0) to [short]++(0,2*\y) to [inductor,l={$2s$}]++(\x,0) to [short,-o]++(\x/2,0);
\draw(\x,0)node[circ]{} to [short]++(0,\y) to [resistor,-*,l={$2$}]++(0,\y);
\draw(0,0) to [short,-o]++(\x+\x/2,0);
\draw[latex-] (\x+\x/4,\y)--++(\x/4,0)--++(0,-\y/8)node[below]{$\bZ_{\text{تھونن}}$};
\end{tikzpicture}
\caption*{(ب)}
\end{subfigure}
\begin{subfigure}{1\textwidth}
\centering
\begin{tikzpicture}[american voltages]
\draw(0,0) to [american voltage source,l={$\frac{2(s+4)}{s(s+1)}$}]++(0,2*\y) to [european resistor,l={$\frac{2s}{s+1}$}]++(\x,0) to [capacitor,l={$\frac{4}{s}$}]++(\x,0) to [resistor,l_={$6$},v^<={$\bV_0(s)$}]++(0,-2*\y) to [short](0,0); 
\end{tikzpicture}
\caption*{(پ)}
\end{subfigure}
\caption{مثال \حوالہ{مثال_لاپلاس_استعمال_متعدد_طریقے_الف} کے دور کا تھونن سے حل۔}
\label{شکل_لاپلاس_استعمال_متعدد_تھونن}
\end{figure}
\انتہا{مثال}
%==========
\ابتدا{مشق}
شکل \حوالہ{شکل_لاپلاس_استعمال_متعدد_طریقے_الف}-الف کو مسئلہ نارٹن سے حل کریں۔

جواب مثال \حوالہ{مثال_لاپلاس_استعمال_متعدد_طریقے_الف} میں دیا گیا ہے۔
\انتہا{مشق}
%=================
\ابتدا{مثال}\شناخت{مثال_لاپلاس_استعمال_مخلوط_جوڑ}
شکل \حوالہ{شکل_لاپلاس_استعمال_مخلوط_جوڑ}-الف میں \عددی{v_0(t)} دریافت کریں۔
\begin{figure}
\centering
\begin{subfigure}{1\textwidth}
\centering
\begin{tikzpicture}[american voltages]
\draw(0,0) to [american controlled current source,l={$4i(t)$}]++(0,\y) to [short]++(\x,0) to [american voltage source,l={$10u(t)$}]++(\x,0) to [resistor,l={$\SI{1}{\ohm}$}]++(\x,0) to [capacitor,l_={$\SI{1}{\farad}$},v^<={$v_0(t)$}]++(0,-\y) to [short] (0,0);
\draw(\x,0) to [capacitor,*-*,l={$\SI{0.5}{\farad}$}]++(0,\y)node[above]{$v_1(t)$};
\draw(2*\x,0) to [resistor,*-*,l={$\SI{6}{\ohm}$},i<_={$i(t)$}]++(0,\y)node[above]{$v_2(t)$};
\draw[dashed,gray] (3/4*\x,3/4*\y) rectangle ++(1.5*\x,3/4*\y)node[right]{\RL{مخلوط جوڑ}};
\end{tikzpicture}
\caption*{(الف)}
\end{subfigure}
\begin{subfigure}{1\textwidth}
\centering
\begin{tikzpicture}[american voltages]
\draw(0,0) to [american controlled current source,l={$4\bI(s)$}]++(0,\y) to [short]++(\x,0) to [american voltage source,l={$\frac{10}{s}$}]++(\x,0) to [resistor,l={$1$}]++(\x,0) to [capacitor,l_={$\frac{1}{s}$},v^<={$\bV_0(s)$}]++(0,-\y) to [short] (0,0);
\draw(\x,0) to [capacitor,*-*,l={$\frac{2}{s}$}]++(0,\y)node[above]{$\bV_1(s)$};
\draw(2*\x,0) to [resistor,*-*,l={$6$},i<_={$\bI(s)$}]++(0,\y)node[above]{$\bV_2(s)$};
\draw[dashed,gray] (3/4*\x,3/4*\y) rectangle ++(1.5*\x,3/4*\y+0.2)node[right]{\RL{مخلوط جوڑ}};
\end{tikzpicture}
\caption*{(ب)}
\end{subfigure}
\caption{مثال \حوالہ{مثال_لاپلاس_استعمال_مخلوط_جوڑ} کا دور۔}
\label{شکل_لاپلاس_استعمال_مخلوط_جوڑ}
\end{figure}

حل: اگر \عددی{v_2(t)} معلوم کیا جائے تو \عددی{v_0(t)} کو تقسیم دباو کے کلیے سے حاصل کیا جا سکتا ہے۔اس دور میں مخلوط جوڑ پایا جاتا ہے لہٰذا مساوات جوڑ کی تعداد کم ہو گی۔ شکل-ب میں لاپلاس بدل  دکھایا گیا ہے جس سے کرخوف مساوات جوڑ لکھتے ہیں
\begin{align*}
\frac{\bV_2(s)}{6}+\frac{\bV_2(s)}{1+\frac{1}{s}}+\frac{\bV_2(s)-\frac{10}{s}}{\frac{2}{s}}-4\bI(s)=0
\end{align*}
جہاں
\begin{align*}
\bI(s)=\frac{\bV_2(s)}{6}
\end{align*}
ہے لہٰذا
\begin{align*}
\frac{\bV_2(s)}{6}+\frac{\bV_2(s)}{1+\frac{1}{s}}+\frac{\bV_2(s)-\frac{10}{s}}{\frac{2}{s}}-\frac{4\bV_2(s)}{6}=0
\end{align*}
یعنی
\begin{align*}
\frac{\bV_2(s)}{6}+\frac{s\bV_2(s)}{s+1}+\frac{s\bV_2(s)-10}{2}-\frac{2\bV_2(s)}{3}=0
\end{align*}
یا
\begin{align*}
\bV_2(s)=\frac{10(s+1)}{s^2+2s-1}
\end{align*}
حاصل ہوتا ہے۔تقسیم دباو کے کلیے سے درکار جواب لکھتے ہیں۔
\begin{align*}
\bV_0(s)&=\bV_2(s)\left(\frac{\frac{1}{s}}{1+\frac{1}{s}}\right)\\
&=\frac{10(s+1)}{s^2+2s-1}\left(\frac{\frac{1}{s}}{1+\frac{1}{s}}\right)\\
&=\frac{10}{s^2+2s-1}
\end{align*}
جزوی کسری پھیلاو حاصل کرتے ہوئے  وقتی دائرہ کار میں دباو حاصل ہو گا۔  نسب نما کے جذر \عددی{-1\mp \sqrt{2}} ہیں لہٰذا درج ذیل لکھا جا سکتا ہے
\begin{align*}
\bV_0(s)&=\frac{10}{(s+1-\sqrt{2})(s+1+\sqrt{2})}\\
&=\frac{K_1}{s+1-\sqrt{2}}+\frac{K_2}{s+1+\sqrt{2}}
\end{align*}
جس سے 
\begin{align*}
K_1&=\left.\frac{10}{s+1+\sqrt{2}} \right|_{s=-1+\sqrt{2}}\\
&=\frac{5}{\sqrt{2}}\\
K_2&=\left.\frac{10}{s+1-\sqrt{2}} \right|_{s=-1-\sqrt{2}}\\
&=-\frac{5}{\sqrt{2}}
\end{align*}
حاصل ہوتے ہیں۔یوں
\begin{align*}
\bV_0(s)=\frac{5}{\sqrt{2}}\left(\frac{1}{s+1-\sqrt{2}}-\frac{1}{s+1+\sqrt{2}}\right)
\end{align*}
لکھ کر الٹ لاپلاس بدل لیتے ہوئے درکار دباو حاصل ہو گا۔
\begin{align*}
v_0(t)&=\frac{5}{\sqrt{2}}\left[e^{-(1-\sqrt{2})t}-e^{-(1+\sqrt{2})t}\right]u(t)\\
&=5\sqrt{2}e^{-t}\sinh (\sqrt{2}t) u(t) \, \si{\volt}
\end{align*}
\انتہا{مثال}
%===================
\ابتدا{مشق}\شناخت{مثال_لاپلاس_استعمال_مشق_الف}
شکل \حوالہ{شکل_لاپلاس_استعمال_مشق_الف} میں \عددی{i_0(t)} بذریعہ مساوات جوڑ دریافت کریں۔
\begin{figure}
\centering
\begin{tikzpicture}
\draw(0,0) to [american current source,l={$4u(t)\,\si{\ampere}$}]++(0,\y) to [short]++(\x,0) to [american voltage source,l={$6u(t)\,\si{\volt}$}]++(\x,0) to [short]++(\x,0) to [resistor,l={$\SI{2}{\ohm}$},i={$i_0(t)$}]++(0,-\y) to [short](0,0);
\draw(\x,0) to [inductor,*-*,l={$\SI{2}{\henry}$}]++(0,\y);
\draw(2*\x,0) to [capacitor,*-*,l={$\SI{0.25}{\farad}$}]++(0,\y);
\end{tikzpicture}
\caption{مثال \حوالہ{مثال_لاپلاس_استعمال_مشق_الف} کا دور۔}
\label{شکل_لاپلاس_استعمال_مشق_الف}
\end{figure}

جواب:\عددی{i_0(t)=[e^{-t}(5\sin t-3\cos t)+3]u(t) \, \si{\ampere}}
\انتہا{مشق}
%===================
\ابتدا{مشق}\شناخت{مثال_لاپلاس_استعمال_مشق_ب}
شکل \حوالہ{شکل_لاپلاس_استعمال_مشق_ب} میں \عددی{v_0(t)} بذریعہ مساوات جوڑ دریافت کریں۔
\begin{figure}
\centering
\begin{tikzpicture}[american voltages]
\draw(0,0) to [american current source,l={$2u(t)\,\si{\ampere}$}]++(0,2*\y) to [short]++(\x,0) to [resistor,l={$\SI{2}{\ohm}$}]++(\x,0) to [short]++(\x,0) to [resistor,l_={$\SI{6}{\ohm}$},v^<={$v_0(t)$}]++(0,-2*\y) to [short](0,0);
\draw(\x,0) to [capacitor,*-*,l={$\SI{0.2}{\farad}$}]++(0,2*\y);
\draw(2*\x,0) to [american voltage source,*-,l={$12u(t)\,\si{\volt}$}]++(0,\y) to [inductor,-*,l={$\SI{4}{\henry}$}]++(0,\y);
\end{tikzpicture}
\caption{مثال \حوالہ{مثال_لاپلاس_استعمال_مشق_ب} کا دور۔}
\label{شکل_لاپلاس_استعمال_مشق_ب}
\end{figure}

جواب:\عددی{v_0(t)=\left[e^{-\frac{t}{2}}\left(7.24\sin \frac{\sqrt{11}}{4}t-12\cos \frac{\sqrt{11}}{4}t\right)+12\right]u(t)\,\si{\volt}}
\انتہا{مشق}
%=============================
\ابتدا{مشق}\شناخت{مثال_لاپلاس_استعمال_مشق_پ}
شکل \حوالہ{شکل_لاپلاس_استعمال_مشق_پ} میں \عددی{v_0(t)} بذریعہ دائری مساوات دریافت کریں۔
\begin{figure}
\centering
\begin{tikzpicture}[american voltages]
\draw(0,0) to [resistor,l={$2$}]++(0,\y) to [inductor,l={$4s$}]++(\x,0) to [capacitor,l={$\frac{2}{s}$}]++(\x,0) to [american voltage source,l={$\frac{6}{s}$}]++(\x,0) to [resistor,l_={$4$},v^<={$v_0(t)$}]++(0,-\y) to [short](0,0);
\draw(\x,0) to [american current source,*-*,l_={$\frac{3}{s}$}]++(0,\y);
%currents
\draw[stealth-] ([shift={(-150:\x/4)}]\x/2,\y/2) arc (-150:150:\x/4);
\draw(\x/2,\y/2) node{$\bI_1(s)$};
\draw[stealth-] ([shift={(-150:\x/4)}]\x+\x,\y/2) arc (-150:150:\x/4);
\draw(\x+\x,\y/2) node{$\bI_2(s)$};
\end{tikzpicture}
\caption{مثال \حوالہ{مثال_لاپلاس_استعمال_مشق_پ} اور مثال \حوالہ{مثال_لاپلاس_استعمال_مشق_ت} کا دور۔}
\label{شکل_لاپلاس_استعمال_مشق_پ}
\end{figure}

جواب:\عددی{v_0(t)=12e^{-\frac{t}{2}}\,\si{\volt}}
\انتہا{مشق}
%=================
\ابتدا{مشق}\شناخت{مثال_لاپلاس_استعمال_مشق_ت}
مسئلہ تھونن کی مدد سے شکل \حوالہ{شکل_لاپلاس_استعمال_مشق_پ} میں  \عددی{v_0(t)} حاصل کریں۔
\انتہا{مشق}
%==========================

لاپلاس بدل کی مدد سے کچھ ادوار ہم حل کر چکے جن میں ابتدائی رو اور دباو صفر تھے۔ آئیں اب چند ایسے ادوار دیکھیں جن میں ابتدائی رو یا ابتدائی دباو پایا جاتا ہو۔اس طرز کے ادوار  ہم پہلے باب \حوالہ{باب_عارضی_رد_عمل} میں حل کر چکے ہیں۔اس باب کے شروع میں ابتدائی رو اور ابتدائی دباو کو شامل کرتے ہوئے پرزوں کے لاپلاس بدل حاصل کئے گئے نہیں شکل \حوالہ{شکل_لاپلاس_دور_مزاحمت_اظہار}، شکل \حوالہ{شکل_لاپلاس_دور_برق_گیر_اظہار} اور شکل \حوالہ{شکل_لاپلاس_دور_امالہ_گیر_اظہار} میں دکھایا گیا ہے۔انہیں کو استعمال کرتے ہوئے ادوار حل کئے جائیں گے۔
%======================

\ابتدا{مثال}\شناخت{مثال_لاپلاس_استعمال_عارضی_ردعمل_الف}
شکل \حوالہ{شکل_لاپلاس_استعمال_عارضی_ردعمل_الف} میں ازل سے ایک سوئچ منقطع اور ایک سوئچ چالو ہے۔عین \عددی{t=\SI{0}{\second}} پر چالو سوئچ کو منقطع کر دیا جاتا ہے جبکہ منقطع سوئچ کو چالو کر دیا جاتا ہے۔لمحہ \عددی{t<0} پر دور کو حل کرتے ہوئے ابتدائی دباو اور ابتدائی رو حاصل کرتے ہوئے \عددی{t\ge 0} پر \عددی{i_0(t)} دریافت کریں۔
\begin{figure}
\centering
\begin{subfigure}{1\textwidth}
\centering
\begin{tikzpicture}
\draw(0,0) to [american voltage source,l={$\SI{6}{\volt}$}]++(0,2*\y) to [cspst,l={${t=0}$}]++(\x,0) to [resistor,l={$\SI{2}{\ohm}$}]++(\x,0) to [short]++(\x,0) to [resistor,l={$\SI{4}{\ohm}$}]++(\x,0) to [inductor,l={$\SI{3}{\henry}$},i={$i(t)$}]++(0,-2*\y) to [short] (0,0);
\draw(2*\x,0) to [capacitor,*-*,l={$\SI{0.2}{\farad}$}]++(0,2*\y);
\draw(3*\x,0) to [american voltage source,*-,l={$\SI{2}{\volt}$}]++(0,\y) to [ospst,-*,l={${t=0}$}]++(0,\y);
\end{tikzpicture}
\caption*{(الف)}
\end{subfigure}
\begin{subfigure}{0.4\textwidth}
\centering
\begin{tikzpicture}
\draw(0,0) to [american voltage source,l_={$\SI{2}{\volt}$}]++(0,\y) to [resistor,l={$\SI{4}{\ohm}$}]++(\x,0) to [short,i={$i_L(0)$}]++(0,-\y) to [short]++(-\x,0) to [short,*-o]++(-\x,0);
\draw(0,\y) to [short,*-o]++(-\x,0);
\draw(-\x,\y/2)node{$\begin{aligned} &+ \\ &v_C(0) \\ &- \end{aligned}$};
\end{tikzpicture}
\caption*{(ب)}
\end{subfigure}%
\begin{subfigure}{0.6\textwidth}
\centering
\begin{tikzpicture}
\draw(0,0) to [american voltage source,l={$\frac{6}{s}$}]++(0,2*\y)  to [resistor,l={$2$}]++(\x,0) to [resistor,l={$4$}]++(\x,0) to [inductor,l={$3s$},i={$\bI(s)$}]++(0,-\y) to [american voltage source,l={$1.5$}]++(0,-\y)to [short] (0,0);
\draw(\x,0)node[ground]{} to [american voltage source,*-,l={$\frac{2}{s}$}]++(0,\y)to [capacitor,-*,l={$\frac{5}{s}$}]++(0,\y)node[above]{$\bV_1(s)$};
\end{tikzpicture}
\caption*{(پ)}
\end{subfigure}
\caption{مثال \حوالہ{مثال_لاپلاس_استعمال_عارضی_ردعمل_الف} کا دور۔}
\label{شکل_لاپلاس_استعمال_عارضی_ردعمل_الف}
\end{figure}

حل: لمحہ \عددی{t<0} پر برق گیر کو کھلے دور جبکہ امالہ گیر کو قصر دور تصور کرتے ہوئے شکل-ب حاصل ہوتا ہے جہاں سے امالہ گیر کی ابتدائی رو \عددی{i_L(0)} اور برق گیر کا ابتدائی دباو \عددی{v_C(0)} حاصل ہوتے ہیں۔
\begin{align*}
i_L(0)&=\frac{2}{4}=\SI{0.5}{\ampere}\\
v_C(0)&=\SI{2}{\volt}
\end{align*}
ابتدائی معلومات کو شامل کرتے ہوئے پرزوں کے لاپلاس مساوی دور پر کرنے سے  لمحہ \عددی{t\ge 0} کے لئے شکل حاصل ہوتا ہے۔مساوات جوڑ لکھتے ہیں
\begin{align*}
\frac{\bV_1(s)-\frac{6}{s}}{2}+\frac{\bV_1(s)-\frac{2}{s}}{\frac{5}{s}}+\frac{\bV_1(s)+1.5}{3s}=0
\end{align*}
جس سے 
\begin{align*}
\bV_1(s)&=\frac{12s^2+91s+120}{s(6s^2+23s+30)}
\end{align*}
حاصل ہوتا ہے۔یوں رو درج ذیل ہے
\begin{align*}
\bI(s)&=\frac{\bV_1(s)}{3s+4}\\
&=\frac{12s^2+91s+120}{s(s+4)(6s^2+23s+30)}
\end{align*}
الٹ لاپلاس بدل لیتے ہوئے درج ذیل ملتا ہے۔
\begin{align*}
i(t)=\left[e^{-\frac{23}{12}t}\left(\frac{44}{\sqrt{191}} \sin \frac{\sqrt{191} t}{12}-2\cos\frac{\sqrt{191} t}{12} \right)+4\right]u(t)\,\si{\ampere}
\end{align*}
\انتہا{مثال}
%==============
\ابتدا{مثال}\شناخت{مثال_لاپلاس_استعمال_عارضی_ردعمل_ب}
شکل \حوالہ{شکل_لاپلاس_استعمال_عارضی_ردعمل_ب} میں ازل سے چالو سوئچ کو لمحہ \عددی{} پر منقطع کیا جاتا ہے۔سوئچ منقطع ہونے کے بعد کی رو \عددی{i(t)} دریافت کریں۔

\begin{figure}
\centering
\begin{subfigure}{1\textwidth}
\centering
\begin{tikzpicture}
\draw(0,0) to [american voltage source,l={$\SI{5}{\volt}$}]++(0,\y) to [ospst,l={${t=0}$}]++(\x,0) to [resistor,l={$\SI{5}{\ohm}$}]++(\x,0) to [resistor,l={$\SI{15}{\ohm}$}]++(\x,0) to [inductor,l={$\SI{3}{\henry}$},i={$i(t)$}]++(\x,0);
\draw(0,0) to [short]++(4*\x,0) to [american voltage source,l_={$\SI{15}{\volt}$}]++(0,\y);
\draw(2*\x,0) to [capacitor,*-*,l={$\SI{0.5}{\farad}$}]++(0,\y);
\end{tikzpicture}
\caption*{(الف)}
\end{subfigure}
\begin{subfigure}{0.4\textwidth}
\centering
\begin{tikzpicture}
\draw(0,0) to [american voltage source,l={$\SI{5}{\volt}$}]++(0,\y) to [resistor,l={$\SI{5}{\ohm}$}]++(\x,0) to [resistor,l={$\SI{15}{\ohm}$},i={$i_L(0)$}]++(\x,0);
\draw(0,0) to [short]++(2*\x,0) to [american voltage source,l_={$\SI{15}{\volt}$}]++(0,\y);
\draw(\x,0) to [short,*-o]++(0,\y/8);
\draw(\x,\y) to [short,*-o]++(0,-\y/8);
\draw(\x+0.3,\y/2)node{$\begin{aligned} &+ \\ &v_C(0) \\ &- \end{aligned}$};
\end{tikzpicture}
\caption*{(ب)}
\end{subfigure}%
\begin{subfigure}{0.6\textwidth}
\centering
\begin{tikzpicture}
\draw(0,0)  to [american voltage source,l={$\frac{7.5}{s}$}] ++(0,\y)to [capacitor,l={$\frac{2}{s}$}]++(0,\y)to [resistor,l={$\frac{15}{s}$}]++(\x,0) to [inductor,l={$3s$},i={$\bI(s)$}]++(\x,0) ;
\draw(0,0) to [short]++(2*\x,0) to [american voltage source,l_={$\frac{15}{s}$}]++(0,\y)to [american voltage source,l_={$1.5$}]++(0,\y);
\end{tikzpicture}
\caption*{(پ)}
\end{subfigure}
\caption{مثال \حوالہ{مثال_لاپلاس_استعمال_عارضی_ردعمل_ب} کا دور۔}
\label{شکل_لاپلاس_استعمال_عارضی_ردعمل_ب}
\end{figure}

حل:چالو سوئچ کی صورت میں برق گیر کو کھلا دور اور امالہ گیر کو قصر دور تصور کرتے ہوئے شکل-ب حاصل ہوتی ہے جہاں سے  امالہ گیر کی ابتدائی رو \عددی{i_L(0)} اور برق گیر کی ابتدائی دباو \عددی{v_C(0)} حاصل کرتے ہیں۔
\begin{align*}
i_L(0)&=\frac{10-20}{5+15}=\SI{-0.5}{\ampere}\\
v_C(0)&=\frac{5\times 15+15\times 5 }{5+15}=\SI{7.5}{\volt}
\end{align*}
ابتدائی معلومات کو استعمال کرتے ہوئے، سوئچ منقطع ہونے کے بعد کا لاپلاس بدل دور شکل-پ میں دکھایا گیا ہے۔ابتدائی رو منفی ہونے کی وجہ سے امالہ کے لاپلاسی اظہار میں \عددی{\SI{1.5}{\volt}} منبع کے قطبین شکل \حوالہ{شکل_لاپلاس_دور_امالہ_گیر_اظہار} کے الٹ ہیں۔ شکل \حوالہ{شکل_لاپلاس_استعمال_عارضی_ردعمل_ب}-ب سے \عددی{\bI(s)} لکھتے ہیں۔
\begin{align*}
\bI(s)&=\frac{\frac{7.5}{s}-1.5-\frac{15}{s}}{\frac{2}{s}+15+3s}\\
&=\frac{-(s+5)}{2(s^2+5s+\frac{2}{3})}\\
&=\frac{-(s+5)}{2(s+\frac{5}{2}-\frac{\sqrt{201}}{6})(s+\frac{5}{2}+\frac{\sqrt{201}}{6})}
\end{align*}
اس کا الٹ لاپلاس بدل لیتے ہوئے درج ذیل حاصل ہوتا ہے۔
 \begin{align*}
i(t)=-e^{-\frac{5}{2}t}\left[\frac{45}{6\sqrt{201}} \sinh \left(\frac{\sqrt{201}}{6}t\right)+\frac{1}{2}\cosh \left(\frac{\sqrt{201}}{6}t\right)\right]u(t)\,\si{\ampere}
\end{align*}
\انتہا{مثال}
%================
\ابتدا{مشق}\شناخت{مشق_لاپلاس_استعمال_سوئچ_الف}
شکل \حوالہ{شکل_لاپلاس_استعمال_سوئچ_الف} میں \عددی{i_0(t)} حاصل کریں۔
\begin{figure}
\centering
\begin{tikzpicture}
\draw(0,0) to [american voltage source,l={$\SI{10}{\volt}$}]++(0,\y) to [ospst,l={${t=0}$}]++(\x,0) to [resistor,l={$\SI{8}{\ohm}$}]++(\x,0) to [resistor,l={$\SI{4}{\ohm}$}]++(\x,0) to [inductor,l={$\SI{4}{\henry}$},i={$i(t)$}]++(\x,0);
\draw(0,0) to [short]++(4*\x,0) to [american voltage source,l_={$\SI{12}{\volt}$}]++(0,\y);
\draw(2*\x,0) to [capacitor,*-*,l={$\SI{1}{\farad}$}]++(0,\y);
\end{tikzpicture}
\caption{مشق \حوالہ{مشق_لاپلاس_استعمال_سوئچ_الف} کا دور۔}
\label{شکل_لاپلاس_استعمال_سوئچ_الف}
\end{figure}

جواب:\عددی{i_0(t)=-\frac{e^{-\frac{t}{2}}}{6}(1+\frac{t}{2})u(t)\,\si{\ampere}}
\انتہا{مشق}
%====================
\ابتدا{مشق}\شناخت{مشق_لاپلاس_استعمال_سوئچ_ب}
شکل \حوالہ{شکل_لاپلاس_استعمال_سوئچ_ب}-الف میں \عددی{v_0(t)} حاصل کریں۔شکل-ب میں داخلی دباو کی مستطیل صورت دی گئی ہے۔
\begin{figure}
\centering
\begin{subfigure}{0.6\textwidth}
\centering
\begin{tikzpicture}[american voltages]
\draw(0,0) to [american voltage source,l={$v_d(t)$}]++(0,\y) to [resistor,l={$\SI{4}{\ohm}$}]++(\x,0) to [inductor,l={$\SI{4}{\henry}$}]++(\x,0) to [resistor,l={$\SI{2}{\ohm}$},v={$v_0(t)$}]++(0,-\y) to [short](0,0);
\draw(\x,0) to [resistor,*-*,l={$\SI{2}{\ohm}$}]++(0,\y);
\end{tikzpicture}
\caption*{(الف)}
\end{subfigure}%
\begin{subfigure}{0.4\textwidth}
\centering
\begin{tikzpicture}
\draw(0,0)--++(0,2)node[left]{$v_d(t)$};
\draw(0,0)--++(3,0)node[below]{$t\,(\si{\second})$};
\draw(0,0)--++(0,1)node[left]{$\SI{20}{\volt}$}--++(1.5,0)--++(0,-1)node[below]{$2$}--(3,0);
\end{tikzpicture}
\caption*{(ب)}
\end{subfigure}%
\caption{مشق \حوالہ{مشق_لاپلاس_استعمال_سوئچ_ب} کا دور۔}
\label{شکل_لاپلاس_استعمال_سوئچ_ب}
\end{figure}

جواب:\عددی{v_0(t)=4(1-e^{-\frac{5}{6}t})u(t)+4(1-e^{-(\frac{5}{6}-2)t})u(t-2) \, \si{\volt}}
\انتہا{مشق}
%=================

\حصہ{تبادلی تفاعل جال}
دور میں کسی بھی دباو یا رو اور داخلی اشارے کے تناسب کو جال کی \اصطلاح{تبادلی تفاعل}\فرہنگ{تبادلی تفاعل}\حاشیہب{transfer function}\فرہنگ{transfer function} یا \اصطلاح{تفاعل جال}\فرہنگ{تفاعل!جال}\فرہنگ{جال!تفاعل}\حاشیہب{network function}\فرہنگ{network!function}\فرہنگ{function!network}  کہتے ہیں۔اگر دونوں متغیرات دباو ہوں تب تبادلی تفاعل \اصطلاح{افزائش دباو}\فرہنگ{افزائش!دباو}\فرہنگ{دباو!افزائش}\حاشیہب{voltage gain}\فرہنگ{voltage!gain}\فرہنگ{gain!voltage} کہلاتا ہے، اگر دونوں متغیرات رو ہوں تب اس کو \اصطلاح{افزائش رو}\فرہنگ{افزائش!رو}\فرہنگ{رو!افزائش}\حاشیہب{current gain}\فرہنگ{current!gain}\فرہنگ{gain!current} کہتے ہیں۔اسی طرح دباو اور رو کے تناسب کو \اصطلاح{افزائش مزاحمت نما}\فرہنگ{افزائش!مزاحمت نما}\فرہنگ{مزاحمت نما!افزائش}\حاشیہب{transresistance gain}\فرہنگ{transcresistance!gain}\فرہنگ{gain!transresistance} کہتے ہیں جبکہ رو اور دباو کے تناسب کو \اصطلاح{افزائش موصلیت نما}\فرہنگ{افزائش!موصلیت نما}\فرہنگ{موصلیت نما!افزائش}\حاشیہب{transconductance gain}\فرہنگ{transconductance!gain}\فرہنگ{gain!transconductance} کہتے ہیں۔تبادلی تفاعل کے حصول میں ابتدائی دباو اور ابتدائی رو کو صفر لیا جاتا ہے۔

فرض کریں کہ کسی دور کا تبادلی تفاعل درج ذیل مساوات دیتی ہے جہاں \عددی{x_d(t)} داخلی اشارہ اور \عددی{y_0(t)} خارجی اشارہ ہیں۔
\begin{multline*}
b_n\frac{\dif^{\, n} y_0(t)}{\dif t^{n}}+b_{n-1}\frac{\dif^{\, n-1} y_0(t)}{\dif t^{n-1}}+\cdots+b_{1}\frac{\dif^{\, 1} y_0(t)}{\dif t^{1}}+b_0 y_0(t)=\\
a_m\frac{\dif^{\, m} x_d(t)}{\dif t^{m}}+a_{m-1}\frac{\dif^{\, m-1} x_d(t)}{\dif t^{m-1}}+\cdots+a_{1}\frac{\dif^{\, 1} x_d(t)}{\dif t^{1}}+a_0 x_d(t)
\end{multline*}
تمام ابتدائی معلومات صفر ہونے کی صورت میں درج بالا کا لاپلاس بدل درج ذیل ہو گا
\begin{multline*}
\left(b_n s^n+b_{n-1}s^{n-1}+\cdots +b_1 s+b_0\right){\kB{Y}_0(s)}=\\
\left(a_m s^m+a_{m-1}s^{m-1}+\cdots+a_1 s+a_0\right){\kB{X}_d(s)}
\end{multline*}
جس سے تبادلی تفاعل \عددی{\bH(s)} 
\begin{align*}
\bH(s)=\frac{\kB{Y}_0(s)}{\kB{X}_d(s)}=\frac{a_m s^m+a_{m-1}s^{m-1}+\cdots+a_1 s+a_0}{b_n s^n+b_{n-1}s^{n-1}+\cdots +b_1 s+b_0}
\end{align*}
یا
\begin{align}\label{مساوات_لاپلاس_استعمال_عمومی_الف}
{\kB{Y}_0(s)}=\bH(s){\kB{X}_d(s)}
\end{align}
لکھتے ہیں۔

مساوات \حوالہ{مساوات_لاپلاس_استعمال_عمومی_الف} کہتی ہے کہ تبادلی تفاعل \عددی{\bH(s)} اور داخلی تفاعل \عددی{\kB{X}_d} کا حاصل ضرب خارجی
 تفاعل \عددی{\kB{Y}_0(s)} کے برابر ہے۔یوں \عددی{x_d(t)=\delta(t)} کی صورت میں چونکہ \عددی{{\kB{X}_d(s)}=1} ہے لہٰذا \عددی{{\kB{Y}_0(s)}=\bH(s)} ہو گا۔
\begin{align}\label{مساوات_لاپلاس_استعمال_تبادلی_تفاعل_اور_خارجی_اشارہ}
{\kB{Y}_0(s)}=\bH(s)\quad  \delta(t)
\end{align}
یہ ایک اہم نتیجہ ہے جس کے تحت کسی بھی دور پر اکائی ضرب تفاعل لاگو کرتے ہوئے خارجی اشارے سے دور کا تبادلی تفاعل حاصل کیا جا سکتا ہے۔ایک بار دور کا تبادلی تفاعل معلوم ہو جائے اس کے بعد کسی بھی داخلی اشارے پر دور کا ردعمل مساوات \حوالہ{مساوات_لاپلاس_استعمال_عمومی_الف} سے حاصل کیا جا سکتا ہے۔اکائی ضرب تفاعل لاگو کرتے ہوئے خارجی ردعمل \عددی{h(t)} دے گا جس کا لاپلاس بدل لیتے ہوئے \عددی{\bH(s)} حاصل کیا جائے گا۔چونکہ تجزیہ گاہ\فرہنگ{تجزیہ گاہ}\حاشیہب{lab}\فرہنگ{lab} میں اکائی ضرب تفاعل پیدا کرنا مشکل بلکہ ناممکن کام ہے لہٰذا ہم دور پر اکائی سیڑھی تفاعل لاگو کرتے ہوئے تبادلی تفاعل حاصل کر سکتے ہیں۔چونکہ \عددی{u(t)} کا لاپلاس بدل \عددی{\tfrac{1}{s}} ہے لہٰذا دور پر اکائی سیڑھی تفاعل لاگو کرتے ہوئے مساوات \حوالہ{مساوات_لاپلاس_استعمال_عمومی_الف} کے تحت درج ذیل لکھا جا سکتا ہے۔
\begin{align}
{\kB{Y}_0(s)}=\frac{\bH(s)}{s} \quad u(t)
\end{align}
یوں اکائی سیڑھی تفاعل لاگو کرتے ہوئے دور کا خارجی اشارہ \عددی{y_0(t)} ناپا جاتا ہے۔خارجی اشارے کا لاپلاس بدل \عددی{\kB{Y}_0(s)} دے گا۔درج بالا مساوات کے تحت \عددی{\bH(s)={\kB{Y}_0(s)}} کے برابر ہے۔اس کو یوں بھی بیان کیا جا سکتا ہے کہ ناپے گئے خارجی اشارے کے تفرق \عددی{\tfrac{\dif y_0(t)}{\dif t}} کا لاپلاس بدل نظام کا تبادلی تفاعل \عددی{\bH(s)} ہو گا۔

%==============

\ابتدا{مثال}
دور کا اکائی ضرب تفاعل رد عمل \عددی{\bH(s)=\tfrac{2}{s+5}} ہے۔داخلی اشارہ \عددی{v_d(t)=3e^{-4t}u(t)\,\si{\volt}} لاگو کرتے ہوئے خارجی اشارہ \عددی{v_0(t)} دریافت کریں۔

حل:داخلی اشارے کا لاپلاس بدل لکھتے ہیں۔
\begin{align*}
\bV_d(s)=\frac{3}{s+4}
\end{align*}
یوں مساوات استعمال کرتے ہوئے
\begin{align*}
\bV_0(s)&=\bH(s) \bV_d(s)\\
&=\frac{6}{(s+5)(s+4)}\\
&=\frac{6}{s+4}-\frac{6}{s+5}
\end{align*}
الٹ لاپلاس بدل لیتے ہوئے خارجی اشارہ حاصل کرتے ہیں۔
\begin{align*}
v_0(t)=6\left(e^{-4t}-e^{-5t}\right)u(t)\,\si{\volt}
\end{align*}
\انتہا{مثال}
%=================

تبادلی تفاعل کے قطب سے دور کے ردعمل کے بارے میں بہت کچھ جانا جاتا ہے۔ہم ایک درجی اور دو درجہ ادوار پر باب \حوالہ{باب_عارضی_رد_عمل} میں غور کر چکے ہیں۔یہاں نتائج کو دوبارہ پیش کرتے ہیں۔ایک عدد امالہ گیر یا برق گیر کی صورت میں ردعمل \عددی{y(t)=y_0e^{-\frac{t}{\tau}}} صورت رکھتا ہے جہاں \عددی{\tau} دور کا وقتی مستقل ہے۔دو درجی ادوار کا ردعمل دور کے \اصطلاح{امتیازی مساوات}
\begin{align*}
s^2+2\zeta \omega_0 s+\omega_0^2=0
\end{align*}
کے قطبین پر منحصر ہوتا ہے۔یاد رہے کہ تبادلی تفاعل کا نسب نما امتیازی مساوات کہلاتا ہے۔امتیازی مساوات میں \عددی{ \zeta} \اصطلاح{تقصیری مستقل} اور \عددی{\omega_0} \اصطلاح{بلا تقصیر قدرتی تعدد} ہے اور یہی دو قیمتیں ردعمل کی تین ممکنہ صورتیں تعین کرتی ہیں۔
%=====================
\begin{description}
\جزو{زیادہ تقصیر:} امتیازی مساوات میں \عددی{\zeta>1} اور مساوات  کے جذر
\begin{align*}
s_1&=-\zeta\omega_0-\omega_0\sqrt{\zeta^2-1}\\
s_2&=-\zeta\omega_0+\omega_0\sqrt{\zeta^2-1}
\end{align*}
ہیں لہٰذا جال کا ردعمل درج ذیل ہے۔
\begin{align*}
y(t)=K_1e^{-(\zeta\omega_0+\omega_0\sqrt{\zeta^2-1})t}+K_2e^{-(\zeta\omega_0-\omega_0\sqrt{\zeta^2-1})t}
\end{align*}
\جزو{کم تقصیر:} امتیازی مساوات میں \عددی{\zeta<1} اور مساوات  کے جذر
\begin{align*}
s_1&=-\zeta\omega_0-j\omega_0\sqrt{1-\zeta^2}\\
s_2&=-\zeta\omega_0+j\omega_0\sqrt{1-\zeta^2}
\end{align*}
ہیں لہٰذا جال کا ردعمل درج ذیل ہے۔
\begin{align*}
y(t)&=K_1e^{-(\zeta\omega_0+j\omega_0\sqrt{1-\zeta^2})t}+K_2e^{-(\zeta\omega_0-j\omega_0\sqrt{1-\zeta^2})t}\\
&=Ke^{-\zeta \omega_0 t} \cos (\omega_0\sqrt{1-\zeta^2} t+\phi)
\end{align*}
\جزو{فاصل تقصیر:} امتیازی مساوات میں \عددی{\zeta=1} اور مساوات  کے جذر
\begin{align*}
s_1=s_2=-\omega_0\\
\end{align*}
ہیں لہٰذا جال کا ردعمل درج ذیل ہے۔
\begin{align*}
y(t)&=K_1e^{-\omega_0 t}+K_2 te^{-\omega_0 t}
\end{align*}
\end{description}
%====================

جال کے قطبین اور صفروں کو عموماً \اصطلاح{مخلوط سطح}\فرہنگ{مخلوط!سطح}\فرہنگ{سطح!مخلوط}\حاشیہب{complex plane}\فرہنگ{complex!plane}\فرہنگ{plane!complex} یا \عددی{s} سطح\فرہنگ{plane!s}\فرہنگ{s!plane} پر دکھایا جاتا ہے۔مخلوط سطح کے افقی محور پر \عددی{\sigma} اور عمودی محور پر \عددی{j\omega} رکھتے ہوئے مخلوط تعدد \عددی{s=\sigma+j\omega} دکھایا جاتا ہے۔اس سطح پر صفر کو \عددی{0} جبکہ قطبین کو \عددی{\times} سے ظاہر کیا جاتا ہے۔
\begin{figure}
\centering
\begin{subfigure}{0.5\textwidth}
\centering
\begin{tikzpicture}
%axis
\draw(-2.25,0)--(2.25,0)node[below]{$\sigma$};
\draw(0,-2.25)--(0,2.25)node[left]{$j\omega$};
%
\draw(-0.5,0)node[cross out,draw=black]{};
\draw(-1.5,0)node[cross out,draw=black]{};
\end{tikzpicture}
\caption*{(الف) مخلوط سطح پر حقیقی سادہ قطبین کا اظہار۔}
\end{subfigure}%
\begin{subfigure}{0.5\textwidth}
\centering
\begin{tikzpicture}
\begin{axis}[kStyleCircuitsA,small,xlabel={$t$},ylabel={$y(t)$},xmin=0,ymin=0,ymax=2.5,xtick=\empty,ytick=\empty]
\addplot[mark=none,color=black,domain=0:4]{e^(-0.5*x)+e^(-1.5*x)};
\end{axis}
\end{tikzpicture}
\caption*{(ب) حقیقی منفی سادہ قطبین سے پیدا ردعمل۔}
\end{subfigure}
\begin{subfigure}{0.5\textwidth}
\centering
\begin{tikzpicture}
%axis
\draw(-2.25,0)--(2.25,0)node[below]{$\sigma$};
\draw(0,-2.25)--(0,2.25)node[left]{$j\omega$};
%
\draw(-1,0.5)node[cross out,draw=black]{};
\draw(-1,-0.5)node[cross out,draw=black]{};
\end{tikzpicture}
\caption*{(پ) مخلوط سطح پر جوڑی دار مخلوط قطبین کا اظہار۔}
\end{subfigure}%
\begin{subfigure}{0.5\textwidth}
\centering
\begin{tikzpicture}
\begin{axis}[kStyleCircuitsA,small,xlabel={$t$},ylabel={$y(t)$},xmin=0,xtick=\empty,ytick=\empty]
\addplot[mark=none,color=black,domain=0:10,samples=100]{2*e^(-0.1*x)*cos(50*x)};
\end{axis}
\end{tikzpicture}
\caption*{(ت) مخلوط جوڑی دار قطبین سے پیدا ردعمل۔}
\end{subfigure}
\begin{subfigure}{0.5\textwidth}
\centering
\begin{tikzpicture}
%axis
\draw(-2.25,0)--(2.25,0)node[below]{$\sigma$};
\draw(0,-2.25)--(0,2.25)node[left]{$j\omega$};
%
\draw(-1,0.15)node[cross out,draw=black]{};
\draw(-1,-0.15)node[cross out,draw=black]{};
\end{tikzpicture}
\caption*{(ٹ) کثیر ہم رقمی حقیقی منفی قطبین۔}
\end{subfigure}%
\begin{subfigure}{0.5\textwidth}
\centering
\begin{tikzpicture}
\begin{axis}[kStyleCircuitsA,small,xlabel={$t$},ylabel={$y(t)$},xmin=0,ymin=0,ymax=2.5,xtick=\empty,ytick=\empty]
\addplot[mark=none,color=black,domain=0:4]{e^(-1*x)+e^(-1*x)};
\end{axis}
\end{tikzpicture}
\caption*{(ث) کثیر ہم رقمی حقیقی منفی قطبین سے پیدا ردعمل۔}
\end{subfigure}
\caption{قطبین اور ردعمل}
\label{شکل_لاپلاس_استعمال_قطبین_اور_ردعمل}
\end{figure}

شکل \حوالہ{شکل_لاپلاس_استعمال_قطبین_اور_ردعمل} میں سادہ اور علیحدہ قطبین، مخلوط قطبین اور کثیر  ہم رقمی قطبین مخلوط سطح پر دکھائے گئے ہیں۔شکل-ٹ میں دو عدد ہم رقمی قطبین کو علیحدہ علیحدہ کر کے دکھایا گیا ہے۔حقیقت میں یہ دونوں حقیقی محور پر ایک ہی نقطے پر پائے جاتے ہیں۔ساتھ ہی ساتھ ان سے حاصل ردعمل بھی دکھایا گیا ہے۔سادہ اور علیحدہ قطبین کے تفاعل  کی شرح تبدیلی کم ہوتی ہے لہٰذا اس کو صفر تک پہنچنے میں زیادہ وقت لگتا ہے۔ مخلوط قطبین کے تفاعل کی شرح تبدیلی زیادہ ہوتی ہے البتہ یہ صفر پر پہنچ کر دوسری جانب نکل جاتا ہے۔یوں مخلوط قطبین کا تفاعل \اصطلاح{مقصور سائن نما}\فرہنگ{مقصور سائن نما}\فرہنگ{سائن نما!مقصور}\حاشیہب{damped sinusoidal}\فرہنگ{damped!sinusoidal} ہوتا ہے۔ کثیر ہم رقمی قطبین کا ردعمل ان دونوں کے درمیان ہے۔یہ تیز تر ممکنہ رفتار سے صفر تک پہنچتا ہے، البتہ اتنا تیز نہیں کہ صفر پر رکھ نہ سکے اور دوسری جانب نکل جائے۔
%======================

\begin{figure}
\centering
\includegraphics{figLaplaceApplicationComplexPlane}
\caption{مخلوط سطح پر مختلف قطبین اور ان کے تفاعل کے ردعمل۔}
\label{شکل_لاپلاس_استعمال_مختلف_قطبین}
\end{figure}

شکل \حوالہ{شکل_لاپلاس_استعمال_مختلف_قطبین} میں مخلوط سطح پر مختلف تفاعل اور تفاعل کے  قطبین  دکھائے گئے۔اس شکل سے کئی حقائق کی وضاحت ہوتی ہے لہٰذا اس پر کچھ وقت صرف کرتے ہیں۔فرض کریں کہ \عددی{p_1} تا \عددی{p_5} بالترتیب \عددی{f_1(t)} تا \عددی{f_5(t)} تفاعل کو ظاہر کرتے ہیں۔مخلوط قطبین جوڑیوں میں پائے جاتے ہیں۔یوں \عددی{p_2} اور \عددی{p^*_2} مخلوط جوڑی ہے جو \عددی{f_2(t)} کو ظاہر کرتے ہیں۔حقیقی جزو صفر ہونے کی صورت میں خیالی قطبین کی جوڑی مثلاً \عددی{p_3} اور \عددی{p^*_3} ملتی ہے۔قطب کا حقیقی جزو اگر مثبت ہو تو تفاعل مسلسل بڑھتا ہے اور اگر حقیقی  جزو منفی ہو تب تفاعل مسلسل گھٹتا ہے۔یوں \عددی{f_4(t)} یا \عددی{f_5(t)} مسلسل بڑھتے تفاعل ہیں جبکہ \عددی{f_1(t)} اور \عددی{f_2(t)} مسلسل گھٹتے تفاعل ہیں۔مسلسل بڑھتا تفاعل غیر متوازن صورت حال کو ظاہر کرتی ہے جو حقیقی دنیا میں زیادہ دیر برقرار نہیں رہ سکتی جیسے مسلسل بڑھتی رو آخر کار کسی نہ کسی چیز کو تباہ کر کے ہی رہے گی۔مسلسل گھٹتا تفاعل متوازن صورت حال کو ظاہر کرتی ہے۔یوں خیالی محور کے دائیں جانب قطب غیر متوازن جبکہ محور کے بائیں جانب قطب متوازن نظام کو ظاہر کرتی ہے۔کسی بھی متوازن نظام کی تخلیق کے دوران مخلوط سطح میں قطبین کے مقام پر کھڑی نظر رکھی جاتی ہے اور خیالی محور کے دائیں جانب قطبین سے ہر صورت چھٹکارا حاصل کیا جاتا ہے۔قطب کا خیالی جزو صفر نہ ہونے کی صورت میں تفاعل سائن نما ہو گا لہٰذا \عددی{f_5(t)} مسلسل بڑھتا سائن نما تفاعل ہے جبکہ  \عددی{f_2(t)} مسلسل گھٹتا سائن نما یعنی \اصطلاح{مقصور سائن نما}\فرہنگ{مقصور سائن نما}\فرہنگ{سائن نما!مقصور}\حاشیہب{damped sinusoidal}\فرہنگ{damped!sinusoidal} ہے۔خیالی قطبین کی جوڑی سائن نما تفاعل کو ظاہر کرتی ہے لہٰذا \عددی{f_3(t)} برقرار حیطے کا سائن نما تفاعل ہے۔حقیقی محور سے جتنا دور جایا جائے، تعدد اتنی بڑھتی ہے لہٰذا \عددی{f_5(t)} سے \عددی{f_2(t)} کا تعدد زیادہ ہے اور \عددی{f_3(t)} کا تعدد اس سے بھی زیادہ ہے۔اسی طرح خیالی محور سے جتنا دور جایا جائے، بڑھنے یا گھٹنے کی شرح اتنی بڑھتی ہے لہٰذا \عددی{f_4(t)} کے بڑھنے کی شرح سے \عددی{f_2(t)} کے گھٹنے کی شرح زیادہ ہو گی جبکہ \عددی{f_5(t)} اس سے زیادہ اور \عددی{f_1(t)} تمام سے زیادہ تیزی سے تبدیل ہو گا۔
%================
\ابتدا{مثال}\شناخت{مثال_لاپلاس_استعمال_تغیر_پذیر_برق_گیر_الف}
شکل \حوالہ{شکل_لاپلاس_استعمال_تغیر_پذیر_برق_گیر_الف}-الف میں \اصطلاح{تغیر پذیر برق گیر}\فرہنگ{تغیر پذیر برق گیر}\فرہنگ{برق گیر!تغیر پذیر} استعمال کیا گیا ہے۔خارجی دباو \عددی{v_C(t)} کو \عددی{C=\SI{1}{\farad}}، \عددی{C=\SI{4}{\farad}} اور \عددی{C=\SI{10}{\farad}} کے لئے حاصل کریں۔
\begin{figure}
\centering
\begin{subfigure}{1\textwidth}
\centering
\begin{tikzpicture}[american voltages]
\draw(0,0) to [american voltage source,l={$u(t) \, \si{\volt}$}]++(0,\y) to [resistor,l={$\SI{1}{\ohm}$}]++(\x,0) to [inductor,l={$\SI{1}{\henry}$}]++(\x,0);
\draw(0,0) to [short]++(2*\x,0) to [vC={$C$},v_>={$v_C(t)$}]++(0,\y);
\end{tikzpicture}
\caption*{(الف)}
\end{subfigure}
\begin{subfigure}{1\textwidth}
\centering
\begin{tikzpicture}[american voltages]
\draw(0,0) to [american voltage source,l={$\frac{1}{s}$}]++(0,\y) to [resistor,l={$1$}]++(\x,0) to [inductor,l={$s$}]++(\x,0);
\draw(0,0) to [short]++(2*\x,0) to [vC={$\frac{1}{sC}$},v_>={$\bV_C(s)$}]++(0,\y);
\end{tikzpicture}
\caption*{(ب)}
\end{subfigure}%
\caption{مثال \حوالہ{مثال_لاپلاس_استعمال_تغیر_پذیر_برق_گیر_الف} کا دور۔}
\label{شکل_لاپلاس_استعمال_تغیر_پذیر_برق_گیر_الف}
\end{figure}

شکل \حوالہ{شکل_لاپلاس_استعمال_تغیر_پذیر_برق_گیر_الف}-ب میں لاپلاس بدل دور دکھایا گیا ہے جس سے تقسیم دباو کے کلیے سے خارجی دباو لکھتے ہیں۔
\begin{align*}
\bV_C(s)&=\left(\frac{\frac{1}{sC}}{1+s+\frac{1}{sC}}\right)\frac{1}{s}\\
&=\frac{\frac{1}{C}}{s(s^2+s+\frac{1}{C})}
\end{align*}
\عددی{C=\SI{1}{\farad}}  کے لئے \عددی{\bV_C(s)} کے مساوات کو حل کرتے ہیں۔
\begin{align*}
\bV_C(s)&=\frac{1}{s(s^2+s+1)}\\
&=\frac{1}{s}-\frac{\frac{1}{6}(3+j\sqrt{3})}{s+\frac{1}{2}+j\frac{\sqrt{3}}{2}}-\frac{\frac{1}{6}(3-j\sqrt{3})}{s+\frac{1}{2}-j\frac{\sqrt{3}}{2}}
\end{align*}
امتیازی مساوات کے جذر یعنی \عددی{\bV_C(s)} کے قطبین \عددی{p_1=0}، \عددی{p_2=-\tfrac{1}{2}-j\tfrac{\sqrt{3}}{2}} اور
 \عددی{p_3=-\tfrac{1}{2}+j\tfrac{\sqrt{3}}{2}} ہیں۔یوں ایک عدد حقیقی اور مخلوط جوڑی دار قطبین پائے جاتے ہیں جو \اصطلاح{کم مقصور صورت}\فرہنگ{مقصور!کم}\فرہنگ{underdamped} حال ہے۔الٹ لاپلاس بدل سے وقتی دائرہ کار میں خارجی دباو حاصل کرتے ہیں۔
\begin{align*}
v_C(t)=\left[1-e^{-\frac{t}{2}}\left(\cos \frac{\sqrt{3}t}{2}+\frac{1}{\sqrt{3}}\sin\frac{\sqrt{3}t}{2}\right)\right]u(t)\,\si{\volt}
\end{align*}
\عددی{C=\SI{4}{\farad}}  کے لئے \عددی{\bV_C(s)} کے مساوات کو حل کرتے ہیں۔
\begin{align*}
\bV_C(s)&=\frac{0.25}{s(s^2+s+0.25)}\\
&=\frac{0.25}{s(s+\frac{1}{2})^2}\\
&=\frac{1}{s}-\frac{1}{s+\frac{1}{2}}-\frac{1}{2(s+\frac{1}{2})^2}
\end{align*}
یہاں تینوں قطبین حقیقی ہیں جن میں \عددی{p=-\tfrac{1}{2}} کثیر رقمی قطب ہے جو \اصطلاح{فاصل مقصور حال}\فرہنگ{مقصور!فاصل}\فرہنگ{critical damped} کو ظاہر کرتی ہے۔الٹ لاپلاس لیتے ہوئے \عددی{v_C(t)} حاصل  کرتے ہیں۔
\begin{align*}
v_C(t)=\left(1-e^{-\frac{t}{2}}-\frac{t}{2}e^{-\frac{t}{2}}\right)u(t) \,\si{\volt}
\end{align*}
\عددی{C=\SI{10}{\farad}}  کے لئے \عددی{\bV_C(s)} کے مساوات کو حل کرتے ہیں۔
\begin{align*}
\bV_C(s)&=\frac{0.1}{s(s^2+s+0.1)}\\
&=\frac{1}{s}+\frac{0.145}{s+0.887}-\frac{1.145}{s+0.113}
\end{align*}
اس مساوات کے قطبین \عددی{p_1=0}، \عددی{p_2=-0.887} اور \عددی{p_3=-0.113} ہیں۔یوں تینوں سادہ علیحدہ علیحدہ حقیقی قطبین ہیں لہٰذا تفاعل کا ردعمل \اصطلاح{زیادہ مقصور}\فرہنگ{مقصور!زیادہ}\فرہنگ{damped!over} ہو گا۔الٹ لاپلاس بدل سے \عددی{v_C(t)} حاصل کرتے ہیں۔
\begin{align*}
v_C(t)=\left(1+0.145e^{-0.887t}-1.145e^{-0.113t}\right)u(t)\,\si{\volt}
\end{align*}
\انتہا{مثال}
%==================
\ابتدا{مثال}
اکائی ضرب ردعمل \عددی{y(t)=2e^{-5t}-4e^{-2t}} ہے۔اکائی سیڑھی ردعمل دریافت کریں۔

حل:اکائی ضرب ردعمل تبادلی تفاعل دیتا ہے لہٰذا دیے گیے ردعمل کا لاپلاس بدل \عددی{\bH(s)} ہو گا۔
\begin{align*}
\bH(s)=\frac{2}{s+5}-\frac{4}{s+2}
\end{align*}
یوں اکائی سیڑھی ردعمل درج ذیل ہو گا۔
\begin{align*}
\kB{Y}(s)=\left(\frac{2}{s+5}-\frac{4}{s+2}\right)\frac{1}{s}
\end{align*}
مخلوط تعددی دائرہ کار میں \عددی{s} سے تقسیم سے مراد وقتی دائرہ کار میں تفاعل کا تکمل ہے لہٰذا اکائی سیڑھی ردعمل وقتی دائرہ کار میں درج ذیل ہو گا۔
\begin{align*}
y(t)&=\int_0^t 2e^{-5t}-4e^{-2t} \dif t\\
&=\left. \frac{2e^{-5t}}{-5}-\frac{4e^{-2t}}{-2}\right|_0^t\\
&=\left(-\frac{8}{5}-\frac{2}{5}e^{-5t}+2e^{-2t}\right)u(t)
\end{align*}
\انتہا{مثال}
%===================
\ابتدا{مشق}
اکائی ضرب ردعمل \عددی{y(t)=2\cos 2t+3\sin 2t} ہے۔اکائی سیڑھی ردعمل دریافت کریں۔

جواب:\عددی{y(t)=\left(\frac{3}{2}-\frac{3}{2}\cos 2t+\sin 2t\right)u(t)}
\انتہا{مشق}
%==================
\ابتدا{مشق}\شناخت{مشق_لاپلاس_استعمال_سیڑھی_ردعمل}
تبادلی تفاعل \عددی{\bH(s)=\tfrac{s+4}{s^2+4s+13}} کے صفر اور قطب حاصل کرتے ہوئے مخلوط سطح پر دکھائیں۔ اس کا اکائی سیڑھی ردعمل بھی حاصل کریں۔

جواب:قطبین اور صفر کو شکل \حوالہ{شکل_لاپلاس_استعمال_سیڑھی_ردعمل} میں دکھایا گیا ہے۔\عددی{y(t)=e^{-2t}\left(\cos 3t+\tfrac{2}{3}\sin 3t\right)u(t)}
\begin{figure}
\centering
\begin{tikzpicture}
\draw(-3,0) to [short,-o](-2,0)node[below]{$-4$} to [short](1,0);
\draw(0,-2)--(0,2);
%
\draw (-1,1.5)node[cross out,draw=black]{};
\draw (-1,-1.5)node[cross out,draw=black]{};
\draw[dashed](0,-1.5)node[right]{$-j3$}--(-1,-1.5)--(-1,1.5)--(0,1.5)node[right]{$j3$};
\draw(-1,0)node[above]{$-2$};
\end{tikzpicture}
\caption{مشق \حوالہ{مشق_لاپلاس_استعمال_سیڑھی_ردعمل} کے قطبین اور صفر۔}
\label{شکل_لاپلاس_استعمال_سیڑھی_ردعمل}
\end{figure}
\انتہا{مشق}
%==================
\ابتدا{مشق}\شناخت{مشق_لاپلاس_استعمال_حسابی_ایمپلیفائر_ردعمل}
شکل \حوالہ{شکل_لاپلاس_استعمال_حسابی_ایمپلیفائر_ردعمل} کا تبادلی تفاعل \عددی{\kB{A}_v(s)=\tfrac{\bV_0(s)}{\bV_d(s)}} حاصل کریں۔
\begin{figure}
\centering
\begin{tikzpicture}
\draw(0,0)node[op amp](u){}; 
\draw(u.+)--++(-\x/4,0)--++(0,-\y/8)node[ground]{};
\draw(u.-)to  [short]++(-\x/4,0)to [capacitor,l_={$C_1$}]++(-\x,0) to [resistor,l_={$R_1$}]++(-\x,0)++(0,-\y)node[ground]{} to [american voltage source,l={$v_d(t)$}]++(0,\y);
\draw(u.-)++(-\x/4,0) to [short,*-]++(0,\y/2)coordinate(kL) to [resistor,l={$R_2$}]++(\x,0) to [capacitor,-*,l={$C_2$}]++(\x,0)coordinate(kR) |-(u.out);
\draw(kL) to [short,*-]++(0,3/4*\y) to [capacitor,l_={$C_3$}]++(2*\x,0)-|(kR);
\draw(u.out)node[above]{$v_0(t)$};
\end{tikzpicture}
\caption{مشق \حوالہ{مشق_لاپلاس_استعمال_حسابی_ایمپلیفائر_ردعمل} کا دور۔}
\label{شکل_لاپلاس_استعمال_حسابی_ایمپلیفائر_ردعمل}
\end{figure}

جواب:
\begin{align*}
\kB{A}_v(s)=-\frac{\frac{1}{R_1 C_3 \left(s+\frac{1}{R_2 C_2}\right)}}{\left(s+\frac{1}{R_1C_1}\right)\left[s+\frac{1}{R_2}\left(\frac{1}{C_2}+\frac{1}{C_3}\right)\right]}
\end{align*}
\انتہا{مشق}
%===================

آپ جانتے ہیں کہ دو درجی کم قصری جال کا امتیازی مساوات درج ذیل ہے
\begin{align*}
s^2+2\zeta \omega_0 s+\omega_0^2
\end{align*}
جس کے مخلوط جوڑی دار قطبین
\begin{align*}
s_1&=-\zeta \omega_0-j\omega_0 \sqrt{1-\zeta^2}\\
s_2&=-\zeta \omega_0+j\omega_0 \sqrt{1-\zeta^2}
\end{align*}
کو مخلوط سطح پر شکل \حوالہ{شکل_لاپلاس_استعمال_کم_قصری_جوڑی_دار} میں دکھایا گیا ہے۔قطب \عددی{p} کو زاویائی صورت میں لکھتے ہیں۔ محدد کے مرکز \عددی{(0,0)} سے قطب کا فاصلہ مسئلہ فیثاغورث کی مدد سے حاصل کرتے ہیں
\begin{align*}
\text{رداس}=\sqrt{(\zeta \omega_0)^2+\left(\omega_0\sqrt{1-\zeta^2}\right)^2}=\omega_0
\end{align*}
جسے شکل میں  \عددی{\omega_0} دکھایا گیا ہے۔اسی طرح زاویہ \عددی{\theta} شکل سے دیکھ کر لکھا جا سکتا ہے۔شکل میں تکون کا قاعدہ \عددی{\omega_0 \zeta} اور وتر \عددی{\omega_0} ہیں لہٰذا درج ذیل لکھا جا سکتا ہے۔
\begin{align*}
\cos \theta&=\frac{\omega_0 \zeta}{\omega_0}\\
&=\zeta
\end{align*}
یوں درج ذیل لکھے جا سکتے ہیں۔
\begin{gather}
\begin{aligned}
\text{رداس}&=\omega_0\\
\text{زاویہ}&=\theta=\cos^{-1} \zeta
\end{aligned}
\end{gather}
آپ دیکھ سکتے ہیں کہ محدد کے مرکز سے قطب تک فاصلہ \عددی{\omega_0} کے برابر ہے جبکہ زاویہ \عددی{\cos^{-1}\zeta} ہے۔یوں \عددی{\zeta} تبدیل کرنے سے رداس تبدیل نہیں ہوتا البتہ زاویہ تبدیل ہونے سے قطب دائری حرکت کرتا ہے۔شکل-ب میں \عددی{\zeta} تبدیل کرنے سے مخلوط جوڑی دار قطبین نقطہ دار دائرے پر حرکت کرتے ہیں۔
\begin{figure}
\centering
\begin{subfigure}{0.5\textwidth}
\centering
\begin{tikzpicture}
\draw(-3,0)--(1,0)node[below]{$\sigma$};
\draw(0,-2.5)--(0,2.5)node[left]{$j\omega$};
%
\draw(-1.5,1.5)node[cross out, draw=black]{}node[above]{$p$};
\draw(-1.5,-1.5)node[cross out, draw=black]{}node[below]{$p^*$};
\draw[](0,0)--(-1.5,1.5)node[pos=0.5,fill=white]{$\omega_0$};
\draw([shift={(135:0.3)}]0,0) arc (135:180:0.3);
\draw(158:0.5)node{$\theta$};
\draw[dashed](0,-1.5)--(-1.5,-1.5)--(-1.5,1.5)--(0,1.5);
\draw(0,-1.5)node[right]{$-j\omega_0\sqrt{1-\zeta^2}$};
\draw(0,1.5)node[right]{$+j\omega_0\sqrt{1-\zeta^2}$};
\draw(-1.5,0)node[below]{$-\zeta \omega_0$};
\end{tikzpicture}
\caption*{(الف) قطب تک رداس \عددی{\omega_0}  ہے جبکہ زاویے \عددی{\cos^{-1}\zeta} ہے۔ }
\end{subfigure}%
\begin{subfigure}{0.5\textwidth}
\centering
\begin{tikzpicture}
\draw(-3,0)--(1,0)node[below]{$\sigma$};
\draw(0,-2.5)--(0,2.5)node[left]{$j\omega$};
%
\draw(-1.5,1.5)node[cross out, draw=black]{}node[left]{$p$};
\draw(-1.5,-1.5)node[cross out, draw=black]{}node[left]{$p^*$};
\draw[](0,0)--(-1.5,1.5)node[pos=0.6,above right]{$\omega_0$};
\draw([shift={(135:0.3)}]0,0) arc (135:180:0.3);
\draw(158:0.5)node{$\theta$};
\draw[dashed] ([shift={(90:2.1213)}]0,0) arc (90:270:2.1213);
\end{tikzpicture}
\caption*{(ب) تقصیر تبدیل کرنے سے قطب نقطہ دار دائرے پر حرکت کرتا ہے۔}
\end{subfigure}
\caption{کم قصری، دو درجی جال کے مخلوط جوڑی دار قطبین۔}
\label{شکل_لاپلاس_استعمال_کم_قصری_جوڑی_دار}
\end{figure}
%===================

\حصہ{ترسیم قطبین و صفر اور بوڈا خط}
ہم تعددی ردعمل پر غور کے دوران بوڈا خطوط پر بحث کر چکے ہیں۔آئیں تبادلی تفاعل کے ترسیم قطبین و صفر اور بوڈا خط کے تعلق پر غور کریں۔ایسا کرنے کی خاطر ہم شکل \حوالہ{شکل_لاپلاس_استعمال_سلسلہ_وار_دور_قطبین_صفر} میں دیے \عددی{RLC} کا تبادلی تفاعل
\begin{align*}
\bH(s)&=\frac{\bV_R(s)}{\bV_d(s)}\\
&=\frac{\frac{R}{L} s}{s^2+\frac{R}{L}s+\frac{1}{LC}}
\end{align*}
 استعمال کریں گے جو پرزوں کی دی گئی قیمتیں پر کرنے سے درج ذیل صورت اختیار کر لیتا ہے۔
\begin{gather}
\begin{aligned}\label{مساوات_لاپلاس_استعمال_تین_بعدی_الف}
\bH(s)&=\frac{4 s}{s^2+4s+8}\\
&=\frac{4s}{(s+2-j2)(s+2+j2)}
\end{aligned}
\end{gather}
%
\begin{figure}
\centering
\begin{tikzpicture}[american voltages]
\draw(0,0) to [american voltage source,l={$v_d(t)$}]++(0,\y) to [inductor,l={$\substack{\displaystyle L \hfill\\ \displaystyle \SI{1}{\henry}}$}]++(\x,0) to [capacitor,l={$\substack{\displaystyle C \hfill \\ \displaystyle \SI{0.125}{\farad}}$}]++(\x,0) to [resistor,l_={$\substack{\displaystyle R \hfill \\ \displaystyle \SI{4}{\ohm}}$},v^<={$v_R(t)$}]++(0,-\y) to [short] (0,0);
\end{tikzpicture}
\caption{\عددی{RLC} دور۔}
\label{شکل_لاپلاس_استعمال_سلسلہ_وار_دور_قطبین_صفر}
\end{figure}

درج بالا تبادلی تفاعل کی تین بعدی مقداری ترسیم شکل \حوالہ{شکل_لاپلاس_استعمال_تین_بعدی_ترسیم_الف} میں دکھائی گئی ہے۔تبادلی تفاعل کا صفر \عددی{s=0} پر پایا جاتا ہے جبکہ \عددی{s=-2\mp j2} پر قطبین پائے جاتے ہیں۔یوں قطبین پر تین بعدی ترسیم لامتناہی ہو گی جبکہ صفر پر اس کی قیمت صفر ہو گی۔
\begin{figure}
\centering
\includegraphics{figLaplaceApplicationPoleZeroPlotRLCA}
\caption{مساوات \حوالہ{مساوات_لاپلاس_استعمال_تین_بعدی_الف} کا تین بعدی ترسیم۔}
\label{شکل_لاپلاس_استعمال_تین_بعدی_ترسیم_الف}
\end{figure}

حقیقی دنیا میں تعدد \عددی{\omega} ہوتا ہے نا کہ \عددی{\delta+j\omega} جو کہ مخلوط تعدد ہے۔بوڈا مقداری خط \عددی{\omega} بالمقابل مقدار کا خط ہے۔مخلوط سطح کے  خیالی محور پر \عددی{s=j\omega} ہوتا ہے لہٰذا مخلوط سطح کے خیالی محور پر بوڈا مقداری خط پایا جاتا ہے۔تین بعدی ترسیم کو \عددی{\delta=0} پر کاٹتے ہوئے شکل \حوالہ{شکل_لاپلاس_استعمال_تین_بعدی_ترسیم_ب} ملتی ہے جس کے خیالی محور پر بوڈا مقداری خط کو موٹی لکیر سے دکھایا گیا ہے۔    
\begin{figure}
\centering
\includegraphics{figLaplaceApplicationPoleZeroPlotRLCB}
\caption{تین بعدی ترسیم کے خیالی محور پر بوڈا خط پایا جاتا ہے۔}
\label{شکل_لاپلاس_استعمال_تین_بعدی_ترسیم_ب}
\end{figure}
شکل \حوالہ{شکل_لاپلاس_استعمال_تین_بعدی_ترسیم_ب} کو یوں گھماتے ہیں کہ حقیقی محور صفحہ کتاب کے عمودی ہو۔اس طرح شکل \حوالہ{شکل_لاپلاس_استعمال_تین_بعدی_ترسیم_پ} ملتا ہے جہاں حقیقی محور کے دونوں جانب برابر فاصلے پر قطبین دیکھے جا سکتے ہیں۔
\begin{figure}
\centering
\includegraphics{figLaplaceApplicationPoleZeroPlotRLCC}
\caption{تین بعدی ترسیم کا حقیقی محور صفحہ کتاب کے عمودی ہے۔}
\label{شکل_لاپلاس_استعمال_تین_بعدی_ترسیم_پ}
\end{figure}
اس شکل میں حقیقی محور \عددی{\delta} کے دونوں جانب بوڈا خط بالکل یکساں ہے لہٰذا ہم خیالی محور کا مثبت حصہ لیتے ہوئے شکل \حوالہ{شکل_لاپلاس_استعمال_تین_بعدی_ترسیم_ت} حاصل کرتے ہیں جہاں صرف اور صرف خیالی محور پر تفاعل کا مقدار دکھایا گیا ہے۔یہی بوڈا مقداری خط ہے۔
\begin{figure}
\centering
\includegraphics{figLaplaceApplicationPoleZeroPlotRLCD}
\caption{تین بعدی ترسیم کے مثبت خیالی محور پر بوڈا مقداری خط پایا جاتا ہے۔}
\label{شکل_لاپلاس_استعمال_تین_بعدی_ترسیم_ت}
\end{figure}
%==================

\حصہ{برقرار حال ردعمل}
کسی بھی نظام کے عارضی ردعمل اور برقرار ردعمل کا مجموعہ مکمل ردعمل ہوتا ہے۔عارضی ردعمل  \عددی{t \to \infty} پر ختم ہو جاتا ہے جبکہ برقرار ردعمل تمام اوقات پر پایا جاتا ہے۔آئیں برقرار ردعمل کو براہ راست حاصل کرنے کا طریقہ سیکھیں۔آپ جانتے ہیں کہ ردعمل کو درج ذیل لکھا جا سکتا ہے
\begin{align}\label{مساوات_لاپلاس_استعمال_برقرار_ردعمل_عمومی}
\kB{Y}(s)=\bH(s)\kB{X}(s)
\end{align}
جہاں \عددی{\kB{X}(s)} داخلی اشارہ، \عددی{\kB{Y}(s)} ردعمل اور \عددی{\bH(s)} نظام کا تبادلی تفاعل ہے۔عارضی ردعمل \عددی{\bH(s)} کے قطبین سے پیدا ہوتا ہے جبکہ برقرار ردعمل داخلی اشارے یعنی جبری تفاعل کے قطبین سے پیدا ہوتا ہے۔

بالکل حصہ \حوالہ{حصہ_برقرار_مخلوط_جبری_تفاعل} کی طرح چلتے ہوئے ہم فرض کرتے ہیں کہ داخلی اشارہ مخلوط تفاعل
\begin{align}\label{مساوات_لاپلاس_استعمال_مخلوط_جبری_الف}
x(t)=X_0e^{j(\omega_0 t+\theta) }
\end{align}
ہے جس کا لاپلاس بدل درج ذیل ہے۔
\begin{align*}
\kB{X}(s)=\frac{X_0 e^{j\theta}}{s-j\omega_0}
\end{align*}
یوں ردعمل
\begin{align*}
\kB{Y}(s)&=\bH(s) \kB{X}(s)\\
&=\bH(s)\left(\frac{X_0 e^{j\theta}}{s-j\omega_0}\right)
\end{align*}
ہو گا۔یہاں ہم فرض کرتے ہیں کہ داخلی اشارے میں \عددی{\tfrac{1}{s-j\omega_0}} نہیں پایا جاتا یعنی اس میں \عددی{j\omega_0} قطب نہیں پایا جاتا۔اگر داخلی اشارے میں \عددی{j\omega_0} قطب پایا جاتا ہو تب ہمیں برقرار حالت دریافت کرنے میں دشواری پیش آتی ہے۔ درج بالا کا جزوی کسری پھیلاو لکھتے ہیں۔
\begin{align*}
\kB{Y}(s)=\frac{K_1}{s-j\omega_0}+\text{\RL{تبادلی تفاعل $\bH(s)$ کے قطبین سے پیدا کسر}}
\end{align*}
مستقل \عددی{K_1} حاصل کرنے کی خاطر مساوات کے دونوں اطراف کو \عددی{s-j\omega_0} سے ضرب دیتے ہوئے \عددی{s=j\omega_0} پر کرتے ہیں۔
\begin{align*}
K_1&=\bH(\omega_0)X_0 e^{j\theta}\\
&=\abs{\bH(\omega_0)}X_0 e^{j(\phi_{H0}+\theta)}
\end{align*}
یوں جزوی کسری پھیلاو درج ذیل لکھی جا سکتی ہے
\begin{align*}
\kB{Y}(s)=\frac{\abs{\bH(\omega_0)}X_0 e^{j(\phi_{H0}+\theta)}}{s-j\omega_0}+\cdots
\end{align*}
جس کا الٹ لاپلاس بدل لیتے ہیں۔
\begin{align*}
y(t)&=\abs{\bH(\omega_0)}X_0 e^{j(\phi_{H0}+\theta)}e^{j\omega_0 t}+\cdots\\
&=\abs{\bH(\omega_0)}X_0 e^{j(\omega_0 t+\phi_{H0}+\theta)}+\cdots
\end{align*}
درج بالا مساوات میں دیا جزو جبری ردعمل یا برقرار ردعمل ہے جبکہ بقایا اجزاء فطری ردعمل یا عارضی ردعمل کو ظاہر کریں گی۔یوں برقرار حال یا جبری ردعمل درج ذیل ہو گا
  \begin{align*}
y_j(t)=y_{\text{برقرار}}(t)&=\abs{\bH(\omega_0)}X_0 e^{j(\omega_0 t+\phi_{H0}+\theta)}
\end{align*}
جو مخلوط ردعمل ہے۔اصل جبری تفاعل مساوات \حوالہ{مساوات_لاپلاس_استعمال_مخلوط_جبری_الف} کا حقیقی جزو یعنی \عددی{x(t)=X_0\cos(\omega_0 t +\theta)} ہو گا۔اسی طرح اصل برقرار ردعمل درج بالا مساوات کا حقیقی جزو ہو گا یعنی
  \begin{align}\label{مساوات_لاپلاس_استعمال_مخلوط_جبری_ب}
y_j(t)=y_{\text{برقرار}}(t)&=\abs{\bH(\omega_0)}X_0 \cos (\omega_0 t+\phi_{H0}+\theta)
\end{align}
%========================

\ابتدا{مثال}\شناخت{مثال_لاپلاس_استعمال_برقرار_ردعمل_الف}
شکل \حوالہ{شکل_لاپلاس_استعمال_برقرار_ردعمل_الف}-الف میں برقرار خارجی اشارہ \عددی{v_0(t)} دریافت کریں جہاں \عددی{v_d(t)=10\cos 4t \,u(t)\,\si{\volt}} ہے۔
\begin{figure}
\centering
\begin{subfigure}{1\textwidth}
\centering
\begin{tikzpicture}[american voltages]
\draw(0,0) to [american voltage source,l={${v_d(t),\si{\volt}}$}]++(0,\y) to [resistor,l={$\SI{1}{\ohm}$}]++(\x,0) to [inductor,l={$\SI{4}{\henry}$}]++(\x,0) to [resistor,l_={$\SI{2}{\ohm}$},v^<={$v_0(t)$}]++(0,-\y) to [short](0,0);
\draw(\x,0)node[ground]{} to [capacitor,*-*,l={$\SI{0.5}{\farad}$}]++(0,\y)node[above]{$v_1(t)$};
\end{tikzpicture}
\caption*{(الف)}
\end{subfigure}
\begin{subfigure}{1\textwidth}
\centering
\begin{tikzpicture}[american voltages]
\draw(0,0) to [american voltage source,l={$\bV_d(s)$}]++(0,\y) to [resistor,l={$1$}]++(\x,0) to [inductor,l={$4s$}]++(\x,0) to [resistor,l_={$2$},v^<={$\bV_0(s)$}]++(0,-\y) to [short](0,0);
\draw(\x,0)node[ground]{} to [capacitor,*-*,l={$\frac{2}{s}$}]++(0,\y)node[above]{$\bV_1(s)$};
\end{tikzpicture}
\caption*{(ب)}
\end{subfigure}%
\caption{مثال \حوالہ{مثال_لاپلاس_استعمال_برقرار_ردعمل_الف} کا دور۔}
\label{شکل_لاپلاس_استعمال_برقرار_ردعمل_الف}
\end{figure}

حل:شکل-ب میں لاپلاس بدل دکھایا گیا ہے۔اس دور کو کسی بھی طریقے سے حل کیا جا سکتا ہے۔ہم ترکیب جوڑ سے \عددی{\bV_1(s)} حاصل کرتے ہوئے تقسیم دباو کے کلیے سے  \عددی{\bV_0(s)} حاصل کریں گے۔کرخوف مساوات جوڑ لکھتے ہیں
\begin{align*}
\frac{\bV_1(s)-\bV_d(s)}{1}+\frac{\bV_1(s)}{\frac{2}{s}}+\frac{\bV_1(s)}{4s+2}=0
\end{align*}
جس سے
\begin{align*}
\bV_1(s)&=\frac{\bV_d(s)}{1+\frac{s}{2}+\frac{1}{4s+2}}\\
&=\frac{(4s+2)\bV_d(s)}{2s^2+5s+3}
\end{align*}
ملتا ہے۔تقسیم دباو کے کلیے سے خارجی دباو لکھتے ہیں۔
\begin{align*}
\bV_0(s)&=\bV_1(s)\left(\frac{2}{4s+2}\right)\\
&=\frac{(4s+2)\bV_d(s)}{2s^2+5s+3}\left(\frac{2}{4s+2}\right)\\
&=\frac{2\bV_d(s)}{2s^2+5s+3}
\end{align*}
مساوات \حوالہ{مساوات_لاپلاس_استعمال_برقرار_ردعمل_عمومی} کے ساتھ موازنہ کرتے ہوئے تبادلی تفاعل لکھا جا سکتا ہے۔
\begin{align*}
\bH(s)=\frac{2}{2s^2+5s+3}
\end{align*}
دی گئی داخلی اشارے کی تعدد \عددی{\omega_0=\SI{4}{\radian\per\second}} ہے لہٰذا اس تعدد پر تبادلی تفاعل کو درج ذیل لکھا جا سکتا ہے۔
\begin{align*}
\bH(j4)&=\frac{2}{2(j4)^2+5(j4)+3}\\
&=0.057\phase{214.6^{\circ}}
\end{align*}
یوں مساوات \حوالہ{مساوات_لاپلاس_استعمال_مخلوط_جبری_ب} سے برقرار ردعمل لکھی جا سکتی ہے۔
\begin{align*}
v_0(t)&=10(0.057)\cos(4t+214.6^{\circ})\\
&=0.57\cos(4t+214.6^{\circ})\,\si{\volt}
\end{align*}
مکمل ردعمل 
\begin{align*}
\bV_0(s)&=\frac{2\bV_d(s)}{2s^2+5s+3}\\
&=\frac{20s}{(s^2+4^2)(2s^2+5s+3)}
\end{align*}
کے  الٹ لاپلاس بدل سے حاصل کیا جا سکتا ہے۔
\انتہا{مثال}
%========================
\ابتدا{مشق}\شناخت{مشق_لاپلاس_استعمال_برقرار_الف}
شکل \حوالہ{شکل_لاپلاس_استعمال_مشق_برقرار_الف} کا برقرار ردعمل حاصل کریں۔
\begin{figure}
\centering
\begin{tikzpicture}[american voltages]
\draw(0,0) to [american voltage source,l={$6\cos 10t \, u(t) \, \si{\volt}$}]++(0,\y) to [resistor,l={$\SI{2}{\ohm}$}]++(\x,0) to [capacitor,l={$\SI{0.25}{\farad}$}] ++(\x,0) to [resistor,l_={$\SI{4}{\ohm}$},v^<={$v_0(t)$}]++(0,-\y) to [short](0,0);
\draw(\x,0)node[ground]{} to [inductor,*-*,l={$\SI{2}{\henry}$}]++(0,\y)node[above]{$v_1(t)$};
\end{tikzpicture}
\caption{مشق \حوالہ{مشق_لاپلاس_استعمال_برقرار_الف} کا دور۔}
\label{شکل_لاپلاس_استعمال_مشق_برقرار_الف}
\end{figure}

جواب:\عددی{v_0(t)=3.99\cos(10t+7.6^{\circ})u(t)\,\si{\volt}}
\انتہا{مشق}
%=======================
\ابتدا{مشق}\شناخت{مشق_لاپلاس_استعمال_برقرار_ب}
شکل \حوالہ{شکل_لاپلاس_استعمال_مشق_برقرار_ب} کا برقرار ردعمل حاصل کریں۔
\begin{figure}
\centering
\begin{tikzpicture}[american voltages]
\draw(0,0) to [capacitor,l={$\SI{0.2}{\farad}$}]++(0,\y) to [resistor,l={$\SI{2}{\ohm}$}]++(\x+\x/2,0) to [short]++(\x,0) to [resistor,l={$\SI{2}{\ohm}$}]++(\x,0) to [capacitor,l_={$\SI{0.25}{\farad}$},v^<={$v_0(t)$}]++(0,-\y) to [short](0,0);
\draw(\x+\x/2,\y) to [american current source,*-*,l_={$3\cos 8t \,\si{\ampere}$}]++(0,-\y);
\draw(2*\x+\x/2,0) to [inductor,*-*,l={$\SI{2}{H}$}]++(0,\y);
\end{tikzpicture}
\caption{مشق \حوالہ{مشق_لاپلاس_استعمال_برقرار_ب} کا دور۔}
\label{شکل_لاپلاس_استعمال_مشق_برقرار_ب}
\end{figure}

جواب:\عددی{v_0(t)=0.768\cos(8t+92^{\circ})u(t)\,\si{\volt}}
\انتہا{مشق}
%========================
\ابتدا{مثال}\شناخت{مثال_لاپلاس_استعمال_فلٹر_الف}
شکل \حوالہ{شکل_لاپلاس_استعمال_فلٹر_الف} میں پست گزار چھلنی دکھائی گئی ہے۔اس کو استعمال کرتے ہوئے دیکھا گیا کہ مستطیل داخلی دباو  پر خارجی دباو درکار قیمت سے تجاوز کرتے ہوئے آگے نکل جاتا ہے جو کم تقصیر کی نشانی ہے۔تقصیر بڑھاتے ہوئے اس مسئلے کو حل کریں۔
\begin{figure}
\centering
\begin{tikzpicture}[american voltages]
\draw(0,0) to [inductor,o-,l={$\substack{\displaystyle L \hfill \\ \displaystyle \SI{250}{\micro\henry}}$}]++(\x,0) to [short]++(\x,0) to [capacitor,l={$\substack{\displaystyle C \hfill \\ \displaystyle \SI{0.1}{\micro\farad}}$}]++(0,-\y) to [short,-o]++(-2*\x,0);
\draw(\x,0) to [resistor,*-*,l={$\substack{\displaystyle R \hfill \\ \displaystyle \SI{62.5}{\ohm}}$}]++(0,-\y); 
\draw(0.3,-\y/2)node{$\begin{aligned} &+ \\ &v_d(t)  \\ &-\end{aligned}$};
\draw(2*\x,0) to [short,*-o]++(\x,0);
\draw(2*\x,-\y) to [short,*-o]++(\x,0);
\draw(2*\x+\x+0.2,-\y/2)node{$\begin{aligned} &+ \\ &v_0(t)  \\ &-\end{aligned}$};
\end{tikzpicture}
\caption{مثال \حوالہ{مثال_لاپلاس_استعمال_فلٹر_الف} کا دور۔}
\label{شکل_لاپلاس_استعمال_فلٹر_الف}
\end{figure}

حل:متوازی جڑے برق گیر اور مزاحمت کی رکاوٹ \عددی{\tfrac{R}{1+sRC}} لیتے ہوئے تقسیم دباو کے کلیے سے چھلنی کا تبادلی تفاعل لکھتے ہیں۔
\begin{align*}
\bH(s)&=\frac{\frac{R}{1+sRC}}{sL+\frac{R}{1+sRC}}\\
&=\frac{\frac{1}{LC}}{s^2+\frac{s}{RC}+\frac{1}{LC}}\\
&=\frac{\omega_0^2}{s^2+2\zeta \omega_0 s +\omega_0^2}
\end{align*}
پرزوں کی دی گئی قیمتیں پر کرتے ہوئے درج ذیل حاصل ہوتا ہے۔
\begin{align*}
\bH(s)=\frac{4\times 10^{10}}{s^2+1.6\times 10^5 s+4\times 10^{10}}
\end{align*}
یعنی \عددی{\omega_0=\tfrac{1}{\sqrt{LC}}=\SI{200}{\kilo\radian\per\second}} اور \عددی{\zeta=\tfrac{160000}{2\times 200000}=0.4} ہیں۔یقیناً تقصیری مستقل کی قیمت نہایت کم ہے جس کو بڑھا کر \عددی{\zeta=1} کرنے سے ہمارا مسئلہ حل ہو سکتا ہے۔تعدد کو تبدیل کئے بغیر ایسا مزاحمت کو تبدیل کرنے سے ہو گا۔مزاحمت کی نئی
 قیمت \عددی{2\zeta \omega_0=\tfrac{1}{RC}} سے حاصل کرتے ہیں۔
\begin{align*}
R&=\frac{1}{2\zeta \omega_0 C}\\
&=\frac{1}{2\times 1 \times 2\times 10^5\times 0.1\times 10^{-1}}\\
&=\SI{25}{\ohm}
\end{align*}
یوں مزاحمت کو تبدیل کرتے ہوئے \عددی{\SI{62.5}{\ohm}} کی جگہ \عددی{\SI{25}{\ohm}} نسب کرنے سے خارجی اشارہ درکار حد سے آگے گزرنا بند کر دیگا۔
\انتہا{مثال}
%=======================
\ابتدا{مشق}
شکل میں سلسلہ وار \عددی{RLC} دکھایا گیا ہے۔خارجی اشارہ مختلف پرزوں کے متوازی حاصل کرتے ہوئے اس دور کو بطور پست گزار، بلند گزار اور پٹی گزار چھلنی استعمال کیا جا سکتا ہے۔تینوں  کے تبادلی تفاعل \عددی{\bH(s)=\tfrac{\bV_0(s)}{\bV_d(s)}} لکھیں۔
\begin{figure}
\centering
\begin{subfigure}{1\textwidth}
\centering
\begin{tikzpicture}
\draw(0,0) to [resistor,o-,l={$R$}]++(\x,0) to [inductor,l={$sL$}]++(\x,0) to [capacitor,l={$\frac{1}{sC}$}]++(0,-\y) to [short,-o]++(-2*\x,0);
\draw(2*\x,0) to [short,*-o]++(\x,0);
\draw(2*\x,-\y) to [short,*-o]++(\x,0);
\draw(0.3,-\y/2)node{$\begin{aligned} &+  \\ &\bV_d(s)  \\  &-\end{aligned}$};
\draw(3*\x+0.3,-\y/2)node{$\begin{aligned} &+  \\ &\bV_0(s)  \\  &-\end{aligned}$};
\end{tikzpicture}
\caption*{(الف) پست گزار چھلنی۔}
\end{subfigure}
\begin{subfigure}{1\textwidth}
\centering
\begin{tikzpicture}
\draw(0,0) to [resistor,o-,l={$R$}]++(\x,0)  to [capacitor,l={$\frac{1}{sC}$}]++(\x,0) to [inductor,l={$sL$}]++(0,-\y) to [short,-o]++(-2*\x,0);
\draw(2*\x,0) to [short,*-o]++(\x,0);
\draw(2*\x,-\y) to [short,*-o]++(\x,0);
\draw(0.3,-\y/2)node{$\begin{aligned} &+  \\ &\bV_d(s)  \\  &-\end{aligned}$};
\draw(3*\x+0.3,-\y/2)node{$\begin{aligned} &+  \\ &\bV_0(s)  \\  &-\end{aligned}$};
\end{tikzpicture}
\caption*{(ب) بلند گزار چھلنی۔}
\end{subfigure}
\begin{subfigure}{1\textwidth}
\centering
\begin{tikzpicture}
\draw(0,0) to [inductor,l={$sL$}]++(\x,0) to [capacitor,l={$\frac{1}{sC}$}]++(\x,0)  to [resistor,o-,l={$R$}]++(0,-\y) to [short,-o]++(-2*\x,0);
\draw(2*\x,0) to [short,*-o]++(\x,0);
\draw(2*\x,-\y) to [short,*-o]++(\x,0);
\draw(0.3,-\y/2)node{$\begin{aligned} &+  \\ &\bV_d(s)  \\  &-\end{aligned}$};
\draw(3*\x+0.3,-\y/2)node{$\begin{aligned} &+  \\ &\bV_0(s)  \\  &-\end{aligned}$};
\end{tikzpicture}
\caption*{(پ) پٹی گزار چھلنی۔}
\end{subfigure}
\end{figure}

جوابات:
\begin{align*}
\bH(s)&=\frac{\frac{1}{LC}}{s^2+\frac{R}{L}s+\frac{1}{LC}} \quad \text{\RL{پست گزار}}\\
\bH(s)&=\frac{s^2}{s^2+\frac{R}{L}s+\frac{1}{LC}} \quad \text{\RL{بلند گزار}}\\
\bH(s)&=\frac{\frac{R}{L}s}{s^2+\frac{R}{L}s+\frac{1}{LC}}  \quad \text{\RL{پٹی گزار}}
\end{align*}
\انتہا{مشق}
%=======================

\حصہء{سوالات}

%==================
\ابتدا{سوال}\شناخت{سوال_لاپلاس_حل_رکاوٹ_الف}
شکل \حوالہ{شکل_سوال_لاپلاس_حل_رکاوٹ_الف} کی داخلی رکاوٹ \عددی{\bZ(s)} حاصل کریں۔
\begin{figure}
\centering
\begin{tikzpicture}
\draw(0,0) to [short,o-]++(2*\xx,0) to [resistor,l={$\SI{2}{\ohm}$}]++(\xx,0) to [capacitor,l={$\SI{0.5}{\farad}$}]++(0,-\yy) to [resistor,l={$\SI{2}{\ohm}$}]++(-\xx,0) to [short,-o]++(-2*\xx,0);
\draw(\xx,0) to [resistor,*-*,l_={$\SI{1}{\ohm}$}]++(0,-\yy);
\draw(\xx+\xx/2,0) to [inductor,*-*,l={$\SI{1}{\henry}$}]++(0,-\yy);
\draw(3*\xx,0) to [resistor,*-*,l={$\SI{1}{\ohm}$}]++(-\xx-\xx/2,-\yy);
\draw[stealth-](\xx/4,-\yy/2)--++(-\xx/4,0)--++(0,-\yy/8)node[below]{$\bZ(s)$};
\end{tikzpicture}
\caption{سوال \حوالہ{سوال_لاپلاس_حل_رکاوٹ_الف} کا دور۔}
\label{شکل_سوال_لاپلاس_حل_رکاوٹ_الف}
\end{figure}

جواب:\عددی{\bZ(s)=\tfrac{2s(4s+3)}{11s^2+16s+6}}
\انتہا{سوال}
%=====================
\ابتدا{سوال}\شناخت{سوال_لاپلاس_حل_رکاوٹ_ب}
شکل \حوالہ{شکل_سوال_لاپلاس_حل_رکاوٹ_ب} میں \عددی{c} اور \عددی{d} کو کھلے سر رکھتے ہوئے \عددی{a} اور \عددی{b} کے مابین رکاوٹ دریافت کریں۔
\begin{figure}
\centering
\begin{tikzpicture}
\pgfmathsetmacro{\ang}{atan(\yyy/(\xxx+\xxx/2))}
\pgfmathsetmacro{\len}{\yyy/sin(\ang)}
\draw(0,0)node[left]{$a$} to [short,o-]++(\xxx/2,0) to [resistor,l={$\SI{1}{\ohm}$}]++(\xxx+\xxx/2,0) to [short,-o]++(\xxx/2,0)node[right]{$c$};
\draw(0,0-\yyy)node[left]{$b$} to [short,o-]++(\xxx/2,0) to [capacitor,l_={$\SI{0.5}{\farad}$}]++(\xxx+\xxx/2,0) to [short,-o]++(\xxx/2,0)node[right]{$d$};
\draw(\xx/2,0) to [resistor,*-,l_={$\SI{2}{\ohm}$}]++(-\ang:\xxx) to [short,-*]++(-\ang:\len-\xxx);
\draw(\xxx+\xxx,0) to [inductor,*-,l={$\SI{1}{\henry}$}]++(-180+\ang:\xxx) to [short,-*]++(-180+\ang:\len-\xxx);
\end{tikzpicture}
\caption{سوال \حوالہ{سوال_لاپلاس_حل_رکاوٹ_ب} اور سوال \حوالہ{سوال_لاپلاس_حل_رکاوٹ_پ} کا دور۔}
\label{شکل_سوال_لاپلاس_حل_رکاوٹ_ب}
\end{figure}

جواب:\عددی{\bZ(s)=\tfrac{2s+2}{s+2}}
\انتہا{سوال}
%====================
\ابتدا{سوال}\شناخت{سوال_لاپلاس_حل_رکاوٹ_پ}
شکل \حوالہ{شکل_سوال_لاپلاس_حل_رکاوٹ_ب} میں \عددی{c} اور \عددی{d} کو آپس میں قصر دور کرتے ہوئے \عددی{a} اور \عددی{b} کے مابین رکاوٹ دریافت کریں۔

جواب:\عددی{\bZ(s):\tfrac{2s^2+6s+4}{3s^2+6}}
\انتہا{سوال}
%===========================
\ابتدا{سوال}\شناخت{سوال_لاپلاس_حل_رکاوٹ_ت}
شکل \حوالہ{شکل_سوال_لاپلاس_حل_رکاوٹ_ت}-الف میں \عددی{v_C(t)} حاصل کریں۔
\begin{figure}
\centering
\begin{subfigure}{0.5\textwidth}
\centering
\begin{tikzpicture}[american voltages]
\draw(0,0) to [american voltage source,l_={$10\,u(t)\,\si{\volt}$}]++(0,2*\y) to [resistor,l={$\SI{2}{\ohm}$}]++(\x,0) to [resistor,l={$\SI{2}{\ohm}$}]++(\x,0) to [inductor,l={$\SI{1}{\henry}$},v={$v_L(t)$}]++(0,-2*\y) to [short](0,0);
\draw(\x,0) to[capacitor,*-,l={$\SI{0.5}{\farad}$},v_>={$v_C(t)$}]++(0,\y) to  [resistor,-*,l={$\SI{1}{\ohm}$}]++(0,\y);
\end{tikzpicture}
\caption*{(الف)}
\end{subfigure}%
\begin{subfigure}{0.5\textwidth}
\centering
\begin{tikzpicture}[american voltages]
\draw(0,0) to [american voltage source,l_={$6\,u(t)\,\si{\volt}$}]++(0,2*\y) to [resistor,l={$\SI{2}{\ohm}$}]++(\x,0) to [resistor,l={$\SI{2}{\ohm}$}]++(\x,0) to [capacitor,l={$\SI{1}{\farad}$},v={$v_0(t)$}]++(0,-2*\y) to [short](0,0);
\draw(\x,0) to [resistor,*-,l={$\SI{2}{\ohm}$}]++(0,\y) to [inductor,-*,l={$\SI{2}{\henry}$}]++(0,\y);
\end{tikzpicture}
\caption*{(ب)}
\end{subfigure}%
\caption{سوال \حوالہ{سوال_لاپلاس_حل_رکاوٹ_ت} تا سوال \حوالہ{سوال_لاپلاس_حل_رکاوٹ_ث} کے ادوار۔}
\label{شکل_سوال_لاپلاس_حل_رکاوٹ_ت}
\end{figure}

جواب:\عددی{v_C(t)=[5-5e^{-\tfrac{4}{3}t}]\,u(t)\,\si{\volt}}

\انتہا{سوال}
%================
\ابتدا{سوال}\شناخت{سوال_لاپلاس_حل_رکاوٹ_ٹ}
شکل \حوالہ{شکل_سوال_لاپلاس_حل_رکاوٹ_ت}-الف میں \عددی{v_L(t)} حاصل کریں۔

جواب:\عددی{v_L(t)=\tfrac{10}{3}e^{-\tfrac{4}{3}t}\, u(t)\,\si{\volt}}
\انتہا{سوال}
%=========================
\ابتدا{سوال}\شناخت{سوال_لاپلاس_حل_رکاوٹ_ث}
شکل \حوالہ{شکل_سوال_لاپلاس_حل_رکاوٹ_ت}-ب میں \عددی{v_0(t)} حاصل کریں۔

جواب:
$v_0(t)=\frac{1}{2\sqrt{17}}\left[6\sqrt{17}-(9+3\sqrt{17})e^{-\tfrac{7}{8}t}+(9-3\sqrt{17})e^{-\big(\tfrac{7+\sqrt{17}}{8}\big)t}\right]\,u(t)$
\انتہا{سوال}
%====================
\ابتدا{سوال}\شناخت{سوال_لاپلاس_حل_رکاوٹ_ج}
شکل \حوالہ{شکل_سوال_لاپلاس_حل_رکاوٹ_ج} میں \عددی{v_0(t)} حاصل کریں۔
\begin{figure}
\centering
\begin{tikzpicture}[american voltages]
\draw(0,0) to [resistor,l={$\SI{6}{\ohm}$}]++(0,\yy);
\draw(\xx+\xx/2,0) to [american current source,*-*,l={$10\,u(t)\,\si{\ampere}$}]++(0,\yy);
\draw(2*\xx+\xx/2,0) to [capacitor,*-*,l={$\SI{1}{\farad}$}]++(0,\yy);
\draw(3*\xx+\xx/2,0) to [resistor,l={$\SI{1}{\ohm}$},v_>={$v_0(t)$}]++(0,\yy);
\draw(0,0) to [short]++(3*\xx+\xx/2,0);
\draw(0,\yy) to [resistor,l={$\SI{4}{\ohm}$}]++(\xx+\xx/2,0) to [inductor,l={$\SI{1}{\henry}$}]++(\xx,0) to [resistor,l={$\SI{2}{\ohm}$}]++(\xx,0);
\end{tikzpicture}
\caption{سوال \حوالہ{سوال_لاپلاس_حل_رکاوٹ_ج} کا دور۔}
\label{شکل_سوال_لاپلاس_حل_رکاوٹ_ج}
\end{figure}

جواب:
$v_0(t)=\tfrac{100}{13}[1-e^{-\tfrac{31}{6}t}(\cosh \tfrac{\sqrt{805}t}{6}+\tfrac{31}{\sqrt{805}}\sinh \tfrac{\sqrt{805}t}{6})]\, u(t)\,\si{\volt}$
\انتہا{سوال}
%================
\ابتدا{سوال}\شناخت{سوال_لاپلاس_حل_رکاوٹ_چ}
شکل \حوالہ{شکل_سوال_لاپلاس_حل_رکاوٹ_چ} میں \عددی{v_C(t)} حاصل کریں۔
\begin{figure}
\centering
\begin{tikzpicture}[american voltages]
\draw(0,0) to [american voltage source,l={$3e^{-t}\,u(t)\,\si{\volt}$}]++(0,\yy) to [resistor,l={$\SI{2}{\ohm}$}]++(\xx,0) to [inductor,l={$\SI{1}{\henry}$}]++(\xx,0) to [capacitor,l={$\SI{0.25}{\farad}$},v={$v_C(t)$}]++(0,-\yy) to [short] (0,0);
\end{tikzpicture}
\caption{سوال \حوالہ{سوال_لاپلاس_حل_رکاوٹ_چ} کا دور۔}
\label{شکل_سوال_لاپلاس_حل_رکاوٹ_چ}
\end{figure}

جواب:$v_C(t)=4e^{-t}(1-\cos \sqrt{3}t)\,u(t)\,\si{\volt}$
\انتہا{سوال}
%================
\ابتدا{سوال}\شناخت{سوال_لاپلاس_حل_رکاوٹ_ح}
شکل \حوالہ{شکل_سوال_لاپلاس_حل_رکاوٹ_ح} میں \عددی{i_x(t)} حاصل کریں۔
\begin{figure}
\centering
\begin{tikzpicture}[american voltages]
\draw(0,0) to [american voltage source,l={$12\,u(t)\,\si{\volt}$}]++(0,\yy) to [resistor,l={$\SI{1}{\ohm}$},i={$i_x(t)$}]++(\xx,0) to [capacitor,l={$\SI{1}{\farad}$}]++(\xx,0) to [american voltage source,l={$6\,u(t)\,\si{\volt}$}]++(0,-\yy) to [short] (0,0);
\draw(\xx,0) to [inductor,*-*,l={$\SI{1}{\henry}$}]++(0,\yy);
\end{tikzpicture}
\caption{سوال \حوالہ{سوال_لاپلاس_حل_رکاوٹ_ح} کا دور۔}
\label{شکل_سوال_لاپلاس_حل_رکاوٹ_ح}
\end{figure}

جواب:
$i_x(t)=[12-e^{-\tfrac{t}{2}}(10\sqrt{3} \sin \tfrac{\sqrt{3}t}{2}-6\cos \tfrac{\sqrt{3}t}{2})]\,u(t)\,\si{\ampere}$
\انتہا{سوال}
%================
\ابتدا{سوال}\شناخت{سوال_لاپلاس_حل_رکاوٹ_خ}
شکل \حوالہ{شکل_سوال_لاپلاس_حل_رکاوٹ_خ} میں \عددی{v_x=8\,u(t)\,\si{\volt}} ہے۔آپ سے گزارش ہے کہ \عددی{v_C(t)} حاصل کریں۔
\begin{figure}
\centering
\begin{tikzpicture}[american voltages]
\draw(0,0) to [american voltage source,l={$v_x(t)$}]++(0,\yy) to [resistor,l={$\SI{1}{\ohm}$}]++(\xx,0) to [resistor,l={$\SI{2}{\ohm}$}]++(\xx,0) to [american voltage source,l={$4\,u(t)\,\si{\volt}$}]++(0,-\yy) to [short] (0,0);
\draw(\xx,0) to [capacitor,*-*,l={$\SI{1}{\farad}$},v_>={$v_C(t)$}]++(0,\yy);
\end{tikzpicture}
\caption{سوال \حوالہ{سوال_لاپلاس_حل_رکاوٹ_خ} کا دور۔}
\label{شکل_سوال_لاپلاس_حل_رکاوٹ_خ}
\end{figure}

جواب:
$v_C(t)=4(1-e^{-\tfrac{3}{2}t})\,u(t)\,\si{\volt}$
\انتہا{سوال}
%================
\ابتدا{سوال}\شناخت{سوال_لاپلاس_حل_رکاوٹ_د}
شکل \حوالہ{شکل_سوال_لاپلاس_حل_رکاوٹ_خ} میں \عددی{v_x=8e^{-t}\,u(t)\,\si{\volt}} ہے۔ \عددی{v_C(t)} حاصل کریں۔

جواب:
$v_C(t)=\left(16e^{-t}-\tfrac{44}{3}e^{-\tfrac{3}{2}t}-\tfrac{4}{3}\right)\,u(t) \,\si{\volt}$
\انتہا{سوال}
%==========================
\ابتدا{سوال}\شناخت{سوال_لاپلاس_حل_رکاوٹ_ڈ}
شکل \حوالہ{شکل_سوال_لاپلاس_حل_رکاوٹ_ڈ} میں  \عددی{v_C(t)} حاصل کریں۔
\begin{figure}
\centering
\begin{tikzpicture}[american voltages]
\draw(0,0) to [american voltage source,l={$17\cos t \,u(t)\,\si{\volt}$}]++(0,\yy) to [resistor,l={$\SI{1}{\ohm}$}]++(\xx,0) to [resistor,l={$\SI{1}{\ohm}$}]++(\xx,0) to [american voltage source,l={$1\,u(t)\,\si{\volt}$}]++(0,-\yy) to [short] (0,0);
\draw(\xx,0) to [capacitor,*-*,l={$\SI{0.5}{\farad}$},v_>={$v_C(t)$}]++(0,\yy);
\end{tikzpicture}
\caption{سوال \حوالہ{سوال_لاپلاس_حل_رکاوٹ_ڈ} کا دور۔}
\label{شکل_سوال_لاپلاس_حل_رکاوٹ_ڈ}
\end{figure}

جواب:
$v_C(t)=\left(8\cos t+2\sin t-7.5e^{-4t}-0.5\right)\,u(t)\,\si{\volt}$
\انتہا{سوال}
%================
\ابتدا{سوال}\شناخت{سوال_لاپلاس_حل_دور_الف}
شکل \حوالہ{شکل_سوال_لاپلاس_حل_دور_الف} میں \عددی{v_s(t)=4 \,u(t)\,\si{\volt}} ہے۔آپ سے گزارش ہے کہ  \عددی{v_x(t)} حاصل کریں۔
\begin{figure}
\centering
\begin{tikzpicture}[american voltages]
\draw(0,0) to [american voltage source,l={$v_s(t)$}]++(0,\yy) to [inductor,l={$\SI{1}{\henry}$}]++(\xx,0) to [resistor,l={$\SI{1}{\ohm}$}]++(\xx,0) to [american voltage source,l={$e^{-t}\,u(t)\,\si{\volt}$}]++(0,-\yy) to [short] (0,0);
\draw(\xx,0) to [resistor,*-*,l={$\SI{2}{\ohm}$},v_>={$v_x(t)$}]++(0,\yy);
\draw(0,\yy) to [short,*-]++(0,3/4*\yy) to [resistor,l={$\SI{1}{\ohm}$}]++(\xx,0) to [capacitor,l={$\SI{1}{\farad}$}]++(\xx,0) to [short,-*]++(0,-3/4*\yy);
\end{tikzpicture}
\caption{سوال \حوالہ{سوال_لاپلاس_حل_دور_الف} کا دور۔}
\label{شکل_سوال_لاپلاس_حل_دور_الف}
\end{figure}

جواب:
$v_x(t)=[4-2e^{-t}-\tfrac{8}{3}e^{-\tfrac{2}{3}t}]\,u(t)\,\si{\volt}$
\انتہا{سوال}
%================
\ابتدا{سوال}\شناخت{سوال_لاپلاس_حل_دور_ب}
شکل \حوالہ{شکل_سوال_لاپلاس_حل_دور_الف} میں \عددی{v_s(t)=4e^{-2t} \,u(t)\,\si{\volt}} ہے۔آپ سے گزارش ہے کہ  \عددی{v_x(t)} حاصل کریں۔

جواب:
$v_x(t)=[\tfrac{10}{3}e^{-\tfrac{2}{3}t}-2e^{-t}-2e^{-2t}]\,u(t)\,\si{\volt}$
\انتہا{سوال}
%===============================
\ابتدا{سوال}\شناخت{سوال_لاپلاس_حل_دور_پ}
شکل \حوالہ{شکل_سوال_لاپلاس_حل_دور_پ} میں \عددی{v_0(t)} حاصل کریں۔
\begin{figure}
\centering
\begin{tikzpicture}[american voltages]
\draw(0,0) to [american voltage source,l={$6\,u(t)\,\si{\volt}$}]++(0,\yy) to [inductor,l={$\SI{1}{\henry}$}]++(\xx+\xx/2,0) to [capacitor,l={$\SI{0.5}{\farad}$}]++(\xx,0) to [resistor,l={$\SI{2}{\ohm}$},v={$v_0(t)$}]++(0,-\yy) to [short] (0,0);
\draw(\xx+\xx/2,0) to [american current source,*-*,l={$2\,u(t)\,\si{\ampere}$}]++(0,\yy);
\end{tikzpicture}
\caption{سوال \حوالہ{سوال_لاپلاس_حل_دور_پ} کا دور۔}
\label{شکل_سوال_لاپلاس_حل_دور_پ}
\end{figure}

جواب:
$v_0(t)=e^{-t}(4\cos t+8\sin t)\,u(t)\,\si{\volt}$
\انتہا{سوال}
%============================
\ابتدا{سوال}\شناخت{سوال_لاپلاس_حل_دور_ت}
شکل \حوالہ{شکل_سوال_لاپلاس_حل_دور_ت} میں \عددی{v_0(t)} حاصل کریں۔
\begin{figure}
\centering
\begin{tikzpicture}[american voltages]
\draw(0,0) to [american voltage source,l={$6\,u(t)\,\si{\volt}$}]++(0,\yy) to [capacitor,l={$\SI{0.5}{\farad}$}]++(\xx+\xx/2,0) to [inductor,l={$\SI{1}{\henry}$}]++(\xx,0)  to [resistor,l={$\SI{2}{\ohm}$},v={$v_0(t)$}]++(0,-\yy) to [short] (0,0);
\draw(\xx+\xx/2,0) to [american current source,*-*,l={$2\,u(t)\,\si{\ampere}$}]++(0,\yy);
\end{tikzpicture}
\caption{سوال \حوالہ{سوال_لاپلاس_حل_دور_ت} کا دور۔}
\label{شکل_سوال_لاپلاس_حل_دور_ت}
\end{figure}

جواب:
$v_0(t)=[4-e^{-t}(4\cos t-8\sin t)]\,u(t)\,\si{\volt}$
\انتہا{سوال}
%=============================
\ابتدا{سوال}\شناخت{سوال_لاپلاس_حل_دور_ٹ}
شکل \حوالہ{شکل_سوال_لاپلاس_حل_دور_ٹ} میں \عددی{v_0(t)} حاصل کریں۔
\begin{figure}
\centering
\begin{tikzpicture}[american voltages]
\draw(0,0) to [american current source,l={$2\,u(t)\,\si{\ampere}$}]++(0,2*\yy) to [resistor,l={$\SI{1}{\ohm}$}]++(\xx,0) to [capacitor,l={$\SI{0.5}{\farad}$}]++(\xx,0) to [inductor,l={$\SI{1}{\henry}$}]++(\xx,0)  to [resistor,l={$\SI{1}{\ohm}$},v={$v_0(t)$}]++(0,-2*\yy) to [short] (0,0);
\draw(\xx,0) to [american voltage source,*-,l={$4\,u(t)\,\si{\volt}$}]++(0,\yy) to [resistor,-*,l={$\SI{1}{\ohm}$}]++(0,\yy);
\draw(2*\xx,0) to [resistor,*-*,l={$\SI{1}{\ohm}$}]++(0,2*\yy);
\end{tikzpicture}
\caption{سوال \حوالہ{سوال_لاپلاس_حل_دور_ٹ} کا دور۔}
\label{شکل_سوال_لاپلاس_حل_دور_ٹ}
\end{figure}

جواب:
$v_0(t)=\tfrac{12}{\sqrt{15}}e^{-\tfrac{7}{4}t}\sin \tfrac{\sqrt{15} t}{4} \, u(t)\,\si{\volt}$
\انتہا{سوال}
%=======================
\ابتدا{سوال}\شناخت{سوال_لاپلاس_حل_دور_ث}
شکل \حوالہ{شکل_سوال_لاپلاس_حل_دور_ث} میں \عددی{v_0(t)} حاصل کریں۔
\begin{figure}
\centering
\begin{tikzpicture}[american voltages]
\draw(0,0) to [resistor,l={$\SI{1}{\ohm}$}]++(0,\yy) to [american current source,l={$4e^{-t}\,u(t)\,\si{\ampere}$}]++(0,\yy);
\draw(\xx+\xx/2,2*\yy) to [capacitor,*-,l_={$\SI{1}{\farad}$}]++(0,-\yy) to [american current source,-*,l_={$2\,u(t)\,\si{\ampere}$}]++(0,-\yy);
\draw(2*\xx+\xx/2,0) to [resistor,l_={$\SI{1}{\ohm}$},v^>={$v_0(t)$}]++(0,\yy) to [american current source,l_={$6\,u(t)\,\si{\ampere}$}]++(0,\yy);
\draw(0,0) to [short]++(2*\xx+\xx/2,0);
\draw(0,2*\yy) to [short]++(2*\xx+\xx/2,0);
\draw(0,\yy) to [resistor,*-*,l={$\SI{1}{\ohm}$}]++(\xx+\xx/2,0) to [inductor,-*,l={$\SI{1}{\henry}$}]++(\xx,0);
\end{tikzpicture}
\caption{سوال \حوالہ{سوال_لاپلاس_حل_دور_ث} کا دور۔}
\label{شکل_سوال_لاپلاس_حل_دور_ث}
\end{figure}

جواب:
$v_0(t)=[2e^{-t}-\tfrac{22}{3}e^{-3t}-\tfrac{2}{3}]\,u(t)\,\si{\volt}$
\انتہا{سوال}
%============================================
\ابتدا{سوال}\شناخت{سوال_لاپلاس_حل_دور_ج}
شکل \حوالہ{شکل_سوال_لاپلاس_حل_دور_ج} میں \عددی{v_0(t)} حاصل کریں۔
\begin{figure}
\centering
\begin{tikzpicture}[american voltages]
\draw(0,0)  to [american voltage source,l={$6e^{-t}\,u(t)\,\si{\volt}$}]++(0,\yy) to [inductor,l={$\SI{1}{\henry}$}]++(0,\yy);
\draw(\xx,0) to [resistor,*-,l_={$\SI{1}{\ohm}$}]++(0,\yy) to [american voltage source,-*,l_={$12\,u(t)\,\si{\volt}$}]++(0,\yy);
\draw(2*\xx,0) to [resistor,l_={$\SI{2}{\ohm}$},v^>={$v_0(t)$}]++(0,\yy) to [resistor,l_={$\SI{1}{\ohm}$}]++(0,\yy);
\draw(0,0) to [short]++(2*\xx,0);
\draw(0,2*\yy) to [short]++(2*\xx,0);
\draw(2*\xx,\yy) to [capacitor,*-*,l={$\SI{0.5}{\farad}$}]++(-\xx,0) to [american voltage source,-*,l={$4e^{-t}\,u(t)\,\si{\volt}$}]++(-\xx,0);
\end{tikzpicture}
\caption{سوال \حوالہ{سوال_لاپلاس_حل_دور_ج} کا دور۔}
\label{شکل_سوال_لاپلاس_حل_دور_ج}
\end{figure}

جواب:
$v_0(t)=(e^{-t}-7e^{-3t}+8)\,u(t)\,\si{\volt}$
\انتہا{سوال}
%=============================
\ابتدا{سوال}\شناخت{سوال_لاپلاس_مسئلہ_الف}
شکل \حوالہ{شکل_سوال_لاپلاس_مسئلہ_الف} میں مسئلہ خطی میل سے  \عددی{v_0(t)} حاصل کریں۔
\begin{figure}
\centering
\begin{tikzpicture}[american voltages]
\draw(0,0) to [american voltage source,l={$6\,u(t)\,\si{\volt}$}]++(0,\yy) to [capacitor,l={$\SI{0.5}{\farad}$}]++(\xx,0) to [inductor,l={$\SI{1}{\henry}$}]++(\xx,0) to [resistor,l={$\SI{1}{\ohm}$}]++(\xx,0) to [resistor,l={$\SI{1}{\ohm}$},v={$v_0(t)$}]++(0,-\yy) to [short] (0,0);
\draw(\xx,0) to [american current source,*-*,l_={$4\,u(t)\,\si{\ampere}$}]++(0,\yy);
\end{tikzpicture}
\caption{سوال \حوالہ{سوال_لاپلاس_مسئلہ_الف} کا دور۔}
\label{شکل_سوال_لاپلاس_مسئلہ_الف}
\end{figure}

جواب:
$v_0(t)=[4-e^{-t}(4\cos t-2\sin t)]\,u(t)\,\si{\volt}$
\انتہا{سوال}
%=============================
\ابتدا{سوال}\شناخت{سوال_لاپلاس_مسئلہ_ب}
شکل \حوالہ{شکل_سوال_لاپلاس_مسئلہ_ب} میں مسئلہ خطی میل سے  \عددی{v_0(t)} حاصل کریں۔
\begin{figure}
\centering
\begin{tikzpicture}[american voltages]
\draw(0,0) to [american voltage source,l={$4\,u(t)\,\si{\volt}$}]++(0,\yy) to [inductor,l={$\SI{1}{\henry}$}]++(\xx,0) to [capacitor,l={$\SI{0.5}{\farad}$}]++(\xx,0)  to [resistor,l={$\SI{1}{\ohm}$}]++(\xx,0) to [resistor,l={$\SI{1}{\ohm}$},v={$v_0(t)$}]++(0,-\yy) to [short] (0,0);
\draw(\xx,0) to [american current source,*-*,l_={$4\,u(t)\,\si{\ampere}$}]++(0,\yy);
\end{tikzpicture}
\caption{سوال \حوالہ{سوال_لاپلاس_مسئلہ_ب} کا دور۔}
\label{شکل_سوال_لاپلاس_مسئلہ_ب}
\end{figure}

جواب:
$v_0(t)=4e^{-t}\cos t \,\u(t)\,\si{\volt}$
\انتہا{سوال}
%=============================
\ابتدا{سوال}\شناخت{سوال_لاپلاس_مسئلہ_پ}
شکل \حوالہ{شکل_سوال_لاپلاس_مسئلہ_پ}  کو تبادلہ منبع  سے حل کرتے ہوئے  \عددی{v_0(t)} حاصل کریں۔
\begin{figure}
\centering
\begin{tikzpicture}[american voltages]
\draw(0,0) to [american voltage source,l={$2\,u(t)\,\si{\volt}$}]++(0,\yy) to [inductor,l={$\SI{2}{\henry}$}]++(\xx,0) to [capacitor,l={$\SI{1}{\farad}$}]++(\xx,0)  to [resistor,l={$\SI{2}{\ohm}$}]++(\xx,0) to [resistor,l={$\SI{1}{\ohm}$},v={$v_0(t)$}]++(0,-\yy) to [short] (0,0);
\draw(\xx,0) to [american current source,*-*,l_={$6\,u(t)\,\si{\ampere}$}]++(0,\yy);
\end{tikzpicture}
\caption{سوال \حوالہ{سوال_لاپلاس_مسئلہ_پ} کا دور۔}
\label{شکل_سوال_لاپلاس_مسئلہ_پ}
\end{figure}

جواب:
$v_0(t)=(10e^{-t}-4e^{-0.5t})\,u(t)\,\si{\volt}$
\انتہا{سوال}
%=============================
\ابتدا{سوال}\شناخت{سوال_لاپلاس_مسئلہ_تبادل_منبع}
شکل \حوالہ{شکل_سوال_لاپلاس_مسئلہ_تبادل_منبع}  کو تبادلہ منبع  سے حل کرتے ہوئے  \عددی{v_0(t)} حاصل کریں۔
\begin{figure}
\centering
\begin{tikzpicture}[american voltages]
\draw(0,0) to [american voltage source,l={$2\,u(t)\,\si{\volt}$}]++(0,\y) to [resistor,l={$\SI{1}{\ohm}$}]++(0,\y) to [inductor,l={$\SI{1}{\henry}$}]++(\x,0) to [capacitor,l={$\SI{1}{\farad}$}]++(\x,0) to [short]++(\x,0)  to [american current source,l={$2\,u(t)\,\si{\ampere}$}]++(0,-2*\y) to  [short] (0,0);
\draw(\x,0) to [resistor,*-,l={$\SI{1}{\ohm}$},v_>={$v_0(t)$}]++(0,\y) to [resistor,-*,l={$\SI{1}{\ohm}$}]++(0,\y);
\draw(2*\x,0) to [resistor,*-*,l={$\SI{1}{\ohm}$}]++(0,2*\y);
\end{tikzpicture}
\caption{سوال \حوالہ{سوال_لاپلاس_مسئلہ_تبادل_منبع} کا دور۔}
\label{شکل_سوال_لاپلاس_مسئلہ_تبادل_منبع}
\end{figure}

جواب:
$v_0(t)=(\tfrac{1}{2}-\tfrac{3}{4}e^{-\tfrac{t}{2}})\,u(t)\,\si{\volt}$
\انتہا{سوال}
%=============================
\ابتدا{سوال}\شناخت{سوال_لاپلاس_مسئلہ_ت}
شکل \حوالہ{شکل_سوال_لاپلاس_مسئلہ_ت}  کو مسئلہ تھونن سے حل کرتے ہوئے  \عددی{v_0(t)} حاصل کریں۔
\begin{figure}
\centering
\begin{tikzpicture}[american voltages]
\draw(0,0) to [american voltage source,l={$4\,u(t)\,\si{\volt}$}]++(0,\yy) to [inductor,l={$\SI{1}{\henry}$}]++(\xx,0) to [capacitor,l={$\SI{2}{\farad}$}]++(\xx,0)  to [resistor,l={$\SI{1}{\ohm}$}]++(\xx,0) to [resistor,l={$\SI{1}{\ohm}$},v={$v_0(t)$}]++(0,-\yy) to [short] (0,0);
\draw(\xx,0) to [american current source,*-*,l_={$6\,u(t)\,\si{\ampere}$}]++(0,\yy);
\draw(2*\xx,0) to [resistor,*-*,l_={$\SI{2}{\ohm}$}]++(0,\yy);
\end{tikzpicture}
\caption{سوال \حوالہ{سوال_لاپلاس_مسئلہ_ت} کا دور۔}
\label{شکل_سوال_لاپلاس_مسئلہ_ت}
\end{figure}

جواب:
$v_0(t)=e^{-\tfrac{t}{2}}(3\cos \tfrac{t}{2}+\sin\tfrac{t}{2})\,u(t)\,\si{\volt}$
\انتہا{سوال}
%=============================
\ابتدا{سوال}\شناخت{سوال_لاپلاس_مسئلہ_ٹ}
شکل \حوالہ{شکل_سوال_لاپلاس_مسئلہ_ٹ}  کو مسئلہ تھونن سے حل کرتے ہوئے  \عددی{i_0(t)} حاصل کریں۔
\begin{figure}
\centering
\begin{tikzpicture}[american voltages]
\draw(0,0) to [american voltage source,l={$4\,u(t)\,\si{\volt}$}]++(0,\yy) to [resistor,l={$\SI{1}{\ohm}$}]++(\xx,0) to [resistor,l={$\SI{2}{\ohm}$},i={$i_0(t)$}]++(\xx,0) to [resistor,l={$\SI{1}{\ohm}$}]++(\xx,0) to [american voltage source,l={$e^{-t}\,u(t)\,\si{\volt}$}]++(0,-\yy) to [short] (0,0);
\draw(\xx,0) to [capacitor,*-*,l={$\SI{1}{\farad}$}]++(0,\yy);
\draw(2*\xx,0)  to [inductor,*-*,l_={$\SI{1}{\henry}$}]++(0,\yy);
\end{tikzpicture}
\caption{سوال \حوالہ{سوال_لاپلاس_مسئلہ_ٹ} کا دور۔}
\label{شکل_سوال_لاپلاس_مسئلہ_ٹ}
\end{figure}

جواب:
$i_0(t)=(\tfrac{1}{3}te^{-t}+e^{-t}-\tfrac{4}{3})\,u(t)\,\si{\ampere}$
\انتہا{سوال}
%=============================
\ابتدا{سوال}\شناخت{سوال_لاپلاس_مسئلہ_ث}
شکل \حوالہ{شکل_سوال_لاپلاس_مسئلہ_ث}  کو مسئلہ تھونن سے حل کرتے ہوئے  \عددی{i_0(t)} حاصل کریں۔
\begin{figure}
\centering
\begin{tikzpicture}[american voltages]
\draw(0,0) to [american voltage source,l={$2e^{-t}\,u(t)\,\si{\volt}$}]++(0,\yy) to [resistor,l={$\SI{1}{\ohm}$}]++(\xx,0) to [resistor,l={$\SI{2}{\ohm}$},i={$i_0(t)$}]++(\xx,0) to [resistor,l={$\SI{1}{\ohm}$}]++(\xx,0) to [american voltage source,l={$2\,u(t)\,\si{\volt}$}]++(0,-\yy) to [short] (0,0);
\draw(\xx,0) to [capacitor,*-*,l={$\SI{1}{\farad}$}]++(0,\yy);
\draw(2*\xx,0)  to [inductor,*-*,l_={$\SI{1}{\henry}$}]++(0,\yy);
\end{tikzpicture}
\caption{سوال \حوالہ{سوال_لاپلاس_مسئلہ_ث} کا دور۔}
\label{شکل_سوال_لاپلاس_مسئلہ_ث}
\end{figure}

جواب:
$i_0(t)=-\tfrac{2}{3}e^{-t}(t+1)\,u(t)\,\si{\ampere}$
\انتہا{سوال}
%=============================
\ابتدا{سوال}\شناخت{سوال_لاپلاس_مسئلہ_ج}
شکل \حوالہ{شکل_سوال_لاپلاس_مسئلہ_ج}  کو مسئلہ تھونن سے حل کرتے ہوئے  \عددی{i_0(t)} حاصل کریں۔
\begin{figure}
\centering
\begin{tikzpicture}[american voltages]
\draw(0,0) to [american voltage source,l={$4e^{-t}\,u(t)\,\si{\volt}$}]++(0,\yy) to [resistor,l={$\SI{1}{\ohm}$}]++(\xx,0) to [resistor,l={$\SI{2}{\ohm}$},i={$i_0(t)$}]++(\xx,0) to [resistor,l={$\SI{1}{\ohm}$}]++(\xx,0) to [american voltage source,l={$2e^{-2t}\,u(t)\,\si{\volt}$}]++(0,-\yy) to [short] (0,0);
\draw(\xx,0) to [capacitor,*-*,l={$\SI{1}{\farad}$}]++(0,\yy);
\draw(2*\xx,0)  to [inductor,*-*,l_={$\SI{1}{\henry}$}]++(0,\yy);
\end{tikzpicture}
\caption{سوال \حوالہ{سوال_لاپلاس_مسئلہ_ج} کا دور۔}
\label{شکل_سوال_لاپلاس_مسئلہ_ج}
\end{figure}

جواب:
$i_0(t)=(-\tfrac{4}{3}te^{-t}+\tfrac{2}{3}e^{-t}-\tfrac{4}{3}e^{-2t})\,u(t)\,\si{\ampere}$
\انتہا{سوال}
%=============================
\ابتدا{سوال}\شناخت{سوال_لاپلاس_تبادلی_الف}
شکل \حوالہ{شکل_سوال_لاپلاس_تبادلی_الف} کا تبادلی تفاعل \عددی{\tfrac{\bI_0(s)}{\bI_s(s)}} حاصل کریں۔
\begin{figure}
\centering
\begin{tikzpicture}[american voltages]
\draw(0,0) to [american current source,l={$i_s(t)$}]++(0,\yy) to [short]++(\xx,0) to [resistor,l={$\SI{1}{\ohm}$}]++(\xx,0) to [short]++(\xx,0) to [resistor,l={$\SI{2}{\ohm}$},i={$i_0(t)$}]++(0,-\yy) to [short] (0,0);
\draw(\xx,0) to [capacitor,*-*,l={$\SI{1}{\farad}$}]++(0,\yy);
\draw(2*\xx,0)  to [inductor,*-*,l_={$\SI{1}{\henry}$}]++(0,\yy);
\end{tikzpicture}
\caption{سوال \حوالہ{سوال_لاپلاس_تبادلی_الف} کا دور۔}
\label{شکل_سوال_لاپلاس_تبادلی_الف}
\end{figure}

جواب:
$\tfrac{\bI_0(s)}{\bI_s(s)}=\tfrac{s}{3s^2+3s+2}$
\انتہا{سوال}
%=============================
\ابتدا{سوال}\شناخت{سوال_لاپلاس_تبادلی_ب}
شکل \حوالہ{شکل_سوال_لاپلاس_تبادلی_ب} کا تبادلی تفاعل \عددی{\tfrac{\bV_0(s)}{\bV_s(s)}} حاصل کریں۔
\begin{figure}
\centering
\begin{tikzpicture}
\draw(0,0)node[op amp](u){};
\draw(u.+) to [short,-o]++(-\x/4,0)node[left]{$v_s(t)$};
\draw(u.-) to [resistor,l_={$R_1$}]++(-\x,0) node[ground]{};
\draw(u.-) to [short,*-]++(0,\y/2) coordinate(kA)to [resistor,l_={$R_2$}]++(\x,0);
\draw(kA) to [short,*-]++(0,\y/2) to [capacitor,l={$C$}]++(\x,0)--++(0,-\y/2)coordinate(kB);
\draw(kB)++(0,\y/4) to [short,*-]++(\x/8,0)-|(u.out) to [short,*-o]++(\x/4,0)node[right]{$v_0(t)$};
\end{tikzpicture}
\caption{سوال \حوالہ{سوال_لاپلاس_تبادلی_ب} کا دور۔}
\label{شکل_سوال_لاپلاس_تبادلی_ب}
\end{figure}

جواب:
$\tfrac{\bV_0(s)}{\bV_s(s)}=\tfrac{\tfrac{1}{R_1 C}}{s+\tfrac{1}{R_2 C}}$
\انتہا{سوال}
%=======================
\ابتدا{سوال}\شناخت{سوال_لاپلاس_تبادلی_پ}
شکل \حوالہ{شکل_سوال_لاپلاس_تبادلی_پ} کا تبادلی تفاعل \عددی{\tfrac{\bV_0(s)}{\bV_s(s)}} حاصل کریں۔
\begin{figure}
\centering
\begin{tikzpicture}
\draw(0,0)node[op amp](u){};
\draw(u.+) to [short]++(-\x/4,0)node[ground]{};
\draw(u.-) to [capacitor,l_={$C_2$}]++(-\x,0) coordinate(kL)to [resistor,l_={$R_1$},-o]++(-\x,0) node[left]{$v_s(t)$};
\draw(u.-) to [short,*-]++(0,\y/2) to [resistor,l={$R_3$}]++(\x,0);
\draw(kL) to [short,*-]++(0,\y) to [capacitor,l={$C_1$}]++(2*\x,0)--++(0,-\y/2)coordinate(kB);
\draw(kL) to [resistor,l={$R_2$}]++(0,-\y)node[ground]{};
\draw(kB)++(0,\y/4) to [short,*-]++(\x/8,0)-|(u.out) to [short,*-o]++(\x/4,0)node[right]{$v_0(t)$};
\end{tikzpicture}
\caption{سوال \حوالہ{سوال_لاپلاس_تبادلی_پ} کا دور۔}
\label{شکل_سوال_لاپلاس_تبادلی_پ}
\end{figure}

جواب:
$\tfrac{\bV_0(s)}{\bV_s(s)}=\tfrac{-sR_2R_3C_2}{s^2R_1R_2R_3C_1C_2+sR_1R_2(C_1+C_2)+R_1+R_2}$
\انتہا{سوال}

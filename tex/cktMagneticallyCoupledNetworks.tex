\باب{مقناطیسی جڑے ادوار}

\حصہ{مشترکہ امالہ}
شکل \حوالہ{شکل_مقناطیسی_خود_امالہ}-الف میں \عددی{N} چکر کا \اصطلاح{لچھا}\فرہنگ{لچھا}\حاشیہب{coil}\فرہنگ{coil} مقناطیسی مادے سے بنائے گئے \اصطلاح{قالب}\فرہنگ{قالب!مقناطیسی}\حاشیہب{core}\فرہنگ{core!magnetic} پر لپیٹا گیا  دکھایا گیا ہے۔اس لچھے میں \عددی{i} رو گزر رہی ہے۔ایمپیئر کے قانون کے تحت رو کے گزرنے سے مقناطیسی میدان پیدا ہوتا ہے۔یوں رو کے گزرنے سے لچھے میں \عددی{\phi} \اصطلاح{مقناطیسی بہاو}\فرہنگ{مقناطیسی بہاو}\فرہنگ{بہاو!مقناطیسی}\حاشیہب{magnetic flux}\فرہنگ{magnetic flux}\فرہنگ{flux!magnetic} پیدا ہوتا ہے جسے ہلکی سیاہی میں نقطہ دار لکیر سے دکھایا گیا ہے۔

لچھے میں رو کی سمت اور مقناطیسی بہاو کی سمت کے تعلق پر غور کریں۔ان کا تعلق دائیں ہاتھ کا قانون کہلاتا ہے۔دائیں ہاتھ کا قانون درج ذیل ہے۔

\ابتدا{قانون}
اگر لچھے کو دائیں ہاتھ سے یوں پکڑا جائے کہ ہاتھ کی چار انگلیاں رو کی سمت میں لپیٹے جائیں تب اسی ہاتھ کا انگوٹھا بہاو کی سمت دے گا۔
\انتہا{قانون}
مقناطیسی بہاو کو کسی مخصوص خطے میں رکھنے کی خاطر مقناطیسی قالب استعمال کیا جاتا ہے۔مقناطیسی بہاو کے لئے مقناطیسی مادے سے گزرنا زیادہ آسان ثابت ہوتا ہے لہٰذا شکل \حوالہ{شکل_مقناطیسی_خود_امالہ}-الف میں بہاو قالب کے اندر ہی رہتے ہوئے گھڑی کے سوئیوں کے گھومنے  کی سمت میں گھومتا ہے۔یوں مقناطیسی بہاو \عددی{\phi} لچھے کے تمام چکروں کے اندر سے گزرتا ہے۔لچھے کا \اصطلاح{ارتباط بہاو}\فرہنگ{ارتباط بہاو}\فرہنگ{بہاو!ارتباط}\حاشیہب{flux linkage}\فرہنگ{flux linkage} \عددی{\lambda} درج ذیل ہے۔
\begin{align}\label{مساوات_مشترک_ارتباط_بہاو_الف}
\lambda=N \phi
\end{align}
اس کتاب میں صرف خطی نظام پر غور کیا گیا ہے۔خطی صورت میں ارتباط بہاو اور رو کا تعلق درج ذیل ہے
\begin{align}\label{مساوات_مشترک_ارتباط_بہاو_ب}
\lambda=L i
\end{align}
جہاں مساوات کے مستقل \عددی{L} کو \اصطلاح{خود امالہ}\فرہنگ{امالہ!خود}\فرہنگ{خود امالہ}\حاشیہب{self inductance}\فرہنگ{inductance!self}\فرہنگ{self inductance} یا \اصطلاح{امالہ} کہتے ہیں۔باب \حوالہ{باب_برق_گیر_امالہ_گیر} میں امالہ پر غور کیا گیا ہے۔درج بالا دو مساوات کو ملاتے ہوئے  بہاو اور رو کا تعلق ملتا ہے۔
\begin{align}
\phi=\frac{Li}{N}
\end{align}
قانون فیراڈے کے تحت بدلتی ارتباط بہاو لچھے میں امالی دباو پیدا کرتا ہے۔
\begin{align}
v=\frac{\dif \lambda}{\dif t}
\end{align}
مساوات \حوالہ{مساوات_مشترک_ارتباط_بہاو_ب} کو درج بالا مساوات میں پر کرتے ہیں۔
\begin{align*}
v=\frac{\dif \lambda}{\dif t}=\frac{\dif (Li)}{\dif t}=L\frac{\dif i}{\dif t}+i\frac{\dif L}{\dif t}
\end{align*}
مستقل امالہ کی صورت میں اس مساوات سے امالہ کی جانی پہچانی درج ذیل مساوات حاصل ہوتی ہے۔
\begin{align}\label{مساوات_مقناطیسی_امالہ_کی_مساوات}
v=L\frac{\dif i}{\dif t}
\end{align}
اس کتاب میں مستقل امالہ پر ہی غور کیا جائے گا۔شکل \حوالہ{شکل_مقناطیسی_خود_امالہ}-ب میں اس امالہ کو دکھایا گیا ہے۔یہاں غور کریں کہ مزاحمت کی طرح امالہ کے دباو اور رو بھی انفعالی رائج سمت کے تحت ہیں۔یوں امالہ میں رو مثبت دباو والے سر سے داخلی ہوتی ہے۔مساوات \حوالہ{مساوات_مقناطیسی_امالہ_کی_مساوات} کہتا ہے کہ  بدلتی رو کے گزرنے سے امالہ میں دباو پیدا ہوتا ہے۔
%
\begin{figure}
\centering
\begin{subfigure}{0.6\textwidth}
\centering
\begin{tikzpicture}[american voltages]
\def\height{3};
\def\width{1.5};
\def\thick{0.4};
\def\depthX{0.2};
\def\depthY{0.2};
\def\gap{0.05};
\def\p{0.2};      %pitch
\def\cTop{2.4}; %top of coil
\def\TL{7};    %number of turns
%flux
\draw[gray,dashed,-stealth](\thick/2,\height-\thick) to [out=90,in=180]++(\thick/2,\thick/2) to [short]++(\width-2*\thick,0) to [out=0,in=90]++(\thick/2,-\thick/2) to [short]++(0,-\height+2*\thick) to [out=-90,in=0]++(-\thick/2,-\thick/2) to [short]++(-\width+2*\thick,0) to [out=180,in=-90]++(-\thick/2,\thick/2) to [short]++(0,0.2);
\draw[gray,-stealth](\width-\thick/2,\height/2)++(0,0.05)--++(0,-0.1);
\draw(\width/2,\height-\thick/2)node[fill=white]{$\phi$};
%core
\draw(0,0)--++(0,\height)--++(\width,0)--++(0,-\height)--cycle;
\draw(0,0)++(\thick,\thick)--++(0,\height-2*\thick)--++(\width-2*\thick,0)--++(0,-\height+2*\thick)--cycle;
%
\draw(\thick,\thick)--++(\depthX,\depthY) --++(0,\height-2*\thick-\depthY);
\draw(\thick,\thick)--++(\depthX,\depthY) --++(\width-2*\thick-\depthX,0);
\draw(0,\height)--++(\depthX,\depthY)--++(\width,0)--++(-\depthX,-\depthY);
\draw(\width,0)--++(\depthX,\depthY)--++(0,\height)--++(-\depthX,-\depthY);
%left winding
\draw (\thick+\depthX,\cTop) to [out=45,in=0] ++(-\thick/2-\depthX,\p/2) to [short]++(-\thick/2,0) to [short] ++(-\x/4,0)coordinate(kTop);
\foreach \l in {0,1,2,...,\TL}{
\draw (0,\cTop-\l*\p) to [out=-135,in=45] ++(\thick+\depthX,-\p);
}
\draw(0,\cTop-\TL*\p-\p) to [short]++ (-\x/4,0)coordinate(kBot);
%current
\draw(kBot) to [short]++(-\x,0) to [american current source,l={$i$}]++(0,1.7)coordinate(currT)|- (kTop);
\draw(currT)++(1,-0.85) node{$\begin{aligned}&+ \\ &v \\ &-   \end{aligned}$};
%text
\draw(0,\height/2)node[left]{$N$};
\draw[stealth-](\width+\depthX,2/3*\height) to [out=45,in=180]++(0.5,0.5)node[right]{\RL{مقناطیسی مادہ}};
\end{tikzpicture}
\caption*{(الف)}
\end{subfigure}%
\begin{subfigure}{0.4\textwidth}
\centering
\begin{circuitikz}[american voltages]
\draw(0,0) to [short,i={$i$},o-]++(\x,0) to [inductor,l={$L$}]++(0,-\y) to [short,-o] ++(-\x,0);
\draw ($(0,0)!0.5!(0,-\y)$)node{$\begin{aligned} &+ \\ &v \\ &-  \end{aligned}$};
\end{circuitikz}
\caption*{(ب)}
\end{subfigure}
\caption{خود امالہ کی تعریف۔}
\label{شکل_مقناطیسی_خود_امالہ}
\end{figure}
%
\begin{figure}
\centering
\begin{tikzpicture}[american voltages]
\def\height{3};
\def\width{1.5};
\def\thick{0.4};
\def\depthX{0.2};
\def\depthY{0.2};
\def\p{0.2};      %pitch
\def\cTop{2.4}; %top of coil
\def\TL{7};    %number of turns
\def\cTopR{2.3}; %top of right coil
\def\TR{6};    %number of right turns
%flux
\draw[gray,-stealth](\thick/2,\height-\thick) to [out=90,in=180]++(\thick/2,\thick/2) to [short]++(\width-2*\thick,0) to [out=0,in=90]++(\thick/2,-\thick/2);
\draw(\width/2,\height-\thick/2)node[fill=white]{$\phi$};
%core
\draw(0,0)--++(0,\height)--++(\width,0)--++(0,-\height)--cycle;
\draw(0,0)++(\thick,\thick)--++(0,\height-2*\thick)--++(\width-2*\thick,0)--++(0,-\height+2*\thick)--cycle;
%
\draw(\thick,\thick)--++(\depthX,\depthY) --++(0,\height-2*\thick-\depthY);
\draw(\thick,\thick)--++(\depthX,\depthY) --++(\width-2*\thick-\depthX,0);
\draw(0,\height)--++(\depthX,\depthY)--++(\width,0)--++(-\depthX,-\depthY);
\draw(\width,0)--++(\depthX,\depthY)--++(0,\height)--++(-\depthX,-\depthY);
%left winding
\draw (\thick+\depthX,\cTop) to [out=45,in=0] ++(-\thick/2-\depthX,\p/2) to [short]++(-\thick/2,0) to [short] ++(-\x/4,0)coordinate(kTop);
\foreach \l in {0,1,2,...,\TL}{
\draw (0,\cTop-\l*\p) to [out=-135,in=45] ++(\thick+\depthX,-\p);
}
\draw(0,\cTop-\TL*\p-\p) to [short]++ (-\x/4,0)coordinate(kBot);
%right winding
\draw (\width-\thick,\cTopR) to [out=135,in=180] ++(\thick/2,\p/2) to [short]++(\thick/2+\depthX,0) to [short,-o] ++(\x,0)coordinate(kTopR);
\foreach \l in {0,1,2,...,\TR}{
\draw (\width+\depthX,\cTopR-\l*\p) to [out=-45,in=135] ++(-\thick-\depthX,-\p);
}
\draw(\width+\depthX,\cTopR-\TR*\p-\p) to [short,-o]++ (\x,0)coordinate(kBotR);
%current
\draw(kBot) to [short]++(-\x,0) to [american current source,l={$i$}]++(0,1.7)coordinate(currT)|- (kTop);
%text
\draw(0,\height/2) node [left]{$N_1$};
\draw(\width+\depthX,\height/2) node [right]{$N_2$};
\draw(currT)++(1,-0.85) node{$\begin{aligned} &+ \\ &v_1 \\ &- \end{aligned}$};
\draw($(kTopR)!0.5!(kBotR)$) node{$\begin{aligned} &+ \\ &v_2 \\ &- \end{aligned}$};
\end{tikzpicture}
\caption{لچھے مقناطیسی میدان کے ذریعے رابطے میں ہیں۔}
\label{شکل_مقناطیسی_مشترکہ_امالہ}
\end{figure}

شکل \حوالہ{شکل_مقناطیسی_خود_امالہ}-الف میں موجود لچھے کے قریب دوسرا لچھا رکھنے سے شکل \حوالہ{شکل_مقناطیسی_مشترکہ_امالہ} حاصل ہوتا ہے۔دوسرے لچھے میں رو نہیں گزر رہی ہے۔پہلے لچھے  کا ارتباط بہاو درج ذیل ہے۔
\begin{align}
\lambda_1=N_1 \phi=L_1 i_1
\end{align}
بدلتی رو کی صورت میں ارتباط بہاو بھی وقت کے ساتھ تبدیل ہو گا۔بدلتا ارتباط بہاو پہلے لچھے میں دباو \عددی{v_1=\tfrac{\dif \lambda_1}{\dif t}=L_1 \tfrac{\dif i_1}{\dif t}} پیدا کرے گا۔متعدد لچھوں کی صورت میں \عددی{L_1} کو \اصطلاح{خود امالہ}\فرہنگ{خود امالہ}\فرہنگ{امالہ!خود}\حاشیہب{self inductance}\فرہنگ{inductance!self} کہا جاتا ہے۔

 دوسرے لچھے کا ارتباط بہاو \عددی{\lambda_2=N_2 \phi} ہے جو دوسرے لچھے میں قانون فیراڈے کے تحت درج ذیل دباو پیدا کرے گا۔
\begin{align}
v_2=\frac{\dif \lambda_2}{\dif t}=\frac{\dif}{\dif t}\left(N_2 \phi\right)=\frac{\dif}{\dif t}\left(N_2 \frac{L_1 i_1}{N_1}\right)=\frac{N_2}{N_1} L_1 \frac{\dif i_1}{\dif t}=L_{21}\frac{\dif i_1}{\dif t}
\end{align}
دوسرے لچھے کا دباو پہلے لچھے کی رو کے وقتی تفرق کے راست تناسب ہے۔راست تناسب کے مستقل \عددی{L_{21}} کو دونوں لچھوں کا \اصطلاح{مشترکہ امالہ}\فرہنگ{مشترکہ امالہ}\فرہنگ{امالہ!مشترکہ}\حاشیہب{mutual inductance}\فرہنگ{mutual inductance}\فرہنگ{inductance!mutual} کہا جاتا ہے جسے ہینری \عددی{\si{\henry}} میں ناپا جاتا ہے۔ ہم کہتے ہیں کہ یہ لچھے آپ میں مقناطیسی میدان کے ذریعہ رابطے میں ہیں۔یوں ان لچھوں کو \اصطلاح{مربوط لچھے}\فرہنگ{مربوط لچھے}\حاشیہب{coupled coils}\فرہنگ{coupled coils} کہا جاتا ہے۔
\begin{figure}
\centering
\begin{subfigure}{1\textwidth}
\centering
\begin{tikzpicture}[american voltages]
\def\height{3};
\def\width{1.5};
\def\thick{0.4};
\def\depthX{0.2};
\def\depthY{0.2};
\def\p{0.2};      %pitch
\def\cTop{2.4}; %top of coil
\def\TL{7};    %number of turns
\def\cTopR{2.3}; %top of right coil
\def\TR{6};    %number of right turns
%flux
\draw[gray,-stealth](\thick/2,\height-\thick) to [out=90,in=180]++(\thick/2,\thick/2) to [short]++(\width-2*\thick,0) to [out=0,in=90]++(\thick/2,-\thick/2);
\draw(\width/2,\height-\thick/2)node[fill=white]{$\phi$};
%core
\draw(0,0)--++(0,\height)--++(\width,0)--++(0,-\height)--cycle;
\draw(0,0)++(\thick,\thick)--++(0,\height-2*\thick)--++(\width-2*\thick,0)--++(0,-\height+2*\thick)--cycle;
%
\draw(\thick,\thick)--++(\depthX,\depthY) --++(0,\height-2*\thick-\depthY);
\draw(\thick,\thick)--++(\depthX,\depthY) --++(\width-2*\thick-\depthX,0);
\draw(0,\height)--++(\depthX,\depthY)--++(\width,0)--++(-\depthX,-\depthY);
\draw(\width,0)--++(\depthX,\depthY)--++(0,\height)--++(-\depthX,-\depthY);
%left winding
\draw (\thick+\depthX,\cTop) to [out=45,in=0] ++(-\thick/2-\depthX,\p/2) to [short]++(-\thick/2,0) to [short] ++(-\x/4,0)coordinate(kTop);
\foreach \l in {0,1,2,...,\TL}{
\draw (0,\cTop-\l*\p) to [out=-135,in=45] ++(\thick+\depthX,-\p);
}
\draw(0,\cTop-\TL*\p-\p) to [short]++ (-\x/4,0)coordinate(kBot);
%right winding
\draw (\width-\thick,\cTopR) to [out=135,in=180] ++(\thick/2,\p/2) to [short]++(\thick/2+\depthX,0) to [short] ++(\x,0)coordinate(kTopR);
\foreach \l in {0,1,2,...,\TR}{
\draw (\width+\depthX,\cTopR-\l*\p) to [out=-45,in=135] ++(-\thick-\depthX,-\p);
}
\draw(\width+\depthX,\cTopR-\TR*\p-\p) to [short]++ (\x,0)coordinate(kBotR);
%current
\draw(kBot) to [short]++(-2*\x,0) to [american current source,l={$i_1$}]++(0,1.7)coordinate(currT)|- (kTop);
\draw(kBotR) to [short]++(\x+\x/4,0)coordinate(kRB) to [american current source,l_={$i_2$}]++(0,1.5)|- (kTopR);
%text
\draw(0,\height/2) node [left]{$N_1$};
\draw(\width+\depthX,\height/2) node [right]{$N_2$};
\draw(currT)++(0.75,-0.85) node{$\begin{aligned} &+ \\ &v_1 \\ &- \end{aligned}$};
\draw(kRB)++(-0.75,0.75) node[]{$\begin{aligned} &+ \\ &v_2 \\ &- \end{aligned}$};
\end{tikzpicture}
\caption*{(الف)}
\end{subfigure}
\begin{subfigure}{1\textwidth}
\centering
\begin{circuitikz}
 \draw(0,0) to [american current source,l={$i_1$}]++(0,\y) to [short]++(2*\x+\x/2,0) to [inductor,l_={$L_1$}]++(0,-\y) to [short](0,0);
\draw(2*\x+\x/2+\x/3,0) to [inductor,l_={$L_2$}]++(0,\y) to [short]++(2*\x+\x/2,0);
\draw(2*\x+\x/2+\x/3,0) to [short]++(2*\x+\x/2,0) to [american current source,l_={$i_2$}]++(0,\y);
%mutual
\draw(2*\x+\x/2+\x/6,\y) node[above]{$M$};
\draw[fill](2*\x+\x/2,\y)++(-0.5,-0.5) circle (\kdot);
\draw[fill](2*\x+\x/2+\x/3,\y)++(0.5,-0.5) circle (\kdot);
%voltages
\draw(0.4,\y/2)node[right]{$\begin{aligned} &+ \\ & v_1=L_1 \frac{\dif i_1}{\dif t}+M\frac{\dif i_2}{\dif t} \\ &-  \end{aligned}$};
\draw(4*\x+\x/2+\x/2+\x/3-0.4,\y/2)node[left]{$\begin{aligned} &+ \\ M\frac{\dif i_1}{\dif t}+L_2\frac{\dif i_2}{\dif t}=& v_2 \\ &-  \end{aligned}$};
\end{circuitikz}
\caption*{(ب)}
\end{subfigure}
\caption{قالب میں لچھوں کے بہاو ایک ہی سمت میں ہیں۔}
\label{شکل_مقناطیسی_مشترکہ_امالہ_ب}
\end{figure}
شکل \حوالہ{شکل_مقناطیسی_مشترکہ_امالہ_ب}-الف میں دونوں لچھوں کو انفرادی منبع سے رو فراہم کی گئی ہے۔دونوں لچھوں پر باری باری غور کریں۔ان کی رو اور قالب کے گرد لچھے کے چکروں کی سمت کو دیکھیں۔انفرادی لچھے کی رو گھڑی کی سمت میں گھومتی بہاو پیدا کرتی ہے۔ اس طرح دونوں رو مل کر مقناطیسی بہاو \عددی{\phi} پیدا کرتی ہیں۔یوں لچھوں کی ارتباط بہاو درج ذیل ہو گی۔
\begin{align}
\lambda_1&=L_1 i_1 +L_{12} i_2\\
\lambda_2&=L_{21} i_1+L_2 i_2
\end{align}
فیراڈے کے قانون کے تحت لچھوں کے دباو حاصل کرتے ہیں۔
\begin{align}
v_1&=\frac{\dif \lambda_1}{\dif t}=L_1\frac{\dif  i_1}{\dif t} +L_{12} \frac{\dif i_2}{\dif t} \label{مساوات_مقناطیسی_مشترک_لچھے_دباو_الف}\\
v_2&=\frac{\dif \lambda_{2}}{\dif t}=L_{21}\frac{\dif  i_1}{\dif t} +L_{2} \frac{\dif i_2}{\dif t}\label{مساوات_مقناطیسی_مشترک_لچھے_دباو_ب}
\end{align}
ان مساوات میں \عددی{L_{12}=L_{21}=M} کے برابر ہے جہاں مشترکہ امالہ کو \عددی{M} سے ظاہر کیا گیا ہے۔لچھے کے دباو کے دو اجزاء ہیں۔پہلا جزو لچھے کی اپنی رو کی بنا ہے اور یہ خود جزو کہلاتا ہے۔دوسرا جزو قریبی لچھے کی رو کے بنا ہے اور یہ مشترک جزو کہلاتا ہے۔ 

شکل \حوالہ{شکل_مقناطیسی_مشترکہ_امالہ_ب}-ب میں \اصطلاح{مربوط} لچھوں کو ظاہر کرنا دکھایا گیا ہے۔لچھوں کے انفرادی خود امالہ کو \عددی{L_1} اور \عددی{L_2} سے ظاہر کیا گیا ہے جبکہ ان کے مابین مشترکہ امالہ کو \عددی{M} سے ظاہر کیا گیا ہے۔
\begin{figure}
\centering
\begin{subfigure}{1\textwidth}
\centering
\begin{tikzpicture}[american voltages]
\def\height{3};
\def\width{1.5};
\def\thick{0.4};
\def\depthX{0.2};
\def\depthY{0.2};
\def\p{0.2};      %pitch
\def\cTop{2.4}; %top of coil
\def\TL{7};    %number of turns
\def\cTopR{2.3}; %top of right coil
\def\TR{6};    %number of right turns
%flux
\draw[gray,-stealth](\thick/2,\height-\thick) to [out=90,in=180]++(\thick/2,\thick/2) to [short]++(\width-2*\thick,0) to [out=0,in=90]++(\thick/2,-\thick/2);
\draw(\width/2,\height-\thick/2)node[fill=white]{$\phi$};
%core
\draw(0,0)--++(0,\height)--++(\width,0)--++(0,-\height)--cycle;
\draw(0,0)++(\thick,\thick)--++(0,\height-2*\thick)--++(\width-2*\thick,0)--++(0,-\height+2*\thick)--cycle;
%
\draw(\thick,\thick)--++(\depthX,\depthY) --++(0,\height-2*\thick-\depthY);
\draw(\thick,\thick)--++(\depthX,\depthY) --++(\width-2*\thick-\depthX,0);
\draw(0,\height)--++(\depthX,\depthY)--++(\width,0)--++(-\depthX,-\depthY);
\draw(\width,0)--++(\depthX,\depthY)--++(0,\height)--++(-\depthX,-\depthY);
%left winding
\draw (\thick+\depthX,\cTop) to [out=45,in=0] ++(-\thick/2-\depthX,\p/2) to [short]++(-\thick/2,0) to [short] ++(-\x/4,0)coordinate(kTop);
\foreach \l in {0,1,2,...,\TL}{
\draw (0,\cTop-\l*\p) to [out=-135,in=45] ++(\thick+\depthX,-\p);
}
\draw(0,\cTop-\TL*\p-\p) to [short]++ (-\x/4,0)coordinate(kBot);
%right winding
\draw(\width+\depthX,\cTopR) to [short]++ (\x,0)coordinate(kTopR);
\foreach \l in {0,1,2,...,\TR}{
\draw (\width-\thick,\cTopR-\l*\p) to [out=-135,in=455] ++(\thick+\depthX,-\p);
}
\draw (\width-\thick,\cTopR-\TR*\p-\p) to [out=-135,in=180] ++(\thick/2,-\p/2) to [short]++(\thick/2+\depthX,0) to [short] ++(\x,0)coordinate(kBotR);
%current
\draw(kBot) to [short]++(-2*\x,0) to [american current source,l={$i_1$}]++(0,1.7)coordinate(currT)|- (kTop);
\draw(kBotR) to [short]++(\x+\x/4,0)coordinate(kRB) to [american current source,l_={$i_2$}]++(0,1.5)|- (kTopR);
%text
\draw(0,\height/2) node [left]{$N_1$};
\draw(\width+\depthX,\height/2) node [right]{$N_2$};
\draw(currT)++(0.75,-0.85) node{$\begin{aligned} &+ \\ &v_1 \\ &- \end{aligned}$};
\draw(kRB)++(-0.75,0.75) node[]{$\begin{aligned} &+ \\ &v_2 \\ &- \end{aligned}$};
\end{tikzpicture}
\caption*{(الف)}
\end{subfigure}
\begin{subfigure}{1\textwidth}
\centering
\begin{circuitikz}
 \draw(0,0) to [american current source,l={$i_1$}]++(0,\y) to [short]++(2*\x+\x/2,0) to [inductor,l_={$L_1$}]++(0,-\y) to [short](0,0);
\draw(2*\x+\x/2+\x/3,0) to [inductor,l_={$L_2$}]++(0,\y) to [short]++(2*\x+\x/2,0);
\draw(2*\x+\x/2+\x/3,0) to [short]++(2*\x+\x/2,0) to [american current source,l_={$i_2$}]++(0,\y);
%mutual
\draw(2*\x+\x/2+\x/6,\y) node[above]{$M$};
\draw[fill](2*\x+\x/2,\y)++(-0.5,-0.5) circle (\kdot);
\draw[fill](2*\x+\x/2+\x/3,0)++(0.5,0.5) circle (\kdot);
%voltages
\draw(0.4,\y/2)node[right]{$\begin{aligned} &+ \\ & v_1=L_1 \frac{\dif i_1}{\dif t}-M\frac{\dif i_2}{\dif t} \\ &-  \end{aligned}$};
\draw(4*\x+\x/2+\x/2+\x/3-0.4,\y/2)node[left]{$\begin{aligned} &+ \\ -M\frac{\dif i_1}{\dif t}+L_2\frac{\dif i_2}{\dif t}=& v_2 \\ &-  \end{aligned}$};
\end{circuitikz}
\caption*{(ب)}
\end{subfigure}
\caption{قالب میں لچھوں کے بہاو آپس میں الٹ سمت ہیں۔}
\label{شکل_مقناطیسی_مشترکہ_الٹ_بہاو}
\end{figure}

شکل \حوالہ{شکل_مقناطیسی_مشترکہ_الٹ_بہاو}-الف میں قالب کے گرد، دائیں لچھے کے چکر الٹائے گئے ہیں۔یوں قالب میں بائیں لچھے کا بہاو گھڑی کی سمت میں گھومتا ہے جبکہ دائیں لچھے کا بہاو گھڑی کی الٹ سمت میں گھومتا ہے لہٰذا کل بہاو \عددی{\phi} حاصل کرنے کی خاطر بائیں لچھے کے بہاو سے دائیں لچھے کا بہاو منفی کرنا ہو گا۔ اس طرح لچھوں کی ارتباط بہاو
\begin{align}
\lambda_1&=L_1 i_1-M i_2\\
\lambda_2&=-M i_1+L_2 i_2
\end{align}
لکھی جائے گی اور ان کے دباو درج ذیل لکھے جائیں گے۔
\begin{align}
v_1&=L_1 \frac{\dif i_1}{\dif t}-M\frac{\dif i_2}{\dif t}\label{مساوات_مقناطیسی_مشترک_لچھے_دباو_پ}\\
v_2&=-M\frac{\dif i_1}{\dif t}+L_2 \frac{\dif i_2}{\dif t}\label{مساوات_مقناطیسی_مشترک_لچھے_دباو_ت}
\end{align}

شکل \حوالہ{شکل_مقناطیسی_مشترکہ_امالہ_ب}-الف میں دونوں لچھوں کی انفرادی بہاو کا مجموعہ قالب میں کل بہاو دیتا ہے جبکہ  شکل \حوالہ{شکل_مقناطیسی_مشترکہ_الٹ_بہاو}-الف  میں بائیں لچھے کے بہاو سے دائیں لچھے کا بہاو تفریق کرنے سے قالب میں کل بہاو  \عددی{\phi} حاصل ہوتا ہے۔لچھوں میں رو کی سمت، قالب کے گرد چکر کی سمت اور قالب میں بہاو کی سمت کو نہایت عمدگی سے نقطوں کی مدد سے ظاہر کیا جاتا ہے۔شکل \حوالہ{شکل_مقناطیسی_مشترکہ_امالہ_ب}-ب اور شکل \حوالہ{شکل_مقناطیسی_مشترکہ_الٹ_بہاو}-ب میں ان نقطوں کا استعمال دکھایا گیا ہے۔

انفرادی لچھے کی رو اور دباو کو انفعالی رائج سمت کے تحت چننیں۔دونوں لچھوں میں نقطوں والے سر سے رو داخل ہونے کی صورت میں دباو کا مشترک جزو مثبت لکھا جاتا ہے جبکہ ایک لچھے کی رو نقطے والے سر اور دوسرے لچھے کی رو بے نقطے والے سر سے داخل ہونے کی صورت میں مشترک دباو منفی لکھا جاتا ہے۔دونوں رو بے نقطے سروں سے داخل ہونے کی صورت میں مشترک دباو  مثبت لکھا جائے گا۔دباو کا خود جزو تمام صورتوں میں انفعالی رائج سمت کے تحت مثبت لکھا جاتا ہے۔یوں شکل \حوالہ{شکل_مقناطیسی_مشترکہ_امالہ_ب}  میں  مساوات \حوالہ{مساوات_مقناطیسی_مشترک_لچھے_دباو_الف} اور مساوات \حوالہ{مساوات_مقناطیسی_مشترک_لچھے_دباو_ب} دباو دیں گے جبکہ شکل \حوالہ{شکل_مقناطیسی_مشترکہ_الٹ_بہاو} میں مساوات \حوالہ{مساوات_مقناطیسی_مشترک_لچھے_دباو_پ} اور مساوات \حوالہ{مساوات_مقناطیسی_مشترک_لچھے_دباو_ت} دباو دیں گے۔

مشترک امالہ کے کرخوف مساوات دباو نسبتاً زیادہ آسانی سے لکھے جاتے ہیں۔
%==============
\ابتدا{مثال}\شناخت{مثال_مقناطیسی_مشترک_امالہ_دباو_الف}
شکل \حوالہ{شکل_مقناطیسی_مشترک_امالہ_دباو_الف} میں دیے دور کے  دونوں اطراف کے دباو کے مساوات لکھیں۔
\begin{figure}
\centering
\begin{circuitikz}
\draw(0,0) rectangle ++(-\boxW,\boxH);
\draw(-0.25,\boxH/2) node[rotate=90]{\RL{بایاں دور}};
\draw(0,0.25) to [short]++(\x,0)coordinate(BL) to [inductor,l={$L_1$}]++(0,\y)coordinate(TL) to [short,i<_={$i_1$}]++(-\x,0);
\draw(\x+\x/3+\x,0.25) to [short]++(-\x,0)coordinate(BR) to [inductor,l_={$L_2$}]++(0,\y)coordinate(TR) to [short,i<^={$i_2$}]++(\x,0);
\draw($(TL)!0.5!(TR)$)node[above]{$M$};
\draw[fill](BL)++(-0.5,0.5) circle (\kdot); 
\draw[fill](BR)++(0.5,0.5) circle (\kdot); 
\draw(2*\x+\x/3,0) rectangle ++(\boxW,\boxH);
\draw(2*\x+\x/3,0)++(\boxW/2,\boxH/2) node[rotate=90]{\RL{دایاں دور}};
\draw(0,\boxH/2) node[right]{$\begin{aligned} &+ \\ &v_1 \\ &-  \end{aligned}$};
\draw(2*\x+\x/3,\boxH/2) node[left]{$\begin{aligned} &- \\ &v_2 \\ &+  \end{aligned}$};
\end{circuitikz}
\caption{مثال \حوالہ{مثال_مقناطیسی_مشترک_امالہ_دباو_الف} کا دور۔}
\label{شکل_مقناطیسی_مشترک_امالہ_دباو_الف}
\end{figure}

حل:بائیں جانب \عددی{v_1} اور \عددی{i_1} عین انفعالی رائج سمت کے تحت لکھے گئے ہیں۔یوں دباو کا خود جزو مثبت لکھا جائے گا۔دونوں لچھوں میں رو بے نقطے سروں سے داخل ہوتی ہے لہٰذا دباو کا مشترک جزو مثبت لکھا جائے گا۔یوں بائیں جانب کرخوف کی مساوات درج ذیل ہو گی۔
\begin{align*}
v_1=L_1 \frac{\dif i_1}{\dif t}+M\frac{\dif i_2}{\dif t}
\end{align*} 
دائیں جانب \عددی{v_2} اور \عددی{i_2} انفعالی رائج سمت کے تحت نہیں چننے گئے ہیں۔یوں دباو کے اجزاء لکھتے ہوئے اس کا خیال رکھا جائے گا۔دوسرے لچھے کی مساوات درج ذیل
\begin{align*}
-v_2=L_2 \frac{\dif i_2}{\dif t}+M\frac{\dif i_1}{\dif t}
\end{align*}
یعنی
\begin{align*}
v_2=-M\frac{\dif i_1}{\dif t}-L_2 \frac{\dif i_2}{\dif t}
\end{align*}
لکھی جائے گی۔ 
\انتہا{مثال}
%=======================
\ابتدا{مثال}\شناخت{مثال_مقناطیسی_مشترک_امالہ_دباو_ب}
شکل \حوالہ{شکل_مقناطیسی_مشترک_امالہ_دباو_ب} کے دور کے کرخوف مساوات دباو لکھیں۔

\begin{figure}
\centering
\begin{circuitikz}[american voltages]
\draw(0,0) to [american voltage source,l={$v_m$}]++(0,\y) to [resistor,l={$R_1$}]++(\x,0)coordinate(kUL) to [inductor,l_={$L_1$},v^<=$v_1$]++(\x,0) to [inductor,l={$L_2$},v>={$v_2$}]++(0,-\y)coordinate(kSB) to [short](0,0);
\draw(kSB) to [short,*-]++(\x,0) to [resistor,l_={$R_2$}]++(0,\y) to [short,-*]++(-\x,0);
\draw[fill](kUL)++(0.5,-0.5) circle (\kdot)++(0.2,-0.2)coordinate(kA);
\draw[fill](kSB)++(-0.5,0.5) circle (\kdot)++(-0.2,0.2)coordinate(kB);
\draw[stealth-stealth] (kA) to [out=-90,in=180] (kB);
\draw($(kA)!0.5!(kB)$)node[shift={(-135:0.5)}]{$M$};
%currents
\draw[stealth-] ([shift={(-135:\x/4)}]\x/2,\y/2) arc (-135:135:\x/4);
\draw(\x/2,\y/2)node{$i_1$};
\draw[stealth-] ([shift={(-135:\x/4)}]2*\x+\x/2,\y/2) arc (-135:135:\x/4);
\draw(2*\x+\x/2,\y/2)node{$i_2$};
\end{circuitikz}
\caption{مثال \حوالہ{مثال_مقناطیسی_مشترک_امالہ_دباو_ب} کا دور۔}
\label{شکل_مقناطیسی_مشترک_امالہ_دباو_ب}
\end{figure}

حل:مشترکہ امالہ کے انفرادی دباو کی نشاندہی \عددی{v_1} اور \عددی{v_2} سے کی گئی ہے جنہیں بالترتیب \عددی{i_1} اور \عددی{i_2} کو دیکھتے ہوئے انفعالی رائج سمت کے تحت چننا گیا ہے۔امالہ \عددی{L_1} کے دباو کے دو اجزاء ہیں۔اس کے خود جزو \عددی{L_1 \tfrac{\dif i_1}{\dif t}} ہے۔امالہ \عددی{L_2} میں رو امالہ \عددی{L_1} کے دباو کا مشترک جزو دیتی ہے۔امالہ \عددی{L_2} کے نقطے والے سر سے کل داخلی ہونے والی رو \عددی{i_2-i_1} لکھی جا سکتی ہے جو \عددی{L_1} کے نقطے والے سر پر مثبت دباو پیدا کرتی ہے۔یوں \عددی{L_1} کا مشترک جزو \عددی{M\tfrac{\dif }{\dif t}(i_2-i_1)} ہے۔اس طرح پہلے امالہ کے لئے درج ذیل لکھا جا سکتا ہے۔
\begin{align}\label{مساوات_مقناطیسی_مثال_مشترک_الف}
v_1=L_1 \frac{\dif i_1}{\dif t}+M\frac{\dif}{\dif t}(i_2-i_1)
\end{align}
امالہ \عددی{L_2} کا خود جزو \عددی{L_2\tfrac{\dif }{\dif t}(i_2-i_1)} ہے۔امالہ \عددی{L_1} کے نقطے والے سر سے \عددی{i_1} داخل ہوتا ہے جو امالہ \عددی{L_2} کے نقطے والے سر پر مثبت دباو پیدا کرے گا۔یوں \عددی{L_2} کے دباو کا مشترک جزو \عددی{M\tfrac{\dif i_1}{\dif t}} ہو گا۔یوں درج ذیل لکھا جا سکتا ہے۔
\begin{align}\label{مساوات_مقناطیسی_مثال_مشترک_ب}
v_2=L_2 \frac{\dif}{\dif t}(i_2-i_1)+M\frac{\dif i_1}{\dif t}
\end{align}
اب دور کو دیکھتے ہوئے کرخوف مساوات لکھتے ہیں۔
\begin{align}\label{مساوات_مقناطیسی_مثال_مشترک_پ}
v_m&=i_1 R_1+v_1-v_2\\
0&=v_2+i_2 R_2
\end{align}
ان میں مساوات \حوالہ{مساوات_مقناطیسی_مثال_مشترک_الف} اور مساوات  \حوالہ{مساوات_مقناطیسی_مثال_مشترک_ب} پر کرتے ہوئے جواب لکھتے ہیں۔
\begin{align}
v_m&=i_1 R_1+L_1 \frac{\dif i_1}{\dif t}+M\frac{\dif}{\dif t}(i_2-i_1)-L_2 \frac{\dif}{\dif t}(i_2-i_1)-M\frac{\dif i_1}{\dif t}\\
0&=L_2 \frac{\dif}{\dif t}(i_2-i_1)+M\frac{\dif i_1}{\dif t}+i_2 R_2
\end{align}
\انتہا{مثال}
%======================
%==============
\ابتدا{مشق}\شناخت{مشق_مقناطیسی_مشترک_امالہ_دباو_پ}
شکل \حوالہ{شکل_مقناطیسی_مشترک_امالہ_دباو_پ} میں دیے دور کے  دونوں اطراف کے دباو لکھیں۔
\begin{figure}
\centering
\begin{circuitikz}
\draw(0,0) rectangle ++(-\boxW,\boxH);
\draw(-0.25,\boxH/2) node[rotate=90]{\RL{بایاں دور}};
\draw(0,0.25) to [short]++(\x,0)coordinate(BL) to [inductor,l={$L_1$}]++(0,\y)coordinate(TL) to [short,i>_={$i_1$}]++(-\x,0);
\draw(\x+\x/3+\x,0.25) to [short]++(-\x,0)coordinate(BR) to [inductor,l_={$L_2$}]++(0,\y)coordinate(TR) to [short,i>^={$i_2$}]++(\x,0);
\draw($(TL)!0.5!(TR)$)node[above]{$M$};
\draw[fill](TL)++(-0.5,-0.5) circle (\kdot); 
\draw[fill](BR)++(0.5,0.5) circle (\kdot); 
\draw(2*\x+\x/3,0) rectangle ++(\boxW,\boxH);
\draw(2*\x+\x/3,0)++(\boxW/2,\boxH/2) node[rotate=90]{\RL{دایاں دور}};
\draw(0,\boxH/2) node[right]{$\begin{aligned} &+ \\ &v_1 \\ &-  \end{aligned}$};
\draw(2*\x+\x/3,\boxH/2) node[left]{$\begin{aligned} &- \\ &v_2 \\ &+  \end{aligned}$};
\end{circuitikz}
\caption{مشق \حوالہ{مشق_مقناطیسی_مشترک_امالہ_دباو_پ} کا دور۔}
\label{شکل_مقناطیسی_مشترک_امالہ_دباو_پ}
\end{figure}

جوابات:\عددی{v_1=-L_1\tfrac{\dif i_1}{\dif t}+M\tfrac{\dif i_2}{\dif t}}، \عددی{v_2=L_2 \tfrac{\dif i_2}{\dif t}-M\tfrac{\dif i_1}{\dif t}}
\انتہا{مشق}
%========================

شکل \حوالہ{شکل_مقناطیسی_وقتی_تعددی_دائرہ_کار}-الف میں وقتی دائرہ کار کا دور جبکہ شکل-ب میں اسی کو تعددی دائرہ کار کی صورت میں دکھایا گیا ہے۔ شکل-ب کے کرخوف مساوات درج ذیل ہیں۔
\begin{align*}
\hat{V}_1&=j\omega L_1 \hat{I}_1+j\omega M \hat{I}_2\\
\hat{V}_2&=j\omega M \hat{I}_1+j\omega L_2 \hat{I}_2
\end{align*}
%
\begin{figure}
\centering
\begin{subfigure}{0.5\textwidth}
\centering
\begin{circuitikz}
\draw(0,0) rectangle ++(-\boxW,\boxH);
\draw(0,0.25) to [short]++(\x,0)coordinate(BL) to [inductor,l={$L_1$}]++(0,\y)coordinate(TL) to [short,i<_={$i_1$}]++(-\x,0);
\draw(\x+\x/3+\x,0.25) to [short]++(-\x,0)coordinate(BR) to [inductor,l_={$L_2$}]++(0,\y)coordinate(TR) to [short,i<^={$i_2$}]++(\x,0);
\draw($(TL)!0.5!(TR)$)node[above]{$M$};
\draw[fill](TL)++(-0.5,-0.5) circle (\kdot); 
\draw[fill](TR)++(0.5,-0.5) circle (\kdot); 
\draw(2*\x+\x/3,0) rectangle ++(\boxW,\boxH);
\draw(0,\boxH/2) node[right]{$\begin{aligned} &+ \\ &v_1 \\ &-  \end{aligned}$};
\draw(2*\x+\x/3,\boxH/2) node[left]{$\begin{aligned} &+ \\ &v_2 \\ &-  \end{aligned}$};
\end{circuitikz}
\caption*{(الف)}
\end{subfigure}%
\begin{subfigure}{0.5\textwidth}
\centering
\begin{circuitikz}
\draw(0,0) rectangle ++(-\boxW,\boxH);
\draw(0,0.25) to [short]++(\x,0)coordinate(BL) to [inductor,l={$j \omega L_1$}]++(0,\y)coordinate(TL) to [short,i<_={$\hat{I}_1$}]++(-\x,0);
\draw(\x+\x/3+\x,0.25) to [short]++(-\x,0)coordinate(BR) to [inductor,l_={$j \omega L_2$}]++(0,\y)coordinate(TR) to [short,i<^={$\hat{I}_2$}]++(\x,0);
\draw($(TL)!0.5!(TR)$)node[above]{$j \omega M$};
\draw[fill](TL)++(-0.5,-0.5) circle (\kdot); 
\draw[fill](TR)++(0.5,-0.5) circle (\kdot); 
\draw(2*\x+\x/3,0) rectangle ++(\boxW,\boxH);
\draw(0,\boxH/2) node[right]{$\begin{aligned} &+ \\ &\hat{V}_1 \\ &-  \end{aligned}$};
\draw(2*\x+\x/3,\boxH/2) node[left]{$\begin{aligned} &+ \\ &\hat{V}_2 \\ &-  \end{aligned}$};
\end{circuitikz}
\caption*{(ب)}
\end{subfigure}%
\caption{وقتی دائرہ کار سے تعددی دائرہ کار کا حصول۔}
\label{شکل_مقناطیسی_وقتی_تعددی_دائرہ_کار}
\end{figure}

%====================
\ابتدا{مثال}
دو عدد مربوط لچھے چار مختلف طریقوں سے آپس میں جوڑے جا سکتے ہیں جنہیں شکل \حوالہ{شکل_مقناطیسی_مربوط_چار_ممکنات} میں دکھایا گیا ہے۔چاروں صورتوں میں ان کا مساوی امالہ حاصل کریں۔شکل میں ان مساوی امالہ \عددی{L_{\text{مساوی}}} کو بھی لکھا گیا ہے۔ 
\begin{figure}
\centering
\begin{subfigure}{0.5\textwidth}
\centering
\begin{tikzpicture}
\draw(0,0) to [american voltage source,l={$\hat{V}_m$}]++(-2*\x,0) to [short,i={$\hat{I}$}]++(0,\y)coordinate(kA) to [inductor,l={$j\omega L_1$}]++(\x,0)coordinate(kB) to [inductor,l={$j\omega L_2$}]++(\x,0)coordinate(kC) to [short]++(0,-\y);
\draw[fill] (kA)++(0.5,-0.5) circle(\kdot)++(0.2,-0.2)coordinate(kD);
\draw[fill] (kB)++(0.5,-0.5) circle(\kdot);
\draw[stealth-stealth](kD) to [out=-45,in=-135]++(\x-0.4,0);
\draw(kD)++(\x/2-0.2,-0.4)node[fill=white]{$j\omega M$};
\draw(-\x,\y+\y/2)node[above]{$L_{\text{مساوی}}=L_1+L_2+2M$};
\end{tikzpicture}
\caption*{(الف)}
\end{subfigure}%
\begin{subfigure}{0.5\textwidth}
\centering
\begin{tikzpicture}
\draw(0,0) to [american voltage source,l={$\hat{V}_m$}]++(-2*\x,0) to [short,i={$\hat{I}$}]++(0,\y)coordinate(kA) to [inductor,l={$j\omega L_1$}]++(\x,0)coordinate(kB) to [inductor,l={$j\omega L_2$}]++(\x,0)coordinate(kC) to [short]++(0,-\y);
\draw[fill] (kA)++(0.5,-0.5) circle(\kdot)++(0.2,-0.2)coordinate(kD);
\draw[fill] (kC)++(-0.5,-0.5) circle(\kdot)++(-0.2,-0.2)coordinate(kE);
\draw[stealth-stealth](kD) to [out=-35,in=-145](kE);
\draw($(kD)!0.5!(kE)$)++(0,-0.4)node[fill=white]{$j \omega M$};
\draw(-\x,\y+\y/2)node[above]{$L_{\text{مساوی}}=L_1+L_2-2M$};
\end{tikzpicture}
\caption*{(ب)}
\end{subfigure}
\begin{subfigure}{0.5\textwidth}
\centering
\begin{tikzpicture}
\draw(0,0) to [american voltage source,l={$\hat{V}_m$}]++(0,\y) to [short,i={$\hat{I}$}]++(3/4*\x,0)coordinate(kA) to [inductor,l={$j\omega L_1$},i={$\hat{I}_1$}]++(0,-\y)coordinate(kB) to [short](0,0);
\draw(3/4*\x,\y) to [short,*-]++(\x,0)coordinate(kC) to [inductor,l={$j\omega L_2$},i={$\hat{I}_2$}]++(0,-\y)coordinate(kD) to [short,-*]++(-\x,0);
\draw[fill](kA)++(-0.5,-0.5) circle (\kdot);
\draw[fill](kC)++(-0.5,-0.5) circle (\kdot);
\draw(3/4*\x+\x/2,\y)node[above]{$j\omega M$};
\draw(\x/2+3/8*\x,\y+\y/2)node[above]{$L_{\text{مساوی}}=\frac{L_1 L_2 -M^2}{L_1+L_2-2M}$};
\end{tikzpicture}
\caption*{(پ)}
\end{subfigure}%
\begin{subfigure}{0.5\textwidth}
\centering
\begin{tikzpicture}
\draw(0,0) to [american voltage source,l={$\hat{V}_m$}]++(0,\y) to [short,i={$\hat{I}$}]++(3/4*\x,0)coordinate(kA) to [inductor,l={$j\omega L_1$}]++(0,-\y)coordinate(kB) to [short](0,0);
\draw(3/4*\x,\y) to [short,*-]++(\x,0)coordinate(kC) to [inductor,l={$j\omega L_2$}]++(0,-\y)coordinate(kD) to [short,-*]++(-\x,0);
\draw[fill](kA)++(-0.5,-0.5) circle (\kdot);
\draw[fill](kD)++(-0.5,0.5) circle (\kdot);
\draw(3/4*\x+\x/2,\y)node[above]{$j\omega M$};
\draw(\x/2+3/8*\x,\y+\y/2)node[above]{$L_{\text{مساوی}}=\frac{L_1 L_2 -M^2}{L_1+L_2+2M}$};
\end{tikzpicture}
\caption*{(ت)}
\end{subfigure}%
\caption{دو مربوط لچھوں کے چار ممکنہ ادوار اور ان کا مساوی امالہ۔}
\label{شکل_مقناطیسی_مربوط_چار_ممکنات}
\end{figure}

حل:شکل \حوالہ{شکل_مقناطیسی_مربوط_چار_ممکنات}-الف کو دیکھتے ہوئے کرخوف مساوات دباو لکھتے ہیں
\begin{align*}
\hat{V}_m&=j\omega L_1 \hat{I}+j\omega M \hat{I}+j\omega L_2 \hat{I}+j\omega M \hat{I}\\
&=j\omega \hat{I}(L_1+L_2+2M)\\
&=j\omega \hat{I}L_{\text{\RL{مساوی}}} 
\end{align*}
جہاں آخری قدم پر قوسین میں بند جزو کو مساوی امالہ \عددی{L_\text{مساوی}} کہا گیا ہے۔
\begin{align}
L_{\text{مساوی}}=L_1+L_2+2M
\end{align}
شکل \حوالہ{شکل_مقناطیسی_مربوط_چار_ممکنات}-ب کو دیکھتے ہوئے کرخوف مساوات دباو لکھتے ہیں
\begin{align*}
\hat{V}_m&=j\omega L_1 \hat{I}-j\omega M \hat{I}+j\omega L_2 \hat{I}-j\omega M \hat{I}\\
&=j\omega \hat{I}(L_1+L_2-2M)\\
&=j\omega \hat{I}L_{\text{\RL{مساوی}}} 
\end{align*}
جہاں آخری قدم پر قوسین میں بند جزو کو مساوی امالہ \عددی{L_\text{مساوی}} کہا گیا ہے۔
\begin{align}
L_{\text{مساوی}}=L_1+L_2-2M
\end{align}
شکل \حوالہ{شکل_مقناطیسی_مربوط_چار_ممکنات}-پ کو دیکھتے ہوئے دونوں لچھوں کے مساوات لکھتے ہیں۔
\begin{align*}
\hat{V}_m&=j\omega L_1 \hat{I}_1+j\omega M \hat{I}_2\\
\hat{V}_m&=j\omega L_2 \hat{I}_2+j\omega M \hat{I}_1
\end{align*}
ان دو عدد ہمزاد مساوات کو حل کرتے ہوئے درج ذیل ملتا ہے۔
\begin{align*}
\hat{I}_1&=\frac{\hat{V}_m(L_2-M)}{j\omega (L_1 L_2-M^2)}\\
\hat{I}_2&=\frac{\hat{V}_m(L_1-M)}{j\omega (L_1 L_2-M^2)}
\end{align*}
کرخوف مساوات رو سے \عددی{\hat{I}=\hat{I}_1+\hat{I}_2} لکھا جا سکتا ہے جس میں درج بالا حاصل شدہ نتائج پر کرتے ہوئے  ترتیب دیتے ہیں
\begin{align*}
\hat{I}&=\hat{I}_1+\hat{I}_2\\
&=\frac{\hat{V}_m(L_1+L_2-M)}{j\omega (L_1 L_2-M^2)}\\
&=\frac{\hat{V}_m}{j\omega L_{\text{مساوی}}}
\end{align*}
جہاں آخری قدم پر مساوی امالہ کی نشاندہی کی گئی ہے یعنی
\begin{align}
L_{\text{مساوی}}=\frac{L_1 L_2-M^2}{L_1+L_2-2M}
\end{align}
\انتہا{مثال}
%======================
\ابتدا{مشق}
شکل \حوالہ{شکل_مقناطیسی_مربوط_چار_ممکنات}-ت میں دیے دور کا مساوی امالہ دریافت کریں۔

جواب:
\begin{align}
L_{\text{مساوی}}=\frac{L_1 L_2-M^2}{L_1+L_2+2M}
\end{align}
\انتہا{مشق}
%=====================
\ابتدا{مثال}\شناخت{مثال_مقناطیسی_مربوط_دور_الف}
شکل \حوالہ{شکل_مقناطیسی_مربوط_دور_الف} میں \عددی{\hat{V}_0} دریافت کریں۔
\begin{figure}
\centering
\begin{tikzpicture}[american voltages]
\draw(0,0) to [american voltage source,l={$30\phase{45^{\circ}}$}\,\si{\volt}]++(0,\y) to [resistor,l={$\SI{2}{\ohm}$}]++(2*\x,0)coordinate(kA) to [inductor,l_={$j2\,\si{\ohm}$}]++(0,-\y)coordinate(kB) to [short] (0,0);
\draw(2*\x+\x/3,0)coordinate(kC) to [inductor,l_={$j4\,\si{\ohm}$}]++(0,\y)coordinate(kD) to [resistor,l={$\SI{4}{\ohm}$}]++(2*\x,0) to [inductor,l={$j6\,\si{\ohm}$},v={$\hat{V}_0$}]++(0,-\y) to [short]++(-2*\x,0);
\draw[fill] (kA)++(-0.5,-0.5) circle (\kdot);
\draw[fill] (kD)++(0.5,-0.5) circle (\kdot);
\draw(2*\x+\x/6,\y)node[above]{$j3\,\si{\ohm}$};
%currents
\draw[stealth-]([shift={(-135:\x/4)}]\x,\y/2) arc (-135:135:\x/4);
\draw(\x,\y/2)node{$\hat{I}_1$};
\draw[stealth-]([shift={(-135:\x/4)}]2*\x+\x/3+\x,\y/2) arc (-135:135:\x/4);
\draw(3*\x+\x/3,\y/2)node{$\hat{I}_2$};
\end{tikzpicture}
\caption{مثال \حوالہ{مثال_مقناطیسی_مربوط_دور_الف} کا دور۔}
\label{شکل_مقناطیسی_مربوط_دور_الف}
\end{figure}

حل:کرخوف مساوات لکھتے ہیں۔
\begin{align*}
30\phase{45^{\circ}}&=(2+j2)\hat{I}_1-j3\hat{I}_2\\
0&=-j3\hat{I}_1+(j4+4+j6)\hat{I}_2
\end{align*}
ان ہمزاد مساوات کو حل کرنے سے درج ذیل ملتا ہے۔
\begin{align*}
\hat{I}_1&=11.474\phase{17.08^{\circ}} \, \si{\ampere}\\
\hat{I}_2&=3.196\phase{38.88^{\circ}}\,\si{\ampere}
\end{align*}
رو \عددی{\hat{I}_2}  کو استعمال کرتے ہوئے خارجی دباو حاصل کرتے ہیں۔
\begin{align*}
\hat{V}_0&=(j6)(\hat{I}_2)=(6\phase{90^{\circ}})(3.196\phase{38.88^{\circ}})=19.176\phase{128.88^{\circ}}\,\si{\volt}
\end{align*}
\انتہا{مثال}
%=====================
\ابتدا{مثال}\شناخت{مثال_مقناطیسی_مربوط_دور_ب}
شکل \حوالہ{شکل_مقناطیسی_مربوط_دور_ب} کر دائری کرخوف مساوات لکھیں۔بعض اوقات دور میں دو عدد سے زیادہ مربوط امالہ موجود ہوتے ہیں۔ایسی صورت میں تیر کے لکیروں سے دو دو امالہ کی نشاندہی کی جاتی ہے۔اس شکل میں \عددی{L_1} اور \عددی{L_2} کے تعلق \عددی{j\omega M} کی نشاندہی کی گئی ہے۔
\begin{figure}
\centering
\begin{tikzpicture}
\draw(0,0) to [american voltage source,l={$\hat{V}_m$}]++(0,2*\y) to [capacitor,l={$\frac{1}{j\omega C_1}$}]++(\x,0) to [resistor,l={$R_1$}]++(\x,0) to [resistor,l={$R_2$}]++(\x,0) to [capacitor,l={$\frac{1}{j\omega C_3}$}]++(0,-2*\y) to [short] (0,0);
\draw(\x,0) to [resistor,*-,l={$R_3$}]++(0,\y) to [inductor,-*,l={$j\omega L_1$}]++(0,\y)coordinate(kA);
\draw(2*\x,0) to [capacitor,*-,l_={$\frac{1}{j\omega C_2}$}]++(0,\y) to [inductor,-*,l_={$j\omega L_2$}]++(0,\y)coordinate(kB);
\draw[fill](kA)++(-0.5,-0.5) circle (\kdot);
\draw[fill](kB)++(0.5,-0.5) circle (\kdot);
%mutual
\draw[-stealth](\x+\x/2,2*\y-0.7)--++(-0.5,-0.5);
\draw[-stealth](\x+\x/2,2*\y-0.7)--++(0.5,-0.5);
\draw(\x+\x/2,2*\y-0.7)node[fill=white]{$j\omega M$};
%currents
\draw[stealth-]([shift={(-135:\x/5)}]\x/2,\y) arc (-135:135:\x/5);
\draw(\x/2,\y)node{$\hat{I}_1$};
\draw[stealth-]([shift={(-135:\x/5)}]\x+\x/2,\y) arc (-135:135:\x/5);
\draw(\x+\x/2,\y)node{$\hat{I}_2$};
\draw[stealth-]([shift={(-135:\x/5)}]2*\x+\x/2,\y) arc (-135:135:\x/5);
\draw(2*\x+\x/2,\y)node{$\hat{I}_3$};
\end{tikzpicture}
\caption{مثال \حوالہ{مثال_مقناطیسی_مربوط_دور_ب} کا دور۔}
\label{شکل_مقناطیسی_مربوط_دور_ب}
\end{figure}

حل:کرخوف مساوات لکھتے ہوئے محتاط اور چوکس رہیں۔تین خانوں کے مساوات درج ذیل ہیں۔
\begin{equation*}
 \begin{split}
\hat{V}_m&=\frac{\hat{I}_1}{j\omega C_1}+j\omega L_1 (\hat{I}_1-\hat{I}_2)+R_3(\hat{I}_1-\hat{I}_2)+j\omega M(\hat{I}_2-\hat{I}_3)\\
0&=R_3(\hat{I}_2-\hat{I}_1)+j\omega L_1(\hat{I}_2-\hat{I}_1)+R_1 I_2+j\omega L_2(\hat{I}_2-\hat{I}_3)\\
&\hspace{2cm} +\frac{1}{j\omega C_2}(\hat{I}_2-\hat{I}_3)-j\omega M(\hat{I}_2-\hat{I}_3)+j\omega M(\hat{I}_1-\hat{I}_2) \\
0&=\frac{\hat{I}_3}{j\omega C_3}+j\omega L_2(\hat{I}_3-\hat{I}_2)+R_2 \hat{I}_3+\frac{\hat{I}_3}{j\omega C_3}-j\omega M(\hat{I}_1-\hat{I}_2)
\end{split}
\end{equation*}
انہیں ترتیب دیتے ہوئے دوبارہ لکھتے ہیں۔ترتیب دینے سے متشاکل مساوات حاصل ہوتے ہیں۔
\begin{equation*}
\begin{split}
\left(\frac{1}{j\omega C_1}+j\omega L_1+R_3\right)\hat{I}_1-\left(j\omega L_2+R_3-j\omega M\right)\hat{I}_2-j\omega M \hat{I}_3&=\hat{V}_m\\
-\left(j\omega L_1+R_3-j\omega M\right)\hat{I}_1+\left(R_3+j\omega L_1+R_1+j\omega L_2+\frac{1}{j\omega C_2}-2j\omega M\right)\hat{I}_2 &\\
-\left(\frac{1}{j\omega C_2+j\omega L_2+R_2+\frac{1}{j\omega C_3}}-j\omega M\right)\hat{I}_3&=0\\
-j\omega M \hat{I}_1-\left(j\omega L_2+\frac{1}{j\omega C_2}-j\omega M\right)\hat{I}_2+\left(\frac{1}{j\omega C_2}+j\omega L_2+R_2+\frac{1}{j\omega C_3}\right)\hat{I}_3&=0
\end{split}
\end{equation*}
\انتہا{مثال}
%=====================
\ابتدا{مشق}\شناخت{مشق_مقناطیسی_مشق_مربوط_دور_الف}
شکل \حوالہ{شکل_مقناطیسی_مشق_مربوط_دور_الف} میں \عددی{\hat{I}_1}، \عددی{\hat{I}_2} اور \عددی{\hat{V}_0} دریافت کریں۔

\begin{figure}
\centering
\begin{tikzpicture}[american voltages]
\draw(0,0) to [american voltage source,l={$20\phase{0^{\circ}}$}\,\si{\volt}]++(0,\y) to [capacitor,l={$-j1\,\si{\ohm}$}]++(2*\x,0)coordinate(kA) to [inductor,l_={$j4\,\si{\ohm}$}]++(0,-\y)coordinate(kB) to [short] (0,0);
\draw(2*\x+\x/3,0)coordinate(kC) to [inductor,l_={$j6\,\si{\ohm}$}]++(0,\y)coordinate(kD) to [resistor,l={$\SI{2}{\ohm}$}]++(2*\x,0) to [capacitor,l={$-j2\,\si{\ohm}$},v={$\hat{V}_0$}]++(0,-\y) to [short]++(-2*\x,0);
%dots
\draw[fill] (kA)++(-0.5,-0.5) circle (\kdot);
\draw[fill] (kC)++(0.5,0.5) circle (\kdot);
\draw(2*\x+\x/6,\y)node[above]{$j2\,\si{\ohm}$};
%currents
\draw[stealth-]([shift={(-135:\x/4)}]\x,\y/2) arc (-135:135:\x/4);
\draw(\x,\y/2)node{$\hat{I}_1$};
\draw[stealth-]([shift={(-135:\x/4)}]2*\x+\x/3+\x,\y/2) arc (-135:135:\x/4);
\draw(3*\x+\x/3,\y/2)node{$\hat{I}_2$};
\end{tikzpicture}
\caption{مشق \حوالہ{مشق_مقناطیسی_مشق_مربوط_دور_الف} کا دور۔}
\label{شکل_مقناطیسی_مشق_مربوط_دور_الف}
\end{figure}

جوابات:\عددی{\hat{I}_1=8.9\phase{-79.7^{\circ}}\,\si{\ampere}}، \عددی{\hat{I}_2=4\phase{126.9^{\circ}}\,\si{\ampere}}، \عددی{\hat{V}_0=8\phase{36.9^{\circ}}\,\si{\volt}}
\انتہا{مشق}
%================
\ابتدا{مشق}\شناخت{مشق_مقناطیسی_مشق_مربوط_دور_ب}
شکل \حوالہ{شکل_مقناطیسی_مشق_مربوط_دور_ب} کے کرخوف مساوات لکھیں۔
\begin{figure}
\centering
\begin{tikzpicture}
\draw(0,0) to [american voltage source,l={$\hat{V}_1$}]++(0,\y) to [resistor,l={$R_1$}]++(\x,0) to [inductor,l={$j\omega L_1$}]++(\x,0)coordinate(kA) to [inductor,l={$j\omega L_2$}]++(\x,0) to [resistor,l={$R_2$}]++(\x,0);
\draw(0,0) to [short]++(4*\x,0) to [american voltage source,l_={$\hat{V}_2$}] ++(0,\y);
\draw(2*\x,0) to [capacitor,*-*,l={$\frac{1}{j\omega C_1}$}]++(0,\y);
%dots
\draw[fill](kA)++(-0.5,-0.25) circle (\kdot);
\draw[fill](kA)++(0.5,-0.25) circle (\kdot);
%mutual
\draw[-stealth](2*\x,\y+0.8)--++(-0.5,-0.5);
\draw[-stealth](2*\x,\y+0.8)--++(0.5,-0.5);
\draw(2*\x,\y+0.8)node[fill=white]{$j\omega M$};
%currents
\draw[stealth-]([shift={(-135:\x/5)}]\x,\y/2) arc (-135:135:\x/5);
\draw(\x,\y/2)node{$\hat{I}_1$};
\draw[stealth-]([shift={(-135:\x/5)}]3*\x,\y/2) arc (-135:135:\x/5);
\draw(3*\x,\y/2)node{$\hat{I}_2$};
\end{tikzpicture}
\caption{مشق \حوالہ{مشق_مقناطیسی_مشق_مربوط_دور_ب} کا دور۔}
\label{شکل_مقناطیسی_مشق_مربوط_دور_ب}
\end{figure}

جوابات:
\begin{align*}
\left(R_1+j\omega L_1+\frac{1}{j\omega C_1}\right)\hat{I}_1-\left(\frac{1}{j\omega C_1}+j\omega M \right)\hat{I}_2&=\hat{V}_1\\
-\left(\frac{1}{j\omega C_1}+j\omega M\right)\hat{I}_1+\left(\frac{1}{j\omega C_1}+j\omega L_2+R_2\right)\hat{I}_2&=-\hat{V}_2
\end{align*}
\انتہا{مشق}
%================
\ابتدا{مشق}\شناخت{مشق_مقناطیسی_مشق_مربوط_دور_پ}
شکل \حوالہ{شکل_مقناطیسی_مشق_مربوط_دور_پ} میں \عددی{\hat{I}_1} اور \عددی{\hat{I}_2} معلوم کرتے ہوئے \عددی{\hat{V}_0} دریافت کریں جہاں تیر والے لکیر سے ان نقطوں کی نشاندہی کی گئی ہے جن کے مابین دباو درکار ہے۔تیر والا سر مثبت دباو کے مقام کی نشاندہی کرتا ہے۔یوں \عددی{j8\,\si{\ohm}} امالہ کا نچلی سرا حوالہ لیتے ہوئے \عددی{-j6\,\si{\ohm}} برق گیر کے بائیں سر پر دباو حاصل کرنا درکار ہے۔
\begin{figure}
\centering
\begin{tikzpicture}
\draw(0,0) to [american voltage source,l={$20\phase{-30^{\circ}}\,\si{\volt}$}]++(0,\y) to [resistor,l={$\SI{2}{\ohm}$}]++(\x,0) to [inductor,l={$j2 \,\si{\ohm}$}]++(\x,0)coordinate(kA) to [inductor,l_={$j4\,\si{\ohm}$}]++(0,-\y) to [capacitor,l={$-j1\,\si{\ohm}$}]++(-2*\x,0);
\draw(2*\x+\x/3,0) to [american voltage source,l_={$40\phase{60^{\circ}}$}] ++(2*\x,0)coordinate(kN) to [inductor,l_={$j8 \,\si{\ohm}$}]++(0,\y) to [capacitor,l_={$-j6\,\si{\ohm}$}]++(-\x,0)coordinate(kP) to [resistor,l_={$\SI{4}{\ohm}$}]++(-\x,0)coordinate(kB) to [inductor,l={$j2\,\si{\ohm}$}]++(0,-\y);
%dots
\draw[fill](kA)++(-0.5,-0.5) circle (\kdot);
\draw[fill](kB)++(0.5,-0.5) circle (\kdot);
%mutual
\draw[-stealth](2*\x+\x/6,\y+0.8)--++(-0.5,-0.5);
\draw[-stealth](2*\x+\x/6,\y+0.8)--++(0.5,-0.5);
\draw(2*\x+\x/6,\y+0.88)node[fill=white]{$j 2.5 \,\si{\ohm}$};
%currents
\draw[stealth-]([shift={(-135:\x/5)}]\x,\y/2) arc (-135:135:\x/5);
\draw(\x,\y/2)node{$\hat{I}_1$};
\draw[stealth-]([shift={(-135:\x/5)}]3*\x,\y/2) arc (-135:135:\x/5);
\draw(3*\x,\y/2)node{$\hat{I}_2$};
%output voltage
\draw(kP) to [open,v={$\hat{V}_0$}](kN);
\end{tikzpicture}
\caption{مشق \حوالہ{مشق_مقناطیسی_مشق_مربوط_دور_پ} کا دور۔}
\label{شکل_مقناطیسی_مشق_مربوط_دور_پ}
\end{figure}

جوابات:\عددی{6.89\phase{252.3^{\circ}}\,\si{\ampere}}، \عددی{7.08\phase{219.9^{\circ}}\,\si{\si{\ampere}}}، \عددی{14.15\phase{-50.1^{\circ}}\,\si{\volt}}
\انتہا{مشق}
%=======================
\ابتدا{مشق}\شناخت{مشق_مقناطیسی_مشق_مربوط_دور_ت}
شکل \حوالہ{شکل_مقناطیسی_مشق_مربوط_دور_ت} میں  بائیں اور دائیں دائروں کی رو حاصل کرتے ہوئے \عددی{\hat{V}_0} دریافت کریں-دباو حاصل کرتے ہوئے دباو کا مشترک جزو شامل کرنا مت بھولیں۔
\begin{figure}
\centering
\begin{tikzpicture}
\draw(0,0) to [american voltage source,l={$60\phase{20^{\circ}}\,\si{\volt}$}]++(0,2*\y) to [inductor,l={$j4 \,\si{\ohm}$}]++(\x,0) to [resistor,l={$\SI{4}{\ohm}$}]++(\x,0) to [capacitor,l={$-j6 \, \si{\ohm}$}]++(\x,0) to [resistor,l={$\SI{2}{\ohm}$}]++(0,-\y) to [inductor,l={$j8\,\si{\ohm}$}]++(0,-\y)coordinate(kA) to [short] (0,0);
\draw(\x,0) to [capacitor,*-,l={$-j6 \,\si{\ohm}$}] ++(0,\y) to [inductor,-*,l={$j8\,\si{\ohm}$}]++(0,\y)coordinate(kB);
%dots
\draw[fill](kA)++(-0.5,0.5) circle (\kdot);
\draw[fill](kB)++(0.5,-0.5) circle (\kdot);
%mutual
\draw(kA)++(-0.5,\y/2)coordinate(kC);
\draw(kB)++(0.5,-\y/2)coordinate(kD);
\draw[stealth-stealth](kC)--(kD)node[pos=0.5,fill=white]{$j 3 \,\si{\ohm}$};
%output voltage
\draw(3*\x,2*\y) to [short,*-o]++(\x,0)coordinate(kLT);
\draw(3*\x,0) to [short,*-o]++(\x,0)coordinate(kLB);
\draw(kLB) to [open,v_>={$\hat{V}_0$}](kLT);
\end{tikzpicture}
\caption{مشق \حوالہ{مشق_مقناطیسی_مشق_مربوط_دور_ت} کا دور۔}
\label{شکل_مقناطیسی_مشق_مربوط_دور_ت}
\end{figure}

جوابات:\عددی{13.9\phase{-55.2^{\circ}}\,\si{\ampere}}، \عددی{5.97\phase{-24.2^{\circ}}\,\si{\ampere}}،\عددی{31.4\phase{83.55^{\circ}}\,\si{\volt}} 
\انتہا{مشق}
%=======================

\ابتدا{مشق}\شناخت{مشق_مقناطیسی_مشق_مربوط_دور_ٹ}
شکل \حوالہ{شکل_مقناطیسی_مشق_مربوط_دور_ٹ} میں   \عددی{\hat{V}_{ab}} دریافت کریں-دونوں امالہ کے دباو کے مشترک جزو شامل کرنا مت بھولیں۔
\begin{figure}
\centering
\begin{tikzpicture}
\draw(0,0) to [american voltage source,l={$10\phase{0^{\circ}}\,\si{\volt}$}]++(0,\y) to [resistor,l={$\SI{4}{\ohm}$}]++(\x,0)coordinate(kA) to [inductor,l={$j6\,\si{\ohm}$}]++(\x,0) coordinate(kB) to [inductor,l={$j4 \,\si{\ohm}$}]++(\x,0) coordinate(kC)to [capacitor,l={$-j2 \, \si{\ohm}$}]++(0,-\y)to [short] (0,0);
\draw(2*\x,0) to [resistor,*-*,l={$\SI{8}{\ohm}$}]++(0,\y);
\draw(kA)node[above]{$a$};
\draw(kC)node[above]{$b$};
%dots
\draw[fill](kA)++(0.5,0.5) circle (\kdot);
\draw[fill](kB)++(0.5,0.5) circle (\kdot);
%mutual
\draw[-stealth](2*\x,\y+0.8)--++(-0.5,-0.5);
\draw[-stealth](2*\x,\y+0.8)--++(0.5,-0.5);
\draw(2*\x,\y)++(0,0.9)node[fill=white]{$j2 \,\si{\ohm}$};
%output voltage
\end{tikzpicture}
\caption{مشق \حوالہ{مشق_مقناطیسی_مشق_مربوط_دور_ٹ} کا دور۔}
\label{شکل_مقناطیسی_مشق_مربوط_دور_ٹ}
\end{figure}

جواب:\عددی{10.5\phase{15^{\circ}}\,\si{\volt}}
\انتہا{مشق}
%=======================
\ابتدا{مثال}\شناخت{مثال_مقناطیسی_داخلی_رکاوٹ_مشترک_الف}
شکل \حوالہ{شکل_مقناطیسی_داخلی_رکاوٹ_مشترک_الف} میں منبع دباو کو نظر آنے والا داخلی رکاوٹ \عددی{\bZ_{\text{داخلی}}} دریافت کریں۔
\begin{figure}
\centering
\begin{tikzpicture}
\draw(0,0) to [american voltage source,l={$\hat{V}_m$}]++(0,\y) to [european resistor,i>^={$\hat{I}_1$},l={$\bZ_1$}]++(2*\x,0)coordinate(kA) to [inductor,l_={$j\omega L_1$}]++(0,-\y) to [short]++(-2*\x,0);
\draw(2*\x+\x/3,0) to [inductor,l_={$j\omega L_2$}]++(0,\y)coordinate(kB) to [short,i>^={$\hat{I}_2$}] ++(2*\x,0) to [european resistor,l={$\bZ_2$}]++(0,-\y) to [short] ++(-2*\x,0);
\draw[fill] (kA)++(-0.5,-0.5) circle (\kdot);
\draw[fill] (kB)++(0.5,-0.5) circle (\kdot);
\draw(2*\x+\x/6,\y+0.5) node{$j\omega M$};
%text
\draw[stealth-] (\x/2,\y/2)--++(-\x/8,0)--++(0,-\y/8)node[below]{$\bZ_{\text{داخلی}}$};
\end{tikzpicture}
\caption{مثال \حوالہ{مثال_مقناطیسی_داخلی_رکاوٹ_مشترک_الف} کا دور۔}
\label{شکل_مقناطیسی_داخلی_رکاوٹ_مشترک_الف}
\end{figure}

حل:رو \عددی{\hat{I}_1} دریافت کرتے ہوئے  رکاوٹ کو \عددی{\tfrac{\hat{V}_m}{\hat{I}_1}} سے حاصل کیا جائے گا۔دونوں دائروں کے کرخوف مساوات لکھتے ہیں۔
\begin{align*}
\hat{V}_m&=(\bZ_1+j\omega L_1)\hat{I}_1-j\omega M \hat{I}_2\\
0&=-j\omega M \hat{I}_1+(j\omega L_2+\bZ_2)\hat{I}_2
\end{align*}
دوسری مساوات سے \عددی{\hat{I}_2} حاصل کرتے ہوئے
\begin{align*}
\hat{I}_2=\frac{j\omega M}{j\omega L_2+\bZ_2} \hat{I}_1
\end{align*}
اس کو بائیں دائرے کی کرخوف مساوات میں پر کرتے ہیں
\begin{align*}
\hat{V}_m&=(\bZ_1+j\omega L_1)\hat{I}_1-j\omega M \frac{j\omega M}{j\omega L_2+\bZ_2} \hat{I}_1
\end{align*}
جہاں سے داخلی رکاوٹ درج ذیل لکھی جا سکتی ہے۔
\begin{align*}
\bZ_{\text{داخلی}}=\frac{\hat{V}_m}{\hat{I}_1}=\bZ_1+j\omega L_1+ \frac{\omega^2 M^2}{j\omega L_2+\bZ_2}
\end{align*}
\انتہا{مثال}
%=========================

\ابتدا{مشق}\شناخت{مشق_مقناطیسی_داخلی_رکاوٹ_مشترک_ب}
درج بالا مثال کے دور میں مشترکہ امالہ پر ایک نقطے کا مقام تبدیل کرتے ہوئے شکل \حوالہ{شکل_مقناطیسی_داخلی_رکاوٹ_مشترک_ب} حاصل کیا گیا ہے۔اس میں منبع دباو کو نظر آنے والا داخلی رکاوٹ \عددی{\bZ_{\text{داخلی}}} دریافت کریں۔
\begin{figure}
\centering
\begin{tikzpicture}
\draw(0,0) to [american voltage source,l={$\hat{V}_m$}]++(0,\y) to [european resistor,i>^={$\hat{I}_1$},l={$\bZ_1$}]++(2*\x,0) to [inductor,l_={$j\omega L_1$}]++(0,-\y)coordinate(kA) to [short]++(-2*\x,0);
\draw(2*\x+\x/3,0) to [inductor,l_={$j\omega L_2$}]++(0,\y)coordinate(kB) to [short,i>^={$\hat{I}_2$}] ++(2*\x,0) to [european resistor,l={$\bZ_2$}]++(0,-\y) to [short] ++(-2*\x,0);
\draw[fill] (kA)++(-0.5,0.5) circle (\kdot);
\draw[fill] (kB)++(0.5,-0.5) circle (\kdot);
\draw(2*\x+\x/6,\y+0.5) node{$j\omega M$};
%text
\draw[stealth-] (\x/2,\y/2)--++(-\x/8,0)--++(0,-\y/8)node[below]{$\bZ_{\text{داخلی}}$};
\end{tikzpicture}
\caption{مشق \حوالہ{مشق_مقناطیسی_داخلی_رکاوٹ_مشترک_ب} کا دور۔}
\label{شکل_مقناطیسی_داخلی_رکاوٹ_مشترک_ب}
\end{figure}

جواب:
\begin{align*}
\bZ_{\text{داخلی}}=\frac{\hat{V}_m}{\hat{I}_1}=\bZ_1+j\omega L_1+ \frac{\omega^2 M^2}{j\omega L_2+\bZ_2}
\end{align*}
آپ نے دیکھا کہ اس دور میں نقطے کا مقام تبدیل کرنے سے داخلی رکاوٹ تبدیل نہیں ہوتا۔
\انتہا{مشق}
%===========================
\ابتدا{مشق}
شکل \حوالہ{شکل_مقناطیسی_مشق_مربوط_دور_ٹ} میں منبع دباو کو کیا رکاوٹ نظر آتا ہے۔

جواب:\عددی{5.88+j11.53\,\si{\ohm}}
\انتہا{مشق}
%==================

\حصہ{مشترکہ امالہ میں توانائی کا ذخیرہ}
شکل \حوالہ{شکل_مقناطیسی_مشترک_امالہ_ذخیرہ-توانائی} کو دیکھیے۔رو مقناطیسی میدان پیدا کرتی ہے۔رو کی غیر موجودگی میں اس دور میں مقناطیسی بہاو نہیں پایا جائے گا۔یوں اس میں ذخیرہ مقناطیسی توانائی بھی صفر کے برابر ہو گی۔اب تصور کریں کہ دایاں لچھا کھلے سر رکھتے ہوئے بائیں لچھے کی رو \عددی{t_1} دورانیے میں \عددی{I_1} کر دی جاتی ہے۔اس دورانیے کے دوران بائیں لچھے کو درج ذیل توانائی فراہم کی جائے گی۔
\begin{figure}
\centering
\begin{circuitikz}
\draw(0,0) rectangle ++(-\boxW,\boxH);
\draw(0,0.25) to [short]++(\x,0)coordinate(BL) to [inductor,l={$L_1$}]++(0,\y)coordinate(TL) to [short,i<_={$i_1$}]++(-\x,0);
\draw(\x+\x/3+\x,0.25) to [short]++(-\x,0)coordinate(BR) to [inductor,l_={$L_2$}]++(0,\y)coordinate(TR) to [short,i<^={$i_2$}]++(\x,0);
\draw($(TL)!0.5!(TR)$)node[above]{$M$};
\draw[fill](TL)++(-0.5,-0.5) circle (\kdot); 
\draw[fill](TR)++(0.5,-0.5) circle (\kdot); 
\draw(2*\x+\x/3,0) rectangle ++(\boxW,\boxH);
\draw(0,\boxH/2) node[right]{$\begin{aligned} &+ \\ &v_1 \\ &-  \end{aligned}$};
\draw(2*\x+\x/3,\boxH/2) node[left]{$\begin{aligned} &+ \\ &v_2 \\ &-  \end{aligned}$};
\end{circuitikz}
\caption{مشترکہ امالہ میں ذخیرہ توانائی۔}
\label{شکل_مقناطیسی_مشترک_امالہ_ذخیرہ-توانائی}
\end{figure}
%
\begin{align*}
\int_{0}^{t_1}v_1(t) i_1(t)\dif t=\int_{0}^{t_1} \left[L_1 \frac{\dif i_1(t)}{\dif t}\right] i_1(t) \dif t=\int_{0}^{I_1} L_1 i_1 \dif i_1=\frac{L_1 I_1^2}{2}
\end{align*}
اس دوران دائیں لچھے کی رو صفر کے برابر ہے لہٰذا \عددی{t_1} کے دوران دائیں لچھے کو کوئی توانائی فراہم نہیں کی جاتی۔اب فرض کریں کہ بائیں لچھے کی رو اسی قیمت پر رکھی جاتی ہے جبکہ دائیں لچھے کی رو \عددی{t_1} تا \عددی{t_2} بڑھا کر \عددی{I_2} کر دی جاتی ہے۔چونکہ  \عددی{t_1} تا \عددی{t_2} بائیں لچھے کی رو تبدیل نہیں ہو رہی  ہے لہٰذا دائیں لچھے کے دباو میں مشترک جزو صفر کے برابر ہو گا۔یوں دائیں لچھے کا دباو \عددی{v_2=L_2\tfrac{\dif i_2}{\dif t}} لکھا جائے گا۔اس طرح دائیں لچھے کو درج ذیل توانائی فراہم کی جاتی ہے۔
\begin{align*}
\int_{t_1}^{t_2}v_2(t) i_2(t)\dif t=\int_{t_1}^{t_2} \left[L_2 \frac{\dif i_2(t)}{\dif t}\right] i_2(t) \dif t=\int_{0}^{I_2} L_2 i_2 \dif i_2=\frac{L_2 I_2^2}{2}
\end{align*}
اسی دورانیے (\عددی{t_1} تا \عددی{t_2})  میں چونکہ دائیں لچھے کی رو تبدیل ہو رہی ہے (جبکہ \عددی{i_1=I_1} مستقل ہے) لہٰذا بائیں لچھے کے دباو میں مشترک جزو پایا جائے گا اور یوں اس کا دباو درج ذیل لکھا جائے گا
\begin{align*}
v_1=L_1 \frac{\dif i_1}{\dif t}+M \frac{\dif i_2}{\dif t}=M\frac{\dif i_2}{\dif t}
\end{align*}
جہاں \عددی{i_1} مستقل ہونے کی وجہ سے \عددی{\tfrac{\dif i_1}{\dif t}=0} ہے۔یوں \عددی{t_1} تا \عددی{t_2} کے دوران بائیں لچھے کو درج ذیل توانائی مہیا کی جاتی ہے۔
\begin{align*}
\int_{t_1}^{t_2} v_1(t) i(t)\dif t=\int_{t_1}^{t_2} \left[M \frac{\dif i_2(t)}{\dif t}\right] I_1 \dif t=\int_{0}^{I_2} M I_1 \dif i_2=M I_1 I_2
\end{align*}
ان تینوں جوابات کا مجموعہ لمحہ \عددی{t_2} تک مشترکہ امالہ کو فراہم کی گئی توانائی دیتا ہے۔
\begin{align}\label{مساوات_مقناطیسی_توانائی_مشترکہ_امالہ_الف}
w=\frac{L_1 I_1^2}{2}+\frac{L_2 I_2^2}{2}+MI_1 I_2
\end{align}
اگر ایک لچھے پر نقطے کا مقام تبدیل کرتے ہوئے جواب حاصل کیا جائے تب درج ذیل جواب حاصل ہوتا ہے۔
\begin{align}\label{مساوات_مقناطیسی_توانائی_مشترکہ_امالہ_ب}
w=\frac{L_1 I_1^2}{2}+\frac{L_2 I_2^2}{2}-MI_1 I_2
\end{align}
آپ نے دیکھا کہ ذخیرہ توانائی کا دارومدار رو پر ہے نا کہ \عددی{t_1} اور \عددی{t_2} پر۔یوں کسی بھی لمحے لچھوں کی رو \عددی{i_1(t)} اور \عددی{i_2(t)} لکھتے ہوئے اس لمحے ذخیرہ توانائی کو درج ذیل لکھا جا سکتا ہے۔
\begin{align}\label{مساوات_مقناطیسی_توانائی_مشترکہ_امالہ_پ}
w(t)=\frac{L_1 i^2_1(t)}{2}+\frac{L_2 i^2_2(t)}{2}\mp M i_1(t) i_2(t)
\end{align}
چونکہ مشترکہ امالہ غیر عامل پرزہ ہے لہٰذا یہ توانائی پیدا نہیں کرتا۔یوں اس کی توانائی کبھی بھی منفی نہیں ہو سکتی۔یوں درج بالا مساوات میں غیر ضروری معلومات نہ لکھتے ہوئے درج ذیل لکھتے ہیں
\begin{align}
w(t)&=\frac{L_1 i^2_1}{2}+\frac{L_2 i^2_2}{2}\mp M i_1 i_2
\end{align}
جس میں \عددی{\tfrac{M^2 i^2_1}{2L_2}} جمع اور منفی کر کے ترتیب دیتے ہوئے درج ذیل لکھا جا سکتا ہے۔
\begin{align}
w=\frac{1}{2} \left(L_1-\frac{M^2}{L_2}\right)i^2_1+\frac{L_2}{2}\left(i_2+\frac{M}{L_2 i_1}\right)^2
\end{align}
درج بالا مساوات کا دوسرا جزو مربع ہے لہٰذا یہ ہر صورت مثبت ہو گا۔چونکہ غیر عامل مشترکہ امالہ کی توانائی مثبت ہے لہٰذا اس مساوات کا پہلا جزو بھی مثبت ہو گا جس سے درج ذیل شرط حاصل ہوتا ہے۔
\begin{align}
M\le \sqrt{L_1 L_2}
\end{align}
یہ مساوات مشترکہ امالہ کی زیادہ سے زیادہ قیمت کا حد بیان کرتا ہے۔  یوں مشترکہ امالہ صفر تا \عددی{\sqrt{L_1 L_2}} ممکن ہے۔
\begin{align}
0\le M \le \sqrt{L_1 L_2}
\end{align}
 کسی بھی مشترکہ امالہ کو درج ذیل لکھا جا سکتا ہے
\begin{align}\label{مساوات_مقناطیسی_مشترک_امالہ_تعریف_الف}
M=k\sqrt{L_1 L_2}
\end{align}
جہاں \عددی{k} کو \اصطلاح{ارتباطی مستقل}\فرہنگ{ارتباطی مستقل}\حاشیہب{coupling coefficient}\فرہنگ{coupling coefficient} کہتے ہیں۔آپ دیکھ سکتے ہیں کہ ارتباطی مستقل صفر تا اکائی ممکن ہے۔
\begin{align}
0 \le k \le 1
\end{align}
ارتباطی مستقل کی تعریف درج ذیل مساوات دیتی ہے۔
\begin{align}
k=\frac{M}{\sqrt{L_1 L_2}}
\end{align}
ارتباطی مستقل یہ بتلاتا ہے کہ ایک لچھے کی کتنی بہاو دوسرے لچھے کے اندر سے گزرتی ہے۔اس باب کے شروع میں مشترکہ امالہ کے اشکال بناتے ہوئے ہم نے مقناطیسی قالب استعمال کیا۔مقناطیسی قالب کے استعمال سے ایک لچھے کی تقریباً تمام بہاو دوسرے لچھے سے بھی گزاری جا سکتی ہے۔ایسی صورت میں \عددی{k \approx 1} ہو گا۔اس کے برعکس ایک دونوں سے دور، قالب سے نہ جوڑے گئے لچھوں کی صورت میں \عددی{k=0} ہو گا چونکہ ایک لچھے کا بہاو دوسرے لچھے تک نہیں پہنچ پائے گا۔ارتباطی مستقل کی قیمت زیادہ \عددی{(k \ge 0.5)} ہونے کی صورت میں ہم کہتے ہیں کہ لچھوں کا \اصطلاح{رابطہ مضبوط}\فرہنگ{رابطہ!مضبوط}\حاشیہب{strongly coupled}\فرہنگ{coupling!strong} ہے جبکہ \عددی{k < 0.5} کی صورت میں ہم کہتے ہیں کہ لچھوں کا \اصطلاح{رابطہ کمزور}\فرہنگ{رابطہ!کمزور}\حاشیہب{weakly coupled}\فرہنگ{coupling!weak} ہے۔    
%========================
\ابتدا{مثال}\شناخت{مثال_مقناطیسی_ارتباطی_مستقل_الف}
شکل \حوالہ{شکل_مقناطیسی_ارتباطی_مستقل_الف}-الف میں \عددی{L_1=\SI{25}{\milli\henry}}، \عددی{L_2=\SI{12}{\milli\henry}} اور \عددی{k=1} ہیں۔لمحہ \عددی{t=\SI{6.2}{\milli\second}} پر مشترکہ امالہ میں ذخیرہ توانائی دریافت کریں۔
\begin{figure}
\centering
\begin{subfigure}{1\textwidth}
\centering
\begin{tikzpicture}
\draw(0,0) to [american voltage source,l={$20\cos 314t \, \si{\volt}$}]++(0,\y) to [resistor,l={$\SI{4}{\ohm}$}]++(2*\x,0)coordinate(kA) to [inductor,l_={$L_1$}]++(0,-\y) to [short] ++(-2*\x,0);
\draw(2*\x+\x/3,0) to [inductor,l_={$L_2$}]++(0,\y)coordinate(kB) to [short]++(2*\x,0) to [resistor,l={$\SI{2}{\ohm}$}]++(0,-\y) to [short]++(-2*\x,0);
%dots
\draw[fill](kA)++(-0.5,-0.5) circle (\kdot);
\draw[fill](kB)++(0.5,-0.5) circle (\kdot);
\draw(2*\x+\x/6,\y)node[above]{$M$};
%currents
\draw[stealth-]([shift={(-135:\x/4)}]\x,\y/2) arc (-135:135:\x/4);
\draw(\x,\y/2) node{$i_1(t)$};
\draw[stealth-]([shift={(-135:\x/4)}]2*\x+\x/3+\x,\y/2) arc (-135:135:\x/4);
\draw(2*\x+\x/3+\x,\y/2) node{$i_2(t)$};
\end{tikzpicture}
\caption*{(الف)}
\end{subfigure}
\begin{subfigure}{1\textwidth}
\centering
\begin{tikzpicture}
\draw(0,0) to [american voltage source,l={${20\phase{0^{\circ}}\, \si{\volt}}$}]++(0,\y) to [resistor,l={$\SI{4}{\ohm}$}]++(2*\x,0)coordinate(kA) to [inductor,l_={$j7.85\,\si{\ohm}$}]++(0,-\y) to [short] ++(-2*\x,0);
\draw(2*\x+\x/3,0) to [inductor,l_={$j3.768\,\si{\ohm}$}]++(0,\y)coordinate(kB) to [short]++(2*\x,0) to [resistor,l={$\SI{2}{\ohm}$}]++(0,-\y) to [short]++(-2*\x,0);
%dots
\draw[fill](kA)++(-0.5,-0.5) circle (\kdot);
\draw[fill](kB)++(0.5,-0.5) circle (\kdot);
\draw(2*\x+\x/6,\y)node[above]{$j5.439\,\si{\ohm}$};
%currents
\draw[stealth-]([shift={(-135:\x/5)}]\x,\y/2) arc (-135:135:\x/5);
\draw(\x,\y/2) node{$\hat{I}_1$};
\draw[stealth-]([shift={(-135:\x/5)}]2*\x+\x/3+\x,\y/2) arc (-135:135:\x/5);
\draw(2*\x+\x/3+\x,\y/2) node{$\hat{I}_2$};
\end{tikzpicture}
\caption*{(ب)}
\end{subfigure}
\caption{مثال \حوالہ{مثال_مقناطیسی_ارتباطی_مستقل_الف} کا دور۔}
\label{شکل_مقناطیسی_ارتباطی_مستقل_الف}
\end{figure}

حل:منبع دباو سے تعدد \عددی{\omega=\SI{314}{\radian\per\second}} اور مساوات \حوالہ{مساوات_مقناطیسی_مشترک_امالہ_تعریف_الف} سے مشترکہ امالہ
\begin{align*}
M&=k\sqrt{L_1 L_2}=1\sqrt{(0.025)(0.012)}=\SI{17.321}{\milli\henry}
\end{align*}
لیتے ہوئے شکل-ب میں تعددی دائرہ کار میں دور کو دوبارہ دکھایا گیا ہے جہاں درج ذیل قیمتیں استعمال کی گئی ہیں۔
\begin{align*}
j\omega L_1&=j (314)(0.025)=j7.85\,\si{\ohm}\\
j\omega L_2&=j(314)(0.012)=j3.768\,\si{\ohm}\\
j\omega M&=j (314)(0.017321)=j5.439\,\si{\ohm}
\end{align*}
دونوں دائروں کے کرخوف مساوات لکھتے ہیں۔
\begin{align*}
20\phase{0^{\circ}}&=(4+j7.85)\hat{I}_1-j5.439\hat{I}_2\\
0&=-j5.439\hat{I}_1+(2+j3.768)\hat{I}_2
\end{align*}
ان میں سے دوسری مساوات سے \عددی{\hat{I}_2} لیتے ہوئے پہلی میں پر کرتے
\begin{align*}
20\phase{0^{\circ}}&=(4+j7.85)\hat{I}_1-j5.439\left(\frac{j5.439 }{2+j3.768}\right)\hat{I}_1
\end{align*}
ہوئے \عددی{\hat{I}_1} حاصل کرتے ہیں۔
\begin{align*}
\hat{I}_1=\frac{20}{7.251+j1.725}=2.610-j0.621=2.683\phase{-13.38^{\circ}}\,\si{\ampere}
\end{align*}
اسی طرح \عددی{\hat{I}_2} درج ذیل حاصل ہوتا ہے۔
\begin{align*}
\hat{I}_2=\left(\frac{j5.439}{2+j3.768}\right) \hat{I}_1=3.421\phase{14.57^{\circ}}\,\si{\ampere}
\end{align*}
حاصل شدہ رو کو وقتی دائرہ کار میں لکھتے ہیں۔
\begin{align*}
i_1(t)&=2.683\cos(314t-13.38^{\circ}) \, \si{\ampere}\\
i_2(t)&=3.421\cos(314t+14.57^{\circ})\,\si{\ampere}
\end{align*}
لمحہ \عددی{t=\SI{6.2}{\milli\second}} پر رو کی قیمتیں حاصل کرتے ہیں۔ایسا کرتے ہوئے زاویہ ہٹاو کو ریڈیئن میں لکھا جائے گا۔
\begin{align*}
i_1(t=\SI{6.2}{\milli\second})&=I_1=2.683\cos\left[(314)(0.0062)-13.38\left(\frac{\pi}{180}\right)\right]=\SI{2.487}{\ampere}\\
i_2(t=\SI{6.2}{\milli\second})&=I_2=3.421\cos\left[(314)(0.062)+14.57\left(\frac{\pi}{180}\right)\right]=\SI{2.199}{\ampere}
\end{align*}
لمحہ \عددی{\SI{6.2}{\milli\second}} پر رو کی قیمتیں جاننے کے بعد مساوات \حوالہ{مساوات_مقناطیسی_توانائی_مشترکہ_امالہ_ب} سے ذخیرہ توانائی حاصل کرتے ہیں۔
\begin{align*}
w(t=\SI{6.2}{\milli\second})&=\frac{L_1 I_1^2}{2}+\frac{L_2 I_2^2}{2}+M I_1 I_2\\
&=\frac{(0.025)(2.487)^2}{2}+\frac{(0.012)(2.199)^2}{2}+0.0173(2.487)(2.199)\\
&=\SI{0.201}{\joule}
\end{align*}
\انتہا{مثال}
%=========================
\ابتدا{مشق}\شناخت{مشق_مقناطیسی_ارتباطی_مستقل_الف}
شکل \حوالہ{شکل_مقناطیسی_مشق_ارتباطی_مستقل_الف} میں تعدد \عددی{\SI{50}{\hertz}} اور \عددی{k=0.6} ہیں۔لمحہ \عددی{t=\SI{5.5}{\milli\second}} پر مشترکہ امالہ میں ذخیرہ توانائی دریافت کریں۔
\begin{figure}
\centering
\begin{tikzpicture}
\draw(0,0) to [american voltage source,l={${36\phase{45^{\circ}}\, \si{\volt}}$}]++(0,\y) to [capacitor,l={$-j44\,\si{\ohm}$}]++(2*\x,0)coordinate(kA) to [inductor,l_={$j20\,\si{\ohm}$}]++(0,-\y) to [short] ++(-2*\x,0);
\draw(2*\x+\x/3,0) to [inductor,l_={$j15\,\si{\ohm}$}]++(0,\y)coordinate(kB) to [capacitor,l={$-j10\,\si{\ohm}$}]++(2*\x,0) to [resistor,l={$\SI{10}{\ohm}$}]++(0,-\y) to [short]++(-2*\x,0);
%dots
\draw[fill](kA)++(-0.5,-0.5) circle (\kdot);
\draw[fill](kB)++(0.5,-0.5) circle (\kdot);
\draw(2*\x+\x/6,\y)node[above]{$M$};
\end{tikzpicture}
\caption{مشق \حوالہ{مشق_مقناطیسی_ارتباطی_مستقل_الف} کا دور۔}
\label{شکل_مقناطیسی_مشق_ارتباطی_مستقل_الف}
\end{figure}

جواب:\عددی{\SI{24.4}{\milli\joule}}
\انتہا{مشق}
%====================

\حصہ{کامل ٹرانسفارمر}
شکل \حوالہ{شکل_مقناطیسی_کامل_ٹرانسفارمر} کو دیکھیے جہاں دو لچھوں کو مقناطیسی قالب پر لپیٹا گیا ہے۔یہ روزمرہ میں استعمال ہونے والا ٹرانسفارمر ہے۔ قالب میں \عددی{\phi} مقناطیسی بہاو پائی جاتی ہے۔یوں دونوں لچھوں سے یکساں بہاو گزرتی ہے۔یہ بہاو لچھوں میں درج ذیل دباو پیدا کرتی ہے۔
\begin{align}
v_1(t)&=N_1\frac{\dif \phi}{\dif t} \label{مساوات_مقناطیسی_ٹرانسفارمر_الف}\\
v_2(t)&=N_2\frac{\dif \phi}{\dif t}\label{مساوات_مقناطیسی_ٹرانسفارمر_ب}
\end{align}
مساوات \حوالہ{مساوات_مقناطیسی_ٹرانسفارمر_الف} کو مساوات \حوالہ{مساوات_مقناطیسی_ٹرانسفارمر_ب} سے تقسیم کرتے  ہیں۔
\begin{align}\label{مساوات_مقناطیسی_ٹرانسفارمر_پ}
\frac{v_1(t)}{v_2(t)}=\frac{N_1\frac{\dif \phi}{\dif t}}{N_2\frac{\dif \phi}{\dif t}}=\frac{N_1}{N_2}
\end{align}
قانون ایمپیئر  کے تحت قالب کے گرد درج ذیل لکھا جا سکتا ہے
\begin{align*}
\oint H \cdot \dif l=i_{\text{گھیرا}}=N_1 i_1+N_2 i_2
\end{align*} 
جہاں تکمل کو قالب کے اندر گھومتے ہوئے حاصل کیا جاتا ہے جبکہ \عددی{H} قالب کے اندر \اصطلاح{مقناطیسی شدت}\فرہنگ{مقناطیسی!شدت}\حاشیہب{magnetic field intensity}\فرہنگ{magnetic field!intensity} ہے۔مقناطیسی قالب میں \عددی{H} کی قیمت قابل نظر انداز ہوتی ہے۔یوں \عددی{H} کا تکمل بھی قابل نظر انداز ہوتا ہے۔درج بالا مساوات میں تکمل کو صفر کے برابر پر کرنے سے درج ذیل 
\begin{align}\label{مساوات_مقناطیسی_ٹرانسفارمر_ت}
N_1 i_1 +N_2 i_2=0
\end{align}
یعنی
\begin{align}\label{مساوات_مقناطیسی_ٹرانسفارمر_ٹ}
\frac{i_1}{i_2}=-\frac{N_2}{N_1}
\end{align}
حاصل ہوتا ہے۔مساوات \حوالہ{مساوات_مقناطیسی_ٹرانسفارمر_ت} کو \عددی{\tfrac{v_1}{N_1}} سے ضرب دینے سے
\begin{align}
v_1 i_1 +\frac{N_2}{N_1}v_1 i_2=0
\end{align}
ملتا ہے  جس میں مساوات \حوالہ{مساوات_مقناطیسی_ٹرانسفارمر_پ} سے \عددی{\tfrac{N_2}{N_1}v_1=v_2} پر کرتے ہوئے  درج ذیل حاصل ہوتا ہے۔ 
\begin{align}
v_1 i_1+v_2 i_2=0
\end{align}
یہ مساوات کہتا ہے کہ کامل ٹرانسفارمر کو کل صفر طاقت درکار ہے یعنی کامل ٹرانسفارمر بے ضیاع پرزہ ہے۔
%  
\begin{figure}
\centering
\begin{tikzpicture}[american voltages]
\def\height{3};
\def\width{1.5};
\def\thick{0.4};
\def\depthX{0.2};
\def\depthY{0.2};
\def\p{0.2};      %pitch
\def\cTop{2.4}; %top of coil
\def\TL{7};    %number of turns
\def\cTopR{2.3}; %top of right coil
\def\TR{6};    %number of right turns
%flux
\draw[gray,-stealth](\thick/2,\height-\thick) to [out=90,in=180]++(\thick/2,\thick/2) to [short]++(\width-2*\thick,0) to [out=0,in=90]++(\thick/2,-\thick/2);
\draw(\width/2,\height-\thick/2)node[fill=white]{$\phi$};
%core
\draw(0,0)--++(0,\height)--++(\width,0)--++(0,-\height)--cycle;
\draw(0,0)++(\thick,\thick)--++(0,\height-2*\thick)--++(\width-2*\thick,0)--++(0,-\height+2*\thick)--cycle;
%
\draw(\thick,\thick)--++(\depthX,\depthY) --++(0,\height-2*\thick-\depthY);
\draw(\thick,\thick)--++(\depthX,\depthY) --++(\width-2*\thick-\depthX,0);
\draw(0,\height)--++(\depthX,\depthY)--++(\width,0)--++(-\depthX,-\depthY);
\draw(\width,0)--++(\depthX,\depthY)--++(0,\height)--++(-\depthX,-\depthY);
%left winding
\draw (\thick+\depthX,\cTop) to [out=45,in=0] ++(-\thick/2-\depthX,\p/2) to [short]++(-\thick/2,0) to [short] ++(-\x/4,0)coordinate(kTop);
\foreach \l in {0,1,2,...,\TL}{
\draw (0,\cTop-\l*\p) to [out=-135,in=45] ++(\thick+\depthX,-\p);
}
\draw(0,\cTop-\TL*\p-\p) to [short]++ (-\x/4,0)coordinate(kBot);
%right winding
\draw (\width-\thick,\cTopR) to [out=135,in=180] ++(\thick/2,\p/2) to [short]++(\thick/2+\depthX,0) to [short,i<={$i_2(t)$}] ++(\x,0)coordinate(kTopR);
\foreach \l in {0,1,2,...,\TR}{
\draw (\width+\depthX,\cTopR-\l*\p) to [out=-45,in=135] ++(-\thick-\depthX,-\p);
}
\draw(\width+\depthX,\cTopR-\TR*\p-\p) to [short]++ (\x,0)coordinate(kBotR);
%current
\draw(kBot) to [short]++(-3/4*\x,0);
\draw(kTop) to [short,i<_={$i_1(t)$}]++(-3/4*\x,0);
%box ckt
\draw(kBot)++(-3/4*\x,-0.25)coordinate(ka) rectangle ++(-0.5,2.25);
\draw(kBotR)++(0,-0.25)coordinate(kb) rectangle ++(0.5,2.25);
%text
\draw(0,\height/2) node [left]{$N_1$};
\draw(\width+\depthX,\height/2) node [right]{$N_2$};
\draw(ka)++(0,1) node[right]{$\begin{aligned} &+ \\ &v_1(t) \\ &- \end{aligned}$};
\draw(kb)++(0,1) node[left]{$\begin{aligned} &+ \\ &v_2(t) \\ &- \end{aligned}$};
\end{tikzpicture}
\caption{ٹرانسفارمر کی ساخت۔}
\label{شکل_مقناطیسی_کامل_ٹرانسفارمر}
\end{figure}

شکل \حوالہ{شکل_مقناطیسی_کامل_ٹرانسفارمر} میں \عددی{i_2} کی سمت الٹ کرنے سے شکل \حوالہ{شکل_مقناطیسی_کامل_ٹرانسفارمر_ب} حاصل ہوتا ہے۔اس کے مساوات درج ذیل  ہیں جہاں دونوں اطراف کے رو کی تناسب میں منفی کی علامت نہیں پائی جاتی۔  
\begin{align}
\frac{v_1(t)}{v_2(t)}&=\frac{N_1}{N_2}\\
\frac{i_1(t)}{i_2(t)}&=\frac{N_2}{N_1}
\end{align}
ٹرانسفارمر کو شکل \حوالہ{شکل_مقناطیسی_کامل_ٹرانسفارمر} سے ظاہر کیا جاتا ہے اور درج بالا دو عدد مساوات ٹرانسفارمر کے بنیادی مساوات لکھنے کا عمومی طریقہ ہے۔ 

\begin{figure}
\centering
\begin{tikzpicture}[american voltages]
\def\height{3};
\def\width{1.5};
\def\thick{0.4};
\def\depthX{0.2};
\def\depthY{0.2};
\def\p{0.2};      %pitch
\def\cTop{2.4}; %top of coil
\def\TL{7};    %number of turns
\def\cTopR{2.3}; %top of right coil
\def\TR{6};    %number of right turns
%flux
\draw[gray,-stealth](\thick/2,\height-\thick) to [out=90,in=180]++(\thick/2,\thick/2) to [short]++(\width-2*\thick,0) to [out=0,in=90]++(\thick/2,-\thick/2);
\draw(\width/2,\height-\thick/2)node[fill=white]{$\phi$};
%core
\draw(0,0)--++(0,\height)--++(\width,0)--++(0,-\height)--cycle;
\draw(0,0)++(\thick,\thick)--++(0,\height-2*\thick)--++(\width-2*\thick,0)--++(0,-\height+2*\thick)--cycle;
%
\draw(\thick,\thick)--++(\depthX,\depthY) --++(0,\height-2*\thick-\depthY);
\draw(\thick,\thick)--++(\depthX,\depthY) --++(\width-2*\thick-\depthX,0);
\draw(0,\height)--++(\depthX,\depthY)--++(\width,0)--++(-\depthX,-\depthY);
\draw(\width,0)--++(\depthX,\depthY)--++(0,\height)--++(-\depthX,-\depthY);
%left winding
\draw (\thick+\depthX,\cTop) to [out=45,in=0] ++(-\thick/2-\depthX,\p/2) to [short]++(-\thick/2,0) to [short] ++(-\x/4,0)coordinate(kTop);
\foreach \l in {0,1,2,...,\TL}{
\draw (0,\cTop-\l*\p) to [out=-135,in=45] ++(\thick+\depthX,-\p);
}
\draw(0,\cTop-\TL*\p-\p) to [short]++ (-\x/4,0)coordinate(kBot);
%right winding
\draw (\width-\thick,\cTopR) to [out=135,in=180] ++(\thick/2,\p/2) to [short]++(\thick/2+\depthX,0) to [short,i={$i_2(t)$}] ++(\x,0)coordinate(kTopR);
\foreach \l in {0,1,2,...,\TR}{
\draw (\width+\depthX,\cTopR-\l*\p) to [out=-45,in=135] ++(-\thick-\depthX,-\p);
}
\draw(\width+\depthX,\cTopR-\TR*\p-\p) to [short]++ (\x,0)coordinate(kBotR);
%current
\draw(kBot) to [short]++(-3/4*\x,0);
\draw(kTop) to [short,i<_={$i_1(t)$}]++(-3/4*\x,0);
%box ckt
\draw(kBot)++(-3/4*\x,-0.25)coordinate(ka) rectangle ++(-0.5,2.25);
\draw(kBotR)++(0,-0.25)coordinate(kb) rectangle ++(0.5,2.25);
%text
\draw(0,\height/2) node [left]{$N_1$};
\draw(\width+\depthX,\height/2) node [right]{$N_2$};
\draw(ka)++(0,1) node[right]{$\begin{aligned} &+ \\ &v_1(t) \\ &- \end{aligned}$};
\draw(kb)++(0,1) node[left]{$\begin{aligned} &+ \\ &v_2(t) \\ &- \end{aligned}$};
\end{tikzpicture}
\caption{کامل ٹرانسفارمر کے دباو اور رو۔}
\label{شکل_مقناطیسی_کامل_ٹرانسفارمر_ب}
\end{figure}

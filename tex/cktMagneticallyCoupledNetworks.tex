\باب{مقناطیسی جڑے ادوار}

\حصہ{مشترکہ امالہ}
شکل \حوالہ{شکل_مقناطیسی_خود_امالہ}-الف میں \عددی{N} چکر کا \اصطلاح{لچھا}\فرہنگ{لچھا}\حاشیہب{coil}\فرہنگ{coil} مقناطیسی مادے سے بنائے گئے \اصطلاح{قالب}\فرہنگ{قالب!مقناطیسی}\حاشیہب{core}\فرہنگ{core!magnetic} پر لپیٹا گیا  دکھایا گیا ہے۔اس لچھے میں \عددی{i} رو گزر رہی ہے۔ایمپیئر کے قانون کے تحت رو کے گزرنے سے مقناطیسی میدان پیدا ہوتا ہے۔یوں رو کے گزرنے سے لچھے میں \عددی{\phi} \اصطلاح{مقناطیسی بہاو}\فرہنگ{مقناطیسی بہاو}\فرہنگ{بہاو!مقناطیسی}\حاشیہب{magnetic flux}\فرہنگ{magnetic flux}\فرہنگ{flux!magnetic} پیدا ہوتا ہے جسے ہلکی سیاہی میں نقطہ دار لکیر سے دکھایا گیا ہے۔

لچھے میں رو کی سمت اور مقناطیسی بہاو کی سمت کے تعلق پر غور کریں۔ان کا تعلق دائیں ہاتھ کا قانون کہلاتا ہے۔دائیں ہاتھ کا قانون درج ذیل ہے۔

\ابتدا{قانون}
اگر لچھے کو دائیں ہاتھ سے یوں پکڑا جائے کہ ہاتھ کی چار انگلیاں رو کی سمت میں لپیٹے جائیں تب اسی ہاتھ کا انگوٹھا بہاو کی سمت دے گا۔
\انتہا{قانون}
مقناطیسی بہاو کو کسی مخصوص خطے میں رکھنے کی خاطر مقناطیسی قالب استعمال کیا جاتا ہے۔مقناطیسی بہاو کے لئے مقناطیسی مادے سے گزرنا زیادہ آسان ثابت ہوتا ہے لہٰذا شکل \حوالہ{شکل_مقناطیسی_خود_امالہ}-الف میں بہاو قالب کے اندر ہی رہتے ہوئے گھڑی کے سوئیوں کے گھومنے  کی سمت میں گھومتا ہے۔یوں مقناطیسی بہاو \عددی{\phi} لچھے کے تمام چکروں کے اندر سے گزرتا ہے۔لچھے کا \اصطلاح{ارتباط بہاو}\فرہنگ{ارتباط بہاو}\فرہنگ{بہاو!ارتباط}\حاشیہب{flux linkage}\فرہنگ{flux linkage} \عددی{\lambda} درج ذیل ہے۔
\begin{align}\label{مساوات_مشترک_ارتباط_بہاو_الف}
\lambda=N \phi
\end{align}
اس کتاب میں صرف خطی نظام پر غور کیا گیا ہے۔خطی صورت میں ارتباط بہاو اور رو کا تعلق درج ذیل ہے
\begin{align}\label{مساوات_مشترک_ارتباط_بہاو_ب}
\lambda=L i
\end{align}
جہاں مساوات کے مستقل \عددی{L} کو \اصطلاح{خود امالہ}\فرہنگ{امالہ!خود}\فرہنگ{خود امالہ}\حاشیہب{self inductance}\فرہنگ{inductance!self}\فرہنگ{self inductance} یا \اصطلاح{امالہ} کہتے ہیں۔باب \حوالہ{باب_برق_گیر_امالہ_گیر} میں امالہ پر غور کیا گیا ہے۔درج بالا دو مساوات کو ملاتے ہوئے  بہاو اور رو کا تعلق ملتا ہے۔
\begin{align}
\phi=\frac{Li}{N}
\end{align}
قانون فیراڈے کے تحت بدلتی ارتباط بہاو لچھے میں امالی دباو پیدا کرتا ہے۔
\begin{align}
v=\frac{\dif \lambda}{\dif t}
\end{align}
مساوات \حوالہ{مساوات_مشترک_ارتباط_بہاو_ب} کو درج بالا مساوات میں پر کرتے ہیں۔
\begin{align*}
v=\frac{\dif \lambda}{\dif t}=\frac{\dif (Li)}{\dif t}=L\frac{\dif i}{\dif t}+i\frac{\dif L}{\dif t}
\end{align*}
مستقل امالہ کی صورت میں اس مساوات سے امالہ کی جانی پہچانی درج ذیل مساوات حاصل ہوتی ہے۔
\begin{align}\label{مساوات_مقناطیسی_امالہ_کی_مساوات}
v=L\frac{\dif i}{\dif t}
\end{align}
اس کتاب میں مستقل امالہ پر ہی غور کیا جائے گا۔شکل \حوالہ{شکل_مقناطیسی_خود_امالہ}-ب میں اس امالہ کو دکھایا گیا ہے۔یہاں غور کریں کہ مزاحمت کی طرح امالہ کے دباو اور رو بھی انفعالی رائج سمت کے تحت ہیں۔یوں امالہ میں رو مثبت دباو والے سر سے داخلی ہوتی ہے۔مساوات \حوالہ{مساوات_مقناطیسی_امالہ_کی_مساوات} کہتا ہے کہ  بدلتی رو کے گزرنے سے امالہ میں دباو پیدا ہوتا ہے۔
%
\begin{figure}
\centering
\begin{subfigure}{0.6\textwidth}
\centering
\begin{tikzpicture}[american voltages]
\def\height{3};
\def\width{1.5};
\def\thick{0.4};
\def\depthX{0.2};
\def\depthY{0.2};
\def\gap{0.05};
\def\p{0.2};      %pitch
\def\cTop{2.4}; %top of coil
\def\TL{7};    %number of turns
%flux
\draw[gray,dashed,-stealth](\thick/2,\height-\thick) to [out=90,in=180]++(\thick/2,\thick/2) to [short]++(\width-2*\thick,0) to [out=0,in=90]++(\thick/2,-\thick/2) to [short]++(0,-\height+2*\thick) to [out=-90,in=0]++(-\thick/2,-\thick/2) to [short]++(-\width+2*\thick,0) to [out=180,in=-90]++(-\thick/2,\thick/2) to [short]++(0,0.2);
\draw[gray,-stealth](\width-\thick/2,\height/2)++(0,0.05)--++(0,-0.1);
\draw(\width/2,\height-\thick/2)node[fill=white]{$\phi$};
%core
\draw(0,0)--++(0,\height)--++(\width,0)--++(0,-\height)--cycle;
\draw(0,0)++(\thick,\thick)--++(0,\height-2*\thick)--++(\width-2*\thick,0)--++(0,-\height+2*\thick)--cycle;
%
\draw(\thick,\thick)--++(\depthX,\depthY) --++(0,\height-2*\thick-\depthY);
\draw(\thick,\thick)--++(\depthX,\depthY) --++(\width-2*\thick-\depthX,0);
\draw(0,\height)--++(\depthX,\depthY)--++(\width,0)--++(-\depthX,-\depthY);
\draw(\width,0)--++(\depthX,\depthY)--++(0,\height)--++(-\depthX,-\depthY);
%left winding
\draw (\thick+\depthX,\cTop) to [out=45,in=0] ++(-\thick/2-\depthX,\p/2) to [short]++(-\thick/2,0) to [short] ++(-\x/4,0)coordinate(kTop);
\foreach \l in {0,1,2,...,\TL}{
\draw (0,\cTop-\l*\p) to [out=-135,in=45] ++(\thick+\depthX,-\p);
}
\draw(0,\cTop-\TL*\p-\p) to [short]++ (-\x/4,0)coordinate(kBot);
%current
\draw(kBot) to [short]++(-\x,0) to [american current source,l={$i$}]++(0,1.7)coordinate(currT)|- (kTop);
\draw(currT)++(1,-0.85) node{$\begin{aligned}&+ \\ &v \\ &-   \end{aligned}$};
%text
\draw(0,\height/2)node[left]{$N$};
\draw[stealth-](\width+\depthX,2/3*\height) to [out=45,in=180]++(0.5,0.5)node[right]{\RL{مقناطیسی مادہ}};
\end{tikzpicture}
\caption*{(الف)}
\end{subfigure}%
\begin{subfigure}{0.4\textwidth}
\centering
\begin{circuitikz}[american voltages]
\draw(0,0) to [short,i={$i$},o-]++(\x,0) to [inductor,l={$L$}]++(0,-\y) to [short,-o] ++(-\x,0);
\draw ($(0,0)!0.5!(0,-\y)$)node{$\begin{aligned} &+ \\ &v \\ &-  \end{aligned}$};
\end{circuitikz}
\caption*{(ب)}
\end{subfigure}
\caption{خود امالہ کی تعریف۔}
\label{شکل_مقناطیسی_خود_امالہ}
\end{figure}
%
\begin{figure}
\centering
\begin{tikzpicture}[american voltages]
\def\height{3};
\def\width{1.5};
\def\thick{0.4};
\def\depthX{0.2};
\def\depthY{0.2};
\def\p{0.2};      %pitch
\def\cTop{2.4}; %top of coil
\def\TL{7};    %number of turns
\def\cTopR{2.3}; %top of right coil
\def\TR{6};    %number of right turns
%flux
\draw[gray,-stealth](\thick/2,\height-\thick) to [out=90,in=180]++(\thick/2,\thick/2) to [short]++(\width-2*\thick,0) to [out=0,in=90]++(\thick/2,-\thick/2);
\draw(\width/2,\height-\thick/2)node[fill=white]{$\phi$};
%core
\draw(0,0)--++(0,\height)--++(\width,0)--++(0,-\height)--cycle;
\draw(0,0)++(\thick,\thick)--++(0,\height-2*\thick)--++(\width-2*\thick,0)--++(0,-\height+2*\thick)--cycle;
%
\draw(\thick,\thick)--++(\depthX,\depthY) --++(0,\height-2*\thick-\depthY);
\draw(\thick,\thick)--++(\depthX,\depthY) --++(\width-2*\thick-\depthX,0);
\draw(0,\height)--++(\depthX,\depthY)--++(\width,0)--++(-\depthX,-\depthY);
\draw(\width,0)--++(\depthX,\depthY)--++(0,\height)--++(-\depthX,-\depthY);
%left winding
\draw (\thick+\depthX,\cTop) to [out=45,in=0] ++(-\thick/2-\depthX,\p/2) to [short]++(-\thick/2,0) to [short] ++(-\x/4,0)coordinate(kTop);
\foreach \l in {0,1,2,...,\TL}{
\draw (0,\cTop-\l*\p) to [out=-135,in=45] ++(\thick+\depthX,-\p);
}
\draw(0,\cTop-\TL*\p-\p) to [short]++ (-\x/4,0)coordinate(kBot);
%right winding
\draw (\width-\thick,\cTopR) to [out=135,in=180] ++(\thick/2,\p/2) to [short]++(\thick/2+\depthX,0) to [short,-o] ++(\x,0)coordinate(kTopR);
\foreach \l in {0,1,2,...,\TR}{
\draw (\width+\depthX,\cTopR-\l*\p) to [out=-45,in=135] ++(-\thick-\depthX,-\p);
}
\draw(\width+\depthX,\cTopR-\TR*\p-\p) to [short,-o]++ (\x,0)coordinate(kBotR);
%current
\draw(kBot) to [short]++(-\x,0) to [american current source,l={$i$}]++(0,1.7)coordinate(currT)|- (kTop);
%text
\draw(0,\height/2) node [left]{$N_1$};
\draw(\width+\depthX,\height/2) node [right]{$N_2$};
\draw(currT)++(1,-0.85) node{$\begin{aligned} &+ \\ &v_1 \\ &- \end{aligned}$};
\draw($(kTopR)!0.5!(kBotR)$) node{$\begin{aligned} &+ \\ &v_2 \\ &- \end{aligned}$};
\end{tikzpicture}
\caption{لچھے مقناطیسی میدان کے ذریعے رابطے میں ہیں۔}
\label{شکل_مقناطیسی_مشترکہ_امالہ}
\end{figure}

شکل \حوالہ{شکل_مقناطیسی_خود_امالہ}-الف میں موجود لچھے کے قریب دوسرا لچھا رکھنے سے شکل \حوالہ{شکل_مقناطیسی_مشترکہ_امالہ} حاصل ہوتا ہے۔دوسرے لچھے میں رو نہیں گزر رہی ہے۔پہلے لچھے  کا ارتباط بہاو درج ذیل ہے۔
\begin{align}
\lambda_1=N_1 \phi=L_1 i_1
\end{align}
بدلتی رو کی صورت میں ارتباط بہاو بھی وقت کے ساتھ تبدیل ہو گا۔بدلتا ارتباط بہاو پہلے لچھے میں دباو \عددی{v_1=\tfrac{\dif \lambda_1}{\dif t}=L_1 \tfrac{\dif i_1}{\dif t}} پیدا کرے گا۔متعدد لچھوں کی صورت میں \عددی{L_1} کو \اصطلاح{خود امالہ}\فرہنگ{خود امالہ}\فرہنگ{امالہ!خود}\حاشیہب{self inductance}\فرہنگ{inductance!self} کہا جاتا ہے۔

 دوسرے لچھے کا ارتباط بہاو \عددی{\lambda_2=N_2 \phi} ہے جو دوسرے لچھے میں قانون فیراڈے کے تحت درج ذیل دباو پیدا کرے گا۔
\begin{align}
v_2=\frac{\dif \lambda_2}{\dif t}=\frac{\dif}{\dif t}\left(N_2 \phi\right)=\frac{\dif}{\dif t}\left(N_2 \frac{L_1 i_1}{N_1}\right)=\frac{N_2}{N_1} L_1 \frac{\dif i_1}{\dif t}=L_{21}\frac{\dif i_1}{\dif t}
\end{align}
دوسرے لچھے کا دباو پہلے لچھے کی رو کے وقتی تفرق کے راست تناسب ہے۔راست تناسب کے مستقل \عددی{L_{21}} کو دونوں لچھوں کا \اصطلاح{مشترکہ امالہ}\فرہنگ{مشترکہ امالہ}\فرہنگ{امالہ!مشترکہ}\حاشیہب{mutual inductance}\فرہنگ{mutual inductance}\فرہنگ{inductance!mutual} کہا جاتا ہے جسے ہینری \عددی{\si{\henry}} میں ناپا جاتا ہے۔ ہم کہتے ہیں کہ یہ لچھے آپ میں مقناطیسی میدان کے ذریعہ رابطے میں ہیں۔یوں ان لچھوں کو \اصطلاح{مربوط لچھے}\فرہنگ{مربوط لچھے}\حاشیہب{coupled coils}\فرہنگ{coupled coils} کہا جاتا ہے۔
\begin{figure}
\centering
\begin{subfigure}{1\textwidth}
\centering
\begin{tikzpicture}[american voltages]
\def\height{3};
\def\width{1.5};
\def\thick{0.4};
\def\depthX{0.2};
\def\depthY{0.2};
\def\p{0.2};      %pitch
\def\cTop{2.4}; %top of coil
\def\TL{7};    %number of turns
\def\cTopR{2.3}; %top of right coil
\def\TR{6};    %number of right turns
%flux
\draw[gray,-stealth](\thick/2,\height-\thick) to [out=90,in=180]++(\thick/2,\thick/2) to [short]++(\width-2*\thick,0) to [out=0,in=90]++(\thick/2,-\thick/2);
\draw(\width/2,\height-\thick/2)node[fill=white]{$\phi$};
%core
\draw(0,0)--++(0,\height)--++(\width,0)--++(0,-\height)--cycle;
\draw(0,0)++(\thick,\thick)--++(0,\height-2*\thick)--++(\width-2*\thick,0)--++(0,-\height+2*\thick)--cycle;
%
\draw(\thick,\thick)--++(\depthX,\depthY) --++(0,\height-2*\thick-\depthY);
\draw(\thick,\thick)--++(\depthX,\depthY) --++(\width-2*\thick-\depthX,0);
\draw(0,\height)--++(\depthX,\depthY)--++(\width,0)--++(-\depthX,-\depthY);
\draw(\width,0)--++(\depthX,\depthY)--++(0,\height)--++(-\depthX,-\depthY);
%left winding
\draw (\thick+\depthX,\cTop) to [out=45,in=0] ++(-\thick/2-\depthX,\p/2) to [short]++(-\thick/2,0) to [short] ++(-\x/4,0)coordinate(kTop);
\foreach \l in {0,1,2,...,\TL}{
\draw (0,\cTop-\l*\p) to [out=-135,in=45] ++(\thick+\depthX,-\p);
}
\draw(0,\cTop-\TL*\p-\p) to [short]++ (-\x/4,0)coordinate(kBot);
%right winding
\draw (\width-\thick,\cTopR) to [out=135,in=180] ++(\thick/2,\p/2) to [short]++(\thick/2+\depthX,0) to [short] ++(\x,0)coordinate(kTopR);
\foreach \l in {0,1,2,...,\TR}{
\draw (\width+\depthX,\cTopR-\l*\p) to [out=-45,in=135] ++(-\thick-\depthX,-\p);
}
\draw(\width+\depthX,\cTopR-\TR*\p-\p) to [short]++ (\x,0)coordinate(kBotR);
%current
\draw(kBot) to [short]++(-2*\x,0) to [american current source,l={$i_1$}]++(0,1.7)coordinate(currT)|- (kTop);
\draw(kBotR) to [short]++(\x+\x/4,0)coordinate(kRB) to [american current source,l_={$i_2$}]++(0,1.5)|- (kTopR);
%text
\draw(0,\height/2) node [left]{$N_1$};
\draw(\width+\depthX,\height/2) node [right]{$N_2$};
\draw(currT)++(0.75,-0.85) node{$\begin{aligned} &+ \\ &v_1 \\ &- \end{aligned}$};
\draw(kRB)++(-0.75,0.75) node[]{$\begin{aligned} &+ \\ &v_2 \\ &- \end{aligned}$};
\end{tikzpicture}
\caption*{(الف)}
\end{subfigure}
\begin{subfigure}{1\textwidth}
\centering
\begin{circuitikz}
 \draw(0,0) to [american current source,l={$i_1$}]++(0,\y) to [short]++(2*\x+\x/2,0) to [inductor,l_={$L_1$}]++(0,-\y) to [short](0,0);
\draw(2*\x+\x/2+\x/3,0) to [inductor,l_={$L_2$}]++(0,\y) to [short]++(2*\x+\x/2,0);
\draw(2*\x+\x/2+\x/3,0) to [short]++(2*\x+\x/2,0) to [american current source,l_={$i_2$}]++(0,\y);
%mutual
\draw(2*\x+\x/2+\x/6,\y) node[above]{$M$};
\draw[fill](2*\x+\x/2,\y)++(-0.5,-0.5) circle (\kdot);
\draw[fill](2*\x+\x/2+\x/3,\y)++(0.5,-0.5) circle (\kdot);
%voltages
\draw(0.4,\y/2)node[right]{$\begin{aligned} &+ \\ & v_1=L_1 \frac{\dif i_1}{\dif t}+M\frac{\dif i_2}{\dif t} \\ &-  \end{aligned}$};
\draw(4*\x+\x/2+\x/2+\x/3-0.4,\y/2)node[left]{$\begin{aligned} &+ \\ M\frac{\dif i_1}{\dif t}+L_2\frac{\dif i_2}{\dif t}=& v_2 \\ &-  \end{aligned}$};
\end{circuitikz}
\caption*{(ب)}
\end{subfigure}
\caption{قالب میں لچھوں کے بہاو ایک ہی سمت میں ہیں۔}
\label{شکل_مقناطیسی_مشترکہ_امالہ_ب}
\end{figure}
شکل \حوالہ{شکل_مقناطیسی_مشترکہ_امالہ_ب}-الف میں دونوں لچھوں کو انفرادی منبع سے رو فراہم کی گئی ہے۔دونوں لچھوں پر باری باری غور کریں۔ان کی رو اور قالب کے گرد لچھے کے چکروں کی سمت کو دیکھیں۔انفرادی لچھے کی رو گھڑی کی سمت میں گھومتی بہاو پیدا کرتی ہے۔ اس طرح دونوں رو مل کر مقناطیسی بہاو \عددی{\phi} پیدا کرتی ہیں۔یوں لچھوں کی ارتباط بہاو درج ذیل ہو گی۔
\begin{align}
\lambda_1&=L_1 i_1 +L_{12} i_2\\
\lambda_2&=L_{21} i_1+L_2 i_2
\end{align}
فیراڈے کے قانون کے تحت لچھوں کے دباو حاصل کرتے ہیں۔
\begin{align}
v_1&=\frac{\dif \lambda_1}{\dif t}=L_1\frac{\dif  i_1}{\dif t} +L_{12} \frac{\dif i_2}{\dif t} \label{مساوات_مقناطیسی_مشترک_لچھے_دباو_الف}\\
v_2&=\frac{\dif \lambda_{2}}{\dif t}=L_{21}\frac{\dif  i_1}{\dif t} +L_{2} \frac{\dif i_2}{\dif t}\label{مساوات_مقناطیسی_مشترک_لچھے_دباو_ب}
\end{align}
ان مساوات میں \عددی{L_{12}=L_{21}=M} کے برابر ہے جہاں مشترکہ امالہ کو \عددی{M} سے ظاہر کیا گیا ہے۔لچھے کے دباو کے دو اجزاء ہیں۔پہلا جزو لچھے کی اپنی رو کی بنا ہے اور یہ خود جزو کہلاتا ہے۔دوسرا جزو قریبی لچھے کی رو کے بنا ہے اور یہ مشترک جزو کہلاتا ہے۔ 

شکل \حوالہ{شکل_مقناطیسی_مشترکہ_امالہ_ب}-ب میں \اصطلاح{مربوط} لچھوں کو ظاہر کرنا دکھایا گیا ہے۔لچھوں کے انفرادی خود امالہ کو \عددی{L_1} اور \عددی{L_2} سے ظاہر کیا گیا ہے جبکہ ان کے مابین مشترکہ امالہ کو \عددی{M} سے ظاہر کیا گیا ہے۔
\begin{figure}
\centering
\begin{subfigure}{1\textwidth}
\centering
\begin{tikzpicture}[american voltages]
\def\height{3};
\def\width{1.5};
\def\thick{0.4};
\def\depthX{0.2};
\def\depthY{0.2};
\def\p{0.2};      %pitch
\def\cTop{2.4}; %top of coil
\def\TL{7};    %number of turns
\def\cTopR{2.3}; %top of right coil
\def\TR{6};    %number of right turns
%flux
\draw[gray,-stealth](\thick/2,\height-\thick) to [out=90,in=180]++(\thick/2,\thick/2) to [short]++(\width-2*\thick,0) to [out=0,in=90]++(\thick/2,-\thick/2);
\draw(\width/2,\height-\thick/2)node[fill=white]{$\phi$};
%core
\draw(0,0)--++(0,\height)--++(\width,0)--++(0,-\height)--cycle;
\draw(0,0)++(\thick,\thick)--++(0,\height-2*\thick)--++(\width-2*\thick,0)--++(0,-\height+2*\thick)--cycle;
%
\draw(\thick,\thick)--++(\depthX,\depthY) --++(0,\height-2*\thick-\depthY);
\draw(\thick,\thick)--++(\depthX,\depthY) --++(\width-2*\thick-\depthX,0);
\draw(0,\height)--++(\depthX,\depthY)--++(\width,0)--++(-\depthX,-\depthY);
\draw(\width,0)--++(\depthX,\depthY)--++(0,\height)--++(-\depthX,-\depthY);
%left winding
\draw (\thick+\depthX,\cTop) to [out=45,in=0] ++(-\thick/2-\depthX,\p/2) to [short]++(-\thick/2,0) to [short] ++(-\x/4,0)coordinate(kTop);
\foreach \l in {0,1,2,...,\TL}{
\draw (0,\cTop-\l*\p) to [out=-135,in=45] ++(\thick+\depthX,-\p);
}
\draw(0,\cTop-\TL*\p-\p) to [short]++ (-\x/4,0)coordinate(kBot);
%right winding
\draw(\width+\depthX,\cTopR) to [short]++ (\x,0)coordinate(kTopR);
\foreach \l in {0,1,2,...,\TR}{
\draw (\width-\thick,\cTopR-\l*\p) to [out=-135,in=455] ++(\thick+\depthX,-\p);
}
\draw (\width-\thick,\cTopR-\TR*\p-\p) to [out=-135,in=180] ++(\thick/2,-\p/2) to [short]++(\thick/2+\depthX,0) to [short] ++(\x,0)coordinate(kBotR);
%current
\draw(kBot) to [short]++(-2*\x,0) to [american current source,l={$i_1$}]++(0,1.7)coordinate(currT)|- (kTop);
\draw(kBotR) to [short]++(\x+\x/4,0)coordinate(kRB) to [american current source,l_={$i_2$}]++(0,1.5)|- (kTopR);
%text
\draw(0,\height/2) node [left]{$N_1$};
\draw(\width+\depthX,\height/2) node [right]{$N_2$};
\draw(currT)++(0.75,-0.85) node{$\begin{aligned} &+ \\ &v_1 \\ &- \end{aligned}$};
\draw(kRB)++(-0.75,0.75) node[]{$\begin{aligned} &+ \\ &v_2 \\ &- \end{aligned}$};
\end{tikzpicture}
\caption*{(الف)}
\end{subfigure}
\begin{subfigure}{1\textwidth}
\centering
\begin{circuitikz}
 \draw(0,0) to [american current source,l={$i_1$}]++(0,\y) to [short]++(2*\x+\x/2,0) to [inductor,l_={$L_1$}]++(0,-\y) to [short](0,0);
\draw(2*\x+\x/2+\x/3,0) to [inductor,l_={$L_2$}]++(0,\y) to [short]++(2*\x+\x/2,0);
\draw(2*\x+\x/2+\x/3,0) to [short]++(2*\x+\x/2,0) to [american current source,l_={$i_2$}]++(0,\y);
%mutual
\draw(2*\x+\x/2+\x/6,\y) node[above]{$M$};
\draw[fill](2*\x+\x/2,\y)++(-0.5,-0.5) circle (\kdot);
\draw[fill](2*\x+\x/2+\x/3,0)++(0.5,0.5) circle (\kdot);
%voltages
\draw(0.4,\y/2)node[right]{$\begin{aligned} &+ \\ & v_1=L_1 \frac{\dif i_1}{\dif t}-M\frac{\dif i_2}{\dif t} \\ &-  \end{aligned}$};
\draw(4*\x+\x/2+\x/2+\x/3-0.4,\y/2)node[left]{$\begin{aligned} &+ \\ -M\frac{\dif i_1}{\dif t}+L_2\frac{\dif i_2}{\dif t}=& v_2 \\ &-  \end{aligned}$};
\end{circuitikz}
\caption*{(ب)}
\end{subfigure}
\caption{قالب میں لچھوں کے بہاو آپس میں الٹ سمت ہیں۔}
\label{شکل_مقناطیسی_مشترکہ_الٹ_بہاو}
\end{figure}

شکل \حوالہ{شکل_مقناطیسی_مشترکہ_الٹ_بہاو}-الف میں قالب کے گرد، دائیں لچھے کے چکر الٹائے گئے ہیں۔یوں قالب میں بائیں لچھے کا بہاو گھڑی کی سمت میں گھومتا ہے جبکہ دائیں لچھے کا بہاو گھڑی کی الٹ سمت میں گھومتا ہے لہٰذا کل بہاو \عددی{\phi} حاصل کرنے کی خاطر بائیں لچھے کے بہاو سے دائیں لچھے کا بہاو منفی کرنا ہو گا۔ اس طرح لچھوں کی ارتباط بہاو
\begin{align}
\lambda_1&=L_1 i_1-M i_2\\
\lambda_2&=-M i_1+L_2 i_2
\end{align}
لکھی جائے گی اور ان کے دباو درج ذیل لکھے جائیں گے۔
\begin{align}
v_1&=L_1 \frac{\dif i_1}{\dif t}-M\frac{\dif i_2}{\dif t}\label{مساوات_مقناطیسی_مشترک_لچھے_دباو_پ}\\
v_2&=-M\frac{\dif i_1}{\dif t}+L_2 \frac{\dif i_2}{\dif t}\label{مساوات_مقناطیسی_مشترک_لچھے_دباو_ت}
\end{align}

شکل \حوالہ{شکل_مقناطیسی_مشترکہ_امالہ_ب}-الف میں دونوں لچھوں کی انفرادی بہاو کا مجموعہ قالب میں کل بہاو دیتا ہے جبکہ  شکل \حوالہ{شکل_مقناطیسی_مشترکہ_الٹ_بہاو}-الف  میں بائیں لچھے کے بہاو سے دائیں لچھے کا بہاو تفریق کرنے سے قالب میں کل بہاو  \عددی{\phi} حاصل ہوتا ہے۔لچھوں میں رو کی سمت، قالب کے گرد چکر کی سمت اور قالب میں بہاو کی سمت کو نہایت عمدگی سے نقطوں کی مدد سے ظاہر کیا جاتا ہے۔شکل \حوالہ{شکل_مقناطیسی_مشترکہ_امالہ_ب}-ب اور شکل \حوالہ{شکل_مقناطیسی_مشترکہ_الٹ_بہاو}-ب میں ان نقطوں کا استعمال دکھایا گیا ہے۔

انفرادی لچھے کی رو اور دباو کو انفعالی رائج سمت کے تحت چننیں۔دونوں لچھوں میں نقطوں والے سر سے رو داخل ہونے کی صورت میں دباو کا مشترک جزو مثبت لکھا جاتا ہے جبکہ ایک لچھے کی رو نقطے والے سر اور دوسرے لچھے کی رو بے نقطے والے سر سے داخل ہونے کی صورت میں مشترک دباو منفی لکھا جاتا ہے۔دونوں رو بے نقطے سروں سے داخل ہونے کی صورت میں مشترک دباو  مثبت لکھا جائے گا۔دباو کا خود جزو تمام صورتوں میں انفعالی رائج سمت کے تحت مثبت لکھا جاتا ہے۔یوں شکل \حوالہ{شکل_مقناطیسی_مشترکہ_امالہ_ب}  میں  مساوات \حوالہ{مساوات_مقناطیسی_مشترک_لچھے_دباو_الف} اور مساوات \حوالہ{مساوات_مقناطیسی_مشترک_لچھے_دباو_ب} دباو دیں گے جبکہ شکل \حوالہ{شکل_مقناطیسی_مشترکہ_الٹ_بہاو} میں مساوات \حوالہ{مساوات_مقناطیسی_مشترک_لچھے_دباو_پ} اور مساوات \حوالہ{مساوات_مقناطیسی_مشترک_لچھے_دباو_ت} دباو دیں گے۔

مشترک امالہ کے کرخوف مساوات دباو نسبتاً زیادہ آسانی سے لکھے جاتے ہیں۔
%==============
\ابتدا{مثال}\شناخت{مثال_مقناطیسی_مشترک_امالہ_دباو_الف}
شکل \حوالہ{شکل_مقناطیسی_مشترک_امالہ_دباو_الف} میں دیے دور کے  دونوں اطراف کے دباو کے مساوات لکھیں۔
\begin{figure}
\centering
\begin{circuitikz}
\draw(0,0) rectangle ++(-\boxW,\boxH);
\draw(-0.25,\boxH/2) node[rotate=90]{\RL{بایاں دور}};
\draw(0,0.25) to [short]++(\x,0)coordinate(BL) to [inductor,l={$L_1$}]++(0,\y)coordinate(TL) to [short,i<_={$i_1$}]++(-\x,0);
\draw(\x+\x/3+\x,0.25) to [short]++(-\x,0)coordinate(BR) to [inductor,l_={$L_2$}]++(0,\y)coordinate(TR) to [short,i<_={$i_2$}]++(\x,0);
\draw($(TL)!0.5!(TR)$)node[above]{$M$};
\draw[fill](BL)++(-0.5,0.5) circle (\kdot); 
\draw[fill](BR)++(0.5,0.5) circle (\kdot); 
\draw(2*\x+\x/3,0) rectangle ++(\boxW,\boxH);
\draw(2*\x+\x/3,0)++(\boxW/2,\boxH/2) node[rotate=90]{\RL{دایاں دور}};
\draw(0,\boxH/2) node[right]{$\begin{aligned} &+ \\ &v_1 \\ &-  \end{aligned}$};
\draw(2*\x+\x/3,\boxH/2) node[left]{$\begin{aligned} &- \\ &v_2 \\ &+  \end{aligned}$};
\end{circuitikz}
\caption{مثال \حوالہ{مثال_مقناطیسی_مشترک_امالہ_دباو_الف} کا دور۔}
\label{شکل_مقناطیسی_مشترک_امالہ_دباو_الف}
\end{figure}

حل:بائیں جانب \عددی{v_1} اور \عددی{i_1} عین انفعالی رائج سمت کے تحت لکھے گئے ہیں۔یوں دباو کا خود جزو مثبت لکھا جائے گا۔دونوں لچھوں میں رو بے نقطے سروں سے داخل ہوتی ہے لہٰذا دباو کا مشترک جزو مثبت لکھا جائے گا۔یوں بائیں جانب کرخوف کی مساوات درج ذیل ہو گی۔
\begin{align*}
v_1=L_1 \frac{\dif i_1}{\dif t}+M\frac{\dif i_2}{\dif t}
\end{align*} 
دائیں جانب \عددی{v_2} اور \عددی{i_2} انفعالی رائج سمت کے تحت نہیں چننے گئے ہیں۔یوں دباو کے اجزاء لکھتے ہوئے اس کا خیال رکھا جائے گا۔دوسرے لچھے کی مساوات درج ذیل
\begin{align*}
-v_2=L_2 \frac{\dif i_2}{\dif t}+M\frac{\dif i_1}{\dif t}
\end{align*}
یعنی
\begin{align*}
v_2=-M\frac{\dif i_1}{\dif t}-L_2 \frac{\dif i_2}{\dif t}
\end{align*}
لکھی جائے گی۔ 
\انتہا{مثال}

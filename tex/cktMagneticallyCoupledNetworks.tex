\باب{مقناطیسی جڑے ادوار}

\حصہ{مشترکہ امالہ}
شکل \حوالہ{شکل_مقناطیسی_خود_امالہ}-الف میں \عددی{N} چکر کا \اصطلاح{لچھا}\فرہنگ{لچھا}\حاشیہب{coil}\فرہنگ{coil} مقناطیسی مادے سے بنائے گئے \اصطلاح{قالب}\فرہنگ{قالب!مقناطیسی}\حاشیہب{core}\فرہنگ{core!magnetic} پر لپیٹا گیا  دکھایا گیا ہے۔اس لچھے میں \عددی{i} رو گزر رہی ہے۔ایمپیئر کے قانون کے تحت رو کے گزرنے سے مقناطیسی میدان پیدا ہوتا ہے۔یوں رو کے گزرنے سے لچھے میں \عددی{\phi} \اصطلاح{مقناطیسی بہاو}\فرہنگ{مقناطیسی بہاو}\فرہنگ{بہاو!مقناطیسی}\حاشیہب{magnetic flux}\فرہنگ{magnetic flux}\فرہنگ{flux!magnetic} پیدا ہوتا ہے جسے ہلکی سیاہی میں نقطہ دار لکیر سے دکھایا گیا ہے۔

لچھے میں رو کی سمت اور مقناطیسی بہاو کی سمت کے تعلق پر غور کریں۔ان کا تعلق دائیں ہاتھ کا قانون کہلاتا ہے۔دائیں ہاتھ کا قانون درج ذیل ہے۔

\ابتدا{قانون}
اگر لچھے کو دائیں ہاتھ سے یوں پکڑا جائے کہ ہاتھ کی چار انگلیاں رو کی سمت میں لپیٹے جائیں تب اسی ہاتھ کا انگوٹھا بہاو کی سمت دے گا۔
\انتہا{قانون}
مقناطیسی بہاو کو کسی مخصوص خطے میں رکھنے کی خاطر مقناطیسی قالب استعمال کیا جاتا ہے۔مقناطیسی بہاو کے لئے مقناطیسی مادے سے گزرنا زیادہ آسان ثابت ہوتا ہے لہٰذا شکل \حوالہ{شکل_مقناطیسی_خود_امالہ}-الف میں بہاو قالب کے اندر ہی رہتے ہوئے گھڑی کے سوئیوں کے گھومنے  کی سمت میں گھومتا ہے۔یوں مقناطیسی بہاو \عددی{\phi} لچھے کے تمام چکروں کے اندر سے گزرتا ہے۔لچھے کا \اصطلاح{ارتباط بہاو}\فرہنگ{ارتباط بہاو}\فرہنگ{بہاو!ارتباط}\حاشیہب{flux linkage}\فرہنگ{flux linkage} \عددی{\lambda} درج ذیل ہے۔
\begin{align}\label{مساوات_مشترک_ارتباط_بہاو_الف}
\lambda=N \phi
\end{align}
اس کتاب میں صرف خطی نظام پر غور کیا گیا ہے۔خطی صورت میں ارتباط بہاو اور رو کا تعلق درج ذیل ہے
\begin{align}\label{مساوات_مشترک_ارتباط_بہاو_ب}
\lambda=L i
\end{align}
جہاں مساوات کے مستقل \عددی{L} کو \اصطلاح{خود امالہ}\فرہنگ{امالہ!خود}\فرہنگ{خود امالہ}\حاشیہب{self inductance}\فرہنگ{inductance!self}\فرہنگ{self inductance} یا \اصطلاح{امالہ} کہتے ہیں۔باب \حوالہ{باب_برق_گیر_امالہ_گیر} میں امالہ پر غور کیا گیا ہے۔درج بالا دو مساوات کو ملاتے ہوئے  بہاو اور رو کا تعلق ملتا ہے۔
\begin{align}
\phi=\frac{Li}{N}
\end{align}
قانون فیراڈے کے تحت بدلتی ارتباط بہاو لچھے میں امالی دباو پیدا کرتا ہے۔
\begin{align}
v=\frac{\dif \lambda}{\dif t}
\end{align}
مساوات \حوالہ{مساوات_مشترک_ارتباط_بہاو_ب} کو درج بالا مساوات میں پر کرتے ہیں۔
\begin{align*}
v=\frac{\dif \lambda}{\dif t}=\frac{\dif (Li)}{\dif t}=L\frac{\dif i}{\dif t}+i\frac{\dif L}{\dif t}
\end{align*}
مستقل امالہ کی صورت میں اس مساوات سے امالہ کی جانی پہچانی درج ذیل مساوات حاصل ہوتی ہے۔
\begin{align}\label{مساوات_مقناطیسی_امالہ_کی_مساوات}
v=L\frac{\dif i}{\dif t}
\end{align}
اس کتاب میں مستقل امالہ پر ہی غور کیا جائے گا۔شکل \حوالہ{شکل_مقناطیسی_خود_امالہ}-ب میں اس امالہ کو دکھایا گیا ہے۔یہاں غور کریں کہ مزاحمت کی طرح امالہ کے دباو اور رو بھی انفعالی رائج سمت کے تحت ہیں۔یوں امالہ میں رو مثبت دباو والے سر سے داخل ہوتی ہے۔مساوات \حوالہ{مساوات_مقناطیسی_امالہ_کی_مساوات} کہتا ہے کہ  بدلتی رو کے گزرنے سے امالہ میں دباو پیدا ہوتا ہے۔
%
\begin{figure}
\centering
\begin{subfigure}{0.6\textwidth}
\centering
\begin{tikzpicture}[american voltages]
\def\height{3};
\def\width{1.5};
\def\thick{0.4};
\def\depthX{0.2};
\def\depthY{0.2};
\def\gap{0.05};
\def\p{0.2};      %pitch
\def\cTop{2.4}; %top of coil
\def\TL{7};    %number of turns
%flux
\draw[gray,dashed,-stealth](\thick/2,\height-\thick) to [out=90,in=180]++(\thick/2,\thick/2) to [short]++(\width-2*\thick,0) to [out=0,in=90]++(\thick/2,-\thick/2) to [short]++(0,-\height+2*\thick) to [out=-90,in=0]++(-\thick/2,-\thick/2) to [short]++(-\width+2*\thick,0) to [out=180,in=-90]++(-\thick/2,\thick/2) to [short]++(0,0.2);
\draw[gray,-stealth](\width-\thick/2,\height/2)++(0,0.05)--++(0,-0.1);
\draw(\width/2,\height-\thick/2)node[fill=white]{$\phi$};
%core
\draw(0,0)--++(0,\height)--++(\width,0)--++(0,-\height)--cycle;
\draw(0,0)++(\thick,\thick)--++(0,\height-2*\thick)--++(\width-2*\thick,0)--++(0,-\height+2*\thick)--cycle;
%
\draw(\thick,\thick)--++(\depthX,\depthY) --++(0,\height-2*\thick-\depthY);
\draw(\thick,\thick)--++(\depthX,\depthY) --++(\width-2*\thick-\depthX,0);
\draw(0,\height)--++(\depthX,\depthY)--++(\width,0)--++(-\depthX,-\depthY);
\draw(\width,0)--++(\depthX,\depthY)--++(0,\height)--++(-\depthX,-\depthY);
%left winding
\draw (\thick+\depthX,\cTop) to [out=45,in=0] ++(-\thick/2-\depthX,\p/2) to [short]++(-\thick/2,0) to [short] ++(-\x/4,0)coordinate(kTop);
\foreach \l in {0,1,2,...,\TL}{
\draw (0,\cTop-\l*\p) to [out=-135,in=45] ++(\thick+\depthX,-\p);
}
\draw(0,\cTop-\TL*\p-\p) to [short]++ (-\x/4,0)coordinate(kBot);
%current
\draw(kBot) to [short]++(-\x,0) to [american current source,l={$i$}]++(0,1.7)coordinate(currT)|- (kTop);
\draw(currT)++(1,-0.85) node{$\begin{aligned}&+ \\ &v \\ &-   \end{aligned}$};
%text
\draw(0,\height/2)node[left]{$N$};
\draw[stealth-](\width+\depthX,2/3*\height) to [out=45,in=180]++(0.5,0.5)node[right]{\RL{مقناطیسی قالب}};
\end{tikzpicture}
\caption*{(الف)}
\end{subfigure}%
\begin{subfigure}{0.4\textwidth}
\centering
\begin{circuitikz}[american voltages]
\draw(0,0) to [short,i={$i$},o-]++(\x,0) to [inductor,l={$L$}]++(0,-\y) to [short,-o] ++(-\x,0);
\draw ($(0,0)!0.5!(0,-\y)$)node{$\begin{aligned} &+ \\ &v \\ &-  \end{aligned}$};
\end{circuitikz}
\caption*{(ب)}
\end{subfigure}
\caption{خود امالہ کی تعریف۔}
\label{شکل_مقناطیسی_خود_امالہ}
\end{figure}
%
\begin{figure}
\centering
\begin{tikzpicture}[american voltages]
\def\height{3};
\def\width{1.5};
\def\thick{0.4};
\def\depthX{0.2};
\def\depthY{0.2};
\def\p{0.2};      %pitch
\def\cTop{2.4}; %top of coil
\def\TL{7};    %number of turns
\def\cTopR{2.3}; %top of right coil
\def\TR{6};    %number of right turns
%flux
\draw[gray,-stealth](\thick/2,\height-\thick) to [out=90,in=180]++(\thick/2,\thick/2) to [short]++(\width-2*\thick,0) to [out=0,in=90]++(\thick/2,-\thick/2);
\draw(\width/2,\height-\thick/2)node[fill=white]{$\phi$};
%core
\draw(0,0)--++(0,\height)--++(\width,0)--++(0,-\height)--cycle;
\draw(0,0)++(\thick,\thick)--++(0,\height-2*\thick)--++(\width-2*\thick,0)--++(0,-\height+2*\thick)--cycle;
%
\draw(\thick,\thick)--++(\depthX,\depthY) --++(0,\height-2*\thick-\depthY);
\draw(\thick,\thick)--++(\depthX,\depthY) --++(\width-2*\thick-\depthX,0);
\draw(0,\height)--++(\depthX,\depthY)--++(\width,0)--++(-\depthX,-\depthY);
\draw(\width,0)--++(\depthX,\depthY)--++(0,\height)--++(-\depthX,-\depthY);
%left winding
\draw (\thick+\depthX,\cTop) to [out=45,in=0] ++(-\thick/2-\depthX,\p/2) to [short]++(-\thick/2,0) to [short] ++(-\x/4,0)coordinate(kTop);
\foreach \l in {0,1,2,...,\TL}{
\draw (0,\cTop-\l*\p) to [out=-135,in=45] ++(\thick+\depthX,-\p);
}
\draw(0,\cTop-\TL*\p-\p) to [short]++ (-\x/4,0)coordinate(kBot);
%right winding
\draw (\width-\thick,\cTopR) to [out=135,in=180] ++(\thick/2,\p/2) to [short]++(\thick/2+\depthX,0) to [short,-o] ++(\x,0)coordinate(kTopR);
\foreach \l in {0,1,2,...,\TR}{
\draw (\width+\depthX,\cTopR-\l*\p) to [out=-45,in=135] ++(-\thick-\depthX,-\p);
}
\draw(\width+\depthX,\cTopR-\TR*\p-\p) to [short,-o]++ (\x,0)coordinate(kBotR);
%current
\draw(kBot) to [short]++(-\x,0) to [american current source,l={$i$}]++(0,1.7)coordinate(currT)|- (kTop);
%text
\draw(0,\height/2) node [left]{$N_1$};
\draw(\width+\depthX,\height/2) node [right]{$N_2$};
\draw(currT)++(1,-0.85) node{$\begin{aligned} &+ \\ &v_1 \\ &- \end{aligned}$};
\draw($(kTopR)!0.5!(kBotR)$) node{$\begin{aligned} &+ \\ &v_2 \\ &- \end{aligned}$};
\end{tikzpicture}
\caption{لچھے مقناطیسی میدان کے ذریعے رابطے میں ہیں۔}
\label{شکل_مقناطیسی_مشترکہ_امالہ}
\end{figure}

شکل \حوالہ{شکل_مقناطیسی_خود_امالہ}-الف میں موجود لچھے کے قریب دوسرا لچھا رکھنے سے شکل \حوالہ{شکل_مقناطیسی_مشترکہ_امالہ} حاصل ہوتا ہے۔دوسرے لچھے میں رو نہیں گزر رہی ہے۔پہلے لچھے  کا ارتباط بہاو درج ذیل ہے۔
\begin{align}
\lambda_1=N_1 \phi=L_1 i_1
\end{align}
بدلتی رو کی صورت میں ارتباط بہاو بھی وقت کے ساتھ تبدیل ہو گا۔بدلتا ارتباط بہاو پہلے لچھے میں دباو 
\begin{align}
v_1=\tfrac{\dif \lambda_1}{\dif t}=L_1 \tfrac{\dif i_1}{\dif t}
\end{align}
پیدا کرے گا۔متعدد لچھوں کی صورت میں \عددی{L_1} کو \اصطلاح{خود امالہ}\فرہنگ{خود امالہ}\فرہنگ{امالہ!خود}\حاشیہب{self inductance}\فرہنگ{inductance!self} کہا جاتا ہے۔

 دوسرے لچھے کا ارتباط بہاو \عددی{\lambda_2=N_2 \phi} ہے جو دوسرے لچھے میں قانون فیراڈے کے تحت درج ذیل دباو پیدا کرے گا۔
\begin{align}
v_2=\frac{\dif \lambda_2}{\dif t}=\frac{\dif}{\dif t}\left(N_2 \phi\right)=\frac{\dif}{\dif t}\left(N_2 \frac{L_1 i_1}{N_1}\right)=\frac{N_2}{N_1} L_1 \frac{\dif i_1}{\dif t}=L_{21}\frac{\dif i_1}{\dif t}
\end{align}
دوسرے لچھے کا دباو پہلے لچھے کی رو کے وقتی تفرق کے راست تناسب ہے۔راست تناسب کے مستقل \عددی{L_{21}} کو دونوں لچھوں کا \اصطلاح{مشترکہ امالہ}\فرہنگ{مشترکہ امالہ}\فرہنگ{امالہ!مشترکہ}\حاشیہب{mutual inductance}\فرہنگ{mutual inductance}\فرہنگ{inductance!mutual} کہا جاتا ہے جسے ہینری \عددی{\si{\henry}} میں ناپا جاتا ہے۔ ہم کہتے ہیں کہ یہ لچھے آپ میں مقناطیسی میدان کے ذریعہ رابطے میں ہیں۔یوں ان لچھوں کو \اصطلاح{مربوط لچھے}\فرہنگ{مربوط لچھے}\حاشیہب{coupled coils}\فرہنگ{coupled coils} کہا جاتا ہے۔
\begin{figure}
\centering
\begin{subfigure}{1\textwidth}
\centering
\begin{tikzpicture}[american voltages]
\def\height{3};
\def\width{1.5};
\def\thick{0.4};
\def\depthX{0.2};
\def\depthY{0.2};
\def\p{0.2};      %pitch
\def\cTop{2.4}; %top of coil
\def\TL{7};    %number of turns
\def\cTopR{2.3}; %top of right coil
\def\TR{6};    %number of right turns
%flux
\draw[gray,-stealth](\thick/2,\height-\thick) to [out=90,in=180]++(\thick/2,\thick/2) to [short]++(\width-2*\thick,0) to [out=0,in=90]++(\thick/2,-\thick/2);
\draw(\width/2,\height-\thick/2)node[fill=white]{$\phi$};
%core
\draw(0,0)--++(0,\height)--++(\width,0)--++(0,-\height)--cycle;
\draw(0,0)++(\thick,\thick)--++(0,\height-2*\thick)--++(\width-2*\thick,0)--++(0,-\height+2*\thick)--cycle;
%
\draw(\thick,\thick)--++(\depthX,\depthY) --++(0,\height-2*\thick-\depthY);
\draw(\thick,\thick)--++(\depthX,\depthY) --++(\width-2*\thick-\depthX,0);
\draw(0,\height)--++(\depthX,\depthY)--++(\width,0)--++(-\depthX,-\depthY);
\draw(\width,0)--++(\depthX,\depthY)--++(0,\height)--++(-\depthX,-\depthY);
%left winding
\draw (\thick+\depthX,\cTop) to [out=45,in=0] ++(-\thick/2-\depthX,\p/2) to [short]++(-\thick/2,0) to [short] ++(-\x/4,0)coordinate(kTop);
\foreach \l in {0,1,2,...,\TL}{
\draw (0,\cTop-\l*\p) to [out=-135,in=45] ++(\thick+\depthX,-\p);
}
\draw(0,\cTop-\TL*\p-\p) to [short]++ (-\x/4,0)coordinate(kBot);
%right winding
\draw (\width-\thick,\cTopR) to [out=135,in=180] ++(\thick/2,\p/2) to [short]++(\thick/2+\depthX,0) to [short] ++(\x,0)coordinate(kTopR);
\foreach \l in {0,1,2,...,\TR}{
\draw (\width+\depthX,\cTopR-\l*\p) to [out=-45,in=135] ++(-\thick-\depthX,-\p);
}
\draw(\width+\depthX,\cTopR-\TR*\p-\p) to [short]++ (\x,0)coordinate(kBotR);
%current
\draw(kBot) to [short]++(-2*\x,0) to [american current source,l={$i_1$}]++(0,1.7)coordinate(currT)|- (kTop);
\draw(kBotR) to [short]++(\x+\x/4,0)coordinate(kRB) to [american current source,l_={$i_2$}]++(0,1.5)|- (kTopR);
%text
\draw(0,\height/2) node [left]{$N_1$};
\draw(\width+\depthX,\height/2) node [right]{$N_2$};
\draw(currT)++(0.75,-0.85) node{$\begin{aligned} &+ \\ &v_1 \\ &- \end{aligned}$};
\draw(kRB)++(-0.75,0.75) node[]{$\begin{aligned} &+ \\ &v_2 \\ &- \end{aligned}$};
\end{tikzpicture}
\caption*{(الف)}
\end{subfigure}
\begin{subfigure}{1\textwidth}
\centering
\begin{circuitikz}
 \draw(0,0) to [american current source,l={$i_1$}]++(0,\y) to [short]++(2*\x+\x/2,0) to [inductor,l_={$L_1$}]++(0,-\y) to [short](0,0);
\draw(2*\x+\x/2+\x/3,0) to [inductor,l_={$L_2$}]++(0,\y) to [short]++(2*\x+\x/2,0);
\draw(2*\x+\x/2+\x/3,0) to [short]++(2*\x+\x/2,0) to [american current source,l_={$i_2$}]++(0,\y);
%mutual
\draw(2*\x+\x/2+\x/6,\y) node[above]{$M$};
\draw(2*\x+\x/2,\y)++(-0.5,-0.5) node[circ]{};
\draw(2*\x+\x/2+\x/3,\y)++(0.5,-0.5) node[circ]{};
%voltages
\draw(0.4,\y/2)node[right]{$\begin{aligned} &+ \\ & v_1=L_1 \frac{\dif i_1}{\dif t}+M\frac{\dif i_2}{\dif t} \\ &-  \end{aligned}$};
\draw(4*\x+\x/2+\x/2+\x/3-0.4,\y/2)node[left]{$\begin{aligned} &+ \\ M\frac{\dif i_1}{\dif t}+L_2\frac{\dif i_2}{\dif t}=& v_2 \\ &-  \end{aligned}$};
\end{circuitikz}
\caption*{(ب)}
\end{subfigure}
\caption{قالب میں لچھوں کے بہاو ایک ہی سمت میں ہیں۔}
\label{شکل_مقناطیسی_مشترکہ_امالہ_ب}
\end{figure}
شکل \حوالہ{شکل_مقناطیسی_مشترکہ_امالہ_ب}-الف میں دونوں لچھوں کو انفرادی منبع سے رو فراہم کی گئی ہے۔دونوں لچھوں پر باری باری غور کریں۔ان کی رو اور قالب کے گرد لچھے کے چکروں کی سمت کو دیکھیں۔انفرادی لچھے کی رو گھڑی کی سمت میں گھومتی بہاو پیدا کرتی ہے۔ اس طرح دونوں رو مل کر مقناطیسی بہاو \عددی{\phi} پیدا کرتی ہیں۔یوں لچھوں کی ارتباط بہاو درج ذیل ہو گی۔
\begin{align}
\lambda_1&=L_1 i_1 +L_{12} i_2\\
\lambda_2&=L_{21} i_1+L_2 i_2
\end{align}
فیراڈے کے قانون کے تحت لچھوں کے دباو حاصل کرتے ہیں۔
\begin{align}
v_1&=\frac{\dif \lambda_1}{\dif t}=L_1\frac{\dif  i_1}{\dif t} +L_{12} \frac{\dif i_2}{\dif t} \label{مساوات_مقناطیسی_مشترک_لچھے_دباو_الف}\\
v_2&=\frac{\dif \lambda_{2}}{\dif t}=L_{21}\frac{\dif  i_1}{\dif t} +L_{2} \frac{\dif i_2}{\dif t}\label{مساوات_مقناطیسی_مشترک_لچھے_دباو_ب}
\end{align}
ان مساوات میں \عددی{L_{12}=L_{21}=M} کے برابر ہے جہاں مشترکہ امالہ کو \عددی{M} سے ظاہر کیا گیا ہے۔لچھے کے دباو کے دو اجزاء ہیں۔پہلا جزو لچھے کی اپنی رو کی بنا ہے اور یہ خود جزو کہلاتا ہے۔دوسرا جزو قریبی لچھے کی رو کے بنا ہے اور یہ مشترک جزو کہلاتا ہے۔ 

شکل \حوالہ{شکل_مقناطیسی_مشترکہ_امالہ_ب}-ب میں \اصطلاح{مربوط} لچھوں کو ظاہر کرنا دکھایا گیا ہے۔لچھوں کے انفرادی خود امالہ کو \عددی{L_1} اور \عددی{L_2} سے ظاہر کیا گیا ہے جبکہ ان کے مابین مشترکہ امالہ کو \عددی{M} سے ظاہر کیا گیا ہے۔
\begin{figure}
\centering
\begin{subfigure}{1\textwidth}
\centering
\begin{tikzpicture}[american voltages]
\def\height{3};
\def\width{1.5};
\def\thick{0.4};
\def\depthX{0.2};
\def\depthY{0.2};
\def\p{0.2};      %pitch
\def\cTop{2.4}; %top of coil
\def\TL{7};    %number of turns
\def\cTopR{2.3}; %top of right coil
\def\TR{6};    %number of right turns
%flux
\draw[gray,-stealth](\thick/2,\height-\thick) to [out=90,in=180]++(\thick/2,\thick/2) to [short]++(\width-2*\thick,0) to [out=0,in=90]++(\thick/2,-\thick/2);
\draw(\width/2,\height-\thick/2)node[fill=white]{$\phi$};
%core
\draw(0,0)--++(0,\height)--++(\width,0)--++(0,-\height)--cycle;
\draw(0,0)++(\thick,\thick)--++(0,\height-2*\thick)--++(\width-2*\thick,0)--++(0,-\height+2*\thick)--cycle;
%
\draw(\thick,\thick)--++(\depthX,\depthY) --++(0,\height-2*\thick-\depthY);
\draw(\thick,\thick)--++(\depthX,\depthY) --++(\width-2*\thick-\depthX,0);
\draw(0,\height)--++(\depthX,\depthY)--++(\width,0)--++(-\depthX,-\depthY);
\draw(\width,0)--++(\depthX,\depthY)--++(0,\height)--++(-\depthX,-\depthY);
%left winding
\draw (\thick+\depthX,\cTop) to [out=45,in=0] ++(-\thick/2-\depthX,\p/2) to [short]++(-\thick/2,0) to [short] ++(-\x/4,0)coordinate(kTop);
\foreach \l in {0,1,2,...,\TL}{
\draw (0,\cTop-\l*\p) to [out=-135,in=45] ++(\thick+\depthX,-\p);
}
\draw(0,\cTop-\TL*\p-\p) to [short]++ (-\x/4,0)coordinate(kBot);
%right winding
\draw(\width+\depthX,\cTopR) to [short]++ (\x,0)coordinate(kTopR);
\foreach \l in {0,1,2,...,\TR}{
\draw (\width-\thick,\cTopR-\l*\p) to [out=-135,in=455] ++(\thick+\depthX,-\p);
}
\draw (\width-\thick,\cTopR-\TR*\p-\p) to [out=-135,in=180] ++(\thick/2,-\p/2) to [short]++(\thick/2+\depthX,0) to [short] ++(\x,0)coordinate(kBotR);
%current
\draw(kBot) to [short]++(-2*\x,0) to [american current source,l={$i_1$}]++(0,1.7)coordinate(currT)|- (kTop);
\draw(kBotR) to [short]++(\x+\x/4,0)coordinate(kRB) to [american current source,l_={$i_2$}]++(0,1.5)|- (kTopR);
%text
\draw(0,\height/2) node [left]{$N_1$};
\draw(\width+\depthX,\height/2) node [right]{$N_2$};
\draw(currT)++(0.75,-0.85) node{$\begin{aligned} &+ \\ &v_1 \\ &- \end{aligned}$};
\draw(kRB)++(-0.75,0.75) node[]{$\begin{aligned} &+ \\ &v_2 \\ &- \end{aligned}$};
\end{tikzpicture}
\caption*{(الف)}
\end{subfigure}
\begin{subfigure}{1\textwidth}
\centering
\begin{circuitikz}
 \draw(0,0) to [american current source,l={$i_1$}]++(0,\y) to [short]++(2*\x+\x/2,0) to [inductor,l_={$L_1$}]++(0,-\y) to [short](0,0);
\draw(2*\x+\x/2+\x/3,0) to [inductor,l_={$L_2$}]++(0,\y) to [short]++(2*\x+\x/2,0);
\draw(2*\x+\x/2+\x/3,0) to [short]++(2*\x+\x/2,0) to [american current source,l_={$i_2$}]++(0,\y);
%mutual
\draw(2*\x+\x/2+\x/6,\y) node[above]{$M$};
\draw(2*\x+\x/2,\y)++(-0.5,-0.5) node[circ]{};
\draw(2*\x+\x/2+\x/3,0)++(0.5,0.5) node[circ]{};
%voltages
\draw(0.4,\y/2)node[right]{$\begin{aligned} &+ \\ & v_1=L_1 \frac{\dif i_1}{\dif t}-M\frac{\dif i_2}{\dif t} \\ &-  \end{aligned}$};
\draw(4*\x+\x/2+\x/2+\x/3-0.4,\y/2)node[left]{$\begin{aligned} &+ \\ -M\frac{\dif i_1}{\dif t}+L_2\frac{\dif i_2}{\dif t}=& v_2 \\ &-  \end{aligned}$};
\end{circuitikz}
\caption*{(ب)}
\end{subfigure}
\caption{قالب میں لچھوں کے بہاو آپس میں الٹ سمت ہیں۔}
\label{شکل_مقناطیسی_مشترکہ_الٹ_بہاو}
\end{figure}

شکل \حوالہ{شکل_مقناطیسی_مشترکہ_الٹ_بہاو}-الف میں قالب کے گرد، دائیں لچھے کے چکر الٹائے گئے ہیں۔یوں قالب میں بائیں لچھے کا بہاو گھڑی کی سمت میں گھومتا ہے جبکہ دائیں لچھے کا بہاو گھڑی کی الٹ سمت میں گھومتا ہے لہٰذا گھڑی کی سمت میں کل بہاو \عددی{\phi} حاصل کرنے کی خاطر بائیں لچھے کے بہاو سے دائیں لچھے کا بہاو منفی کرنا ہو گا۔ اس طرح لچھوں کی ارتباط بہاو
\begin{align}
\lambda_1&=L_1 i_1-M i_2\\
\lambda_2&=-M i_1+L_2 i_2
\end{align}
لکھی جائے گی اور ان کے دباو درج ذیل لکھے جائیں گے۔
\begin{align}
v_1&=L_1 \frac{\dif i_1}{\dif t}-M\frac{\dif i_2}{\dif t}\label{مساوات_مقناطیسی_مشترک_لچھے_دباو_پ}\\
v_2&=-M\frac{\dif i_1}{\dif t}+L_2 \frac{\dif i_2}{\dif t}\label{مساوات_مقناطیسی_مشترک_لچھے_دباو_ت}
\end{align}

شکل \حوالہ{شکل_مقناطیسی_مشترکہ_امالہ_ب}-الف میں دونوں لچھوں کی انفرادی بہاو کا مجموعہ قالب میں کل بہاو دیتا ہے جبکہ  شکل \حوالہ{شکل_مقناطیسی_مشترکہ_الٹ_بہاو}-الف  میں بائیں لچھے کے بہاو سے دائیں لچھے کا بہاو تفریق کرنے سے قالب میں کل بہاو  \عددی{\phi} حاصل ہوتا ہے۔لچھوں میں رو کی سمت، قالب کے گرد چکر کی سمت اور قالب میں بہاو کی سمت کو نہایت عمدگی سے نقطوں کی مدد سے ظاہر کیا جاتا ہے۔شکل \حوالہ{شکل_مقناطیسی_مشترکہ_امالہ_ب}-ب اور شکل \حوالہ{شکل_مقناطیسی_مشترکہ_الٹ_بہاو}-ب میں ان نقطوں کا استعمال دکھایا گیا ہے۔

انفرادی لچھے کی رو اور دباو کو انفعالی رائج سمت کے تحت چننیں۔دونوں لچھوں میں نقطوں والے سر سے رو داخل ہونے کی صورت میں دباو کا مشترک جزو مثبت لکھا جاتا ہے جبکہ ایک لچھے کی رو نقطے والے سر اور دوسرے لچھے کی رو بے نقطے والے سر سے داخل ہونے کی صورت میں مشترک دباو منفی لکھا جاتا ہے۔دونوں رو بے نقطے سروں سے داخل ہونے کی صورت میں مشترک دباو  مثبت لکھا جائے گا۔دباو کا خود جزو تمام صورتوں میں انفعالی رائج سمت کے تحت مثبت لکھا جاتا ہے۔یوں شکل \حوالہ{شکل_مقناطیسی_مشترکہ_امالہ_ب}  میں  مساوات \حوالہ{مساوات_مقناطیسی_مشترک_لچھے_دباو_الف} اور مساوات \حوالہ{مساوات_مقناطیسی_مشترک_لچھے_دباو_ب} دباو دیں گے جبکہ شکل \حوالہ{شکل_مقناطیسی_مشترکہ_الٹ_بہاو} میں مساوات \حوالہ{مساوات_مقناطیسی_مشترک_لچھے_دباو_پ} اور مساوات \حوالہ{مساوات_مقناطیسی_مشترک_لچھے_دباو_ت} دباو دیں گے۔

مشترک امالہ کے کرخوف مساوات دباو نسبتاً زیادہ آسانی سے لکھے جاتے ہیں۔
%==============
\ابتدا{مثال}\شناخت{مثال_مقناطیسی_مشترک_امالہ_دباو_الف}
شکل \حوالہ{شکل_مقناطیسی_مشترک_امالہ_دباو_الف} میں دیے دور کے  دونوں اطراف کے دباو کے مساوات لکھیں۔
\begin{figure}
\centering
\begin{circuitikz}
\draw(0,0) rectangle ++(-\boxW,\boxH);
\draw(-0.25,\boxH/2) node[rotate=90]{\RL{بایاں دور}};
\draw(0,0.25) to [short]++(\x,0)coordinate(BL) to [inductor,l={$L_1$}]++(0,\y)coordinate(TL) to [short,i<_={$i_1$}]++(-\x,0);
\draw(\x+\x/3+\x,0.25) to [short]++(-\x,0)coordinate(BR) to [inductor,l_={$L_2$}]++(0,\y)coordinate(TR) to [short,i<^={$i_2$}]++(\x,0);
\draw($(TL)!0.5!(TR)$)node[above]{$M$};
\draw(BL)++(-0.5,0.5) node[circ]{}; 
\draw(BR)++(0.5,0.5) node[circ]{}; 
\draw(2*\x+\x/3,0) rectangle ++(\boxW,\boxH);
\draw(2*\x+\x/3,0)++(\boxW/2,\boxH/2) node[rotate=90]{\RL{دایاں دور}};
\draw(0,\boxH/2) node[right]{$\begin{aligned} &+ \\ &v_1 \\ &-  \end{aligned}$};
\draw(2*\x+\x/3,\boxH/2) node[left]{$\begin{aligned} &- \\ &v_2 \\ &+  \end{aligned}$};
\end{circuitikz}
\caption{مثال \حوالہ{مثال_مقناطیسی_مشترک_امالہ_دباو_الف} کا دور۔}
\label{شکل_مقناطیسی_مشترک_امالہ_دباو_الف}
\end{figure}

حل:بائیں جانب \عددی{v_1} اور \عددی{i_1} عین انفعالی رائج سمت کے تحت لکھے گئے ہیں۔یوں دباو کا خود جزو مثبت لکھا جائے گا۔دونوں لچھوں میں رو بے نقطے سروں سے داخل ہوتی ہے لہٰذا دباو کا مشترک جزو مثبت لکھا جائے گا۔یوں بائیں جانب کرخوف کی مساوات درج ذیل ہو گی۔
\begin{align*}
v_1=L_1 \frac{\dif i_1}{\dif t}+M\frac{\dif i_2}{\dif t}
\end{align*} 
دائیں جانب \عددی{v_2} اور \عددی{i_2} انفعالی رائج سمت کے تحت نہیں چننے گئے ہیں۔یوں دباو کے اجزاء لکھتے ہوئے اس کا خیال رکھا جائے گا۔دوسرے لچھے کی مساوات درج ذیل
\begin{align*}
-v_2=L_2 \frac{\dif i_2}{\dif t}+M\frac{\dif i_1}{\dif t}
\end{align*}
یعنی
\begin{align*}
v_2=-M\frac{\dif i_1}{\dif t}-L_2 \frac{\dif i_2}{\dif t}
\end{align*}
لکھی جائے گی۔ 
\انتہا{مثال}
%=======================
\ابتدا{مثال}\شناخت{مثال_مقناطیسی_مشترک_امالہ_دباو_ب}
شکل \حوالہ{شکل_مقناطیسی_مشترک_امالہ_دباو_ب} کے دور کے کرخوف مساوات دباو لکھیں۔

\begin{figure}
\centering
\begin{circuitikz}[american voltages]
\draw(0,0) to [american voltage source,l={$v_m$}]++(0,\y) to [resistor,l={$R_1$}]++(\x,0)coordinate(kUL) to [inductor,l_={$L_1$},v^<=$v_1$]++(\x,0) to [inductor,l={$L_2$},v>={$v_2$}]++(0,-\y)coordinate(kSB) to [short](0,0);
\draw(kSB) to [short,*-]++(\x,0) to [resistor,l_={$R_2$}]++(0,\y) to [short,-*]++(-\x,0);
\draw(kUL)++(0.5,-0.5) node[circ]{} ++(0.2,-0.2)coordinate(kA);
\draw(kSB)++(-0.5,0.5) node[circ]{} ++(-0.2,0.2)coordinate(kB);
\draw[stealth-stealth] (kA) to [out=-90,in=180] (kB);
\draw($(kA)!0.5!(kB)$)node[shift={(-135:0.5)}]{$M$};
%currents
\draw[stealth-] ([shift={(-135:\x/4)}]\x/2,\y/2) arc (-135:135:\x/4);
\draw(\x/2,\y/2)node{$i_1$};
\draw[stealth-] ([shift={(-135:\x/4)}]2*\x+\x/2,\y/2) arc (-135:135:\x/4);
\draw(2*\x+\x/2,\y/2)node{$i_2$};
\end{circuitikz}
\caption{مثال \حوالہ{مثال_مقناطیسی_مشترک_امالہ_دباو_ب} کا دور۔}
\label{شکل_مقناطیسی_مشترک_امالہ_دباو_ب}
\end{figure}

حل:مشترکہ امالہ کے انفرادی دباو کی نشاندہی \عددی{v_1} اور \عددی{v_2} سے کی گئی ہے جنہیں بالترتیب \عددی{i_1} اور \عددی{i_2} کو دیکھتے ہوئے انفعالی رائج سمت کے تحت چننا گیا ہے۔امالہ \عددی{L_1} کے دباو کے دو اجزاء ہیں۔اس کے خود جزو \عددی{L_1 \tfrac{\dif i_1}{\dif t}} ہے۔امالہ \عددی{L_2} میں رو امالہ \عددی{L_1} کے دباو کا مشترک جزو دیتی ہے۔امالہ \عددی{L_2} کے نقطے والے سر سے کل داخلی ہونے والی رو \عددی{i_2-i_1} لکھی جا سکتی ہے جو \عددی{L_1} کے نقطے والے سر پر مثبت دباو پیدا کرتی ہے۔یوں \عددی{L_1} کا مشترک جزو \عددی{M\tfrac{\dif }{\dif t}(i_2-i_1)} ہے۔اس طرح پہلے امالہ کے لئے درج ذیل لکھا جا سکتا ہے۔
\begin{align}\label{مساوات_مقناطیسی_مثال_مشترک_الف}
v_1=L_1 \frac{\dif i_1}{\dif t}+M\frac{\dif}{\dif t}(i_2-i_1)
\end{align}
امالہ \عددی{L_2} کا خود جزو \عددی{L_2\tfrac{\dif }{\dif t}(i_2-i_1)} ہے۔امالہ \عددی{L_1} کے نقطے والے سر سے \عددی{i_1} داخل ہوتا ہے جو امالہ \عددی{L_2} کے نقطے والے سر پر مثبت دباو پیدا کرے گا۔یوں \عددی{L_2} کے دباو کا مشترک جزو \عددی{M\tfrac{\dif i_1}{\dif t}} ہو گا۔یوں درج ذیل لکھا جا سکتا ہے۔
\begin{align}\label{مساوات_مقناطیسی_مثال_مشترک_ب}
v_2=L_2 \frac{\dif}{\dif t}(i_2-i_1)+M\frac{\dif i_1}{\dif t}
\end{align}
اب دور کو دیکھتے ہوئے کرخوف مساوات لکھتے ہیں۔
\begin{align}\label{مساوات_مقناطیسی_مثال_مشترک_پ}
v_m&=i_1 R_1+v_1-v_2\\
0&=v_2+i_2 R_2
\end{align}
ان میں مساوات \حوالہ{مساوات_مقناطیسی_مثال_مشترک_الف} اور مساوات  \حوالہ{مساوات_مقناطیسی_مثال_مشترک_ب} پر کرتے ہوئے جواب لکھتے ہیں۔
\begin{align}
v_m&=i_1 R_1+L_1 \frac{\dif i_1}{\dif t}+M\frac{\dif}{\dif t}(i_2-i_1)-L_2 \frac{\dif}{\dif t}(i_2-i_1)-M\frac{\dif i_1}{\dif t}\\
0&=L_2 \frac{\dif}{\dif t}(i_2-i_1)+M\frac{\dif i_1}{\dif t}+i_2 R_2
\end{align}
\انتہا{مثال}
%======================
%==============
\ابتدا{مشق}\شناخت{مشق_مقناطیسی_مشترک_امالہ_دباو_پ}
شکل \حوالہ{شکل_مقناطیسی_مشترک_امالہ_دباو_پ} میں دیے دور کے  دونوں اطراف کے دباو لکھیں۔
\begin{figure}
\centering
\begin{circuitikz}
\draw(0,0) rectangle ++(-\boxW,\boxH);
\draw(-0.25,\boxH/2) node[rotate=90]{\RL{بایاں دور}};
\draw(0,0.25) to [short]++(\x,0)coordinate(BL) to [inductor,l={$L_1$}]++(0,\y)coordinate(TL) to [short,i>_={$i_1$}]++(-\x,0);
\draw(\x+\x/3+\x,0.25) to [short]++(-\x,0)coordinate(BR) to [inductor,l_={$L_2$}]++(0,\y)coordinate(TR) to [short,i>^={$i_2$}]++(\x,0);
\draw($(TL)!0.5!(TR)$)node[above]{$M$};
\draw(TL)++(-0.5,-0.5) node[circ]{}; 
\draw(BR)++(0.5,0.5) node[circ]{}; 
\draw(2*\x+\x/3,0) rectangle ++(\boxW,\boxH);
\draw(2*\x+\x/3,0)++(\boxW/2,\boxH/2) node[rotate=90]{\RL{دایاں دور}};
\draw(0,\boxH/2) node[right]{$\begin{aligned} &+ \\ &v_1 \\ &-  \end{aligned}$};
\draw(2*\x+\x/3,\boxH/2) node[left]{$\begin{aligned} &- \\ &v_2 \\ &+  \end{aligned}$};
\end{circuitikz}
\caption{مشق \حوالہ{مشق_مقناطیسی_مشترک_امالہ_دباو_پ} کا دور۔}
\label{شکل_مقناطیسی_مشترک_امالہ_دباو_پ}
\end{figure}

جوابات:\عددی{v_1=-L_1\tfrac{\dif i_1}{\dif t}+M\tfrac{\dif i_2}{\dif t}}، \عددی{v_2=L_2 \tfrac{\dif i_2}{\dif t}-M\tfrac{\dif i_1}{\dif t}}
\انتہا{مشق}
%========================

شکل \حوالہ{شکل_مقناطیسی_وقتی_تعددی_دائرہ_کار}-الف میں وقتی دائرہ کار کا دور جبکہ شکل-ب میں اسی کو تعددی دائرہ کار کی صورت میں دکھایا گیا ہے۔ شکل-ب کے کرخوف مساوات درج ذیل ہیں۔
\begin{align*}
\hat{V}_1&=j\omega L_1 \hat{I}_1+j\omega M \hat{I}_2\\
\hat{V}_2&=j\omega M \hat{I}_1+j\omega L_2 \hat{I}_2
\end{align*}
%
\begin{figure}
\centering
\begin{subfigure}{0.5\textwidth}
\centering
\begin{circuitikz}
\draw(0,0) rectangle ++(-\boxW,\boxH);
\draw(0,0.25) to [short]++(\x,0)coordinate(BL) to [inductor,l={$L_1$}]++(0,\y)coordinate(TL) to [short,i<_={$i_1$}]++(-\x,0);
\draw(\x+\x/3+\x,0.25) to [short]++(-\x,0)coordinate(BR) to [inductor,l_={$L_2$}]++(0,\y)coordinate(TR) to [short,i<^={$i_2$}]++(\x,0);
\draw($(TL)!0.5!(TR)$)node[above]{$M$};
\draw(TL)++(-0.5,-0.5) node[circ]{}; 
\draw(TR)++(0.5,-0.5) node[circ]{}; 
\draw(2*\x+\x/3,0) rectangle ++(\boxW,\boxH);
\draw(0,\boxH/2) node[right]{$\begin{aligned} &+ \\ &v_1 \\ &-  \end{aligned}$};
\draw(2*\x+\x/3,\boxH/2) node[left]{$\begin{aligned} &+ \\ &v_2 \\ &-  \end{aligned}$};
\end{circuitikz}
\caption*{(الف)}
\end{subfigure}%
\begin{subfigure}{0.5\textwidth}
\centering
\begin{circuitikz}
\draw(0,0) rectangle ++(-\boxW,\boxH);
\draw(0,0.25) to [short]++(\x,0)coordinate(BL) to [inductor,l={$j \omega L_1$}]++(0,\y)coordinate(TL) to [short,i<_={$\hat{I}_1$}]++(-\x,0);
\draw(\x+\x/3+\x,0.25) to [short]++(-\x,0)coordinate(BR) to [inductor,l_={$j \omega L_2$}]++(0,\y)coordinate(TR) to [short,i<^={$\hat{I}_2$}]++(\x,0);
\draw($(TL)!0.5!(TR)$)node[above]{$j \omega M$};
\draw(TL)++(-0.5,-0.5) node[circ]{}; 
\draw(TR)++(0.5,-0.5) node[circ]{}; 
\draw(2*\x+\x/3,0) rectangle ++(\boxW,\boxH);
\draw(0,\boxH/2) node[right]{$\begin{aligned} &+ \\ &\hat{V}_1 \\ &-  \end{aligned}$};
\draw(2*\x+\x/3,\boxH/2) node[left]{$\begin{aligned} &+ \\ &\hat{V}_2 \\ &-  \end{aligned}$};
\end{circuitikz}
\caption*{(ب)}
\end{subfigure}%
\caption{وقتی دائرہ کار سے تعددی دائرہ کار کا حصول۔}
\label{شکل_مقناطیسی_وقتی_تعددی_دائرہ_کار}
\end{figure}

%====================
\ابتدا{مثال}
دو عدد مربوط لچھے چار مختلف طریقوں سے آپس میں جوڑے جا سکتے ہیں جنہیں شکل \حوالہ{شکل_مقناطیسی_مربوط_چار_ممکنات} میں دکھایا گیا ہے۔چاروں صورتوں میں ان کا مساوی امالہ حاصل کریں۔شکل میں ان مساوی امالہ \عددی{L_{\text{مساوی}}} کو بھی لکھا گیا ہے۔ 
\begin{figure}
\centering
\begin{subfigure}{0.5\textwidth}
\centering
\begin{tikzpicture}
\draw(0,0) to [american voltage source,l={$\hat{V}_m$}]++(-2*\x,0) to [short,i={$\hat{I}$}]++(0,\y)coordinate(kA) to [inductor,l={$j\omega L_1$}]++(\x,0)coordinate(kB) to [inductor,l={$j\omega L_2$}]++(\x,0)coordinate(kC) to [short]++(0,-\y);
\draw (kA)++(0.5,-0.5) node[circ]{}++(0.2,-0.2)coordinate(kD);
\draw (kB)++(0.5,-0.5) node[circ]{};
\draw[stealth-stealth](kD) to [out=-45,in=-135]++(\x-0.4,0);
\draw(kD)++(\x/2-0.2,-0.4)node[fill=white]{$j\omega M$};
\draw(-\x,\y+\y/2)node[above]{$L_{\text{مساوی}}=L_1+L_2+2M$};
\end{tikzpicture}
\caption*{(الف)}
\end{subfigure}%
\begin{subfigure}{0.5\textwidth}
\centering
\begin{tikzpicture}
\draw(0,0) to [american voltage source,l={$\hat{V}_m$}]++(-2*\x,0) to [short,i={$\hat{I}$}]++(0,\y)coordinate(kA) to [inductor,l={$j\omega L_1$}]++(\x,0)coordinate(kB) to [inductor,l={$j\omega L_2$}]++(\x,0)coordinate(kC) to [short]++(0,-\y);
\draw (kA)++(0.5,-0.5) node[circ]{}++(0.2,-0.2)coordinate(kD);
\draw (kC)++(-0.5,-0.5) node[circ]{}++(-0.2,-0.2)coordinate(kE);
\draw[stealth-stealth](kD) to [out=-35,in=-145](kE);
\draw($(kD)!0.5!(kE)$)++(0,-0.4)node[fill=white]{$j \omega M$};
\draw(-\x,\y+\y/2)node[above]{$L_{\text{مساوی}}=L_1+L_2-2M$};
\end{tikzpicture}
\caption*{(ب)}
\end{subfigure}
\begin{subfigure}{0.5\textwidth}
\centering
\begin{tikzpicture}
\draw(0,0) to [american voltage source,l={$\hat{V}_m$}]++(0,\y) to [short,i={$\hat{I}$}]++(3/4*\x,0)coordinate(kA) to [inductor,l={$j\omega L_1$},i={$\hat{I}_1$}]++(0,-\y)coordinate(kB) to [short](0,0);
\draw(3/4*\x,\y) to [short,*-]++(\x,0)coordinate(kC) to [inductor,l={$j\omega L_2$},i={$\hat{I}_2$}]++(0,-\y)coordinate(kD) to [short,-*]++(-\x,0);
\draw(kA)++(-0.5,-0.5) node[circ]{};
\draw(kC)++(-0.5,-0.5) node[circ]{};
\draw(3/4*\x+\x/2,\y)node[above]{$j\omega M$};
\draw(\x/2+3/8*\x,\y+\y/2)node[above]{$L_{\text{مساوی}}=\frac{L_1 L_2 -M^2}{L_1+L_2-2M}$};
\end{tikzpicture}
\caption*{(پ)}
\end{subfigure}%
\begin{subfigure}{0.5\textwidth}
\centering
\begin{tikzpicture}
\draw(0,0) to [american voltage source,l={$\hat{V}_m$}]++(0,\y) to [short,i={$\hat{I}$}]++(3/4*\x,0)coordinate(kA) to [inductor,l={$j\omega L_1$}]++(0,-\y)coordinate(kB) to [short](0,0);
\draw(3/4*\x,\y) to [short,*-]++(\x,0)coordinate(kC) to [inductor,l={$j\omega L_2$}]++(0,-\y)coordinate(kD) to [short,-*]++(-\x,0);
\draw(kA)++(-0.5,-0.5) node[circ]{};
\draw(kD)++(-0.5,0.5) node[circ]{};
\draw(3/4*\x+\x/2,\y)node[above]{$j\omega M$};
\draw(\x/2+3/8*\x,\y+\y/2)node[above]{$L_{\text{مساوی}}=\frac{L_1 L_2 -M^2}{L_1+L_2+2M}$};
\end{tikzpicture}
\caption*{(ت)}
\end{subfigure}%
\caption{دو مربوط لچھوں کے چار ممکنہ ادوار اور ان کا مساوی امالہ۔}
\label{شکل_مقناطیسی_مربوط_چار_ممکنات}
\end{figure}

حل:شکل \حوالہ{شکل_مقناطیسی_مربوط_چار_ممکنات}-الف کو دیکھتے ہوئے کرخوف مساوات دباو لکھتے ہیں
\begin{align*}
\hat{V}_m&=j\omega L_1 \hat{I}+j\omega M \hat{I}+j\omega L_2 \hat{I}+j\omega M \hat{I}\\
&=j\omega \hat{I}(L_1+L_2+2M)\\
&=j\omega \hat{I}L_{\text{\RL{مساوی}}} 
\end{align*}
جہاں آخری قدم پر قوسین میں بند جزو کو مساوی امالہ \عددی{L_\text{مساوی}} کہا گیا ہے۔
\begin{align}
L_{\text{مساوی}}=L_1+L_2+2M
\end{align}
شکل \حوالہ{شکل_مقناطیسی_مربوط_چار_ممکنات}-ب کو دیکھتے ہوئے کرخوف مساوات دباو لکھتے ہیں
\begin{align*}
\hat{V}_m&=j\omega L_1 \hat{I}-j\omega M \hat{I}+j\omega L_2 \hat{I}-j\omega M \hat{I}\\
&=j\omega \hat{I}(L_1+L_2-2M)\\
&=j\omega \hat{I}L_{\text{\RL{مساوی}}} 
\end{align*}
جہاں آخری قدم پر قوسین میں بند جزو کو مساوی امالہ \عددی{L_\text{مساوی}} کہا گیا ہے۔
\begin{align}
L_{\text{مساوی}}=L_1+L_2-2M
\end{align}
شکل \حوالہ{شکل_مقناطیسی_مربوط_چار_ممکنات}-پ کو دیکھتے ہوئے دونوں لچھوں کے مساوات لکھتے ہیں۔
\begin{align*}
\hat{V}_m&=j\omega L_1 \hat{I}_1+j\omega M \hat{I}_2\\
\hat{V}_m&=j\omega L_2 \hat{I}_2+j\omega M \hat{I}_1
\end{align*}
ان دو عدد ہمزاد مساوات کو حل کرتے ہوئے درج ذیل ملتا ہے۔
\begin{align*}
\hat{I}_1&=\frac{\hat{V}_m(L_2-M)}{j\omega (L_1 L_2-M^2)}\\
\hat{I}_2&=\frac{\hat{V}_m(L_1-M)}{j\omega (L_1 L_2-M^2)}
\end{align*}
کرخوف مساوات رو سے \عددی{\hat{I}=\hat{I}_1+\hat{I}_2} لکھا جا سکتا ہے جس میں درج بالا حاصل شدہ نتائج پر کرتے ہوئے  ترتیب دیتے ہیں
\begin{align*}
\hat{I}&=\hat{I}_1+\hat{I}_2\\
&=\frac{\hat{V}_m(L_1+L_2-M)}{j\omega (L_1 L_2-M^2)}\\
&=\frac{\hat{V}_m}{j\omega L_{\text{مساوی}}}
\end{align*}
جہاں آخری قدم پر مساوی امالہ کی نشاندہی کی گئی ہے یعنی
\begin{align}
L_{\text{مساوی}}=\frac{L_1 L_2-M^2}{L_1+L_2-2M}
\end{align}
\انتہا{مثال}
%======================
\ابتدا{مشق}
شکل \حوالہ{شکل_مقناطیسی_مربوط_چار_ممکنات}-ت میں دیے دور کا مساوی امالہ دریافت کریں۔

جواب:
\begin{align}
L_{\text{مساوی}}=\frac{L_1 L_2-M^2}{L_1+L_2+2M}
\end{align}
\انتہا{مشق}
%=====================
\ابتدا{مثال}\شناخت{مثال_مقناطیسی_مربوط_دور_الف}
شکل \حوالہ{شکل_مقناطیسی_مربوط_دور_الف} میں \عددی{\hat{V}_0} دریافت کریں۔
\begin{figure}
\centering
\begin{tikzpicture}[american voltages]
\draw(0,0) to [american voltage source,l={$30\phase{45^{\circ}}$}\,\si{\volt}]++(0,\y) to [resistor,l={$\SI{2}{\ohm}$}]++(2*\x,0)coordinate(kA) to [inductor,l_={$j2\,\si{\ohm}$}]++(0,-\y)coordinate(kB) to [short] (0,0);
\draw(2*\x+\x/3,0)coordinate(kC) to [inductor,l_={$j4\,\si{\ohm}$}]++(0,\y)coordinate(kD) to [resistor,l={$\SI{4}{\ohm}$}]++(2*\x,0) to [inductor,l={$j6\,\si{\ohm}$},v={$\hat{V}_0$}]++(0,-\y) to [short]++(-2*\x,0);
\draw (kA)++(-0.5,-0.5) node[circ]{};
\draw (kD)++(0.5,-0.5) node[circ]{};
\draw(2*\x+\x/6,\y)node[above]{$j3\,\si{\ohm}$};
%currents
\draw[stealth-]([shift={(-135:\x/4)}]\x,\y/2) arc (-135:135:\x/4);
\draw(\x,\y/2)node{$\hat{I}_1$};
\draw[stealth-]([shift={(-135:\x/4)}]2*\x+\x/3+\x,\y/2) arc (-135:135:\x/4);
\draw(3*\x+\x/3,\y/2)node{$\hat{I}_2$};
\end{tikzpicture}
\caption{مثال \حوالہ{مثال_مقناطیسی_مربوط_دور_الف} کا دور۔}
\label{شکل_مقناطیسی_مربوط_دور_الف}
\end{figure}

حل:کرخوف مساوات لکھتے ہیں۔
\begin{align*}
30\phase{45^{\circ}}&=(2+j2)\hat{I}_1-j3\hat{I}_2\\
0&=-j3\hat{I}_1+(j4+4+j6)\hat{I}_2
\end{align*}
ان ہمزاد مساوات کو حل کرنے سے درج ذیل ملتا ہے۔
\begin{align*}
\hat{I}_1&=11.474\phase{17.08^{\circ}} \, \si{\ampere}\\
\hat{I}_2&=3.196\phase{38.88^{\circ}}\,\si{\ampere}
\end{align*}
رو \عددی{\hat{I}_2}  کو استعمال کرتے ہوئے خارجی دباو حاصل کرتے ہیں۔
\begin{align*}
\hat{V}_0&=(j6)(\hat{I}_2)=(6\phase{90^{\circ}})(3.196\phase{38.88^{\circ}})=19.176\phase{128.88^{\circ}}\,\si{\volt}
\end{align*}
\انتہا{مثال}
%=====================
\ابتدا{مثال}\شناخت{مثال_مقناطیسی_مربوط_دور_ب}
شکل \حوالہ{شکل_مقناطیسی_مربوط_دور_ب} کر دائری کرخوف مساوات لکھیں۔بعض اوقات دور میں دو عدد سے زیادہ مربوط امالہ موجود ہوتے ہیں۔ایسی صورت میں تیر کے لکیروں سے دو دو امالہ کی نشاندہی کی جاتی ہے۔اس شکل میں \عددی{L_1} اور \عددی{L_2} کے تعلق \عددی{j\omega M} کی نشاندہی کی گئی ہے۔
\begin{figure}
\centering
\begin{tikzpicture}
\draw(0,0) to [american voltage source,l={$\hat{V}_m$}]++(0,2*\y) to [capacitor,l={$\frac{1}{j\omega C_1}$}]++(\x,0) to [resistor,l={$R_1$}]++(\x,0) to [resistor,l={$R_2$}]++(\x,0) to [capacitor,l={$\frac{1}{j\omega C_3}$}]++(0,-2*\y) to [short] (0,0);
\draw(\x,0) to [resistor,*-,l={$R_3$}]++(0,\y) to [inductor,-*,l={$j\omega L_1$}]++(0,\y)coordinate(kA);
\draw(2*\x,0) to [capacitor,*-,l_={$\frac{1}{j\omega C_2}$}]++(0,\y) to [inductor,-*,l_={$j\omega L_2$}]++(0,\y)coordinate(kB);
\draw(kA)++(-0.5,-0.5) node[circ]{};
\draw(kB)++(0.5,-0.5) node[circ]{};
%mutual
\draw[-stealth](\x+\x/2,2*\y-0.7)--++(-0.5,-0.5);
\draw[-stealth](\x+\x/2,2*\y-0.7)--++(0.5,-0.5);
\draw(\x+\x/2,2*\y-0.7)node[fill=white]{$j\omega M$};
%currents
\draw[stealth-]([shift={(-135:\x/5)}]\x/2,\y) arc (-135:135:\x/5);
\draw(\x/2,\y)node{$\hat{I}_1$};
\draw[stealth-]([shift={(-135:\x/5)}]\x+\x/2,\y) arc (-135:135:\x/5);
\draw(\x+\x/2,\y)node{$\hat{I}_2$};
\draw[stealth-]([shift={(-135:\x/5)}]2*\x+\x/2,\y) arc (-135:135:\x/5);
\draw(2*\x+\x/2,\y)node{$\hat{I}_3$};
\end{tikzpicture}
\caption{مثال \حوالہ{مثال_مقناطیسی_مربوط_دور_ب} کا دور۔}
\label{شکل_مقناطیسی_مربوط_دور_ب}
\end{figure}

حل:کرخوف مساوات لکھتے ہوئے محتاط اور چوکس رہیں۔تین خانوں کے مساوات درج ذیل ہیں۔
\begin{equation*}
 \begin{split}
\hat{V}_m&=\frac{\hat{I}_1}{j\omega C_1}+j\omega L_1 (\hat{I}_1-\hat{I}_2)+R_3(\hat{I}_1-\hat{I}_2)+j\omega M(\hat{I}_2-\hat{I}_3)\\
0&=R_3(\hat{I}_2-\hat{I}_1)+j\omega L_1(\hat{I}_2-\hat{I}_1)+R_1 I_2+j\omega L_2(\hat{I}_2-\hat{I}_3)\\
&\hspace{2cm} +\frac{1}{j\omega C_2}(\hat{I}_2-\hat{I}_3)-j\omega M(\hat{I}_2-\hat{I}_3)+j\omega M(\hat{I}_1-\hat{I}_2) \\
0&=\frac{\hat{I}_3}{j\omega C_3}+j\omega L_2(\hat{I}_3-\hat{I}_2)+R_2 \hat{I}_3+\frac{\hat{I}_3}{j\omega C_3}-j\omega M(\hat{I}_1-\hat{I}_2)
\end{split}
\end{equation*}
انہیں ترتیب دیتے ہوئے دوبارہ لکھتے ہیں۔ترتیب دینے سے متشاکل مساوات حاصل ہوتے ہیں۔
\begin{equation*}
\begin{split}
\left(\frac{1}{j\omega C_1}+j\omega L_1+R_3\right)\hat{I}_1-\left(j\omega L_2+R_3-j\omega M\right)\hat{I}_2-j\omega M \hat{I}_3&=\hat{V}_m\\
-\left(j\omega L_1+R_3-j\omega M\right)\hat{I}_1+\left(R_3+j\omega L_1+R_1+j\omega L_2+\frac{1}{j\omega C_2}-2j\omega M\right)\hat{I}_2 &\\
-\left(\frac{1}{j\omega C_2+j\omega L_2+R_2+\frac{1}{j\omega C_3}}-j\omega M\right)\hat{I}_3&=0\\
-j\omega M \hat{I}_1-\left(j\omega L_2+\frac{1}{j\omega C_2}-j\omega M\right)\hat{I}_2+\left(\frac{1}{j\omega C_2}+j\omega L_2+R_2+\frac{1}{j\omega C_3}\right)\hat{I}_3&=0
\end{split}
\end{equation*}
\انتہا{مثال}
%=====================
\ابتدا{مشق}\شناخت{مشق_مقناطیسی_مشق_مربوط_دور_الف}
شکل \حوالہ{شکل_مقناطیسی_مشق_مربوط_دور_الف} میں \عددی{\hat{I}_1}، \عددی{\hat{I}_2} اور \عددی{\hat{V}_0} دریافت کریں۔

\begin{figure}
\centering
\begin{tikzpicture}[american voltages]
\draw(0,0) to [american voltage source,l={$20\phase{0^{\circ}}$}\,\si{\volt}]++(0,\y) to [capacitor,l={$-j1\,\si{\ohm}$}]++(2*\x,0)coordinate(kA) to [inductor,l_={$j4\,\si{\ohm}$}]++(0,-\y)coordinate(kB) to [short] (0,0);
\draw(2*\x+\x/3,0)coordinate(kC) to [inductor,l_={$j6\,\si{\ohm}$}]++(0,\y)coordinate(kD) to [resistor,l={$\SI{2}{\ohm}$}]++(2*\x,0) to [capacitor,l={$-j2\,\si{\ohm}$},v={$\hat{V}_0$}]++(0,-\y) to [short]++(-2*\x,0);
%dots
\draw (kA)++(-0.5,-0.5) node[circ]{};
\draw (kC)++(0.5,0.5) node[circ]{};
\draw(2*\x+\x/6,\y)node[above]{$j2\,\si{\ohm}$};
%currents
\draw[stealth-]([shift={(-135:\x/4)}]\x,\y/2) arc (-135:135:\x/4);
\draw(\x,\y/2)node{$\hat{I}_1$};
\draw[stealth-]([shift={(-135:\x/4)}]2*\x+\x/3+\x,\y/2) arc (-135:135:\x/4);
\draw(3*\x+\x/3,\y/2)node{$\hat{I}_2$};
\end{tikzpicture}
\caption{مشق \حوالہ{مشق_مقناطیسی_مشق_مربوط_دور_الف} کا دور۔}
\label{شکل_مقناطیسی_مشق_مربوط_دور_الف}
\end{figure}

جوابات:\عددی{\hat{I}_1=8.9\phase{-79.7^{\circ}}\,\si{\ampere}}، \عددی{\hat{I}_2=4\phase{126.9^{\circ}}\,\si{\ampere}}، \عددی{\hat{V}_0=8\phase{36.9^{\circ}}\,\si{\volt}}
\انتہا{مشق}
%================
\ابتدا{مشق}\شناخت{مشق_مقناطیسی_مشق_مربوط_دور_ب}
شکل \حوالہ{شکل_مقناطیسی_مشق_مربوط_دور_ب} کے کرخوف مساوات لکھیں۔
\begin{figure}
\centering
\begin{tikzpicture}
\draw(0,0) to [american voltage source,l={$\hat{V}_1$}]++(0,\y) to [resistor,l={$R_1$}]++(\x,0) to [inductor,l={$j\omega L_1$}]++(\x,0)coordinate(kA) to [inductor,l={$j\omega L_2$}]++(\x,0) to [resistor,l={$R_2$}]++(\x,0);
\draw(0,0) to [short]++(4*\x,0) to [american voltage source,l_={$\hat{V}_2$}] ++(0,\y);
\draw(2*\x,0) to [capacitor,*-*,l={$\frac{1}{j\omega C_1}$}]++(0,\y);
%dots
\draw(kA)++(-0.5,-0.25) node[circ]{};
\draw(kA)++(0.5,-0.25) node[circ]{};
%mutual
\draw[-stealth](2*\x,\y+0.8)--++(-0.5,-0.5);
\draw[-stealth](2*\x,\y+0.8)--++(0.5,-0.5);
\draw(2*\x,\y+0.8)node[fill=white]{$j\omega M$};
%currents
\draw[stealth-]([shift={(-135:\x/5)}]\x,\y/2) arc (-135:135:\x/5);
\draw(\x,\y/2)node{$\hat{I}_1$};
\draw[stealth-]([shift={(-135:\x/5)}]3*\x,\y/2) arc (-135:135:\x/5);
\draw(3*\x,\y/2)node{$\hat{I}_2$};
\end{tikzpicture}
\caption{مشق \حوالہ{مشق_مقناطیسی_مشق_مربوط_دور_ب} کا دور۔}
\label{شکل_مقناطیسی_مشق_مربوط_دور_ب}
\end{figure}

جوابات:
\begin{align*}
\left(R_1+j\omega L_1+\frac{1}{j\omega C_1}\right)\hat{I}_1-\left(\frac{1}{j\omega C_1}+j\omega M \right)\hat{I}_2&=\hat{V}_1\\
-\left(\frac{1}{j\omega C_1}+j\omega M\right)\hat{I}_1+\left(\frac{1}{j\omega C_1}+j\omega L_2+R_2\right)\hat{I}_2&=-\hat{V}_2
\end{align*}
\انتہا{مشق}
%================
\ابتدا{مشق}\شناخت{مشق_مقناطیسی_مشق_مربوط_دور_پ}
شکل \حوالہ{شکل_مقناطیسی_مشق_مربوط_دور_پ} میں \عددی{\hat{I}_1} اور \عددی{\hat{I}_2} معلوم کرتے ہوئے \عددی{\hat{V}_0} دریافت کریں جہاں تیر والے لکیر سے ان نقطوں کی نشاندہی کی گئی ہے جن کے مابین دباو درکار ہے۔تیر والا سر مثبت دباو کے مقام کی نشاندہی کرتا ہے۔یوں \عددی{j8\,\si{\ohm}} امالہ کا نچلی سرا حوالہ لیتے ہوئے \عددی{-j6\,\si{\ohm}} برق گیر کے بائیں سر پر دباو حاصل کرنا درکار ہے۔
\begin{figure}
\centering
\begin{tikzpicture}
\draw(0,0) to [american voltage source,l={$20\phase{-30^{\circ}}\,\si{\volt}$}]++(0,\y) to [resistor,l={$\SI{2}{\ohm}$}]++(\x,0) to [inductor,l={$j2 \,\si{\ohm}$}]++(\x,0)coordinate(kA) to [inductor,l_={$j4\,\si{\ohm}$}]++(0,-\y) to [capacitor,l={$-j1\,\si{\ohm}$}]++(-2*\x,0);
\draw(2*\x+\x/3,0) to [american voltage source,l_={$40\phase{60^{\circ}}$}] ++(2*\x,0)coordinate(kN) to [inductor,l_={$j8 \,\si{\ohm}$}]++(0,\y) to [capacitor,l_={$-j6\,\si{\ohm}$}]++(-\x,0)coordinate(kP) to [resistor,l_={$\SI{4}{\ohm}$}]++(-\x,0)coordinate(kB) to [inductor,l={$j2\,\si{\ohm}$}]++(0,-\y);
%dots
\draw(kA)++(-0.5,-0.5) node[circ]{};
\draw(kB)++(0.5,-0.5) node[circ]{};
%mutual
\draw[-stealth](2*\x+\x/6,\y+0.8)--++(-0.5,-0.5);
\draw[-stealth](2*\x+\x/6,\y+0.8)--++(0.5,-0.5);
\draw(2*\x+\x/6,\y+0.88)node[fill=white]{$j 2.5 \,\si{\ohm}$};
%currents
\draw[stealth-]([shift={(-135:\x/5)}]\x,\y/2) arc (-135:135:\x/5);
\draw(\x,\y/2)node{$\hat{I}_1$};
\draw[stealth-]([shift={(-135:\x/5)}]3*\x,\y/2) arc (-135:135:\x/5);
\draw(3*\x,\y/2)node{$\hat{I}_2$};
%output voltage
\draw(kP) to [open,v={$\hat{V}_0$}](kN);
\end{tikzpicture}
\caption{مشق \حوالہ{مشق_مقناطیسی_مشق_مربوط_دور_پ} کا دور۔}
\label{شکل_مقناطیسی_مشق_مربوط_دور_پ}
\end{figure}

جوابات:\عددی{6.89\phase{252.3^{\circ}}\,\si{\ampere}}، \عددی{7.08\phase{219.9^{\circ}}\,\si{\si{\ampere}}}، \عددی{14.15\phase{-50.1^{\circ}}\,\si{\volt}}
\انتہا{مشق}
%=======================
\ابتدا{مشق}\شناخت{مشق_مقناطیسی_مشق_مربوط_دور_ت}
شکل \حوالہ{شکل_مقناطیسی_مشق_مربوط_دور_ت} میں  بائیں اور دائیں دائروں کی رو حاصل کرتے ہوئے \عددی{\hat{V}_0} دریافت کریں-دباو حاصل کرتے ہوئے دباو کا مشترک جزو شامل کرنا مت بھولیں۔
\begin{figure}
\centering
\begin{tikzpicture}
\draw(0,0) to [american voltage source,l={$60\phase{20^{\circ}}\,\si{\volt}$}]++(0,2*\y) to [inductor,l={$j4 \,\si{\ohm}$}]++(\x,0) to [resistor,l={$\SI{4}{\ohm}$}]++(\x,0) to [capacitor,l={$-j6 \, \si{\ohm}$}]++(\x,0) to [resistor,l={$\SI{2}{\ohm}$}]++(0,-\y) to [inductor,l={$j8\,\si{\ohm}$}]++(0,-\y)coordinate(kA) to [short] (0,0);
\draw(\x,0) to [capacitor,*-,l={$-j6 \,\si{\ohm}$}] ++(0,\y) to [inductor,-*,l={$j8\,\si{\ohm}$}]++(0,\y)coordinate(kB);
%dots
\draw(kA)++(-0.5,0.5) node[circ]{};
\draw(kB)++(0.5,-0.5) node[circ]{};
%mutual
\draw(kA)++(-0.5,\y/2)coordinate(kC);
\draw(kB)++(0.5,-\y/2)coordinate(kD);
\draw[stealth-stealth](kC)--(kD)node[pos=0.5,fill=white]{$j 3 \,\si{\ohm}$};
%output voltage
\draw(3*\x,2*\y) to [short,*-o]++(\x,0)coordinate(kLT);
\draw(3*\x,0) to [short,*-o]++(\x,0)coordinate(kLB);
\draw(kLB) to [open,v_>={$\hat{V}_0$}](kLT);
\end{tikzpicture}
\caption{مشق \حوالہ{مشق_مقناطیسی_مشق_مربوط_دور_ت} کا دور۔}
\label{شکل_مقناطیسی_مشق_مربوط_دور_ت}
\end{figure}

جوابات:\عددی{13.9\phase{-55.2^{\circ}}\,\si{\ampere}}، \عددی{5.97\phase{-24.2^{\circ}}\,\si{\ampere}}،\عددی{31.4\phase{83.55^{\circ}}\,\si{\volt}} 
\انتہا{مشق}
%=======================

\ابتدا{مشق}\شناخت{مشق_مقناطیسی_مشق_مربوط_دور_ٹ}
شکل \حوالہ{شکل_مقناطیسی_مشق_مربوط_دور_ٹ} میں   \عددی{\hat{V}_{ab}} دریافت کریں-دونوں امالہ کے دباو کے مشترک جزو شامل کرنا مت بھولیں۔
\begin{figure}
\centering
\begin{tikzpicture}
\draw(0,0) to [american voltage source,l={$10\phase{0^{\circ}}\,\si{\volt}$}]++(0,\y) to [resistor,l={$\SI{4}{\ohm}$}]++(\x,0)coordinate(kA) to [inductor,l={$j6\,\si{\ohm}$}]++(\x,0) coordinate(kB) to [inductor,l={$j4 \,\si{\ohm}$}]++(\x,0) coordinate(kC)to [capacitor,l={$-j2 \, \si{\ohm}$}]++(0,-\y)to [short] (0,0);
\draw(2*\x,0) to [resistor,*-*,l={$\SI{8}{\ohm}$}]++(0,\y);
\draw(kA)node[above]{$a$};
\draw(kC)node[above]{$b$};
%dots
\draw(kA)++(0.5,0.5) node[circ]{};
\draw(kB)++(0.5,0.5) node[circ]{};
%mutual
\draw[-stealth](2*\x,\y+0.8)--++(-0.5,-0.5);
\draw[-stealth](2*\x,\y+0.8)--++(0.5,-0.5);
\draw(2*\x,\y)++(0,0.9)node[fill=white]{$j2 \,\si{\ohm}$};
%output voltage
\end{tikzpicture}
\caption{مشق \حوالہ{مشق_مقناطیسی_مشق_مربوط_دور_ٹ} کا دور۔}
\label{شکل_مقناطیسی_مشق_مربوط_دور_ٹ}
\end{figure}

جواب:\عددی{10.5\phase{15^{\circ}}\,\si{\volt}}
\انتہا{مشق}
%=======================
\ابتدا{مثال}\شناخت{مثال_مقناطیسی_داخلی_رکاوٹ_مشترک_الف}
شکل \حوالہ{شکل_مقناطیسی_داخلی_رکاوٹ_مشترک_الف} میں منبع دباو کو نظر آنے والا داخلی رکاوٹ \عددی{\bZ_{\text{داخلی}}} دریافت کریں۔
\begin{figure}
\centering
\begin{tikzpicture}
\draw(0,0) to [american voltage source,l={$\hat{V}_m$}]++(0,\y) to [european resistor,i>^={$\hat{I}_1$},l={$\bZ_1$}]++(2*\x,0)coordinate(kA) to [inductor,l_={$j\omega L_1$}]++(0,-\y) to [short]++(-2*\x,0);
\draw(2*\x+\x/3,0) to [inductor,l_={$j\omega L_2$}]++(0,\y)coordinate(kB) to [short,i>^={$\hat{I}_2$}] ++(2*\x,0) to [european resistor,l={$\bZ_2$}]++(0,-\y) to [short] ++(-2*\x,0);
\draw (kA)++(-0.5,-0.5) node[circ]{};
\draw (kB)++(0.5,-0.5) node[circ]{};
\draw(2*\x+\x/6,\y+0.5) node{$j\omega M$};
%text
\draw[stealth-] (\x/2,\y/2)--++(-\x/8,0)--++(0,-\y/8)node[below]{$\bZ_{\text{داخلی}}$};
\end{tikzpicture}
\caption{مثال \حوالہ{مثال_مقناطیسی_داخلی_رکاوٹ_مشترک_الف} کا دور۔}
\label{شکل_مقناطیسی_داخلی_رکاوٹ_مشترک_الف}
\end{figure}

حل:رو \عددی{\hat{I}_1} دریافت کرتے ہوئے  رکاوٹ کو \عددی{\tfrac{\hat{V}_m}{\hat{I}_1}} سے حاصل کیا جائے گا۔دونوں دائروں کے کرخوف مساوات لکھتے ہیں۔
\begin{align*}
\hat{V}_m&=(\bZ_1+j\omega L_1)\hat{I}_1-j\omega M \hat{I}_2\\
0&=-j\omega M \hat{I}_1+(j\omega L_2+\bZ_2)\hat{I}_2
\end{align*}
دوسری مساوات سے \عددی{\hat{I}_2} حاصل کرتے ہوئے
\begin{align*}
\hat{I}_2=\frac{j\omega M}{j\omega L_2+\bZ_2} \hat{I}_1
\end{align*}
اس کو بائیں دائرے کی کرخوف مساوات میں پر کرتے ہیں
\begin{align*}
\hat{V}_m&=(\bZ_1+j\omega L_1)\hat{I}_1-j\omega M \frac{j\omega M}{j\omega L_2+\bZ_2} \hat{I}_1
\end{align*}
جہاں سے داخلی رکاوٹ درج ذیل لکھی جا سکتی ہے۔
\begin{align*}
\bZ_{\text{داخلی}}=\frac{\hat{V}_m}{\hat{I}_1}=\bZ_1+j\omega L_1+ \frac{\omega^2 M^2}{j\omega L_2+\bZ_2}
\end{align*}
\انتہا{مثال}
%=========================

\ابتدا{مشق}\شناخت{مشق_مقناطیسی_داخلی_رکاوٹ_مشترک_ب}
درج بالا مثال کے دور میں مشترکہ امالہ پر ایک نقطے کا مقام تبدیل کرتے ہوئے شکل \حوالہ{شکل_مقناطیسی_داخلی_رکاوٹ_مشترک_ب} حاصل کیا گیا ہے۔اس میں منبع دباو کو نظر آنے والا داخلی رکاوٹ \عددی{\bZ_{\text{داخلی}}} دریافت کریں۔
\begin{figure}
\centering
\begin{tikzpicture}
\draw(0,0) to [american voltage source,l={$\hat{V}_m$}]++(0,\y) to [european resistor,i>^={$\hat{I}_1$},l={$\bZ_1$}]++(2*\x,0) to [inductor,l_={$j\omega L_1$}]++(0,-\y)coordinate(kA) to [short]++(-2*\x,0);
\draw(2*\x+\x/3,0) to [inductor,l_={$j\omega L_2$}]++(0,\y)coordinate(kB) to [short,i>^={$\hat{I}_2$}] ++(2*\x,0) to [european resistor,l={$\bZ_2$}]++(0,-\y) to [short] ++(-2*\x,0);
\draw (kA)++(-0.5,0.5) node[circ]{};
\draw (kB)++(0.5,-0.5) node[circ]{};
\draw(2*\x+\x/6,\y+0.5) node{$j\omega M$};
%text
\draw[stealth-] (\x/2,\y/2)--++(-\x/8,0)--++(0,-\y/8)node[below]{$\bZ_{\text{داخلی}}$};
\end{tikzpicture}
\caption{مشق \حوالہ{مشق_مقناطیسی_داخلی_رکاوٹ_مشترک_ب} کا دور۔}
\label{شکل_مقناطیسی_داخلی_رکاوٹ_مشترک_ب}
\end{figure}

جواب:
\begin{align*}
\bZ_{\text{داخلی}}=\frac{\hat{V}_m}{\hat{I}_1}=\bZ_1+j\omega L_1+ \frac{\omega^2 M^2}{j\omega L_2+\bZ_2}
\end{align*}
آپ نے دیکھا کہ اس دور میں نقطے کا مقام تبدیل کرنے سے داخلی رکاوٹ تبدیل نہیں ہوتا۔
\انتہا{مشق}
%===========================
\ابتدا{مشق}
شکل \حوالہ{شکل_مقناطیسی_مشق_مربوط_دور_ٹ} میں منبع دباو کو کیا رکاوٹ نظر آتا ہے۔

جواب:\عددی{5.88+j11.53\,\si{\ohm}}
\انتہا{مشق}
%==================

\حصہ{مشترکہ امالہ میں توانائی کا ذخیرہ}
شکل \حوالہ{شکل_مقناطیسی_مشترک_امالہ_ذخیرہ-توانائی} کو دیکھیے۔رو مقناطیسی میدان پیدا کرتی ہے۔رو کی غیر موجودگی میں اس دور میں مقناطیسی بہاو نہیں پایا جائے گا۔یوں اس میں ذخیرہ مقناطیسی توانائی بھی صفر کے برابر ہو گی۔اب تصور کریں کہ دایاں لچھا کھلے سر رکھتے ہوئے بائیں لچھے کی رو \عددی{t_1} دورانیے میں \عددی{I_1} کر دی جاتی ہے۔اس دورانیے کے دوران بائیں لچھے کو درج ذیل توانائی فراہم کی جائے گی۔
\begin{figure}
\centering
\begin{circuitikz}
\draw(0,0) rectangle ++(-\boxW,\boxH);
\draw(0,0.25) to [short]++(\x,0)coordinate(BL) to [inductor,l={$L_1$}]++(0,\y)coordinate(TL) to [short,i<_={$i_1$}]++(-\x,0);
\draw(\x+\x/3+\x,0.25) to [short]++(-\x,0)coordinate(BR) to [inductor,l_={$L_2$}]++(0,\y)coordinate(TR) to [short,i<^={$i_2$}]++(\x,0);
\draw($(TL)!0.5!(TR)$)node[above]{$M$};
\draw(TL)++(-0.5,-0.5) node[circ]{}; 
\draw(TR)++(0.5,-0.5) node[circ]{}; 
\draw(2*\x+\x/3,0) rectangle ++(\boxW,\boxH);
\draw(0,\boxH/2) node[right]{$\begin{aligned} &+ \\ &v_1 \\ &-  \end{aligned}$};
\draw(2*\x+\x/3,\boxH/2) node[left]{$\begin{aligned} &+ \\ &v_2 \\ &-  \end{aligned}$};
\end{circuitikz}
\caption{مشترکہ امالہ میں ذخیرہ توانائی۔}
\label{شکل_مقناطیسی_مشترک_امالہ_ذخیرہ-توانائی}
\end{figure}
%
\begin{align*}
\int_{0}^{t_1}v_1(t) i_1(t)\dif t=\int_{0}^{t_1} \left[L_1 \frac{\dif i_1(t)}{\dif t}\right] i_1(t) \dif t=\int_{0}^{I_1} L_1 i_1 \dif i_1=\frac{L_1 I_1^2}{2}
\end{align*}
اس دوران دائیں لچھے کی رو صفر کے برابر ہے لہٰذا \عددی{t_1} کے دوران دائیں لچھے کو کوئی توانائی فراہم نہیں کی جاتی۔اب فرض کریں کہ بائیں لچھے کی رو اسی قیمت پر رکھی جاتی ہے جبکہ دائیں لچھے کی رو \عددی{t_1} تا \عددی{t_2} بڑھا کر \عددی{I_2} کر دی جاتی ہے۔چونکہ  \عددی{t_1} تا \عددی{t_2} بائیں لچھے کی رو تبدیل نہیں ہو رہی  ہے لہٰذا دائیں لچھے کے دباو میں مشترک جزو صفر کے برابر ہو گا۔یوں دائیں لچھے کا دباو \عددی{v_2=L_2\tfrac{\dif i_2}{\dif t}} لکھا جائے گا۔اس طرح دائیں لچھے کو درج ذیل توانائی فراہم کی جاتی ہے۔
\begin{align*}
\int_{t_1}^{t_2}v_2(t) i_2(t)\dif t=\int_{t_1}^{t_2} \left[L_2 \frac{\dif i_2(t)}{\dif t}\right] i_2(t) \dif t=\int_{0}^{I_2} L_2 i_2 \dif i_2=\frac{L_2 I_2^2}{2}
\end{align*}
اسی دورانیے (\عددی{t_1} تا \عددی{t_2})  میں چونکہ دائیں لچھے کی رو تبدیل ہو رہی ہے (جبکہ \عددی{i_1=I_1} مستقل ہے) لہٰذا بائیں لچھے کے دباو میں مشترک جزو پایا جائے گا اور یوں اس کا دباو درج ذیل لکھا جائے گا
\begin{align*}
v_1=L_1 \frac{\dif i_1}{\dif t}+M \frac{\dif i_2}{\dif t}=M\frac{\dif i_2}{\dif t}
\end{align*}
جہاں \عددی{i_1} مستقل ہونے کی وجہ سے \عددی{\tfrac{\dif i_1}{\dif t}=0} ہے۔یوں \عددی{t_1} تا \عددی{t_2} کے دوران بائیں لچھے کو درج ذیل توانائی مہیا کی جاتی ہے۔
\begin{align*}
\int_{t_1}^{t_2} v_1(t) i(t)\dif t=\int_{t_1}^{t_2} \left[M \frac{\dif i_2(t)}{\dif t}\right] I_1 \dif t=\int_{0}^{I_2} M I_1 \dif i_2=M I_1 I_2
\end{align*}
ان تینوں جوابات کا مجموعہ لمحہ \عددی{t_2} تک مشترکہ امالہ کو فراہم کی گئی توانائی دیتا ہے۔
\begin{align}\label{مساوات_مقناطیسی_توانائی_مشترکہ_امالہ_الف}
w=\frac{L_1 I_1^2}{2}+\frac{L_2 I_2^2}{2}+MI_1 I_2
\end{align}
اگر ایک لچھے پر نقطے کا مقام تبدیل کرتے ہوئے جواب حاصل کیا جائے تب درج ذیل جواب حاصل ہوتا ہے۔
\begin{align}\label{مساوات_مقناطیسی_توانائی_مشترکہ_امالہ_ب}
w=\frac{L_1 I_1^2}{2}+\frac{L_2 I_2^2}{2}-MI_1 I_2
\end{align}
آپ نے دیکھا کہ ذخیرہ توانائی کا دارومدار رو پر ہے نا کہ \عددی{t_1} اور \عددی{t_2} پر۔یوں کسی بھی لمحے لچھوں کی رو \عددی{i_1(t)} اور \عددی{i_2(t)} لکھتے ہوئے اس لمحے ذخیرہ توانائی کو درج ذیل لکھا جا سکتا ہے۔
\begin{align}\label{مساوات_مقناطیسی_توانائی_مشترکہ_امالہ_پ}
w(t)=\frac{L_1 i^2_1(t)}{2}+\frac{L_2 i^2_2(t)}{2}\mp M i_1(t) i_2(t)
\end{align}
چونکہ مشترکہ امالہ غیر عامل پرزہ ہے لہٰذا یہ توانائی پیدا نہیں کرتا۔یوں اس کی توانائی کبھی بھی منفی نہیں ہو سکتی۔یوں درج بالا مساوات میں غیر ضروری معلومات نہ لکھتے ہوئے درج ذیل لکھتے ہیں
\begin{align}
w(t)&=\frac{L_1 i^2_1}{2}+\frac{L_2 i^2_2}{2}\mp M i_1 i_2
\end{align}
جس میں \عددی{\tfrac{M^2 i^2_1}{2L_2}} جمع اور منفی کر کے ترتیب دیتے ہوئے درج ذیل لکھا جا سکتا ہے۔
\begin{align}
w=\frac{1}{2} \left(L_1-\frac{M^2}{L_2}\right)i^2_1+\frac{L_2}{2}\left(i_2+\frac{M}{L_2 i_1}\right)^2
\end{align}
درج بالا مساوات کا دوسرا جزو مربع ہے لہٰذا یہ ہر صورت مثبت ہو گا۔چونکہ غیر عامل مشترکہ امالہ کی توانائی مثبت ہے لہٰذا اس مساوات کا پہلا جزو بھی مثبت ہو گا جس سے درج ذیل شرط حاصل ہوتا ہے۔
\begin{align}
M\le \sqrt{L_1 L_2}
\end{align}
یہ مساوات مشترکہ امالہ کی زیادہ سے زیادہ قیمت کا حد بیان کرتا ہے۔  یوں مشترکہ امالہ صفر تا \عددی{\sqrt{L_1 L_2}} ممکن ہے۔
\begin{align}
0\le M \le \sqrt{L_1 L_2}
\end{align}
 کسی بھی مشترکہ امالہ کو درج ذیل لکھا جا سکتا ہے
\begin{align}\label{مساوات_مقناطیسی_مشترک_امالہ_تعریف_الف}
M=k\sqrt{L_1 L_2}
\end{align}
جہاں \عددی{k} کو \اصطلاح{ارتباطی مستقل}\فرہنگ{ارتباطی مستقل}\حاشیہب{coupling coefficient}\فرہنگ{coupling coefficient} کہتے ہیں۔آپ دیکھ سکتے ہیں کہ ارتباطی مستقل صفر تا اکائی ممکن ہے۔
\begin{align}
0 \le k \le 1
\end{align}
ارتباطی مستقل کی تعریف درج ذیل مساوات دیتی ہے۔
\begin{align}
k=\frac{M}{\sqrt{L_1 L_2}}
\end{align}
ارتباطی مستقل یہ بتلاتا ہے کہ ایک لچھے کی کتنی بہاو دوسرے لچھے کے اندر سے گزرتی ہے۔اس باب کے شروع میں مشترکہ امالہ کے اشکال بناتے ہوئے ہم نے مقناطیسی قالب استعمال کیا۔مقناطیسی قالب کے استعمال سے ایک لچھے کی تقریباً تمام بہاو دوسرے لچھے سے بھی گزاری جا سکتی ہے۔ایسی صورت میں \عددی{k \approx 1} ہو گا۔اس کے برعکس ایک دونوں سے دور، قالب سے نہ جوڑے گئے لچھوں کی صورت میں \عددی{k=0} ہو گا چونکہ ایک لچھے کا بہاو دوسرے لچھے تک نہیں پہنچ پائے گا۔ارتباطی مستقل کی قیمت زیادہ \عددی{(k \ge 0.5)} ہونے کی صورت میں ہم کہتے ہیں کہ لچھوں کا \اصطلاح{رابطہ مضبوط}\فرہنگ{رابطہ!مضبوط}\حاشیہب{strongly coupled}\فرہنگ{coupling!strong} ہے جبکہ \عددی{k < 0.5} کی صورت میں ہم کہتے ہیں کہ لچھوں کا \اصطلاح{رابطہ کمزور}\فرہنگ{رابطہ!کمزور}\حاشیہب{weakly coupled}\فرہنگ{coupling!weak} ہے۔    
%========================
\ابتدا{مثال}\شناخت{مثال_مقناطیسی_ارتباطی_مستقل_الف}
شکل \حوالہ{شکل_مقناطیسی_ارتباطی_مستقل_الف}-الف میں \عددی{L_1=\SI{25}{\milli\henry}}، \عددی{L_2=\SI{12}{\milli\henry}} اور \عددی{k=1} ہیں۔لمحہ \عددی{t=\SI{6.2}{\milli\second}} پر مشترکہ امالہ میں ذخیرہ توانائی دریافت کریں۔
\begin{figure}
\centering
\begin{subfigure}{1\textwidth}
\centering
\begin{tikzpicture}
\draw(0,0) to [american voltage source,l={$20\cos 314t \, \si{\volt}$}]++(0,\y) to [resistor,l={$\SI{4}{\ohm}$}]++(2*\x,0)coordinate(kA) to [inductor,l_={$L_1$}]++(0,-\y) to [short] ++(-2*\x,0);
\draw(2*\x+\x/3,0) to [inductor,l_={$L_2$}]++(0,\y)coordinate(kB) to [short]++(2*\x,0) to [resistor,l={$\SI{2}{\ohm}$}]++(0,-\y) to [short]++(-2*\x,0);
%dots
\draw(kA)++(-0.5,-0.5) node[circ]{};
\draw(kB)++(0.5,-0.5) node[circ]{};
\draw(2*\x+\x/6,\y)node[above]{$M$};
%currents
\draw[stealth-]([shift={(-135:\x/4)}]\x,\y/2) arc (-135:135:\x/4);
\draw(\x,\y/2) node{$i_1(t)$};
\draw[stealth-]([shift={(-135:\x/4)}]2*\x+\x/3+\x,\y/2) arc (-135:135:\x/4);
\draw(2*\x+\x/3+\x,\y/2) node{$i_2(t)$};
\end{tikzpicture}
\caption*{(الف)}
\end{subfigure}
\begin{subfigure}{1\textwidth}
\centering
\begin{tikzpicture}
\draw(0,0) to [american voltage source,l={${20\phase{0^{\circ}}\, \si{\volt}}$}]++(0,\y) to [resistor,l={$\SI{4}{\ohm}$}]++(2*\x,0)coordinate(kA) to [inductor,l_={$j7.85\,\si{\ohm}$}]++(0,-\y) to [short] ++(-2*\x,0);
\draw(2*\x+\x/3,0) to [inductor,l_={$j3.768\,\si{\ohm}$}]++(0,\y)coordinate(kB) to [short]++(2*\x,0) to [resistor,l={$\SI{2}{\ohm}$}]++(0,-\y) to [short]++(-2*\x,0);
%dots
\draw(kA)++(-0.5,-0.5) node[circ]{};
\draw(kB)++(0.5,-0.5) node[circ]{};
\draw(2*\x+\x/6,\y)node[above]{$j5.439\,\si{\ohm}$};
%currents
\draw[stealth-]([shift={(-135:\x/5)}]\x,\y/2) arc (-135:135:\x/5);
\draw(\x,\y/2) node{$\hat{I}_1$};
\draw[stealth-]([shift={(-135:\x/5)}]2*\x+\x/3+\x,\y/2) arc (-135:135:\x/5);
\draw(2*\x+\x/3+\x,\y/2) node{$\hat{I}_2$};
\end{tikzpicture}
\caption*{(ب)}
\end{subfigure}
\caption{مثال \حوالہ{مثال_مقناطیسی_ارتباطی_مستقل_الف} کا دور۔}
\label{شکل_مقناطیسی_ارتباطی_مستقل_الف}
\end{figure}

حل:منبع دباو سے تعدد \عددی{\omega=\SI{314}{\radian\per\second}} اور مساوات \حوالہ{مساوات_مقناطیسی_مشترک_امالہ_تعریف_الف} سے مشترکہ امالہ
\begin{align*}
M&=k\sqrt{L_1 L_2}=1\sqrt{(0.025)(0.012)}=\SI{17.321}{\milli\henry}
\end{align*}
لیتے ہوئے شکل-ب میں تعددی دائرہ کار میں دور کو دوبارہ دکھایا گیا ہے جہاں درج ذیل قیمتیں استعمال کی گئی ہیں۔
\begin{align*}
j\omega L_1&=j (314)(0.025)=j7.85\,\si{\ohm}\\
j\omega L_2&=j(314)(0.012)=j3.768\,\si{\ohm}\\
j\omega M&=j (314)(0.017321)=j5.439\,\si{\ohm}
\end{align*}
دونوں دائروں کے کرخوف مساوات لکھتے ہیں۔
\begin{align*}
20\phase{0^{\circ}}&=(4+j7.85)\hat{I}_1-j5.439\hat{I}_2\\
0&=-j5.439\hat{I}_1+(2+j3.768)\hat{I}_2
\end{align*}
ان میں سے دوسری مساوات سے \عددی{\hat{I}_2} لیتے ہوئے پہلی میں پر کرتے
\begin{align*}
20\phase{0^{\circ}}&=(4+j7.85)\hat{I}_1-j5.439\left(\frac{j5.439 }{2+j3.768}\right)\hat{I}_1
\end{align*}
ہوئے \عددی{\hat{I}_1} حاصل کرتے ہیں۔
\begin{align*}
\hat{I}_1=\frac{20}{7.251+j1.725}=2.610-j0.621=2.683\phase{-13.38^{\circ}}\,\si{\ampere}
\end{align*}
اسی طرح \عددی{\hat{I}_2} درج ذیل حاصل ہوتا ہے۔
\begin{align*}
\hat{I}_2=\left(\frac{j5.439}{2+j3.768}\right) \hat{I}_1=3.421\phase{14.57^{\circ}}\,\si{\ampere}
\end{align*}
حاصل شدہ رو کو وقتی دائرہ کار میں لکھتے ہیں۔
\begin{align*}
i_1(t)&=2.683\cos(314t-13.38^{\circ}) \, \si{\ampere}\\
i_2(t)&=3.421\cos(314t+14.57^{\circ})\,\si{\ampere}
\end{align*}
لمحہ \عددی{t=\SI{6.2}{\milli\second}} پر رو کی قیمتیں حاصل کرتے ہیں۔ایسا کرتے ہوئے زاویہ ہٹاو کو ریڈیئن میں لکھا جائے گا۔
\begin{align*}
i_1(t=\SI{6.2}{\milli\second})&=I_1=2.683\cos\left[(314)(0.0062)-13.38\left(\frac{\pi}{180}\right)\right]=\SI{2.487}{\ampere}\\
i_2(t=\SI{6.2}{\milli\second})&=I_2=3.421\cos\left[(314)(0.062)+14.57\left(\frac{\pi}{180}\right)\right]=\SI{2.199}{\ampere}
\end{align*}
لمحہ \عددی{\SI{6.2}{\milli\second}} پر رو کی قیمتیں جاننے کے بعد مساوات \حوالہ{مساوات_مقناطیسی_توانائی_مشترکہ_امالہ_ب} سے ذخیرہ توانائی حاصل کرتے ہیں۔
\begin{align*}
w(t=\SI{6.2}{\milli\second})&=\frac{L_1 I_1^2}{2}+\frac{L_2 I_2^2}{2}+M I_1 I_2\\
&=\frac{(0.025)(2.487)^2}{2}+\frac{(0.012)(2.199)^2}{2}+0.0173(2.487)(2.199)\\
&=\SI{0.201}{\joule}
\end{align*}
\انتہا{مثال}
%=========================
\ابتدا{مشق}\شناخت{مشق_مقناطیسی_ارتباطی_مستقل_الف}
شکل \حوالہ{شکل_مقناطیسی_مشق_ارتباطی_مستقل_الف} میں تعدد \عددی{\SI{50}{\hertz}} اور \عددی{k=0.6} ہیں۔لمحہ \عددی{t=\SI{5.5}{\milli\second}} پر مشترکہ امالہ میں ذخیرہ توانائی دریافت کریں۔
\begin{figure}
\centering
\begin{tikzpicture}
\draw(0,0) to [american voltage source,l={${36\phase{45^{\circ}}\, \si{\volt}}$}]++(0,\y) to [capacitor,l={$-j44\,\si{\ohm}$}]++(2*\x,0)coordinate(kA) to [inductor,l_={$j20\,\si{\ohm}$}]++(0,-\y) to [short] ++(-2*\x,0);
\draw(2*\x+\x/3,0) to [inductor,l_={$j15\,\si{\ohm}$}]++(0,\y)coordinate(kB) to [capacitor,l={$-j10\,\si{\ohm}$}]++(2*\x,0) to [resistor,l={$\SI{10}{\ohm}$}]++(0,-\y) to [short]++(-2*\x,0);
%dots
\draw(kA)++(-0.5,-0.5) node[circ]{};
\draw(kB)++(0.5,-0.5) node[circ]{};
\draw(2*\x+\x/6,\y)node[above]{$M$};
\end{tikzpicture}
\caption{مشق \حوالہ{مشق_مقناطیسی_ارتباطی_مستقل_الف} کا دور۔}
\label{شکل_مقناطیسی_مشق_ارتباطی_مستقل_الف}
\end{figure}

جواب:\عددی{\SI{24.4}{\milli\joule}}
\انتہا{مشق}
%====================

\حصہ{کامل ٹرانسفارمر}
شکل \حوالہ{شکل_مقناطیسی_کامل_ٹرانسفارمر_ب}-الف کو دیکھیے جہاں دو لچھوں کو مقناطیسی قالب پر لپیٹا گیا ہے۔یہ روزمرہ میں استعمال ہونے والا ٹرانسفارمر ہے۔شکل میں دباو اور رو کی سمتیں یوں چننی گئی ہیں کہ بائیں  لچھے کو بائیں دور سے طاقت مہیا کی جا رہی ہے جبکہ دایاں لچھا دائیں ہاتھ کے دور کو طاقت فراہم کرتا ہے۔ بایاں لچھا \عددی{N_1} چکر پر مشتمل ہے اور اور یہ قالب میں گھڑی کے سوئیوں کے گھومنے کی سمت میں مقناطیسی بہاو \عددی{\phi_1} پیدا کرتا ہے۔دایاں لچھا \عددی{N_2} چکر پر مشتمل ہے اور اور یہ قالب میں گھڑی کے سوئیوں کے گھومنے کی سمت کے الٹ سمت میں مقناطیسی بہاو \عددی{\phi_2} پیدا کرتا ہے۔یوں قالب میں گھڑی کے سوئیوں کے گھومنے کی سمت میں کل مقناطیسی بہاو \عددی{\phi=\phi_1-\phi_2} پایا جائے گا۔لچھوں کو یہی بہاو \عددی{\phi} نظر آتی ہے جو لچھوں میں درج ذیل دباو پیدا کرتی ہے۔
\begin{align}
v_1(t)&=N_1\frac{\dif \phi}{\dif t} \label{مساوات_مقناطیسی_ٹرانسفارمر_الف}\\
v_2(t)&=N_2\frac{\dif \phi}{\dif t}\label{مساوات_مقناطیسی_ٹرانسفارمر_ب}
\end{align}
مساوات \حوالہ{مساوات_مقناطیسی_ٹرانسفارمر_الف} کو مساوات \حوالہ{مساوات_مقناطیسی_ٹرانسفارمر_ب} سے تقسیم کرنے سے
\begin{align*}
\frac{v_1(t)}{v_2(t)}=\frac{N_1\frac{\dif \phi}{\dif t}}{N_2\frac{\dif \phi}{\dif t}}
\end{align*}
یعنی درج ذیل \اصطلاح{تبادلہ دباو}\فرہنگ{تبادلہ!دباو}\فرہنگ{دباو!تبادلہ}\حاشیہب{voltage transformation}\فرہنگ{voltage!transformation}\فرہنگ{transformation!voltage} کی مساوات ملتی ہے۔
\begin{align}\label{مساوات_مقناطیسی_ٹرانسفارمر_پ}
\frac{v_1(t)}{v_2(t)}=\frac{N_1}{N_2} \quad \quad \text{\RL{تبادلہ دباو}}
\end{align}
قانون ایمپیئر  کے تحت قالب کے گرد درج ذیل لکھا جا سکتا ہے
\begin{align*}
\oint H \cdot \dif l=i_{\text{گھیرا}}=N_1 i_1-N_2 i_2
\end{align*} 
جہاں تکمل کو قالب کے اندر گھومتے ہوئے حاصل کیا جاتا ہے جبکہ \عددی{H} قالب کے اندر \اصطلاح{مقناطیسی شدت}\فرہنگ{مقناطیسی!شدت}\حاشیہب{magnetic field intensity}\فرہنگ{magnetic field!intensity} ہے۔مقناطیسی قالب میں \عددی{H} کی قیمت قابل نظر انداز ہوتی ہے۔یوں \عددی{H} کا تکمل بھی قابل نظر انداز ہوتا ہے۔درج بالا مساوات میں تکمل کو صفر کے برابر پر کرنے سے
\begin{align}\label{مساوات_مقناطیسی_ٹرانسفارمر_ت}
N_1 i_1 -N_2 i_2=0
\end{align}
یعنی درج ذیل \اصطلاح{تبادلہ رو}\فرہنگ{تبادلہ!رو}\فرہنگ{رو!تبادلہ}\حاشیہب{current transformation}\فرہنگ{current!transformation}\فرہنگ{transformation!current} کی مساوات ملتی ہے۔
\begin{align}\label{مساوات_مقناطیسی_ٹرانسفارمر_ٹ}
\frac{i_1}{i_2}=\frac{N_2}{N_1}\quad \quad \text{\RL{تبادلہ رو}}
\end{align}
تبادلہ رو کی مساوات سے ظاہر ہے کہ کم چکر والے لچھے میں زیادہ چکر والے لچھے کی نسبت زیادہ رو پائی جاتی ہے۔یوں کم چکر والے لچھے کے لئے زیادہ موٹی تار استعمال کی جائے گی۔
 
مساوات \حوالہ{مساوات_مقناطیسی_ٹرانسفارمر_ت} کو \عددی{\tfrac{v_1}{N_1}} سے ضرب دینے سے
\begin{align}
v_1 i_1 -\frac{N_2}{N_1}v_1 i_2=0
\end{align}
ملتا ہے  جس میں مساوات \حوالہ{مساوات_مقناطیسی_ٹرانسفارمر_پ} سے \عددی{\tfrac{N_2}{N_1}v_1=v_2} پر کرتے ہوئے  درج ذیل لکھا جا سکتا ہے۔ 
\begin{align}
v_1 i_1=v_2 i_2
\end{align}
یہ مساوات کہتا ہے کہ کامل ٹرانسفارمر وہی طاقت دائیں ہاتھ کے دور کو فراہم کرتا ہے جو اسے بائیں ہاتھ کے دور سے ملتا ہے۔اس حقیقت کو یوں بھی بیان کیا جا سکتا ہے کہ کامل ٹرانسفارمر میں طاقت کا ضیاع صفر ہے۔حقیقی ٹرانسفارمر میں طاقت کا ضیاع انتہائی کم ہوتا ہے جسے اس کتاب میں نظر انداز کیا جائے گا۔ 
\begin{figure}
\centering
\begin{subfigure}{1\textwidth}
\centering
\begin{tikzpicture}[american voltages]
\def\height{3};
\def\width{1.5};
\def\thick{0.4};
\def\depthX{0.2};
\def\depthY{0.2};
\def\p{0.2};      %pitch
\def\cTop{2.4}; %top of coil
\def\TL{7};    %number of turns
\def\cTopR{2.3}; %top of right coil
\def\TR{6};    %number of right turns
%flux
\draw[gray,-stealth](\thick/2,\height-\thick) to [out=90,in=180]++(\thick/2,\thick/2) to [short]++(\width-2*\thick,0) to [out=0,in=90]++(\thick/2,-\thick/2);
\draw(\width/2,\height-\thick/2)node[fill=white]{$\phi$};
%core
\draw(0,0)--++(0,\height)--++(\width,0)--++(0,-\height)--cycle;
\draw(0,0)++(\thick,\thick)--++(0,\height-2*\thick)--++(\width-2*\thick,0)--++(0,-\height+2*\thick)--cycle;
%
\draw(\thick,\thick)--++(\depthX,\depthY) --++(0,\height-2*\thick-\depthY);
\draw(\thick,\thick)--++(\depthX,\depthY) --++(\width-2*\thick-\depthX,0);
\draw(0,\height)--++(\depthX,\depthY)--++(\width,0)--++(-\depthX,-\depthY);
\draw(\width,0)--++(\depthX,\depthY)--++(0,\height)--++(-\depthX,-\depthY);
%left winding
\draw (\thick+\depthX,\cTop) to [out=45,in=0] ++(-\thick/2-\depthX,\p/2) to [short]++(-\thick/2,0) to [short] ++(-\x/4,0)coordinate(kTop);
\foreach \l in {0,1,2,...,\TL}{
\draw (0,\cTop-\l*\p) to [out=-135,in=45] ++(\thick+\depthX,-\p);
}
\draw(0,\cTop-\TL*\p-\p) to [short]++ (-\x/4,0)coordinate(kBot);
%right winding
\draw (\width-\thick,\cTopR) to [out=135,in=180] ++(\thick/2,\p/2) to [short]++(\thick/2+\depthX,0) to [short,-o]++(\x/2,0)coordinate(kTopR) to  [short,o-,i={$i_2$}] ++(\x/2,0)++(0,0.25)coordinate(kka);
\foreach \l in {0,1,2,...,\TR}{
\draw (\width+\depthX,\cTopR-\l*\p) to [out=-45,in=135] ++(-\thick-\depthX,-\p);
}
\draw(\width+\depthX,\cTopR-\TR*\p-\p) to [short,-o]++(\x/2,0)coordinate(kBotR) to [short,o-]++ (\x/2,0)++(0.5,-0.25)coordinate(kkb);
%current
\draw(kTop) to [short]++(-\x/2,0)coordinate(kvT) to [short,o-,i<_={$i_1$}]++(-\x/2,0)++(-0.5,0.25)coordinate(ka);
\draw(kBot) to [short] ++(-\x/2,0)coordinate(kvB) to [short,o-]++(-\x/2,0)++(0,-0.25)coordinate(kb);
%box ckt
\draw(ka) rectangle (kb);
\draw(kka) rectangle (kkb);
%dots
\draw[](-0.3,\height-0.3) node[circ]{};
\draw[](\width+\depthX+0.3,\height-0.3) node[circ]{};
%text
\draw($(ka)!0.5!(kb)$)node[rotate=90]{\RL{بایاں دور}};
\draw($(kka)!0.5!(kkb)$)node[rotate=90]{\RL{دایاں دور}};
\draw(0,\height/2) node [left]{$N_1$};
\draw(\width+\depthX,\height/2) node [right]{$N_2$};
\draw($(kvT)!0.5!(kvB)$) node[]{$\begin{aligned} &+ \\ &v_1 \\ &- \end{aligned}$};
\draw($(kTopR)!0.5!(kBotR)$) node[]{$\begin{aligned} &+ \\ &v_2 \\ &- \end{aligned}$};
\draw[stealth-](4/5*\width,\height+\depthY) to [out=90,in=180]++(0.5,0.3)node[right]{\RL{قالب}};
\end{tikzpicture}
\caption*{(الف) کامل ٹرانسفارمر کی بناوٹ۔}
\end{subfigure}
\begin{subfigure}{1\textwidth}
\centering
\begin{tikzpicture}
\draw(0,0)node[transformer core](T){};
\draw (T)node[above]{$N_1:N_2$};
\draw(T.A1)++(0.4,-0.4)node[circ]{};
\draw(T.B1)++(-0.4,-0.4)node[circ]{};
\draw($(T.A1)!0.5!(T.A2)$)node{$\begin{aligned} &+ \\ &v_1 \\ &- \end{aligned}$};
\draw($(T.B1)!0.5!(T.B2)$)node{$\begin{aligned} &+ \\ & v_2 \\ &- \end{aligned}$};
\draw(T.A1) to [short,o-,i<_={$ i_1$}]++(-\x,0)++(0,0.25)coordinate(kTL);
\draw(T.A2) to [short,o-]++(-\x,0)++(-0.5,-0.25)coordinate(kBL);
\draw(kBL) rectangle (kTL);
\draw($(kBL)!0.5!(kTL)$)node[rotate=90]{\RL{بایاں دور}};
\draw(T.B1) to [short,o-,i={$ i_2$}]++(\x,0)++(0,0.25)coordinate(kA);
\draw(T.B2) to [short,o-]++(\x,0)++(0.5,-0.25)coordinate(kB);
\draw(kA) rectangle (kB);
\draw($(kA)!0.5!(kB)$)node[rotate=90]{\RL{دایاں دور}};
\end{tikzpicture}
\caption*{(ب) کامل ٹرانسفارمر کی علامت۔}
\end{subfigure}
\caption{کامل ٹرانسفارمر۔}
\label{شکل_مقناطیسی_کامل_ٹرانسفارمر_ب}
\end{figure}

شکل \حوالہ{شکل_مقناطیسی_کامل_ٹرانسفارمر_ب}-ب میں کامل ٹرانسفارمر کی علامت دکھائی گئی ہے جہاں لچھوں کے درمیان افقی لکیریں مقناطیسی قالب کو ظاہر کرتی ہیں۔ بالائی جانب لچھوں کے چکروں کی نسبت \عددی{(N_1:N_2)} لکھی گئی ہے۔ہم جلد دیکھیں گے کہ ٹرانسفارمر استعمال کرنے والے ادوار میں  ٹرانسفارمر کے لچھوں کی امالہ \عددی{L_1} اور \عددی{L_2} جاننا ضروری نہیں ہوتا۔اسی طرح لچھوں کی مشترکہ امالہ \عددی{M} جاننا بھی درکار نہیں ہوتا۔یہی وجہ ہے کہ ان معلومات کا ذکر ٹرانسفارمر کی علامت پر نہیں کیا گیا ہے۔اس شکل میں ٹرانسفارمر کی علامت پر دونوں جانب کی رو اور دباو بھی دکھائے گئے ہیں۔ 

شکل \حوالہ{شکل_مقناطیسی_ٹرانسفارمر_تبادلہ_بوجھ}-الف میں ٹرانسفارمر کے دائیں ہاتھ بوجھ \عددی{\bZ_L} لدا ہے۔دائیں جانب کے دباو اور رو کا تعلق درج ذیل ہے۔ 
\begin{align}\label{مساوات_مقناطیسی_بوجھ}
\bZ_L=\frac{\hat{V}_2}{\hat{I}_2}
\end{align}
یوں \عددی{\hat{V}_2=20\phase{30^{\circ}}\,\si{\volt}} اور \عددی{\hat{I}_2=2\phase{45^{\circ}}\,\si{\ampere}} کی صورت میں ٹرانسفارمر کے دائیں لچھے کو \عددی{10\phase{-15^{\circ}}\,\si{\ohm}} رکاوٹ نظر آئے گی۔اسی طرح بائیں جانب کے دور کو رکاوٹ \عددی{\bZ_L'} نظر آئے گی
\begin{align}
\bZ_L'=\frac{\hat{V}_1}{\hat{I}_1}
\end{align}
جسے مساوات \حوالہ{مساوات_مقناطیسی_ٹرانسفارمر_پ} اور مساوات \حوالہ{مساوات_مقناطیسی_ٹرانسفارمر_ٹ} کی مدد سے درج ذیل لکھا جا سکتا ہے۔
\begin{align*}
\bZ_L' &=\frac{\frac{N_1}{N_2}\hat{V}_2}{\frac{N_2}{N_1}\hat{I}_2}=\left(\frac{N_1}{N_2}\right)^2\frac{\hat{V}_2}{\hat{I}_2}
\end{align*}
مساوات \حوالہ{مساوات_مقناطیسی_بوجھ} کو استعمال کرتے ہوئے ٹرانسفارمر پر لدے رکاوٹ \عددی{\bZ_L} اور بائیں جانب دور کو نظر آنے والی رکاوٹ \عددی{\bZ_L'} کا تعلق ملتا ہے جسے \اصطلاح{تبادلہ رکاوٹ}\فرہنگ{تبادلہ!رکاوٹ}\فرہنگ{رکاوٹ!تبادلہ}\حاشیہب{impedance transformation}\فرہنگ{impedance!transformation}\فرہنگ{transformation!impedance} کی مساوات کہتے ہیں۔
\begin{align}
\bZ_L'=\left(\frac{N_1}{N_2}\right)^2 \bZ_L \quad \quad \text{\RL{تبادلہ رکاوٹ}}
\end{align}
یوں جیسا شکل \حوالہ{شکل_مقناطیسی_ٹرانسفارمر_تبادلہ_بوجھ}-ب میں دکھایا گیا ہے، بائیں ہاتھ کے دور کو ٹرانسفارمر بمع بوجھ \عددی{\bZ_L} کی جگہ متبادل بوجھ \عددی{\bZ_L'} نظر آتا ہے۔
% 
\begin{figure}
\centering
\begin{subfigure}{0.6\textwidth}
\centering
\begin{tikzpicture}
\draw(0,0)node[transformer core](T){};
\draw (T)node[above]{$N_1:N_2$};
\draw(T.A1)++(0.4,-0.4)node[circ]{};
\draw(T.B1)++(-0.4,-0.4)node[circ]{};
\draw($(T.A1)!0.5!(T.A2)$)node{$\begin{aligned} &+ \\ &\hat{V}_1 \\ &- \end{aligned}$};
\draw($(T.B1)!0.5!(T.B2)$)node{$\begin{aligned} &+ \\ &\hat{V}_2 \\ &- \end{aligned}$};
\draw(T.A1) to [short,o-,i<_={$\hat{I}_1$}]++(-\x,0)++(0,0.25)coordinate(kTL);
\draw(T.A2) to [short,o-]++(-\x,0)++(-0.5,-0.25)coordinate(kBL);
\draw(kBL) rectangle (kTL);
\draw($(kBL)!0.5!(kTL)$)node[rotate=90]{\RL{بایاں دور}};
\draw(T.B1) to [short,o-,i={$\hat{I}_2$}]++(\x,0) to [european resistor,l={$\bZ_L$}]++(0,-\y)coordinate(kB);
\draw(T.B2) to [short,o-]++(\x,0)-|(kB);
\end{tikzpicture}
\caption*{(الف)}
\end{subfigure}%
\begin{subfigure}{0.4\textwidth}
\centering
\begin{tikzpicture}
\draw(0,0) to [short]++(\x/2,0) to [short,o-]++(\x/2,0) to [european resistor,l_={$\bZ_L'$}]++(0,\y) to [short,-o]++(-\x/2,0) to [short,i<_={$\hat{I}_1$}]++(-\x/2,0);
\draw(\x/2,\y/2)node{$\begin{aligned} &+ \\ & \hat{V}_1 \\ &- \end{aligned}$};
\draw(0,-0.25) rectangle (-0.5,\y+0.5);
\draw(-0.25,\y/2)node[rotate=90]{\RL{بایاں دور}};
\end{tikzpicture}
\caption*{(ب)}
\end{subfigure}%
\caption{تبادلہ بوجھ۔}
\label{شکل_مقناطیسی_ٹرانسفارمر_تبادلہ_بوجھ}
\end{figure}
%===========================
\ابتدا{مثال}\شناخت{مثال_مقناطیسی_ٹرانسفارمر_مثال_الف}
شکل \حوالہ{شکل_مقناطیسی_ٹرانسفارمر_مثال_الف} میں تمام دباو اور رو حاصل کریں۔ٹرانسفارمر کے دائیں لچھے کو کتنی مزاحمت نظر آتی ہے۔منبع کو کتنی مزاحمت نظر آتی ہے۔مساوات \حوالہ{مساوات_مقناطیسی_ٹرانسفارمر_پ} کے تحت زیادہ چکر والی جانب پر زیادہ دباو پایا جاتا ہے۔زیادہ چکروں والی جانب کو ٹرانسفارمر کی \اصطلاح{زیادہ دباو والی جانب}\فرہنگ{زیادہ دباو!جانب}\فرہنگ{جانب!زیادہ دباو}\حاشیہب{high voltage side, high tension side, HT}\فرہنگ{high voltage side}\فرہنگ{HT} کہا جاتا ہے جبکہ کم چکروں والی جانب کو ٹرانسفارمر کی \اصطلاح{کم دباو والی جانب}\فرہنگ{کم دباو!جانب}\فرہنگ{جانب!کم دباو}\حاشیہب{low voltage side,low tension side, LT}\فرہنگ{low voltage side}\فرہنگ{LT} کہا جاتا ہے۔ 
\begin{figure}
\centering
\begin{subfigure}{0.6\textwidth}
\centering
\begin{tikzpicture}[american voltages]
\draw(0,0) node[transformer core](T){};
\draw (T)node[above]{$200:50$};
\draw(T.A1)++(0.4,-0.4)node[circ]{};
\draw(T.B1)++(-0.4,-0.4)node[circ]{};
\draw(T.A2) to [short,o-]++(-\x/2,0) to [american voltage source,l={$40\phase{0^{\circ}}\,\si{\volt}$}]++(0,\y)coordinate(kTop);
\draw(T.A1) to  [short,o-,i<_={$\hat{I}_m$}]++(-\x/2,0) |-(kTop);
\draw(T.B1) to [short,o-,i={$\hat{I}_L$}]++(\x/2,0) to [resistor,l={$\SI{1}{\ohm}$},v={$\hat{V}_L$}]++(0,-\y)coordinate(kB);
\draw(T.B2) to [short,o-]++(\x/3,0)-|(kB);
\end{tikzpicture}
\caption*{(الف)}
\end{subfigure}%
\begin{subfigure}{0.4\textwidth}
\centering
\begin{tikzpicture}
\draw(0,0) to [short]++(\x/3,0) to [short,o-]++(\x/3,0) to [resistor,l_={$\SI{16}{\ohm}$}]++(0,\y) to [short,-o]++(-\x/3,0) to [short,i<_={$\hat{I}_m$}]++(-\x/3,0);
\draw(0,0) to [american voltage source,l={$40\phase{0^{\circ}}\,\si{\volt}$}]++(0,\y);
\end{tikzpicture}
\caption*{(ب)}
\end{subfigure}%
\caption{مثال \حوالہ{مثال_مقناطیسی_ٹرانسفارمر_مثال_الف} کا دور۔}
\label{شکل_مقناطیسی_ٹرانسفارمر_مثال_الف}
\end{figure}

حل:ٹرانسفارمر کے چکروں کی تناسب \عددی{200:50} سے ٹرانسفارمر کی کم دباو جانب دباو حاصل کرتے ہیں۔
\begin{align*}
\hat{V}_L=(40\phase{0^{\circ}})\left(\frac{50}{200}\right)=10\phase{0^{\circ}} \, \si{\volt}
\end{align*}
بوجھ کا دباو جانتے ہوئے اس کی رو حاصل کرتے ہیں
\begin{align*}
\hat{I}_L=\frac{\hat{V}_L}{\bZ_L}=\frac{10\phase{0^{\circ}}}{1}=10\phase{0^{\circ}}\,\si{\ampere}
\end{align*}
جس سے منبع کی رو، یعنی زیادہ دباو والی جانب کی رو، حاصل کی جا سکتی ہے۔
\begin{align*}
\hat{I}_m=\left(\frac{N_2}{N_1}\right)\hat{I}_L=\left(\frac{50}{200}\right)(10\phase{0^{\circ}})=2.5\phase{0^{\circ}}\,\si{\ampere}
\end{align*}
جیسا شکل \حوالہ{شکل_مقناطیسی_ٹرانسفارمر_مثال_الف}-ب میں دکھایا گیا ہے، منبع کو ٹرانسفارمر بمع بوجھ، متبادل مزاحمت \عددی{\bZ_L'} نظر آتا ہے
\begin{align*}
\bZ_L'=\left(\frac{N_1}{N_2}\right)^2 \bZ_L=\left(\frac{200}{50}\right)^2 (\SI{1}{\ohm})=\SI{16}{\ohm}
\end{align*} 
جس سے منبع کی رو حاصل کی جا سکتی ہے۔
\begin{align*}
\hat{I}_m=\frac{40\phase{0^{\circ}}}{16}=2.5\phase{0^{\circ}}\,\si{\ampere}
\end{align*}
\انتہا{مثال}
%============================

\ابتدا{مثال}\شناخت{مثال_مقناطیسی_ٹرانسفارمر_مثال_ب}
شکل \حوالہ{شکل_مقناطیسی_ٹرانسفارمر_مثال_ب}-الف میں تمام اجزاء کو ٹرانسفارمر کے بائیں جانب منتقل کرتے ہوئے منبع کی رو حاصل کریں۔حاصل کردہ رو سے بوجھ کی رو حاصل کریں۔ٹرانسفارمر کے ایک لچھے پر منبع سے طاقت فراہم کی جاتی ہے جبکہ ٹرانسفارمر اسی طاقت کو دوسرے لچھے  پر بیرونی جڑے دور کو فراہم کرتا ہے۔ٹرانسفارمر کو جس لچھے پر منبع سے طاقت فراہم کی جاتی ہے اس کو ٹرانسفارمر کا \اصطلاح{داخلی  لچھا}\فرہنگ{داخلی لچھا} یا \اصطلاح{ابتدائی لچھا}\فرہنگ{ابتدائی لچھا}\حاشیہب{primary coil}\فرہنگ{primary coil} کہا جاتا ہے اور ٹرانسفارمر کے اس ہاتھ کو \اصطلاح{داخلی ہاتھ}\فرہنگ{داخلی ہاتھ}\حاشیہب{input side}\فرہنگ{input side} یا \اصطلاح{ابتدائی جانب}\حاشیہب{primary side}\فرہنگ{primary side}کہا جاتا ہے۔جو لچھا بقایا دور کو طاقت فراہم کرتا ہے اس کو \اصطلاح{خارجی لچھا}\فرہنگ{خارجی لچھا} یا \اصطلاح{ثانوی لچھا}\فرہنگ{ثانوی لچھا}\حاشیہب{secondary coil}\فرہنگ{secondary coil} کہا جاتا ہے جبکہ ٹرانسفارمر کے اس ہاتھ کو \اصطلاح{خارجی ہاتھ}\فرہنگ{خارجی ہاتھ}\حاشیہب{output side}\فرہنگ{output side} یا \اصطلاح{ثانوی جانب}\حاشیہب{secondary side} کہا جاتا ہے۔ یوں شکل \حوالہ{شکل_مقناطیسی_ٹرانسفارمر_مثال_ب} میں ٹرانسفارمر کے بائیں جانب کو داخلی جانب یا زیادہ دباو والا ہاتھ یا ابتدائی ہاتھ پکارا جا سکتا ہے جبکہ ٹرانسفارمر کے دائیں جانب کو خارجی جانب یا کم دباو والا جانب یا ثانوی جانب پکارا جا سکتا ہے۔یہاں یہ جاننا ضروری ہے کہ ٹرانسفارمر کے کسی بھی لچھے پر طاقت مہیا کرتے ہوئے دوسرے لچھے سے طاقت حاصل کی جا سکتی ہے لہٰذا ان اصطلاحات کو استعمال کرتے ہوئے ذرہ خیال رکھنا ضروری ہے۔ ادوار بناتے ہوئے داخلی لچھے کو عموماً بائیں جانب دکھایا جاتا ہے۔
\begin{figure}
\centering
\begin{subfigure}{1\textwidth}
\centering
\begin{tikzpicture}[american voltages]
\draw(0,0) node[transformer core](T){};
\draw (T)node[above]{$400:40$};
\draw(T.A1)++(0.4,-0.4)node[circ]{};
\draw(T.B1)++(-0.4,-0.4)node[circ]{};
\draw(T.A2) to [short,o-]++(-\x,0) to [american voltage source,l={$230\phase{0^{\circ}}\,\si{volt}$}]++(0,\y)coordinate(kTop);
\draw(T.A1) to  [european resistor,o-,i_<={$\hat{I}_m$},l_={$6-j2 \,\si{\ohm}$}]++(-\x,0) |-(kTop);
\draw(T.B1) to [short,o-,i={$\hat{I}_L$}]++(\x/2,0) to [european resistor,l={$0.04+j0.05 \, \si{\ohm}$},v={$\hat{V}_L$}]++(0,-\y)coordinate(kB);
\draw(T.B2) to [short,o-]++(\x/3,0)-|(kB);
\end{tikzpicture}
\caption*{(الف)}
\end{subfigure}
\begin{subfigure}{1\textwidth}
\centering
\begin{tikzpicture}
\draw(0,0) to [short]++(\x,0) to [short,o-]++(\x/3,0) to [european resistor,l_={$4+j5 \, \si{\ohm}$}]++(0,\y) to [short,-o]++(-\x/3,0)
 to [european resistor,i_<={$\hat{I}_m$},l_={$6-j2 \,\si{\ohm}$}]++(-\x,0);
\draw(0,0) to [american voltage source,l={$230\phase{0^{\circ}}\,\si{\volt}$}]++(0,\y);
\end{tikzpicture}
\caption*{(ب)}
\end{subfigure}%
\caption{مثال \حوالہ{مثال_مقناطیسی_ٹرانسفارمر_مثال_ب} کا دور۔}
\label{شکل_مقناطیسی_ٹرانسفارمر_مثال_ب}
\end{figure}

حل:چکروں کی تناسب سے متبادل رکاوٹ حاصل کرتے ہیں۔
\begin{align*}
\bZ_L'=\left(\frac{N_1}{N_2}\right)^2\bZ_L=\left(\frac{400}{40}\right)^2 (0.04+j0.05)=4+j5 \, \si{\ohm}
\end{align*}
ٹرانسفارمر اور بوجھ کی جگہ متبادل رکاوٹ نسب کرتے ہوئے شکل \حوالہ{شکل_مقناطیسی_ٹرانسفارمر_مثال_ب}-ب ملتا ہے جہاں سے منبع کی رو حاصل کی جا سکتی ہے۔
\begin{align*}
\hat{I}_m=\frac{230\phase{0^{\circ}}}{6-j2+4+j5}=22\phase{-16.7^{\circ}}\,\si{\ampere}
\end{align*}
ٹرانسفارمر کی بوجھ جانب رو درج ذیل ہو گی۔
\begin{align*}
\hat{I}_L=\left(\frac{400}{40}\right)\hat{I}_m=\left(\frac{400}{40}\right)(22\phase{-16.7^{\circ}})=220\phase{-16.7^{\circ}} \, \si{\ampere}
\end{align*}
\انتہا{مثال}
%=======================
\ابتدا{مثال}\شناخت{مثال_مقناطیسی_ثانوی_جانب_منتقلی}
شکل \حوالہ{شکل_مقناطیسی_ٹرانسفارمر_مثال_ب} میں تمام معلومات کو ثانوی جانب منتقل کرتے ہوئے بوجھ کی رو اور دباو دریافت کریں۔

حل:تبادلہ کئے گئی قیمتوں پر \عددی{(')} کی علامت استعمال کی جاتی ہے۔ ابتدائی جانب لاگو دباو کو ثانوی جانب منتقل کرتے ہوئے درج ذیل حاصل ہوتا ہے۔
\begin{align*}
\hat{V}_m'=\frac{N_2}{N_1} \hat{V}_m=\frac{40}{400} (230\phase{0^{\circ}})=23\phase{0^{\circ}}\,\si{\volt}
\end{align*}
اسی طرح ابتدائی جانب رکاوٹ کو ثانوی جانب منتقل کرتے ہوئے درج ذیل رکاوٹ ملتی ہے۔
\begin{align*}
\bZ'=\frac{40^2}{400^2} (6-j2)=0.06-j0.02\,\si{\ohm}
\end{align*}
ان معلومات کو استعمال کرتے ہوئے شکل \حوالہ{شکل_مقناطیسی_ثانوی_جانب_منتقلی} حاصل ہوتا ہے جس سے  بوجھ کی رو لکھتے ہیں۔
\begin{align*}
\hat{I}_L=\frac{23\phase{0^{\circ}}}{0.06-j0.02+0.04+j0.05}=220\phase{-16.7^{\circ}}\,\si{\ampere}
\end{align*}
بوجھ کا دباو تقسیم دباو سے حاصل کرتے ہیں۔
\begin{align*}
\hat{V}_L=23\phase{0^{\circ}} \left(\frac{0.04+j0.05}{0.06-j0.02+0.04+j0.05}\right)=14.1\phase{34.64^{\circ}}\,\si{\volt}
\end{align*}
اسی جواب کو قانون اوہم سے بھی حاصل کیا جا سکتا ہے یعنی
\begin{align*}
\hat{V}_L=\hat{I}_L \bZ_L=(220\phase{-16.7^{\circ}})(0.04+j0.05)=14.1\phase{34.64^{\circ}}\,\si{\volt}
\end{align*}
ابتدائی رو کو تبادلہ رو کی مساوات سے حاصل کرتے ہیں۔
\begin{align*}
\hat{I}_m=\frac{N_2}{N_1} \hat{I}_L=\frac{40}{400} (220\phase{-16.7^{\circ}})=22\phase{-16.7^{\circ}}\,\si{\ampere}
\end{align*}
%
\begin{figure}
\centering
\begin{tikzpicture}
\draw(0,0) to [short]++(\x+\x/2,0) to [short,o-]++(\x/3,0) to [european resistor,l_={$0.04+j0.05 \, \si{\ohm}$}]++(0,\y) to [short,i_<={$\hat{I}_L$},-o]++(-\x/3,0)
 to [european resistor,l_={$0.06-j0.02 \,\si{\ohm}$}]++(-\x,0) to [short,i_<={$\hat{I}'_m$}]++(-\x/2,0);
\draw(0,0) to [american voltage source,l={$23\phase{0^{\circ}}\,\si{\volt}$}]++(0,\y);
\end{tikzpicture}
\caption{مثال \حوالہ{مثال_مقناطیسی_ثانوی_جانب_منتقلی} کا دور جس میں تمام معلومات ثانوی جانب منتقل کی گئی ہیں۔}
\label{شکل_مقناطیسی_ثانوی_جانب_منتقلی}
\end{figure}

\انتہا{مثال}
%=====================
\ابتدا{مثال}\شناخت{مثال_مقناطیسی_ٹرانسفارمر_الٹ_دباو_الف}
شکل\حوالہ{شکل_مقناطیسی_ٹرانسفارمر_الٹ_دباو_الف}-الف میں تمام نا معلوم مقدار حاصل کریں۔
\begin{figure}
\centering
\begin{subfigure}{1\textwidth}
\centering
\begin{tikzpicture}[american voltages]
\draw(0,0) node[transformer core](T){};
\draw (T)node[above]{$5:1$};
\draw(T.A1)++(0.4,-0.4)node[circ]{};
\draw(T.B2)++(-0.4,0.4)node[circ]{};
\draw(T.A2) to [short,o-]++(-\x-\x/2,0) to [american voltage source,l={$100\phase{0^{\circ}}\,\si{volt}$}]++(0,\y)coordinate(kTop);
\draw(T.A1) to  [european resistor,o-,l_={$2+j4 \,\si{\ohm}$}]++(-\x,0) to [short,i_<={$\hat{I}_1$}]++(-\x/2,0) |-(kTop);
\draw(T.B1) to [european resistor,l={$0.1+j0.6$},o-]++(\x,0) to [short,i={$\hat{I}_2$}] ++(\x/2,0) to [european resistor,l={$0.2-j1 \, \si{\ohm}$},v={$\hat{V}_L$}]++(0,-\y)coordinate(kB);
\draw(T.B2) to [short,o-]++(\x/3,0)-|(kB);
\draw($(T.A1)!0.5!(T.A2)$)node{$\begin{aligned}&+ \\ &\hat{V}_1 \\ &-  \end{aligned}$};
\draw($(T.B1)!0.5!(T.B2)$)node{$\begin{aligned}&+ \\ &\hat{V}_2 \\ &-  \end{aligned}$};
\end{tikzpicture}
\caption*{(الف)}
\end{subfigure}
\begin{subfigure}{1\textwidth}
\centering
\begin{tikzpicture}[american voltages]
\draw(0,0) to [short]++(\x+\x/2,0) to [short,o-]++(\x/3,0) to [european resistor,l_={$7.5-j10 \, \si{\ohm}$},v^>={$\hat{V}_1$}]++(0,\y) to [short,-o]++(-\x/3,0)
 to [european resistor,l_={$2+j4 \,\si{\ohm}$}]++(-\x,0) to [short,i_<={$\hat{I}_1$}]++(-\x/2,0);
\draw(0,0) to [american voltage source,l={$100\phase{0^{\circ}}\,\si{\volt}$}]++(0,\y);
\end{tikzpicture}
\caption*{(ب)}
\end{subfigure}
\caption{مثال \حوالہ{مثال_مقناطیسی_ٹرانسفارمر_الٹ_دباو_الف} کا دور۔}
\label{شکل_مقناطیسی_ٹرانسفارمر_الٹ_دباو_الف}
\end{figure}

حل:شکل \حوالہ{شکل_مقناطیسی_ٹرانسفارمر_الٹ_دباو_الف}-الف میں نقطوں کا مقام شکل \حوالہ{شکل_مقناطیسی_کامل_ٹرانسفارمر_ب}-ب سے مختلف ہے لہٰذا دھیان سے چلنا ہو گا۔موجودہ شکل میں تبادلہ دباو اور تبادلہ رو کے مساوات درج ذیل لکھے جائیں گے۔
\begin{align*}
\frac{\hat{V}_1}{\hat{V}_2}&=-\frac{5}{1}\\
\frac{\hat{I}_1}{\hat{I}_2}&=-\frac{1}{5}
\end{align*}
جبکہ تبادلہ رکاوٹ کی مساوات تبدیل نہیں ہو گی۔ثانوی جانب کل رکاوٹ کو ابتدائی جانب منتقل کرتے ہیں۔
\begin{align*}
\bZ_2'=\left(\frac{5}{1}\right)^2 (0.1+0.6j+0.2-j1)=7.5-j10\,\si{\ohm}
\end{align*} 
یوں شکل \حوالہ{شکل_مقناطیسی_ٹرانسفارمر_الٹ_دباو_الف}-ب حاصل ہوتا ہے جس سے \عددی{\hat{I}_1} اور \عددی{\hat{V}_1} حاصل کرتے ہیں۔
\begin{align*}
\hat{I}_1&=\frac{100\phase{0^{\circ}}}{2+j4+7.5-j10}=8.9\phase{32.3^{\circ}}\,\si{\ampere}\\
\hat{V}_1&=\hat{I}_1 \bZ_2'=(8.9\phase{32.3^{\circ}})(7.5-j10)=111\phase{-20.9^{\circ}}\,\si{\volt}
\end{align*}
یوں تبادلہ دباو اور تبادلہ رو کے مساوات سے ثانوی جانب کے مقدار حاصل کرتے ہیں۔
\begin{align*}
\hat{I}_2&=-\hat{I}_1\left(\frac{N_1}{N_2}\right)=(8.9\phase{32.3^{\circ}})\left(\frac{5}{1}\right)=44.5\phase{212^{\circ}}\,\si{\ampere}\\
\hat{V}_2&=-\hat{V}_1\left(\frac{N_2}{N_1}\right)=-(111\phase{-20.9^{\circ}})\left(\frac{1}{5}\right)=22.2\phase{159^{\circ}}\,\si{\volt}
\end{align*}
دباو \عددی{\hat{V}_2} جانتے ہوئے تقسیم دباو کے کلیے سے بوجھ کا دباو حاصل کرتے ہیں۔
\begin{align*}
\hat{V}_L=(22.2\phase{159^{\circ}}) \left(\frac{0.2-j1}{0.1+j0.6+0.2-j1}\right)=45\phase{134^{\circ}}\,\si{\volt}
\end{align*}
\انتہا{مثال}
%======================
\ابتدا{مشق}\شناخت{مشق_مقناطیسی_سادہ_ابتدائی_رو_الف}
شکل \حوالہ{شکل_مقناطیسی_سادہ_ابتدائی_رو_الف} میں \عددی{\hat{I}_1} حاصل کریں۔
\begin{figure}
\centering
\begin{tikzpicture}[american voltages]
\draw(0,0) node[transformer core](T){};
\draw (T)node[above]{$1:2$};
\draw(T.A1)++(0.4,-0.4)node[circ]{};
\draw(T.B1)++(-0.4,-0.4)node[circ]{};
\draw(T.A2) to [short,o-]++(-\x-\x/2,0) to [american voltage source,l={$22\phase{0^{\circ}}\,\si{volt}$}]++(0,\y)coordinate(kTop);
\draw(T.A1) to  [european resistor,o-,l_={$1-j1 \,\si{\ohm}$}]++(-\x,0) to [short,i_<={$\hat{I}_1$}]++(-\x/2,0) |-(kTop);
\draw(T.B1) to [european resistor,l={$2+j2$},o-]++(\x,0) to [short] ++(\x/2,0) to [european resistor,l={$4 \, \si{\ohm}$}]++(0,-\y)coordinate(kB);
\draw(T.B2) to [short,o-]++(\x/3,0)-|(kB);
\draw($(T.A1)!0.5!(T.A2)$)node{$\begin{aligned}&+ \\ &\hat{V}_1 \\ &-  \end{aligned}$};
\draw($(T.B1)!0.5!(T.B2)$)node{$\begin{aligned}&+ \\ &\hat{V}_2 \\ &-  \end{aligned}$};
\end{tikzpicture}
\caption{مشق \حوالہ{مشق_مقناطیسی_سادہ_ابتدائی_رو_الف} کا دور۔}
\label{شکل_مقناطیسی_سادہ_ابتدائی_رو_الف}
\end{figure}

جواب:\عددی{8.63\phase{11.3^{\circ}}\,\si{\volt}}
\انتہا{مشق}
%====================

%======================
\ابتدا{مشق}\شناخت{مشق_مقناطیسی_سادہ_ابتدائی_رو_ب}
شکل \حوالہ{شکل_مقناطیسی_سادہ_ابتدائی_رو_ب} میں \عددی{\hat{I}_1}، \عددی{\hat{I}_2} اور \عددی{\hat{V}_1} حاصل کریں۔
\begin{figure}
\centering
\begin{tikzpicture}[american voltages]
\draw(0,0) node[transformer core](T){};
\draw (T)node[above]{$2:1$};
\draw(T.A1)++(0.4,-0.4)node[circ]{};
\draw(T.B2)++(-0.4,0.4)node[circ]{};
\draw(T.A2) to [european resistor,o-,l={$4-j4\,\si{\ohm}$}]++(-\x,0) to [short]++(-\x/2,0) to [american voltage source,l={$220\phase{0^{\circ}}\,\si{volt}$}]++(0,\y)coordinate(kTop);
\draw(T.A1) to  [european resistor,o-,l_={$10+j14 \,\si{\ohm}$}]++(-\x,0) to [short,i_<={$\hat{I}_1$}]++(-\x/2,0) |-(kTop);
\draw(T.B1) to [european resistor,l={$4+j4$},o-]++(\x,0)coordinate(kTR);
\draw(T.B2) to [short,o-,i={$\hat{I}_2$}]++(\x,0) to [american voltage source,l_={${80\phase{30^{\circ}}}$}]++(0,\y)-|(kTR);
\draw($(T.A1)!0.5!(T.A2)$)node{$\begin{aligned}&+ \\ &\hat{V}_1 \\ &-  \end{aligned}$};
\draw($(T.B1)!0.5!(T.B2)$)node{$\begin{aligned}&- \\ &\hat{V}_2 \\ &+  \end{aligned}$};
\end{tikzpicture}
\caption{مشق \حوالہ{مشق_مقناطیسی_سادہ_ابتدائی_رو_ب} کا دور۔}
\label{شکل_مقناطیسی_سادہ_ابتدائی_رو_ب}
\end{figure}

جواب:\عددی{9.25\phase{-28.3^{\circ}}\,\si{\ampere}}، \عددی{18.5\phase{-28.3^{\circ}}\,\si{\ampere}}، \عددی{65.2\phase{-17.8^{\circ}}\,\si{\volt}}
\انتہا{مشق}
%====================
\ابتدا{مشق}\شناخت{مشق_مقناطیسی_سادہ_ابتدائی_رو_پ}
شکل \حوالہ{شکل_مقناطیسی_سادہ_ابتدائی_رو_پ} میں \عددی{\hat{I}_0} دریافت کریں۔
\begin{figure}
\centering
\begin{tikzpicture}
\draw(0,0)node[transformer core](T1){};
\draw(2*\x,0,0)node[transformer core](T2){};
\draw(T1.A1)++(0.4,-0.4)node[circ]{};
\draw(T1.B2)++(-0.4,0.4)node[circ]{};
\draw(T2.A1)++(0.4,-0.4)node[circ]{};
\draw(T2.B1)++(-0.4,-0.4)node[circ]{};
\draw(T1)node[above]{$4:1$};
\draw(T2)node[above]{$1:2$};
%
\draw(T1.B1) to [european resistor,l={$1-j0.5 \, \si{\ohm}$},i_>={$\hat{I}_2$}] (T2.A1);
\draw(T1.B2) to [short] (T2.A2);
%
\draw(T1.A2) to [short]++(-\x,0) to [american voltage source,l={$220\phase{0^{\circ}}\,\si{\volt}$}]++(0,\y)coordinate(kTL);
\draw(T1.A1) to [european resistor,l_={$8+j4\,\si{\ohm}$}]++(-\x,0) -|(kTL);
%
\draw(T2.B2) to [short]++(\x,0) to [american voltage source,l={$60\phase{0^{\circ}}\,\si{\volt}$}]++(0,\y)coordinate(kRT);
\draw(T2.B1) to [european resistor,l={$2+j2\,\si{\ohm}$}]++(\x,0)-|(kRT);
\end{tikzpicture}
\caption{مشق \حوالہ{مشق_مقناطیسی_سادہ_ابتدائی_رو_پ} کا دور۔}
\label{شکل_مقناطیسی_سادہ_ابتدائی_رو_پ}
\end{figure}

جواب:\عددی{42.2\phase{173^{\circ}}\,\si{\ampere}}
\انتہا{مشق}
%======================
\ابتدا{مشق}\شناخت{مشق_مقناطیسی_سادہ_ابتدائی_رو_ت}
شکل \حوالہ{شکل_مقناطیسی_سادہ_ابتدائی_رو_ت} میں \عددی{\hat{I}_0=2\phase{-60^{\circ}}\,\si{ampere}} ہے۔منبع کا دباو \عددی{\hat{V}_m} حاصل کریں۔
\begin{figure}
\centering
\begin{tikzpicture}
\draw(0,0) node[transformer core](T){};
\draw(T)node[above]{$4:3$};
\draw(T.A2)++(0.4,0.4)node[circ]{};
\draw(T.B1)++(-0.4,-0.4)node[circ]{};
%
\draw(T.A2) to [short]++(-2*\x,0) to [american voltage source,l={$\hat{V}_m$}]++(0,\y)coordinate(kTL);
\draw(T.A1) to [capacitor,l_={$-j2\,\si{\ohm}$}]++(-\x,0)coordinate(kLM) to [resistor,l_={$\SI{4}{\ohm}$}]++(-\x,0)-|(kTL);
\draw(T.A2)++(-\x,0) to [inductor,*-*,l={$j6\,\si{\ohm}$}](kLM);
\draw(T.B1) to [resistor,l={$\SI{2}{\ohm}$}]++(\x,0)coordinate(kRM) to [capacitor,l={$-j4 \,\si{\ohm}$}]++(\x,0)coordinate(kTR);
\draw(T.B2) to [short]++(2*\x,0) to [resistor,l_={$\SI{4}{\ohm}$},i_<={$\hat{I}_0$}](kTR);
\draw(T.B2)++(\x,0) to [inductor,*-*,l={$j3\,\si{\ohm}$}](kRM);
\end{tikzpicture}
\caption{مشق \حوالہ{مشق_مقناطیسی_سادہ_ابتدائی_رو_ت} کا دور۔}
\label{شکل_مقناطیسی_سادہ_ابتدائی_رو_ت}
\end{figure}

جواب:\عددی{29.2\phase{13.5^{\circ}}\,\si{\volt}}
\انتہا{مشق}
%=========================

کامل ٹرانسفارمر استعمال کرنے والے ادوار  کو مسئلہ تھونن یا مسئلہ نارٹن سے بھی حل کیا جا سکتا ہے۔اس ترکیب میں ٹرانسفارمر بمع ابتدائی جانب دور یا ثانوی جانب دور  کا تھونن یا نارٹن مساوی دور حاصل کرتے ہوئے مکمل دور کو حل کیا جاتا ہے۔آئیں شکل \حوالہ{شکل_مقناطیسی_تھونن_ترکیب_سے_حل}-الف کو مسئلہ تھونن  سے حل کرتے ہوئے اس ترکیب کو سیکھیں۔ٹرانسفارمر بمع بائیں ہاتھ کے دور کا مساوی تھونن حاصل کرتے ہیں۔
\begin{figure}
\centering
\begin{subfigure}{1\textwidth}
\centering
\begin{tikzpicture}
\draw(0,0) node[transformer core](T){};
\draw(T.A1)++(0.4,-0.4)node[circ]{};
\draw(T.B1)++(-0.4,-0.4)node[circ]{};
\draw(T)node[above]{$N_1:N_2$};
%
\draw(T.A2) to [short]++(-\x,0) to [american voltage source,l={$\hat{V}_{m1}$}]++(0,\y)coordinate(kTL);
\draw(T.A1) to [european resistor,i<_={$\hat{I}_1$},l_={$\bZ_1$}]++(-\x,0)-|(kTL);
\draw(T.B2) to [short]++(\x,0) to [american voltage source,l_={$\hat{V}_{m2}$}]++(0,\y)coordinate(kTR);
\draw(T.B1) to [european resistor,l={$\bZ_2$},i>^={$\hat{I}_2$}]++(\x,0)-|(kTR);
%voltage
\draw($(T.A1)!0.5!(T.A2)$)node{$\begin{aligned}&+ \\ &\hat{V}_1 \\ &-  \end{aligned}$};
\draw($(T.B1)!0.5!(T.B2)$)node{$\begin{aligned}&+ \\ &\hat{V}_2 \\ &-  \end{aligned}$};
\end{tikzpicture}
\caption*{(الف)}
\end{subfigure}
\begin{subfigure}{0.6\textwidth}
\centering
\begin{tikzpicture}[american voltages]
\draw(0,0) node[transformer core](T){};
\draw(T.A1)++(0.4,-0.4)node[circ]{};
\draw(T.B1)++(-0.4,-0.4)node[circ]{};
\draw(T)node[above]{$N_1:N_2$};
%
\draw(T.A2) to [short]++(-\x,0) to [american voltage source,l={$\hat{V}_{m1}$}]++(0,\y)coordinate(kTL);
\draw(T.A1) to [european resistor,i<_={$\hat{I}_1$},l_={$\bZ_1$},v^>={$\hat{V}_{Z1}$}]++(-\x,0)-|(kTL);
\draw(T.B2) to [short ,-o]++(\x/2,0)coordinate(kBR);
\draw(T.B1) to [short,-o,i={$\hat{I}_2$}]++(\x/2,0)coordinate(kTR);
%voltage
\draw($(T.A1)!0.5!(T.A2)$)node{$\begin{aligned}&+ \\ &\hat{V}_1 \\ &-  \end{aligned}$};
\draw($(T.B1)!0.5!(T.B2)$)node{$\begin{aligned}&+ \\ &\hat{V}_2 \\ &-  \end{aligned}$};
\draw($(kTR)!0.5!(kBR)$)node{$\begin{aligned}&+ \\ &\hat{V}_{\text{تھونن}} \\ &-  \end{aligned}$};
\end{tikzpicture}
\caption*{(ب) تھونن دباو کا حصول۔}
\end{subfigure}%
\begin{subfigure}{0.4\textwidth}
\centering
\begin{tikzpicture}[american voltages]
\draw(0,0) node[transformer core](T){};
\draw(T.A1)++(0.4,-0.4)node[circ]{};
\draw(T.B1)++(-0.4,-0.4)node[circ]{};
\draw(T)node[above]{$N_1:N_2$};
%
\draw(T.A1) to [european resistor,l_={$\bZ_1$}]++(0,-\y)-|(T.A2);
\draw(T.B2) to [short ,-o]++(\x/2,0)coordinate(kBR);
\draw(T.B1) to [short,-o]++(\x/2,0)coordinate(kTR);
%
\draw[stealth-]($(kTR)!0.5!(kBR)$)++(-\x/4,0) --++(\x/4,0)--++(0,-\y/8)node[below]{$\bZ_{\text{تھونن}}$};
\end{tikzpicture}
\caption*{(پ) تھونن رکاوٹ کا حصول۔}
\end{subfigure}
\begin{subfigure}{1\textwidth}
\centering
\begin{tikzpicture}
\draw(0,0) to [american voltage source,l_={$\hat{V}_{m2}$}]++(0,\y) to [european resistor,l_={$\bZ_2$},-o,i<_={$\hat{I}_2$}]++(-\x,0) to [european resistor,l_={$\left(\frac{N_2}{N_1}\right)^2\bZ_1$}]++(-\x,0)coordinate(kTL);
\draw(0,0) to [short,-o]++(-\x,0) to [short]++(-\x,0) to [american voltage source,l={$\left(\frac{N_2}{N_1}\right) \hat{V}_{m1}$}]++(0,\y)-|(kTL);
\end{tikzpicture}
\caption*{(ت) تمام اجزاء کو دائیں ہاتھ منتقل کیا گیا ہے۔}
\end{subfigure}
\begin{subfigure}{1\textwidth}
\centering
\begin{tikzpicture}
\draw(0,0) to [american voltage source,l_={$\left(\frac{N_1}{N_2}\right)\hat{V}_{m2}$}]++(0,\y) to [european resistor,l_={$\left(\frac{N_1}{N_2}\right)^2\bZ_2$},-o]++(-\x,0) to [european resistor,i<_={$\hat{I}_1$},l_={$\bZ_1$}]++(-\x,0)coordinate(kTL);
\draw(0,0) to [short,-o]++(-\x,0) to [short]++(-\x,0) to [american voltage source,l={$\hat{V}_{m1}$}]++(0,\y)-|(kTL);
\end{tikzpicture}
\caption*{(ٹ) تمام اجزاء کو بائیں ہاتھ منتقل کیا گیا ہے۔}
\end{subfigure}
\caption{ٹرانسفارمر استعمال کرنے والا دور جسے مسئلہ تھونن سے حل کیا گیا ہے۔}
\label{شکل_مقناطیسی_تھونن_ترکیب_سے_حل}
\end{figure}

شکل \حوالہ{شکل_مقناطیسی_تھونن_ترکیب_سے_حل}-ب میں ٹرانسفارمر کا دایاں جانب کھلا سر کیا گیا ہے جہاں سے تھونن دباو حاصل کیا جا سکتا ہے۔چونکہ دایاں جانب کھلے سر ہے لہٰذا \عددی{\hat{I}_2=\SI{0}{\ampere}} ہو گا اور یوں \عددی{\hat{I}_1} بھی صفر ایمپیئر ہو گا۔ رکاوٹ \عددی{\bZ_1} میں رو نہ گزرنے کی بدولت قانون اوہم کے تحت \عددی{\hat{V}_{Z1}}  بھی صفر ہو گا۔یوں \عددی{\hat{V}_1=\hat{V}_{m1}} ہو گا۔دباو \عددی{\hat{V}_1} جانتے ہوئے  تبادلہ دباو کی مساوات سے \عددی{\hat{V}_2} حاصل کیا جا سکتا ہے جو تھونن دباو کے برابر ہو گا۔
\begin{align}
\hat{V}_2=\left(\frac{N_2}{N_1}\right) \hat{V}_1=\left(\frac{N_2}{N_1}\right) \hat{V}_{m1}=\hat{V}_{\text{تھونن}}
\end{align}
شکل \حوالہ{شکل_مقناطیسی_تھونن_ترکیب_سے_حل}-پ میں منبع کو قصر دور  کرتے ہوئے  تھونن رکاوٹ کا حصول دکھایا گیا ہے۔تبادلہ رکاوٹ کی مساوات سے  تھونن رکاوٹ درج ذیل حاصل ہوتی ہے۔
\begin{align}
\bZ_{\text{تھونن}}=\left(\frac{N_2}{N_1}\right)^2 \bZ_1
\end{align}
تھونن دباو اور تھونن رکاوٹ استعمال کرتے ہوئے شکل-الف کو شکل-ت کے طرز پر بنایا جا سکتا ہے۔شکل \حوالہ{شکل_مقناطیسی_تھونن_ترکیب_سے_حل}-ت میں تمام اجزاء کو ٹرانسفارمر کے دائیں جانب منتقل کیا گیا ہے۔اس شکل سے \عددی{\hat{I}_2} حاصل کرتے ہیں۔
\begin{align}
\hat{I}_2=\frac{\frac{N_2}{N_1}\hat{V}_{m1}-\hat{V}_{m2}}{\frac{N_2^2}{N_1^2} \bZ_1+\bZ_2}
\end{align}
ہم تمام اجزاء کو بالکل اسی طرح بائیں  ہاتھ بھی منتقل کر سکتے ہیں۔ایسا کرنے سے  شکل \حوالہ{شکل_مقناطیسی_تھونن_ترکیب_سے_حل}-ٹ حاصل ہوتا ہے جہاں سے \عددی{\hat{I}_1} حاصل کیا جا سکتا ہے۔
\begin{align}
\hat{I}_1=\frac{\hat{V}_{m1}-\frac{N_1}{N_2} \hat{V}_{m2}}{\bZ_1+\frac{N_1^2}{N_2^2}\bZ_2}
\end{align}

%================================
\ابتدا{مثال}
شکل \حوالہ{شکل_مقناطیسی_ایک_جانب_منتقل} میں تمام پرزوں کو بائیں ہاتھ منتقل کرتے ہوئے \عددی{\hat{I}_1} اور \عددی{\hat{I}_2} حاصل کریں۔تمام پرزوں کو دائیں ہاتھ منتقل کرتے ہوئے دوبارہ حل کریں۔
\begin{figure}
\centering
\begin{subfigure}{1\textwidth}
\centering
\begin{tikzpicture}
\draw(0,0) node[transformer core](T){};
\draw(T)node[above]{$3:1$};
\draw(T.A1)++(0.4,-0.4)node[circ]{};
\draw(T.B2)++(-0.4,0.4)node[circ]{};
%
\draw(T.A1)  to [short]++(-\x/2,0) to  [european resistor,l_={$9+j18\,\si{\ohm}$},i<^={$\hat{I}_1$}]++(-\x,0) to [american voltage source,l_={$66\phase{30^{\circ}}\,\si{\volt}$}]++(0,-\y)|-(T.A2);
\draw(T.B1) to [short]++(\x/2,0) to [european resistor,l={$2-j1\,\si{\ohm}$},i>_={$\hat{I}_2$}]++(\x,0)coordinate(kTR);
\draw(T.B2)to [short]++(\x+\x/2,0) to [american voltage source,l_={$30\phase{0^{\circ}}\,\si{\volt}$}]++(0,\y)-|(kTR);
%voltages
\draw($(T.A1)!0.5!(T.A2)$)node{$\begin{aligned} &+ \\ &\hat{V}_1 \\ &- \end{aligned}$};
\draw($(T.B1)!0.5!(T.B2)$)node{$\begin{aligned} &+ \\ &\hat{V}_2 \\ &- \end{aligned}$};
\end{tikzpicture}
\caption*{(الف)}
\end{subfigure}
\begin{subfigure}{1\textwidth}
\centering
\begin{tikzpicture}
\draw(0,0) to [short,o-]++(-\x/2,0) to [european resistor,l_={$9+j18\,\si{\ohm}$},i<^={$\hat{I}_1$}]++(-\x,0) to [american voltage source,l_={$66\phase{30^{\circ}}\,\si{\volt}$}]++(0,-\y) to [short,-o]++(\x+\x/2,0);
\draw(0,0) to [short,o-]++(\x/2,0) to [european resistor,l={$18-j9\,\si{\ohm}$}]++(\x,0) to [american voltage source,l={$90\phase{0^{\circ}}\,\si{\volt}$}]++(0,-\y) to [short,-o]++(-\x-\x/2,0);
\draw(0,-\y/2)node{$\begin{aligned} &+ \\ &\hat{V}_1 \\ &- \end{aligned}$};
\end{tikzpicture}
\caption*{(ب) بائیں منتقلی۔}
\end{subfigure}
\begin{subfigure}{1\textwidth}
\centering
\begin{tikzpicture}
\draw(0,0) to [short,o-]++(-\x/2,0) to [european resistor,l_={$1+j2\,\si{\ohm}$}]++(-\x,0);
\draw(0,-\y) to [short,o-] ++(-\x-\x/2,0) to [american voltage source,l={$22\phase{30^{\circ}}\,\si{\volt}$}]++(0,\y);
\draw(0,0) to [short,o-]++(\x/2,0) to [european resistor,l={$2-j1\,\si{\ohm}$},i>_={$\hat{I}_2$}]++(\x,0);
\draw(0,-\y) to [short,o-]++(\x+\x/2,0) to [american voltage source,l_={$30\phase{0^{\circ}}\,\si{\volt}$}]++(0,\y);
\draw(0,-\y/2)node{$\begin{aligned} &+ \\ &\hat{V}_2 \\ &- \end{aligned}$};
\end{tikzpicture}
\caption*{(پ) دائیں منتقلی۔}
\end{subfigure}
\caption{تمام پرزوں کی ایک جانب منتقلی۔}
\label{شکل_مقناطیسی_ایک_جانب_منتقل}
\end{figure}

حل:لچھوں پر نقطوں کو دیکھتے ہوئے دباو کو ایک جانب سے دوسری جانب منتقل کریں۔شکل \حوالہ{شکل_مقناطیسی_ایک_جانب_منتقل}-ب میں تمام اجزاء کو بائیں جانب منتقل کیا گیا ہے جبکہ شکل-پ میں تمام اجزاء کو دائیں جانب منتقل کیا گیا ہے۔
\انتہا{مثال}
%===============================
\ابتدا{مثال}
شکل میں \عددی{\hat{V}_0} دریافت کریں۔
\begin{figure}
\centering
\begin{subfigure}{1\textwidth}
\centering
\begin{tikzpicture}[american voltages]
\draw(0,0) node[transformer core](T){};
\draw(T)node[above]{$2:1$};
\draw(T.A1)++(0.4,-0.4)node[circ]{};
\draw(T.B1)++(-0.4,-0.4)node[circ]{};
%
\draw(T.A1) to [american voltage source,l_={$4\phase{0^{\circ}}\,\si{\volt}$}]++(-\x,0) to [european resistor,l_={$4-j1\,\si{\ohm}$}]++(-\x,0)coordinate(kTL) to [short]++(-\x,0) to [inductor,l_={$j2\,\si{\ohm}$}]++(0,-\y)|-(T.A2);
\draw(T.A2)++(-2*\x,0) to [american current source,*-*,l_={$2\phase{0^{\circ}}\,\si{\ampere}$}](kTL);
%
\draw(T.B1) to [european resistor,l={$4-j8\,\si{\ohm}$}]++(\x,0) to [inductor,l={$j20\,\si{\ohm}$},v={$\hat{V}_0$}]++(0,-\y)|-(T.B2);
\end{tikzpicture}
\caption*{(الف)}
\end{subfigure}
\begin{subfigure}{1\textwidth}
\centering
\begin{tikzpicture}[american voltages]
%
\draw(0,0) to [american voltage source,o-,l_={$4\phase{0^{\circ}}\,\si{\volt}$}]++(-\x,0) to [european resistor,l_={$4-j1\,\si{\ohm}$}]++(-\x,0)coordinate(kTL) to [short]++(-\x,0) to [inductor,l_={$j2\,\si{\ohm}$}]++(0,-\y)to [short,-o]++(3*\x,0);
\draw(-2*\x,-\y) to [american current source,*-*,l_={$2\phase{0^{\circ}}\,\si{\ampere}$}](kTL);
\draw(0,-\y/2)node{$\begin{aligned} &+ \\ &\hat{V}_{\text{تھونن}} \\ &- \end{aligned}$};
\end{tikzpicture}
\caption*{(ب)}
\end{subfigure}
\begin{subfigure}{1\textwidth}
\centering
\begin{tikzpicture}[american voltages]
\draw(0,0) node[transformer core](T){};
\draw(T)node[above]{$2:1$};
\draw(T.A1)++(0.4,-0.4)node[circ]{};
\draw(T.B1)++(-0.4,-0.4)node[circ]{};
%
\draw(T.A1)  to [european resistor,l_={$4+j1\,\si{\ohm}$}]++(-\x,0)coordinate(kTL) ;
\draw(T.A2) to [short]++(-\x,0) to [american voltage source,l={$4\phase{150^{\circ}}\,\si{\volt}$}](kTL);
%
\draw(T.B1) to [european resistor,l={$4-j8\,\si{\ohm}$}]++(\x,0) to [inductor,l={$j20\,\si{\ohm}$},v={$\hat{V}_0$}]++(0,-\y)|-(T.B2);
\end{tikzpicture}
\caption*{(پ)}
\end{subfigure}
\begin{subfigure}{1\textwidth}
\centering
\begin{tikzpicture}[american voltages]
\draw(0,0) to [american voltage source,l={$2\phase{150^{\circ}}\,\si{\volt}$}]++(0,\y)to [european resistor,l={$1+j0.25\,\si{\ohm}$}]++(\x,0) to [european resistor,l={$4-j8\,\si{\ohm}$}]++(\x,0) to [inductor,l={$j20\,\si{\ohm}$},v={$\hat{V}_0$}]++(0,-\y)to [short] (0,0);
\end{tikzpicture}
\caption*{(ت)}
\end{subfigure}
\end{figure}

حل:ہم ٹرانسفارمر کے بائیں جانب دور کا تھونن مساوی حاصل کرتے ہیں۔اس دور کو شکل-ب میں دکھایا گیا ہے۔شکل-ب میں منبع رو کی تمام رو امالہ سے گزرتی ہے لہٰذا امالہ پر دباو \عددی{4\phase{90^{\circ}}\,\si{\volt}} ہو گا۔منبع دباو کی رو صفر ہے۔یوں تھونن دباو \عددی{4\phase{90^{\circ}}-4\phase{30^{\circ}}=4\phase{150^{\circ}}\,\si{\volt}} ہو گا۔شکل-ب میں منبع رو کو کھلے سر اور منبع دباو کو قصر دور کرتے ہوئے تھونن رکاوٹ حاصل کرتے ہیں۔
\begin{align*}
\bZ_{\text{تھونن}}=4-j1+j2=4+j1\,\si{\ohm}
\end{align*}
تھونن مساوی دور استعمال کرتے ہوئے شکل-الف سے شکل-پ حاصل ہوتا ہے۔تمام پرزوں کو دائیں منتقل کرتے ہوئے شکل-ت حاصل ہوتا ہے جس کو دیکھ کر تقسیم دباو کے کلیے سے  \عددی{\hat{V}_0} لکھا جا سکتا ہے۔
\begin{align*}
\hat{V}_0=\left(\frac{j20}{1+j0.25+4-j8+j20}\right)(2\phase{150^{\circ}})=3\phase{-188^{\circ}}\,\si{\volt}
\end{align*}
\انتہا{مثال}
%=========================
\ابتدا{مشق}\شناخت{مشق_مقناطیسی_تھونن_مساوی_سے_حل_الف}
شکل \حوالہ{شکل_مقناطیسی_تھونن_مساوی_سے_حل_الف} میں ٹرانسفارمر بمع دائیں ہاتھ دور کا تھونن مساوی حاصل کرتے ہوئے \عددی{\hat{I}_1} دریافت کریں۔
\begin{figure}
\centering
\begin{tikzpicture}
\draw(0,0) node[transformer core](T){};
\draw(T)node[above]{$3:1$};
\draw(T.A2)++(0.4,0.4)node[circ]{};
\draw(T.B1)++(-0.4,-0.4)node[circ]{};
%
\draw(T.A1)to[european resistor,l_={$4-j6\,\si{\ohm}$},i<^={$\hat{I}_1$}]++(-\x,0) coordinate(kTL);
\draw(T.A2) to [short]++(-\x,0) to [american voltage source,l={$40\phase{0^{\circ}}\,\si{\volt}$}]++(0,\y)-|(kTL);
\draw(T.B1) to [european resistor,l={$2+j2\,\si{\ohm}$},i>_={$\hat{I}_2$}]++(\x,0)coordinate(kTR);
\draw(T.B2) to [short]++(\x,0) to [american voltage source,l_={$15\phase{0^{\circ}}\,\si{\volt}$}]++(0,\y)-|(kTR);
\end{tikzpicture}
\caption{مشق \حوالہ{مشق_مقناطیسی_تھونن_مساوی_سے_حل_الف} کا دور۔}
\label{شکل_مقناطیسی_تھونن_مساوی_سے_حل_الف}
\end{figure} 

جواب:\عددی{3.39\phase{-28.6^{\circ}}\,\si{\ampere}}
\انتہا{مشق}
%=======================
\ابتدا{مشق}
شکل \حوالہ{شکل_مقناطیسی_تھونن_مساوی_سے_حل_الف} میں بائیں ہاتھ کے دور اور ٹرانسفارمر کا تھونن مساوی استعمال کرتے ہوئے \عددی{\hat{I}_2} دریافت کریں۔

جواب:\عددی{10.17\phase{151.4^{\circ}}\,\si{\ampere}}
\انتہا{مشق}
%===============

آپ نے دیکھا کہ تمام اجزاء کا تبادلہ ٹرانسفارمر کے ایک جانب کرنے سے دور نہایت آسانی سے حل ہوتا ہے۔یہاں بتلاتا چلوں کہ یہ ترکیب صرف اس صورت قابل استعمال ہے جب تک ٹرانسفارمر کے دونوں اطراف کے ادوار کسی طرح بھی آپس میں نہ جڑے ہوں۔اگر ٹرانسفارمر کے دونوں جانب کے ادوار آپس میں جڑے ہوں تب مکمل دور کے کرخوف مساوات لکھتے ہوئے دور حل کیا جاتا ہے۔  

%==================
\ابتدا{مشق}\شناخت{مشق_مقناطیسی_کرخوف_مساوات_سے_حل}
شکل \حوالہ{شکل_مقناطیسی_کرخوف_مساوات_سے_حل} میں کرخوف کے مساوات اور ٹرانسفارمر کے مساوات تبادلہ استعمال کرتے ہوئے \عددی{\hat{I}_1}، \عددی{\hat{I}_2}، \عددی{\hat{V}_1} اور \عددی{\hat{V}_2} دریافت کریں۔
\begin{figure}
\centering
\begin{tikzpicture}
\draw(0,0) node[transformer core](T){};
\draw(T)node[above]{$3:1$};
\draw(T.A1)++(0.4,-0.4)node[circ]{};
\draw(T.B2)++(-0.4,0.4)node[circ]{};
\draw(T.A2) node[circ]{}node[ground]{};
\draw(T.B2) node[circ]{}node[ground]{};
%
\draw(T.A1) to [short,i<_={$\hat{I}_1$}]++(-\x/2,0)coordinate(kTM) to [resistor,l_={$\SI{1}{\ohm}$}]++(-\x,0)coordinate(kTL);
\draw(T.A2) to [short]++(-\x-\x/2,0) to [american voltage source,l={$20\phase{0^{\circ}}\,\si{\volt}$}]++(0,\y)|-(kTL);
\draw(T.A2)++(-\x/2,0) to [inductor,*-*,l={$j8\,\si{\ohm}$}](kTM);
%
\draw(T.B1) to [short,-*,i={$\hat{I}_2$}]++(\x/2,0)coordinate(kTMR) to [short]++(\x,0) to [resistor,l={$\SI{4}{\ohm}$}]++(0,-\y)|-(T.B2);
\draw(kTM)to [short]++(0,\y/2) to [capacitor,l={$-j6\,\si{\ohm}$}]++(2*\x,0) -|(kTMR);
\draw(T.B2)++(\x/2,0)node[shift={(0.8,0.5)}]{$0.5\phase{0^{\circ}}\,\si{\ampere}$} to [american current source,*-*] (kTMR);

%text
\draw($(T.A1)!0.5!(T.A2)$)node{$\begin{aligned} &+ \\ &\hat{V}_1 \\ &- \end{aligned}$};
\draw($(T.B1)!0.5!(T.B2)$)node{$\begin{aligned} &+ \\ &\hat{V}_2 \\ &- \end{aligned}$};
\end{tikzpicture}
\caption{مشق \حوالہ{مشق_مقناطیسی_کرخوف_مساوات_سے_حل} کا دور۔}
\label{شکل_مقناطیسی_کرخوف_مساوات_سے_حل}
\end{figure}

جوابات:\عددی{1.6\phase{55^{\circ}}\,\si{\ampere}}، \عددی{4.8\phase{235^{\circ}}\,\si{\ampere}}، \عددی{19\phase{-9^{\circ}}\,\si{\volt}}، \عددی{6.3\phase{171^{\circ}}\,\si{\volt}}
\انتہا{مشق}
%========================

ٹرانسفارمر کی ثانوی جانب کو کھلا دور رکھتے ہوئے اس کے ابتدائی جانب منبع دباو نسب کرنے سے ثانوی جانب کی رو صفر حاصل ہوتی ہے۔تبادلہ رو کے تحت یوں ابتدائی رو بھی صفر ہو گی۔حقیقت میں ٹرانسفارمر کے دونوں لچھے عام امالہ ہیں لہٰذا ابتدائی جانب منبع دباو نسب کرنے سے ابتدائی لچھے میں رو ضرور گزرے گی جسے ٹرانسفارمر کی \اصطلاح{ہیجان انگیز رو}\فرہنگ{ہیجان انگیز رو}\حاشیہب{excitation current}\فرہنگ{excitation current} کہتے ہیں۔کسی بھی ٹرانسفارمر جس پر اس کے استعداد کے مطابق پورا برقی بوجھ لدا ہو میں رو کی قیمتیں  ہیجان انگیز رو سے اتنی زیادہ ہوتی ہیں کہ ہیجان انگیز رو کو نظر انداز کیا جا سکتا ہے۔

ٹرانسفارمر پر یک سمتی رو طاقت لاگو کرنے سے قالب میں ساکن مقناطیسی بہاو پیدا ہو گا۔قانون فیراڈے کے تحت ساکن مقناطیسی میدان صفر دباو پیدا کرتا ہے لہٰذا ٹرانسفارمر کے دونوں لچھوں میں صفر دباو پایا جائے گا۔ایسی صورت میں ٹرانسفارمر ہمارے کسی کام کا نہیں ہے۔اس کے برعکس ٹرانسفارمر کے ابتدائی جانب بدلتی رو طاقت لاگو کرنے سے قالب میں بدلتا مقناطیسی میدان پیدا ہوتا ہے جو دونوں لچھوں میں \اصطلاح{امالی دباو}\فرہنگ{امالی دباو}\فرہنگ{دباو!امالی}\حاشیہب{induced voltage}\فرہنگ{induced voltage} \عددی{v=\tfrac{\dif \lambda}{\dif t}} پیدا کرتا ہے۔ٹرانسفارمر کی مدد سے بدلتی دباو کو کم اور زیادہ کیا جاتا ہے۔یہی بنیادی وجہ ہے کہ تمام دنیا میں برقی طاقت کی ترسیل بدلتی رو طاقت کی صورت میں کی جاتی ہے۔

ہر ٹرانسفارمر کسی مخصوص دباو اور تعدد پر چلنے کے لئے بنایا جاتا ہے۔ٹرانسفارمر کی صحیح کارکردگی کے لئے ضروری ہے کہ اسے انہیں دباو اور تعدد پر استعمال کیا جائے۔عموماً مشینوں کی طرح ٹرانسفارمر پر لگی معلوماتی تختی پر ٹرانسفارمر کی معلومات درج ہوتی ہے۔اس تختی پر لچھوں کے چکر کی بجائے دونوں اطراف کے موثر دباو درج ہوتے ہیں۔ گھریلو صارفین کو طاقت مہیا کرنے والے واپڈا کے ٹرانسفارمر پر \عددی{11000\!:\!230\, \si{\volt/\volt}}، \عددی{\SI{50}{\hertz}} درج\حاشیہد{مشین کی معلوماتی تختی پر پر درج \عددی{\SI{230}{\volt}} سے مراد \عددی{\SI{230}{\volt}\,\rms} ہوتا ہے۔} ہو گا۔یہ ٹرانسفارمر \عددی{\SI{11}{\kilo\volt}\,\rms} سے \عددی{\SI{230}{\volt}\,\rms} پیدا کرتا ہے۔عموماً استعمال میں اس ٹرانسفارمر کو زیادہ دباو جانب \عددی{\SI{11}{\kilo\volt}\,\rms} پر طاقت مہیا کی جاتی ہے  جبکہ اس کے کم دباو جانب سے \عددی{\SI{230}{\volt}\,\rms} پر طاقت حاصل کی جاتی ہے۔اس ٹرانسفارمر کے کم دباو جانب \عددی{\SI{230}{\volt}\,\rms} پر طاقت مہیا کرتے ہوئے زیادہ دباو جانب سے \عددی{\SI{11}{\kilo\volt}\,\rms} پر طاقت حاصل کی جا سکتی ہے۔

%=================
\ابتدا{مثال}
تربیلا ڈیم سے \عددی{\SI{10}{\mega\watt}} طاقت اندرون ملک منتقل کرنے کے لئے \اصطلاح{المونیم}\فرہنگ{المونیم}\حاشیہب{aluminium}\فرہنگ{aluminium} کا ترسیلی تار درکار ہے۔ترسیلی تار کی یک طرفہ لمبائی \عددی{\SI{500}{\kilo\meter}} ہے۔تار میں مزاحمتی ضیاع کو کل طاقت کے \عددی{\SI{5}{\percent}} سے کم رکھنا ہے۔طاقت کی ترسیل \عددی{\SI{110}{\kilo\volt}\,\rms} دباو پر کی جائے گی۔تار کی موٹائی دریافت کریں۔اگر طاقت کی ترسیل \عددی{\SI{230}{\volt}\,\rms} پر کی جائے، تب تار کی موٹائی کیا ہو گی۔المونیم کی موصلیت \عددی{\sigma=\SI{3.69e7}{\siemens\per\meter}} ہے۔

حل:طاقت کا ضیاع حاصل کرتے ہیں۔
\begin{align*}
P_{\text{ضیاع}}=(0.05)(\SI{10}{\mega\watt})=\SI{0.5}{\mega\watt}
\end{align*}
ترسیلی دباو پر رو کی قیمت حاصل کرتے ہیں۔
\begin{align*}
I=\frac{\SI{10}{\mega\watt}}{\SI{110}{\kilo\volt}}=\SI{90.9}{\ampere}\,\rms
\end{align*}
طاقت کا ضیاع تار میں مزاحمتی ضیاع کی بنا ہے یعنی
\begin{align*}
P_{\text{ضیاع}}=I^2 R
\end{align*}
لہٰذا
\begin{align*}
R=\frac{P_{\text{ضیاع}}}{I^2}=\frac{\num{500000}}{90.9^2}=\SI{60.5}{\ohm}
\end{align*}
ہو گا۔تار کی یک طرفہ لمبائی پانچ سو کلو میٹر ہے لہٰذا دو عدد تار کی کل لمبائی \عددی{\SI{1000}{\kilo\meter}} ہو گی۔رداس \عددی{r}، موصلیت \عددی{\sigma} اور \عددی{L} لمبی تار کی مزاحمت \عددی{R=\tfrac{L}{\sigma \pi r^2}} ہو گی لہٰذا تار کا رداس درج ذیل ہو گا۔
\begin{align*}
r=\sqrt{\frac{L}{\sigma \pi R}}=\sqrt{\frac{\num{1000000}}{\num{3.69e7} \pi 60.5}}=\SI{1.19}{\centi\meter}
\end{align*}
آئیں اب \عددی{\SI{230}{\volt}\,\rms} پر ترسیل کے لئے حل کرتے ہیں۔طاقت کا ضیاع اب بھی  \عددی{\SI{0.5}{\mega\watt}} ہے جبکہ رو درج ذیل ہے۔
\begin{align*}
I=\frac{\SI{10}{\mega\watt}}{\SI{230}{\volt}}=\SI{43478}{\ampere}\,\rms
\end{align*}
یوں تار کی مزاحمت
\begin{align*}
R=\frac{P_{\text{ضیاع}}}{I^2}=\frac{\num{500000}}{43478^2}=\SI{0.0002645}{\ohm}
\end{align*}
ہو گی جس سے تار کا رداس درج ذیل حاصل ہوتا ہے۔
\begin{align*}
r=\sqrt{\frac{L}{\sigma \pi R}}=\sqrt{\frac{\num{1000000}}{\num{3.69e7} \pi 0.0002645}}=\SI{5.7}{\meter}
\end{align*}
آپ نے یقیناً واپڈا کے کسی کھمبے میں \عددی{\SI{11.4}{\meter}} قطر کا تار نہیں دیکھا ہو گا۔میں نے تو اتنے قطر کا کھمبا بھی کبھی نہیں دیکھا۔

اس مثال سے صاف واضح ہے کہ طاقت کی ترسیل زیادہ سے زیادہ دباو پر کی جاتی ہے تا کہ کم سے کم موٹائی کی تار استعمال کی جائے۔
\انتہا{مثال}
%===================
\ابتدا{مثال}
پانچ کلو واٹ سے کم طاقت استعمال کرنے والے گھریلو صارفین کے ہاں ایک دور (یک فیزہ) توانائی ناپنے والا \اصطلاح{میٹر}\فرہنگ{میٹر}\حاشیہب{energy meter}\فرہنگ{energy meter} نسب کیا جاتا ہے۔ایک قصبے میں پچاس گھر ہیں۔اس قصبے کو کتنی استعداد کا ٹرانسفارمر درکار ہے۔فرض کریں کہ ہر گھرانے کو \عددی{\SI{230}{\volt}\,\rms} دباو پر \عددی{\SI{20}{\ampere}\,\rms} درکار ہوں گے۔

حل:پچاس گھرانوں کو کل درج ذیل رو درکار ہو گی۔
\begin{align*}
(50)(\SI{20}{\ampere}\,\rms)=\SI{1000}{\ampere}\,\rms
\end{align*}
یوں اس قصبے کو
\begin{align*}
(\SI{230}{\volt}\,\rms)(\SI{1000}{\ampere}\,\rms)=\SI{230}{\kilo\volt\ampere}
\end{align*}
استعداد کا ٹرانسفارمر درکار ہو گا۔
\انتہا{مثال}
%=================
\ابتدا{مثال}\شناخت{مثال_مقناطیسی_کامل_ٹرانسفارمر_متعدد_لچھے}
شکل \حوالہ{شکل_مقناطیسی_کامل_ٹرانسفارمر_متعدد_لچھے} میں ٹرانسفارمر کے قالب پر تین عدد لچھے لپیٹے گئے ہیں۔فرض کریں کہ چکروں کی تناسب درج ذیل ہے۔
\begin{align*}
N_1:N_2:N_3=3:2:10
\end{align*}
 یوں \عددی{N_3} پر \عددی{\SI{20}{\volt}\phase{0^{\circ}}\,\rms} مہیا کرنے سے \عددی{N_2} پر \عددی{\SI{6}{\volt}\phase{0^{\circ}}\,\rms} اور \عددی{N_3} پر \عددی{\SI{4}{\volt}\phase{0^{\circ}}\,\rms} ملیں گے۔
\begin{itemize}
\item
اگر \عددی{b} اور \عددی{c} کو جوڑا جائے تب \عددی{V_{ad}=\SI{10}{\volt}\,\rms} ہو گا۔شکل \حوالہ{شکل_مقناطیسی_کامل_ٹرانسفارمر_متعدد_لچھے}-ب میں ایسا دکھایا گیا ہے  جہاں صفائی کی خاطر قالب ظاہر کرنے والی عمودی لکیریں نہیں کھینچی گئی ہیں۔
\item
اس کے برعکس اگر صرف \عددی{b} اور \عددی{d} کو جوڑا جائے تب \عددی{V_{ad}=\SI{2}{\volt}\,\rms} ہو گا۔
\item
اسی طرح  اگر صرف \عددی{c} اور \عددی{f} کو جوڑا جائے تب \عددی{V_{ed}=\SI{24}{\volt}\,\rms} اور \عددی{V_{ab}=\SI{6}{\volt}\,\rms} ہو گا۔
\end{itemize}

\begin{figure}
\centering
\begin{subfigure}{1\textwidth}
\centering
\begin{tikzpicture}[american voltages]
\def\height{3};
\def\width{1.5};
\def\thick{0.4};
\def\depthX{0.2};
\def\depthY{0.2};
\def\p{0.2};      %pitch
\def\cTop{2.4}; %top of coil
\def\TL{2};    %number of turns
\def\cTopL{1.3}; %top of coil
\def\TLL{2};    %number of turns
\def\cTopR{2.3}; %top of right coil
\def\TR{6};    %number of right turns
%flux
\draw[gray,-stealth](\thick/2,\height-\thick) to [out=90,in=180]++(\thick/2,\thick/2) to [short]++(\width-2*\thick,0) to [out=0,in=90]++(\thick/2,-\thick/2);
\draw(\width/2,\height-\thick/2)node[fill=white]{$\phi$};
%core
\draw(0,0)--++(0,\height)--++(\width,0)--++(0,-\height)--cycle;
\draw(0,0)++(\thick,\thick)--++(0,\height-2*\thick)--++(\width-2*\thick,0)--++(0,-\height+2*\thick)--cycle;
%
\draw(\thick,\thick)--++(\depthX,\depthY) --++(0,\height-2*\thick-\depthY);
\draw(\thick,\thick)--++(\depthX,\depthY) --++(\width-2*\thick-\depthX,0);
\draw(0,\height)--++(\depthX,\depthY)--++(\width,0)--++(-\depthX,-\depthY);
\draw(\width,0)--++(\depthX,\depthY)--++(0,\height)--++(-\depthX,-\depthY);
%left winding top
\draw (\thick+\depthX,\cTop) to [out=45,in=0] ++(-\thick/2-\depthX,\p/2) to [short]++(-\thick/2,0) to [short,-o] ++(-\x/4,0)node[left]{$a$}coordinate(kTop);
\foreach \l in {0,1,2,...,\TL}{
\draw (0,\cTop-\l*\p) to [out=-135,in=45] ++(\thick+\depthX,-\p);
}
\draw(0,\cTop-\TL*\p-\p) to [short,-o]++ (-\x/4,0)node[left]{$b$}coordinate(kBot);
%left winding Lower
\draw (\thick+\depthX,\cTopL) to [out=45,in=0] ++(-\thick/2-\depthX,\p/2) to [short]++(-\thick/2,0) to [short,-o] ++(-\x/4,0)node[left]{$c$}coordinate(kTopL);
\foreach \l in {0,1,2,...,\TLL}{
\draw (0,\cTopL-\l*\p) to [out=-135,in=45] ++(\thick+\depthX,-\p);
}
\draw(0,\cTopL-\TLL*\p-\p) to [short,-o]++ (-\x/4,0)node[left]{$d$}coordinate(kBotL);
%right winding
\draw (\width-\thick,\cTopR) to [out=135,in=180] ++(\thick/2,\p/2) to [short]++(\thick/2+\depthX,0) to [short,-o]++(\x/4,0)node[right]{$e$}coordinate(kTopR); %to  [short,o-,i={$i_2$}] ++(\x/2,0)++(0,0.25)coordinate(kka);
\foreach \l in {0,1,2,...,\TR}{
\draw (\width+\depthX,\cTopR-\l*\p) to [out=-45,in=135] ++(-\thick-\depthX,-\p);
}
\draw(\width+\depthX,\cTopR-\TR*\p-\p) to [short,-o]++(\x/4,0)node[right]{$f$}coordinate(kBotR);% to [short,o-]++ (\x/2,0)++(0.5,-0.25)coordinate(kkb);
%current
%\draw(kTop) to [short]++(-\x/2,0)coordinate(kvT) to [short,o-,i<_={$i_1$}]++(-\x/2,0)++(-0.5,0.25)coordinate(ka);
%\draw(kBot) to [short] ++(-\x/2,0)coordinate(kvB) to [short,o-]++(-\x/2,0)++(0,-0.25)coordinate(kb);
%box ckt
%\draw(ka) rectangle (kb);
%\draw(kka) rectangle (kkb);
%dots
\draw[](-0.3,\height-0.3) node[circ]{};
\draw[](-0.3,\height/2) node[circ]{};
\draw[](\width+\depthX+0.3,\height-0.3) node[circ]{};
%text
%\draw($(ka)!0.5!(kb)$)node[rotate=90]{\RL{بایاں دور}};
%\draw($(kka)!0.5!(kkb)$)node[rotate=90]{\RL{دایاں دور}};
\draw(0,3/4*\height)++(0,-0.2) node [left]{$N_1$};
\draw(0,\height/3) node [left]{$N_2$};
\draw(\width+\depthX,\height/2) node [right]{$N_3$};
%\draw($(kvT)!0.5!(kvB)$) node[]{$\begin{aligned} &+ \\ &v_1 \\ &- \end{aligned}$};
%\draw($(kTopR)!0.5!(kBotR)$) node[]{$\begin{aligned} &+ \\ &v_2 \\ &- \end{aligned}$};
%\draw[stealth-](4/5*\width,\height+\depthY) to [out=90,in=180]++(0.5,0.3)node[right]{\RL{قالب}};
\end{tikzpicture}
\caption*{(الف)}
\end{subfigure}
\begin{subfigure}{1\textwidth}
\centering
\begin{tikzpicture}
\draw(0,0)coordinate(ka)node[above]{$a$} to [short,o-]++(\x/2,0)node[shift={(-0.3,-0.3)},circ]{} to [inductor,l_={$\SI{6}{\volt}\phase{0^{\circ}}\,\rms$}]++(0,-\y) to [short,-o]++(-\x/2,0)coordinate(kb)node[above]{$b$};
\draw(kb)++(0,-0.5)coordinate(c)node[below]{$c$} to [short,o-]++(\x/2,0)node[shift={(-0.3,-0.3)},circ]{} to [inductor,l_={$\SI{4}{\volt}\phase{0^{\circ}}\,\rms$}]++(0,-\y) to [short,-o]++(-\x/2,0)coordinate(kc)node[below]{$d$};
\draw(\x+0.8,0)coordinate(ke)node[above]{$e$} to [short,o-]++(-\x/2,0)node[shift={(0.3,-0.3)},circ]{} to [inductor]++(0,-\y) to [short,-o]++(\x/2,0)coordinate(kf)node[below]{$f$};
\draw(kf) to [short,o-] ++(\x/2,0) to [american voltage source,l_={$\SI{20}{\volt}\phase{0^{\circ}}\,\rms$}]++(0,\y) to [short,-o]++(-\x/2,0);
\draw(kb) to [short,o-]++(-\x/3,0) to [short]++(0,-0.5) to [short,-o]++(\x/3,0);
\end{tikzpicture}
\caption*{(ب)}
\end{subfigure}
\caption{مثال \حوالہ{مثال_مقناطیسی_کامل_ٹرانسفارمر_متعدد_لچھے} کا دور۔}
\label{شکل_مقناطیسی_کامل_ٹرانسفارمر_متعدد_لچھے}
\end{figure}

\انتہا{مثال}
%==================

\حصہء{سوالات}

%======================
\ابتدا{سوال}\شناخت{سوال_مقناطیسی_الف}
شکل \حوالہ{شکل_سوال_مقناطیسی_الف}-الف میں \عددی{v_1}، \عددی{v_2}، \عددی{v_3} اور \عددی{v_4} کے مساوات لکھیں۔
\begin{figure}
\centering
\begin{subfigure}{0.5\textwidth}
\centering
\begin{circuitikz}
\draw(0,0) rectangle ++(-\boxW,\boxH);
\draw(-0.25,\boxH/2) node[rotate=90]{\RL{بایاں دور}};
\draw(0,0.25) to [short]++(\x,0)coordinate(BL) to [inductor,l={$L_1$}]++(0,\y)coordinate(TL) to [short,i<_={$i_1$}]++(-\x,0);
\draw(\x+\x/3+\x,0.25) to [short]++(-\x,0)coordinate(BR) to [inductor,l_={$L_2$}]++(0,\y)coordinate(TR) to [short,i<^={$i_2$}]++(\x,0);
\draw($(TL)!0.5!(TR)$)node[above]{$M$};
\draw(BL)++(-0.5,0.5) node[circ]{}; 
\draw(BR)++(0.5,0.5) node[circ]{}; 
\draw(2*\x+\x/3,0) rectangle ++(\boxW,\boxH);
\draw(2*\x+\x/3,0)++(\boxW/2,\boxH/2) node[rotate=90]{\RL{دایاں دور}};
\draw(0,\boxH/2) node[right]{$\begin{aligned} &+ \\ &v_1 \\ &-  \end{aligned}$};
\draw(2*\x+\x/3,\boxH/2) node[left]{$\begin{aligned} &- \\ &v_2 \\ &+  \end{aligned}$};
\draw(\x/4,\boxH/2) node[right]{$\begin{aligned} &- \\ &v_3 \\ &+  \end{aligned}$};
\draw(2*\x+\x/3-\x/4,\boxH/2) node[left]{$\begin{aligned} &+ \\ &v_4 \\ &-  \end{aligned}$};
\end{circuitikz}
\caption*{(الف)}
\end{subfigure}%
\begin{subfigure}{0.5\textwidth}
\centering
\begin{circuitikz}
\draw(0,0) rectangle ++(-\boxW,\boxH);
\draw(-0.25,\boxH/2) node[rotate=90]{\RL{بایاں دور}};
\draw(0,0.25) to [short]++(\x,0)coordinate(BL) to [inductor,l={$L_1$}]++(0,\y)coordinate(TL) to [short,i<_={$i_1$}]++(-\x,0);
\draw(\x+\x/3+\x,0.25) to [short]++(-\x,0)coordinate(BR) to [inductor,l_={$L_2$}]++(0,\y)coordinate(TR) to [short,i>^={$i_2$}]++(\x,0);
\draw($(TL)!0.5!(TR)$)node[above]{$M$};
\draw(BL)++(-0.5,0.5) node[circ]{}; 
\draw(BR)++(0.5,0.5) node[circ]{}; 
\draw(2*\x+\x/3,0) rectangle ++(\boxW,\boxH);
\draw(2*\x+\x/3,0)++(\boxW/2,\boxH/2) node[rotate=90]{\RL{دایاں دور}};
\draw(0,\boxH/2) node[right]{$\begin{aligned} &+ \\ &v_1 \\ &-  \end{aligned}$};
\draw(2*\x+\x/3,\boxH/2) node[left]{$\begin{aligned} &- \\ &v_2 \\ &+  \end{aligned}$};
\draw(\x/4,\boxH/2) node[right]{$\begin{aligned} &- \\ &v_3 \\ &+  \end{aligned}$};
\draw(2*\x+\x/3-\x/4,\boxH/2) node[left]{$\begin{aligned} &+ \\ &v_4 \\ &-  \end{aligned}$};
\end{circuitikz}
\caption*{(ب)}
\end{subfigure}%
\caption{سوال \حوالہ{سوال_مقناطیسی_الف} اور سوال \حوالہ{سوال_مقناطیسی_ب} کے ادوار۔}
\label{شکل_سوال_مقناطیسی_الف}
\end{figure}

جوابات:
\begin{align*}
v_1&=L_1 \frac{\dif i_1}{\dif t}+M\frac{\dif i_2}{\dif t}\\
v_2&=-M\frac{\dif i_1}{\dif t}-L_2 \frac{\dif i_2}{\dif t}\\
v_3&=-L_1 \frac{\dif i_1}{\dif t}-M\frac{\dif i_2}{\dif t}\\
v_4&=M\frac{\dif i_1}{\dif t}+L_2 \frac{\dif i_2}{\dif t}
\end{align*}
\انتہا{سوال}
%=======================
\ابتدا{سوال}\شناخت{سوال_مقناطیسی_ب}
شکل \حوالہ{شکل_سوال_مقناطیسی_الف}-ب میں \عددی{v_1}، \عددی{v_2}، \عددی{v_3} اور \عددی{v_4} کے مساوات لکھیں۔

جوابات:
\begin{align*}
v_1&=L_1 \frac{\dif i_1}{\dif t}-M\frac{\dif i_2}{\dif t}\\
v_2&=-M\frac{\dif i_1}{\dif t}+L_2 \frac{\dif i_2}{\dif t}\\
v_3&=-L_1 \frac{\dif i_1}{\dif t}+M\frac{\dif i_2}{\dif t}\\
v_4&=M\frac{\dif i_1}{\dif t}-L_2 \frac{\dif i_2}{\dif t}
\end{align*}
\انتہا{سوال}
%=======================
\ابتدا{سوال}\شناخت{سوال_مقناطیسی_پ}
شکل \حوالہ{شکل_سوال_مقناطیسی_پ}-الف میں \عددی{v_1}، \عددی{v_2}، \عددی{v_3} اور \عددی{v_4} کے مساوات لکھیں۔
\begin{figure}
\centering
\begin{subfigure}{0.5\textwidth}
\centering
\begin{circuitikz}
\draw(0,0) rectangle ++(-\boxW,\boxH);
\draw(-0.25,\boxH/2) node[rotate=90]{\RL{بایاں دور}};
\draw(0,0.25) to [short,i<_={$i_1$}]++(\x,0)coordinate(BL) to [inductor,l={$L_1$}]++(0,\y)coordinate(TL) to [short]++(-\x,0);
\draw(\x+\x/3+\x,0.25) to [short]++(-\x,0)coordinate(BR) to [inductor,l_={$L_2$}]++(0,\y)coordinate(TR) to [short,i<^={$i_2$}]++(\x,0);
\draw($(TL)!0.5!(TR)$)node[above]{$M$};
\draw(BL)++(-0.5,0.5) node[circ]{}; 
\draw(TR)++(0.5,-0.5) node[circ]{}; 
\draw(2*\x+\x/3,0) rectangle ++(\boxW,\boxH);
\draw(2*\x+\x/3,0)++(\boxW/2,\boxH/2) node[rotate=90]{\RL{دایاں دور}};
\draw(0,\boxH/2) node[right]{$\begin{aligned} &+ \\ &v_1 \\ &-  \end{aligned}$};
\draw(2*\x+\x/3,\boxH/2) node[left]{$\begin{aligned} &- \\ &v_2 \\ &+  \end{aligned}$};
\draw(\x/4,\boxH/2) node[right]{$\begin{aligned} &- \\ &v_3 \\ &+  \end{aligned}$};
\draw(2*\x+\x/3-\x/4,\boxH/2) node[left]{$\begin{aligned} &+ \\ &v_4 \\ &-  \end{aligned}$};
\end{circuitikz}
\caption*{(الف)}
\end{subfigure}%
\begin{subfigure}{0.5\textwidth}
\centering
\begin{circuitikz}
\draw(0,0) rectangle ++(-\boxW,\boxH);
\draw(-0.25,\boxH/2) node[rotate=90]{\RL{بایاں دور}};
\draw(0,0.25) to [short,i>_={$i_1$}]++(\x,0)coordinate(BL) to [inductor,l={$L_1$}]++(0,\y)coordinate(TL) to [short]++(-\x,0);
\draw(\x+\x/3+\x,0.25) to [short,i>^={$i_2$}]++(-\x,0)coordinate(BR) to [inductor,l_={$L_2$}]++(0,\y)coordinate(TR) to [short]++(\x,0);
\draw($(TL)!0.5!(TR)$)node[above]{$M$};
\draw(BL)++(-0.5,0.5) node[circ]{}; 
\draw(TR)++(0.5,-0.5) node[circ]{}; 
\draw(2*\x+\x/3,0) rectangle ++(\boxW,\boxH);
\draw(2*\x+\x/3,0)++(\boxW/2,\boxH/2) node[rotate=90]{\RL{دایاں دور}};
\draw(0,\boxH/2) node[right]{$\begin{aligned} &+ \\ &v_1 \\ &-  \end{aligned}$};
\draw(2*\x+\x/3,\boxH/2) node[left]{$\begin{aligned} &- \\ &v_2 \\ &+  \end{aligned}$};
\draw(\x/4,\boxH/2) node[right]{$\begin{aligned} &- \\ &v_3 \\ &+  \end{aligned}$};
\draw(2*\x+\x/3-\x/4,\boxH/2) node[left]{$\begin{aligned} &+ \\ &v_4 \\ &-  \end{aligned}$};
\end{circuitikz}
\caption*{(ب)}
\end{subfigure}%
\caption{سوال \حوالہ{سوال_مقناطیسی_پ} اور سوال \حوالہ{سوال_مقناطیسی_ت} کے ادوار۔}
\label{شکل_سوال_مقناطیسی_پ}
\end{figure}

جوابات:
\begin{align*}
v_1&=L_1 \frac{\dif i_1}{\dif t}-M\frac{\dif i_2}{\dif t}\\
v_2&=M\frac{\dif i_1}{\dif t}-L_2 \frac{\dif i_2}{\dif t}\\
v_3&=-L_1 \frac{\dif i_1}{\dif t}+M\frac{\dif i_2}{\dif t}\\
v_4&=-M\frac{\dif i_1}{\dif t}+L_2 \frac{\dif i_2}{\dif t}
\end{align*}
\انتہا{سوال}
%=======================
\ابتدا{سوال}\شناخت{سوال_مقناطیسی_ت}
شکل \حوالہ{شکل_سوال_مقناطیسی_پ}-ب میں \عددی{v_1}، \عددی{v_2}، \عددی{v_3} اور \عددی{v_4} کے مساوات لکھیں۔

جوابات:
\begin{align*}
v_1&=-L_1 \frac{\dif i_1}{\dif t}+M\frac{\dif i_2}{\dif t}\\
v_2&=-M\frac{\dif i_1}{\dif t}+L_2 \frac{\dif i_2}{\dif t}\\
v_3&=L_1 \frac{\dif i_1}{\dif t}-M\frac{\dif i_2}{\dif t}\\
v_4&=M\frac{\dif i_1}{\dif t}-L_2 \frac{\dif i_2}{\dif t}
\end{align*}
\انتہا{سوال}
%=======================
\ابتدا{سوال}\شناخت{سوال_مقناطیسی_ٹ}
شکل \حوالہ{شکل_سوال_مقناطیسی_ٹ} میں \عددی{\bV_0} حاصل کریں۔
\begin{figure}
\centering
\begin{tikzpicture}[american voltages]
\draw(0,0) to [american voltage source,l={$8\phase{0^{\circ}}$}\,\si{\volt}]++(0,\y) to [resistor,l={$\SI{1}{\ohm}$}]++(2*\x,0)coordinate(kA) to [inductor,l_={$j3\,\si{\ohm}$}]++(0,-\y)coordinate(kB) to [short] (0,0);
\draw(2*\x+\x/3,0)coordinate(kC) to [inductor,l_={$j4\,\si{\ohm}$}]++(0,\y)coordinate(kD) to [resistor,l={$\SI{1}{\ohm}$}]++(2*\x,0) to [resistor,l={$\SI{1}{\ohm}$},v={$\hat{V}_0$}]++(0,-\y) to [short]++(-2*\x,0);
\draw (kA)++(-0.5,-0.5) node[circ]{};
\draw (kD)++(0.5,-0.5) node[circ]{};
\draw(2*\x+\x/6,\y)node[above]{$j2\,\si{\ohm}$};
%currents
\draw[stealth-]([shift={(-135:\x/4)}]\x,\y/2) arc (-135:135:\x/4);
\draw(\x,\y/2)node{$\hat{I}_1$};
\draw[stealth-]([shift={(-135:\x/4)}]2*\x+\x/3+\x,\y/2) arc (-135:135:\x/4);
\draw(3*\x+\x/3,\y/2)node{$\hat{I}_2$};
\end{tikzpicture}
\caption{سوال \حوالہ{سوال_مقناطیسی_ٹ} کا دور۔}
\label{شکل_سوال_مقناطیسی_ٹ}
\end{figure}

جواب:\عددی{1.37\phase{-30.96^{\circ}}\,\si{\volt}}
\انتہا{سوال}
%=====================
\ابتدا{سوال}\شناخت{سوال_مقناطیسی_ث}
شکل \حوالہ{شکل_سوال_مقناطیسی_ث} میں \عددی{\bV_0} حاصل کریں۔
\begin{figure}
\centering
\begin{tikzpicture}[american voltages]
\draw(0,0) to [american voltage source,l={$20\phase{0^{\circ}}$}\,\si{\volt}]++(0,\y) to [resistor,l={$\SI{1}{\ohm}$}]++(\x,0) to [capacitor,l={$-j1\,\si{\ohm}$}]++(\x,0)coordinate(kA) to [inductor,l_={$j2\,\si{\ohm}$}]++(0,-\y)coordinate(kB) to [short] (0,0);
\draw(2*\x+\x/3,0)coordinate(kC) to [inductor,l_={$j1\,\si{\ohm}$}]++(0,\y)coordinate(kD) to [resistor,l={$\SI{1}{\ohm}$}]++(2*\x,0) to [capacitor,l={$-j2\,\si{\ohm}$},v={$\hat{V}_0$}]++(0,-\y) to [short]++(-2*\x,0);
\draw (kA)++(-0.5,-0.5) node[circ]{};
\draw (kC)++(0.5,0.5) node[circ]{};
\draw(2*\x+\x/6,\y)node[above]{$j2\,\si{\ohm}$};
%currents
\draw[stealth-]([shift={(-135:\x/4)}]\x,\y/2) arc (-135:135:\x/4);
\draw(\x,\y/2)node{$\hat{I}_1$};
\draw[stealth-]([shift={(-135:\x/4)}]2*\x+\x/3+\x,\y/2) arc (-135:135:\x/4);
\draw(3*\x+\x/3,\y/2)node{$\hat{I}_2$};
\end{tikzpicture}
\caption{سوال \حوالہ{سوال_مقناطیسی_ث} کا دور۔}
\label{شکل_سوال_مقناطیسی_ث}
\end{figure}

جواب:\عددی{13.33\phase{180^{\circ}}\,\si{\volt}}
\انتہا{سوال}
%=====================
\ابتدا{سوال}\شناخت{سوال_مقناطیسی_ج}
شکل \حوالہ{شکل_سوال_مقناطیسی_ج} میں \عددی{\bV_0} حاصل کریں۔
\begin{figure}
\centering
\begin{tikzpicture}[american voltages]
\draw(0,0) to [american voltage source,l={$40\phase{-30^{\circ}}$}\,\si{\volt}]++(0,\y) to [resistor,l={$\SI{1}{\ohm}$}]++(\x,0) to [capacitor,l={$-j2\,\si{\ohm}$}]++(\x,0)coordinate(kA) to [inductor,l_={$j3\,\si{\ohm}$}]++(0,-\y)coordinate(kB) to [short] (0,0);
\draw(2*\x+\x/3,0)coordinate(kC) to [inductor,l_={$j2\,\si{\ohm}$}]++(0,\y)coordinate(kD) to [capacitor,l={$-j1\,\si{\ohm}$}]++(2*\x,0) to [resistor,l={$\SI{2}{\ohm}$},v={$\hat{V}_0$}]++(0,-\y) to [short]++(-2*\x,0);
\draw (kB)++(-0.5,0.5) node[circ]{};
\draw (kD)++(0.5,-0.5) node[circ]{};
\draw(2*\x+\x/6,\y)node[above]{$j1\,\si{\ohm}$};
%currents
\draw[stealth-]([shift={(-135:\x/4)}]\x,\y/2) arc (-135:135:\x/4);
\draw(\x,\y/2)node{$\hat{I}_1$};
\draw[stealth-]([shift={(-135:\x/4)}]2*\x+\x/3+\x,\y/2) arc (-135:135:\x/4);
\draw(3*\x+\x/3,\y/2)node{$\hat{I}_2$};
\end{tikzpicture}
\caption{سوال \حوالہ{سوال_مقناطیسی_ج} کا دور۔}
\label{شکل_سوال_مقناطیسی_ج}
\end{figure}

جواب:\عددی{22.2\phase{183.7^{\circ}}\,\si{\volt}}
\انتہا{سوال}
%=====================
\ابتدا{سوال}\شناخت{سوال_مقناطیسی_چ}
شکل \حوالہ{شکل_سوال_مقناطیسی_چ} میں \عددی{\bV_0} حاصل کریں۔
\begin{figure}
\centering
\begin{tikzpicture}[american voltages]
\draw(0,0) to [american voltage source,l={$12\phase{0^{\circ}}$}\,\si{\volt}]++(0,\y) to [resistor,l={$\SI{1}{\ohm}$}]++(\x,0)coordinate(kA) to [inductor,l={$j3\,\si{\ohm}$}]++(\x,0)coordinate(kB) to [resistor,l_={$\SI{1}{\ohm}$}]++(0,-\y) to [short] (0,0);
\draw(2*\x+\x/3,0) to [resistor,l_={$\SI{2}{\ohm}$}]++(0,\y)coordinate(kC) to [inductor,l={$j2\,\si{\ohm}$}]++(\x,0)coordinate(kD) to [resistor,l={$\SI{2}{\ohm}$},v={$\hat{V}_0$}]++(0,-\y) to [short]++(-\x,0);
\draw (kA)++(0.5,-0.3) node[circ]{};
\draw (kC)++(0.5,-0.3) node[circ]{};
\draw[stealth-stealth](2*\x-\x/4,\y+0.4) to [out=20,in=160]++(\x/2+\x/3,0);
\draw(2*\x+\x/6,\y+0.3)node[above,fill=white]{$j1\,\si{\ohm}$};
\end{tikzpicture}
\caption{سوال \حوالہ{سوال_مقناطیسی_چ} کا دور۔}
\label{شکل_سوال_مقناطیسی_چ}
\end{figure}

جواب:\عددی{1.47\phase{190.6^{\circ}}\,\si{\volt}}
\انتہا{سوال}
%=====================
\ابتدا{سوال}\شناخت{سوال_مقناطیسی_ح}
شکل \حوالہ{شکل_سوال_مقناطیسی_ح} میں \عددی{\bV_0} حاصل کریں۔
\begin{figure}
\centering
\begin{tikzpicture}[american voltages]
\draw(0,0) to [american voltage source,l={$60\phase{0^{\circ}}$}\,\si{\volt}]++(0,\y) to [capacitor,l={$-j2\,\si{\ohm}$}]++(\x,0)
 to [resistor,l={$\SI{1}{\ohm}$}]++(\x,0)coordinate(kA) to [inductor,l_={$j3\,\si{\ohm}$}]++(0,-\y)coordinate(kB) to [short] (0,0);
\draw(2*\x+\x/3,0)coordinate(kC) to [inductor,l_={$j6\,\si{\ohm}$}]++(0,\y)coordinate(kD) to [resistor,l={$\SI{1}{\ohm}$}]++(\x,0) to [capacitor,l={$-j1\,\si{\ohm}$}]++(\x,0) to [resistor,l={$\SI{1}{\ohm}$},v={$\hat{V}_0$}]++(0,-\y) to [short]++(-2*\x,0);
\draw(\x,0) to [resistor,*-*,l={$\SI{4}{\ohm}$}]++(0,\y);
\draw (kB)++(-0.5,0.5) node[circ]{};
\draw (kC)++(0.5,0.5) node[circ]{};
\draw(2*\x+\x/6,\y)node[above]{$j2\,\si{\ohm}$};
\end{tikzpicture}
\caption{سوال \حوالہ{سوال_مقناطیسی_ح} کا دور۔}
\label{شکل_سوال_مقناطیسی_ح}
\end{figure}

جواب:\عددی{9.1\phase{29.5^{\circ}}\,\si{\volt}}
\انتہا{سوال}
%=====================
\ابتدا{سوال}\شناخت{سوال_مقناطیسی_خ}
شکل \حوالہ{شکل_سوال_مقناطیسی_خ} میں \عددی{\bV_0} حاصل کریں۔
\begin{figure}
\centering
\begin{tikzpicture}[american voltages]
\draw(0,0) to [american voltage source,l={$24\phase{0^{\circ}}$}\,\si{\volt}]++(0,\y) to [resistor,l={$\SI{1}{\ohm}$}]++(\x,0) to [capacitor,l={$-j1\,\si{\ohm}$}]++(\x,0)coordinate(kA) to [inductor,l_={$j2\,\si{\ohm}$}]++(0,-\y)coordinate(kB) to [short] (0,0);
\draw(2*\x,\y)coordinate(kC) to [inductor,*-,l={$j2\,\si{\ohm}$}]++(\x,0)coordinate(kD) to [resistor,l={$\SI{2}{\ohm}$}]++(\x,0) to [american current source,l={$2\phase{0^{\circ}}\,\si{\ampere}$},v={$\hat{V}_0$}]++(0,-\y) to [short,-*]++(-2*\x,0);
\draw(0,\y)  to [short,*-]++(0,3/4*\y) to [capacitor,l={$-j2\,\si{\ohm}$}]++(4*\x,0) to [short,-*]++(0,-3/4*\y);
\draw (kA)++(-0.5,-0.5) node[circ]{};
\draw (kD)++(-0.5,0.5) node[circ]{};
\draw[stealth-stealth] (2*\x+\x/4,\y/3) to [out=0,in=-90]++(\x/2,\y/2);
\draw(2*\x+\x/2+\x/4,\y/2)node[fill=white]{$j1\,\si{\ohm}$};
\end{tikzpicture}
\caption{سوال \حوالہ{سوال_مقناطیسی_خ} کا دور۔}
\label{شکل_سوال_مقناطیسی_خ}
\end{figure}

جواب:\عددی{23.1\phase{9.73^{\circ}}\,\si{\volt}}
\انتہا{سوال}
%=====================
\ابتدا{سوال}\شناخت{سوال_مقناطیسی_د}
شکل \حوالہ{شکل_سوال_مقناطیسی_د} میں \عددی{\bI_0} حاصل کریں۔
\begin{figure}
\centering
\begin{tikzpicture}[american voltages]
\draw(0,0) to [american voltage source,l={$18\phase{0^{\circ}}$}\,\si{\volt}]++(0,2*\y) to [capacitor,l={$-j6\,\si{\ohm}$}]++(\x,0)
 to [resistor,l={$\SI{2}{\ohm}$}]++(\x,0)coordinate(kA) to [inductor,l_={$j4\,\si{\ohm}$}]++(0,-\y)coordinate(kB) to [short]++ (\x/3,0);
\draw(2*\x+\x/3,\y)coordinate(kC) to [inductor,l_={$j3\,\si{\ohm}$}]++(0,\y)coordinate(kD) to [resistor,l={$\SI{6}{\ohm}$}]++(\x,0) to [capacitor,l={$-j4\,\si{\ohm}$}]++(\x,0) to [short,i={$\bI_0$}]++(0,-2*\y) to [short](0,0);
\draw(2*\x+\x/6,0) to [resistor,*-*,l={$\SI{4}{\ohm}$}]++(0,\y);
\draw (kA)++(-0.5,-0.5) node[circ]{};
\draw (kD)++(0.5,-0.5) node[circ]{};
\draw(2*\x+\x/6,2*\y)node[above]{$j3\,\si{\ohm}$};
\end{tikzpicture}
\caption{سوال \حوالہ{سوال_مقناطیسی_د} کا دور۔}
\label{شکل_سوال_مقناطیسی_د}
\end{figure}

جواب:\عددی{1.26\phase{81.3^{\circ}}\,\si{\ampere}}
\انتہا{سوال}
%=====================
\ابتدا{سوال}\شناخت{سوال_مقناطیسی_ڈ}
شکل \حوالہ{شکل_سوال_مقناطیسی_ڈ} میں منبع کو کیا رکاوٹ نظر آتی ہے؟
\begin{figure}
\centering
\begin{tikzpicture}[american voltages]
\draw(0,0) to [american voltage source,l={$15\phase{0^{\circ}}$}\,\si{\volt}]++(0,\y) to [resistor,l={$\SI{1}{\ohm}$}]++(\x,0)  to [capacitor,l={$-j2\,\si{\ohm}$}]++(\x,0)coordinate(kA) to [inductor,l_={$j3\,\si{\ohm}$}]++(0,-\y)coordinate(kB) to [short] (0,0);
\draw(2*\x+\x/3,0)coordinate(kC) to [inductor,l_={$j2\,\si{\ohm}$}]++(0,\y)coordinate(kD) to [resistor,l={$\SI{2}{\ohm}$}]++(\x,0) to [capacitor,l={$-j1\,\si{\ohm}$}]++(\x,0) to [inductor,l={$j2\,\si{\ohm}$}]++(0,-\y) to [short]++(-2*\x,0);
\draw(\x,0) to [resistor,*-*,l={$\SI{2}{\ohm}$}]++(0,\y);
\draw (kA)++(-0.5,-0.5) node[circ]{};
\draw (kC)++(0.5,0.5) node[circ]{};
\draw(2*\x+\x/6,\y)node[above]{$j1\,\si{\ohm}$};
\end{tikzpicture}
\caption{سوال \حوالہ{سوال_مقناطیسی_ڈ} کا دور۔}
\label{شکل_سوال_مقناطیسی_ڈ}
\end{figure}

جواب:\عددی{1.35+j0.59\,\si{\ohm}}
\انتہا{سوال}
%=====================
\ابتدا{سوال}\شناخت{سوال_مقناطیسی_داخلی_رکاوٹ_الف}
شکل \حوالہ{شکل_سوال_مقناطیسی_داخلی_رکاوٹ_الف} میں داخلی رکاوٹ \عددی{\bZ} حاصل کریں۔
\begin{figure}
\centering
\begin{tikzpicture}[american voltages]
\draw(0,\y) to [resistor,o-,l={$\SI{2}{\ohm}$}]++(\x,0) to [capacitor,l={$-j1\,\si{\ohm}$}]++(\x,0)coordinate(kA) to [inductor,l_={$j2\,\si{\ohm}$}]++(0,-\y)coordinate(kB) to [short,-o] (0,0);
\draw(2*\x,\y)coordinate(kC) to [inductor,*-,l={$j2\,\si{\ohm}$}]++(\x,0)coordinate(kD) to [resistor,l={$\SI{4}{\ohm}$}]++(\x,0)
 to [capacitor,l={$-j6\,\si{\ohm}$}]++(0,-\y) to [short,-*]++(-2*\x,0);
\draw (kA)++(-0.5,-0.5) node[circ]{};
\draw (kD)++(-0.5,0.5) node[circ]{};
\draw[stealth-stealth] (2*\x+\x/4,\y/3) to [out=0,in=-90]++(\x/2,\y/2);
\draw(2*\x+\x/2+\x/4,\y/2)node[fill=white]{$j2\,\si{\ohm}$};
\draw[stealth-](\x/4,\y/2)--++(-\x/4,0)--++(0,-\y/8)node[below]{$\bZ$};
\end{tikzpicture}
\caption{سوال \حوالہ{سوال_مقناطیسی_داخلی_رکاوٹ_الف} کا دور۔}
\label{شکل_سوال_مقناطیسی_داخلی_رکاوٹ_الف}
\end{figure}

جواب:\عددی{5+j36.9\,\si{\ohm}}
\انتہا{سوال}
%=====================
\ابتدا{سوال}\شناخت{سوال_مقناطیسی_داخلی_رکاوٹ_ب}
شکل \حوالہ{شکل_سوال_مقناطیسی_داخلی_رکاوٹ_ب} میں منبع کو کیا رکاوٹ نظر آتی ہے؟
\begin{figure}
\centering
\begin{tikzpicture}[american voltages]
\draw(0,0) to [american voltage source,l={$20\phase{0^{\circ}}$}\,\si{\volt}]++(0,2*\y) to [resistor,l={$\SI{1}{\ohm}$}]++(\x,0)  to [capacitor,l={$-j2\,\si{\ohm}$}]++(\x,0) to [short]++(0,-\y/2)coordinate(kA) to [inductor,l_={$j3\,\si{\ohm}$}]++(0,-\y)coordinate(kB) to [short]++(0,-\y/2) to [short] (0,0);
\draw(2*\x+\x/3,0)coordinate(kBot) to [short] ++(0,\y/2)coordinate(kC) to [inductor,l_={$j4\,\si{\ohm}$}]++(0,\y)coordinate(kD) to [short]++(0,\y/2);
\draw(kBot) to [short]++(2*\x,0);
\draw(kBot)++(0,2*\y) to [short]++(2*\x,0);
\draw(kBot)++(\x,0)  to [capacitor,*-,l={$-j4\,\si{\ohm}$}]++(0,\y) to [resistor,-*,l={$\SI{1}{\ohm}$}]++(0,\y);
\draw(kBot)++(2*\x,0)to [resistor,-*,l_={$\SI{1}{\ohm}$}]++(0,\y)  to [inductor,*-,l_={$j2\,\si{\ohm}$}]++(0,\y) ;
\draw(kBot)++(\x,\y) to [resistor,*-*,l={$\SI{2}{\ohm}$}]++(\x,0);
\draw (kA)++(-0.5,-0.5) node[circ]{};
\draw (kC)++(0.5,0.5) node[circ]{};
\draw(2*\x+\x/6,2*\y)node[above]{$j3\,\si{\ohm}$};
\end{tikzpicture}
\caption{سوال \حوالہ{سوال_مقناطیسی_داخلی_رکاوٹ_ب} کا دور۔}
\label{شکل_سوال_مقناطیسی_داخلی_رکاوٹ_ب}
\end{figure}

جواب:\عددی{1.785-j0.5536\,\si{\ohm}}
\انتہا{سوال}
%=====================
\ابتدا{سوال}\شناخت{سوال_مقناطیسی_داخلی_رکاوٹ_پ}
شکل \حوالہ{شکل_سوال_مقناطیسی_داخلی_رکاوٹ_پ} میں \عددی{X_C} کی وہ قیمت دریافت کریں جس پر منبع کو مزاحمتی رکاوٹ نظر آتی ہے۔
\begin{figure}
\centering
\begin{tikzpicture}[american voltages]
\draw(0,0) to [american voltage source,l={$10\phase{0^{\circ}}$}\,\si{\volt}]++(0,\y) to [resistor,l={$\SI{12}{\ohm}$}]++(\x,0)coordinate(kA) to [inductor,l_={$j1\,\si{\ohm}$}]++(0,-\y)coordinate(kB) to [short] (0,0);
\draw(\x+\x/3,0)coordinate(kBot) coordinate(kC) to [inductor,l_={$j50\,\si{\ohm}$}]++(0,\y)coordinate(kD) to [resistor,l={$\SI{6}{\ohm}$}]++(\x,0) to [capacitor,l={$-jX_C\,\si{\ohm}$}]++(\x,0) to [resistor,l={$\SI{8}{\ohm}$}]++(\x,0) to [inductor,l={$j6\,\si{\ohm}$}]++(0,-\y) to [short] (kC);
\draw (kA)++(-0.5,-0.5) node[circ]{};
\draw (kD)++(0.5,-0.5) node[circ]{};
\draw(\x+\x/6,\y)node[above]{$j6\,\si{\ohm}$};
\end{tikzpicture}
\caption{سوال \حوالہ{سوال_مقناطیسی_داخلی_رکاوٹ_پ} کا دور۔}
\label{شکل_سوال_مقناطیسی_داخلی_رکاوٹ_پ}
\end{figure}

جواب:\عددی{X_C=49.3137}، \عددی{X_C=26.686}
\انتہا{سوال}
%=====================
%=====================
\ابتدا{سوال}\شناخت{سوال_مقناطیسی_داخلی_رکاوٹ_ث}
شکل \حوالہ{شکل_سوال_مقناطیسی_داخلی_رکاوٹ_ث} میں کامل ٹرانسفارمر استعمال کیا گیا ہے۔ تمام دباو اور رو دریافت کریں۔آپ دیکھیں گے کہ داخلی اور خارجی متغیرات ہم قدم ہیں۔
\begin{figure}
\centering
\begin{tikzpicture}[american voltages]
\draw(0,0)node[transformer core](T){};
\draw (T)node[above]{$1:2$};
\draw(T.A1)++(0.4,-0.4)node[circ]{};
\draw(T.B1)++(-0.4,-0.4)node[circ]{};
\draw($(T.A1)!0.5!(T.A2)$)node{$\begin{aligned} &+ \\ &\bV_1 \\ &- \end{aligned}$};
\draw($(T.B1)!0.5!(T.B2)$)node{$\begin{aligned} &+ \\ & \bV_2 \\ &- \end{aligned}$};
\draw(T.A2) to [short]++(-\x,0)to [american voltage source,l={$10\phase{0^{\circ}}$}\,\si{\volt}]++(0,\y)coordinate(kTL);
\draw(T.A1) to [resistor,l_={$\SI{1}{\ohm}$},i_<={$\bI_1$}]++(-\x,0)-|(kTL);
\draw(T.B1) to [resistor,l={$\SI{1}{\ohm}$},i={$\bI_2$}]++(\x,0) to [resistor,l={$\SI{1}{\ohm}$},v={$\bV_0$}]++(0,-\y)coordinate(kBR);
\draw(T.B2)-|(kBR);
\end{tikzpicture}
\caption{سوال \حوالہ{سوال_مقناطیسی_داخلی_رکاوٹ_ث} کا دور۔}
\label{شکل_سوال_مقناطیسی_داخلی_رکاوٹ_ث}
\end{figure}

جواب:\عددی{\bI_1=\tfrac{20}{3}\phase{0^{\circ}}\,\si{\ampere}}، \عددی{\bI_2=\tfrac{10}{3}\phase{0^{\circ}}\,\si{\ampere}}،
 \عددی{\bV_1=\tfrac{10}{3}\phase{0^{\circ}}\,\si{\volt}}،\\ \عددی{\bV_2=\tfrac{20}{3}\phase{0^{\circ}}\,\si{\volt}}،  
\عددی{\bV_0=\tfrac{10}{3}\phase{0^{\circ}}\,\si{\volt}}

\انتہا{سوال}
%=====================
\ابتدا{سوال}\شناخت{سوال_مقناطیسی_داخلی_رکاوٹ_ج}
شکل \حوالہ{شکل_سوال_مقناطیسی_داخلی_رکاوٹ_ج} میں کامل ٹرانسفارمر استعمال کیا گیا ہے۔ تمام دباو اور رو دریافت کریں۔آپ دیکھیں گے کہ داخلی اور خارجی متغیرات میں \عددی{180^{\circ}} زاویائی فرق پایا جاتا ہے۔
\begin{figure}
\centering
\begin{tikzpicture}[american voltages]
\draw(0,0)node[transformer core](T){};
\draw (T)node[above]{$1:2$};
\draw(T.A1)++(0.4,-0.4)node[circ]{};
\draw(T.B2)++(-0.4,0.4)node[circ]{};
\draw($(T.A1)!0.5!(T.A2)$)node{$\begin{aligned} &+ \\ &\bV_1 \\ &- \end{aligned}$};
\draw($(T.B1)!0.5!(T.B2)$)node{$\begin{aligned} &+ \\ & \bV_2 \\ &- \end{aligned}$};
\draw(T.A2) to [short]++(-\x,0)to [american voltage source,l={$10\phase{0^{\circ}}$}\,\si{\volt}]++(0,\y)coordinate(kTL);
\draw(T.A1) to [resistor,l_={$\SI{1}{\ohm}$},i_<={$\bI_1$}]++(-\x,0)-|(kTL);
\draw(T.B1) to [resistor,l={$\SI{1}{\ohm}$},i={$\bI_2$}]++(\x,0) to [resistor,l={$\SI{1}{\ohm}$},v={$\bV_0$}]++(0,-\y)coordinate(kBR);
\draw(T.B2)-|(kBR);
\end{tikzpicture}
\caption{سوال \حوالہ{سوال_مقناطیسی_داخلی_رکاوٹ_ج} کا دور۔}
\label{شکل_سوال_مقناطیسی_داخلی_رکاوٹ_ج}
\end{figure}

جواب:\عددی{\bI_1=\tfrac{20}{3}\phase{0^{\circ}}\,\si{\ampere}}، \عددی{\bI_2=\tfrac{10}{3}\phase{180^{\circ}}\,\si{\ampere}}،
 \عددی{\bV_1=\tfrac{10}{3}\phase{0^{\circ}}\,\si{\volt}}،\\ \عددی{\bV_2=\tfrac{20}{3}\phase{180^{\circ}}\,\si{\volt}}،  
\عددی{\bV_0=\tfrac{10}{3}\phase{180^{\circ}}\,\si{\volt}}

\انتہا{سوال}
%=====================
\ابتدا{سوال}\شناخت{سوال_مقناطیسی_داخلی_رکاوٹ_ت}
شکل \حوالہ{شکل_سوال_مقناطیسی_داخلی_رکاوٹ_ت} میں کامل ٹرانسفارمر استعمال کیا گیا ہے۔ تمام دباو اور رو دریافت کریں۔
\begin{figure}
\centering
\begin{tikzpicture}[american voltages]
\draw(0,0)node[transformer core](T){};
\draw (T)node[above]{$1:4$};
\draw(T.A1)++(0.4,-0.4)node[circ]{};
\draw(T.B1)++(-0.4,-0.4)node[circ]{};
\draw($(T.A1)!0.5!(T.A2)$)node{$\begin{aligned} &+ \\ &\bV_1 \\ &- \end{aligned}$};
\draw($(T.B1)!0.5!(T.B2)$)node{$\begin{aligned} &+ \\ & \bV_2 \\ &- \end{aligned}$};
\draw(T.A2) to [short]++(-\x,0)to [american voltage source,l={$40\phase{0^{\circ}}$}\,\si{\volt}]++(0,\y)coordinate(kTL);
\draw(T.A1) to [resistor,l_={$\SI{1}{\ohm}$},i_<={$\bI_1$}]++(-\x,0)-|(kTL);
\draw(T.B1) to [resistor,l={$\SI{4}{\ohm}$},i={$\bI_2$}]++(\x,0) to [inductor,l={$j8\,\si{\ohm}$}]++(0,-\y)coordinate(kBR);
\draw(T.B2)-|(kBR);
\end{tikzpicture}
\caption{سوال \حوالہ{سوال_مقناطیسی_داخلی_رکاوٹ_ت} کا دور۔}
\label{شکل_سوال_مقناطیسی_داخلی_رکاوٹ_ت}
\end{figure}

جواب:\عددی{\bI_1=29.7\phase{-21.8^{\circ}}\,\si{\ampere}}، \عددی{\bI_2=7.43\phase{-21.8^{\circ}}\,\si{\ampere}}، \\
\عددی{\bV_1=16.6\phase{41.6^{\circ}}\,\si{\volt}}، \عددی{\bV_2=66.4\phase{41.6^{\circ}}\,\si{\volt}}
\انتہا{سوال}
%=====================
\ابتدا{سوال}\شناخت{سوال_مقناطیسی_داخلی_رکاوٹ_ٹ}
شکل \حوالہ{شکل_سوال_مقناطیسی_داخلی_رکاوٹ_ٹ} میں \عددی{\bV_0} دریافت کریں۔
\begin{figure}
\centering
\begin{tikzpicture}[american voltages]
\draw(0,0)node[transformer core](T){};
\draw (T)node[above]{$3:1$};
\draw(T.A2)++(0.4,0.4)node[circ]{};
\draw(T.B1)++(-0.4,-0.4)node[circ]{};
\draw(T.A2) to [short]++(-2*\x,0)to [american voltage source,l={$150\phase{45^{\circ}}$}\,\si{\volt}]++(0,\y)coordinate(kTL);
\draw(T.A1) to [resistor,l_={$\SI{1}{\ohm}$}]++(-\x,0) to [capacitor,l_={$-j2\,\si{\ohm}$}]++(-\x,0)-|(kTL);
\draw(T.B1) to [resistor,l={$\SI{1}{\ohm}$}]++(\x,0) to [capacitor,l={$-j4\,\si{\ohm}$},v={$\bV_0$}]++(0,-\y)coordinate(kBR);
\draw(T.B2)-|(kBR);
\end{tikzpicture}
\caption{سوال \حوالہ{سوال_مقناطیسی_داخلی_رکاوٹ_ٹ} کا دور۔}
\label{شکل_سوال_مقناطیسی_داخلی_رکاوٹ_ٹ}
\end{figure}

جواب:\عددی{\bV_0=45.8\phase{210.3^{\circ}}\,\si{\volt}}
\انتہا{سوال}
%=================================
\ابتدا{سوال}\شناخت{سوال_مقناطیسی_داخلی_رکاوٹ_چ}
شکل \حوالہ{شکل_سوال_مقناطیسی_داخلی_رکاوٹ_چ} میں \عددی{\bV_0} دریافت کریں۔
\begin{figure}
\centering
\begin{tikzpicture}[american voltages]
\draw(0,0)node[transformer core](T){};
\draw (T)node[above]{$2:1$};
\draw(T.A1)++(0.4,-0.4)node[circ]{};
\draw(T.B1)++(-0.4,-0.4)node[circ]{};
\draw(T.A2) to [short]++(-2*\x,0)coordinate(kBotL) to [american voltage source,l={$24\phase{0^{\circ}}$}\,\si{\volt}]++(0,\y)coordinate(kTL);
\draw(T.A1)  to [capacitor,l_={$-j4\,\si{\ohm}$}]++(-\x,0) to [resistor,l_={$\SI{4}{\ohm}$}]++(-\x,0)-|(kTL);
\draw(T.A1)++(-\x,0) to [american current source,*-,l={$\SI{2}{\phase{0^{\circ}}}\,\si{\ampere}$}]++(0,-\y)coordinate(kBotC)--($(T.A2)!(kBotC)!(kBotL)$)node[circ]{};
\draw(T.B1) to [short]++(\x,0)coordinate(kTopR)to [inductor,l={$j2\,\si{\ohm}$}]++(\x,0) to [resistor,l={$\SI{4}{\ohm}$},v={$\bV_0$}]++(0,-\y)coordinate(kBR);
\draw(T.B2)-|(kBR)coordinate[pos=0.2](kA)coordinate[pos=0.8](kB);
\draw(kTopR) to [capacitor,*-,l_={$-j2\,\si{\ohm}$}]++(0,-\y)coordinate(kBotCR)--($(kA)!(kBotCR)!(kB)$)node[circ]{};
\end{tikzpicture}
\caption{سوال \حوالہ{سوال_مقناطیسی_داخلی_رکاوٹ_چ} کا دور۔}
\label{شکل_سوال_مقناطیسی_داخلی_رکاوٹ_چ}
\end{figure}

جواب:\عددی{\bV_0=4.44\phase{-33.7^{\circ}}\,\si{\volt}}
\انتہا{سوال}
%=================================
\ابتدا{سوال}\شناخت{سوال_مقناطیسی_ٹرانسفارمر_داخلی_رکاوٹ_الف}
شکل \حوالہ{شکل_سوال_مقناطیسی_ٹرانسفارمر_داخلی_رکاوٹ_الف} میں منبع کو نظر آنے والی رکاوٹ حاصل کریں۔
\begin{figure}
\centering
\begin{tikzpicture}[american voltages]
\draw(0,0)node[transformer core](T){};
\draw (T)node[above]{$1:2$};
\draw(T.A2)++(0.4,0.4)node[circ]{};
\draw(T.B1)++(-0.4,-0.4)node[circ]{};
\draw(T.A2) to [short]++(-\x,0)to [american voltage source,l={$\bV_S$}]++(0,\y)coordinate(kTL);
\draw(T.A1) to [resistor,l_={$\SI{2}{\ohm}$}]++(-\x,0)-|(kTL);
\draw(T.B1) to [resistor,l={$\SI{2}{\ohm}$}]++(\x,0) to [inductor,l={$j4\,\si{\ohm}$}]++(0,-\y)coordinate(kBR);
\draw(T.B2)-|(kBR);
\end{tikzpicture}
\caption{سوال \حوالہ{سوال_مقناطیسی_ٹرانسفارمر_داخلی_رکاوٹ_الف} کا دور۔}
\label{شکل_سوال_مقناطیسی_ٹرانسفارمر_داخلی_رکاوٹ_الف}
\end{figure}

جواب:\عددی{\bZ=2.5+j1\,\si{\ohm}}
\انتہا{سوال}
%=================================
\ابتدا{سوال}\شناخت{سوال_مقناطیسی_ٹرانسفارمر_داخلی_رکاوٹ_ب}
شکل \حوالہ{شکل_سوال_مقناطیسی_ٹرانسفارمر_داخلی_رکاوٹ_ب} میں داخلی رکاوٹ دریافت کریں۔
\begin{figure}
\centering
\begin{tikzpicture}[american voltages]
\draw(0,0)node[transformer core](T){};
\draw (T)node[above]{$4:1$};
\draw(T.A1)++(0.4,-0.4)node[circ]{};
\draw(T.B1)++(-0.4,-0.4)node[circ]{};
\draw(T.A2) to [short]++(-2*\x,0)coordinate(kBotL) to [american voltage source,l={$\bV_S$}]++(0,\y)coordinate(kTL);
\draw(T.A1)  to [capacitor,l_={$-j8\,\si{\ohm}$}]++(-\x,0) to [resistor,l_={$\SI{16}{\ohm}$}]++(-\x,0)-|(kTL);
\draw(T.A1)++(-\x,0) to [inductor,*-,l={$j20\,\si{\ohm}$}]++(0,-\y)coordinate(kBotC)--($(T.A2)!(kBotC)!(kBotL)$)node[circ]{};
\draw(T.B1) to [short]++(\x,0)coordinate(kTopR)to [inductor,l={$j2\,\si{\ohm}$}]++(\x,0) to [resistor,l={$\SI{1}{\ohm}$}]++(0,-\y)coordinate(kBR);
\draw(T.B2)-|(kBR)coordinate[pos=0.2](kA)coordinate[pos=0.8](kB);
\draw(kTopR) to [capacitor,*-,l_={$-j2\,\si{\ohm}$}]++(0,-\y)coordinate(kBotCR)--($(kA)!(kBotCR)!(kB)$)node[circ]{};
\end{tikzpicture}
\caption{سوال \حوالہ{سوال_مقناطیسی_ٹرانسفارمر_داخلی_رکاوٹ_ب} کا دور۔}
\label{شکل_سوال_مقناطیسی_ٹرانسفارمر_داخلی_رکاوٹ_ب}
\end{figure}

جواب:\عددی{21.69+j21.78\,\si{\ohm}}
\انتہا{سوال}
%=================================
\ابتدا{سوال}\شناخت{سوال_مقناطیسی_ٹرانسفارمر_داخلی_رکاوٹ_پ}
شکل \حوالہ{شکل_سوال_مقناطیسی_ٹرانسفارمر_داخلی_رکاوٹ_پ} میں داخلی رکاوٹ دریافت کریں۔
\begin{figure}
\centering
\begin{tikzpicture}[american voltages]
\draw(0,0)node[transformer core](T){};
\draw (T)node[above]{$2:1$};
\draw(T.A1)++(0.4,-0.4)node[circ]{};
\draw(T.B1)++(-0.4,-0.4)node[circ]{};
\draw(2*\x,0)node[transformer core](Ta){};
\draw (Ta)node[above]{$1:4$};
\draw(Ta.A1)++(0.4,-0.4)node[circ]{};
\draw(Ta.B1)++(-0.4,-0.4)node[circ]{};
%
\draw(T.B1) to [inductor,l={$j2\,\si{\ohm}$}]++(\x,0)--(Ta.A1);
\draw(T.B2) to [resistor,l={$\SI{2}{\ohm}$}]++(\x,0)--(Ta.A2);

\draw(T.A2) to [short]++(-\x,0)coordinate(kBotL) to [american voltage source,l={$\bV_S$}]++(0,\y)coordinate(kTL);
\draw(T.A1)  to [capacitor,l_={$-j16\,\si{\ohm}$}]++(-\x,0) -|(kTL);
\draw(Ta.B1)  to [resistor,l={$\SI{48}{\ohm}$}]++(\x,0) to [capacitor,l={$-j32\,\si{\ohm}$}]++(0,-\y)|-(Ta.B2);
\end{tikzpicture}
\caption{سوال \حوالہ{سوال_مقناطیسی_ٹرانسفارمر_داخلی_رکاوٹ_پ} کا دور۔}
\label{شکل_سوال_مقناطیسی_ٹرانسفارمر_داخلی_رکاوٹ_پ}
\end{figure}

جواب:\عددی{20-j16\,\si{\ohm}}
\انتہا{سوال}
%=================================
\ابتدا{سوال}\شناخت{سوال_مقناطیسی_ٹرانسفارمر_داخلی_رکاوٹ_ت}
شکل \حوالہ{شکل_سوال_مقناطیسی_ٹرانسفارمر_داخلی_رکاوٹ_ت} میں \عددی{\SI{32}{\ohm}} خارجی مزاحمت والے  ایمپلیفائر کے ساتھ ٹرانسفارمر کے ذریعہ  \عددی{\SI{8}{\ohm}} کا لاوڈ سپیکر جوڑا گیا ہے۔ایمپلیفائر کو اس کے تھونن مساوی دور سے ظاہر کیا گیا ہے۔لاوڈ سپیکر میں زیادہ سے زیادہ طاقت منتقل کرنے کی خاطر ٹرانسفارمر کی \عددی{\tfrac{N_1}{N_2}} دریافت کریں۔
\begin{figure}
\centering
\begin{tikzpicture}[american voltages]
\draw(0,0)node[transformer core](T){};
\draw (T)node[above]{$N_1:N_2$};
\draw(T.A1)++(0.4,-0.4)node[circ]{};
\draw(T.B1)++(-0.4,-0.4)node[circ]{};
%

\draw(T.A2) to [short]++(-\x,0)coordinate(kBotL) to [american voltage source,l={$\bV_ m$}]++(0,\y)coordinate(kTL);
\draw(T.A1)  to [resistor,l_={$\SI{32}{\ohm}$}]++(-\x,0) -|(kTL);
\draw(T.B1)  to [short]++(\x,0) to [resistor,l={$\SI{8}{\ohm}$}]++(0,-\y)|-(T.B2);
\end{tikzpicture}
\caption{سوال \حوالہ{سوال_مقناطیسی_ٹرانسفارمر_داخلی_رکاوٹ_ت} کا دور۔}
\label{شکل_سوال_مقناطیسی_ٹرانسفارمر_داخلی_رکاوٹ_ت}
\end{figure}

جواب:\عددی{\tfrac{N_1}{N_2}=\tfrac{2}{1}}
\انتہا{سوال}
%=================================
\ابتدا{سوال}\شناخت{سوال_مقناطیسی_ٹرانسفارمر_معلوم_الف}
شکل \حوالہ{شکل_سوال_مقناطیسی_ٹرانسفارمر_معلوم_الف} میں \عددی{\bI_m} معلوم کریں۔
\begin{figure}
\centering
\begin{tikzpicture}[american voltages]
\draw(0,0)node[transformer core](T){};
\draw (T)node[above]{$1:4$};
\draw(T.A1)++(0.4,-0.4)node[circ]{};
\draw(T.B1)++(-0.4,-0.4)node[circ]{};
%

\draw(T.A2) to [short]++(-2*\x,0)coordinate(kBotL) to [american current source,l={$\bI_ m$}]++(0,\y)coordinate(kTL);
\draw(T.A1)  to [capacitor,l_={$-j2\,\si{\ohm}$}]++(-\x,0)coordinate(resT)--++(-\x,0) -|(kTL);
\draw(T.B1)  to [inductor,l={$j1\,\si{\ohm}$}]++(\x,0) to [resistor,l={$\SI{8}{\ohm}$},i={$4\phase{60^{\circ}}\,\si{\ampere}$}]++(0,-\y)|-(T.B2);
\draw(resT) to [resistor,*-*,l={$\SI{4}{\ohm}$}] ($(T.A2)!(resT)!(kBotL)$);
\end{tikzpicture}
\caption{سوال \حوالہ{سوال_مقناطیسی_ٹرانسفارمر_معلوم_الف} کا دور۔}
\label{شکل_سوال_مقناطیسی_ٹرانسفارمر_معلوم_الف}
\end{figure}

جواب:\عددی{\bI_m=18.2\phase{34.8^{\circ}}\,\si{\ampere}}
\انتہا{سوال}
%=================================
\ابتدا{سوال}\شناخت{سوال_مقناطیسی_ٹرانسفارمر_معلوم_ب}
شکل \حوالہ{شکل_سوال_مقناطیسی_ٹرانسفارمر_معلوم_ب} میں \عددی{v_0(t)} معلوم کریں۔
\begin{figure}
\centering
\begin{tikzpicture}[american voltages]
\draw(0,0)node[transformer core](T){};
\draw (T)node[above]{$1:2$};
\draw(T.A1)++(0.4,-0.4)node[circ]{};
\draw(T.B1)++(-0.4,-0.4)node[circ]{};
%
\draw(T.A2) to [short]++(-\x,0)coordinate(kBotL) to [american voltage source,l={$10\cos 1000t \,\si{\volt}$}]++(0,\y)coordinate(kTL);
\draw(T.A1)  to [resistor,l_={$\SI{2}{\ohm}$}]++(-\x,0) -|(kTL);
\draw(T.B1)  to [capacitor,l={$\SI{500}{\micro\farad}$}]++(\x,0) to [resistor,l_={$\SI{4}{\ohm}$},v^<={$v_0(t)$}]++(0,-\y)|-(T.B2);
\end{tikzpicture}
\caption{سوال \حوالہ{سوال_مقناطیسی_ٹرانسفارمر_معلوم_ب} کا دور۔}
\label{شکل_سوال_مقناطیسی_ٹرانسفارمر_معلوم_ب}
\end{figure}

جواب:\عددی{v_0(t)=12.6\cos(1000t+18.4^{\circ})\,\si{\volt}}
\انتہا{سوال}
%=================================

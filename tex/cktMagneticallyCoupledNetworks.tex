\باب{مقناطیسی جڑے ادوار}

\حصہ{مشترکہ امالہ}
شکل \حوالہ{شکل_مقناطیسی_خود_امالہ} میں \عددی{N} چکر کا \اصطلاح{لچھا}\فرہنگ{لچھا}\حاشیہب{coil}\فرہنگ{coil} دکھایا گیا ہے جس میں \عددی{i} رو گزر رہی ہے۔رو کے گزرنے سے لچھے میں \عددی{\phi} \اصطلاح{مقناطیسی بہاو}\فرہنگ{مقناطیسی بہاو}\فرہنگ{بہاو!مقناطیسی}\حاشیہب{magnetic flux}\فرہنگ{magnetic flux}\فرہنگ{flux!magnetic} پیدا ہوتا ہے۔مقناطیسی بہاو \عددی{\phi} لچھے کے تمام چکروں کے اندر سے گزرنے کی صورت میں لچھے کا ارتباط بہاو \عددی{\lambda} درج ذیل ہے۔
\begin{align}\label{مساوات_مشترک_ارتباط_بہاو_الف}
\lambda=N \phi
\end{align}
اس کتاب میں صرف خطی نظام پر غور کیا گیا ہے۔خطی صورت میں ارتباط بہاو اور رو کا تعلق درج ذیل ہے
\begin{align}\label{مساوات_مشترک_ارتباط_بہاو_ب}
\lambda=L i
\end{align}
جہاں مساوات کے مستقل \عددی{L} کو \اصطلاح{خود امالہ}\فرہنگ{امالہ!خود}\فرہنگ{خود امالہ}\حاشیہب{self inductance}\فرہنگ{inductance!self}\فرہنگ{self inductance} یا \اصطلاح{امالہ} کہتے ہیں۔باب \حوالہ{باب_برق_گیر_امالہ_گیر} میں ہم امالہ پر غور کر چکے ہیں۔درج بالا دو مساوات کو ملاتے ہوئے  بہاو اور رو کا تعلق ملتا ہے۔
\begin{align}
\phi=\frac{Li}{N}
\end{align}
قانون فیراڈے کے تحت بدلتی ارتباط بہاو لچھے میں امالی دباو پیدا کرتا ہے۔
\begin{align}
v=\frac{\dif \lambda}{\dif t}
\end{align}
مساوات \حوالہ{مساوات_مشترک_ارتباط_بہاو_ب} کو درج بالا مساوات میں پر کرتے ہیں۔
\begin{align*}
v=\frac{\dif \lambda}{\dif t}=\frac{\dif (Li)}{\dif t}=L\frac{\dif i}{\dif t}+i\frac{\dif L}{\dif t}
\end{align*}
مستقل امالہ کی صوت میں اس مساوات سے امالہ کی جانی پہچانی درج ذیل مساوات حاصل ہوتی ہے۔
\begin{align}
v=L\frac{\dif i}{\dif t}
\end{align}
اس کتاب میں مستقل امالہ پر ہی غور کیا جائے گا۔
%
\begin{figure}
\centering
\begin{tikzpicture}[american voltages]
\pgfmathsetmacro{\lx}{1}
\pgfmathsetmacro{\ly}{0.2}
\pgfmathsetmacro{\yDiv}{1+\ly*sin(1980)}
%
\draw[dashed,gray] (\lx,\y/2) circle (1 cm and 1.5 cm);
\draw(\lx+1,\y/2)node[fill=white]{$\phi$};
\draw[domain=-180:5.5*360,samples=500,variable=\t,mark position=0(kBot)]  plot ({\lx*cos(\t)},{(\t/1980+\ly*sin(\t))/\yDiv*\y})coordinate(kTop);
\draw(kBot) to [short]++(-2*\x,0) to [american current source,l={$i$}]++(0,\y+0.15) to [short]++(2*\x,0) to [short] (kTop);
\draw(-\lx,\y/2)node[left]{\RL{$N$ چکر}};
\draw(-2*\x,\y) to [open,v={$v$}]++(0,-\y);
\end{tikzpicture}
\caption{خود امالہ کی تعریف۔}
\label{شکل_مقناطیسی_خود_امالہ}
\end{figure}

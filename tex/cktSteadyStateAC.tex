\باب{برقرار حالت بدلتی رو}
جبری تفاعل میں یکدم تبدیلی سے دور عارضی حالت اختیار کرتا ہے۔محدود قیمت کے وقتی مستقل کی صورت میں آخر کار عارضی دورانیہ گزر جاتا ہے اور دور ایک بار پھر برقرار حالت اختیار کر لیتا ہے۔جبری تفاعل میں یکدم تبدیلی کی غیر موجودگی میں دور برقرار صورت میں رہتا ہے۔اس باب میں ایسے ہی ادوار پر غور کیا جائے گا جن کے جبری تفاعل میں یکدم تبدیلی نہیں پائی جاتی۔ایسی صورت میں جبری حل ہی مکمل حل ہو گا۔اس باب میں مکمل حل  سے مراد جبری حل ہو گا۔ 
 
\حصہ{مخلوط اعداد}
\اصطلاح{حقیقی}\فرہنگ{حقیقی!عدد}\حاشیہب{real number}\فرہنگ{real!number} عدد اور \اصطلاح{خیالی}\فرہنگ{خیالی!عدد}\حاشیہب{imaginary number}\فرہنگ{imaginary!number} عدد کے مجموعے کو \اصطلاح{مخلوط}\فرہنگ{مخلوط!عدد}\حاشیہب{complex number}\فرہنگ{complex!number} عدد کہتے ہیں۔مخلوط اعداد کو \اصطلاح{مخلوط سطح}\فرہنگ{مخلوط سطح}\حاشیہب{complex plane}\فرہنگ{complex!plane} پر دکھایا جایا ہے۔مخلوط سطح پر افقی محدد حقیقی اعداد کو ظاہر کرتا ہے جبکہ عمودی محدد خیالی اعداد کو ظاہر کرتا ہے۔

شکل \حوالہ{شکل_بدلتا_مخلوط_اعداد_لکھنا}-الف میں مخلوط عدد \عددی{3+j2} دکھایا گیا ہے۔اسی شکل میں ایک مستطیل بھی دکھایا گیا ہے۔اس عدد کے حقیقی اور خیالی اجزاء مستطیل کے اطراف ہیں۔یوں مخلوط عدد کو حقیقی اور خیالی اجزاء کے مجموعے یعنی \عددی{3+j2} کے طرز پر لکھنے کو \اصطلاح{مستطیلی طرز}\فرہنگ{مستطیلی طرز}\حاشیہب{rectangular form}\فرہنگ{rectangular form} کہتے ہیں۔ 

شکل \حوالہ{شکل_بدلتا_مخلوط_اعداد_لکھنا}-الف میں مخلوط نقطہ \عددی{(3+j2)} سے محدد کے مرکز \عددی{(0,0)} تک لکیر کھینچی گئی ہے۔اس لکیر کی لمبائی \عددی{r} کو مسئلہ فیثاغورث کی مدد سے
\begin{align*}
r=\sqrt{3^2+2^2}=\sqrt{13}
\end{align*}
لکھا جا سکتا ہے۔اسی طرح افقی محدد سے لکیر تک کا زاویہ درج ذیل ہو گا۔
\begin{align*}
\theta=\tan^{-1}\frac{2}{3}=33.69^{\circ}
\end{align*}
شکل \حوالہ{شکل_بدلتا_مخلوط_اعداد_لکھنا}-ب میں اسی مخلوط عدد کو \عددی{r\phase{\theta}} کی شکل میں دکھایا گیا ہے۔مخلوط عدد کو حیطے اور زاویے سے ظاہر کرنے کو \اصطلاح{زاویائی طرز}\فرہنگ{زاویائی طرز}\حاشیہب{angular form}\فرہنگ{angular form} کہتے ہیں۔

کسی بھی مخلوط عدد \عددی{m} کو 
\begin{align}
m=x+jy\quad \quad \text{\RL{مستطیل طرز}}
\end{align}
یا
\begin{align}
m=r\phase{\theta} \quad \quad \text{\RL{زاویائی طرز}}
\end{align}
میں لکھا جا سکتا ہے جہاں مستطیلی طرز سے زاویائی طرز درج ذیل طریقے سے حاصل کی جاتی ہے
\begin{gather}
\begin{aligned}\label{مساوات_بدلتا_مستطیل_سے_زاویائی}
r&=\sqrt{x^2+y^2}\\
\theta&=\tan^{-1}\frac{y}{x}
\end{aligned}
\end{gather}
جبکہ زاویائی طرز سے مستطیل طرز درج ذیل سے حاصل کی جاتی ہے۔
\begin{gather}
\begin{aligned}
x&=r \cos \theta\\
y&=r\sin \theta
\end{aligned}
\end{gather}
%
\begin{figure}
\centering
\begin{subfigure}{0.5\textwidth}
\centering
\begin{tikzpicture}
\pgfmathsetmacro{\ang}{atan(2/3)};
\draw(0,0)--++(3.2,0)node[right]{حقیقی};
\draw(0,0)--++(0,2.7)node[left]{خیالی};
\foreach \x in {1,2,3}{\draw(\x,0)--++(0,-0.1)node[below]{$\x$};}
\foreach \y in {1,2}{\draw(0,\y)--++(-0.1,0)node[left]{$j \y$};}
\draw[gray](3,0)--++(0,2)--++(-3,0);
\draw[gray,dashed](0,0)--(3,2)node[pos=0.6,fill=white]{r};
\draw[fill](3,2) circle (1.5pt)node[right]{$3+j2$};
\draw[gray,-stealth]([shift={(0:0.8)}]0,0) arc (0:\ang:0.8);
\draw[gray](2/3*\ang:0.9)node[right]{$\theta$};
\end{tikzpicture}
\caption*{(الف) مخلوط عدد لکھنے کی مستطیل طرز۔}
\end{subfigure}%
\begin{subfigure}{0.5\textwidth}
\centering
\begin{tikzpicture}
\pgfmathsetmacro{\ang}{atan(2/3)};
\draw(0,0)--++(4.2,0)node[right]{حقیقی};
\draw(0,0)--++(0,2.7)node[left]{خیالی};
\foreach \x in {1,2,3,4}{\draw(\x,0)--++(0,-0.1)node[below]{$\x$};}
\foreach \y in {1,2}{\draw(0,\y)--++(-0.1,0)node[left]{$j \y$};}
\draw[gray](0,0)--++(3,2)node[pos=0.6,fill=white]{$\sqrt{13}$};
\draw[fill](3,2) circle (1.5pt)node[right]{$\sqrt{13}\phase{33.69^{\circ}}$};
\draw[gray,-stealth]([shift={(0:0.8)}]0,0) arc (0:\ang:0.8);
\draw[gray](2/3*\ang:0.9)node[right]{$33.69^{\circ}$};
\end{tikzpicture}
\caption*{(ب) مخلوط عدد لکھنے کی زاویائی طرز۔}
\end{subfigure}%
\caption{مخلوط اعداد کو لکھنے کے طریقے۔}
\label{شکل_بدلتا_مخلوط_اعداد_لکھنا}
\end{figure}

مخلوط اعداد کو جمع، منفی، ضرب اور تقسیم کرنے کی چند مثالیں دیکھتے ہیں۔

%================================
\ابتدا{مثال}
مخلوط اعداد \عددی{a=2+j3} اور \عددی{b=4+j5} دیے گئے ہیں۔درج ذیل حاصل کریں۔
\begin{align*}
a+b, \quad \quad a-b, \quad \quad a b, \quad \quad \frac{a}{b}
\end{align*}

حل:مخلوط اعداد جمع (منفی) کرتے وقت حقیقی اجزاء کو علیحدہ جمع (منفی) کیا جاتا ہے اور خیالی اجزاء کو علیحدہ جمع (منفی) کیا جاتا ہے۔
\begin{align*}
a+b&=(2+j3)+(4+j5)=(2+4)+j(3+5)=6+j8\\
a-b&=(2+j3)-(4+j5)=(2-4)+j(3-5)=-2-j2
\end{align*}
مخلوط اعداد کو ضرب دیتے ہوئے \عددی{j^2=(\sqrt{-1})^2=-1} لکھا جاتا ہے۔
\begin{align*}
ab&=(2+j3)(4+j5)=8+j10+j12+j^2 15=(8-15)+j(10+12)=-7+j22
\end{align*}
مخلوط اعداد کو تقسیم کرتے ہیں۔
\begin{align*}
\frac{a}{b}&=\frac{2+j3}{4+j5}\\
&=\left(\frac{2+j3}{4+j5}\right)\left(\frac{4-j5}{4-j5}\right)\\
&=\frac{8-j10+j12-j^2 15}{4^2-(j 5)^2}\\
&=\frac{23+j2}{16+25}\\
&=\frac{23}{41}+j\frac{2}{41}\\
&=0.56098+j0.04878
\end{align*}
\انتہا{مثال}
%===============================
\ابتدا{مثال}
گزشتہ مثال میں مخلوط اعداد کو زاویائی طرز پر لکھتے ہوئے \عددی{ab} اور \عددی{\tfrac{a}{b}} حاصل کریں۔

حل:مساوات \حوالہ{مساوات_بدلتا_مستطیل_سے_زاویائی} استعمال کرتے ہوئے \عددی{a=2+j3} کا حیطہ اور زاویہ حاصل کرتے ہیں۔
\begin{align*}
r_a&=\sqrt{2^2+3^2}=\sqrt{13}\\
\theta_a&=\tan^{-1}\frac{3}{2}=56.31^{\circ}
\end{align*}
یوں
\begin{align*}
a=\sqrt{13}\phase{56.31^{\circ}}
\end{align*}
لکھا جائے گا۔اسی طرح \عددی{b=4+j5} کا حیطہ اور زاویہ حاصل کرتے ہوئے
\begin{align*}
r_b&=\sqrt{4^2+5^2}=\sqrt{41}\\
\theta_b&=\tan^{-1}\frac{5}{4}=51.34^{\circ}
\end{align*}
درج ذیل لکھا جائے گا۔
\begin{align*}
b=\sqrt{41}\phase{51.34^{\circ}}
\end{align*}
اس طرح
\begin{align*}
ab&=\left(\sqrt{13}\phase{56.31^{\circ}}\right)\left(\sqrt{41}\phase{51.34^{\circ}}\right)\\
&=\sqrt{13}\sqrt{41}\phase{56.31^{\circ}+51.34^{\circ}}\\
&=\sqrt{533}\phase{107.65^{\circ}}
\end{align*}
اور
\begin{align*}
\frac{a}{b}&=\frac{\sqrt{13}\phase{56.31^{\circ}}}{\sqrt{41}\phase{51.34^{\circ}}}\\
&=\frac{\sqrt{13}}{\sqrt{41}}\phase{56.31^{\circ}-51.34^{\circ}}\\
&=\sqrt{\frac{13}{41}}\phase{4.97^{\circ}}
\end{align*}
حاصل ہوتے ہیں۔

ان جوابات کو مستطیلی طرز میں درج ذیل لکھا جائے گا جو گزشتہ مثال کے جوابات ہیں۔
\begin{align*}
ab&=\sqrt{533} \cos 107.65^{\circ}+j \sqrt{533}\sin{107.65^{\circ}}=-7+j22\\
\frac{a}{b}&=\sqrt{\frac{13}{41}} \cos 4.97^{\circ}+j\sqrt{\frac{13}{41}}\sin 4.97^{\circ}=0.56098+j0.04878
\end{align*}
\انتہا{مثال}
%======================

ہم نے دیکھا کہ زاویائی طرز میں لکھا مخلوط عدد \عددی{a=r\phase{\theta}}  مستطیل طرز میں بھی لکھا جا سکتا ہے یعنی
\begin{align}
a=r\phase{\theta}=r \cos \theta +j r \sin \theta
\end{align}
\اصطلاح{یولر مساوات}\فرہنگ{یولر مساوات}\حاشیہب{Euler's equation}\فرہنگ{Euler's equation} درج ذیل ہے۔
\begin{align}
e^{j \theta}=\cos \theta+j \sin \theta
\end{align}
مندرجہ بالا دو مساوات کو ملاتے ہوئے درج ذیل لکھا جا سکتا ہے۔
\begin{align}\label{مساوات_بدلتا_مخلوط_عدد_طرز_لکھائی}
r\phase{\theta}=r e^{j\theta}=r\left(\cos \theta+j  \sin \theta\right)
\end{align}
%=================================
\ابتدا{مثال}
مخلوط عدد \عددی{m=5-j12} کو زاویائی طرز میں لکھیں۔

حل:مساوات \حوالہ{مساوات_بدلتا_مستطیل_سے_زاویائی} کے استعمال سے درج ذیل حاصل کرتے ہیں
\begin{align*}
r&=\sqrt{5^2+12^2}=13\\
\theta&=\tan^{-1}\frac{-12}{5}=-67.38^{\circ}
\end{align*}
لہٰذا درج ذیل لکھے جا سکتے ہیں۔
\begin{align*}
m&=13 e^{-j67.38^{\circ}}\\
m&=13\phase{-67.38^{\circ}}
\end{align*}
\انتہا{مثال}
%======================
\حصہ{سائن نما تفاعل}
\اصطلاح{سائن نما}\فرہنگ{سائن نما}\حاشیہب{sinusoidal}\فرہنگ{sinusoidal} تفاعل سے مراد \اصطلاح{سائن} تفاعل \عددی{\sin \theta} اور \اصطلاح{کوسائن} تفاعل \عددی{\cos \theta} ہیں۔شکل \حوالہ{شکل_بدلتی_رو_سائن_تفاعل}-الف میں رداس \عددی{A_0} کے گول دائرے پر ایک نقطہ یکساں رفتار کے ساتھ، گھڑی کی گردش کی الٹ سمت میں، حرکت کر رہا ہے۔یہ دائرہ \اصطلاح{کارتیسی محدد}\فرہنگ{کارتیسی محدد}\حاشیہب{Cartesian coordinates}\فرہنگ{Cartesian coordinates} کے مرکز \عددی{(0,0)} پر پایا جاتا ہے۔لمحہ \عددی{t} پر زاویہ \عددی{\phase{aox}} کی قیمت \عددی{\theta} کے برابر ہے۔نقطے سے \عددی{x} محدد پر عمودی لکیر محدد کو  \عددی{x(t)}  پر ٹکراتی ہے جبکہ \عددی{y} محدد پر عمودی لکیر \عددی{y(t)} پر ٹکراتی ہے۔شکل کو دیکھتے ہوئے درج ذیل لکھا جا سکتا ہے
\begin{align}\label{مساوات_بدلتا_سائن_نما_تفاعل_الف}
y(t)&=A_0 \sin \theta
\end{align}
جہاں \عددی{A_0} موج کی چوٹی ہے جسے موج کا \اصطلاح{حیطہ}\فرہنگ{حیطہ}\حاشیہب{amplitude}\فرہنگ{amplitude} کہتے ہیں اور \عددی{\theta} کو تفاعل کا \اصطلاح{دلیل}\فرہنگ{دلیل}\حاشیہد{ایک ماہر ریاضی اپنی خیالی دنیا میں کوسائن \عددی{\cos \theta} تفاعل کے ساتھ بحث میں مصورف ہوتا ہے۔ماہر ریاضی تفاعل کو دلیل کے طور پر صفر پیش کرتا ہے۔تفاعل اس کا فوراً جواب اکائی \عددی{(\cos 0=1)} دیتا ہے۔}\حاشیہب{argument}\فرہنگ{argument} کہتے ہیں۔اس مساوات میں \عددی{\theta} از خود وقت\عددی{t} پر منحصر ہے۔

 گردش کرتا نقطہ ایک چکر میں \عددی{360^{\circ}} درجے کا زاویہ یعنی \عددی{2\pi} ریڈیئن طے کرتا ہے۔ایک چکر کاٹنے کے لئے درکار دورانیے کو \اصطلاح{دوری عرصہ}\فرہنگ{دوری عرصہ}\حاشیہب{time period}\فرہنگ{time period} کہتے ہیں جسے \عددی{T} سے ظاہر کیا جاتا ہے۔
%=========================
\ابتدا{مشق}
شکل \حوالہ{شکل_بدلتی_رو_سائن_تفاعل}-الف میں نقطہ ایک چکر \عددی{\SI{20}{\milli\second}} میں پورا کرتا ہے۔یہ نقطہ ایک سیکنڈ میں کتنے چکر پورا کرے گا۔یہ نقطہ ایک سیکنڈ میں کتنے ریڈیئن کا زاویہ طے کرتا ہے۔

جوابات:\عددی{50} چکر، \عددی{100\pi \, \si{\radian}}
\انتہا{مشق}
%========================

اگر ایک چکر کاٹنے کے لئے \عددی{T} سیکنڈ کا وقت درکار ہو تب ایک سیکنڈ میں چکروں کی تعداد  \عددی{\tfrac{1}{T}} ہو گی جسے \اصطلاح{تعدد}\فرہنگ{تعدد}\حاشیہب{frequency}\فرہنگ{frequency} کہتے اور \عددی{f} سے ظاہر کرتے ہیں۔
\begin{align}
f=\frac{1}{T}
\end{align}
تعدد کی اکائی \اصطلاح{ہرٹز}\فرہنگ{ہرٹز}\حاشیہب{Hertz}\فرہنگ{Hertz} ہے جسے \عددی{\si{\hertz}} سے ظاہر کیا جاتا ہے۔

ایک چکر \عددی{2\pi} ریڈیئن کو کہتے ہیں لہٰذا \عددی{f} چکر سے مراد \عددی{2\pi f} ریڈیئن کا زاویہ ہے۔یوں \عددی{f} تعدد پر گردش کرتا نقطہ ایک سیکنڈ میں \عددی{2\pi f} ریڈیئن کا زاویہ طے کرے گا یعنی اس کی \اصطلاح{زاویائی رفتار}\فرہنگ{زاویائی رفتار}\حاشیہب{angular speed}\فرہنگ{angular speed} کی قیمت \عددی{2\pi f} ہو گی۔زاویائی رفتار کو \عددی{\omega} سے ظاہر کیا جاتا ہے جبکہ اس کی اکائی ریڈیئن فی سیکنڈ \عددی{\si{\radian\per\second}} ہے۔
\begin{align}
\omega=2\pi f
\end{align}
زاویائی رفتار \عددی{\omega} سے گردش کرتا ہوا نقطہ \عددی{t} سیکنڈ میں \عددی{2\pi f t} ریڈیئن کا زاویہ طے کرے گا۔یوں اگر \عددی{t=0} پر نقطہ عین \عددی{x} محدد کے مثبت حصے پر ہو تب لمحہ \عددی{t} پر
\begin{align}
\theta=\omega t=2\pi f t
\end{align}
لکھا جائے گا۔یوں مساوات  \حوالہ{مساوات_بدلتا_سائن_نما_تفاعل_الف} کو
\begin{gather}
\begin{aligned}\label{مساوات_بدلتا_سائن_نما_تفاعل_ب}
y(t)&=A_0 \sin 2\pi f t\\
&=A_0\sin \frac{2\pi}{T}t\\
&=A_0 \sin \omega t
\end{aligned}
\end{gather}
لکھا جا سکتا ہے۔

برقی میدان میں \عددی{y(t)} وقت کے ساتھ بدلتے دباو یا وقت کے ساتھ بدلتی رو کو ظاہر کر سکتی ہے۔مساوات \حوالہ{مساوات_بدلتا_سائن_نما_تفاعل_ب} میں دیے تفاعل، جسے شکل \حوالہ{شکل_بدلتی_رو_سائن_تفاعل}-ب میں دکھایا گیا ہے،  کا آزاد متغیرہ وقت \عددی{t} ہے۔آپ دیکھ سکتے ہیں کہ یہ تفاعل ہر \عددی{T} سیکنڈ کے بعد اپنے آپ کو دہراتا ہے۔ اس حقیقت کو ریاضی میں درج ذیل لکھا جاتا ہے۔
\begin{align}
y(t+T)=y( t)
\end{align}
جس سے مراد یہ ہے کہ تفاعل کی قیمت لمحہ \عددی{t} اور لمحہ \عددی{t+T} پر برابر ہیں۔

\begin{figure}
\centering
\begin{subfigure}{0.5\textwidth}
\centering
\begin{tikzpicture}
\pgfmathsetmacro{\ang}{50}
\pgfmathsetmacro{\rad}{1.8}
\pgfmathsetmacro{\xp}{\rad*cos(\ang)}
\pgfmathsetmacro{\yp}{\rad*sin(\ang)}
\draw[](0,0)--++(\rad+0.5,0)node[right]{$x$};
\draw[](0,0)--++(0,\rad+0.5)node[left]{$y$};
\draw[](0,0) circle (\rad);
\draw[fill](\ang:\rad) circle (1.5pt);
\draw(0,0)node[below left]{$o$}--++(\ang:\rad)node[pos=0.7,fill=white]{$A_0$}node[above right]{$a$};
\draw[-stealth] ([shift={(0:0.5)}]0,0) arc (0:\ang:0.5);
\draw(\ang/2:0.8)node{$\theta$};
\draw[dashed](\xp,\yp)--(\xp,0)node[below]{$x(t)$};
\draw[dashed](\xp,\yp)--(0,\yp)node[left]{$y(t)$};
\end{tikzpicture}
\caption*{الف}
\end{subfigure}
\begin{subfigure}{0.5\textwidth}
\centering
\begin{tikzpicture}
\begin{axis}[kStyleCircuitsA,small, xlabel=$t$, ylabel=$y(t)$, xtick={90,180,270,360},xticklabels={$\dfrac{T}{4}$,$\dfrac{T}{2}$,$\dfrac{3T}{2}$,$T$},ytick={-1,1},yticklabels={$-A_0$,$A_0$},
]
\addplot[domain=0:400,samples=100]{sin(x)};
\draw[gray,dashed](axis cs:0,1)--(axis cs:90,1);
\draw[gray,dashed](axis cs:0,-1)--(axis cs:270,-1);
\end{axis}%
\end{tikzpicture}%
\caption*{(ب)}
\end{subfigure}%
\begin{subfigure}{0.5\textwidth}
\centering
\begin{tikzpicture}
\begin{axis}[kStyleCircuitsA,small, xlabel=$\omega t$, ylabel=$y(\omega t)$, xtick={90,180,270,360},xticklabels={$\dfrac{\pi}{2}$,$\pi$,$\dfrac{3\pi}{2}$,$2\pi$},ytick={-1,1},yticklabels={$-A_0$,$A_0$},
]
\addplot[domain=0:400,samples=100]{sin(x)};
\draw[gray,dashed](axis cs:0,1)--(axis cs:90,1);
\draw[gray,dashed](axis cs:0,-1)--(axis cs:270,-1);
\end{axis}%
\end{tikzpicture}%
\caption*{(پ)}
\end{subfigure}%
\caption{سائن موج۔}
\label{شکل_بدلتی_رو_سائن_تفاعل}
\end{figure}%

مساوات \حوالہ{مساوات_بدلتا_سائن_نما_تفاعل_ب} کے خط کو \عددی{\omega t} کے ساتھ بھی کھینچا جا سکتا ہے۔ایسا ہی شکل \حوالہ{شکل_بدلتی_رو_سائن_تفاعل}-پ میں دکھایا گیا ہے جہاں سے واضح ہے کہ یہ تفاعل ہر \عددی{2\pi} ریڈیئن کے بعد اپنے آپ کو دہراتا ہے۔
%====================
\ابتدا{مشق}
شکل \حوالہ{شکل_بدلتی_رو_سائن_تفاعل}-الف میں گردش کرتا نقطہ \عددی{\SI{0.2}{\second}} میں \عددی{40^{\circ}} کا زاویہ طے کرتا ہے۔زاویائی رفتار، تعدد اور دوری عرصہ دریافت کریں۔

جوابات:\عددی{\omega=\tfrac{10\pi}{9}\,\si{\radian\per\second}}، \عددی{f=\SI{1.8}{\hertz}}، \عددی{T=\frac{5}{9} \, \si{\second}}
\انتہا{مشق}
%=================

شکل \حوالہ{شکل_بدلتا_لمحہ_صفر_پر_مقام_الفا_ہے} میں عمومی صورت حال دکھائی گئی ہے جہاں \عددی{\omega} زاویائی رفتار سے گردش کرتا نقطہ، لمحہ \عددی{t=0} پر  زاویہ \عددی{\alpha} پر پایا جاتا ہے۔یہ نقطہ وقت \عددی{t} کے دوران \عددی{\omega t} زاویہ طے کرتے ہوئے \عددی{\theta=\omega t+\alpha} پہنچ جائے گا لہٰذا اس کے لئے
\begin{align}\label{مساوات_بدلتا_سائن_نما_تفاعل_پ}
y(t)=A_0 \sin(\omega t +\alpha)
\end{align}
لکھا جا سکتا ہے جہاں \عددی{\alpha} کو \اصطلاح{زاویائی ہٹاو}\فرہنگ{زاویائی ہٹاو}\حاشیہب{phase angle}\فرہنگ{phase angle} کہتے ہیں۔اس مساوات کا دلیل \عددی{\omega t+\alpha} ہے۔شکل \حوالہ{شکل_بدلتا_لمحہ_صفر_پر_مقام_الفا_ہے}-ب میں مساوات \حوالہ{مساوات_بدلتا_سائن_نما_تفاعل_ب} اور مساوات \حوالہ{مساوات_بدلتا_سائن_نما_تفاعل_پ} کو دکھایا گیا ہے۔آپ دیکھ سکتے ہیں کہ ان مساوات میں \عددی{\alpha} \اصطلاح{زاویائی فرق}\فرہنگ{زاویائی فرق}\حاشیہب{phase difference}\فرہنگ{phase difference} پایا جاتا ہے۔ مساوات \حوالہ{مساوات_بدلتا_سائن_نما_تفاعل_ب} سے  مساوات \حوالہ{مساوات_بدلتا_سائن_نما_تفاعل_پ} \عددی{\alpha} ریڈیئن \اصطلاح{آگے}\فرہنگ{آگے}\حاشیہب{lead}\فرہنگ{lead} ہے۔ یہ بھی کہا جا سکتا ہے کہ مساوات \حوالہ{مساوات_بدلتا_سائن_نما_تفاعل_پ} سے مساوات \حوالہ{مساوات_بدلتا_سائن_نما_تفاعل_ب} \عددی{\alpha} ریڈیئن \اصطلاح{پیچھے}\فرہنگ{پیچھے}\حاشیہب{lag}\فرہنگ{lag} ہے۔ایک ہی تعدد کے دو تفاعل
\begin{gather}
\begin{aligned}
y_1(t)&=A_{01} \sin (\omega t +\alpha)\\
y_2(t)&=A_{02}\sin(\omega t+\beta)
\end{aligned}
\end{gather}
میں \عددی{y_1(t)} تفاعل \عددی{y_2(t)} سے  \عددی{\alpha-\beta} ریڈیئن  آگے ہے۔ہم یہ بھی کہہ سکتے ہیں کہ \عددی{y_2(t)} تفاعل \عددی{y_1(t)} سے 
 \عددی{\beta-\alpha}  ریڈیئن آگے ہے یا کہ \عددی{y_1(t)} تفاعل \عددی{y_2(t)} سے  \عددی{\beta-\alpha} ریڈیئن پیچھے ہے۔اگر \عددی{\alpha=\beta} ہو تب  تفاعل \اصطلاح{ہم زاویہ}\فرہنگ{ہم زاویہ}\حاشیہب{in phase}\فرہنگ{in phase}\فرہنگ{phase!in} کہلاتے ہیں جبکہ \عددی{\alpha\ne\beta} کی صورت میں تفاعل \اصطلاح{الگ زاویہ}\فرہنگ{الگ زاویہ}\حاشیہب{out of phase}\فرہنگ{out of phase}\فرہنگ{phase!out of} کہلاتے ہیں۔


\begin{figure}
\centering
\begin{subfigure}{0.5\textwidth}
\centering
\begin{tikzpicture}
\pgfmathsetmacro{\angA}{20}
\pgfmathsetmacro{\angB}{50}
\pgfmathsetmacro{\rad}{1.8}
\pgfmathsetmacro{\xp}{\rad*cos(\angB)}
\pgfmathsetmacro{\yp}{\rad*sin(\angB)}
\draw[](0,0)--++(\rad+1.5,0)node[right]{$x$};
\draw[](0,0)--++(0,\rad+0.5)node[left]{$y$};
\draw[](0,0) circle (\rad);
\draw[gray,fill](\angA:\rad) circle (1.5pt);
\draw[fill](\angB:\rad) circle (1.5pt);
\draw[dashed,gray](0,0)--++(\angA:\rad);
\draw(0,0)--++(\angB:\rad)node[pos=0.7,fill=white]{$A_0$};
\draw[dashed](\xp,\yp)--(0,\yp)node[left]{$y(t)$};
%angles
\draw[-stealth] ([shift={(0:0.7)}]0,0) arc (0:\angA:0.7);
\draw(\angA/2:0.9)node{$\alpha$};
\draw[-stealth] ([shift={(\angA:0.6)}]0,0) arc (\angA:\angB:0.6);
\draw(\angA/2+\angB/2:0.9)node{$\omega t$};
\draw[-stealth] ([shift={(0:\rad+0.5)}]0,0) arc (0:\angB:\rad+0.5);
\draw(\angB/2:\rad+0.8)node{$\theta$};
\end{tikzpicture}
\caption*{(الف)}
\end{subfigure}%
\begin{subfigure}{0.5\textwidth}
\centering
\begin{tikzpicture}
\begin{axis}[kStyleCircuitsA,small,xlabel=$t$, ylabel=$y(t)$,ytick={1},yticklabels={$A_0$},xtick={180,360},xticklabels={$\pi$,$2\pi$},]
\addplot[domain=0:370,samples=100,dashed]{sin(x)}node[pos=0.42,pin={[font=\small]10:${A_0\sin \omega t}$},inner sep=0pt]{};
\addplot[domain=-60:320,samples=100]{sin(x+60)}node[pos=0.35,pin={[pin distance=0.75cm,font=\small]10:${A_0\sin (\omega t+\alpha)}$},inner sep=0pt]{};
\draw(axis cs:-60,-0.1)--(axis cs:-60,-0.3);
\draw[stealth-](axis cs:0,-0.2)--(axis cs:30,-0.2);
\draw[stealth-](axis cs:-60,-0.2)--(axis cs:-80,-0.2)--(axis cs:-80,-0.4)node[below]{$\alpha$};
\end{axis}
\end{tikzpicture}
\caption*{(ب) زاویائی ہٹاو۔}
\end{subfigure}%
\caption{لمحہ \عددی{t=0} پر زاویہ \عددی{\alpha} ہے۔}
\label{شکل_بدلتا_لمحہ_صفر_پر_مقام_الفا_ہے}
\end{figure}

زاویائی ہٹاو کو عموماً درجوں میں بیان کیا جاتا ہے لہٰذا \عددی{\alpha=\tfrac{\pi}{4}} کی صورت میں درج ذیل لکھا جا سکتا  ہے۔
\begin{align}\label{مساوات_بدلتا_زاویہ_ہٹاو_روایتی_طریقہ}
y(t)=A_0 \sin \left(\omega t +\frac{\pi}{4}\right)=A_0 \sin\left(\omega t+45^{\circ}\right)
\end{align}
با ضابطہ طور پر چونکہ \عددی{\omega t} کی قیمت ریڈیئن میں ہے لہٰذا \عددی{\alpha} کی قیمت بھی ریڈیئن میں ہونا لازم ہے لہٰذا تفاعل لکھنے کا صحیح طریقہ \عددی{y(t)=A_0\sin\left(\omega t+\tfrac{\pi}{4}\right)} ہی ہے لیکن زاویائی ہٹاو کو درجوں میں لکھنے کی روایت نہایت مقبول ہے لہٰذا اس کتاب میں بھی اس روایت کو برقرار رکھا جائے گا۔مساوات \حوالہ{مساوات_بدلتا_زاویہ_ہٹاو_روایتی_طریقہ} میں \عددی{45^{\circ}} لکھتے ہوئے زیر بالا میں درجے کی علامت \عددی{(^\circ)} استعمال کی گئی ہے جبکہ \عددی{\tfrac{\pi}{4}} پر کوئی علامت نہیں لگائی گئی۔اسی علامت سے ریڈیئن یا درجوں کی پہچان کی جاتی ہے۔
%================
\ابتدا{مثال}
مساوات \عددی{y_1(t)=15\sin(100t+60^{\circ})} اور \عددی{y_2(t)=22\sin(200t +0.2\pi)} کی قیمت \عددی{t=\SI{25}{\milli\second}} پر دریافت کریں۔

حل:پہلی تفاعل میں \عددی{50^{\circ}} کا زاویائی ہٹاو \عددی{\tfrac{60^{\circ}}{180^{\circ}}\times \pi=\tfrac{\pi}{3}} ریڈیئن کے برابر ہے۔یوں
 لمحہ \عددی{t=\SI{25}{\milli\second}} پر
\begin{align*}
y_1(0.025)&=15\sin\left(100\times 25\times 10^{-3}+\frac{\pi}{3}\right)=-5.918619766
\end{align*}
اور
\begin{align*}
y_2(0.025)&=22\sin(200\times 0.025+0.2\pi)=-13.39917888
\end{align*}
حاصل ہوتے ہیں۔
\انتہا{مثال}
%================== 

اگرچہ اب تک کی بحث میں ہم نے سائن تفاعل استعمال کیا، ہم اس کی جگہ کوسائن تفاعل بھی استعمال کر سکتے تھے۔ان دو تفاعل کی صورت بالکل یکساں ہے پس دونوں میں \عددی{90^{\circ}} کا زاویائی فرق پایا جاتا ہے۔
\begin{align}
\sin \left(\omega t+\frac{\pi}{2}\right)&=\cos \omega t\\
\cos \left(\omega t-\frac{\pi}{2}\right)&=\sin \omega t
\end{align}
سائن نما تفاعل کے دلیل  کے ساتھ \عددی{2\pi} ریڈیئن یا \عددی{360^{\circ}} کا مضرب جمع کرنے سے  تفاعل کی قیمت تبدیل نہیں ہوتی۔
\begin{align}
\cos(\omega t +\alpha+2\pi n)&=\cos(\omega t +\alpha) \quad \quad  n=0,\pm 1, \pm 2, \cdots \label{مساوات_بدلتا_طول_بعد_وہی_موج}\\
\sin(\omega t +\alpha+2\pi n)&=\sin(\omega t +\alpha) \quad \quad  n=0,\pm 1, \pm 2, \cdots \label{مساوات_بدلتا_طول_بعد_وہی_موج_ب}
\end{align}

دو سائن نما تفاعل میں زاویائی فرق  تین شرائط پورا کرنے کے بعد دریافت کیا جا سکتا ہے۔پہلی شرط یہ ہے کہ دونوں تفاعل کی تعدد برابر ہو۔دوسری شرط یہ ہے کہ دونوں کو  سائن تفاعل اور یا پھر دونوں کو  کوسائن تفاعل کی صورت میں لکھا جائے۔تیسری اور آخری شرط یہ ہے کہ دوسری شرط میں لکھے گئے تفاعل کے حیطے مثبت ہوں۔درج ذیل مماثل ان شرائط کو پورا کرنے میں مدد دیتے ہیں۔
\begin{align}
-\sin(\omega t+\alpha)&=\sin(\omega t+\alpha \pm 180^{\circ})\label{مساوات_بدلتا_آدھی_طول_منفی_موج_الف}\\
-\cos(\omega t+\alpha)&=\cos(\omega t+\alpha \pm 180^{\circ})\label{مساوات_بدلتا_آدھی_طول_منفی_موج_ب}
\end{align}
ان کے علاوہ درج ذیل مماثل بھی نہایت اہم ثابت ہوتے ہیں۔
\begin{align}
\sin(\alpha\pm\beta)&=\sin \alpha \cos \beta \pm \cos \alpha \sin \beta \label{مساوات_بدلتا_آدھی_طول_منفی_موج_پ}\\
\cos(\alpha \pm \beta)&=\cos \alpha \cos \beta \mp \sin \alpha \sin \beta \label{مساوات_بدلتا_آدھی_طول_منفی_موج_ت}
\end{align}
ایک آخری تفاعل جس کا ذکر ضروری ہے درج ذیل ہے۔
\begin{align} \label{مساوات_بدلتا_آدھی_طول_منفی_موج_ٹ}
\cos^2 \alpha+\sin^2 \alpha=1
\end{align}
%===============
\ابتدا{مثال}\شناخت{مثال_بدلتا_کوسائن_تفاعل_بالمقابل_ہٹاو}
درج ذیل تفاعل کے خط کھینچیں۔
\begin{itemize}
\item
$v(t)=1 \cos (\omega t +60^{\circ})$
\item
$v(t)=1 \cos (\omega t +240^{\circ})$
\item
$v(t)=1 \cos (\omega t -300^{\circ})$
\end{itemize}

حل: شکل \حوالہ{شکل_بدلتا_کوسائن_تفاعل_بالمقابل_ہٹاو}-الف میں \عددی{v(\omega t)=1\cos \omega t} کا خط دکھایا گیا ہے۔اس کو افقی محدد پر \عددی{60^{\circ}} درجے  بائیں منتقل کرنے سے \عددی{v(\omega t)=1\cos(\omega t +60^{\circ})} کا خط حاصل ہوتا ہے جسے شکل-ب میں دکھایا گیا ہے۔ ہم درج ذیل لکھ سکتے ہیں
\begin{align*}
v(\omega t)=1\cos (\omega t +240^{\circ})=1\cos (\omega t +60^{\circ}+180^{\circ})=-1\cos (\omega t +60^{\circ})
\end{align*}
جہاں مساوات \حوالہ{مساوات_بدلتا_آدھی_طول_منفی_موج_ب} کا استعمال کیا گیا ہے۔درج بالا مساوات کو شکل-پ میں دکھایا گیا ہے۔آپ دیکھ سکتے ہیں کہ یہ شکل-ب کا منفی ہے۔اسی طرح مساوات \حوالہ{مساوات_بدلتا_طول_بعد_وہی_موج} کی مدد سے
\begin{align*}
v(\omega t)=1\cos(\omega t-300^{\circ})=1\cos(\omega t-300^{\circ}+360^{\circ})=1\cos(\omega t+60^{\circ})
\end{align*}
لکھتے ہوئے شکل-ت حاصل ہوتی ہے جو عین شکل-ب ہی ہے۔
\begin{figure}
\centering
\begin{subfigure}{0.5\textwidth}
\centering
\begin{tikzpicture}
\begin{axis}[kStyleCircuitsA,small,xlabel=$\omega t$, ylabel=$v(\omega t)$,xtick={90,180,270,360}, xticklabels={$90^{\circ}$,$180^{\circ}$,$270^{\circ}$,$360^{\circ}$},ytick={-1,1},yticklabels={$-1$,$1$},]
\addplot[domain=0:360,samples=100]{1*cos(x)}node[pos=0.1,pin={[font=\small]10:${\cos \omega t}$},inner sep=0pt]{};
\end{axis}
\end{tikzpicture}
\caption*{(الف)}
\end{subfigure}%
\begin{subfigure}{0.5\textwidth}
\centering
\begin{tikzpicture}
\begin{axis}[kStyleCircuitsA,small,xlabel=$\omega t$, ylabel=$v(\omega t)$,xtick={-60,30,120,210,300}, xticklabels={$-60^{\circ}$,$30^{\circ}$,$120^{\circ}$,$210^{\circ}$,$300^{\circ}$},ytick={-1,1},yticklabels={$-1$,$1$},]
\addplot[domain=-60:300,samples=100]{1*cos(x+60)}node[pos=0.85,pin={[font=\small]170:${\cos (\omega t+60^{\circ})}$},inner sep=0pt]{};
\end{axis}
\end{tikzpicture}
\caption*{(ب)}
\end{subfigure}
\begin{subfigure}{0.5\textwidth}
\centering
\begin{tikzpicture}
\begin{axis}[kStyleCircuitsA,small,xlabel=$\omega t$, ylabel=$v(\omega t)$,xtick={-60,30,120,210,300},
 xticklabels={$-60^{\circ}$,$30^{\circ}$,$120^{\circ}$,$210^{\circ}$,$300^{\circ}$},ytick={-1,1},yticklabels={$-1$,$1$},]
\addplot[domain=-60:300,samples=100]{1*cos(x+240)}node[pos=0.85,pin={[font=\small]-170:${\cos( \omega t+240^{\circ})}$},inner sep=0pt]{};
\end{axis}
\end{tikzpicture}
\caption*{(پ)}
\end{subfigure}%
\begin{subfigure}{0.5\textwidth}
\centering
\begin{tikzpicture}
\begin{axis}[kStyleCircuitsA,small,xlabel=$\omega t$, ylabel=$v(\omega t)$,xtick={-60,30,120,210,300}, 
xticklabels={$-60^{\circ}$,$30^{\circ}$,$120^{\circ}$,$210^{\circ}$,$300^{\circ}$},ytick={-1,1},yticklabels={$-1$,$1$},]
\addplot[domain=-60:300,samples=100]{1*cos(x-300)}node[pos=0.85,pin={[font=\small]170:${\cos( \omega t-300^{\circ})}$},inner sep=0pt]{};
\end{axis}
\end{tikzpicture}
\caption*{(ت)}
\end{subfigure}%
\caption{مثال \حوالہ{مثال_بدلتا_کوسائن_تفاعل_بالمقابل_ہٹاو} کے خط۔}
\label{شکل_بدلتا_کوسائن_تفاعل_بالمقابل_ہٹاو}
\end{figure}
\انتہا{مثال}
%==========================
\ابتدا{مثال}
درج ذیل امواج کی تعدد ہرٹز میں حاصل کریں۔ امواج کے مابین زاویائی فرق دریافت کریں۔یہ بھی بتلائیں کہ کونسی موج آگے ہے۔
\begin{align*}
v_1(\omega t)&=100\sin(400t -30^{\circ})\\
v_2(\omega t)&=-250\cos(400t+0.2\pi)
\end{align*}

حل:ان امواج میں \عددی{\omega=\SI{400}{\radian\per\second}} ہے لہٰذا
\begin{align*}
f=\frac{\omega}{2\pi}=\frac{400}{2\pi}=\SI{63.66}{\hertz}
\end{align*}
ہو گا۔زاویائی فرق دریافت کرنے کی خاطر دونوں امواج کو مثبت حیطے کے کوسائن موج کی صورت میں لکھتے ہیں۔ساتھ ہی ساتھ ان کے زاویائی ہٹاو کو درجوں میں لکھتے ہیں۔یوں
\begin{align*}
v_1(\omega t)&=100\sin(400t -30^{\circ})\\
&=100\cos(400t-30^{\circ}-90^{\circ})\\
&=100\cos(400t-120^{\circ})\\
&=100\cos(400t+240^{\circ})
\end{align*}
لکھا جا سکتا ہے جہاں آخری قدم پر مساوات \حوالہ{مساوات_بدلتا_طول_بعد_وہی_موج} کا استعمال کیا گیا۔اسی طرح
\begin{align*}
v_2(\omega t)&=-250\cos(400t+0.2\pi)\\
&=250\cos(400t+0.2\pi+\pi)\\
&=250\cos(400t+216^{\circ})
\end{align*}
بھی لکھا جا سکتا ہے جہاں آخری قدم پر \عددی{1.2\pi} ریڈیئن کو \عددی{216^{\circ}} درجے لکھا گیا ہے۔ان امواج کے مابین
\begin{align*}
240^{\circ}-216^{\circ}=24^{\circ}
\end{align*}
کا زاویائی فرق پایا جاتا ہے اور موج \عددی{v_1(\omega t)} آگے ہے۔
\انتہا{مثال}
%========================
\ابتدا{مشق}
ایک دور میں درج ذیل تین رو پائے جاتے ہیں۔
\begin{align*}
i_1(1)&=30\cos(100\pi t+30^{\circ})\\
i_2(2)&=55\sin(100\pi t +40^{\circ})\\
i_3(t)&=20\sin(100 \pi t+60^{\circ})
\end{align*}
\عددی{i_2} سے \عددی{i_1} کتنی آگے ہے اور \عددی{i_3} سے \عددی{i_1} کتنی پیچھے ہے۔

جوابات:\عددی{80^{\circ}}، \عددی{-60^{\circ}} یا \عددی{300^{\circ}} 
\انتہا{مشق}
%=========================
\حصہ{سائن نما اور مخلوط جبری تفاعل}
گزشتہ باب میں دور پر مستقل جبری تفاعل مسلط  کرتے ہوئے، دور کا جبری ردعمل بھی مستقل قیمت کا حاصل ہوا۔تفرقی مساوات کا جبری ردعمل، مسلط جبری تفاعل اور اس کے تمام بلند درجی تفرق کا مجموعہ ہوتا ہے۔یوں  دور پر جبری دباو \عددی{v(t)=\sin \omega t} مسلط کرنے سے رو کا جبری ردعمل \عددی{i(t)=c_1\sin\omega t+c_2\cos\omega t} متوقع ہو گا۔پس جبری ردعمل کے مستقل \عددی{c_1} اور \عددی{c_2} معلوم کرنا باقی ہے۔
%===============
\ابتدا{مثال}\شناخت{مثال_بدلتا_مزاحمت_امالہ_جبری_حل_الف}
شکل \حوالہ{شکل_بدلتا_مزاحمت_امالہ_جبری_حل_الف} میں رو \عددی{i_J(t)} حاصل کریں۔

\begin{figure}
\centering
\begin{tikzpicture}
\draw(0,0) to [american voltage source,l={${v(t)=V_0\cos \omega t}$}]++(0,\y) to [resistor,i={$i(t)$},l={$R$}]++(\x,0) to [inductor,l={$L$}]++(0,-\y) to [short] (0,0);
\end{tikzpicture}
\caption{مثال \حوالہ{مثال_بدلتا_مزاحمت_امالہ_جبری_حل_الف} کا دور۔}
\label{شکل_بدلتا_مزاحمت_امالہ_جبری_حل_الف}
\end{figure}

حل: دور کی تفرقی مساوات لکھتے ہیں۔
\begin{align}\label{مساوات_بدلتا_مزاحمت_امالہ_جبری_حل_الف}
R i(t)+L \frac{\dif i(t)}{\dif t}=V_0 \cos \omega t
\end{align}
دور پر مسلط جبری تفاعل اور اس تفاعل کے تمام بلند درجی تفرق کا مجموعہ جبری حل کے برابر ہو گا۔
\begin{align*}
i_J(t)&=c_1 \cos \omega t+c_2\sin \omega t
\end{align*} 
اس جبری حل کو مساوات \حوالہ{مساوات_بدلتا_مزاحمت_امالہ_جبری_حل_الف} میں پُر کرتے ہوئے \عددی{c_1} اور \عددی{c_2} مستقل دریافت کرتے ہیں۔ 
\begin{align*}
R(c_1 \cos \omega t+c_2\sin \omega t)+L (-c_1 \omega \sin\omega t+c_2 \omega \cos \omega t)=V_0 \cos \omega t
\end{align*}
درج بالا مساوات میں دونوں اطراف \عددی{\cos \omega t} کے  عددی سر برابر ہوں گے۔اسی طرح دونوں اطراف \عددی{\sin \omega t} کے عددی سر برابر ہوں گے۔
\begin{align*}
c_1 R+c_2 \omega L&=V_0\\
-c_1 \omega L+c_2 R&=0
\end{align*}
ان ہمزاد مساوات کو \عددی{c_1} اور \عددی{c_2} کے لئے حل کرتے ہوئے درج ذیل ملتا ہے
\begin{align*}
c_1&=\frac{R V_0}{R^2+\omega^2 L^2}\\
c_2&=\frac{\omega L V_0}{R^2+\omega^2 L^2}
\end{align*}
لہٰذا جبری حل
\begin{align}\label{مساوات_بدلتا_جبری_حل_الف}
i_J(t)=\frac{R V_0}{R^2+\omega^2 L^2} \cos \omega t+\frac{\omega L V_0}{R^2+\omega^2 L^2} \sin \omega t
\end{align}
ہو گا۔
\انتہا{مثال}
%=================
\ابتدا{مثال}\شناخت{مثال_بدلتا_مزاحمت_امالہ_جبری_حل_ب}
درج بالا مثال میں \عددی{R=\SI{100}{\ohm}}، \عددی{L=\SI{5}{\milli\henry}}، \عددی{V_0=\SI{310}{\volt}} اور \عددی{\omega=\SI{10}{\kilo\radian\per\second}} کی صورت میں جبری حل کو مساوات \حوالہ{مساوات_بدلتا_آدھی_طول_منفی_موج_ت} کی مدد سے \عددی{i(t)=I_0\cos(\omega t-\phi)} کے طرز پر لکھیں۔

حل: مساوات \حوالہ{مساوات_بدلتا_جبری_حل_الف} میں دی گئی قیمتیں پُر کرنے سے
\begin{align*}
i_J(t)&=\frac{100\times 310}{100^2+(\num{10000}\times 0.005)^2} \cos \omega t+\frac{\num{10000}\times 0.005\times 310}{100^2+(\num{10000}\times 0.005)^2} \sin \omega t
\end{align*}
یعنی درج ذیل حاصل ہوتا ہے۔
\begin{align}\label{مساوات_بدلتا_جبری_حل_ب}
i_J(t)=2.48\cos \omega t+1.24\sin \omega t
\end{align}
مساوات \حوالہ{مساوات_بدلتا_آدھی_طول_منفی_موج_ت} سے  جبری حل کی درکار صورت کو درج ذیل لکھا جا سکتا ہے۔ 
\begin{align}\label{مساوات_بدلتا_جبری_حل_پ}
i(t)=I_0 \cos(\omega t-\phi)=I_0 \cos \phi \cos \omega t+I_0 \sin \phi \sin \omega t
\end{align}
مساوات \حوالہ{مساوات_بدلتا_جبری_حل_ب} میں \عددی{\cos \omega t} اور \عددی{\sin \omega t} کے عددی سر کو مساوات \حوالہ{مساوات_بدلتا_جبری_حل_پ} کے عددی سر کے برابر پُر کرتے ہیں۔
\begin{align}
I_0 \cos \phi &=2.48 \label{مساوات_بدلتا_جبری_حل_ت}\\
I_0 \sin \phi&=1.24 \label{مساوات_بدلتا_جبری_حل_ٹ}
\end{align}
ان ہمزاد مساوات کے مربع جمع کرتے ہوئے 
\begin{align*}
I_0^2 \cos^2 \phi+I_0^2 \sin^2 \phi=2.48^2+1.24^2
\end{align*}
ملتا ہے جس میں مساوات \حوالہ{مساوات_بدلتا_آدھی_طول_منفی_موج_ٹ} کے استعمال سے  \عددی{\cos^2 \phi+\sin^2 \phi=1} پُر کرتے ہوئے
\begin{align*}
I_0=\sqrt{2.48^2+1.24^2}=2.7727
\end{align*}
ملتا ہے۔اسی طرح مساوات \حوالہ{مساوات_بدلتا_جبری_حل_ٹ} کو مساوات \حوالہ{مساوات_بدلتا_جبری_حل_ت} سے تقسیم کرنے سے
\begin{align*}
\frac{\sin \phi}{\cos \phi}=\frac{1.24}{2.48}=\tan \phi
\end{align*}
یعنی
\begin{align*}
\phi=\tan ^{-1}\frac{1.24}{2.48}=\phase{26.6^{\circ}}
\end{align*}
ملتا ہے۔یوں جبری حل درج ذیل لکھا جائے گا
\begin{align}
i_J(t)=2.77\cos(\omega t-26.6^{\circ})=2.77\cos(\num{10000} t-26.6^{\circ})
\end{align}
جہاں سے ظاہر ہے کہ دباو سے رو \عددی{26.6^{\circ}} درجے پیچھے ہے۔ مخلوط جبری حل درج ذیل لکھا جائے گا جس کا حقیقی جزو درج بالا مساوات ہے۔
\begin{align}\label{مساوات_بدلتا_مخلوط_رو_صورت}
i_M(t)=2.77 e^{j(\num{10000} t-26.6^{\circ})}
\end{align}
\انتہا{مثال}
%===================
\ابتدا{مثال}\شناخت{مثال_بدلتا_مزاحمت_امالہ_جبری_حل_پ}
مثال \حوالہ{مثال_بدلتا_مزاحمت_امالہ_جبری_حل_ب} کے طرز پر مثال \حوالہ{مثال_بدلتا_مزاحمت_امالہ_جبری_حل_الف} میں حاصل کئے گئے جبری حل کو \عددی{i_J(t)=I_0\cos(\omega t -\phi)} کی صورت میں لکھیں۔

حل:مساوات \حوالہ{مساوات_بدلتا_جبری_حل_الف} میں \عددی{\cos \omega t} اور \عددی{\sin \omega t} کے عددی سر کو مساوات \حوالہ{مساوات_بدلتا_جبری_حل_پ} میں \عددی{\cos \omega t} اور \عددی{\sin \omega t} کے عددی سر کے برابر پُر کرتے ہوئے درج ذیل ملتا ہے۔
\begin{align*}
I_0 \cos \phi&=\frac{R V_0}{R^2+\omega^2 L^2}\\
I_0 \sin \phi&=\frac{\omega L V_0}{R^2+\omega^2 L^2}
\end{align*}
ان ہمزاد مساوات میں دوسری مساوات کو پہلی سے تقسیم کرتے ہوئے
\begin{align*}
\frac{\sin \phi}{\cos \phi}=\tan \phi=\frac{\omega L}{R}
\end{align*}
یعنی
\begin{align}
\phi=\tan^{-1}{\frac{\omega L}{R}}
\end{align}
ملتا ہے جبکہ دونوں ہمزاد مساوات کے مربع کا مجموعہ لیتے ہوئے
\begin{align*}
I_0^2 \cos^2 \phi+I_0^2 \sin^2 \phi=I_0^2&=\left(\frac{R V_0}{R^2+\omega^2 L^2}\right)^2+\left(\frac{\omega L V_0}{R^2+\omega^2 L^2}\right)^2 \\
&=\frac{(R^2+\omega^2 L^2)V_0^2}{(R^2+\omega^2 L^2)^2}\\
&=\frac{V_0^2}{R^2+\omega^2 L^2}
\end{align*}
یعنی
\begin{align}
I_0=\frac{V_0}{\sqrt{R^2+\omega^2 L^2}}
\end{align}
ملتا ہے۔یوں جبری حل درج ذیل لکھا جائے گا۔
\begin{align}\label{مساوات_بدلتا_جبری_حل_امالہ_مزاحمت}
i_J(t)=\frac{V_0}{\sqrt{R^2+\omega^2 L^2}} \cos \left(\omega t -\tan^{-1}{\frac{\omega L}{R}}\right)
\end{align}
\انتہا{مثال}
%========================

مساوات \حوالہ{مساوات_بدلتا_جبری_حل_امالہ_مزاحمت} سے ظاہر ہے کہ \عددی{L=0} کی صورت میں \عددی{\phi=0} ہو گا لہٰذا دباو اور رو ہم زاویہ ہوں گے جبکہ \عددی{R=0} کی صورت میں \عددی{\phi=90^{\circ}} ہو گا لہٰذا دباو سے رو \عددی{90^{\circ}} درجے پیچھے ہو گی۔مزاحمت اور امالہ کے دیگر قیمتوں کی صورت میں دباو سے رو \عددی{0^{\circ}} تا \عددی{90^{\circ}} کے مابین کسی مخصوص  درجے پر پیچھے رہے گی۔اسی لئے مزاحمت اور امالہ کے ادوار کو پیچھے رہنے والے ادوار کہا جاتا ہے۔

سلسلہ وار جڑے مزاحمت اور امالہ کے دور کا حل آپ نے دیکھا۔یقیناً اس دور کا حل سلسلہ وار جڑے دو عدد مزاحمتی دور کے حل سے کئی گنا مشکل تھا۔آپ خود تصور کر سکتے ہیں کہ زیادہ تعداد کے پرزوں کا دور حل کرنا کتنا مشکل ہو گا۔اسی مشکل کو مد نظر رکھتے ہوئے ہم \اصطلاح{مخلوط تفاعل}\فرہنگ{مخلوط تفاعل}\حاشیہب{complex function}\فرہنگ{complex function} کو پیش کرتے ہیں جس سے ادوار کا حل انتہائی آسان ثابت ہوتا ہے۔

مخلوط تفاعل اور سائن نما تفاعل کا تعلق \اصطلاح{یولر مساوات}\فرہنگ{یولر مساوات}\حاشیہب{Euler's equation}\فرہنگ{Euler's equation}
\begin{align}
e^{j\omega t}=\cos \omega t +j \sin \omega t \quad \quad \text{\RL{یولر مساوات}}
\end{align}
دیتی ہے جہاں \عددی{j=\sqrt{-1}} خیالی عدد ہے۔یولر مساوات میں \عددی{\cos \omega t} \اصطلاح{حقیقی}\فرہنگ{حقیقی}\حاشیہب{real}\فرہنگ{real} مقدار اور \عددی{\sin \omega t} \اصطلاح{خیالی}\فرہنگ{خیالی}\حاشیہب{imaginary}\فرہنگ{imaginary} مقدار ہیں۔

حقیقی دنیا میں مخلوط جبری تفاعل نہیں پایا جاتا۔اس کے باوجود، دور پر سائن نما جبری تفاعل کی جگہ مخلوط جبری تفاعل مسلط کرتے ہوئے  مخلوط حل حاصل کیا جا سکتا ہے۔مخلوط جبری تفاعل کو حقیقی جبری تفاعل اور خیالی جبری تفاعل کا مجموعہ تصور کیا جا سکتا ہے۔خطی ادوار میں مسئلہ نفاذ کے تحت تمام جبری تفاعل کی علیحدہ علیحدہ اثرات کا مجموعہ لیا جا سکتا ہے۔یوں جبری تفاعل کے حقیقی جزو سے حل کا حقیقی جزو جبکہ جبری تفاعل کے خیالی جزو سے حل کا خیالی جزو حاصل ہو گا۔یوں مخلوط حل کے خیالی جزو کو رد کرتے ہوئے حقیقی جزو کو سائن نما تفاعل کا ردعمل تسلیم کیا جاتا ہے۔اس ترکیب کو مثال کی مدد سے زیادہ آسانی سے سمجھا جا سکتا ہے۔
%====================
\ابتدا{مثال}\شناخت{مثال_بدلتا_مخلوط_تفاعل_الف}
شکل \حوالہ{شکل_بدلتا_مزاحمت_امالہ_جبری_حل_الف} میں حقیقی جبری تفاعل \عددی{V_0\cos \omega t} کی جگہ مخلوط جبری تفاعل نسب کرتے ہوئے حقیقی \عددی{i(t)} کے لئے حل کریں۔

حل:حقیقی جبری تفاعل \عددی{v(t)=V_0\cos \omega t} کی جگہ دور میں مخلوط جبری تفاعل \عددی{v(t)=V_0 e^{j\omega t}} نسب کرتے ہوئے کرخوف مساوات لکھتے ہیں۔
\begin{align*}
R i(t) +L\frac{\dif i(t)}{\dif t}=V_0 e^{j \omega t}
\end{align*}
جبری تفاعل \عددی{e^{j\omega t}} کا تفرق \عددی{j\omega e^{j\omega t}} بھی جبری تفاعل ہی ہے لہٰذا درج بالا مساوات کا مخلوط حل \عددی{i_M(t)=I_0e^{j\omega t}} فرض  کرتے ہیں جہاں \عددی{I_0} نا معلوم مخلوط مستقل ہے۔اس حل کو درج بالا مساوات میں پُر کرتے ہوئے
\begin{align*}
R I_0 e^{j\omega t}+L \frac{\dif}{\dif t}\left(I_0 e^{j\omega t} \right)=V_0 e^{j \omega t}
\end{align*}
درکار تفرق کے بعد
\begin{align}\label{مساوات_بدلتا_مخلوط_الف}
R I_0  e^{j \omega t}+j \omega L I_0e^{j\omega t}&=V_0 e^{j \omega t}
\end{align}
ملتا ہے جس کے دونوں اطراف کو \عددی{e^{j\omega t}} سے تقسیم کرتے ہوئے درج ذیل ملتا ہے۔
\begin{align}\label{مساوات_بدلتا_مخلوط_ب}
R I_0+j \omega L I_0&=V_0 
\end{align}
اس سے \عددی{I_0} حاصل کرتے ہیں۔
\begin{align}\label{مساوات_بدلتا_مخلوط_مستقل_الف}
I_0=\frac{V_0}{R+j\omega L}
\end{align}
یوں مخلوط رو درج ذیل لکھی جا سکتی ہے۔
\begin{gather}
\begin{aligned}\label{مساوات_بدلتا_مخلوط_پ}
i_M(t)&=I_0 e^{j\omega t}\\
&=\frac{V_0 e^{j\omega t}}{R+j\omega L}
\end{aligned}
\end{gather}
ہمیں اس کا حقیقی جزو درکار ہے۔یولر مساوات کی مدد سے درج بالا مساوات کو درج ذیل لکھا جا سکتا ہے۔
\begin{align*}
i_M(t)&=\frac{V_0 (\cos \omega t+j \sin \omega t)}{R+j\omega L}
\end{align*}
دائیں ہاتھ کسر کے بالائی اور نچلے حصے کو \عددی{R-j\omega t} سے ضرب دیتے ہیں
 \begin{align*}
i_M(t)&=\frac{V_0 (\cos \omega t+j \sin \omega t)(R-j\omega L)}{(R+j\omega L)(R-j\omega L)}\\
&=\frac{V_0(R \cos \omega t+\omega L \sin \omega t)+jV_0(R\sin \omega t-\omega L \cos \omega t)}{R^2+\omega^2 L^2}
\end{align*}
جہاں دوسرا قدم ترتیب دیتے ہوئے لکھا گیا ہے۔اس کا حقیقی جزو درکار حل ہے
\begin{align}\label{مساوات_بدلتا_مخلوط_ت}
i(t)=\frac{V_0(R \cos \omega t+\omega L \sin \omega t)}{R^2+\omega^2 L^2}
\end{align}
جو عین مساوات \حوالہ{مساوات_بدلتا_جبری_حل_الف} ہی ہے۔

ہم مساوات \حوالہ{مساوات_بدلتا_مخلوط_مستقل_الف} کے مخلوط مستقل \عددی{I_0} کو زاویائی شکل میں لکھ کر بھی آگے بڑھ سکتے ہیں۔مخلوط مستقل کو درج ذیل لکھا جا سکتا ہے
\begin{align*}
I_0&=\frac{V_0}{R+j\omega L}\\
&=\frac{V_0}{\sqrt{R^2+\omega^2 L^2} \phase{\tan^{-1}\frac{\omega L}{R}}}\\
&=\frac{V_0}{\sqrt{R^2+\omega^2 L^2}\,  e^{j\tan^{-1}\frac{\omega L}{R}}}\\
&=\frac{V_0}{\sqrt{R^2+\omega^2 L^2}} e^{-j\tan^{-1}\frac{\omega L}{R}}
\end{align*}
جہاں دوسری قدم پر کسر کے نچلی حصے کو مساوات \حوالہ{مساوات_بدلتا_مستطیل_سے_زاویائی} کی مدد سے زاویائی صورت میں لکھا گیا ہے اور تیسری قدم پر یولر مساوات کا استعمال کیا گیا ہے۔زاویہ \عددی{\theta=\tan^{-1} \tfrac{\omega L}{R}} کو شکل \حوالہ{شکل_بدلتا_مخلوط_تفاعل_الف} میں دکھایا گیا ہے۔ یوں مخلوط رو درج ذیل لکھی جائے گی۔
\begin{align*}
i_M&=I_0 e^{j\omega t}\\
&=\frac{V_0}{\sqrt{R^2+\omega^2 L^2}} e^{j\left(\omega t-\tan^{-1}\frac{\omega L}{R}\right)} 
\end{align*}
اس مساوات میں \عددی{\tan^{-1}\tfrac{\omega L}{R}=\theta} لکھتے ہوئے حقیقی جزو لے کر حقیقی رو حاصل کرتے ہیں۔
\begin{align*}
i(t)&=\left. \frac{V_0}{\sqrt{R^2+\omega^2 L^2}} e^{j\left(\omega t-\theta\right)} \right|_{\text{حقیقی}} \\
&=\frac{V_0}{\sqrt{R^2+\omega^2 L^2}} \cos (\omega t-\theta)\\
&=\frac{V_0}{\sqrt{R^2+\omega^2 L^2}} \left(\cos \omega t \cos \theta+\sin \omega t \sin \theta \right)
\end{align*}
شکل \حوالہ{شکل_بدلتا_مخلوط_تفاعل_الف} سے  \عددی{\cos \theta=\tfrac{R}{\sqrt{R^2+\omega^2L^2}}} اور
 \عددی{\sin \theta=\tfrac{\omega L}{\sqrt{R^2+\omega^2L^2}}} پُر کرتے ہوئے
\begin{align*}
i(t)&=\frac{V_0}{\sqrt{R^2+\omega^2 L^2}} \left(\cos \omega t \frac{R}{\sqrt{R^2+\omega^2 L^2}}+\sin \omega t \frac{\omega L}{\sqrt{R^2+\omega^2 L^2}}\right)\\
&=\frac{V_0 (R \cos \omega t +\omega L \sin \omega t)}{R^2+\omega^2 L^2}
\end{align*}
%
\begin{figure}
\centering
\begin{tikzpicture}
\pgfmathsetmacro{\l}{3};
\pgfmathsetmacro{\ang}{20};
\pgfmathsetmacro{\x}{\l*cos(\ang)};
\pgfmathsetmacro{\y}{\l*sin(\ang)};
\draw(0,0)--++(4,0)node[right]{حقیقی};
\draw(0,0)--++(0,1.5)node[left]{خیالی};
\draw(0,0)--++(\ang:\l)coordinate(kp)node[pos=0.5,above,sloped]{$\sqrt{R^2+\omega^2 L^2}$};
\draw(\x,0)--(kp)node[pos=0.5,right]{$\omega L$};
\draw(\x/2,0)node[below]{$R$};
\draw[-stealth]([shift={(0:0.8)}]0,0) arc (0:\ang:0.8);
\draw(\ang/2:1)node{$\theta$};
\end{tikzpicture}
\caption{مثال \حوالہ{مثال_بدلتا_مخلوط_تفاعل_الف} کا شکل۔}
\label{شکل_بدلتا_مخلوط_تفاعل_الف}
\end{figure}
\انتہا{مثال}
%================

\حصہ{دوری سمتیہ}
درج بالا حصے میں ہم نے دیکھا کہ حقیقی جبری تفاعل کی جگہ مخلوط جبری تفاعل نسب کرتے ہوئے مخلوط حل حاصل کیا جا سکتا ہے جس کا حقیقی جزو حقیقی جبری رد عمل ہو گا۔ اس ترکیب کو مثال \حوالہ{مثال_بدلتا_مخلوط_تفاعل_الف} میں استعمال کیا گیا جہاں مساوات \حوالہ{مساوات_بدلتا_مخلوط_الف} کو \عددی{e^{j\omega t}} سے تقسیم کرتے ہوئے مساوات \حوالہ{مساوات_بدلتا_مخلوط_ب} حاصل کی گئی۔ مساوات \حوالہ{مساوات_بدلتا_مخلوط_ب} سے \عددی{I_0} حاصل کی گئی جسے  \عددی{e^{j\omega t}} سے ضرب دیتے ہوئے مخلوط حل حاصل کیا گیا۔مخلوط حل کا حقیقی جزو یعنی مساوات \حوالہ{مساوات_بدلتا_مخلوط_ت} درکار جواب ہے۔مثال \حوالہ{مثال_بدلتا_مزاحمت_امالہ_جبری_حل_ب} میں  مخصوص قیمتیں استعمال کرتے  ہوئے مخلوط رو کو مساوات \حوالہ{مساوات_بدلتا_مخلوط_رو_صورت}  میں پیش کیا گیا۔آپ دیکھ سکتے ہیں کہ جزو \عددی{e^{j \omega t}}  جوں کا توں مخلوط جبری تفاعل اور مخلوط جبری حل میں پایا جاتا ہے۔

حقیقت میں کسی بھی خطی دور پر مخلوط جبری تفاعل مثلاً
\begin{align}
v_M=V_0 e^{j \omega t}
\end{align}
مسلط کرنے سے دور میں تمام رو کی صورت \عددی{i_M(t)=I_0e^{j(\omega t+\phi)}} اور دباو کی صورت \عددی{v_M(t)=V_0e^{j(\omega t+\phi)}} ہو گی جہاں تمام رو اور دباو  کی تعدد \عددی{\omega} جبکہ ان کے انفرادی  حیطے مختلف ہوں گے۔ ان کے انفرادی زاویہ ہٹاو بھی مختلف ہوں گے۔یہاں حیطہ حقیقی مقدار ہے۔

یوں تعدد جانتے ہوئے کسی بھی مخلوط تفاعل مثلاً مخلوط رو کو اس کے حیطے \عددی{I_0} اور زاویائی ہٹاو \عددی{\phi} سے مکمل طور پر ظاہر کیا جا سکتا ہے۔مخلوط تفاعل مثلاً 
\begin{align}
i_M(t)=I_0 e^{j(\omega t+\phi)}
\end{align}
سے حقیقی تفاعل درج ذیل
\begin{align}\label{مساوات_بدلتا_حقیقی_رو_الف}
i(t)&=\left. I_0 e^{j(\omega t+\phi)} \right|_{\text{حقیقی}}
\end{align}
لکھا جا سکتا ہے جہاں \عددی{I_0} حقیقی مقدار ہے اور زیر نوشت میں لفظ "حقیقی" لکھنے کا مطلب ہے کہ اس تفاعل کا حقیقی جزو لیا جائے یعنی
\begin{align}\label{مساوات_بدلتا_حقیقی_رو_ب}
i(t)&=I_0 \cos (\omega t+\phi)
\end{align}
مساوات \حوالہ{مساوات_بدلتا_حقیقی_رو_الف} حقیقی رو دیتی ہے۔اس طرز کے تمام مساوات میں \عددی{e^{j\omega t}} پایا جاتا ہے اور مساوات کا حقیقی جزو ہی حقیقی مقدار ہوتا ہے۔یوں ایسے مساوات میں لفظ "حقیقی" اور \عددی{e^{j\omega t}} کو ذہن میں رکھتے ہوئے انہیں لکھنے سے گریز کیا جاتا ہے۔مساوات \حوالہ{مساوات_بدلتا_حقیقی_رو_الف} میں ایسا ہی کرتے ہوئے درج ذیل لکھا جائے گا
\begin{align}\label{مساوات_بدلتا_دوری_سمتیہ_الف}
\hat{I}=I_0e^{j\phi}
\end{align}
جہاں رو کو ٹوپی والے بڑے حرف سے ظاہر کیا گیا ہے۔دباو کی صورت میں تفاعل کو \عددی{\hat{V}} لکھا جاتا۔ٹوپی والے بڑے حرف سے ظاہر کردہ تفاعل کو \عددی{e^{j\omega t}} سے ضرب دے کر حقیقی جزو لینے سے حقیقی تفاعل حاصل کیا جاتا ہے۔

مساوات \حوالہ{مساوات_بدلتا_دوری_سمتیہ_الف} کا صفحہ \حوالہصفحہ{مساوات_بدلتا_مخلوط_عدد_طرز_لکھائی} پر مساوات \حوالہ{مساوات_بدلتا_مخلوط_عدد_طرز_لکھائی} سے موازنہ کریں۔ایسا معلوم ہوتا ہے جیسے  \عددی{\hat{I}} مخلوط عدد کو ظاہر کرتا ہے۔اگرچہ مساوات \حوالہ{مساوات_بدلتا_دوری_سمتیہ_الف} درحقیقت میں مساوات \حوالہ{مساوات_بدلتا_حقیقی_رو_الف} کو چھوٹا لکھنے کا طریقہ ہے لہٰذا \عددی{\hat{I}} مخلوط عدد کو ظاہر نہیں کرتا لیکن دیکھا یہ گیا ہے کہ \عددی{\hat{I}} کو مخلوط عدد تصور کر لینے سے ہمارے لئے آسانی پیدا ہوتی ہے۔آئیں \عددی{\hat{V}} کو مخلوط عدد فرض کرتے ہوئے اس کو مخلوط سطح پر ظاہر کریں۔
%=========================== 
\ابتدا{مثال}\شناخت{مثال_بدلتا_دوری_سمتیہ_مثال_الف}
مخلوط دباو \عددی{v_M(t)=50e^{j(100\pi t-35^{\circ})}}سے \عددی{\hat{V}} حاصل کرتے ہوئے \عددی{\hat{V}} کو مخلوط سطح پر دکھائیں۔

حل:مخلوط دباو سے حقیقی دباو لکھتے ہیں۔
\begin{align*}
v(t)=\left. 50e^{j(100\pi t-35^{\circ})} \right|_{\text{حقیقی}}
\end{align*}
اس مساوات کی تعدد \عددی{(\omega=100\pi)} کو ذہن نشین کرتے ہوئے لفظ "حقیقی" اور \عددی{e^{j100\pi t}} لکھنے سے گریز کرتے ہوئے درج ذیل لکھا جائے گا
\begin{align*}
\hat{V}&=50e^{-j35^{\circ}}\\
&=50\phase{-35^{\circ}}
\end{align*}
جسے شکل \حوالہ{شکل_بدلتا_دوری_سمتیہ_مثال_الف} میں مخلوط سطح پر دکھایا گیا ہے۔حقیقی محدد سے گھڑی کی گردش کی جانب مثبت زاویہ ناپا جاتا ہے لہٰذا منفی زاویے کو گھڑی کی گردش کے الٹ جانب دکھایا گیا ہے۔مخلوط اعداد اور \عددی{\hat{V}} میں فرق رکھنے کی خاطر \عددی{\hat{V}} کو مخلوط سطح پر تیر کی نشان سے ظاہر کیا جاتا ہے۔
\begin{figure}
\centering
\begin{tikzpicture}
\draw(0,0)--++(3,0)node[right]{حقیقی};
\draw(0,-1.5)--(0,0.5)node[left]{خیالی};
\draw[-latex] (0,0)--++(-35:2.5)node[right]{$\hat{V}$}node[pos=0.6,below left]{$50$};
\draw[-stealth]([shift={(0:0.5)}]0,0) arc (0:-35:0.5);
\draw(-35*2/3:0.6)node[right]{$35^{\circ}$};
\end{tikzpicture}
\caption{مثال \حوالہ{مثال_بدلتا_دوری_سمتیہ_مثال_الف} کی دوری سمتیہ۔}
\label{شکل_بدلتا_دوری_سمتیہ_مثال_الف}
\end{figure}
\انتہا{مثال}
%===========================

مثال \حوالہ{مثال_بدلتا_دوری_سمتیہ_مثال_الف} میں \عددی{\hat{V}} کو مخلوط سطح پر تیر کے نشان سے ظاہر کیا گیا ہے جسے دیکھ کر یوں معلوم ہوتا ہے جیسے \عددی{\hat{V}} ایک سمتیہ ہے۔اسی حقیقت کی بنا پر \عددی{\hat{V}} یا \عددی{\hat{I}} کو \اصطلاح{دوری سمتیہ}\فرہنگ{دوری سمتیہ}\حاشیہب{phasor}\فرہنگ{phasor} کہتے ہیں اور شکل \حوالہ{شکل_بدلتا_دوری_سمتیہ_مثال_الف} کو \اصطلاح{دوری سمتیہ شکل}\فرہنگ{دوری سمتیہ شکل}\حاشیہب{phasor diagram}\فرہنگ{phasor diagram} کہتے ہیں۔

مخلوط عدد لکھنے کے تمام طرز پر دوری سمتیہ کو لکھا جاتا ہے لہٰذا درج ذیل لکھنا ممکن ہے۔
\begin{gather}
\begin{aligned}\label{مساوات_بدلتا_تعددی_طرز}
\hat{I}&=I_0e^{j\phi}\\
 &=I_0\phase{\phi}\\
&=I_x+jI_y
\end{aligned}
\end{gather}

دوری سمتیہ کا حیطہ حقیقی اور مثبت مقدار ہوتا ہے۔یوں درج بالا مساوات میں \عددی{I_0} حقیقی مثبت مقدار ہے۔

مساوات \حوالہ{مساوات_بدلتا_حقیقی_رو_ب} کو تفاعل کی \اصطلاح{وقتی دائرہ کار}\فرہنگ{وقتی دائرہ کار}\حاشیہب{time domain}\فرہنگ{time domain} صورت کہتے ہیں جبکہ مساوات \حوالہ{مساوات_بدلتا_تعددی_طرز} کو تفاعل کی \اصطلاح{تعددی دائرہ کار}\فرہنگ{تعددی دائرہ کار}\حاشیہب{frequency domain}\فرہنگ{frequency form} صورت کہتے ہیں۔
%====================
\ابتدا{مثال}
درج ذیل تفاعل کے دوری سمتیہ دریافت کریں۔
\begin{align*}
v_1(t)=20 \cos (100t +30^{\circ}), \quad v_2(t)=-40 \sin(310t -40^{\circ}), \quad i(t)=22\cos(\omega t+0.2\pi)
\end{align*}

حل:دباو \عددی{v_1(t)} کو مخلوط تفاعل کا حقیقی جزو لکھ کر
\begin{align*}
v_1(t)=\left. 20 e^{j(100 t+30^{\circ})}\right|_{\text{حقیقی}}
\end{align*}
تعدد کو ذہن نشین کرتے ہوئے،  \عددی{e^{j 100t}} نہ لکھتے  ہوئے اور زیر نوشت میں لفظ "حقیقی" نہ لکھتے ہوئے  دوری سمتیہ حاصل ہوتا ہے۔
\begin{align*}
\hat{V}_1=20 e^{j30^{\circ}}=20\phase{30^{\circ}}
\end{align*}
اسی طرح \عددی{v_2(t)} کو \عددی{\cos} کی صورت میں یوں لکھتے ہیں کہ حیطہ مثبت لکھا جائے۔
\begin{align*}
v_2=-40 \sin(310t-40^{\circ})=40 \cos (310 t-40^{\circ}+90^{\circ})=40 \cos (310t+50^{\circ})
\end{align*} 
اس کو مخلوط تفاعل کا حقیقی جزو لکھتے ہیں۔
\begin{align*}
v_2=\left.40 e^{j(310t+50^{\circ})}\right|_{\text{حقیقی}}
\end{align*} 
اس مساوات کے زیر نوشت میں لفظ "حقیقی" نہ لکھتے ہوئے اور ساتھ ہی ساتھ \عددی{e^{j 310 t}} نہ لکھتے ہوئے دوری سمتیہ حاصل ہوتی ہے یعنی
\begin{align*}
\hat{V}_2=40 e^{ j50^{\circ}}
\end{align*} 
جس کو درج ذیل بھی لکھا جا سکتا ہے۔
\begin{align*}
\hat{V}_2=40 \phase{50^{\circ}}
\end{align*} 
رو کو بھی مخلوط تفاعل کا حقیقی جزو لکھ کر
\begin{align*}
i(t)=\left. 22 e^{j(\omega t+0.2\pi)} \right|_{\text{حقیقی}}
\end{align*}
دوری سمتیہ حاصل کرتے ہیں۔
\begin{align*}
\hat{I}=22 e^{j 0.2\pi}=22\phase{0.2 \pi}
\end{align*}
\انتہا{مثال}
%=======================
\ابتدا{مشق}
درج ذیل کو تعددی دائرہ کار میں لکھیں جہاں \عددی{\omega=\SI{400}{\radian \per\second}} ہے۔
\begin{align*}
\hat{I}=35\phase{44^{\circ}}, \quad \hat{V}=12 e^{j\tfrac{\pi}{4}}, \quad \hat{I}=33\phase{-77^{\circ}}
\end{align*}

جوابات:\عددی{i(t)=35\cos(400t+44^{\circ})}، \عددی{v(t)=12\cos(400t+\frac{\pi}{4})}،\\
 \عددی{i(t)=33\cos(400t-77^{\circ})}
\انتہا{مشق}
%=====================

شکل \حوالہ{شکل_بدلتا_دوری_سمتیات} میں \عددی{\hat{I}=25\phase{20^{\circ}}} اور \عددی{\hat{V}=30 e^{j55^{\circ}}} کھینچے گئے ہیں جہاں سے دوری سمتیات کا زاویائی تعلق بھی ظاہر ہوتا ہے۔شکل \حوالہ{شکل_بدلتا_دوری_سمتیات} میں دباو سے رو \عددی{33^{\circ}} درجے پیچھے ہے۔
\begin{figure}
\centering
\begin{tikzpicture}
\pgfmathsetmacro{\iMag}{2.5}
\pgfmathsetmacro{\iAng}{20}
\pgfmathsetmacro{\vMag}{3}
\pgfmathsetmacro{\vAng}{55}
\draw(0,0)--++(4,0)node[right]{حقیقی};
\draw(0,0)--++(0,2.7)node[left]{خیالی};
\draw[-latex](0,0)--++(\iAng:\iMag)node[right]{$\hat{I}$};
\draw[-latex](0,0)--++(\vAng:\vMag)node[right]{$\hat{V}$};
\draw[-stealth]([shift={(0:0.8)}]0,0) arc (0:\iAng:0.8);
\draw(2/3*\iAng:0.9)node[right]{$20^{\circ}$};
\draw[-stealth]([shift={(0:1.6)}]0,0) arc (0:\vAng:1.6);
\draw(3/4*\vAng:1.7)node[right]{$55^{\circ}$};
\end{tikzpicture}
\caption{دوری سمتیات کے اشکال۔}
\label{شکل_بدلتا_دوری_سمتیات}
\end{figure}

کسی بھی حقیقی تفاعل مثلاً حقیقی دباو کو \عددی{v(t)=V_0\cos(\omega t+\phi)} صورت میں لکھتے ہوئے جہاں \عددی{V_0} مثبت حقیقی مقدار ہو، \عددی{V_0} اور \عددی{\phi} استعمال کرتے ہوئے دوری سمتیہ فوراً
\begin{align}
\hat{V}=V_0\phase{\phi}
\end{align}
لکھا جا سکتا ہے۔
%=================
\ابتدا{مثال}
درج ذیل کے دوری سمتیات فوراً لکھیں۔
\begin{align*}
i_1(t)&=20\cos(132t-27^{\circ}) \\
 v_1(t)&=-100\cos(20t-60^{\circ})\\
 i_2(t)&=-90\sin(450t-100^{\circ})
\end{align*}

حل:رو \عددی{i_1} میں \عددی{I_0=20} اور \عددی{\phi=-27^{\circ}} ہے لہٰذا درج ذیل لکھا جائے گا۔
\begin{align*}
\hat{I}_1=20\phase{-27^{\circ}}
\end{align*}
دباو کا حیطہ منفی ہے لہٰذا مثبت حیطہ حاصل کرنے کی خاطر دباو کو درج ذیل لکھتے ہیں
\begin{align*}
v_1(t)=100\cos(20t-60^{\circ}+180^{\circ})=100\cos(20t+120^{\circ})
\end{align*}
جس سے دوری سمتیہ درج ذیل لکھا جا سکتا ہے۔
\begin{align*}
\hat{V}_1=100\phase{120^{\circ}}
\end{align*}
رو \عددی{i_2(t)} کو \عددی{i(t)=I_0\cos(\omega t+\phi)} کی صورت میں لکھتے ہیں۔
\begin{align*}
i_2(t)=90\cos(450t-100^{\circ}+90^{\circ})=90\cos(450t-10^{\circ})
\end{align*}
یوں دوری سمتیہ درج ذیل ہو گا۔
\begin{align*}
\hat{I}_2=90\phase{-10^{\circ}}
\end{align*}
\انتہا{مثال}
%========================

\حصہ{مزاحمت، امالہ گیر اور برق گیر کے انفرادی دوری سمتی تعلق}
شکل \حوالہ{شکل_بدلتا_مزاحمت_تعددی_اور_وقتی_تفاعل} پر نظر رکھتے ہوئے پڑھیں۔مزاحمت \عددی{R} پر مخلوط دباو \عددی{v(t)=V_0e^{j(\omega t+\phi_v)}} مسلط کرنے سے  مزاحمت میں مخلوط رو \عددی{i(t)=I_0e^{j(\omega t+\phi_i)}} گزرے گی۔اوہم کے قانون کے تحت
\begin{align*}
V_0 e^{j(\omega t +\phi_v)}=R I_0 e^{j(\omega t+\phi_i)}
\end{align*}
یعنی
\begin{align*}
V_0 e^{j\phi_v}=R I_0 e^{j\phi_i}
\end{align*}
ہو گا۔اس کو دوری سمتیہ کی صورت میں
\begin{align}\label{مساوات_بدلتا_مزاحمت_دوری_تعلق}
\hat{V}=R \hat{I}
\end{align}
لکھا جا سکتا ہے جہاں
\begin{align*}
\hat{V}&=V_0e^{j\phi_v}\\
\hat{I}&=I_0 e^{j \phi_i}
\end{align*}
یعنی
\begin{gather}
\begin{aligned}\label{مساوات_بدلتا_مزاحمت_دوری_تعلق_الف}
\hat{V}&=V_0 \phase{\phi_v}\\
\hat{I}&=I_0 \phase{\phi_i}
\end{aligned}
\end{gather}
کے برابر ہیں۔اس طرح مساوات \حوالہ{مساوات_بدلتا_مزاحمت_دوری_تعلق} کو درج ذیل لکھا جا سکتا ہے۔
\begin{align*}
V_0 \phase{\phi_v} = R I_0 \phase{\phi_i}
\end{align*}

یاد رہے کہ دوری سمتیات میں \عددی{V_0} اور \عددی{I_0} حقیقی اور مثبت مقدار ہیں۔درج بالا مساوات میں بائیں ہاتھ اور دائیں ہاتھ کے مخلوط اعداد صرف اور صرف اس صورت برابر ہوں گے جب ان کے حیطے برابر ہوں اور ان کے زاویے برابر ہوں یعنی
\begin{gather}
\begin{aligned}\label{مساوات_بدلتا_مزاحمت_دوری_تعلق_ب}
V_0&=I_0 R\\
\phi_v&=\phi_i
\end{aligned}
\end{gather} 
اس طرح مزاحمت کی رو اور دباو ہم زاویہ ہیں۔مساوات \حوالہ{مساوات_بدلتا_مزاحمت_دوری_تعلق_ب} کی مدد سے مساوات \حوالہ{مساوات_بدلتا_مزاحمت_دوری_تعلق_الف} درج ذیل صورت اختیار کرتے ہیں۔
\begin{gather}
\begin{aligned}\label{مساوات_بدلتا_مزاحمت_دوری_تعلق_پ}
\hat{V}&=V_0\phase{\phi_v}\\
\hat{I}&=\frac{V_0}{R} \phase{\phi_v}
\end{aligned}
\end{gather}

شکل \حوالہ{شکل_بدلتا_مزاحمت_تعددی_اور_وقتی_تفاعل}-پ میں مزاحمت کے \عددی{\hat{I}} اور \عددی{\hat{V}} دوری سمتیات دکھائے گئے ہیں جو تعددی تفاعل ہیں جبکہ شکل \حوالہ{شکل_بدلتا_مزاحمت_تعددی_اور_وقتی_تفاعل}-ت میں مزاحمت کے \عددی{i(t)} اور \عددی{v(t)} دکھائے گئے ہیں جو وقتی تفاعل ہیں۔
\begin{figure}
\centering
\begin{subfigure}{0.5\textwidth}
\centering
\begin{tikzpicture}
\coordinate (a) at (0,0);
\coordinate (b) at (-0.025,0.5);
\coordinate (c) at (-0.04,1);
\coordinate (d) at (-0.12,1.5);
\coordinate (e) at (-0.2,2);
\coordinate (f) at (-0.15,2.5);
\coordinate (g) at (0.5,3);

\coordinate (h) at (0.7,2.5);
\coordinate (i) at (0.6,2);
\coordinate (j) at (0.75,1.5);
\coordinate (k) at (0.7,1);
\coordinate (l) at (0.7,0.5);
\coordinate (m) at (0.6,0);
%box circuit
\draw [] plot [smooth cycle] coordinates {(a) (b) (c) (d) (e) (f) (g) (h) (i) (j) (k) (l) (m)};
%controlled circuit
\draw (h) to [short,-o]++(1,0)coordinate(HH);
\draw (l) to [short,-o]++(1,0)coordinate(LL);
\draw ($(HH)!0.5!(LL)$)++(-0.3,0) node[shift={(\x/4,0)}]{$\begin{aligned} &+ \\ \hat{V} &=\hat{I} R\\&-   \end{aligned}$};
\draw(HH) to [short,i={$\hat{I}$},o-]++(\x,0) to [resistor,l={$R$}]++(0,-\y) to [short,-o] (LL);
\end{tikzpicture}
\caption*{(الف)}
\end{subfigure}%
\begin{subfigure}{0.5\textwidth}
\centering
\begin{tikzpicture}
\coordinate (a) at (0,0);
\coordinate (b) at (-0.025,0.5);
\coordinate (c) at (-0.04,1);
\coordinate (d) at (-0.12,1.5);
\coordinate (e) at (-0.2,2);
\coordinate (f) at (-0.15,2.5);
\coordinate (g) at (0.5,3);

\coordinate (h) at (0.7,2.5);
\coordinate (i) at (0.6,2);
\coordinate (j) at (0.75,1.5);
\coordinate (k) at (0.7,1);
\coordinate (l) at (0.7,0.5);
\coordinate (m) at (0.6,0);
%box circuit
\draw [] plot [smooth cycle] coordinates {(a) (b) (c) (d) (e) (f) (g) (h) (i) (j) (k) (l) (m)};
%controlled circuit
\draw (h) to [short,-o]++(1,0)coordinate(HH);
\draw (l) to [short,-o]++(1,0)coordinate(LL);
\draw ($(HH)!0.5!(LL)$)++(-0.3,0) node[shift={(\x/4,0)}]{$\begin{aligned} &+ \\ v(t) &=i(t)R\\&-   \end{aligned}$};
\draw(HH) to [short,i={$i(t)$},o-]++(\x,0) to [resistor,l={$R$}]++(0,-\y) to [short,-o] (LL);
\end{tikzpicture}
\caption*{(ب)}
\end{subfigure}
\begin{subfigure}{0.5\textwidth}
\centering
\begin{tikzpicture}
\pgfmathsetmacro{\l}{3}
\pgfmathsetmacro{\ang}{30}
%
\draw(0,0)--++(3,0)node[right]{حقیقی};
\draw(0,0)--++(0,2)node[left]{خیالی};
\draw[-latex](0,0)--++(\ang:\l)node[above left]{$\hat{V}$};
\draw[-latex](0,0)--++(\ang:0.6*\l)node[above left]{$\hat{I}$};
\draw[-stealth]([shift={(0:0.5)}]0,0) arc (0:\ang:0.5);
\draw(2/3*\ang:0.6)node[right]{$\phi_v=\phi_i$};
\end{tikzpicture}
\caption*{(پ)}
\end{subfigure}%
\begin{subfigure}{0.5\textwidth}
\centering
\begin{tikzpicture}
\begin{axis}[kStyleCircuitsA,small,xlabel=$\omega t$, ylabel=${v,i}$,ticks=none]
\addplot[domain=-120:250,samples=100]{cos(x+30)}node[pos=0.15,above left]{$v$};
\addplot[domain=-120:250,samples=100]{0.6*cos(x+30)}node[pos=0.15,below right]{$i$};
\draw[dashed](axis cs:-30,1)--(axis cs:-30,-0.3);
\draw[stealth-](axis cs:0,-0.2)--(axis cs:30,-0.2);
\draw[stealth-](axis cs:-30,-0.2)--(axis cs:-60,-0.2)--(axis cs:-60,-0.4)node[below]{$\phi_v$};
\end{axis}
\end{tikzpicture}
\caption*{(ت)}
\end{subfigure}%
\caption{مزاحمت کے دباو اور رو کے تعددی اور وقتی تفاعل۔}
\label{شکل_بدلتا_مزاحمت_تعددی_اور_وقتی_تفاعل}
\end{figure}
%==============
\ابتدا{مثال}
شکل \حوالہ{شکل_بدلتا_مزاحمت_تعددی_اور_وقتی_تفاعل}-ب میں \عددی{\SI{10}{\ohm}} کے مزاحمت پر \عددی{v(t)=22\cos(30t-66^{\circ})} دباو مسلط کی گئی ہے۔ مزاحمت کے رو کو وقتی دائرہ کار میں لکھیں۔رو کی تعددی دائرہ کار صورت شکل \حوالہ{شکل_بدلتا_مزاحمت_تعددی_اور_وقتی_تفاعل}-الف سے دریافت کریں۔

حل:اوہم کے قانون کی مدد سے رو کی وقتی دائرہ کار صورت معلوم کرتے ہیں۔
\begin{align*}
i(t)=\frac{v(t)}{R}=\frac{22\cos(30t-66^{\circ})}{10}=2.2\cos(30t-66^{\circ}) \, \si{\ampere}
\end{align*}
آئیں اب رو کی تعددی دائرہ کار صورت حاصل کرتے ہیں۔دوری دباو
\begin{align*}
\hat{V}=22\phase{-66^{\circ}}\, \si{\volt}
\end{align*}
ہے لہٰذا دوری رو درج ذیل ہو گی۔
\begin{align*}
\hat{I}=\frac{\hat{V}}{R}=\frac{22\phase{-66^{\circ}}}{10}=2.2\phase{-66^{\circ}} \, \si{\ampere}
\end{align*}
\انتہا{مثال}
%=============================
\ابتدا{مشق}
بارہ اوہم کے مزاحمت میں دوری رو \عددی{\hat{I}=37\phase{43^{\circ}} \, \si{\ampere}} ہے جبکہ تعدد \عددی{\omega=\SI{172}{\radian\per\second}} ہے۔دباو کی وقتی دائرہ کار صورت لکھیں۔

جواب:\عددی{v(t)=444\cos(172 t +43^{\circ}) \, \si{\volt}}
\انتہا{مشق}
%============================
شکل \حوالہ{شکل_بدلتا_امالہ_تعددی_اور_وقتی_تفاعل} پر نظر رکھتے ہوئے پڑھیں۔امالہ گیر کے دباو اور رو کا تعلق درج ذیل ہے۔
\begin{align}\label{مساوات_بدلتا_امالہ_وقتی_تعلق}
v=L\frac{\dif i(t)}{\dif t}
\end{align}
امالہ گیر پر مخلوط دباو \عددی{v(t)=V_0e^{j(\omega t+\phi_v)}} مسلط کرنے سے اس میں مخلوط رو \عددی{i(t)=I_0e^{j(\omega t+\phi_i)}} پیدا ہو گی۔ان قیمتوں کو درج بالا مساوات میں پُر کرنے سے
\begin{align*}
V_0 e^{j(\omega t+\phi_v)}&=L \frac{\dif}{\dif t} \left[I_0 e^{j(\omega t +\phi_i)}\right]\\
&=j \omega L I_0 e^{j(\omega t+\phi_i)}
\end{align*}
یعنی
\begin{align*}
V_0 e^{j\phi_v}=j\omega L I_0 e^{j\phi_i}
\end{align*}
ملتا ہے جو دوری مساوات ہے۔یہ دوری مساوات درج ذیل لکھی جائے گی۔
\begin{align}\label{مساوات_بدلتا_امالہ_تعددی_تعلق_الف}
\hat{V}=j\omega L \hat{I}
\end{align} 
آپ نے دیکھا کہ مساوات \حوالہ{مساوات_بدلتا_امالہ_وقتی_تعلق} جو تفرقی اور وقتی مساوات ہے سے مساوات \حوالہ{مساوات_بدلتا_امالہ_تعددی_تعلق_الف} حاصل ہوتا ہے جو تعددی اور الجبرائی مساوات ہے۔دوری سمتیات کی مدد سے تفرقی مساوات سے الجبرائی مساوات حاصل ہوتے ہیں۔آپ جانتے ہیں کہ الجبرائی مساوات حل کرنا نہایت آسان ہوتا ہے جبکہ تفرقی مساوات کو حل کرنا دشوار ہوتا ہے۔یہی وجہ ہے کہ دوری سمتیات اتنے مقبول ہیں۔

آپ جانتے ہیں کہ
\begin{align}\label{مساوات_بدلتا_عمود}
\phase{90^{\circ}}=e^{j90^{\circ}}=\cos 90^{\circ}+j \sin 90^{\circ}=j
\end{align}
لکھا جا سکتا ہے لہٰذا مساوات \حوالہ{مساوات_بدلتا_امالہ_تعددی_تعلق_الف} کو 
\begin{align*}
\hat{V}=\omega L \hat{I} e^{j90^{\circ}}
\end{align*} 
یعنی
\begin{align}\label{مساوات_بدلتا_امالہ_تعددی_تعلق_ب}
V_0 e^{j\phi_v}=\omega L I_0 e^{j (\phi_i+90^{\circ})}
\end{align} 
لکھا جا سکتا ہے۔مساوات \حوالہ{مساوات_بدلتا_امالہ_تعددی_تعلق_ب} میں دونوں ہاتھ کے مخلوط اعداد صرف اور صرف اس وقت برابر ہوں گے جب ان کے حیطے برابر ہوں اور ان کے زاویے برابر ہوں لہٰذا اس مساوات کے تحت
\begin{gather}
\begin{aligned}\label{مساوات_بدلتا_امالہ_رو_پیچھے_ہے}
V_0&=\omega L I_0\\
\phi_v&=\phi_i+90^{\circ}
\end{aligned}
\end{gather}
ہوں گے۔یوں دباو کا زاویہ، رو کے زاویے سے \عددی{90^{\circ}} درجے زیادہ ہے لہٰذا رو سے دباو \عددی{90^{\circ}} درجے آگے ہے یا دباو سے رو \عددی{90^{\circ}} پیچھے ہے۔شکل \حوالہ{شکل_بدلتا_امالہ_تعددی_اور_وقتی_تفاعل}-پ میں دوری سمتیات دکھائے گئے ہیں جہاں دباو سے رو \عددی{90^{\circ}} درجے پیچھے دکھایا گیا ہے۔

مساوات \حوالہ{مساوات_بدلتا_امالہ_تعددی_تعلق_ب} سے وقتی مساوات درج ذیل لکھی جائے گی جہاں مساوات \حوالہ{مساوات_بدلتا_امالہ_رو_پیچھے_ہے} کے تحت \عددی{\phi_v=\phi_i+90^{\circ}} ہو گا۔
\begin{align}
V_0 \cos (\omega t +\phi_v)=\omega L I_0 \cos (\omega t +\phi_i+90^{\circ})
\end{align}
درج بالا مساوات میں دیے دباو اور رو کو شکل \حوالہ{شکل_بدلتا_امالہ_تعددی_اور_وقتی_تفاعل}-ت میں دکھایا گیا ہے۔
%
\begin{figure}
\centering
\begin{subfigure}{0.5\textwidth}
\centering
\begin{tikzpicture}
\coordinate (a) at (0,0);
\coordinate (b) at (-0.025,0.5);
\coordinate (c) at (-0.04,1);
\coordinate (d) at (-0.12,1.5);
\coordinate (e) at (-0.2,2);
\coordinate (f) at (-0.15,2.5);
\coordinate (g) at (0.5,3);

\coordinate (h) at (0.7,2.5);
\coordinate (i) at (0.6,2);
\coordinate (j) at (0.75,1.5);
\coordinate (k) at (0.7,1);
\coordinate (l) at (0.7,0.5);
\coordinate (m) at (0.6,0);
%box circuit
\draw [] plot [smooth cycle] coordinates {(a) (b) (c) (d) (e) (f) (g) (h) (i) (j) (k) (l) (m)};
%controlled circuit
\draw (h) to [short,-o]++(1,0)coordinate(HH);
\draw (l) to [short,-o]++(1,0)coordinate(LL);
\draw ($(HH)!0.5!(LL)$)++(-0.3,0) node[shift={(\x/4,0)}]{$\begin{aligned} &+ \\ \hat{V} &=j \omega L \hat{I} \\&-   \end{aligned}$};
\draw(HH) to [short,i={$\hat{I}$},o-]++(\x,0) to [inductor,l={$L$}]++(0,-\y) to [short,-o] (LL);
\end{tikzpicture}
\caption*{(الف)}
\end{subfigure}%
\begin{subfigure}{0.5\textwidth}
\centering
\begin{tikzpicture}
\coordinate (a) at (0,0);
\coordinate (b) at (-0.025,0.5);
\coordinate (c) at (-0.04,1);
\coordinate (d) at (-0.12,1.5);
\coordinate (e) at (-0.2,2);
\coordinate (f) at (-0.15,2.5);
\coordinate (g) at (0.5,3);

\coordinate (h) at (0.7,2.5);
\coordinate (i) at (0.6,2);
\coordinate (j) at (0.75,1.5);
\coordinate (k) at (0.7,1);
\coordinate (l) at (0.7,0.5);
\coordinate (m) at (0.6,0);
%box circuit
\draw [] plot [smooth cycle] coordinates {(a) (b) (c) (d) (e) (f) (g) (h) (i) (j) (k) (l) (m)};
%controlled circuit
\draw (h) to [short,-o]++(1,0)coordinate(HH);
\draw (l) to [short,-o]++(1,0)coordinate(LL);
\draw ($(HH)!0.5!(LL)$)++(-0.3,0) node[shift={(\x/4,0)}]{$\begin{aligned} &+ \\ v(t) &=L \frac{\dif i(t)}{\dif t}\\&-   \end{aligned}$};
\draw(HH) to [short,i={$i(t)$},o-]++(\x,0) to [inductor,l={$L$}]++(0,-\y) to [short,-o] (LL);
\end{tikzpicture}
\caption*{(ب)}
\end{subfigure}
\begin{subfigure}{0.5\textwidth}
\centering
\begin{tikzpicture}
\pgfmathsetmacro{\l}{3}
\pgfmathsetmacro{\ang}{30}
\pgfmathsetmacro{\angA}{\ang-90}
%
\draw(0,0)--++(3,0)node[right]{حقیقی};
\draw(0,0)--++(0,2)node[left]{خیالی};
\draw[-latex](0,0)--++(\ang:\l)node[above left]{$\hat{V}$};
\draw[-latex](0,0)--++(\ang-90:0.7*\l)node[below right]{$\hat{I}$};
\draw[-stealth]([shift={(0:0.6)}]0,0) arc (0:\ang:0.6);
\draw(2/3*\ang:0.7)node[right]{$\phi_v$};
\draw[-stealth]([shift={(0:0.5)}]0,0) arc (0:\angA:0.5);
\draw(2/3*\angA:0.6)node[right]{$\phi_i$};
\draw[stealth-stealth]([shift={(\angA:1.5)}]0,0) arc (\angA:\ang:1.5);
\draw(0.5*\angA:1.6)node[right]{$90^{\circ}$};
\end{tikzpicture}
\caption*{(پ)}
\end{subfigure}%
\begin{subfigure}{0.5\textwidth}
\centering
\begin{tikzpicture}
\begin{axis}[kStyleCircuitsA,small,xlabel=$\omega t$, ylabel=${v,i}$,ticks=none]
\addplot[domain=-120:250,samples=100]{cos(x+30)}node[pos=0.15,above left]{$v$};
\addplot[domain=-120:250,samples=100]{0.7*cos(x+30-90)}node[pos=0.5,above  right]{$i$};
\draw[dashed](axis cs:-30,1)--(axis cs:-30,-0.3);
\draw[dashed](axis cs:60,0.7)--(axis cs:60,-0.3);
\draw[stealth-](axis cs:60,-0.2)--(axis cs:90,-0.2);
\draw[stealth-](axis cs:-30,-0.2)--(axis cs:-90,-0.2)--(axis cs:-90,-0.3)node[left]{$90^{\circ}$};
\end{axis}
\end{tikzpicture}
\caption*{(ت)}
\end{subfigure}%
\caption{امالہ کے دباو اور رو کے تعددی اور وقتی تفاعل۔}
\label{شکل_بدلتا_امالہ_تعددی_اور_وقتی_تفاعل}
\end{figure}
%============
\ابتدا{مثال}
شکل \حوالہ{شکل_بدلتا_امالہ_تعددی_اور_وقتی_تفاعل} میں \عددی{\SI{4}{\milli\henry}} امالہ گیر پر \عددی{v(t)=12\cos(1000t+22^{\circ})} دباو مسلط کی جاتی ہے۔امالہ گیر کی رو دریافت کریں۔

حل:دوری سمتیہ دباو درج ذیل ہے۔
\begin{align*}
\hat{V}=12\phase{22^{\circ}}
\end{align*}
مساوات \حوالہ{مساوات_بدلتا_امالہ_تعددی_تعلق_الف} کی مدد سے دوری سمتیہ رو حاصل کرتے ہیں
\begin{align*}
\hat{I}&=\frac{\hat{V}}{j \omega L}\\
&=\frac{12\phase{22^{\circ}}}{j1000\times 0.004}\\
&=\frac{12\phase{22^{\circ}}}{4\phase{90^{\circ}}}\\
&=3\phase{-68^{\circ}} \, \si{\ampere}
\end{align*}
جہاں مساوات \حوالہ{مساوات_بدلتا_عمود} کا استعمال کرتے ہوئے \عددی{j=\phase{90^{\circ}}} لکھا گیا  ہے۔یوں رو کی وقتی دائرہ کار صورت درج ذیل ہو گی۔
\begin{align*}
i(t)=3\cos(1000t-68^{\circ}) \, \si{\ampere}
\end{align*}
\انتہا{مثال}
%================
\ابتدا{مشق}
امالہ کی قیمت \عددی{\SI{10}{\milli\henry}} جبکہ اس میں رو \عددی{\hat{I}=8\phase{44^{\circ}}} کی تعدد \عددی{\SI{500}{\radian \per\second}} ہے۔دباو کی وقتی دائرہ کار صورت دریافت کریں۔

جواب:\عددی{v(t)=40\cos(500t+134^{\circ})\, \si{\volt}}
\انتہا{مشق}
%===============
شکل \حوالہ{شکل_بدلتا_برق_گیر_تعددی_اور_وقتی_تفاعل} پر نظر رکھتے ہوئے پڑھیں جہاں برق گیر پر دباو \عددی{v(t)=V_0\cos(\omega t +\phi_v)} مسلط کی گئی ہے۔برق گیر کی تفرقی مساوات
\begin{align}\label{مساوات_بدلتا_برق_گیر_تفرقی_مساوات}
i(t)=C \frac{\dif v(t)}{\dif t}
\end{align}
میں مخلوط دباو اور مخلوط رو پُر کرتے ہوئے
\begin{align*}
I_0 e^{j(\omega t +\phi_i)}&=C \frac{\dif}{\dif t} \left[V_0 e^{j(\omega t +\phi_v)} \right]\\
&=j \omega C V_0 e^{j(\omega t +\phi_v)}
\end{align*}
یعنی
\begin{align*}
I_0 e^{j\phi_i}=j \omega C e^{j \phi_v}
\end{align*}
حاصل ہوتا ہے جس کو دوری سمتیہ کی صورت میں لکھتے ہیں۔
\begin{align}\label{مساوات_بدلتا_برق_گیر_دوری_مساوات_الف}
\hat{I}=j \omega C \hat{V}
\end{align}
مساوات \حوالہ{مساوات_بدلتا_برق_گیر_تفرقی_مساوات} برق گیر کی تفرقی مساوات ہے جبکہ مساوات \حوالہ{مساوات_بدلتا_برق_گیر_دوری_مساوات_الف} برق گیر کی الجبرائی مساوات ہے۔
 
مساوات \حوالہ{مساوات_بدلتا_برق_گیر_دوری_مساوات_الف} میں \عددی{j=e^{j90^{\circ}}} لکھنے سے درج ذیل ملتا ہے۔
\begin{align}\label{مساوات_بدلتا_برق_گیر_دوری_مساوات_ب}
I_0 e^{j\phi_i}= \omega C e^{j (\phi_v+90^{\circ})}
\end{align}
اس مساوات کے دونوں اطراف صرف اور صرف اس صورت برابر ہو سکتے ہیں جب دونوں اطراف کے حیطے برابر ہوں اور ان کے زاویے برابر ہوں۔
\begin{gather}
\begin{aligned}
I_0&=\omega C V_0\\
\phi_i&=\phi_v+90^{\circ}
\end{aligned}
\end{gather}
درج بالا مساوات کے تحت دباو سے رو \عددی{90^{\circ}} درجے آگے ہے۔

مساوات \حوالہ{مساوات_بدلتا_برق_گیر_دوری_مساوات_ب} سے وقتی دائرہ کار صورت لکھتے ہیں جہاں درج بالا مساوات کے تحت \عددی{\phi_i=\phi_v+90^{\circ}} ہو گا۔
\begin{align}
I_0 \cos (\omega t +\phi_i)=\omega C V_0 \cos(\omega t +\phi_v+90^{\circ})
\end{align}
شکل \حوالہ{شکل_بدلتا_برق_گیر_تعددی_اور_وقتی_تفاعل}-پ میں دوری سمتیات دکھائے گئے ہیں جبکہ شکل-ت میں دباو اور رو کی وقتی دائرہ کار صورت دکھائی گئی ہے۔

% 
\begin{figure}
\centering
\begin{subfigure}{0.5\textwidth}
\centering
\begin{tikzpicture}
\coordinate (a) at (0,0);
\coordinate (b) at (-0.025,0.5);
\coordinate (c) at (-0.04,1);
\coordinate (d) at (-0.12,1.5);
\coordinate (e) at (-0.2,2);
\coordinate (f) at (-0.15,2.5);
\coordinate (g) at (0.5,3);

\coordinate (h) at (0.7,2.5);
\coordinate (i) at (0.6,2);
\coordinate (j) at (0.75,1.5);
\coordinate (k) at (0.7,1);
\coordinate (l) at (0.7,0.5);
\coordinate (m) at (0.6,0);
%box circuit
\draw [] plot [smooth cycle] coordinates {(a) (b) (c) (d) (e) (f) (g) (h) (i) (j) (k) (l) (m)};
%controlled circuit
\draw (h) to [short,-o]++(1,0)coordinate(HH);
\draw (l) to [short,-o]++(1,0)coordinate(LL);
\draw ($(HH)!0.5!(LL)$)++(-0.3,0) node[shift={(\x/4,0)}]{$\begin{aligned} &+ \\& \hat{V} \\&-   \end{aligned}$};
\draw(HH) to [short,i={${\hat{I}=j \omega C \hat{V}}$},o-]++(\x,0) to [capacitor,l={$C$}]++(0,-\y) to [short,-o] (LL);
\end{tikzpicture}
\caption*{(الف)}
\end{subfigure}%
\begin{subfigure}{0.5\textwidth}
\centering
\begin{tikzpicture}
\coordinate (a) at (0,0);
\coordinate (b) at (-0.025,0.5);
\coordinate (c) at (-0.04,1);
\coordinate (d) at (-0.12,1.5);
\coordinate (e) at (-0.2,2);
\coordinate (f) at (-0.15,2.5);
\coordinate (g) at (0.5,3);

\coordinate (h) at (0.7,2.5);
\coordinate (i) at (0.6,2);
\coordinate (j) at (0.75,1.5);
\coordinate (k) at (0.7,1);
\coordinate (l) at (0.7,0.5);
\coordinate (m) at (0.6,0);
%box circuit
\draw [] plot [smooth cycle] coordinates {(a) (b) (c) (d) (e) (f) (g) (h) (i) (j) (k) (l) (m)};
%controlled circuit
\draw (h) to [short,-o]++(1,0)coordinate(HH);
\draw (l) to [short,-o]++(1,0)coordinate(LL);
\draw ($(HH)!0.5!(LL)$)++(-0.3,0) node[shift={(\x/4,0)}]{$\begin{aligned} &+ \\& v(t)\\&-   \end{aligned}$};
\draw(HH) to [short,i={${i(t)=C\frac{\dif v(t)}{\dif t}}$},o-]++(\x,0) to [capacitor,l={$C$}]++(0,-\y) to [short,-o] (LL);
\end{tikzpicture}
\caption*{(ب)}
\end{subfigure}
\begin{subfigure}{0.5\textwidth}
\centering
\begin{tikzpicture}
\pgfmathsetmacro{\l}{3}
\pgfmathsetmacro{\ang}{30}
\pgfmathsetmacro{\angA}{\ang+90}
%
\draw(0,0)--++(3,0)node[right]{حقیقی};
\draw(0,0)--++(0,2)node[left]{خیالی};
\draw[-latex](0,0)--++(\ang:\l)node[above left]{$\hat{V}$};
\draw[-latex](0,0)--++(\angA:0.7*\l)node[above left]{$\hat{I}$};
\draw[-stealth]([shift={(0:0.7)}]0,0) arc (0:\ang:0.7);
\draw(2/3*\ang:0.8)node[right]{$\phi_v$};
\draw[-stealth]([shift={(0:0.5)}]0,0) arc (0:\angA:0.5);
\draw(2/3*\angA:0.6)node[right]{$\phi_i$};
\draw[stealth-stealth]([shift={(\angA:1.5)}]0,0) arc (\angA:\ang:1.5);
\draw(0.5*\angA:1.6)node[right]{$90^{\circ}$};
\end{tikzpicture}
\caption*{(پ)}
\end{subfigure}%
\begin{subfigure}{0.5\textwidth}
\centering
\begin{tikzpicture}
\begin{axis}[kStyleCircuitsA,small,xlabel=$\omega t$, ylabel=${v,i}$,ticks=none]
\addplot[domain=-30:340,samples=100]{cos(x-60)}node[pos=0.25,above right]{$v$};
\addplot[domain=-120:340,samples=100]{0.7*cos(x-60+90)}node[pos=0.2,above left]{$i$};
\draw[dashed](axis cs:-30,0)--(axis cs:-30,-0.3);
\draw[dashed](axis cs:-120,0)--(axis cs:-120,-0.3);
\draw[stealth-](axis cs:-30,-0.2)--(axis cs:-10,-0.2);
\draw[stealth-](axis cs:-120,-0.2)--(axis cs:-150,-0.2);
\draw(axis cs:-75,-0.2)node[]{$90^{\circ}$};
\end{axis}
\end{tikzpicture}
\caption*{(ت)}
\end{subfigure}%
\caption{برق گیر کے دباو اور رو کے تعددی اور وقتی تفاعل۔}
\label{شکل_بدلتا_برق_گیر_تعددی_اور_وقتی_تفاعل}
\end{figure}
%=========================
\ابتدا{مثال}
شکل \حوالہ{شکل_بدلتا_برق_گیر_تعددی_اور_وقتی_تفاعل} میں \عددی{\SI{100}{\micro\farad}} برق گیر پر \عددی{v(t)=7\cos(5000t-60^{\circ}) \, \si{\volt}} کا دباو مسلط کیا گیا ہے۔رو حاصل کریں۔

حل:مسلط دباو کی دوری سمتیہ لکھتے ہیں۔
\begin{align*}
\hat{V}=7\phase{-60^{\circ}}
\end{align*}
یوں رو درج ذیل ہو گی
\begin{align*}
\hat{I}&=j \omega C \hat{V}\\
&=j 5000 \times 100 \times 10^{-6} 7 \phase{-60^{\circ}}\\
&=3.5 \phase{-60^{\circ}+90^{\circ}}\\
&=3.5\phase{30^{\circ}} \, \si{\ampere}
\end{align*}
جس کی وقتی دائرہ کار صورت درج ذیل ہے۔
\begin{align*}
i(t)=3.5\cos(5000 t+30^{\circ}) \, \si{\ampere}
\end{align*}
\انتہا{مثال}
%===========================
\ابتدا{مشق}
شکل \حوالہ{شکل_بدلتا_برق_گیر_تعددی_اور_وقتی_تفاعل} میں \عددی{\SI{330}{\micro\farad}} برق گیر کی رو \عددی{\hat{I}=11\phase{-12^{\circ}} \, \si{\ampere}} ہے۔ رو کی تعدد \عددی{\SI{6000}{\hertz}} ہے۔دباو کی وقتی دائرہ کار صورت حاصل کریں۔

جواب:\عددی{v(t)=0.884 \cos(12000\pi t-102^{\circ}) \, \si{\volt}}
\انتہا{مشق}
%=============================

\حصہ{برقی رکاوٹ اور برقی فراوانی}
قانون اوہم کے تحت برقی مزاحمت کو \عددی{R=\tfrac{V}{I}} لکھا جا سکتا ہے۔بالکل اسی طرح، شکل \حوالہ{شکل_بدلتا_برقی_رکاوٹ_تعریف} میں دوری سمتیہ دباو اور دوری سمتیہ رو کی شرح کو \اصطلاح{برقی رکاوٹ}\فرہنگ{رکاوٹ}\فرہنگ{برقی!رکاوٹ}\حاشیہب{impedance}\فرہنگ{impedance} کہتے اور  \عددی{\bZ} سے ظاہر کرتے ہیں۔
\begin{align}\label{مساوات_بدلتا_رکاوٹ_تعریف_الف}
\bZ=\frac{\hat{V}}{\hat{I}}
\end{align}
برقی رکاوٹ کو عموماً \اصطلاح{رکاوٹ} کہا جاتا ہے۔چونکہ \عددی{\hat{V}} اور \عددی{\hat{I}} مخلوط اعداد ہیں لہٰذا \عددی{{\bf{Z}}} بھی مخلوط عدد ہو گا۔
\begin{align}
\bZ=\frac{V_0\phase{\phi_v}}{I_0\phase{\phi_i}}=\frac{V_0}{I_0}\phase{\phi_v-\phi_i}=Z_0\phase{\phi_z}
\end{align}
چونکہ دباو اور رو کی شرح کو اوہم \عددی{\si{\ohm}} میں ناپتے ہیں لہٰذا رکاوٹ کی اکائی بھی اوہم ہے۔یوں بدلتی رو دور کی رکاوٹ یک سمتی رو دور کی مزاحمت کی مانند ہے۔رکاوٹ کو مستطیل طرز میں بھی لکھا جا سکتا ہے
\begin{align}
\bZ(\omega)=R(\omega)+j X(\omega)
\end{align} 
جہاں \عددی{R} حقیقی جزو  یعنی \اصطلاح{مزاحمت}\فرہنگ{مزاحمت}\حاشیہب{resistive}\فرہنگ{resistive} ہے جبکہ \عددی{X} خیالی جزو یعنی \اصطلاح{متعاملیت}\فرہنگ{متعاملیت}\حاشیہب{reactance}\فرہنگ{reactance} ہے۔ رکاوٹ مخلوط عدد ہے نا کہ دوری سمتیہ چونکہ دوری سمتیہ سائن نما تفاعل کو ظاہر کرتی ہے جبکہ رکاوٹ سائن نما تفاعل نہیں ہے۔
\tikzexternaldisable
\begin{figure}
\centering
\begin{tikzpicture}
\draw(0,0)--++(0,2.5)--++(2,0)--++(0,-2.5)--cycle;
\draw(1,1.25)node{\RL{بدلتی رو دور}};
\draw(0,0.25) to [short]++(-\x,0) to [american voltage source,l={${V_0 \phase{\phi_v}}$}]++(0,\y) to [short,i={$\vspace{25pt} {I_0\phase{\phi_i}}$}]++(\x,0);
\draw[stealth-] (-0.5,1.25)--++(-\x/4,0)--++(0,-\y/8)node[below]{$Z \phase{\phi_z}$};
\end{tikzpicture}
\caption{برقی رکاوٹ کی تعریف۔}
\label{شکل_بدلتا_برقی_رکاوٹ_تعریف}
\end{figure}

کسی بھی مخلوط عدد کی طرح، رکاوٹ کو بھی مستطیل طرز اور زاویائی طرز میں لکھا جا سکتا ہے
\begin{align}
\bZ=Z\phase{\phi_z}=R+jX
\end{align}
جہاں ایک طرز سے دوسری طرز میں تبادلہ درج ذیل مساوات سے کیا جاتا ہے۔
\begin{gather}
\begin{aligned}
R&=Z \cos \phi_z\\
X&=Z \sin \phi_z \quad \quad \text{\RL{زاویائی سے مستطیل طرز}}
\end{aligned}
\end{gather}  
%
\begin{gather}
\begin{aligned}
Z&=\sqrt{R^2+X^2}\\
\phi_z&=\tan^{-1}\frac{X}{R} \quad \quad \text{\RL{مستطیل سے زاویائی طرز}}
\end{aligned}
\end{gather}

مساوات \حوالہ{مساوات_بدلتا_رکاوٹ_تعریف_الف} رکاوٹ کی تعریف ہے۔اسے استعمال کرتے ہوئے مساوات \حوالہ{مساوات_بدلتا_مزاحمت_دوری_تعلق}، مساوات \حوالہ{مساوات_بدلتا_امالہ_تعددی_تعلق_الف} اور مساوات \حوالہ{مساوات_بدلتا_برق_گیر_دوری_مساوات_الف} سے بالترتیب مزاحمت، امالہ گیر اور برق گیر کی رکاوٹ لکھتے ہیں۔
\begin{gather}
\begin{aligned}\label{مساوات_بدلتا_پرزوں_کی_رکاوٹ_الف}
Z_R&=R\\
Z_L&=j \omega L=j X_L\\
Z_C&=\frac{1}{j \omega C}=-\frac{j}{\omega C}=-j X_C
\end{aligned}
\end{gather}
درج بالا میں برق گیر کی رکاوٹ لکھتے ہوئے \عددی{\tfrac{1}{j}=\tfrac{1\times j}{j \times j}=\frac{j}{-1}=-j} کا استعمال کیا گیا ہے۔یوں امالی متعاملیت اور برق گیری متعاملیت درج ذیل ہیں۔
\begin{gather}
\begin{aligned}
X_L&=\omega L\\
X_C&=\frac{1}{\omega C}
\end{aligned}
\end{gather} 

%========================
\ابتدا{مشق}
مزاحمت \عددی{R=\SI{30}{\ohm}}، امالہ \عددی{L=\SI{20}{\milli\henry}} اور برق گیر \عددی{C=\SI{2000}{\micro\farad}} کی رکاوٹ \عددی{\SI{100}{\radian\per\second}}، \عددی{\SI{1000}{\radian\per\second}} اور \عددی{\SI{9000}{\hertz}} تعدد پر دریافت کریں۔

جوابات:پہلی تعدد پر \عددی{Z_R=\SI{30}{\ohm}}، \عددی{Z_L=j2 \, \si{\ohm}}، \عددی{Z_C=-j5 \, \si{\ohm}} ہیں۔\\
دوسری تعدد پر \عددی{Z_R=\SI{30}{\ohm}}، \عددی{Z_L=j20 \, \si{\ohm}}، \عددی{Z_C=-j0.5 \, \si{\ohm}} ہیں۔\\
تیسری تعدد پر \عددی{Z_R=\SI{30}{\ohm}}، \عددی{Z_L=j1131 \, \si{\ohm}}، \عددی{Z_C=-j0.00884 \, \si{\ohm}} ہیں۔\\
\انتہا{مشق}
%==================

قوانین کرخوف وقتی دائرہ کار کے علاوہ تعددی دائرہ کار میں بھی لاگو ہوتے ہیں۔صفحہ \حوالہصفحہ{حصہ_مزاحمت_سلسلہ_وار_مساوی} پر حصہ \حوالہ{حصہ_مزاحمت_سلسلہ_وار_مساوی} میں سلسلہ وار جڑے مزاحمتوں کا مساوی مزاحمت مساوات \حوالہ{مساوات_مزاحمتی_متعدد_سلسلہ_مساوی_مزاحمت} میں حاصل کیا گیا۔اسی طرح صفحہ \حوالہصفحہ{حصہ_مزاحمت_متعدد_متوازی_کا_مساوی} پر حصہ \حوالہ{حصہ_مزاحمت_متعدد_متوازی_کا_مساوی} میں متعدد مساوی مزاحمتوں کا مساوی مزاحمت مساوات \حوالہ{مساوات_مزاحمتی_متوازی_مساوی} میں پیش کیا گیا۔بالکل اسی طرح متعدد سلسلہ وار جڑے رکاوٹ اور متعدد متوازی رکاوٹ کے مساوی رکاوٹ حاصل کی جا سکتی ہے۔مشق میں آپ سے ایسا ہی کرنے کو کہا گیا ہے۔

مساوات \حوالہ{مساوات_بدلتا_سلسلہ_وار_مساوی_رکاوٹ_الف} متعدد سلسلہ وار رکاوٹ کی مساوی رکاوٹ دیتی ہے جبکہ مساوات \حوالہ{مساوات_بدلتا_متوازی_مساوی_رکاوٹ_الف} متعدد متوازی رکاوٹوں کی مساوی رکاوٹ دیتی ہے۔
\begin{align}\label{مساوات_بدلتا_سلسلہ_وار_مساوی_رکاوٹ_الف}
\bZ_s=\bZ_1+\bZ_2+\bZ_3+\cdots+\bZ_n \quad \text{\RL{سلسلہ وار رکاوٹوں کا مساوی رکاوٹ}}
\end{align}
%
\begin{align}\label{مساوات_بدلتا_متوازی_مساوی_رکاوٹ_الف}
\frac{1}{\bZ_m}=\frac{1}{\bZ_1}+\frac{1}{\bZ_2}+\frac{1}{\bZ_3}+\cdots+\frac{1}{\bZ_n} \quad \text{\RL{متوازی رکاوٹوں کا مساوی رکاوٹ}}
\end{align}
آپ دیکھ سکتے ہیں کہ یہ مساوات ہوبہو مزاحمتوں کی مساوات کی طرح ہیں۔
%=============
\ابتدا{مشق}
صفحہ \حوالہصفحہ{شکل_مزاحمتی_متعدد_مزاحمت_تقسیم_دباو} پر شکل \حوالہ{شکل_مزاحمتی_متعدد_مزاحمت_تقسیم_دباو} میں سلسلہ وار مزاحمت جڑے دکھائے گئے ہیں۔مزاحمتوں کی جگہ رکاوٹ نسب کرتے ہوئے، مخلوط دباو اور مخلوط رو کے استعمال سے مساوی رکاوٹ کی مساوات حاصل کریں۔ اسی طرح متعدد رکاوٹوں کو متوازی جوڑتے ہوئے ان کا مساوی رکاوٹ حاصل کریں۔

جوابات:مساوات \حوالہ{مساوات_بدلتا_سلسلہ_وار_مساوی_رکاوٹ_الف} اور مساوات \حوالہ{مساوات_بدلتا_متوازی_مساوی_رکاوٹ_الف}
\انتہا{مشق}
%===============
\ابتدا{مثال}
متعدد برق گیر سلسلہ وار جڑے ہیں۔ان کی انفرادی رکاوٹیں استعمال کرتے ہوئے مساوی رکاوٹ حاصل کریں۔مساوی رکاوٹ سے مساوی برقی گیر دریافت کریں۔

حل:برق گیر \عددی{C_1} تا \عددی{C_n} کی \عددی{\omega} تعدد پر رکاوٹیں \عددی{\tfrac{1}{j\omega C_1}}، \عددی{\tfrac{1}{j\omega C_2}} ،\عددی{\cdots}،\عددی{\tfrac{1}{j\omega C_n}} ہوں گی۔ان کے مساوی برق گیر کو \عددی{C_s} کہتے ہوئے مساوی رکاوٹ \عددی{\tfrac{1}{j\omega C_s}} لکھا جائے گا۔یوں مساوات \حوالہ{مساوات_بدلتا_سلسلہ_وار_مساوی_رکاوٹ_الف} کے تحت درج ذیل لکھا جا سکتا ہے۔
\begin{align*}
\frac{1}{j\omega C_s}=\frac{1}{j\omega C_1}+\frac{1}{j\omega C_2}+\frac{1}{j\omega C_3}+\cdots+\frac{1}{j\omega C_n}
\end{align*}
اس مساوات کے دونوں اطراف کو \عددی{j\omega} سے ضرب دیتے ہوئے درج ذیل ملتا ہے جو عین مساوات \حوالہ{مساوات_امالہ_سلسلہ_وار_برق_گیر_کا_مساوی} ہی ہے۔
\begin{align*}
\frac{1}{C_s}=\frac{1}{C_1}+\frac{1}{C_2}+\frac{1}{C_3}+\cdots+\frac{1}{C_n}
\end{align*}
\انتہا{مثال}
%====================================
\ابتدا{مشق}
متعدد برق گیر متوازی جڑے ہیں۔ان کی رکاوٹیں استعمال کرتے ہوئے مساوات \حوالہ{مساوات_بدلتا_متوازی_مساوی_رکاوٹ_الف} کی مدد سے ان کا مساوی رکاوٹ حاصل کریں۔مساوی رکاوٹ سے مساوی برق گیر کی مساوات حاصل کریں۔متعدد متوازی برق گیر کا مساوی برق گیر مساوات \حوالہ{مساوات_امالہ_متوازی_برق_گیر_کا_مساوی} دیتی ہے۔
\انتہا{مشق}
%==================================
\ابتدا{مشق}
متعدد امالہ گیر متوازی جڑے ہیں۔ان کی رکاوٹیں استعمال کرتے ہوئے مساوات \حوالہ{مساوات_بدلتا_متوازی_مساوی_رکاوٹ_الف} کی مدد سے ان کا مساوی رکاوٹ حاصل کریں۔مساوی رکاوٹ سے مساوی امالہ گیر کی مساوات حاصل کریں۔

جواب:صفحہ \حوالہصفحہ{مساوات_امالہ_متوازی_مساوی_مساوات_قیمت} پر مساوات \حوالہ{مساوات_امالہ_متوازی_مساوی_مساوات_قیمت}
\انتہا{مشق}
%==================================
\ابتدا{مشق}
متعدد امالہ گیر سلسہ جڑے ہیں۔ان کی رکاوٹیں استعمال کرتے ہوئے مساوات \حوالہ{مساوات_بدلتا_سلسلہ_وار_مساوی_رکاوٹ_الف} کی مدد سے ان کا مساوی رکاوٹ حاصل کریں۔مساوی رکاوٹ سے مساوی امالہ گیر کی مساوات حاصل کریں۔

جواب:صفحہ \حوالہصفحہ{مساوات_امالہ_سلسلہ_وار_امالہ_گیر_الف} پر مساوات \حوالہ{مساوات_امالہ_سلسلہ_وار_امالہ_گیر_الف}
\انتہا{مشق}
%==================================
\ابتدا{مثال}\شناخت{مثال_بدلتا_سلسلہ_وار_مزاحمت_امالہ_برق_گیر_الف}
شکل \حوالہ{شکل_بدلتا_سلسلہ_وار_مزاحمت_امالہ_برق_گیر_الف} میں منبع دباو کو درپیش مساوی رکاوٹ \عددی{\SI{50}{\hertz}} اور \عددی{\SI{2000}{\radian\per\second}} تعدد پر دریافت کریں۔دباو \عددی{v(t)=30\cos(\omega t+45^{\circ}) \, \si{\volt}} کی صورت میں دونوں تعدد پر وقتی دائرہ کار میں رو دریافت کریں۔
\begin{figure}
\centering
\begin{tikzpicture}
\draw(0,0) to [american voltage source,l={$v(t)$}]++(0,\y) to [resistor,i={$i(t)$},l={$\SI{10}{\ohm}$}]++(\x,0) to [inductor,l={$\SI{2}{\milli\henry}$}]++(\x,0) to [capacitor,l={$\SI{500}{\micro\farad}$}]++(0,-\y) to [short] (0,0);
\end{tikzpicture}
\caption{مثال \حوالہ{مثال_بدلتا_سلسلہ_وار_مزاحمت_امالہ_برق_گیر_الف} کا دور۔}
\label{شکل_بدلتا_سلسلہ_وار_مزاحمت_امالہ_برق_گیر_الف}
\end{figure}

حل:مساوات \حوالہ{مساوات_بدلتا_پرزوں_کی_رکاوٹ_الف} سے انفرادی پرزوں کی رکاوٹ \عددی{\SI{50}{\hertz}} تعدد پر حاصل کرتے ہیں۔
\begin{align*}
Z_R&=\SI{10}{\ohm}\\
Z_L&=j 2 \pi \times 50 \times 2\times 10^{-3}=j0.6283 \, \si{\ohm}\\
Z_C&=\frac{1}{j 2\pi \times 50 \times 500 \times 10^{-6}}=-j6.3662\, \si{\ohm}
\end{align*}
چونکہ تمام پرزے سلسلہ وار جڑے ہیں لہٰذا ان کا مساوی رکاوٹ درج ذیل ہو گا۔
\begin{align*}
\bZ_s=10+j0.6283-j6.3662=10-j5.7379 \, \si{\ohm}
\end{align*}
دباو کو دوری سمتیہ صورت میں لکھتے ہوئے تعددی دائرہ کار میں رو حاصل کرتے ہیں۔
\begin{align*}
\hat{I}=\frac{\hat{V}}{\bZ_s}=\frac{30\phase{45^{\circ}}}{10-j5.7379}=\frac{30\phase{45^{\circ}}}{11.5292\phase{-29.85^{\circ}}}=2.6\phase{74.85^{\circ}} \, \si{\ampere}
\end{align*}
اس سے وقتی دائرہ کار میں رو لکھتے ہیں۔
\begin{align*}
i(t)=2.6\cos(100\pi t +74.85^{\circ})\, \si{\ampere}
\end{align*}
اب \عددی{\SI{2000}{\radian\per\second}} پر قیمتیں دریافت کرتے ہیں۔انفرادی رکاوٹ درج ذیل ہیں
\begin{align*}
Z_R&=\SI{10}{\ohm}\\
Z_L&=j 2000\times 2\times 10^{-3}=j4 \, \si{\ohm}\\
Z_C&=\frac{1}{j 2000\times 500 \times 10^{-6}}=-j1\, \si{\ohm}
\end{align*}
جن سے مساوی رکاوٹ درج ذیل حاصل ہوتا ہے۔
\begin{align*}
\bZ_s=10+j4-j1=10+j3=10.44\phase{16.7^{\circ}} \, \si{\ohm}
\end{align*}
یوں دوری رو درج ذیل ہو گی
\begin{align*}
\hat{I}=\frac{30\phase{45^{\circ}}}{10.44\phase{16.7^{\circ}}}=2.87\phase{28.3^{\circ}}
\end{align*}
جس سے وقتی دائرہ کار میں رو لکھتے ہیں۔
\begin{align*}
i(t)=2.87\cos(2000t+28.3^{\circ})\, \si{\ampere}
\end{align*}
آپ نے دیکھا کہ \عددی{\SI{50}{\hertz}} پر کُل رکاوٹ برق گیر کی خاصیت رکھتا ہے یعنی \عددی{\bZ=R-jX} لکھا جاتا ہے جبکہ \عددی{\SI{2000}{\radian\per\second}} پر \عددی{\bZ=R+jX} لکھا جاتا ہے جو امالی خاصیت کو ظاہر کرتا ہے۔
\انتہا{مثال}
%=================================
\ابتدا{مشق}\شناخت{مشق_بدلتا_متوازی_الف}
شکل \حوالہ{شکل_بدلتا_متوازی_الف} میں وقتی دائرہ کار میں رو حاصل کریں۔تعدد \عددی{\SI{1000}{\radian\per\second}} ہے۔

\begin{figure}
\centering
\begin{tikzpicture}
\draw(0,0) to [american voltage source,l={${55\phase{-66^{\circ}}}$}]++(0,\y) to [short,i={$i(t)$}]++(\x,0) to [short]++(2*\x,0) to [capacitor,l_={$\SI{50}{\micro\farad}$}]++(0,-\y) to [short]++(-3*\x,0);
\draw(\x,0) to [resistor,*-*,l={$\SI{80}{\ohm}$}]++(0,\y);
\draw(2*\x,0) to [inductor,*-*,l={$\SI{5}{\milli\henry}$}]++(0,\y);
\end{tikzpicture}
\caption{مشق \حوالہ{مشق_بدلتا_متوازی_الف} کا دور۔}
\label{شکل_بدلتا_متوازی_الف}
\end{figure} 

جواب:\عددی{8.28\cos(1000t-151.23^{\circ}) \, \si{\ampere}}
\انتہا{مشق}
%=======================

بدلتی رو ادوار میں برقی رکاوٹ \عددی{\bZ} کے علاوہ \اصطلاح{برقی فراوانی}\فرہنگ{فراوانی}\حاشیہب{admittance}\فرہنگ{admittance} \عددی{\bY} بھی نہایت اہم ثابت ہوتی ہے۔رکاوٹ کے بالعکس متناسب کو فراوانی کہتے ہیں۔
\begin{align}
\bY=\frac{1}{\bZ}
\end{align}
مخلوط رکاوٹ کی صورت میں فراوانی بھی مخلوط ہو گی۔فراوانی کو سیمنز \عددی{\si{\siemens}} میں ناپا جاتا ہے۔فراوانی کو مستطیل طرز درج ذیل لکھا جاتا ہے
\begin{align}
\bY=G+jB
\end{align}
جہاں \عددی{G} کو \اصطلاح{ایصالیت}\فرہنگ{ایصالیت}\حاشیہب{conductance}\فرہنگ{conductance} اور \عددی{B} کو \اصطلاح{تاثریت}\فرہنگ{تاثریت}\حاشیہب{susceptance}\فرہنگ{susceptance} کہتے ہیں۔

رکاوٹ سے فراوانی کے اجزاء درج ذیل مساوات سے شروع کرتے ہوئے
\begin{align}
G+jB=\frac{1}{R+jX}
\end{align}
حاصل کی جا سکتی ہے یعنی
\begin{gather}
\begin{aligned}
G&=\frac{R}{R^2+X^2}\\
B&=\frac{-X}{R^2+X^2}
\end{aligned}
\end{gather}
اسی طرح فراوانی کے اجزاء سے رکاوٹ کے اجزاء درج ذیل حاصل ہوتے ہیں۔
\begin{gather}
\begin{aligned}
R&=\frac{G}{G^2+B^2}\\
X&=\frac{-B}{G^2+B^2}
\end{aligned}
\end{gather}
آپ دیکھ سکتے ہیں کہ مخلوط رکاوٹ کی صورت میں \عددی{G} اور \عددی{R} آپس میں بالعکس متناسب نہیں ہیں۔اسی طرح \عددی{B} اور \عددی{X} بھی آپس میں بالعکس متناسب نہیں ہیں۔اگر رکاوٹ میں \عددی{X=0} ہو تب \عددی{G=\tfrac{1}{R}} ہو گا۔

انفرادی پرزوں کی فراوانی درج ذیل ہے۔
\begin{gather}
\begin{aligned}
Y_R&=\frac{1}{R}=G\\
Y_L&\frac{1}{j \omega L}=\frac{1}{\omega L} \phase{-90^{\circ}}\\
Y_C&=j \omega C=\omega C \phase{90^{\circ}}
\end{aligned}
\end{gather}
جہاں انفرادی مزاحمت کی صورت میں \عددی{\tfrac{1}{R}=G} لکھا گیا ہے۔

قوانین کرخوف فراوانی پر بھی لاگو ہوتے ہیں لہٰذا باب دوم کی طرز پر سلسلہ وار اور متوازی جڑے فراوانی کی مساوی فراوانی بالترتیب درج ذیل مساوات سے حاصل کی جا سکتی ہے۔
\begin{align}
\frac{1}{\bZ_s}&=\frac{1}{\bY_1}+\frac{1}{\bY_2}+\frac{1}{\bY_3}+\cdots+\frac{1}{\bY_n} \quad \text{\RL{سلسلہ وار جڑے}}\\
\bY_m &=\bY_1+\bY_2+\bY_3+\cdots+\bY_n\quad \text{\RL{متوازی جڑے}}
\end{align}
%================
\ابتدا{مثال}\شناخت{مثال_بدلتا_متوازی_ب}
شکل \حوالہ{شکل_بدلتا_متوازی_ب} میں منبع کے متوازی جڑے دور کی فراوانی \عددی{\SI{500}{\radian\per\second}} پر دریافت کرتے ہوئے رو \عددی{\hat{I}} حاصل کریں۔ 

\begin{figure}
\centering
\begin{tikzpicture}
\draw(0,0) to [american voltage source,l={${120\phase{56^{\circ}}}$}]++(0,2*\y) to [short,i={${\hat{I}}$}]++(\x,0) to [short]++(2*\x,0) to [capacitor,l_={$\SI{100}{\micro\farad}$}]++(0,-2*\y) to [short] (0,0);
\draw (\x,0)  to [resistor,*-*,l={$\SI{10}{\ohm}$}]++(0,2*\y);
\draw(2*\x,0) to [resistor,*-,l={$\SI{5}{\ohm}$}]++(0,\y) to [inductor,-*,l={$\SI{10}{\milli\henry}$}]++(0,\y);
\end{tikzpicture}
\caption{مثال \حوالہ{مثال_بدلتا_متوازی_ب} کا دور۔}
\label{شکل_بدلتا_متوازی_ب}
\end{figure}

حل:دور میں تین متوازی حصوں کے انفرادی رکاوٹ لکھتے  ہیں۔
\begin{align*}
Z_1&=\SI{10}{\ohm}\\
Z_2&=5+j 500\times 10\times 10^{-3}=5+j5 \, \si{\ohm}\\
Z_3&=\frac{1}{j 500 \times 100 \times 10^{-6}}=-j20 \, \si{\ohm}
\end{align*}
یوں تینوں حصوں کے فراوانی درج ذیل ہو گی۔
\begin{align*}
Y_1&=\frac{1}{10}=\SI{0.1}{\siemens}\\
Y_2&=\frac{1}{5+j5}=0.1-j0.1 \, \si{\siemens}\\
Y_3&=\frac{1}{-j20}=j0.05 \, \si{\siemens}
\end{align*}
یوں تینوں حصوں کو متوازی جوڑنے سے درج ذیل مساوی فراوانی حاصل ہو گی
\begin{align*}
\bY_m=(0.1)+(0.1-j0.1)+(j0.05)=0.2-j0.05 \, \si{\siemens}
\end{align*}
جسے استعمال کرتے ہوئے رو حاصل کرتے ہیں۔
\begin{align*}
\hat{I}&=\bY \hat{V}\\
&=(0.2-j0.05) (120 \phase{56^{\circ}})\\
&=24.74\phase{41.96^{\circ}} \, \si{\ampere}
\end{align*} 
\انتہا{مثال}
%=================
\ابتدا{مشق}\شناخت{مشق_بدلتا_متوازی_پ}
شکل \حوالہ{شکل_بدلتا_متوازی_پ} میں منبع کے متوازی دور کی فراوانی دریافت کرتے ہوئے \عددی{\hat{I}} حاصل کریں۔
\begin{figure}
\centering
\begin{tikzpicture}
\draw(0,0) to [american voltage source,l={${22\phase{10^{\circ}}}$}]++(0,\y) to [short,i={${\hat{I}}$}]++(\x,0) to [short]++(3*\x,0) to [inductor,l_={${j4 \, \si{\ohm}}$}]++(0,-\y) to [short] (0,0);
\draw (\x,0)  to [resistor,*-*,l={$\SI{4}{\ohm}$}]++(0,\y);
\draw(2*\x,0) to [inductor,*-*,l={${j2 \, \si{\ohm}}$}]++(0,\y);
\draw(3*\x,0) to [capacitor,*-*,l={${-j4 \, \si{\ohm}}$}]++(0,\y);
\end{tikzpicture}
\caption{مشق \حوالہ{مشق_بدلتا_متوازی_پ} کا دور۔}
\label{شکل_بدلتا_متوازی_پ}
\end{figure}

جواب:\عددی{12.298\phase{-53.4^{\circ}}\,\si{\ampere}}
\انتہا{مشق}
%============

آئیں مختلف انداز میں جڑے متعدد پرزوں کی مساوی رکاوٹ حاصل کرنا ایک مثال کی مدد سے سیکھیں۔مساوی رکاوٹ حاصل کرنے کا عمل بالکل ویسا ہی ہے جیسا مزاحمتی دور کا مساوی مزاحمت حاصل کرنے کا عمل۔مزاحمتی دور میں حقیقی اعداد استعمال ہوتے ہیں جبکہ رکاوٹی دور میں مخلوط اعداد استعمال ہوتے ہیں۔
%=============
\ابتدا{مثال}\شناخت{مثال_بدلتا_متعدد_رکاوٹ_مساوی_الف}
شکل \حوالہ{شکل_بدلتا_متعدد_رکاوٹ_مساوی_الف}-الف میں متعدد پرزے مختلف طرز پر جڑے دکھائے گئے ہیں۔دور کے دو سروں پر مساوی رکاوٹ \عددی{\bZ} دریافت کریں۔

\begin{figure}
\centering
\begin{subfigure}{1\textwidth}
\centering
\begin{circuitikz}
\draw(0,0) to [short,o-]++(\x/2,0)coordinate(kA) to [inductor,l={$j6\,\si{\ohm}$}]++(\x,0) to [resistor,l={$\SI{8}{\ohm}$}]++(\x,0)coordinate(kB);
\draw(kA) to [short,*-]++(0,\y) to [capacitor,l={$-j4 \,\si{\ohm}$}]++(\x,0) to [resistor,l={$\SI{2}{\ohm}$}]++(\x,0) to [short,-*] (kB);
\draw(kB) to [inductor,l={$j12\,\si{\ohm}$}]++(0,-\y) to [capacitor,l={$-j10\,\si{\ohm}$}]++(0,-\y) to [resistor,-*,l={$\SI{6}{\ohm}$}]++(0,-\y)coordinate(kD);
\draw(kB) to [resistor,l={$\SI{4}{\ohm}$}]++(\x,0) to [inductor,l={$j2 \, \si{\ohm}$}]++(\x,0) to [capacitor,l={$-j6\,\si{\ohm}$}]++(\x,0) to [short,-*]++(0,-\y)coordinate(kC);
\draw(0,-3*\y) to [short,o-] ++(5*\x+\x/2,0) to [short,-*]++(0,\y) to [short]++(\x/2,0) to [capacitor,l={$-j2 \,\si{\ohm}$}]++(0,\y) to [short]++(-\x,0) to [inductor,l_={$j10\,\si{\ohm}$}]++(0,-\y) to [short]++(\x/2,0);
\draw[stealth-](\x/4,-\y-\y/2) --++(-\x/4,0) --++(0,-\y/8)node[below]{$\bZ$};
\end{circuitikz}
\caption*{(الف)}
\end{subfigure}
\begin{subfigure}{0.5\textwidth}
\centering
\begin{tikzpicture}
\draw(0,0) to [short,o-]++(\x/2,0)coordinate(kA) to [european resistor,l={$\bZ_1$}]++(\x,0)coordinate(kB);
\draw(kA) to [short,*-]++(0,\y/2) to [european resistor,l=${\bZ_2}$]++(\x,0) to [short,-*] (kB);
\draw(kB) to [european resistor,l={$\bZ_3$}]++(\x,0) to [european resistor,l={$\bZ_5$}]++(0,-\y) to [short,-o]++(-2.5*\x,0);
\draw(kB) to [european resistor,-*,l={$\bZ_4$}]++(0,-\y);
\draw[stealth-](\x/4,-\y/2) --++(-\x/4,0) --++(0,-\y/8)node[below]{$\bZ$};
\end{tikzpicture}
\caption*{(ب)}
\end{subfigure}
\caption{مثال \حوالہ{مثال_بدلتا_متعدد_رکاوٹ_مساوی_الف} کا دور۔}
\label{شکل_بدلتا_متعدد_رکاوٹ_مساوی_الف}
\end{figure}

حل:شکل \حوالہ{شکل_بدلتا_متعدد_رکاوٹ_مساوی_الف}-ب میں دور کے مختلف حصوں کی نشاندہی کی گئی ہے جن کا مساوی رکاوٹ آسانی سے حاصل کیا جا سکتا ہے۔ان حصوں کی رکاوٹ دریافت کرتے ہیں۔
\begin{align*}
\bZ_1&=8+j6 \, \si{\ohm}\\
\bZ_2&=2-j4 \, \si{\ohm}\\
\bZ_3&=4+j2-j6=4-j4 \, \si{\ohm}\\
\bZ_4&=6-j10+j12=6+j2 \,\si{\ohm}
\end{align*}
اور 
\begin{align*}
\frac{1}{\bZ_5}&=\frac{1}{j10}+\frac{1}{-j2}\\
&=\frac{1}{j10}-\frac{1}{j2}\\
&=\frac{j2-j10}{(j10)(j2)}\\
&=\frac{-j8}{-20} \, \si{\siemens}
\end{align*}
سے درج ذیل ملتا ہے۔
\begin{align*}
\bZ_5&=\frac{20}{j8}=-j\frac{5}{2} \, \si{\ohm}
\end{align*}
رکاوٹ \عددی{\bZ_3} اور \عددی{\bZ_5} سلسلہ وار جڑے ہیں لہٰذا ان کا مساوی رکاوٹ درج ذیل ہو گا۔
\begin{align*}
\bZ_{35}=\bZ_3+\bZ_5=4-j4-j\frac{5}{2}=4-j7.5 \, \si{\ohm}
\end{align*}
اب \عددی{\bZ_4} اور \عددی{\bZ_{35}} متوازی ہیں لہٰذا ان رکاوٹ کی فراوانی دریافت کرتے ہیں۔یوں
\begin{align*}
\bY_4&=\frac{1}{\bZ_4}\\
&=\frac{1}{6+j2}\\
&=\left(\frac{1}{6+j2}\right)\left( \frac{6-j2}{6-j2}\right)\\
&=\frac{6-j2}{36+4}\\
&=0.15-j 0.05
\end{align*}
اور
\begin{align*}
\bY_{35}&=\frac{1}{\bZ_{35}}\\
&=\frac{1}{4-j7.5}\\
&=\frac{4+j7.5}{4^2+7.5^2}\\
&=0.05536+j0.10381
\end{align*}
حاصل کرتے ہوئے درج ذیل لکھا جا سکتا ہے
\begin{align*}
\bY_{435}&=\bY_4+\bY_{35}\\
&=0.15-j 0.05+0.05536+j0.10381\\
&=0.20536+j0.05381 \, \si{\siemens}
\end{align*}
جس سے 
\begin{align*}
\bZ_{435}&=\frac{1}{\bY_{435}}\\
&=\frac{1}{0.20536+j0.05381}\\
&=4.55665-j1.19397 \, \si{\ohm}
\end{align*}
حاصل ہوتا ہے جو متوازی جڑے \عددی{\bZ_4} اور \عددی{\bZ_{35}} کا مساوی رکاوٹ ہے۔رکاوٹ \عددی{\bZ_1} اور \عددی{\bZ_2} متوازی جڑے ہیں۔ان کا مساوی رکاوٹ درج ذیل ہو گا۔
\begin{align*}
\bZ_{12}&=\frac{\bZ_1 \bZ_2}{\bZ_1+\bZ_2}\\
&=\frac{(8+j6)(2-j4)}{(8+j6)+(2-j4)}\\
&=3.46154-j2.69231 \, \si{\ohm}
\end{align*}
یوں شکل \حوالہ{شکل_بدلتا_متعدد_رکاوٹ_مساوی_الف} میں دیے دور کا مساوی مزاحمت درج ذیل ہو گا۔
\begin{align*}
\bZ&=\bZ_{12}+\bZ_{435}\\
&=3.46154-j2.69231+4.55665-j1.19397\\
&=8.01819-j3.88628 \, \si{\ohm}
\end{align*}
\انتہا{مثال}
%=================
\ابتدا{مشق}\شناخت{مشق_بدلتا_متعدد_متوازی_رکاوٹ_مساوی_الف}
شکل \حوالہ{شکل_بدلتا_متعدد_متوازی_رکاوٹ_مساوی_الف} میں \عددی{\bZ} حاصل کریں۔
\begin{figure}
\centering
\begin{tikzpicture}
\draw(0,0) to [inductor,o-,l={$j4 \, \si{\ohm}$}] ++(\x,0) to [resistor,l={$\SI{2}{\ohm}$}]++(0,-\y) to [capacitor,l={$-j6 \, \si{\ohm}$}]++(0,-\y) to [capacitor,l={$-j2$},-o]++(-\x,0);
\draw(\x,0) to [short,*-]++(\x,0) to [resistor,l={$\SI{8}{\ohm}$}]++(0,-\y) to [inductor,l={$j16\, \si{\ohm}$}]++(0,-\y) to [short,-*]++(-\x,0);
\draw[stealth-](\x/4,-\y)--++(-\x/4,0)--++(0,-\y/8)node[below]{$\bZ$};
\end{tikzpicture}
\caption{مشق \حوالہ{مشق_بدلتا_متعدد_متوازی_رکاوٹ_مساوی_الف} کا دور۔}
\label{شکل_بدلتا_متعدد_متوازی_رکاوٹ_مساوی_الف}
\end{figure}

جواب:\عددی{\bZ=\tfrac{24}{5}-j\tfrac{22}{5} \, \si{\ohm}}
\انتہا{مشق}
%============================

\حصہ{دوری سمتیات کے اشکال}
رکاوٹ کی قیمت تعدد پر منحصر ہوتی ہے۔یوں دور میں رو اور دباو کا دارومدار بھی تعدد پر ہو گا۔دوری سمتی اشکال کی مدد سے رو اور دباو پر تعدد کے اثر پر غور کرنے میں مدد ملتی ہے۔آئیں اس پر چند مثال دیکھیں۔
%=======================
\ابتدا{مثال}\شناخت{مثال_بدلتا_دوری_سمتیات_الف}
شکل \حوالہ{شکل_بدلتا_دوری_سمتیات_الف} میں \عددی{\hat{I}}، \عددی{\hat{V}_R}، \عددی{\hat{V}_L} اور \عددی{\hat{V}_m} کے دوری سمتیہ مختلف تعدد پر  کھینچیں۔تعدد \عددی{\omega=\SI{1000}{\radian\per\second}} پر \عددی{\hat{I}=5\phase{0^{\circ}}\, \si{\ampere}} کی صورت میں \عددی{\hat{V}_m} حاصل کریں۔

\begin{figure}
\centering
\begin{subfigure}{0.5\textwidth}
\centering
\begin{tikzpicture}[american voltages]
\draw(0,0) to [american voltage source,l={$\hat{V}_m$}]++(0,\y) to [short,i={$\hat{I}$}]++(\x/4,0) to [resistor,l={$\SI{2}{\ohm}$},v={$\hat{V}_R$}]++(\x,0) to [short]++(\x/4,0) to [inductor,l={$\SI{1.5}{\milli\henry}$},v={$\hat{V}_L$}]++(0,-\y) to [short]++(-\x-\x/2,0);
\end{tikzpicture}
\caption*{(الف)}
\end{subfigure}%
\begin{subfigure}{0.5\textwidth}
\centering
\begin{tikzpicture}
\draw[gray](0,0)--++(3,0);
\draw[gray](0,0)--++(0,2);
\draw[-latex](0,0)--++(1,0)node[below]{$\hat{I}$};
\draw[-latex](0,0)--++(2,0)node[below]{$\hat{V}_R$};
\draw[-latex](0,0)--++(0,1.5)node[left]{$\hat{V}_L$};
\end{tikzpicture}
\caption*{(ب)}
\end{subfigure}
\begin{subfigure}{0.5\textwidth}
\centering
\begin{tikzpicture}
\draw[gray](0,0)--++(3,0);
\draw[gray](0,0)--++(0,2);
\draw[-latex](0,0)--++(1,0)node[pos=0.5,below]{$\hat{I}$};
\draw[-latex](0,0)--++(2,0)node[below]{$\hat{V}_R$};
\draw[-latex](2,0)--++(0,1.5)node[pos=0.5,right]{$\hat{V}_L$};
\draw[-latex](0,0)--++(2,1.5)node[pos=0.75,above left]{$\hat{V}_m$};
\draw[dashed](2,1.5)--++(0,0.5);
\end{tikzpicture}
\caption*{(پ)}
\end{subfigure}%
\begin{subfigure}{0.5\textwidth}
\centering
\begin{tikzpicture}
\draw[gray](0,0)--++(3,0);
\draw[gray](0,0)--++(0,2);
\draw[-stealth]([shift={(0:0.6)}]0,0) arc (0:20:0.6);
\draw(2/3*20:0.8)node[right]{$\phi$};
\begin{scope}[rotate=20]
\draw[-latex](0,0)--++(1,0)node[pos=0.7,above]{$\hat{I}$};
\draw[-latex](0,0)--++(2,0)node[below]{$\hat{V}_R$};
\draw[-latex](2,0)--++(0,1.5)node[pos=0.5,right]{$\hat{V}_L$};
\draw[-latex](0,0)--++(2,1.5)node[pos=0.75,above left]{$\hat{V}_m$};
\draw[dashed](2,1.5)--++(0,0.5);
\end{scope}
\end{tikzpicture}
\caption*{(ت)}
\end{subfigure}
\caption{مثال \حوالہ{مثال_بدلتا_دوری_سمتیات_الف} کے اشکال۔}
\label{شکل_بدلتا_دوری_سمتیات_الف}
\end{figure}


حل:دوری سمتیات کے خط کسی ایک دوری سمتیہ کے حوالے سے کھینچے جاتے ہیں۔ہم \عددی{\hat{I}} کو حوالہ سمتیہ تصور کرتے ہوئے آگے بڑھتے ہیں۔مزید، ہم اس دوری سمتیہ کو صفر زاویے پر تصور کرتے ہیں یعنی ہم \عددی{\hat{I}=I_0\phase{0^{\circ}}} تصور کرتے ہیں۔تعدد \عددی{\omega} پر مزاحمتی رکاوٹ \عددی{\bZ_R=R} جبکہ  امالی رکاوٹ \عددی{\bZ_L=\omega L \phase{90^{\circ}}} ہو گی لہٰذا ان پرزوں پر دباو درج ذیل ہو گا۔
\begin{align*}
\hat{V}_R&=\hat{I} \bZ_R=I_0 R \phase{0^{\circ}}\\
\hat{V}_L&=\hat{I} \bZ_L=I_0 \omega L \phase{90^{\circ}}
\end{align*}
یوں مزاحمت پر دباو عین رو کے ہم زاویہ ہے جبکہ امالہ پر دباو، رو سے \عددی{90^{\circ}} آگے ہے۔شکل \حوالہ{شکل_بدلتا_دوری_سمتیات_الف}-ب میں ان دوری سمتیات کو دکھایا گیا ہے۔چونکہ مزاحمتی رکاوٹ کی قیمت پر تعدد کا کوئی اثر نہیں لہٰذا \عددی{\hat{V}_R} کی قیمت اور زاویہ تعدد تبدیل کرنے سے تبدیل نہیں ہوتے۔اس کے برعکس امالی رکاوٹ تعدد کے راست متناسب ہے لہٰذا تعدد بڑھانے سے \عددی{\bZ_L} کی قیمت بڑھے گی اور یوں \عددی{\hat{V}_L} کا حیطہ بھی بڑھے گا جبکہ اس کا زاویہ جوں کا توں رہے گا۔آپ دیکھ سکتے ہیں کہ \عددی{\omega=\SI{0}{\radian\per\second}} پر \عددی{\hat{V}_L} کی قیمت صفر ہو گی جبکہ تعدد بڑھانے سے \عددی{\hat{V}_L} کی نوک خیالی محدد پر رہتے ہوئے  مرکز سے دور ہو گی۔

شکل \حوالہ{شکل_بدلتا_دوری_سمتیات_الف}-الف سے درج ذیل لکھا جا سکتا ہے۔
\begin{align*}
\hat{V}_m=\hat{V}_R+\hat{V}_L
\end{align*}
شکل \حوالہ{شکل_بدلتا_دوری_سمتیات_الف}-پ میں اس سمتی جمع کو دکھایا گیا ہے جہاں دم سے سر جوڑنے کا طریقہ استعمال کیا گیا ہے۔تعدد کو کم یا زیادہ کرنے  سے \عددی{\hat{V}_L} کا حیطہ کم اور زیادہ ہو گا لہٰذا شکل میں \عددی{\hat{V}_m} کی نوک نقطہ دار لکیر پر حرکت کرے گی۔صفر تعدد کی صورت میں \عددی{\hat{V}_m=\hat{V}_R} ہو گا جبکہ لامتناہی  تعدد پر \عددی{\hat{V}_m} کا زاویہ تقریباً \عددی{90^{\circ}} ہو گا۔
 
\عددی{\omega=\SI{1000}{\radian\per\second}} اور \عددی{\hat{I}=5\phase{0^{\circ}}\, \si{\ampere}} کی صورت میں درج ذیل لکھا جا سکتا ہے
\begin{align*}
\hat{V}_R&=\hat{I} \bZ_R=(5\phase{0^{\circ}})(2)=10\phase{0^{\circ}} \, \si{\volt}\\
\hat{V}_L&=\hat{I} \bZ_L=(5\phase{0^{\circ}})(1000 \times 1.5\times 10^{-3})=7.5\phase{90^{\circ}}\, \si{\volt}
\end{align*}
جس سے منبع کا دباو درج ذیل ملتا ہے۔
\begin{align*}
\hat{V}_m&=10\phase{0^{\circ}}+7.5\phase{90^{\circ}}\\
&=10+j7.5\\
&=12.5\phase{36.87^{\circ}}
\end{align*}
یہی جواب شکل \حوالہ{شکل_بدلتا_دوری_سمتیات_الف}-پ سے  بھی ترسیمی طریقے سے حاصل کیا جا سکتا ہے۔

یہاں بتلاتا چلوں کہ حوالہ دوری سمتیہ کا زاویہ صفر درجے رکھنا ضروری نہیں ہے۔ہم \عددی{\hat{I}=I_0\phase{\phi}} لے سکتے ہیں۔ایسی صورت میں تمام سمتیات اسی زاویے سے گھوم جائیں گے۔شکل \حوالہ{شکل_بدلتا_دوری_سمتیات_الف}-ت میں ایسا ہی دکھایا گیا ہے۔
\انتہا{مثال}
%===============

\ابتدا{مثال}\شناخت{مثال_بدلتا_دوری_سمتیات_ب}
شکل \حوالہ{شکل_بدلتا_دوری_سمتیات_ب} میں \عددی{\hat{I}}، \عددی{\hat{V}_R}، \عددی{\hat{V}_L} اور \عددی{\hat{V}_m} کے دوری سمتیہ مختلف تعدد پر  کھینچیں۔

\begin{figure}
\centering
\begin{subfigure}{0.5\textwidth}
\centering
\begin{tikzpicture}[american voltages]
\draw(0,0) to [american voltage source,l={$\hat{V}_m$}]++(0,\y) to [short,i={$\hat{I}$}]++(\x/4,0) to [resistor,l={$\SI{2}{\ohm}$},v={$\hat{V}_R$}]++(\x,0) to [short]++(\x/4,0) to [capacitor,l={$\SI{150}{\micro\farad}$},v={$\hat{V}_L$}]++(0,-\y) to [short]++(-\x-\x/2,0);
\end{tikzpicture}
\caption*{(الف)}
\end{subfigure}
\begin{subfigure}{0.5\textwidth}
\centering
\begin{tikzpicture}
\draw[gray](0,0)--++(3,0);
\draw[gray](0,0.5)--(0,-2);
\draw[-latex](0,0)--++(1,0)node[above]{$\hat{I}$};
\draw[-latex](0,0)--++(2,0)node[above]{$\hat{V}_R$};
\draw[-latex](0,0)--++(0,-1.5)node[left]{$\hat{V}_C$};
\end{tikzpicture}
\caption*{(ب)}
\end{subfigure}%
\begin{subfigure}{0.5\textwidth}
\centering
\begin{tikzpicture}
\draw[gray](0,0)--++(3,0);
\draw[gray](0,0.5)--(0,-2);
\draw[-latex](0,0)--++(1,0)node[pos=0.5,above]{$\hat{I}$};
\draw[-latex](0,0)--++(2,0)node[above]{$\hat{V}_R$};
\draw[-latex](2,0)--++(0,-1.5)node[pos=0.5,right]{$\hat{V}_C$};
\draw[-latex](0,0)--++(2,-1.5)node[pos=0.75,below left]{$\hat{V}_m$};
\draw[dashed](2,-1.5)--++(0,-0.5);
\end{tikzpicture}
\caption*{(پ)}
\end{subfigure}
\caption{مثال \حوالہ{مثال_بدلتا_دوری_سمتیات_ب} کے اشکال۔}
\label{شکل_بدلتا_دوری_سمتیات_ب}
\end{figure}


حل:رو کو حوالہ لیتے ہیں۔یوں \عددی{\hat{I}=I_0\phase{0^{\circ}}} ہو گا۔تعدد \عددی{\omega} پر مزاحمتی رکاوٹ \عددی{\bZ_R=R} جبکہ  برق گیر کی رکاوٹ \عددی{\bZ_C=\tfrac{1}{\omega C} \phase{-90^{\circ}}} ہو گی لہٰذا ان پرزوں پر دباو درج ذیل ہو گا۔
\begin{align*}
\hat{V}_R&=\hat{I} \bZ_R=I_0 R \phase{0^{\circ}}\\
\hat{V}_C&=\hat{I} \bZ_C=\frac{I_0}{ \omega C} \phase{-90^{\circ}}
\end{align*}
یوں مزاحمت پر دباو عین رو کے ہم زاویہ ہے جبکہ برق گیر پر دباو، رو سے \عددی{90^{\circ}} پیچھے ہے۔شکل \حوالہ{شکل_بدلتا_دوری_سمتیات_ب}-ب میں ان دوری سمتیات کو دکھایا گیا ہے۔چونکہ مزاحمتی رکاوٹ کی قیمت پر تعدد کا کوئی اثر نہیں لہٰذا \عددی{\hat{V}_R} کی قیمت اور زاویہ تعدد تبدیل کرنے سے تبدیل نہیں ہوتے۔اس کے برعکس برق گیر رکاوٹ تعدد کے بالعکس متناسب ہے لہٰذا تعدد بڑھانے سے \عددی{\bZ_C} کی قیمت کم ہو گی اور یوں \عددی{\hat{V}_C} کا حیطہ بھی کم ہو گا جبکہ اس کا زاویہ جوں کا توں رہے گا۔آپ دیکھ سکتے ہیں کہ لامتناہی تعدد پر \عددی{\hat{V}_C} کی قیمت صفر ہو گی جبکہ تعدد کم کرنے سے \عددی{\hat{V}_C} کی نوک خیالی محدد پر رہتے ہوئے  مرکز سے دور ہو گی۔

شکل \حوالہ{شکل_بدلتا_دوری_سمتیات_ب}-الف سے درج ذیل لکھا جا سکتا ہے۔
\begin{align*}
\hat{V}_m=\hat{V}_R+\hat{V}_C
\end{align*}
شکل \حوالہ{شکل_بدلتا_دوری_سمتیات_ب}-پ میں اس سمتی جمع کو دکھایا گیا ہے جہاں دم سے سر جوڑنے کا طریقہ استعمال کیا گیا ہے۔تعدد کو کم یا زیادہ کرنے  سے \عددی{\hat{V}_C} کا حیطہ زیادہ اور کم ہو گا لہٰذا شکل میں \عددی{\hat{V}_m} کی نوک نقطہ دار لکیر پر حرکت کرے گی۔لامتناہی تعدد کی صورت میں \عددی{\hat{V}_m=\hat{V}_R} ہو گا جبکہ صفر  تعدد پر \عددی{\hat{V}_m} کا زاویہ تقریباً \عددی{-90^{\circ}} ہو گا۔ 
\انتہا{مثال}
%===============
\ابتدا{مثال}\شناخت{مثال_بدلتا_دوری_سمتیات_پ}
شکل \حوالہ{شکل_بدلتا_دوری_سمتیات_پ}-الف  میں دکھائے دور کے  \عددی{\hat{I}}، \عددی{\hat{V}_R}، \عددی{\hat{V}_L}، \عددی{\hat{V}_C} اور \عددی{\hat{V}_m} دوری سمتیہ مختلف تعدد پر کھینچیں۔

\begin{figure}
\centering
\begin{subfigure}{0.5\textwidth}
\centering
\begin{tikzpicture}[american voltages]
\draw(0,0) to [american voltage source,l={$\hat{V}_m$}]++(0,\y) to [short,i={$\hat{I}$}]++(\x/2,0)to [resistor,l={$\SI{8}{\ohm}$},v_={$\hat{V}_R$}]++(\x,0) to [short]++(\x/2,0) to [inductor,l={$\SI{4}{\milli\henry}$},v_={$\hat{V}_L$}]++(0,-\y) to [short]++(-\x/2,0) to [capacitor,l={$\SI{100}{\micro\farad}$},v_={$\hat{V}_C$}]++(-\x,0) to [short]++(-\x/2,0);
\end{tikzpicture}
\caption*{(الف)}
\end{subfigure}%
\begin{subfigure}{0.5\textwidth}
\centering
\begin{tikzpicture}
\draw[gray](0,0)--++(3,0);
\draw[gray](0,-2)--++(0,4);
\draw[-latex](0,0)--++(1,0)node[below]{$\hat{I}$};
\draw[-latex](0,0)--(2,0)node[below]{$\hat{V}_R$};
\draw[-latex](0,0)--(0,1.5)node[left]{$\hat{V}_L$};
\draw[-latex](0,0)--(0,-1)node[left]{$\hat{V}_C$};
\end{tikzpicture}
\caption*{(ب)}
\end{subfigure}
\begin{subfigure}{0.5\textwidth}
\centering
\begin{tikzpicture}
\draw[gray](0,0)--++(3,0);
\draw[gray](0,-0.75)--++(0,2.5);
\draw[-latex](0,0)--++(1,0)node[pos=0.5,below]{$\hat{I}$};
\draw[-latex](0,0)--(2,0)node[pos=0.75,below]{$\hat{V}_R$};
\draw[-latex](2,0)--++(0,1)node[pos=0.5,right]{$\hat{V}_L+\hat{V}_C$};
\draw[-latex](0,0)--(2,1)node[pos=0.6,above left]{$\hat{V}_m$};
\draw[dashed](2,1)--++(0,0.5);
\draw[dashed](2,0)--++(0,-0.5);
\end{tikzpicture}
\caption*{(پ)}
\end{subfigure}
\caption{مثال \حوالہ{مثال_بدلتا_دوری_سمتیات_پ} کے اشکال۔}
\label{شکل_بدلتا_دوری_سمتیات_پ}
\end{figure}

حل:یہاں بھی رو کو حوالہ دوری سمتیہ \عددی{\hat{I}=I_0\phase{0^{\circ}}\,\si{\ampere}} تصور کرتے ہیں۔مزاحمت کا دباو اسی سمت میں ہو گا جبکہ امالہ کا دباو \عددی{90^{\circ}} آگے اور برق گیر کا دباو \عددی{90^{\circ}} پیچھے ہو گا۔شکل \حوالہ{شکل_بدلتا_دوری_سمتیات_پ}-ب میں انہیں دکھایا گیا ہے۔

جن تعدد پر \عددی{\omega L > \tfrac{1}{\omega C}} ہو، ان تعدد پر امالہ کا دباو، برق گیر کے دباو سے زیادہ ہو گا لہٰذا \عددی{\hat{V}_L+\hat{V}_C} کا زاویہ \عددی{90^{\circ}} ہو گا یعنی ان کا مجموعی تاثیر امالی ہو گا۔شکل \حوالہ{شکل_بدلتا_دوری_سمتیات_پ}-پ میں ایسی ہی تعدد پر درج ذیل سمتی مجموعہ دکھایا گیا ہے۔
\begin{align*}
\hat{V}_m=\hat{V}_R+\hat{V}_L+\hat{V}_C
\end{align*}
جس تعدد پر \عددی{\omega L=\tfrac{1}{\omega C}} ہو یعنی \عددی{(\omega_0=\tfrac{1}{\sqrt{LC}})} اس تعدد پر \عددی{\hat{V}_L+\hat{V}_C=0} ہو گا لہٰذا درج بالا مجموعے سے \عددی{\hat{V}_m=\hat{V}_R} حاصل ہو گا۔تعدد \عددی{\omega_0} سلسلہ وار جڑے \عددی{LC} کی \اصطلاح{قدرتی تعدد}\فرہنگ{قدرتی تعدد} یا اس کی \اصطلاح{گھمکی تعدد}\فرہنگ{گھمکی تعدد}\حاشیہب{resonant frequency}\فرہنگ{resonant frequency} کہلاتی ہے۔تعدد کم اور زیادہ کرنے سے \عددی{\hat{V}_m} کی نوک نقطہ دار لکیر پر حرکت کرتی ہے۔عین \عددی{\omega_0} پر \عددی{\hat{V}_m=\hat{V}_R} ہو گا۔گھمکی تعدد سے زیادہ تعدد پر \عددی{\hat{V}_m} کی نوک، نقطہ دار لکیر پر رہتے ہوئے، افقی محدد سے اوپر ہو گی جبکہ \عددی{\omega_0} سے کم تعدد پر \عددی{\hat{V}_m} کی نوک، نقطہ دار لکیر پر رہتے ہوئے، افقی محدد سے نیچے ہو گی۔ شکل \حوالہ{شکل_بدلتا_دوری_سمتیات_پ}-پ  گھمکی تعدد سے زیادہ تعدد کی صورت حال دکھا رہا ہے۔ 
\انتہا{مثال} 
%================
\ابتدا{مثال}\شناخت{مثال_بدلتا_رو_بالمقابل_دباو}
گزشتہ مثال میں \عددی{\hat{V}_m=120\phase{40^{\circ}} \, \si{\volt}} ہے۔شکل \حوالہ{شکل_بدلتا_دوری_سمتیات_پ}-الف میں پرزوں کی قیمتیں استعمال کرتے ہوئے \عددی{\omega=\SI{1000}{\radian\per\second}} پر  \عددی{\hat{V}_m} اور \عددی{\hat{I}} دوری سمتیوں کے خط کھینچیں۔

حل:اس مرتبہ ہم \عددی{\hat{V}_m} کو حوالہ لیتے ہیں۔دی گئی تعدد پر درج ذیل ہو گا۔
\begin{align*}
\bZ_R&=\SI{8}{\ohm}\\
\bZ_L&=1000\times 0.004 \phase{90^{\circ}}=4\phase{90^{\circ}}\, \si{\ohm}\\
\bZ_C&=\frac{1}{1000 \times 100\times 10^{-6}} \phase{-90^{\circ}}=10\phase{-90^{\circ}} \, \si{\ohm}
\end{align*}
شکل \حوالہ{شکل_بدلتا_دوری_سمتیات_پ}-الف کو دیکھ کر درج ذیل لکھا جا سکتا ہے
\begin{align*}
\hat{V}_m=\hat{V}_R+\hat{V}_L+\hat{V}_C
\end{align*}
جس میں قیمتیں پُر کرتے ہوئے
\begin{align*}
120 \phase{40^{\circ}}&=\hat{I} \bZ_R+\hat{I}\bZ_L+\hat{I} \bZ_C\\
&=\hat{I}\left(8+4\phase{90^{\circ}}+10\phase{-90^{\circ}}\right)\\
&=\hat{I}\left(8+j4-j10\right)\\
&=\hat{I}\left(8-j6\right)\\
&=\hat{I} \left(10\phase{-36.87^{\circ}}\right)
\end{align*}
ملتا ہے۔اس مساوات سے درج ذیل حاصل ہوتا ہے۔
\begin{align*}
\hat{I}&=\frac{120\phase{40^{\circ}}}{10\phase{-36.87^{\circ}}}\\
&=12\phase{76.87^{\circ}}
\end{align*}
\عددی{\hat{V}_m} اور \عددی{\hat{I}} کو شکل \حوالہ{شکل_بدلتا_رو_بالمقابل_دباو}-الف میں دکھایا گیا ہے جہاں حیطوں کو درست تناسب سے نہیں دکھایا گیا ہے۔

دور کی قدرتی تعدد درج ذیل ہے
\begin{align*}
\omega_0&=\frac{1}{\sqrt{0.004\times 100 \times 10^{-6}}}=\SI{1581}{\radian\per\second}
\end{align*}
جبکہ دور کو \عددی{\SI{1000}{\radian\per\second}} پر حل کیا گیا ہے۔یہی وجہ ہے کہ دور برق گیر تاثیر رکھتا ہے اور رو منبع کے دباو سے \عددی{37.87^{\circ}} درجے  آگے ہے۔عین قدرتی تعدد پر
\begin{align*}
\bZ_L&=j1581\times 0.004=6.32\phase{90^{\circ}} \, \si{\ohm}\\
\bZ_C&=\frac{1}{j1581\times 100\times 10^{-6}}=6.32\phase{-90^{\circ}}\,\si{\ohm}
\end{align*}
حاصل ہوتے ہیں۔

عموماً حوالہ دوری سمتیہ کا زاویہ \عددی{0^{\circ}} رکھا جاتا ہے۔یوں اگر ہم منبع دباو کا زاویہ \عددی{40^{\circ}} کی جگہ \عددی{0^{\circ}} چنتے تب \عددی{\hat{V}_m=120\phase{0^{\circ}}} لکھا جاتا اور رو درج ذیل حاصل ہوتی۔
\begin{align*}
\hat{I}=\frac{\hat{V}_m}{\bZ}=\frac{120\phase{0^{\circ}}}{10\phase{-36.87^{\circ}}}=12\phase{36.87^{\circ}}\,\si{\ampere}
\end{align*}
انہیں شکل  \حوالہ{شکل_بدلتا_رو_بالمقابل_دباو}-ب میں دکھایا گیا ہے۔آپ دیکھ سکتے ہیں کہ شکل-الف کو  گھڑی کے گردش کی سمت میں \عددی{40^{\circ}} گھمانے سے شکل-ب ملتا ہے۔یوں حوالہ سمتیہ کا زاویہ تبدیل کرنے سے تمام دوری سمتیہ کی شکل گھوم جاتی ہے، البتہ انفرادی دوری سمتیات کے تعلق پر کوئی فرق نہیں پڑتا۔یوں شکل-الف اور شکل-ب دونوں میں رو \عددی{36.87^{\circ}} درجے دباو سے آگے ہے۔ 
\begin{figure}
\centering
\begin{subfigure}{0.5\textwidth}
\centering
\begin{tikzpicture}
\draw[gray](0,0)--++(4,0);
\draw[gray](0,0)--++(0,2.5);
\draw[-latex](0,0)--++(40:2.4)node[right]{$\hat{V}_m$};
\draw[-latex](0,0)--++(76.87:1.2)node[above]{$\hat{I}$};
\draw[-stealth]([shift={(0:0.7)}]0,0) arc (0:40:0.7);
\draw(2/3*40:0.8)node[right]{$40^{\circ}$};
\draw[-stealth]([shift={(40:0.8)}]0,0) arc (40:76.87:0.8);
\draw(3/4*76.87:0.9)node[right,rotate=40]{$36.87^{\circ}$};
\end{tikzpicture}
\caption*{(الف)}
\end{subfigure}%
\begin{subfigure}{0.5\textwidth}
\centering
\begin{tikzpicture}
\draw[gray](0,0)--++(4,0);
\draw[gray](0,0)--++(0,2.5);
\draw[-latex](0,0)--++(0:2.4)node[below]{$\hat{V}_m$};
\draw[-latex](0,0)--++(36.87:1.2)node[above]{$\hat{I}$};
\draw[-stealth]([shift={(0:0.7)}]0,0) arc (0:36.87:0.7);
\draw(2/3*40:0.8)node[right]{$36.87^{\circ}$};
\end{tikzpicture}
\caption*{(ب)}
\end{subfigure}%
\caption{مثال \حوالہ{مثال_بدلتا_رو_بالمقابل_دباو} کے دوری سمتیوں کے خط۔}
\label{شکل_بدلتا_رو_بالمقابل_دباو}
\end{figure}
\انتہا{مثال}
%===================
\ابتدا{مثال}\شناخت{مثال_بدلتا_متوازی_تینوں_پرزے_الف}
شکل \حوالہ{شکل_بدلتا_متوازی_تینوں_پرزے_الف} کے دور میں دیے تمام دوری سمتیات کے خط کھینچیں۔
\begin{figure}
\centering
\begin{subfigure}{0.7\textwidth}
\centering
\begin{tikzpicture}[american voltages]
\draw(0,0) to [american voltage source,l={$\hat{V}_m$}]++(0,\y) to [short,i={$\hat{I}_m$}]++(\x,0) to [short]++(2*\x,0)  to [capacitor,i>_={$\hat{I}_C$},l_={$C$},v^={$\hat{V}_C$}]++(0,-\y) to [short] (0,0);
\draw(\x,0) to [resistor,*-*,i<={$\hat{I}_R$},l={$R$},v={$\hat{V}_R$}]++(0,\y);
\draw(2*\x,0) to [inductor,*-*,i<={$\hat{I}_L$},l={$L$},v={$\hat{V}_L$}]++(0,\y);
\end{tikzpicture}
\caption*{(الف)}
\end{subfigure}%
\begin{subfigure}{0.3\textwidth}
\centering
\begin{tikzpicture}[american voltages]
\draw[gray](0,0)--++(3,0);
\draw[gray](0,-1.5)--(0,2);
\draw[-latex](0,0)--++(0:2.7)node[below]{$\hat{V}_m$};
\draw[-latex](0,0)--++(0:1.75)node[below]{$\hat{I}_R$};
\draw[-latex](0,0)--++(0,1.5)node[left]{$\hat{I}_C$};
\draw[-latex](0,0)--++(0,-1)node[left]{$\hat{I}_L$};
\end{tikzpicture}
\caption*{(ب)}
\end{subfigure}
\begin{subfigure}{0.5\textwidth}
\centering
\begin{tikzpicture}[american voltages]
\draw[gray](0,0)--++(3,0);
\draw[gray](0,-1.5)--(0,2);
\draw[-latex](0,0)--++(0:2.7)node[below]{$\hat{V}_m$};
\draw[-latex](0,0)--++(0:1.75)node[below]{$\hat{I}_R$};
\draw[-latex,gray](0,0)--++(0,1.5)node[left]{$\hat{I}_C$};
\draw[-latex,gray](0,0)--++(0,-1)node[left]{$\hat{I}_L$};
\draw[-latex](0,0)--++(0,0.5)node[left]{$\hat{I}_L+\hat{I}_C$};
\end{tikzpicture}
\caption*{(پ)}
\end{subfigure}%
\begin{subfigure}{0.5\textwidth}
\centering
\begin{tikzpicture}[american voltages]
\draw[gray](0,0)--++(3,0);
\draw[gray](0,-1.5)--(0,2);
\draw[-latex](0,0)--++(0:2.7)node[below]{$\hat{V}_m$};
\draw[-latex](0,0)--++(0:1.75)node[pos=0.8,below]{$\hat{I}_R$};
\draw[-latex](1.75,0)--++(0,0.5)node[right]{$\hat{I}_L+\hat{I}_C$};
\draw[-latex](0,0)--++(1.75,0.5)node[pos=0.7,above]{$\hat{I}_m$};
\draw[gray,dashed](1.75,0.5)--++(0,1.5);
\draw[gray,dashed](1.75,0)--++(0,-1.5);
\draw[-latex,gray](0,0)--++(1.75,1.5)node[pos=0.7,above]{$\hat{I}'_m$};
\draw[-latex,gray](0,0)--++(1.75,-1)node[pos=0.7,below]{$\hat{I}''_m$};
\end{tikzpicture}
\caption*{(ت)}
\end{subfigure}
\caption{مثال \حوالہ{مثال_بدلتا_متوازی_تینوں_پرزے_الف} کے اشکال۔}
\label{شکل_بدلتا_متوازی_تینوں_پرزے_الف}
\end{figure}

حل:دباو \عددی{\hat{V}_m} کو حوالہ دوری سمتیہ لیتے ہوئے اس کا زاویہ صفر درجے چنتے ہیں۔ تینوں پرزوں پر \عددی{\hat{V}_m}  دباو پایا جاتا ہے لہٰذا ان کی انفرادی رو درج ذیل ہوں گے۔
\begin{align*}
\hat{I}_R&=\frac{\hat{V}_m}{\bZ_R}=\frac{V_m\phase{0^{\circ}}}{R}=\frac{V_m}{R}\phase{0^{\circ}}\\
\hat{I}_L&=\frac{\hat{V}_m}{\bZ_L}=\frac{V_m\phase{0^{\circ}}}{\omega L \phase{90^{\circ}}}=\frac{V_m}{\omega L} \phase{-90^{\circ}}\\
\hat{I}_C&=\frac{\hat{V}_m}{\bZ_C}=\frac{V_m \phase{0^{\circ}}}{\frac{1}{\omega C} \phase{-90^{\circ}}}=\omega C V_m \phase{90^{\circ}}
\end{align*}
انہیں شکل \حوالہ{شکل_بدلتا_متوازی_تینوں_پرزے_الف}-ب میں دکھایا گیا ہے۔

قدرتی تعدد \عددی{\omega_0=\tfrac{1}{\sqrt{LC}}} پر امالی رکاوٹ اور برق گیری رکاوٹ کے  مقدار برابر \عددی{(\omega L=\tfrac{1}{\omega C})} ہوتے ہیں۔قدرتی تعدد پر \عددی{\hat{I}_m=\hat{I}_R} ہو گا۔قدرتی تعدد سے زیادہ تعدد پر \عددی{\bZ_L>\bZ_C} لہٰذا \عددی{I_C>I_L} ہو گا۔اس صورت حال کو شکل-پ میں دکھایا گیا ہے۔

کرخوف قانون رو سے درج ذیل لکھا جا سکتا ہے
\begin{align*}
\hat{I}_m=\hat{I}_R+\hat{I}_L+\hat{I}_C
\end{align*}
جسے \عددی{\omega>\omega_0}  کی صورت میں  شکل-ت میں دکھایا گیا ہے۔تعدد مزید بڑھانے سے \عددی{\hat{I}_m} کی نوک نقطہ دار لکیر پر رہتے ہوئے افقی محدد سے مزید دور ہو گی۔دوری سمتیہ \عددی{\hat{I}'_m} ایسی صورت کو ظاہر کرتی ہے۔

قدرتی تعدد سے کم تعدد \عددی{(\omega<\omega_0)} پر \عددی{\bZ_C>\bZ_L} اور \عددی{I_L>I_C} ہو گا لہٰذا دوری سمتیہ کی نوک نقطہ دار لکیر پر افقی محدد سے نیچے کی طرف ہو گی۔دوری سمتیہ \عددی{\hat{I}''_m} ایسی صورت کو ظاہر کرتی ہے۔
\انتہا{مثال}
%===================
\ابتدا{مشق}\شناخت{مشق_بدلتا_دو_پرزے_متوازی_الف}
شکل \حوالہ{شکل_بدلتا_دو_پرزے_متوازی_الف}-الف میں تمام رو اور دباو کے دوری سمتیات کے خط کھینچیں۔تقسیم رو کا کلیہ استعمال کیا جا سکتا ہے۔ جواب کو شکل-ب میں دکھایا گیا ہے۔
\begin{figure}
\centering
\begin{subfigure}{1\textwidth}
\centering
\begin{tikzpicture}[american voltages]
\draw(0,0) to [american current source,l={${\hat{I}_m=10\phase{60^{\circ}}\,\si{\ampere}}$}]++(0,\y) to [short]++(2*\x,0) to [inductor,l_={$j10 \, \si{\ohm}$},i>={$\hat{I}_L$}]++(0,-\y) to [short] (0,0);
\draw(\x,0) to [resistor,*-*,i<_={$\hat{I}_R$},l={$\SI{20}{\ohm}$}]++(0,\y);
\draw(2*\x,\y) to [short,*-o]++(\x/2,0)coordinate(kT);
\draw(2*\x,0) to [short,*-o]++(\x/2,0)coordinate(kB);
\draw($(kT)!0.5!(kB)$)node{$\begin{aligned}&+ \\& \hat{V}_0 \\ &- \end{aligned}$};
\end{tikzpicture}
\caption*{(الف)}
\end{subfigure}
\begin{subfigure}{1\textwidth}
\centering
\begin{tikzpicture}[american voltages]
\draw[gray](0,0)--++(4,0);
\draw[gray](0,0)--(0,2.5);
\draw[-latex](0,0)--++(60:2)node[right]{${\hat{I}_m=\SI{10}{\ampere}}$};
\draw[-latex](0,0)--++(123.43:0.894)node[left]{${\hat{I}_R=2\sqrt{5} \, \si{\ampere}}$};
\draw[-latex](0,0)--++(33.43:1.788)node[right]{${\hat{I}_L=4\sqrt{5} \, \si{\ampere}}$};
\draw[-latex](0,0)--++(123.43:3)node[left]{${\hat{V}_0=8\sqrt{125}\,\si{\volt}}$};
\draw[-stealth]([shift={(0:1)}]0,0) arc (0:33.43:1);
\draw(2/3*33.43:1.1)node[right]{$33.43^{\circ}$};
\draw[-stealth]([shift={(0:0.8)}]0,0) arc (0:60:0.8);
\draw(3/4*60:0.9)node[right,rotate=60]{$60^{\circ}$};
\draw[-stealth]([shift={(0:0.6)}]0,0) arc (0:123.43:0.6);
\draw(110:1.2)node[rotate=123.43]{$123.43^{\circ}$};
\end{tikzpicture}
\caption*{(ب)}
\end{subfigure}
\caption{مشق \حوالہ{مشق_بدلتا_دو_پرزے_متوازی_الف} کے اشکال۔}
\label{شکل_بدلتا_دو_پرزے_متوازی_الف}
\end{figure}

\انتہا{مشق}
%===============
\ابتدا{مشق}\شناخت{مشق_بدلتا_دو_پرزے_متوازی_ب}
شکل \حوالہ{شکل_بدلتا_دو_پرزے_متوازی_ب} میں برق گیر کی وہ قیمت دریافت کریں جس پر \عددی{v(t)=120\cos(5500t-30^{\circ}) \, \si{\volt}} اور \عددی{i(t)} ہم زاویہ ہوں گے۔اس تعدد پر \عددی{i(t)} دریافت کریں۔
\begin{figure}
\centering
\begin{tikzpicture}
\draw(0,0) to [american voltage source,l={$v(t)$}]++(0,\y) to [resistor,i_={$i(t)$},l={$\SI{100}{\ohm}$}]++(\x,0) to [inductor,l={$\SI{120}{\micro\henry}$}]++(\x,0) to [capacitor,l={$C$}]++(0,-\y) to [short] (0,0);
\end{tikzpicture}
\caption{مشق \حوالہ{مشق_بدلتا_دو_پرزے_متوازی_ب} کا دور۔}
\label{شکل_بدلتا_دو_پرزے_متوازی_ب}
\end{figure}

جواب:\عددی{C=\SI{275.48}{\micro\farad}}، \عددی{i(t)=1.2\cos(5500t-30^{\circ})\, \si{\ampere}}
\انتہا{مشق}
%===============

\حصہ{کرخوف مساوات}
یک سمتی رو ادوار کو کرخوف کے قوانین سے حل کرنا ہم گزشتہ بابوں میں دیکھ چکے ہیں۔قوانین کرخوف دوری سمتیات پر بھی لاگو ہوتے ہیں۔یوں بدلتی رو ادوار کو کرخوف مساوات سے بالکل یک سمتی رو ادوار کی طرح حل کیا جا سکتا ہے۔یک سمتی رو ادوار حل کرنے کے تمام ترکیب یعنی مسئلہ تھونن، مسئلہ نارٹن، مسئلہ تبادلہ منبع، مسئلہ نفاذ، تقسیم رو اور تقسیم دباو کو بدلتی رو ادوار حل کرنے کے لئے بھی استعمال کیا جاتا ہے۔بدلتی رو ادوار کی صورت میں مخلوط الجبرا کا استعمال کیا جاتا ہے۔گزشتہ حصے میں ہم نے چند سادہ مثال اسی طرح حل کئے۔آئیں نسبتاً مشکل ادوار حل کریں۔متعدد منبع کی صورت میں دوری سمتیات کی ترکیب صرف اس صورت میں قابل استعمال ہو گی جب تمام منبع کی تعدد یکساں ہو البتہ ان کے انفرادی زاویے مختلف ہو سکتے ہیں۔

%==================
\ابتدا{مثال}\شناخت{مثال_بدلتا-متعدد_پرزے_کرخوف_الف}
شکل \حوالہ{شکل_بدلتا-متعدد_پرزے_کرخوف_الف} میں تمام نا معلوم دباو اور رو دریافت کریں۔
\begin{figure}
\centering
\begin{subfigure}{0.5\textwidth}
\centering
\begin{tikzpicture}[american voltages]
\draw(0,0) to [american voltage source,l={$20\phase{30^{\circ}}\,\si{\volt}$}]++(0,\y) to [inductor,i>^={$\hat{I}_1$},l={$j10\,\si{\ohm}$}]++(\x,0) to [resistor,i={$\hat{I}_3$},l={$\SI{1}{\ohm}$}]++(\x,0) to [capacitor,v={$\hat{V}_2$},l={$-j8\,\si{\ohm}$}]++(0,-\y) to [short] (0,0);
\draw(\x,\y) to [resistor,v={$\hat{V}_1$},i={$\hat{I}_2$},l={$\SI{6}{\ohm}$},*-*]++(0,-\y);
\end{tikzpicture}
\caption*{(الف)}
\end{subfigure}%
\begin{subfigure}{0.5\textwidth}
\centering
\begin{tikzpicture}[american voltages]
\draw[gray](0,0)--(3,0);
\draw[gray](0,0)--(0,1);
\draw[-latex](0,0)--++(-33.163:2.395)node[right]{$\hat{I}_1$}coordinate(kA);
\draw[-latex](0,0)--++(-67.224:1.816)node[left]{$\hat{I}_2$}coordinate(kB);
\draw[-latex](0,0)--++(15.65:1.352)node[above]{$\hat{I}_3$}coordinate(kC);
\draw[-stealth]([shift={(0:0.8)}]0,0) arc (0:15.65:0.8);
%\draw(20:0.45)node[above,rotate=15]{$16^{\circ}$};
\draw[-stealth]([shift={(0:0.7)}]0,0) arc (0:-33.163:0.7);
\draw(-33.163*2/3:0.8)node[right]{$33^{\circ}$};
\draw[-stealth]([shift={(0:0.6)}]0,0) arc (0:-67.224:0.6);
\draw(-67.224*2/3:0.7)node[right,rotate={-50}]{$67^{\circ}$};
\draw[gray,dashed](kC)--(kA)--(kB);
\end{tikzpicture}
\caption*{(ب)}
\end{subfigure}%
\caption{مثال \حوالہ{مثال_بدلتا-متعدد_پرزے_کرخوف_الف} کے اشکال۔}
\label{شکل_بدلتا-متعدد_پرزے_کرخوف_الف}
\end{figure}

حل:ہم پہلے منبع سے جڑی کل رکاوٹ حاصل کرتے ہوئے \عددی{\hat{I}_1} دریافت کرتے ہیں جسے جانتے ہوئے \عددی{j2\,\si{\ohm}} امالہ کا دباو حاصل کیا جا سکتا ہے۔اس دباو کو \عددی{\hat{V}_m} سے منفی کرتے ہوئے \عددی{\hat{V}_1} حاصل کیا جائے گا۔اب \عددی{\hat{V}_1} جانتے ہوئے  \عددی{\hat{I}_2} حاصل کیا جا سکتا ہے جسے استعمال کرتے ہوئے \عددی{\hat{I}_3=\hat{I}_1-\hat{I}_2} لکھا جا سکتا ہے۔آخر میں \عددی{\hat{V}_2=\hat{I}_3 (-j8)} لکھتے ہوئے برق گیر کا دباو حاصل کیا جائے  گا۔

منبع کو درج ذیل رکاوٹ نظر آتی ہے۔
\begin{align*}
\bZ&=j10+\frac{6(1-j8)}{6+1-j8}\\
&=j10 +\frac{6-j48}{7-j8}\\
&=j10 +\left(\frac{6-j48}{7-j8}\right)\left(\frac{7+j8}{7+j8}\right)\\
&=j10+\frac{426-j288}{113}\\
&=3.7699+j7.4513\\
&=8.3507\phase{63.163^{\circ}}\,\si{\ohm}
\end{align*}
یوں درج ذیل لکھا جا سکتا ہے
\begin{align*}
\hat{I}_1&=\frac{\hat{V}_m}{\bZ}\\
&=\frac{20\phase{30^{\circ}}}{8.3507\phase{63.163^{\circ}}}\\
&=2.395\phase{-33.163^{\circ}}\, \si{\ampere}
\end{align*}
جس سے \عددی{\hat{V}_1} حاصل کرتے ہیں۔
\begin{align*}
\hat{V}_1&=\hat{V}_m-\hat{I}_1 (j2)\\
&=20\phase{30^{\circ}}-(2.395\phase{-33.163^{\circ}})( 2\phase{90^{\circ}})\\
&=4.219-j10.049\\
&=10.8987\phase{-67.224^{\circ}}\,\si{\volt}
\end{align*}
آپ \عددی{\hat{V}_1} کو یوں بھی حاصل کر سکتے ہیں۔
\begin{align*}
\hat{V}_1&=\frac{6(1-j8)}{6+1-j8} \hat{I}_1\\
&=10.8987\phase{-67.224^{\circ}}\,\si{\volt}
\end{align*}
اس کے علاوہ \عددی{\hat{V}_1} کو تقسیم دباو کے کلیے سے بھی حاصل کیا جا سکتا ہے یعنی
\begin{align*}
\hat{V}_1&=\left(\frac{\frac{6(1-j8)}{6+1-j8}}{j10+\frac{6(1-j8)}{6+1-j8}}\right) 20\phase{30^{\circ}}\\
&=10.8987\phase{-67.224^{\circ}}\,\si{\volt}
\end{align*}
دباو \عددی{\hat{V}_1} جانتے ہوئے \عددی{\hat{I}_2} حاصل کرتے ہیں۔
\begin{align*}
\hat{I}_2&=\frac{\hat{V}_1}{6}\\
&=\frac{10.8987\phase{-67.224^{\circ}}}{6}\\
&=1.816\phase{-67.224^{\circ}}\,\si{\ampere}
\end{align*}
یوں \عددی{\hat{I}_3} درج ذیل ہو گا۔
\begin{align*}
\hat{I}_3 &=\hat{I}_1-\hat{I}_2\\
&=2.395\phase{-33.163^{\circ}}-1.816\phase{-67.224^{\circ}}\\
&=(2.005-j1.310)-(0.703-j1.675)\\
&=1.302+j0.365\\
&=1.352\phase{15.65^{\circ}}\,\si{\ampere}
\end{align*}
آپ \عددی{\hat{I}_3} کو درج ذیل سے بھی حاصل کر سکتے ہیں۔
\begin{align*}
\hat{I}_3&=\frac{\hat{V}_1}{1-j8}\\
&=1.352\phase{15.65^{\circ}}\,\si{\ampere}
\end{align*}
برق گیر کا دباو حاصل کرتے ہیں۔
\begin{align*}
\hat{V}_2&=\hat{I}_3 (-j8)\\
&=(1.352\phase{15.65^{\circ}})(8\phase{-90^{\circ}})\\
&=10.816\phase{-74.35^{\circ}}\,\si{\volt}
\end{align*}
اس دباو کو تقسیم دباو کے کلیے سے بھی حاصل کیا جا سکتا ہے یعنی
\begin{align*}
\hat{V}_2&=\left(\frac{-j8}{1-j8}\right) \hat{V}_1\\
&=10.816\phase{-74.35^{\circ}}\,\si{\volt}
\end{align*}
اس کے علاوہ \عددی{\SI{1}{\ohm}}  مزاحمت میں \عددی{\hat{I}_3} گزرتی ہے۔یوں \عددی{\hat{V}_1} سے اس مزاحمت کی دباو منفی کرنے سے بھی برق گیر کا دباو حاصل کیا جا سکتا ہے یعنی
\begin{align*}
\hat{V}_2&=\hat{V}_1-(\hat{I}_3)(1)\\
&=10.816\phase{-74.35^{\circ}}\,\si{\volt}
\end{align*}
آپ نے دیکھا کہ آپ اپنے مرضی کی کوئی بھی ترکیب استعمال کرتے ہوئے جوابات حاصل کر سکتے ہیں۔شکل-ب میں دوری رو دکھائے گئے ہیں جہاں نقطہ دار لکیر قانون متوازی الاضلاع سے \عددی{\hat{I}_1=\hat{I}_2+\hat{I}_3} دکھاتی ہے۔
\انتہا{مثال}
%=========================
\ابتدا{مشق}\شناخت{مشق_بدلتا_کرخوف_بدلتی_رو_الف}
شکل \حوالہ{شکل_بدلتا_کرخوف_بدلتی_رو_الف} میں
\begin{align*}
v_1(t)&=10\cos(300t+30^{\circ}) \, \si{\volt}\\
v_2(t)&=30\cos(300t+60^{\circ})\,\si{\volt}
\end{align*}
ہیں۔ دوری سمتیات استعمال کرتے ہوئے \عددی{i(t)} حاصل کریں۔
\begin{figure}
\centering
\begin{tikzpicture}
\draw(0,0) to [american voltage source,l={$v_1(t)$}]++(0,\y) to [resistor,i={$i(t)$},l={$\SI{4}{\ohm}$}]++(\x,0) to [capacitor,l={$\SI{200}{\micro\farad}$}]++(\x,0) ++(\x,0) to [american voltage source,l_={$v_2(t)$}]++(-\x,0) ++(\x,0) to [inductor,l={$\SI{050}{\milli\henry}$}]++(\x,0) to [resistor,l={$\SI{2}{\ohm}$}]++(0,-\y) to [short] (0,0);
\end{tikzpicture}
\caption{مشق \حوالہ{مشق_بدلتا_کرخوف_بدلتی_رو_الف} کا دور۔}
\label{شکل_بدلتا_کرخوف_بدلتی_رو_الف}
\end{figure}

جواب:\عددی{i(t)=3.52\cos(300t-91.3^{\circ})\,\si{\ampere}}
\انتہا{مشق}
%============================
 \ابتدا{مشق}\شناخت{مشق_بدلتا_کرخوف_بدلتی_رو_ب}
شکل \حوالہ{شکل_بدلتا_کرخوف_بدلتی_رو_ب} میں \عددی{\hat{V}_0} دریافت کریں۔

\begin{figure}
\centering
\begin{tikzpicture}
\draw(0,0) to [american current source,l={$16\phase{0^{\circ}}\, \si{\ampere}$}]++(0,\y) to [resistor,l={$\SI{2}{\ohm}$}]++(\x,0) to [inductor,l={$j6\,\si{\ohm}$}]++(\x,0) to [capacitor,l_={$-j10\,\si{\ohm}$}]++(0,-\y) to [short] (0,0);
\draw(\x,\y) to [inductor,*-*,l_={$j8\,\si{\ohm}$}]++(0,-\y);
\draw(2*\x,\y) to [short,*-o]++(\x/2,0)coordinate(kT);
\draw(2*\x,0) to [short,*-o]++(\x/2,0)coordinate(kB);
\draw($(kT)!0.5!(kB)$) node {$\begin{aligned}&+ \\ & \hat{V}_0 \\ &-  \end{aligned}$};
\end{tikzpicture}
\caption{مشق \حوالہ{مشق_بدلتا_کرخوف_بدلتی_رو_ب} کا دور۔}
\label{شکل_بدلتا_کرخوف_بدلتی_رو_ب}
\end{figure}

جواب:\عددی{\hat{V}_0=320\phase{-90^{\circ}}\,\si{\volt}}
\انتہا{مشق}
%==============

 \ابتدا{مشق}\شناخت{مشق_بدلتا_کرخوف_بدلتی_رو_پ}
شکل \حوالہ{شکل_بدلتا_کرخوف_بدلتی_رو_پ} میں \عددی{\hat{V}_1} دریافت کریں۔

\begin{figure}
\centering
\begin{tikzpicture}
\draw(0,0) to [american current source,l_={$10\phase{30^{\circ}}\,\si{\ampere}$}]++(0,-2*\y) to [short]++(3*\x,0) to [american voltage source,l_={$40\phase{60^{\circ}}\,\si{\volt}$}]++(0,2*\y) to [resistor,l={$\SI{4}{\ohm}$}]++(-\x,0) to [capacitor,l={$-j20\,\si{\ohm}$}]++(-\x,0) to [short] (0,0);
\draw(\x,-2*\y)node[ground]{} to [resistor,*-,l={$\SI{5}{\ohm}$}]++(0,\y) to [inductor,-*,l={$j10\,\si{\ohm}$}]++(0,\y)node[above]{$\hat{V}_1$};
\end{tikzpicture}
\caption{مشق \حوالہ{مشق_بدلتا_کرخوف_بدلتی_رو_پ} کا دور۔}
\label{شکل_بدلتا_کرخوف_بدلتی_رو_پ}
\end{figure}

جواب:\عددی{\hat{V}_1=182.88\phase{-127.16^{\circ}}\,\si{\volt}}
\انتہا{مشق}
%==============

\حصہ{تجزیاتی تراکیب}
اس حصے میں ہم وہ تمام ترکیب استعمال کریں گے جن سے یک سمتی ادوار حل کیے گئے۔ایسا مثالوں کی مدد سے کیا جائے گا۔

%===================
\ابتدا{مثال}\شناخت{مثال_بدلتا_تراکیب_الف}
شکل \حوالہ{شکل_بدلتا_تراکیب_الف} کو ترکیب جوڑ سے حل کرتے ہوئے \عددی{\hat{V}_1} اور \عددی{\hat{V}_2} دریافت کریں۔
\begin{figure}
\centering
\begin{tikzpicture}
\draw(0,0) to [american voltage source,l={$20\phase{70^{\circ}}\,\si{\volt}$}]++(0,\y) to [inductor,l={$j2\,\si{\ohm}$}]++(\x,0)node[above]{$\hat{V}_1$} to [resistor,l={$\SI{10}{\ohm}$}]++(0,-\y)node[ground]{} to [short] (0,0);
\draw(\x,0) to [short,*-]++(2*\x,0) to [resistor,l_={$\SI{2}{\ohm}$}]++(0,\y) to [capacitor,l_={$-j4\,\si{\ohm}$}]++(-\x,0) to [american current source,-*,l_={$2\phase{30^{\circ}}\,\si{\ampere}$}]++(-\x,0);
\draw(2*\x,0) to [inductor,*-*,l_={$j6\,\si{\ohm}$}]++(0,\y)node[above]{$\hat{V}_2$};
\end{tikzpicture}
\caption{مثال \حوالہ{مثال_بدلتا_تراکیب_الف} کا دور۔}
\label{شکل_بدلتا_تراکیب_الف}
\end{figure}

حل:جوڑ \عددی{\hat{V}_1} اور \عددی{\hat{V}_2} پر کرخوف مساوات رو لکھتے ہیں۔
\begin{align}
\frac{\hat{V}_1-20\phase{70^{\circ}}}{j2}+\frac{\hat{V}_1}{10}-2\phase{30^{\circ}}&=0 \label{مساوات_بدلتا_جوڑ_ترکیب_الف}\\
2\phase{30^{\circ}}+\frac{\hat{V}_2}{j6}+\frac{\hat{V}_2}{2-j4}&=0\label{مساوات_بدلتا_جوڑ_ترکیب_ب}
\end{align}
مساوات \حوالہ{مساوات_بدلتا_جوڑ_ترکیب_الف} سے درج ذیل ملتا ہے
\begin{align*}
\hat{V}_1\left(\frac{1}{j2}+\frac{1}{10}\right)&=2\phase{30^{\circ}}+\frac{20\phase{70^{\circ}}}{j2}\\
&=1.7321+j+9.3970-j3.4202
\end{align*}
جسے حل کرتے ہیں۔
\begin{align*}
\hat{V}_1&=\frac{11.1291-j2.4202}{0.1-j0.5}\\
&=8.9347+j20.4713\\
&=22.3361\phase{66.42^{\circ}}\,\si{\volt}
\end{align*}
مساوات \حوالہ{مساوات_بدلتا_جوڑ_ترکیب_ب} سے درج ذیل حاصل ہوتا ہے۔
\begin{align*}
\hat{V}_2&=\frac{-2\phase{30^{\circ}}}{\frac{1}{j6}+\frac{1}{2-j4}}\\
&=-18.5885-j3.8038\\
&=18.97\phase{-168.43^{\circ}} \, \si{\volt}
\end{align*}
\انتہا{مثال}
%===================
\ابتدا{مثال}\شناخت{مثال_بدلتا_تراکیب_ب}
شکل \حوالہ{شکل_بدلتا_تراکیب_ب} کو دائری ترکیب سے حل کرتے ہوئے \عددی{\hat{I}_0} حاصل کریں۔
\begin{figure}
\centering
\begin{tikzpicture}
\draw(0,0) to [american current source,l_={$5\phase{60^{\circ}}\,\si{\ampere}$}]++(0,\yy) to [capacitor,l_={$-j4\,\si{\ohm}$}]++(-\xx,0)node[above]{$b$} to [inductor,l_={$j2\,\si{\ohm}$}]++(0,-\yy)node[below]{$a$} to [short] (0,0);
\draw(0,0) to [short,*-*]++(\xx,0)node[ground]{}node[above left]{$d$} to [resistor,l_={$\SI{10}{\ohm}$}]++(0,\yy)node[above]{$c$} to [american voltage source,-*,l_={$30\phase{-40^{\circ}}\,\si{\volt}$}]++(-\xx,0);
\draw(\xx,0) to [short]++(\xx,0)node[below]{$f$} to [resistor,i<_={$\hat{I}_0$},l_={$\SI{2}{\ohm}$}]++(0,\yy)node[above]{$e$} to [inductor,-*,l_={$j6\,\si{\ohm}$}]++(-\xx,0);
%current
\draw[stealth-] ([shift={(-150:\xx/4)}]-\xx/2,\yy/2) arc (-150:150:\xx/4);
\draw(-\xx/2,0)node[above]{$\hat{I}_1$};
\draw[stealth-] ([shift={(-120:\xx/4)}]\xx/2,\yy/2) arc (-120:120:\xx/4);
\draw(\xx/2,0)node[above]{$\hat{I}_2$};
\draw[stealth-] ([shift={(-120:\xx/4)}]\xx+\xx/2,\yy/2) arc (-120:120:\xx/4);
\draw(\xx+\xx/2,0)node[above]{$\hat{I}_3$};
\end{tikzpicture}
\caption{مثال \حوالہ{مثال_بدلتا_تراکیب_ب} کا دور۔}
\label{شکل_بدلتا_تراکیب_ب}
\end{figure}

حل:تین خانوں میں رو فرض کرتے ہوئے دائرہ \عددی{abcda} اور \عددی{dcefd} پر  کرخوف مساوات دباو لکھتے ہیں۔
\begin{align}
(j2-j4)\hat{I}_1+30\phase{-40^{\circ}}+10(\hat{I}_2-\hat{I}_3)&=0 \label{مساوات_بدلتا_دائری_ترکیب_الف}\\
10(\hat{I_3}-\hat{I}_2)+(2+6j)\hat{I}_3&=0  \label{مساوات_بدلتا_دائری_ترکیب_ب}
\end{align} 
چونکہ \عددی{\hat{I}_1} اور \عددی{\hat{I}_2} منبع رو سے گزرتے ہیں لہٰذا درج ذیل لکھا جائے گا۔
\begin{align}
\hat{I}_2-\hat{I}_1=5\phase{60^{\circ}}  \label{مساوات_بدلتا_دائری_ترکیب_پ}
\end{align}
درج بالا تین ہمزاد مساوات کو حل کرتے ہوئے تینوں دائروں کی رو حاصل ہو گی۔

مساوات \حوالہ{مساوات_بدلتا_دائری_ترکیب_پ} سے \عددی{\hat{I}_1=\hat{I}_2-5\phase{60^{\circ}}} لیتے ہوئے مساوات \حوالہ{مساوات_بدلتا_دائری_ترکیب_الف} میں پُر کرتے اور ترتیب دیتے ہوئے درج ذیل مساوات حاصل ہوتا ہے۔
\begin{gather}
\begin{aligned} \label{مساوات_بدلتا_دائری_ترکیب_ت}
(10-j2)\hat{I}_2-10\hat{I}_3&=-30\phase{-40^{\circ}}-(j2)(5\phase{60^{\circ}})\\
&=-14.3211+j14.2836
\end{aligned}
\end{gather}
مساوات \حوالہ{مساوات_بدلتا_دائری_ترکیب_ب} سے \عددی{\hat{I}_2=\left(\tfrac{12+6j}{10}\right)\hat{I}_3} لیتے ہوئے مساوات \حوالہ{مساوات_بدلتا_دائری_ترکیب_ت} میں پُر کرتے ہیں۔
\begin{align*}
(10-j2)\left(\frac{12+6j}{10}\right)\hat{I}_3-10\hat{I}_3&=-14.3211+j14.2836
\end{align*}
اس کو حل کرنے سے
\begin{align*}
\hat{I}_3&=\frac{-14.3211+j14.2836}{3.2+j3.6}\\
&=0.2411+j4.1924\\
&=4.1993\phase{86.71^{\circ}}\,\si{\ampere}
\end{align*}
حاصل ہوتا ہے۔ شکل سے ظاہر ہے کہ یہی \عددی{\hat{I}_0} ہے یعنی
\begin{align}
\hat{I}_0=\hat{I}_3=4.1993\phase{86.71^{\circ}}\,\si{\ampere}
\end{align}

رو \عددی{\hat{I}_3} جاننے کے بعد مساوات \حوالہ{مساوات_بدلتا_دائری_ترکیب_ب} سے \عددی{\hat{I}_2} حاصل کرتے ہیں۔
\begin{align*}
\hat{I}_2&=\left(\frac{12+6j}{10}\right)\hat{I}_3\\
&=\left(\frac{12+6j}{10}\right)(0.2411+j4.1924)\\
&=-2.2261+j5.1755\\
&=5.63\phase{113.27^{\circ}}\,\si{\volt}
\end{align*}
رو \عددی{\hat{I}_2} جانتے ہوئے مساوات \حوالہ{مساوات_بدلتا_دائری_ترکیب_پ} سے \عددی{\hat{I}_1} حاصل کرتے ہیں۔
\begin{align*}
\hat{I}_1&=\hat{I}_2-5\phase{60^{\circ}} \\
&=(-2.2261+j5.1755)-(2.5+j4.3301)\\
&=-4.7261+j0.8454\\
&=4.8\phase{169.86^{\circ}}\,\si{\volt}
\end{align*}
\انتہا{مثال}
%====================
\ابتدا{مثال}\شناخت{مثال_بدلتا_تراکیب_پ}
شکل \حوالہ{شکل_بدلتا_تراکیب_ب} کو مسئلہ نفاذ سے حل کریں۔
\begin{figure}
\centering
\begin{tikzpicture}
\draw(0,0) to [american current source,l_={$5\phase{60^{\circ}}\,\si{\ampere}$}]++(0,\yy)node[above]{$\hat{V}$} to [capacitor,l_={$-j4\,\si{\ohm}$}]++(-\xx,0) to [inductor,l_={$j2\,\si{\ohm}$}]++(0,-\yy) to [short] (0,0);
\draw(0,0) to [short,*-*]++(\xx,0)node[ground]{} to [resistor,l_={$\SI{10}{\ohm}$}]++(0,\yy) to [short,-*]++(-\xx,0);
\draw(\xx,0) to [short]++(\xx,0) to [resistor,i<_={$\hat{I}'_0$},l_={$\SI{2}{\ohm}$}]++(0,\yy) to [inductor,-*,l_={$j6\,\si{\ohm}$}]++(-\xx,0);
\end{tikzpicture}
\caption{مثال \حوالہ{مثال_بدلتا_تراکیب_پ} کا دور۔ منبع دباو کو قصر دور کیا گیا ہے۔}
\label{شکل_بدلتا_تراکیب_پ}
\end{figure}


حل:مسئلہ نفاذ میں تمام منبع کے انفرادی اثرات کا مجموعہ لیا جاتا ہے۔شکل \حوالہ{شکل_بدلتا_تراکیب_ب} میں منبع دباو کو قصر دور کرتے ہوئے  شکل \حوالہ{شکل_بدلتا_تراکیب_پ} حاصل ہوتا ہے۔منبع رو کے متوازی کل رکاوٹ \عددی{ \bZ_1} حاصل کرتے ہیں۔شکل کو دیکھتے ہوئے درج ذیل لکھا جا سکتا ہے
\begin{align*}
\frac{1}{\bZ_1}&=\frac{1}{j2-j4}+\frac{1}{10}+\frac{1}{2+j6}\\
&=\frac{3}{20}+j\frac{7}{20}
\end{align*}
جس سے
\begin{align*}
\bZ_1&=\frac{1}{\frac{3}{20}+j\frac{7}{20}}\\
&=\frac{30}{29}-j\frac{70}{29}\\
&=2.6261\phase{-66.8^{\circ}} \, \si{\ohm}
\end{align*}
حاصل ہوتا ہے۔یوں دباو \عددی{\hat{V}} درج ذیل ہو گا۔
\begin{align*}
\hat{V}&=(5\phase{60^{\circ}})(2.6261\phase{-66.8^{\circ}})=13.13\phase{-6.8^{\circ}}\,\si{\volt} 
\end{align*}
دباو جانتے ہوئے درکار رو درج ذیل لکھی جا سکتی ہے۔
\begin{gather}
\begin{aligned}\label{مساوات-بدلتا_منبع_رو_کی_رو}
\hat{I}'_0&=\frac{\hat{V}}{2+j6}\\
&=\frac{13.13\phase{-6.8^{\circ}}}{2+j6}\\
&=2.076\phase{-78.37^{\circ}} \, \si{\ampere}
\end{aligned}
\end{gather}
%
\begin{figure}
\centering
\begin{tikzpicture}
\draw(0,0) ++(0,\yy) to [capacitor,l_={$-j4\,\si{\ohm}$}]++(-\xx,0) to [inductor,l_={$j2\,\si{\ohm}$}]++(0,-\yy) to [short] (0,0);
\draw(0,0) to [short]++(\xx,0)node[ground]{} to [resistor,l_={$\SI{10}{\ohm}$}]++(0,\yy) to [american voltage source,i={$\hat{I}$},l_={$30\phase{-40^{\circ}}\,\si{\volt}$}]++(-\xx,0);
\draw(\xx,0) to [short]++(\xx,0) to [resistor,i<_={$\hat{I}''_0$},l_={$\SI{2}{\ohm}$}]++(0,\yy) to [inductor,-*,l_={$j6\,\si{\ohm}$}]++(-\xx,0);
\end{tikzpicture}
\caption{مثال \حوالہ{مثال_بدلتا_تراکیب_ب} کا دور۔منبع رو کو کھلا دور کیا گیا ہے۔}
\label{شکل_بدلتا_تراکیب_ت}
\end{figure}

آئیں اب منبع دباو سے پیدا رو حاصل کریں۔ایسا کرنے کی خاطر منبع رو کو کھلا دور کرتے ہوئے شکل \حوالہ{شکل_بدلتا_تراکیب_ت} حاصل کرتے ہیں۔اس شکل میں \عددی{\hat{I}} جانتے ہوئے تقسیم رو کے کلیے سے \عددی{\hat{I}''_0} حاصل کیا جا سکتا ہے۔آئیں پہلے \عددی{\hat{I}} حاصل کریں۔منبع رو کے ساتھ کل درج ذیل رکاوٹ جڑی ہے۔
\begin{align*}
\bZ_2&=j2-j4+\frac{10(2+j6)}{10+2+j6}\\
&=\frac{10}{3}+j\frac{4}{3}\\
&=3.59\phase{21.8^{\circ}}\,\si{\ohm}
\end{align*}
یوں رو \عددی{\hat{I}} درج ذیل ہو گی۔
\begin{align*}
\hat{I}&=\frac{30\phase{-40^{\circ}}}{3.59\phase{21.8^{\circ}}}\\
&=8.356\phase{-61.8^{\circ}}\,\si{\ampere}
\end{align*}
رو \عددی{\hat{I}} جانتے ہوئے تقسیم رو کے کلیے سے \عددی{\hat{I}''_0} حاصل کرتے ہیں جہاں منفی کی علامت اس لئے استعمال کی گئی ہے کہ  \عددی{\hat{I}} اور \عددی{\hat{I}''_0} کی سمتیں آپس میں الٹ ہیں۔
\begin{align*}
\hat{I}''_0&=-\left(\frac{10}{10+2+j6}\right) \hat{I}\\
&=-\left(\frac{10}{10+2+j6}\right) 8.356\phase{-61.8^{\circ}}\\
&=6.23\phase{91.63^{\circ}}\,\si{\ampere}
\end{align*}
شکل \حوالہ{شکل_بدلتا_تراکیب_پ} اور شکل \حوالہ{شکل_بدلتا_تراکیب_ت} میں حاصل کئے گئے رو کا مجموعہ درکار رو ہے یعنی
\begin{align*}
\hat{I}_0&=\hat{I}'_0+\hat{I}''_0\\
&=2.076\phase{-78.37^{\circ}}+6.23\phase{91.63^{\circ}}\\
&=4.1993\phase{86.71^{\circ}}\,\si{\ampere}
\end{align*}
\انتہا{مثال}
%====================
\ابتدا{مثال}\شناخت{مثال_بدلتا_تراکیب_ٹ}
شکل \حوالہ{شکل_بدلتا_تراکیب_ٹ}-الف کو تبادلہ منبع سے حل کرتے ہوئے \عددی{\hat{I}_0} حاصل کریں۔

\begin{figure}
\centering
\begin{subfigure}{1\textwidth}
\centering
\begin{tikzpicture}
\draw(0,\yy) to [american current source,l_={$10\phase{30^{\circ}}\,\si{\ampere}$}]++(0,-\yy)++(0,\yy) to [capacitor,l_={$-j4\,\si{\ohm}$}]++(-\xx,0) to [resistor,l_={$\SI{2}{\ohm}$}]++(0,-\yy) to [short] (0,0);
\draw(0,0) to [short,*-]++(\xx,0) to [capacitor,*-,l_={$-j12 \, \si{\ohm}$}]++(0,\yy) to [american voltage source,i={$\hat{I}_0$},-*,l_={$100\phase{60^{\circ}}$}]++(-\xx,0);
\draw(\xx,0) to [short]++(\xx,0) to [resistor,l_={$\SI{4}{\ohm}$}]++(0,\yy) to [inductor,-*,l_={$j8\,\si{\ohm}$}]++(-\xx,0);
\end{tikzpicture}
\caption*{(الف)}
\end{subfigure}
\begin{subfigure}{1\textwidth}
\centering
\begin{tikzpicture}
\draw(0,\yy) to [capacitor,l={$-j4\,\si{\ohm}$}]++(-\xx,0)  to [resistor,l_={$\SI{2}{\ohm}$}]++(-\xx,0) to [american voltage source,l={$44.79\phase{-33.43^{\circ}}\,\si{\volt}$}]++(0,-\yy) to [short] (\xx,0);
\draw(0,0) to [short,*-*]++(\xx,0) to [capacitor,l_={$-j12 \, \si{\ohm}$}]++(0,\yy) to [american voltage source,i={$\hat{I}_0$},-*,l_={$100\phase{60^{\circ}}$}]++(-\xx,0);
\draw(\xx,0) to [short]++(\xx,0) to [resistor,l_={$\SI{4}{\ohm}$}]++(0,\yy) to [inductor,-*,l_={$j8\,\si{\ohm}$}]++(-\xx,0);
\end{tikzpicture}
\caption*{(ب)}
\end{subfigure}
\begin{subfigure}{1\textwidth}
\centering
\begin{tikzpicture}
\draw(0,\yy) to [capacitor,l={$-j4\,\si{\ohm}$}]++(-\xx,0)  to [resistor,l_={$\SI{2}{\ohm}$}]++(-\xx,0) to [american voltage source,l={$44.79\phase{-33.43^{\circ}}\,\si{\volt}$}]++(0,-\yy) to [short] (\xx,0);
\draw(0,0) to [short,-*]++(\xx,0)++(0,\yy) to [american voltage source,i={$\hat{I}_0$},l_={$100\phase{60^{\circ}}$}]++(-\xx,0);
\draw(\xx,0) to [short]++(\xx,0) to [resistor,l_={$\SI{18}{\ohm}$}]++(0,\yy) to [inductor,-*,l_={$j6\,\si{\ohm}$}]++(-\xx,0);
\end{tikzpicture}
\caption*{(پ)}
\end{subfigure}
\caption{مثال \حوالہ{مثال_بدلتا_تراکیب_ٹ} کا دور۔}
\label{شکل_بدلتا_تراکیب_ٹ}
\end{figure}

حل:منبع رو کے متوازی رکاوٹ \عددی{2-j4} جڑی ہے۔ان کو نارٹن مساوی دور تصور کرتے ہوئے ان کی جگہ  تھونن مساوی  دور نسب کرتے ہوئے شکل-ب ملتا ہے جہاں درج ذیل متبادل منبع دباو نسب کیا گیا ہے۔
\begin{align*}
(10\phase{30^{\circ}})(2-j4)&=44.72\phase{-33.43^{\circ}}\,\si{\volt}
\end{align*}
شکل \حوالہ{شکل_بدلتا_تراکیب_ٹ}-ب میں دائیں جانب مساوی رکاوٹ درج ذیل ہے
\begin{align*}
\frac{-j12(4+j8)}{-j12+4+j8}=18+j6
\end{align*}
جسے استعمال کرتے ہوئے شکل-پ ملتی ہے۔شکل-پ کو دیکھ کر درج ذیل لکھا جا سکتا ہے۔
\begin{align*}
\hat{I}_0&=\frac{44.72\phase{-33.43^{\circ}}+100\phase{60^{\circ}}}{2-j4+j6+18}\\
&=\frac{87.3205+j61.9615}{20+j2}\\
&=5.33\phase{29.65^{\circ}}\,\si{\ampere}
\end{align*}
\انتہا{مثال}

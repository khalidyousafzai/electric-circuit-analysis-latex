\باب{تجزیہ عارضی حال}
\حصہ{تعارف}
ایسے ادوار جن میں امالہ گیر اور (یا) برق گیر پائے جاتے ہوں میں توانائی ذخیرہ کرنے کی صلاحیت ہوتی ہے۔توانائی ذخیرہ کرنے والے ادوار کا رد عمل منبع طاقت کے علاوہ ذخیرہ توانائی پر بھی منحصر ہوتا ہے۔ایسے ادوار میں کسی بھی طرح کی تبدیلی سے ذخیرہ توانائی میں تبدیلی رونما ہو سکتی ہے۔دور میں تبدیلی مثلاً  کسی سوئچ کے چالو یا غیر چالو کرنے سے پیدا ہو سکتی ہے۔ایسی صورت جہاں دور یکساں ایک ہی حالت میں رہے کو \اصطلاح{برقرار حالت}\فرہنگ{برقرار حالت}\حاشیہب{steady state}\فرہنگ{steady state} کہتے ہیں۔تبدیلی کے بعد دور متبادل برقرار حالت اختیار کرتا ہے۔ایک برقرار حالت سے دوسری برقرار حالت تک پہنچنے کے دوران، دور  \اصطلاح{عارضی حالت}\فرہنگ{عارضی حالت}\حاشیہب{transient state}\فرہنگ{transient state} میں ہوتا ہے۔

\حصہ{ایک درجی ادوار}
وہ ادوار جن میں صرف امالہ گیر توانائی ذخیرہ کرتے ہوں کی کرخوف مساوات \اصطلاح{ایک درجی تفرقی مساوات}\فرہنگ{تفرقی مساوات!ایک درجی}\فرہنگ{ایک درجی!تفرقی مساوات}\حاشیہب{first order differential equation}\فرہنگ{differential equation!first order} ہوتی ہے۔اسی طرح وہ ادوار جن میں صرف برق گیر توانائی ذخیرہ کرتے ہوں بھی ایک درجی کرخوف مساوات  دیتے ہیں۔اسی لئے انہیں \اصطلاح{یک درجی ادوار}\فرہنگ{یک درجی ادوار}\فرہنگ{دور!یک درجی}\حاشیہب{first order circuits}\فرہنگ{first order!circuits} کہتے ہیں۔اس کے برعکس ایسے ادوار جن میں امالہ گیر اور برق گیر دونوں پائے جاتے ہوں \اصطلاح{دو درجی تفرقی مساوات}\فرہنگ{دو درجی!تفرقی مساوات}\فرہنگ{تفرقی مساوات!دو درجی}\حاشیہب{second order differential equations}\فرہنگ{differential equation!second order} دیتے ہیں اور انہیں \اصطلاح{دو درجی ادوار}\فرہنگ{دو درجی!ادوار}\فرہنگ{دور!دو درجی}\حاشیہب{second order circuits}\فرہنگ{second order!circuits} کہا جاتا ہے۔

\begin{figure}
\centering
\begin{subfigure}{0.5\textwidth}
\centering
\begin{tikzpicture}
\draw(0,0) to [american voltage source,l={$v_i(t)$}]++(0,\y) to [resistor,i>^={$i(t)$},l={$R$}]++(\x,0) to [inductor,l={$L$}]++(0,-\y) to [short]++(-\x,0);
\end{tikzpicture}
\caption*{(الف)}
\end{subfigure}%
\begin{subfigure}{0.5\textwidth}
\centering
\begin{tikzpicture}
\draw(0,0) to [american current source,l={$i_i(t)$}]++(0,\y) to [short]++(2*\x,0) to [capacitor,i={$i_C(t)$},l_={$C$}]++(0,-\y) to [short]++(-2*\x,0);
\draw(\x,0)node[ground]{} to [resistor,*-*,i<_={$i_R(t)$},l={$R$}]++(0,\y)node[above]{$v(t)$};
\end{tikzpicture}
\caption*{(ب)}
\end{subfigure}%
\caption{ایک درجی ادوار کی مثالیں۔}
\label{شکل_عارضی_ایک_درجی_ادوار_الف}
\end{figure}

شکل \حوالہ{شکل_عارضی_ایک_درجی_ادوار_الف} میں ایک درجی ادوار کی مثالیں دی گئی ہیں۔آئیں ان کی کرخوف مساوات لکھ کر دیکھیں۔شکل-الف کی مساوات درج ذیل ہے۔
\begin{align}
v(t)=i(t)R+L \frac{\dif i(t)}{\dif t}
\end{align}
اسی طرح شکل-ب کی کرخوف مساوات درج ذیل ہے۔
\begin{align}
i_i(t)=\frac{v(t)}{R}+C\frac{\dif v(t)}{\dif t}
\end{align}
آپ دیکھ سکتے ہیں کہ درج بالا دونوں مساوات ایک درجی تفرقی مساوات ہیں۔

\begin{figure}
\centering
\begin{tikzpicture}
\draw(0,0) to [american voltage source,l={$v_i(t)$}]++(0,\y) to [resistor,i={$i(t)$},l={$R$}]++(\x,0) to [inductor,l={$L$}]++(\x,0) to [capacitor,l={$C$}]++(0,-\y) to [short]++(-2*\x,0);
\end{tikzpicture}
\caption{دو درجی دور۔}
\label{شکل_عارضی_دور_درجی_دور_الف}
\end{figure}

شکل \حوالہ{شکل_عارضی_دور_درجی_دور_الف} میں دو درجی دور دکھایا گیا ہے جس کی کرخوف مساوات درج ذیل ہے۔
\begin{align*}
v_i(t)=R i(t)+L\frac{\dif i(t)}{\dif t}+\frac{1}{C} \int_{-\infty}^{t}i(t) \dif t
\end{align*}
اس مساوات میں تکمل کی علامت ختم کرنے سے تفرقی مساوات حاصل ہو گی۔تکمل کی علامت ختم کرنے کی خاطر اس کا تفرق لیتے ہیں۔
\begin{align}
\frac{\dif v_i(t)}{\dif t}=R \frac{\dif i(t)}{\dif t}+L\frac{\dif^{\,2} i(t)}{\dif t^2}+\frac{i(t)}{C}
\end{align} 
آپ دیکھ سکتے ہیں کہ امالہ گیر اور برق گیر دونوں کی موجودگی سے دو درجی تفرقی مساوات حاصل ہوتی ہے۔

\جزوحصہ{رد عمل کی عمومی مساوات}
ایک درجی ادوار کے رد عمل جاننے کی خاطر ان کی تفرقی مساوات حل کی جاتی ہے جس سے دور کے مختلف مقامات پر دباو اور رو حاصل کی جاتی ہے۔ان یک درجی مساوات کی عمومی صورت درج ذیل ہوتی ہے
\begin{align}\label{مساوات_عارضی_یک_درجی_عمومی_مساوات}
\frac{\dif x(t)}{\dif t}+a x(t)= f(t)
\end{align}
جہاں \عددی{x(t)} دباو یا رو کو ظاہر کرتی ہے، \عددی{a} مستقل ہے اور \عددی{f(t)} تفاعل عملی ہے۔اس مساوات کا آزاد متغیرہ وقت \عددی{t} ہے۔تفرقی مساوات کا ایک بنیادی مسئلہ کہتا ہے کہ مساوات \حوالہ{مساوات_عارضی_یک_درجی_عمومی_مساوات} کا مکمل حل اس کے \اصطلاح{فطری رد عمل}\فرہنگ{فطری رد عمل}\فرہنگ{رد عمل!فطری}\حاشیہب{natural response, complementary solution}\فرہنگ{complementary solution}\فرہنگ{natural response}\فرہنگ{response!natural} \عددی{x_c(t)} اور \اصطلاح{جبری رد عمل}\فرہنگ{جبری رد عمل}\فرہنگ{رد عمل!جبری}\حاشیہب{forced response, particular solution}\فرہنگ{particular solution}\فرہنگ{forced response} \عددی{x_p(t)} کا مجموعہ ہے۔مساوات \حوالہ{مساوات_عارضی_یک_درجی_عمومی_مساوات} کے کسی بھی حل کو بطور جبری رد عمل لیا جا سکتا ہے جبکہ درج ذیل مساوات
\begin{align}\label{مساوات_عارضی_یک_درجی_عمومی_مساوات_ب}
\frac{\dif x(t)}{\dif t}+a x(t)=0
\end{align}
 کے کسی بھی حل کو فطری رد عمل تصور کیا جا سکتا ہے۔مساوات \حوالہ{مساوات_عارضی_یک_درجی_عمومی_مساوات} میں \عددی{f(t)=0} پُر کرنے سے مساوات \حوالہ{مساوات_عارضی_یک_درجی_عمومی_مساوات_ب} حاصل ہوتا ہے۔مساوات \حوالہ{مساوات_عارضی_یک_درجی_عمومی_مساوات_ب} \اصطلاح{ہم جنسی مساوات}\فرہنگ{ہم جنسی مساوات}\فرہنگ{مساوات!ہم جنسی}\حاشیہب{homogenous equation}\فرہنگ{homogenous equation} مساوات کہلاتی ہے۔

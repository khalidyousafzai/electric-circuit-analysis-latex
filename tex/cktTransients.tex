\باب{عارضی رد عمل}
\حصہ{تعارف}
ایسے ادوار جن میں امالہ گیر اور (یا) برق گیر پائے جاتے ہوں میں توانائی ذخیرہ کرنے کی صلاحیت ہوتی ہے۔توانائی ذخیرہ کرنے والے ادوار کا رد عمل منبع طاقت کے علاوہ ذخیرہ توانائی پر بھی منحصر ہوتا ہے۔ایسے ادوار میں کسی بھی طرح کی تبدیلی سے ذخیرہ توانائی میں تبدیلی رونما ہو سکتی ہے۔دور میں تبدیلی مثلاً  کسی سوئچ کے چالو یا غیر چالو کرنے سے پیدا ہو سکتی ہے۔ایسی صورت جہاں دور یکساں ایک ہی حالت میں رہے کو \اصطلاح{برقرار حالت}\فرہنگ{برقرار حالت}\حاشیہب{steady state}\فرہنگ{steady state} کہتے ہیں۔تبدیلی کے بعد دور متبادل برقرار حالت اختیار کرتا ہے۔ایک برقرار حالت سے دوسری برقرار حالت تک پہنچنے کے دوران، دور  \اصطلاح{عارضی حالت}\فرہنگ{عارضی حالت}\حاشیہب{transient state}\فرہنگ{transient state} میں ہوتا ہے۔

\حصہ{ایک درجی ادوار}
وہ ادوار جن میں صرف امالہ گیر توانائی ذخیرہ کرتے ہوں کی کرخوف مساوات \اصطلاح{ایک درجی تفرقی مساوات}\فرہنگ{تفرقی مساوات!ایک درجی}\فرہنگ{ایک درجی!تفرقی مساوات}\حاشیہب{first order differential equation}\فرہنگ{differential equation!first order} ہوتی ہے۔اسی طرح وہ ادوار جن میں صرف برق گیر توانائی ذخیرہ کرتے ہوں بھی ایک درجی کرخوف مساوات  دیتے ہیں۔اسی لئے انہیں \اصطلاح{یک درجی ادوار}\فرہنگ{یک درجی ادوار}\فرہنگ{دور!یک درجی}\حاشیہب{first order circuits}\فرہنگ{first order!circuits} کہتے ہیں۔اس کے برعکس ایسے ادوار جن میں امالہ گیر اور برق گیر دونوں پائے جاتے ہوں \اصطلاح{دو درجی تفرقی مساوات}\فرہنگ{دو درجی!تفرقی مساوات}\فرہنگ{تفرقی مساوات!دو درجی}\حاشیہب{second order differential equations}\فرہنگ{differential equation!second order} دیتے ہیں اور انہیں \اصطلاح{دو درجی ادوار}\فرہنگ{دو درجی!ادوار}\فرہنگ{دور!دو درجی}\حاشیہب{second order circuits}\فرہنگ{second order!circuits} کہا جاتا ہے۔

\begin{figure}
\centering
\begin{subfigure}{0.5\textwidth}
\centering
\begin{tikzpicture}
\draw(0,0) to [american voltage source,l={$v_i(t)$}]++(0,\y) to [resistor,i>^={$i(t)$},l={$R$}]++(\x,0) to [inductor,l={$L$}]++(0,-\y) to [short]++(-\x,0);
\end{tikzpicture}
\caption*{(الف)}
\end{subfigure}%
\begin{subfigure}{0.5\textwidth}
\centering
\begin{tikzpicture}
\draw(0,0) to [american current source,l={$i_i(t)$}]++(0,\y) to [short]++(2*\x,0) to [capacitor,i={$i_C(t)$},l_={$C$}]++(0,-\y) to [short]++(-2*\x,0);
\draw(\x,0)node[ground]{} to [resistor,*-*,i<_={$i_R(t)$},l={$R$}]++(0,\y)node[above]{$v(t)$};
\end{tikzpicture}
\caption*{(ب)}
\end{subfigure}%
\caption{ایک درجی ادوار کی مثالیں۔}
\label{شکل_عارضی_ایک_درجی_ادوار_الف}
\end{figure}

شکل \حوالہ{شکل_عارضی_ایک_درجی_ادوار_الف} میں ایک درجی ادوار کی مثالیں دی گئی ہیں۔آئیں ان کی کرخوف مساوات لکھ کر دیکھیں۔شکل-الف کی مساوات درج ذیل ہے۔
\begin{align}
v(t)=i(t)R+L \frac{\dif i(t)}{\dif t}
\end{align}
اسی طرح شکل-ب کی کرخوف مساوات درج ذیل ہے۔
\begin{align}
i_i(t)=\frac{v(t)}{R}+C\frac{\dif v(t)}{\dif t}
\end{align}
آپ دیکھ سکتے ہیں کہ درج بالا دونوں مساوات ایک درجی تفرقی مساوات ہیں۔

\begin{figure}
\centering
\begin{tikzpicture}
\draw(0,0) to [american voltage source,l={$v_i(t)$}]++(0,\y) to [resistor,i={$i(t)$},l={$R$}]++(\x,0) to [inductor,l={$L$}]++(\x,0) to [capacitor,l={$C$}]++(0,-\y) to [short]++(-2*\x,0);
\end{tikzpicture}
\caption{دو درجی دور۔}
\label{شکل_عارضی_دور_درجی_دور_الف}
\end{figure}

شکل \حوالہ{شکل_عارضی_دور_درجی_دور_الف} میں دو درجی دور دکھایا گیا ہے جس کی کرخوف مساوات درج ذیل ہے۔
\begin{align*}
v_i(t)=R i(t)+L\frac{\dif i(t)}{\dif t}+\frac{1}{C} \int_{-\infty}^{t}i(t) \dif t
\end{align*}
اس مساوات میں تکمل کی علامت ختم کرنے سے تفرقی مساوات حاصل ہو گی۔تکمل کی علامت ختم کرنے کی خاطر اس کا تفرق لیتے ہیں۔
\begin{align}
\frac{\dif v_i(t)}{\dif t}=R \frac{\dif i(t)}{\dif t}+L\frac{\dif^{\,2} i(t)}{\dif t^2}+\frac{i(t)}{C}
\end{align} 
آپ دیکھ سکتے ہیں کہ امالہ گیر اور برق گیر دونوں کی موجودگی سے دو درجی تفرقی مساوات حاصل ہوتی ہے۔

\جزوحصہ{رد عمل کی عمومی مساوات}
ایک درجی ادوار کے رد عمل جاننے کی خاطر ان کی تفرقی مساوات حل کی جاتی ہے جس سے دور کے مختلف مقامات پر دباو اور رو حاصل کی جاتی ہے۔ان یک درجی مساوات کی عمومی صورت درج ذیل ہوتی ہے
\begin{align}\label{مساوات_عارضی_یک_درجی_عمومی_مساوات}
\frac{\dif y(t)}{\dif t}+a y(t)= g(t)
\end{align}
جہاں \عددی{y(t)} دباو یا رو کو ظاہر کرتی ہے، \عددی{a} مستقل ہے اور \عددی{g(t)} \اصطلاح{تفاعل عملی}\فرہنگ{تفاعل عملی}\حاشیہب{forcing function}\فرہنگ{forcing function} ہے۔اس مساوات کا آزاد متغیرہ وقت \عددی{t} ہے۔تفرقی مساوات کا ایک بنیادی مسئلہ کہتا ہے کہ مساوات \حوالہ{مساوات_عارضی_یک_درجی_عمومی_مساوات} کا مکمل حل اس کے \اصطلاح{فطری رد عمل}\فرہنگ{فطری رد عمل}\فرہنگ{رد عمل!فطری}\حاشیہب{natural response, complementary solution}\فرہنگ{complementary solution}\فرہنگ{natural response}\فرہنگ{response!natural} \عددی{y_f(t)} اور \اصطلاح{جبری رد عمل}\فرہنگ{جبری رد عمل}\فرہنگ{رد عمل!جبری}\حاشیہب{forced response, particular solution}\فرہنگ{particular solution}\فرہنگ{forced response} 
\عددی{y_j(t)} کا مجموعہ ہے۔مساوات \حوالہ{مساوات_عارضی_یک_درجی_عمومی_مساوات} کے کسی بھی حل کو بطور جبری رد عمل لیا جا سکتا ہے جبکہ درج ذیل \اصطلاح{ہم جنسی مساوات}\فرہنگ{ہم جنسی مساوات}\فرہنگ{مساوات!ہم جنسی}\حاشیہب{homogenous equation}\فرہنگ{homogenous equation}
\begin{align}\label{مساوات_عارضی_یک_درجی_عمومی_مساوات_ب}
\frac{\dif y(t)}{\dif t}+a y(t)=0
\end{align}
 کے کسی بھی حل کو فطری رد عمل تصور کیا جا سکتا ہے۔مساوات \حوالہ{مساوات_عارضی_یک_درجی_عمومی_مساوات} میں \عددی{g(t)=0} پُر کرنے سے ہم جنسی مساوات  حاصل ہوتی ہے۔

آئیں \عددی{g(t)=A} کی صورت میں مساوات \حوالہ{مساوات_عارضی_یک_درجی_عمومی_مساوات} کا حل حاصل کریں جہاں \عددی{A} ایک مستقل ہے۔یوں ہمیں درج ذیل دو مساوات کے حل درکار ہیں۔
\begin{align}
\frac{\dif y_j(t)}{\dif t}+a y_j(t)&=A \label{مساوات_عارضی_یک_درجی_عمومی_مساوات_پ}\\
\frac{\dif y_f(t)}{\dif t}+a y_f(t)&=0\label{مساوات_عارضی_یک_درجی_عمومی_مساوات_ت}
\end{align}
جبری حل کو قیاس کے ذریعہ حاصل کیا جائے گا۔  جبری حل کو تفاعل عملی اور اس کے تمام ممکنہ تفرق کے مجموعے کے برابر تصور کرتے ہوئے آگے بڑھتے ہیں۔چونکہ مستقل کا تفرق \عددی{(\tfrac{\dif A}{\dif t}=0)} صفر کے برابر ہے لہٰذا جبری حل کو مستقل \عددی{K_1} تصور کرتے ہیں۔
\begin{align}
y_j(t)=K_1
\end{align}
اس قیمت کو مساوات \حوالہ{مساوات_عارضی_یک_درجی_عمومی_مساوات_پ} میں پُر کرتے ہوئے حل کرنے سے
\begin{align*}
\frac{\dif K_1}{\dif t}+a K_1&=A \\
0+a K_1&=A
\end{align*}
یعنی
\begin{align}\label{مساوات_عارضی_یک_درجی_عمومی_مساوات_ٹ}
K_1=\frac{A}{a}
\end{align}
حاصل ہوتا ہے۔مساوات \حوالہ{مساوات_عارضی_یک_درجی_عمومی_مساوات_ت} کو ترتیب دیتے ہوئے
\begin{align*}
\frac{\dif y_f(t)}{y_f(t)}=-a \dif t
\end{align*}
لکھا جا سکتا ہے  جس کا تکمل
\begin{align*}
\ln y_f(t)=-a t +c
\end{align*}
یعنی
\begin{align}\label{مساوات_عارضی_یک_درجی_عمومی_مساوات_ث}
y_f(t)=K_2e^{-at}
\end{align}
کے برابر ہے جہاں \عددی{c} تکمل کا مستقل ہے اور \عددی{K_2=e^{c}} کے برابر ہے۔مساوات \حوالہ{مساوات_عارضی_یک_درجی_عمومی_مساوات_ٹ} اور مساوات \حوالہ{مساوات_عارضی_یک_درجی_عمومی_مساوات_ث} سے مکمل حل درج ذیل حاصل ہوتا ہے۔
\begin{align}
y(t)=\frac{A}{a}+K_2 e^{-at}
\end{align}
کسی بھی لمحے پر \عددی{y(t)} جاننے سے درج بالا مساوات میں نا معلوم مستقل \عددی{K_2} دریافت کیا جا سکتا ہے۔درج بالا مساوات کو درج ذیل عمومی حل کی صورت میں لکھا جا سکتا ہے
\begin{align}\label{مساوات_عارضی_یک_درجی_عمومی_مساوات_ج}
y(t)=K_1+K_2 e^{-\frac{t}{\tau}}
\end{align}
جہاں \عددی{\tau=\tfrac{1}{a}} کے برابر ہے۔


مساوات \حوالہ{مساوات_عارضی_یک_درجی_عمومی_مساوات_ج} کے مختلف اجزاء کو نام دیے گئے ہیں۔یوں \عددی{\tau} \اصطلاح{وقتی مستقل}\فرہنگ{وقتی مستقل}\حاشیہب{time constant}\فرہنگ{time constant} کہلاتا ہے جبکہ \عددی{K_1} \اصطلاح{برقرار حالت حل}\فرہنگ{برقرار حالت حل}\فرہنگ{حل:برقرار حالت}\فرہنگ{برقرار حالت:حل}\حاشیہب{steady state solution}\فرہنگ{steady state solution} کہلاتا ہے۔مساوات \حوالہ{مساوات_عارضی_یک_درجی_عمومی_مساوات_ج} میں \عددی{t=\infty} پُر کرنے سے برقرار حالت حل حاصل ہوتا ہے۔یوں کسی بھی تبدیلی کے بہت دیر بعد دور برقرار حالت میں ہو گا یعنی ابدی صورت کو برقرار حالت کہا جاتا ہے۔

\begin{figure}
\begin{tikzpicture}
\begin{axis}[name=ka,axis lines*=middle,
	 every axis x label/.style={
    at={(ticklabel* cs:1.05)},
    anchor=east,}, 
	every axis y label/.style={
    at={(ticklabel* cs:1.05)},
    anchor=east,}
,xlabel=$t$,ylabel=$K_2 e^{-\frac{t}{\tau}}$,ytick={1,0.368},yticklabel style={/pgf/number format/precision=3},yticklabels={$K_2$,$0.368 K_2$},xtick={0.5,1,1.5,2,2.5},xticklabels={$\tau$,$2\tau$,$3\tau$,$4\tau$,$5\tau$}]
\addplot[width=4cm,domain=0:3,samples=100]{e^(-x/0.5)};
\draw[dashed](axis cs:0,1)--(axis cs:0.5,0);
\draw[dashed](axis cs:0,0.368)--(axis cs:0.5,0.368)--(axis cs:0.5,0);
\end{axis}
\node [anchor=north] at (ka.south){(الف)};
\end{tikzpicture}%
\begin{tikzpicture}
\begin{axis}[name=kb,axis lines*=middle,
 every axis x label/.style={
    at={(ticklabel* cs:1.05)},
    anchor=east,}, 
	every axis y label/.style={
    at={(ticklabel* cs:1.05)},
    anchor=east,},
 xlabel=$t$,ylabel=$e^{-\frac{t}{\tau}}$]
\addplot[width=4cm,domain=0:3,samples=100]{e^(-x/0.5)}node[pos=0.25,above right]{$\tau=0.5$};
\addplot[width=4cm,domain=0:3,samples=100]{e^(-x/2)}node[pos=0.25,above right]{$\tau=2$};
\end{axis}
\node[anchor=north] at (ka.south){(ب)};
\end{tikzpicture}
\caption{وقتی مستقل}
\label{شکل_عارضی_وقتی_مستقل_الف}
\end{figure}

شکل \حوالہ{شکل_عارضی_وقتی_مستقل_الف}-الف میں مثبت \عددی{a} کی صورت میں جبری حل دکھایا گیا ہے۔ابتدائی لمحہ \عددی{t=0} پر \عددی{y_j(0)=K_2} کے برابر ہے جبکہ ایک وقتی مستقل برابر وقت بعد اس کی قیمت \عددی{y_j(\tau)=0.368K_2} رہ گئی ہے یعنی \عددی{\tau} دورانیے میں جبری حل کی قیمت میں \عددی{\SI{63.2}{\percent}} کمی واقع ہوئی ہے۔اسی طرح دو وقتی مستقل وقفے کے بعد \عددی{y_j(2\tau)=0.135K_2} ہے جو \عددی{y_p(\tau)} کے \عددی{0.368} گنا ہے۔حقیقت میں کسی بھی لمحہ \عددی{t_1} پر \عددی{y_j} کی قیمت میں لمحہ \عددی{t_1+\tau} پر \عددی{\SI{63.2}{\percent}} کمی واقع ہو گی۔پانچ وقتی مستقل وقفے کے بعد \عددی{y_j(5\tau)=0.0067K_2} رہ جاتا ہے جو ابتدائی قیمت کے \عددی{\SI{0.67}{\percent}} ہے۔

مساوات \حوالہ{مساوات_عارضی_یک_درجی_عمومی_مساوات_ث}  \اصطلاح{قوت نمائی انحطاطی}\فرہنگ{قوت نمائی!انحطاط}\حاشیہب{exponential decaying}\فرہنگ{exponential decay} خط ہے۔قوت نمائی انحطاطی خط کی ایک خصوصیت یہ ہے کہ ابتدائی لمحے  پر اس کا مماس افقی محور کو \عددی{\tau} پر کاٹتا ہے۔اس مماس کو شکل \حوالہ{شکل_عارضی_وقتی_مستقل_الف}-الف میں \عددی{(0,K_2)} تا \عددی{(\tau,0)} نقطہ دار لکیر سے دکھایا گیا ہے۔ شکل \حوالہ{شکل_عارضی_وقتی_مستقل_الف}-ب میں مختلف \عددی{\tau} کی قیمتوں کے لئے مساوات \حوالہ{مساوات_عارضی_یک_درجی_عمومی_مساوات_ث}  کو کھینچا گیا ہے۔آپ دیکھ سکتے ہیں کہ کم وقتی مستقل کا خط جلد اختتامی قیمت تک پہنچتا ہے۔یوں وقتی مستقل کسی بھی دور کے رد عمل کے دورانیے کی ناپ ہے۔
%=======================

\ابتدا{مثال}\شناخت{مثال_عارضی_یک_درجی_دور_الف}
شکل \حوالہ{شکل_عارضی_سلسلہ_وار_مزاحمت_برق_گیر_الف} میں مزاحمت اور بے بار برق گیر سلسلہ وار جڑے ہیں۔لمحہ \عددی{t=0} پر سوئچ چالو کرتے ہوئے انہیں مستقل منبع دباو \عددی{V_I} کے ساتھ جوڑا جاتا ہے۔برق گیر کا دباو \عددی{v(t)} اور رو \عددی{i(t)} دریافت کریں۔

\begin{figure}
\centering
\begin{tikzpicture}
\draw(0,0) to [american voltage source,l={$V_I$}]++(0,\y) to [cspst,l={${t=0}$}]++(\x,0) to [resistor,l={$R$}]++(\x,0)node[above]{$v(t)$} to [capacitor,-*,l={$C$}]++(0,-\y) node[ground]{} to [short] (0,0);
\draw(\x,-0.5)node{(الف)};
\end{tikzpicture}%
\begin{tikzpicture}
\begin{axis}[name=kb,axis lines*=middle,
	 every axis x label/.style={
    at={(ticklabel* cs:1.05)},
    anchor=east,}, 
	every axis y label/.style={
    at={(ticklabel* cs:1.05)},
    anchor=east,}
,xlabel=$t$,ylabel=$v(t)$,ytick={1,0.5},yticklabel style={/pgf/number format/precision=3},yticklabels={$V_I$,$0.5 V_I$},xtick={1,2,3,4,5},xticklabels={$RC$,$2RC$,$3RC$,$4RC$,$5RC$}]
\addplot[width=4cm,domain=0:5,samples=100]{1-e^(-x)}node[pos=0.3,below right]{$v(t)=V_I \left(1-e^{-\frac{t}{RC}}\right)$};
\end{axis}%
\node [anchor=north] at (kb.south){(ب)};
\end{tikzpicture}%
\begin{tikzpicture}
\begin{axis}[name=kc,axis lines*=middle,
	 every axis x label/.style={
    at={(ticklabel* cs:1.05)},
    anchor=east,}, 
	every axis y label/.style={
    at={(ticklabel* cs:1.05)},
    anchor=east,}
,xlabel=$t$,ylabel=$i(t)$,ytick={1,0.5},yticklabel style={/pgf/number format/precision=3},yticklabels={$\frac{V_I}{R}$,$0.5\frac{V_I}{R}$},xtick={1,2,3,4,5},xticklabels={$RC$,$2RC$,$3RC$,$4RC$,$5RC$}]
\addplot[width=4cm,domain=0:5,samples=100]{e^(-x)}node[pos=0.3,above right]{$i(t)=\frac{V_I}{R}e^{-\frac{t}{RC}}$};
\end{axis}%
\node [anchor=north] at (kc.south){(پ)};
\end{tikzpicture}%
\caption{مثال \حوالہ{مثال_عارضی_یک_درجی_دور_الف} کا دور، دباو اور رو۔}
\label{شکل_عارضی_سلسلہ_وار_مزاحمت_برق_گیر_الف}
\end{figure} 

حل:سوئچ چالو کرنے سے پہلے برق گیر بے بار ہے لہٰذا اس پر دباو صفر کے برابر ہے۔صفحہ \حوالہصفحہ{مساوات_امالہ_برق_گیر_دباو_مسلسل_ہے} پر مساوات \حوالہ{مساوات_امالہ_برق_گیر_دباو_مسلسل_ہے} کے تحت \عددی{v_C(0_+)=v_C(0_-)} ہو گا یعنی یوں سوئچ چالو کرنے کے فوراً بعد برق گیر کا دباو صفر ہی ہو گا۔سوئچ چالو کرنے کے بعد  دباو جوڑ \عددی{v(t)} کے استعمال سے کرخوف مساوات رو لکھتے ہیں
\begin{align*}
\frac{v(t)-V_I}{R}+C\frac{\dif v(t)}{\dif t}=0
\end{align*}
جسے ترتیب دیتے ہوئے
\begin{align}\label{مساوات_عارضی_برق_گیر_عارضی_حل_الف}
\frac{\dif v(t)}{\dif t}+\frac{v(t)}{RC}=\frac{V_I}{RC}
\end{align}
لکھا  جا سکتا ہے جو عمومی مساوات \حوالہ{مساوات_عارضی_یک_درجی_عمومی_مساوات} کی طرح ہے۔چونکہ \عددی{V_I} مستقل قیمت ہے لہٰذا اس مساوات کا جبری حل
\begin{align*}
v_j(t)=K_1
\end{align*}
 تصور کیا جا سکتا ہے جسے  مساوات \حوالہ{مساوات_عارضی_برق_گیر_عارضی_حل_الف} میں پُر کرتے ہوئے حل کرنے سے
\begin{align*}
\frac{\dif  K_1}{\dif t}+\frac{K_1}{RC}&=\frac{V_I}{RC}\\
0+\frac{K_1}{RC}&=\frac{V_I}{RC}
\end{align*}
یعنی
\begin{align*}
K_1=V_I
\end{align*}
حاصل ہوتا ہے۔یوں جبری حل درج ذیل حاصل ہوتا ہے۔
\begin{align*}
v_j(t)=V_I
\end{align*}
اس نتیجے کے تحت سوئچ چالو کرنے کے بہت دیر بعد برق گیر پر دباو عین منبع دباو کے برابر ہو گا۔شکل کو دیکھتے ہوئے اسی نتیجے تک یوں پہنچا جا سکتا ہے کہ سوئچ چالو کرنے کے بعد دور میں رو کی وجہ سے برق گیر پر بار جمع ہونا شروع ہو جائے گا۔جب تک برق گیر کا دباو منبع کے دباو سے کم ہو، مزاحمت پر دباو پایا جائے گا لہٰذا اس میں رو پائی جائے گی۔یہ رو برق گیر پر جمع بار میں اضافہ کرتی رہے گی۔عین اس وقت جب برق گیر اور منبع کے دباو برابر ہو جائیں، رو کی قیمت صفر ہو جائے گی اور برق گیر کا دباو اسی قیمت پر ابد تک برقرار رہے گا۔ 

آئیں اب فطری حل دریافت کریں۔فطری حل ہم جنسی مساوات سے حاصل ہوتا ہے۔مساوات \حوالہ{مساوات_عارضی_برق_گیر_عارضی_حل_الف} کے دائیں بازو کو صفر کے برابر پُر کرنے سے ہم جنسی مساوات
\begin{align}\label{مساوات_عارضی_برق_گیر_ہم_جنسی_الف}
\frac{\dif v(t)}{\dif t}+\frac{v(t)}{RC}=0
\end{align}
 حاصل ہوتی ہے۔اس کو
\begin{align*}
\frac{\dif v(t)}{v(t)}=-\frac{\dif t}{RC}
\end{align*}
لکھتے ہوئے تکمل لینے سے
\begin{align*}
\ln v(t)=-\frac{t}{RC}+c
\end{align*}
یعنی
\begin{align*}
v_f(t)=K_2 e^{-\frac{t}{RC}}
\end{align*}
فطری حل حاصل ہوتا ہے۔ جبری اور فطری حل کا مجموعہ مکمل حل ہو گا۔
\begin{align*}
v(t)=V_I+K_2 e^{-\frac{t}{RC}}
\end{align*}
مکمل حل میں نا معلوم مستقل کو \اصطلاح{ابتدائی شرائط}\فرہنگ{ابتدائی شرائط}\حاشیہب{initial conditions}\فرہنگ{initial conditions} سے حاصل کرتے ہیں جس کے تحت \عددی{t=0_+} پر \عددی{v_C(0_+)=0} کی قیمت معلوم ہے۔ان قیمتوں کو درج بالا مساوات میں پُر کرتے ہوئے حل کرنے سے
\begin{align*}
0&=V_I+K_2 e^{-\frac{0}{RC}}\\
0&=V_I+K_2
\end{align*}
یعنی
\begin{align*}
K_2=-V_I
\end{align*}
حاصل ہوتا ہے۔ یوں مکمل حل درج ذیل حاصل ہوتا ہے۔
\begin{align*}
v(t)=V_I\left(1-e^{-\frac{t}{RC}}\right)
\end{align*}
آپ دیکھ سکتے ہیں کہ وقتی مستقل \عددی{RC} کے برابر ہے۔ یوں \عددی{R} یا (اور) \عددی{C} بڑھانے سے  وقتی مستقل بڑھے گا جس سے دور برقرار صورت زیادہ دیر میں اختیار کرے گا۔ 

رو \عددی{i(t)} کو درج بالا مساوات سے حاصل کرتے ہیں۔
\begin{align*}
i(t)&=C\frac{\dif v(t)}{\dif t}\\
&=C V_I \left(0+\frac{1}{RC}e^{-\frac{t}{RC}}\right)\\
&=\frac{V_I}{R}e^{-\frac{t}{RC}}
\end{align*}
یہی رو مزاحمت پر اوہم کے قانون کی مدد سے بھی حاصل کی جا سکتی ہے یعنی
\begin{align*}
i(t)&=\frac{V_I-v(t)}{R}\\
&=\frac{V_I}{R}e^{-\frac{t}{RC}}
\end{align*} 
\انتہا{مثال}
%===============

\باب{عارضی رد عمل}\شناخت{باب_عارضی_رد_عمل}
\حصہ{تعارف}
ایسے ادوار جن میں امالہ گیر اور (یا) برق گیر پائے جاتے ہوں میں توانائی ذخیرہ کرنے کی صلاحیت ہوتی ہے۔توانائی ذخیرہ کرنے والے ادوار کا رد عمل منبع طاقت کے علاوہ ذخیرہ توانائی پر بھی منحصر ہوتا ہے۔ایسے ادوار میں کسی بھی طرح کی تبدیلی سے ذخیرہ توانائی میں تبدیلی رونما ہو سکتی ہے۔دور میں تبدیلی مثلاً  کسی سوئچ کے چالو یا غیر چالو کرنے سے پیدا ہو سکتی ہے۔ایسی صورت جہاں دور یکساں ایک ہی حالت میں رہے کو \اصطلاح{برقرار حالت}\فرہنگ{برقرار حالت}\حاشیہب{steady state}\فرہنگ{steady state} کہتے ہیں۔تبدیلی کے بعد دور متبادل برقرار حالت اختیار کرتا ہے۔ایک برقرار حالت سے دوسری برقرار حالت تک پہنچنے کے دوران، دور  \اصطلاح{عارضی حالت}\فرہنگ{عارضی حالت}\حاشیہب{transient state}\فرہنگ{transient state} میں ہوتا ہے۔

\حصہ{ایک درجی ادوار}
وہ ادوار جن میں صرف امالہ گیر توانائی ذخیرہ کرتے ہوں کی کرخوف مساوات \اصطلاح{ایک درجی تفرقی مساوات}\فرہنگ{تفرقی مساوات!ایک درجی}\فرہنگ{ایک درجی!تفرقی مساوات}\حاشیہب{first order differential equation}\فرہنگ{differential equation!first order} ہوتی ہے۔اسی طرح وہ ادوار جن میں صرف برق گیر توانائی ذخیرہ کرتے ہوں بھی ایک درجی کرخوف مساوات  دیتے ہیں۔اسی لئے انہیں \اصطلاح{یک درجی ادوار}\فرہنگ{یک درجی ادوار}\فرہنگ{دور!یک درجی}\حاشیہب{first order circuits}\فرہنگ{first order!circuits} کہتے ہیں۔اس کے برعکس ایسے ادوار جن میں امالہ گیر اور برق گیر دونوں پائے جاتے ہوں \اصطلاح{دو درجی تفرقی مساوات}\فرہنگ{دو درجی!تفرقی مساوات}\فرہنگ{تفرقی مساوات!دو درجی}\حاشیہب{second order differential equations}\فرہنگ{differential equation!second order} دیتے ہیں اور انہیں \اصطلاح{دو درجی ادوار}\فرہنگ{دو درجی!ادوار}\فرہنگ{دور!دو درجی}\حاشیہب{second order circuits}\فرہنگ{second order!circuits} کہا جاتا ہے۔

\begin{figure}
\centering
\begin{subfigure}{0.5\textwidth}
\centering
\begin{tikzpicture}
\draw(0,0) to [american voltage source,l={$v_i(t)$}]++(0,\y) to [resistor,i>^={$i(t)$},l={$R$}]++(\x,0) to [inductor,l={$L$}]++(0,-\y) to [short]++(-\x,0);
\end{tikzpicture}
\caption*{(الف)}
\end{subfigure}%
\begin{subfigure}{0.5\textwidth}
\centering
\begin{tikzpicture}
\draw(0,0) to [american current source,l={$i_i(t)$}]++(0,\y) to [short]++(2*\x,0) to [capacitor,i={$i_C(t)$},l_={$C$}]++(0,-\y) to [short]++(-2*\x,0);
\draw(\x,0)node[ground]{} to [resistor,*-*,i<_={$i_R(t)$},l={$R$}]++(0,\y)node[above]{$v(t)$};
\end{tikzpicture}
\caption*{(ب)}
\end{subfigure}%
\caption{ایک درجی ادوار کی مثالیں۔}
\label{شکل_عارضی_ایک_درجی_ادوار_الف}
\end{figure}

شکل \حوالہ{شکل_عارضی_ایک_درجی_ادوار_الف} میں ایک درجی ادوار کی مثالیں دی گئی ہیں۔آئیں ان کی کرخوف مساوات لکھ کر دیکھیں۔شکل-الف کی مساوات درج ذیل ہے۔
\begin{align}
v(t)=i(t)R+L \frac{\dif i(t)}{\dif t}
\end{align}
اسی طرح شکل-ب کی کرخوف مساوات درج ذیل ہے۔
\begin{align}
i_i(t)=\frac{v(t)}{R}+C\frac{\dif v(t)}{\dif t}
\end{align}
آپ دیکھ سکتے ہیں کہ درج بالا دونوں مساوات ایک درجی تفرقی مساوات ہیں۔

\begin{figure}
\centering
\begin{tikzpicture}
\draw(0,0) to [american voltage source,l={$v_i(t)$}]++(0,\y) to [resistor,i={$i(t)$},l={$R$}]++(\x,0) to [inductor,l={$L$}]++(\x,0) to [capacitor,l_={$C$}]++(0,-\y) to [short]++(-2*\x,0);
\draw(2*\x+\dx,\y/2)node[right]{$\begin{aligned} &+ \\ & v_C(t) \\ &- \end{aligned}$};
\end{tikzpicture}
\caption{دو درجی دور۔}
\label{شکل_عارضی_دور_درجی_دور_الف}
\end{figure}

شکل \حوالہ{شکل_عارضی_دور_درجی_دور_الف} میں دو درجی دور دکھایا گیا ہے جس کی کرخوف مساوات درج ذیل ہے جہاں \عددی{v_C(0)} لمحہ \عددی{t=0} پر برق گیر کا دباو ہے۔
\begin{gather}
\begin{aligned}\label{مساوات_عارضی_تکمل_و_تفرقی_الف}
v_i(t)&=R i(t)+L\frac{\dif i(t)}{\dif t}+\frac{1}{C} \int_{-\infty}^{t}i(t) \dif t\\
&=R i(t)+L\frac{\dif i(t)}{\dif t}+\frac{1}{C} \int_{0}^{t}i(t) \dif t+v_C(0)
\end{aligned}
\end{gather}
اس \اصطلاح{تکمل و تفرقی مساوات}\فرہنگ{تکمل و تفرقی}\حاشیہب{integro-differential equation}\فرہنگ{integro-differential equation} میں تکمل کی علامت ختم کرنے سے \اصطلاح{تفرقی مساوات}\فرہنگ{تفرقی مساوات}\حاشیہب{differential equation}\فرہنگ{differential equation} حاصل ہو گی۔تکمل کی علامت ختم کرنے کی خاطر اس کا تفرق لیتے ہیں۔
\begin{align}
\frac{\dif v_i(t)}{\dif t}=R \frac{\dif i(t)}{\dif t}+L\frac{\dif^{\,2} i(t)}{\dif t^2}+\frac{i(t)}{C}
\end{align} 
آپ دیکھ سکتے ہیں کہ امالہ گیر اور برق گیر دونوں کی موجودگی سے دو درجی تفرقی مساوات حاصل ہوتی ہے۔

مساوات \حوالہ{مساوات_عارضی_تکمل_و_تفرقی_الف} رو \عددی{i(t)}  کی تکمل و ترقی مساوات ہے۔اس مساوات میں 
\begin{align*}
i(t)&=C\frac{\dif v_C(t)}{\dif t}\\
v_C(t)&=\frac{1}{C}\int_{-\infty}^{t} i(t)\dif t
\end{align*}
پُر کرنے سے دباو \عددی{v_C(t)} کی تفرقی مساوات حاصل ہوتی ہے۔
\begin{align}\label{مساوات_عارضی_تکمل_و_تفرقی_ب}
v_i(t)&=R C\frac{\dif v_C(t)}{\dif t}+LC\frac{\dif^{\,2} i(t)}{\dif t^2}+v_C(t)
\end{align}

\جزوحصہ{رد عمل کی عمومی مساوات}
ایک درجی ادوار کے رد عمل جاننے کی خاطر ان کی تفرقی مساوات حل کی جاتی ہے جس سے دور کے مختلف مقامات پر دباو اور رو حاصل کی جاتی ہے۔ان یک درجی مساوات کی عمومی صورت درج ذیل ہوتی ہے
\begin{align}\label{مساوات_عارضی_یک_درجی_عمومی_مساوات}
\frac{\dif y(t)}{\dif t}+a y(t)= g(t)
\end{align}
جہاں \عددی{y(t)} دباو یا رو کو ظاہر کرتی ہے، \عددی{a} مستقل ہے اور \عددی{g(t)} \اصطلاح{جبری قوت}\فرہنگ{جبری قوت}\حاشیہب{forcing function}\فرہنگ{forcing function} ہے۔اس مساوات کا آزاد متغیرہ وقت \عددی{t} ہے۔تفرقی مساوات کا ایک بنیادی مسئلہ کہتا ہے کہ مساوات \حوالہ{مساوات_عارضی_یک_درجی_عمومی_مساوات} کا عمومی حل اس کے \اصطلاح{فطری رد عمل}\فرہنگ{فطری رد عمل}\فرہنگ{رد عمل!فطری}\حاشیہب{natural response, complementary solution}\فرہنگ{complementary solution}\فرہنگ{natural response}\فرہنگ{response!natural} \عددی{y_f(t)} اور \اصطلاح{جبری رد عمل}\فرہنگ{جبری رد عمل}\فرہنگ{رد عمل!جبری}\حاشیہب{forced response}\فرہنگ{forced response} \عددی{y_j(t)} کا مجموعہ ہے۔مساوات \حوالہ{مساوات_عارضی_یک_درجی_عمومی_مساوات} کے کسی بھی حل کو بطور جبری رد عمل لیا جا سکتا ہے جبکہ درج ذیل \اصطلاح{متجانس مساوات}\فرہنگ{متجانس مساوات}\فرہنگ{مساوات!متجانس}\حاشیہب{homogenous equation}\فرہنگ{homogenous equation}
\begin{align}\label{مساوات_عارضی_یک_درجی_عمومی_مساوات_ب}
\frac{\dif y(t)}{\dif t}+a y(t)=0
\end{align}
 کے کسی بھی حل کو فطری رد عمل تصور کیا جا سکتا ہے۔مساوات \حوالہ{مساوات_عارضی_یک_درجی_عمومی_مساوات} میں \عددی{g(t)=0} پُر کرنے سے متجانس مساوات  حاصل ہوتی ہے۔

آئیں \عددی{g(t)=A} کی صورت میں مساوات \حوالہ{مساوات_عارضی_یک_درجی_عمومی_مساوات} کا حل حاصل کریں جہاں \عددی{A} ایک مستقل ہے۔یوں ہمیں درج ذیل دو مساوات کے حل درکار ہیں۔
\begin{align}
\frac{\dif y_j(t)}{\dif t}+a y_j(t)&=A \label{مساوات_عارضی_یک_درجی_عمومی_مساوات_پ}\\
\frac{\dif y_f(t)}{\dif t}+a y_f(t)&=0\label{مساوات_عارضی_یک_درجی_عمومی_مساوات_ت}
\end{align}
جبری حل کو قیاس کے ذریعہ \عددی{K_1} تصور کرتے ہیں جہاں \عددی{K} ایک مستقل ہے۔
\begin{align}
y_j(t)=K_1
\end{align}
جبری حل \عددی{y_j(t)=K_1} کو مساوات \حوالہ{مساوات_عارضی_یک_درجی_عمومی_مساوات_پ} میں پُر کرتے ہوئے حل کرنے سے
\begin{align*}
\frac{\dif K_1}{\dif t}+a K_1&=A \\
0+a K_1&=A
\end{align*}
یعنی
\begin{align}\label{مساوات_عارضی_یک_درجی_عمومی_مساوات_ٹ}
K_1=\frac{A}{a}
\end{align}
حاصل ہوتا ہے۔مساوات \حوالہ{مساوات_عارضی_یک_درجی_عمومی_مساوات_ت} کو ترتیب دیتے ہوئے
\begin{align*}
\frac{\dif y_f(t)}{y_f(t)}=-a \dif t
\end{align*}
لکھا جا سکتا ہے  جس کا تکمل
\begin{align*}
\ln y_f(t)=-a t +c
\end{align*}
یعنی
\begin{align}\label{مساوات_عارضی_یک_درجی_عمومی_مساوات_ث}
y_f(t)=K_2e^{-at}
\end{align}
کے برابر ہے جہاں \عددی{c} تکمل کا مستقل ہے اور \عددی{K_2=e^{c}} کے برابر ہے۔مساوات \حوالہ{مساوات_عارضی_یک_درجی_عمومی_مساوات_ٹ} اور مساوات \حوالہ{مساوات_عارضی_یک_درجی_عمومی_مساوات_ث} سے عمومی حل درج ذیل حاصل ہوتا ہے۔
\begin{align}
y(t)=\frac{A}{a}+K_2 e^{-at}
\end{align}
کسی بھی لمحے پر \عددی{y(t)} جاننے سے درج بالا مساوات میں نا معلوم مستقل \عددی{K_2} دریافت کیا جا سکتا ہے۔درج بالا مساوات کو درج ذیل عمومی حل کی صورت میں لکھا جا سکتا ہے
\begin{align}\label{مساوات_عارضی_یک_درجی_عمومی_مساوات_ج}
y(t)=K_1+K_2 e^{-\frac{t}{\tau}}
\end{align}
جہاں \عددی{\tau=\tfrac{1}{a}} کے برابر ہے۔


مساوات \حوالہ{مساوات_عارضی_یک_درجی_عمومی_مساوات_ج} کے مختلف اجزاء کو نام دیے گئے ہیں۔یوں \عددی{\tau} \اصطلاح{وقتی مستقل}\فرہنگ{وقتی مستقل}\حاشیہب{time constant}\فرہنگ{time constant} کہلاتا ہے جبکہ \عددی{K_1} \اصطلاح{برقرار حالت حل}\فرہنگ{برقرار حالت حل}\فرہنگ{حل:برقرار حالت}\فرہنگ{برقرار حالت:حل}\حاشیہب{steady state solution}\فرہنگ{steady state solution} کہلاتا ہے۔مساوات \حوالہ{مساوات_عارضی_یک_درجی_عمومی_مساوات_ج} میں \عددی{t=\infty} پُر کرنے سے برقرار حالت حل حاصل ہوتا ہے۔یوں کسی بھی تبدیلی کے بہت دیر بعد دور برقرار حالت میں ہو گا یعنی ابدی صورت کو برقرار حالت کہا جاتا ہے۔

\begin{figure}
\begin{tikzpicture}
\begin{axis}[name=ka,axis lines*=middle,
	 every axis x label/.style={
    at={(ticklabel* cs:1.05)},
    anchor=east,}, 
	every axis y label/.style={
    at={(ticklabel* cs:1.05)},
    anchor=east,}
,xlabel=$t$,ylabel=$K_2 e^{-\frac{t}{\tau}}$,ytick={1,0.368},yticklabel style={/pgf/number format/precision=3},yticklabels={$K_2$,$0.368 K_2$},xtick={0.5,1,1.5,2,2.5},xticklabels={$\tau$,$2\tau$,$3\tau$,$4\tau$,$5\tau$}]
\addplot[width=4cm,domain=0:3,samples=100]{e^(-x/0.5)};
\draw[dashed](axis cs:0,1)--(axis cs:0.5,0);
\draw[dashed](axis cs:0,0.368)--(axis cs:0.5,0.368)--(axis cs:0.5,0);
\end{axis}
\node [anchor=north] at (ka.south){(الف)};
\end{tikzpicture}%
\begin{tikzpicture}
\begin{axis}[name=kb,axis lines*=middle,
 every axis x label/.style={
    at={(ticklabel* cs:1.05)},
    anchor=east,}, 
	every axis y label/.style={
    at={(ticklabel* cs:1.05)},
    anchor=east,},
 xlabel=$t$,ylabel=$e^{-\frac{t}{\tau}}$]
\addplot[width=4cm,domain=0:3,samples=100]{e^(-x/0.5)}node[pos=0.25,above right]{$\tau=0.5$};
\addplot[width=4cm,domain=0:3,samples=100]{e^(-x/2)}node[pos=0.25,above right]{$\tau=2$};
\end{axis}
\node[anchor=north] at (ka.south){(ب)};
\end{tikzpicture}
\caption{وقتی مستقل}
\label{شکل_عارضی_وقتی_مستقل_الف}
\end{figure}

شکل \حوالہ{شکل_عارضی_وقتی_مستقل_الف}-الف میں مثبت \عددی{a} کی صورت میں جبری حل دکھایا گیا ہے۔ابتدائی لمحہ \عددی{t=0} پر \عددی{y_j(0)=K_2} کے برابر ہے جبکہ ایک وقتی مستقل برابر وقت بعد اس کی قیمت \عددی{y_j(\tau)=0.368K_2} رہ گئی ہے یعنی \عددی{\tau} دورانیے میں جبری حل کی قیمت میں \عددی{\SI{63.2}{\percent}} کمی واقع ہوئی ہے۔اسی طرح دو وقتی مستقل وقفے کے بعد \عددی{y_j(2\tau)=0.135K_2} ہے جو \عددی{y_p(\tau)} کے \عددی{0.368} گنا ہے۔حقیقت میں کسی بھی لمحہ \عددی{t_1} پر \عددی{y_j} کی قیمت میں لمحہ \عددی{t_1+\tau} پر \عددی{\SI{63.2}{\percent}} کمی واقع ہو گی۔پانچ وقتی مستقل وقفے کے بعد \عددی{y_j(5\tau)=0.0067K_2} رہ جاتا ہے جو ابتدائی قیمت کے \عددی{\SI{0.67}{\percent}} ہے۔

مساوات \حوالہ{مساوات_عارضی_یک_درجی_عمومی_مساوات_ث}  \اصطلاح{قوت نمائی انحطاطی}\فرہنگ{قوت نمائی!انحطاط}\حاشیہب{exponential decaying}\فرہنگ{exponential decay} خط ہے۔قوت نمائی انحطاطی خط کی ایک خصوصیت یہ ہے کہ ابتدائی لمحے  پر اس کا مماس افقی محور کو \عددی{\tau} پر کاٹتا ہے۔اس مماس کو شکل \حوالہ{شکل_عارضی_وقتی_مستقل_الف}-الف میں \عددی{(0,K_2)} تا \عددی{(\tau,0)} نقطہ دار لکیر سے دکھایا گیا ہے۔ شکل \حوالہ{شکل_عارضی_وقتی_مستقل_الف}-ب میں مختلف \عددی{\tau} کی قیمتوں کے لئے مساوات \حوالہ{مساوات_عارضی_یک_درجی_عمومی_مساوات_ث}  کو کھینچا گیا ہے۔آپ دیکھ سکتے ہیں کہ کم وقتی مستقل کا خط جلد اختتامی قیمت تک پہنچتا ہے۔یوں وقتی مستقل کسی بھی دور کے رد عمل کے دورانیے کی ناپ ہے۔
%=======================

\ابتدا{مثال}\شناخت{مثال_عارضی_یک_درجی_دور_الف}
شکل \حوالہ{شکل_عارضی_سلسلہ_وار_مزاحمت_برق_گیر_الف} میں مزاحمت اور بے بار برق گیر سلسلہ وار جڑے ہیں۔لمحہ \عددی{t=0} پر \اصطلاح{سوئچ}\فرہنگ{سوئچ}\حاشیہد{اس طرز کے سوئچ کا پورا نام ایک قطب ایک چال سوئچ ہے۔}\حاشیہب{switch, spst, single pole single throw}\فرہنگ{switch} چالو کرتے ہوئے انہیں مستقل منبع دباو \عددی{V_I} کے ساتھ جوڑا جاتا ہے۔برق گیر کا دباو \عددی{v(t)} اور رو \عددی{i(t)} دریافت کریں۔

\begin{figure}
\centering
\begin{tikzpicture}
\draw(0,0) to [american voltage source,l={$V_I$}]++(0,\y) to [cspst,l={${t=0}$}]++(\x,0) to [resistor,l={$R$}]++(\x,0)node[above]{$v(t)$} to [capacitor,-*,l={$C$}]++(0,-\y) node[ground]{} to [short] (0,0);
\draw(\x/2+\dx,\y-\dy)node[below]{سوئچ};
\draw(\x,-0.5)node{(الف)};
\end{tikzpicture}%
\begin{tikzpicture}
\begin{axis}[name=kb,axis lines*=middle,
	 every axis x label/.style={
    at={(ticklabel* cs:1.05)},
    anchor=east,}, 
	every axis y label/.style={
    at={(ticklabel* cs:1.05)},
    anchor=east,}
,xlabel=$t$,ylabel=$v(t)$,ytick={1,0.5},yticklabel style={/pgf/number format/precision=3},yticklabels={$V_I$,$0.5 V_I$},xtick={1,2,3,4,5},xticklabels={$\tau$,$2\tau$,$3\tau$,$4\tau$,$5\tau$}]
\addplot[width=4cm,domain=0:5,samples=100]{1-e^(-x)}node[pos=0.3,below right]{$v(t)=V_I \left(1-e^{-\frac{t}{RC}}\right)$};
\end{axis}%
\node [anchor=north] at (kb.south){(ب)};
\end{tikzpicture}%
\begin{tikzpicture}
\begin{axis}[name=kc,axis lines*=middle,
	 every axis x label/.style={
    at={(ticklabel* cs:1.05)},
    anchor=east,}, 
	every axis y label/.style={
    at={(ticklabel* cs:1.05)},
    anchor=east,}
,xlabel=$t$,ylabel=$i(t)$,ytick={1,0.5},yticklabel style={/pgf/number format/precision=3},yticklabels={$\frac{V_I}{R}$,$0.5\frac{V_I}{R}$},xtick={1,2,3,4,5},xticklabels={$\tau$,$2\tau$,$3\tau$,$4\tau$,$5\tau$}]
\addplot[width=4cm,domain=0:5,samples=100]{e^(-x)}node[pos=0.3,above right]{$i(t)=\frac{V_I}{R}e^{-\frac{t}{RC}}$};
\end{axis}%
\node [anchor=north] at (kc.south){(پ)};
\end{tikzpicture}%
\caption{مثال \حوالہ{مثال_عارضی_یک_درجی_دور_الف} کا دور، دباو اور رو۔}
\label{شکل_عارضی_سلسلہ_وار_مزاحمت_برق_گیر_الف}
\end{figure} 

حل:سوئچ چالو کرنے سے پہلے برق گیر بے بار ہے لہٰذا اس پر دباو صفر کے برابر ہے۔صفحہ \حوالہصفحہ{مساوات_امالہ_برق_گیر_دباو_بلا_جوڑ_ہے} پر مساوات \حوالہ{مساوات_امالہ_برق_گیر_دباو_بلا_جوڑ_ہے} کے تحت \عددی{v_C(0_+)=v_C(0_-)} ہو گا یعنی یوں سوئچ چالو کرنے کے فوراً بعد برق گیر کا دباو صفر ہی ہو گا۔سوئچ چالو کرنے کے بعد  دباو جوڑ \عددی{v(t)} کے استعمال سے کرخوف مساوات رو لکھتے ہیں
\begin{align*}
\frac{v(t)-V_I}{R}+C\frac{\dif v(t)}{\dif t}=0
\end{align*}
جسے ترتیب دیتے ہوئے
\begin{align}\label{مساوات_عارضی_برق_گیر_عارضی_حل_الف}
\frac{\dif v(t)}{\dif t}+\frac{v(t)}{RC}=\frac{V_I}{RC}
\end{align}
لکھا  جا سکتا ہے جو عمومی مساوات \حوالہ{مساوات_عارضی_یک_درجی_عمومی_مساوات} کی طرح ہے۔چونکہ \عددی{V_I} مستقل قیمت ہے لہٰذا اس مساوات کا جبری حل
\begin{align*}
v_J(t)=K_1
\end{align*}
 تصور کیا جا سکتا ہے جسے  مساوات \حوالہ{مساوات_عارضی_برق_گیر_عارضی_حل_الف} میں پُر کرتے ہوئے حل کرنے سے
\begin{align*}
\frac{\dif  K_1}{\dif t}+\frac{K_1}{RC}&=\frac{V_I}{RC}\\
0+\frac{K_1}{RC}&=\frac{V_I}{RC}
\end{align*}
یعنی
\begin{align*}
K_1=V_I
\end{align*}
حاصل ہوتا ہے۔یوں جبری حل درج ذیل حاصل ہوتا ہے۔
\begin{align*}
v_J(t)=V_I
\end{align*}
اس نتیجے کے تحت سوئچ چالو کرنے کے بہت دیر بعد برق گیر پر دباو عین منبع دباو کے برابر ہو گا۔شکل کو دیکھتے ہوئے اسی نتیجے تک یوں پہنچا جا سکتا ہے کہ سوئچ چالو کرنے کے بعد دور میں رو کی وجہ سے برق گیر پر بار جمع ہونا شروع ہو جائے گا۔جب تک برق گیر کا دباو منبع کے دباو سے کم ہو، مزاحمت پر دباو پایا جائے گا لہٰذا اس میں رو پائی جائے گی۔یہ رو برق گیر پر جمع بار میں اضافہ کرتی رہے گی۔عین اس وقت جب برق گیر اور منبع کے دباو برابر ہو جائیں، رو کی قیمت صفر ہو جائے گی اور برق گیر کا دباو اسی قیمت پر ابد تک برقرار رہے گا۔ 

آئیں اب فطری حل دریافت کریں۔فطری حل متجانس مساوات سے حاصل ہوتا ہے۔مساوات \حوالہ{مساوات_عارضی_برق_گیر_عارضی_حل_الف} کے دائیں بازو کو صفر کے برابر پُر کرنے سے متجانس مساوات
\begin{align}\label{مساوات_عارضی_برق_گیر_ہم_جنسی_الف}
\frac{\dif v(t)}{\dif t}+\frac{v(t)}{RC}=0
\end{align}
 حاصل ہوتی ہے۔اس کو
\begin{align*}
\frac{\dif v(t)}{v(t)}=-\frac{\dif t}{RC}
\end{align*}
لکھتے ہوئے تکمل لینے سے
\begin{align*}
\ln v(t)=-\frac{t}{RC}+c
\end{align*}
یعنی
\begin{align*}
v_F(t)=K_2 e^{-\frac{t}{RC}}
\end{align*}
فطری حل حاصل ہوتا ہے۔ جبری اور فطری حل کا مجموعہ عمومی حل ہو گا۔
\begin{align*}
v(t)=V_I+K_2 e^{-\frac{t}{RC}}
\end{align*}
عمومی حل میں نا معلوم مستقل کو \اصطلاح{ابتدائی شرائط}\فرہنگ{ابتدائی شرائط}\حاشیہب{initial conditions}\فرہنگ{initial conditions} سے حاصل کرتے ہیں جس کے تحت \عددی{t=0_+} پر \عددی{v_C(0_+)=0} کی قیمت معلوم ہے۔ان قیمتوں کو درج بالا مساوات میں پُر کرتے ہوئے حل کرنے سے
\begin{align*}
0&=V_I+K_2 e^{-\frac{0}{RC}}\\
0&=V_I+K_2
\end{align*}
یعنی
\begin{align*}
K_2=-V_I
\end{align*}
حاصل ہوتا ہے۔ 

جبری حل اور فطری حل کا مجموعہ عمومی حل دیتا ہے
\begin{gather}
\begin{aligned}\label{مساوات_عارضی_مزاحمت_برق_گیر_کا_دباو}
v(t)&=v_J(t)+v_F(t)\\
&=V_I\left(1-e^{-\frac{t}{RC}}\right)\\
&=V_I\left(1-e^{-\frac{t}{\tau}}\right)
\end{aligned}
\end{gather}
درج بالا مساوات میں وقتی مستقل درج ذیل ہے۔
\begin{align}
\tau=RC
\end{align}
یوں \عددی{R} یا (اور) \عددی{C} بڑھانے سے  وقتی مستقل بڑھے گا جس سے دور برقرار صورت زیادہ دیر کے بعد اختیار کرے گا۔ 

مساوات \حوالہ{مساوات_عارضی_مزاحمت_برق_گیر_کا_دباو} کو \عددی{i(t)=C\tfrac{\dif v(t)}{\dif t}} میں  پر کرتے ہوئے رو \عددی{i(t)} حاصل کرتے ہیں۔
\begin{align*}
i(t)&=C\frac{\dif v(t)}{\dif t}\\
&=C V_I \left(0+\frac{1}{RC}e^{-\frac{t}{RC}}\right)\\
&=\frac{V_I}{R}e^{-\frac{t}{RC}}
\end{align*}
یہی رو مزاحمت پر اوہم کے قانون کی مدد سے بھی حاصل کی جا سکتی ہے یعنی
\begin{align*}
i(t)&=\frac{V_I-v(t)}{R}\\
&=\frac{V_I}{R}e^{-\frac{t}{RC}}
\end{align*} 
\انتہا{مثال}
%===============
\ابتدا{مثال}\شناخت{مثال_عارضی_مزاحمت_امالہ_عارضی_الف}
شکل \حوالہ{شکل_عارضی_مزاحمت_امالہ_الف} میں لمحہ \عددی{t=0} پر سوئچ چالو کیا جاتا ہے۔رو کا خط کھینچیں۔
\begin{figure}
\centering
\begin{tikzpicture}
\draw(0,0) to [american voltage source,l={$V_I$}]++(0,\y) to [cspst,l={${t=0}$}]++(\x,0) to [resistor,i>_={$i(t)$},l={$R$}]++(\x,0) to [inductor,-*,l={$L$}]++(0,-\y)node[ground]{} to [short] (0,0);
\end{tikzpicture}%
\begin{tikzpicture}
\begin{axis}[name=kb,xlabel=$t$,ylabel=$i(t)$,ytick={1,0.5},yticklabels={$\frac{V_I}{R}$,$0.5\frac{V_I}{R}$},xtick={1,2,3,4,5},xlabel={$t$},xticklabels={$\tau$,$2\tau$,$3\tau$,$4\tau$,$5\tau$},
axis lines*=middle,
 every axis x label/.style={
    at={(ticklabel* cs:1.05)},
    anchor=east,}, 
	every axis y label/.style={
    at={(ticklabel* cs:1.05)},
    anchor=east,}
]
\addplot[domain=0:5]{1-e^(-x)};
\end{axis}%
\end{tikzpicture}%
\caption{مثال \حوالہ{مثال_عارضی_مزاحمت_امالہ_عارضی_الف} کے اشکال۔}
\label{شکل_عارضی_مزاحمت_امالہ_الف}
\end{figure}

حل:کرخوف مساوات دباو
\begin{align*}
V_I=i(t)R+L\frac{\dif i(t)}{\dif t}
\end{align*}
کو ترتیب دیتے ہوئے عمومی شکل میں لاتے ہیں
\begin{align}\label{مساوات_عارضی_امالہ_گیر_عمومی_الف}
\frac{\dif i(t)}{\dif t}+\frac{R}{L} i(t)=\frac{V_I}{L}
\end{align}
جس کا جبری حل
\begin{align*}
i_J(t)=K_1
\end{align*}
ہو گا۔جبری حل کو عمومی مساوات میں پُر کرتے ہوئے حل کرنے سے
\begin{align*}
\frac{\dif K_1}{\dif t}+\frac{R}{L} K_1&=\frac{V_I}{L}\\
0+\frac{R}{L} K_1&=\frac{V_I}{L}
\end{align*}
یعنی
\begin{align*}
K_1=\frac{V_I}{R}
\end{align*}
حاصل ہوتا ہے جس سے جبری حل درج ذیل لکھا جائے گا۔
\begin{align*}
i_J(t)=\frac{V_I}{R}
\end{align*}
یہی جواب منطق سے بھی حاصل کیا جا سکتا ہے۔چونکہ  یک سمتی رو کے لئے امالہ گیر بطور قصر دور کردار ادا کرتا ہے لہٰذا عارضی دورانیہ گزر جانے کے بعد ہم امالہ گیر کو قصر دور تصور کر سکتے ہیں۔شکل \حوالہ{شکل_عارضی_مزاحمت_امالہ_الف} میں امالہ گیر کو قصر دور کرتے ہوئے اوہم کے قانون سے \عددی{i_J(t)=\tfrac{V_I}{R}} لکھا جا سکتا ہے۔

فطری حل حاصل کرنے کی خاطر مساوات \حوالہ{مساوات_عارضی_امالہ_گیر_عمومی_الف} میں دیے گئے عمومی مساوات کا دایاں ہاتھ صفر کے برابر پُر کرتے ہوئے درج ذیل متجانس مساوات حاصل کرتے ہیں۔
\begin{align*}
\frac{\dif i(t)}{\dif t}+\frac{R}{L} i(t)=0
\end{align*}
اس کو ترتیب دیتے ہوئے
\begin{align*}
\frac{\dif i(t)}{i(t)}=-\frac{R}{L} \dif t
\end{align*}
تکمل لینے سے
\begin{align*}
\ln i(t)=-\frac{R}{L}t +c
\end{align*}
یعنی
\begin{align*}
i_F(t)=K_2e^{-\frac{R}{L}t}
\end{align*}
حاصل ہوتا ہے۔

جبری اور فطری حل کا مجموعہ عمومی حل دیتا ہے
\begin{gather}
\begin{aligned}\label{مساوات_عارضی_مزاحمت_امالہ_مکمل_حل_الف}
i(t)&=i_J(t)+i_F(t)\\
&=\frac{V_I}{R}+K_2e^{-\frac{R}{L}t}\\
&=\frac{V_I}{R}+K_2e^{-\frac{t}{\tau}}
\end{aligned}
\end{gather}
جہاں وقتی مستقل درج ذیل ہے۔
\begin{align}
\tau=\frac{R}{L}
\end{align}
عمومی حل میں نا معلوم مستقل \عددی{K_2} کو ابتدائی معلومات سے حاصل کیا جا سکتا ہے۔سوئچ چالو کرنے سے پہلے دور میں رو صفر کے برابر ہے۔صفحہ \حوالہصفحہ{مساوات_امالہ_امالہ_گیر_رو_بلا_جوڑ_ہے} پر مساوات \حوالہ{مساوات_امالہ_امالہ_گیر_رو_بلا_جوڑ_ہے} کے تحت امالہ کی رو بلا جوڑ تفاعل
\begin{align*}
i_L(t_+)=i_L(t_-)
\end{align*} 
ہے لہٰذا سوئچ چالو کرنے کے فوراً بعد امالہ کی رو وہی ہو گی جو سوئچ چالو کرنے کے فوراً پہلے تھی یعنی لمحہ \عددی{t=0_+} پر \عددی{i_L(0_+)=i_L(0_-)=0} ہو گی۔ان معلومات کو مساوات \حوالہ{مساوات_عارضی_مزاحمت_امالہ_مکمل_حل_الف} میں دیے عمومی حل میں پُر کرنے سے
\begin{align*}
0&=\frac{V_I}{R}+K_2e^{-\frac{0}{\tau}}
\end{align*}
یعنی
\begin{align*}
K_2=-\frac{V_I}{R}
\end{align*}
حاصل ہوتا ہے۔عمومی حل میں ابتدائی معلومات سے حاصل کردہ مستقل پر کرنے سے درج ذیل \اصطلاح{مخصوص حل}\فرہنگ{مخصوص حل}\حاشیہب{particular solution}\فرہنگ{particular solution} حاصل ہوتا ہے ۔
\begin{align}
i(t)=\frac{V_I}{R} \left(1-e^{-\frac{t}{\tau}}\right)
\end{align}
رو کے خط کو شکل \حوالہ{شکل_عارضی_مزاحمت_امالہ_الف}-ب میں دکھایا گیا ہے۔
\انتہا{مثال}
%=====================
\ابتدا{مثال}\شناخت{مثال_عارضی_دو_عدد_مزاحمت_برق_گیر_الف}
ازل سے شکل \حوالہ{شکل_عارضی_دو_عدد_مزاحمت_برق_گیر_الف} میں \اصطلاح{ایک قطب دو چال سوئچ}\فرہنگ{سوئچ!ایک قطب دو چال}\حاشیہب{single pole double throw switch, spdt}\فرہنگ{switch!spdt} اسی جگہ پر ہے۔لمحہ \عددی{t=0} پر اس کی جگہ تبدیل کرتے ہوئے \عددی{\SI{5}{\kilo\ohm}} مزاحمت کو زمین کے ساتھ جوڑا جاتا ہے۔برق گیر پر دباو دریافت کریں۔
\begin{figure}
\centering
\begin{tikzpicture}
\draw(0,0) to [american voltage source,l={$\SI{20}{\volt}$}]++(0,\y)coordinate(kP)++(\x,0) node[spdt,xscale=-1](kspdt){};
\draw[name path=kkk](kspdt.in) to [resistor,l={$\SI{5}{\kilo\ohm}$}]++(\x,0) to [short]++(\x,0) to [capacitor,l={$\SI{200}{\micro\farad}$}]++(0,-\y)coordinate(kbot)-|(0,0);
\draw(kP)|-(kspdt.out 1);
\draw(kbot)++(-\x,0)node[ground]{} to [resistor,*-*,l={$\SI{15}{\kilo\ohm}$}]++(0,\y);
\path[name path=kvert](kspdt.out 2)--++(0,-\y);
\draw[name intersections={of=kvert and kkk}] (kspdt.out 2) to [short,-*](intersection-1);
\draw(\x,-\y/4)node{(الف)};
\end{tikzpicture}%
\begin{tikzpicture}
\draw(0,0) to [american voltage source,l={$\SI{20}{\volt}$}]++(0,\y)coordinate(kP)++(\x,0) node[spdt,xscale=-1](kspdt){};
\draw[](kspdt.in) to [resistor,l={$\SI{5}{\kilo\ohm}$}]++(\x,0) to [short]++(\x,0) to [short,-o]++(0,-\y/8)++(0,-6/8*\y) to [short,o-]++(0,-\y/8)coordinate(kbot);
\draw[name path=kkk](0,0)--++(kbot);
\draw(kP)|-(kspdt.out 1);
\draw(kbot)++(-\x,0)node[ground]{} to [resistor,*-*,l={$\SI{15}{\kilo\ohm}$}]++(0,\y);
\path[name path=kvert](kspdt.out 2)--++(0,-\y);
\draw[name intersections={of=kvert and kkk}] (kspdt.out 2) to [short,-*](intersection-1);
\draw(kbot)++(0,\y/2)node{$\begin{aligned}&+ \\ &v_C \\ &-  \end{aligned}$};
\draw(\x,-\y/4)node{(ب)};
\end{tikzpicture}%
\begin{tikzpicture}
\draw(0,0) to [american voltage source,l={$\SI{20}{\volt}$}]++(0,\y)coordinate(kP)++(\x,0) node[spdt,xscale=-1,yscale=-1](kspdt){};
\draw[name path=kkk](kspdt.in) to [resistor,l={$\SI{5}{\kilo\ohm}$}]++(\x,0) to [short]++(\x,0)node[above]{$v_C(t)$} to [capacitor,l={$\SI{200}{\micro\farad}$}]++(0,-\y)coordinate(kbot)-|(0,0);
\draw(kP)|-(kspdt.out 2);
\draw(kbot)++(-\x,0)node[ground]{} to [resistor,*-*,l={$\SI{15}{\kilo\ohm}$}]++(0,\y);
\path[name path=kvert](kspdt.out 1)--++(0,-\y);
\draw[name intersections={of=kvert and kkk}] (kspdt.out 1) to [short,-*](intersection-1);
\draw(\x,-\y/4)node{(پ)};
\end{tikzpicture}%
\begin{tikzpicture}
\begin{axis}[axis lines*=middle,ytick={5,10},xlabel=$t$,ylabel=$v_C(t)$,
every axis x label/.style={
    at={(ticklabel* cs:1.05)},
    anchor=east,}, 
	every axis y label/.style={
    at={(ticklabel* cs:1.05)},
    anchor=east,}
]
\addplot[domain=0:6,samples=100]{15*e^(-4*x/3)};
\addplot[domain=-1:0,samples=10]{15}node[right]{$\SI{15}{\volt}$};
\end{axis}
\end{tikzpicture}
\caption{مثال \حوالہ{مثال_عارضی_دو_عدد_مزاحمت_برق_گیر_الف} کے اشکال۔}
\label{شکل_عارضی_دو_عدد_مزاحمت_برق_گیر_الف}
\end{figure}

حل:ازل سے دور منبع کے ساتھ جڑا رہا ہے۔یوں دور برقرار حالت میں ہو گا اور برق گیر کو کھلا دور تصور کیا جاتا ہے۔ایسا کرنے سے شکل-ب حاصل ہوتی ہے جہاں سے تقسیم دباو کے کلیے سے برق گیر کا ابتدائی دباو درج ذیل حاصل ہوتا ہے۔
\begin{align*}
v_C(0_-)=20\left(\frac{\SI{15}{\kilo\ohm}}{\SI{5}{\kilo\ohm}+\SI{15}{\kilo\ohm}}\right)=\SI{15}{\volt}
\end{align*}
برق گیر کا دباو بلا جوڑ ہے لہٰذا
\begin{align*}
v_C(0_+)=v_C(0_-)=\SI{15}{\volt} \quad \quad \text{\RL{ابتدائی حالت}}
\end{align*}
ہو گا۔لمحہ \عددیء{t=0} کے بعد کی صورت شکل-پ میں دکھائی گئی ہے۔ہمیں اس شکل میں \عددی{v(t)} درکار ہے جسے کرخوف مساوات رو کی مدد سے حاصل کرتے ہیں۔
\begin{align*}
\frac{v_C(t)}{5000}+\frac{v_C(t)}{15000}+200\times 10^{-6}\frac{\dif v_C(t)}{\dif t}=0
\end{align*}
اس متجانس مساوات کو ترتیب دیتے ہوئے
\begin{align*}
\frac{\dif v_C(t)}{v_C(t)}=-\frac{4}{3}\dif t
\end{align*}
لکھا جا سکتا ہے جس کا تکمل
\begin{align*}
\ln v_C(t)=-\frac{4}{3}t+c
\end{align*}
یا
\begin{align*}
v_C(t)=Ke^{-\frac{4}{3}t}
\end{align*}
کے برابر ہے جہاں تکمل کے مستقل کو \عددی{c} یا \عددی{K} لکھا گیا ہے۔ابتدائی حالت کی معلومات اس مساوات میں پُر کرتے ہوئے 
\begin{align*}
15=Ke^{0}
\end{align*}
سے \عددی{K} کی قیمت درج ذیل
\begin{align*}
K=15
\end{align*}
حاصل ہوتی ہے۔یوں 
\begin{align*}
v_C(t)=15 e^{-\frac{4}{3}t}
\end{align*}
حاصل ہوتا ہے جس میں وقتی مستقل \عددی{\tau=\tfrac{3}{4}} کے برابر ہے۔یوں سوئچ چالو کرنے کے \عددی{\SI{0.75}{\second}} بعد برق گیر کا دباو ابتدائی قیمت کے \عددی{\SI{36.8}{\percent}} یعنی \عددی{0.368 \times 15=\SI{5.52}{\volt}} ہو گا۔ 
\انتہا{مثال}
%======================
\ابتدا{مثال}\شناخت{مثال_عارضی_دو_عدد_مزاحمت_امالہ_گیر_الف}
ازل سے شکل \حوالہ{شکل_عارضی_دو_عدد_مزاحمت_امالہ_گیر_الف} میں سوئچ غیر چالو تھا جسے \عددی{t=0} پر چالو کیا جاتا ہے۔امالہ گیر کی رو \عددی{i_L(t)} دریافت کریں۔

\begin{figure}
\centering
\begin{subfigure}{0.5\textwidth}
\centering
\begin{tikzpicture}
\draw(0,0) to [american voltage source,l={$\SI{24}{\volt}$}]++(0,\y) to [resistor,l={$\SI{1}{\kilo\ohm}$}]++(\x,0) to [cspst,l={${t=0}$}]++(\x,0) to [resistor,l={$\SI{1}{\kilo\ohm}$}]++(\x,0) to [inductor,i={$i_L(t)$},l={$\SI{1}{\milli\henry}$}]++(0,-\y) to [short] (0,0);
\draw(2*\x,0) to [american current source,*-*,l={$\SI{4}{\milli\ampere}$}]++(0,\y);
\end{tikzpicture}
\caption*{(الف)}
\end{subfigure}
\begin{subfigure}{0.5\textwidth}
\centering
\begin{tikzpicture}
\draw(0,0) to [american voltage source,l={$\SI{24}{\volt}$}]++(0,\y) to [resistor,l={$\SI{1}{\kilo\ohm}$}]++(\x,0) to [resistor,-o,l={$\SI{1}{\kilo\ohm}$}]++(\x,0);
\draw(0,0) to [short,-o]++(2*\x,0);
% to [inductor,i={$i_L(t)$},l={$\SI{1}{\milli\henry}$}]++(0,-\y) to [short] (0,0);
\draw(\x,0) to [american current source,*-*,l={$\SI{4}{\milli\ampere}$}]++(0,\y)node[above]{$v_t$};
\draw(2*\x,\y/2)node{$\begin{aligned} &+ \\ &v_{\text{تھونن}}\\ &- \end{aligned}$};
\end{tikzpicture}%
\caption*{(ب)}
\end{subfigure}%
\begin{subfigure}{0.5\textwidth}
\centering
\begin{tikzpicture}
\draw(0,0) to [short]++(0,\y) to [resistor,l={$\SI{1}{\kilo\ohm}$}]++(\x,0) to [resistor,-o,l={$\SI{1}{\kilo\ohm}$}]++(\x,0);
\draw(0,0) to [short,-o]++(2*\x,0);
% to [inductor,i={$i_L(t)$},l={$\SI{1}{\milli\henry}$}]++(0,-\y) to [short] (0,0);
%\draw(\x,0) to [american current source,*-*,l={$\SI{4}{\milli\ampere}$}]++(0,\y);
\draw[stealth-](2*\x,\y/2)--++(\x/4,0)--++(0,-\y/8)node[below]{$R_{\text{تھونن}}$};
\end{tikzpicture}%
\caption*{(پ)}
\end{subfigure}
\begin{subfigure}{0.5\textwidth}
\centering
\begin{tikzpicture}
\draw(0,0) to [american voltage source,l={$\SI{28}{\volt}$}]++(0,\y) to [resistor,l={$\SI{2}{\kilo\ohm}$}]++(\x,0) to [inductor,i={$i_L(t)$},l={$\SI{1}{\milli\henry}$}]++(0,-\y) --(0,0);
\end{tikzpicture}%
\caption*{(ت)}
\end{subfigure}%
\begin{subfigure}{0.5\textwidth}
\centering
\pgfplotsset{scaled y ticks=false, scaled x ticks=false}
\begin{tikzpicture}
\begin{axis}[axis lines*=middle,
ytick={0.014},yticklabels={$\SI{14}{\milli\ampere}$},xlabel=$t(\si{\micro\second})$,ylabel=$i_L(t)$,
xtick={-0.000001,0.000001,0.000002,0.000003,0.000004,0.000005}, xticklabels={$-1$,$1$,$2$,$3$,$4$,$5$},
every axis x label/.style={
    at={(ticklabel* cs:1.15)},
    anchor=east,}, 
	every axis y label/.style={
    at={(ticklabel* cs:1.05)},
    anchor=east,}
]
\addplot[domain=0:3*10^(-6),samples=20]{0.014-0.01*e^(-2000000*x)};
\addplot[domain=-1*10^(-6):0,samples=10]{0.004}node[right]{$\SI{4}{\milli\ampere}$};
\end{axis}
\end{tikzpicture}%
\caption*{(ٹ)}
\end{subfigure}%
\caption{مثال \حوالہ{مثال_عارضی_دو_عدد_مزاحمت_امالہ_گیر_الف} کے اشکال۔}
\label{شکل_عارضی_دو_عدد_مزاحمت_امالہ_گیر_الف}
\end{figure}

حل:غیر چالو سوئچ کی صورت میں منبع رو کی تمام رو امالہ گیر سے گزرتی ہے لہٰذا
\begin{align*}
i_L(0_-)=i_L(0_+)=\SI{4}{\milli\ampere}
\end{align*}
ہو گا۔اس دور کو مسئلہ تھونن کی مدد سے حل کرتے ہیں۔یوں امالہ کو بوجھ تصور کرتے ہوئے بقایا دور کا تھونن مساوی حاصل کرتے ہیں۔تھونن دباو حاصل کرنے کی خاطر بوجھ کو کھلے دور کیا جاتا ہے جس سے شکل \حوالہ{شکل_عارضی_دو_عدد_مزاحمت_امالہ_گیر_الف}-ب حاصل ہوتی ہے۔اس شکل میں منبع رو کی تمام رو بائیں مزاحمت اور منبع دباو سے گزرے گی لہٰذا مزاحمت پر \عددی{\SI{4}{\volt}} کا دباو ہو گا۔یوں
\begin{align*}
v_t=v_{\text{تھونن}}=\SI{24}{\volt}+\SI{4}{\volt}=\SI{28}{\volt}
\end{align*}
لکھا جا سکتا ہے۔یاد رہے کہ بالائی دائیں مزاحمت میں رو صفر کے برابر ہے لہٰذا اس پر دباو بھی صفر ہو گا اور یوں \عددی{v_t} اور \عددی{v_{\text{تھونن}}} برابر ہوں گے۔

منبع دباو کو قصر دور اور منبع رو کو کھلے دور کرتے ہوئے شکل-پ حاصل ہوتی ہے جسے دیکھتے ہوئے تھونن مزاحمت
\begin{align*}
R_{\text{تھونن}}=\SI{2}{\kilo\ohm}
\end{align*}
لکھی جا سکتی ہے۔

تھونن مساوی دور استعمال کرتے ہوئے شکل-الف کو شکل-ت کی طرز پر بنایا جا سکتا ہے۔شکل-ت کی کرخوف مساوات
\begin{align*}
28=2000 i(t)+0.001 \frac{\dif i(t)}{\dif t}
\end{align*}
کو عمومی صورت میں لکھتے ہیں۔
\begin{align*}
\frac{\dif i(t)}{\dif t}+2\times 10^6 i(t)=28000
\end{align*}
اس مساوات کا جبری حل 
\begin{align*}
i_J(t)=K_1=\SI{14}{\milli\ampere}
\end{align*}
حاصل ہوتا ہے اور اس کا فطری حل
\begin{align*}
i_F(t)=K_2 e^{-2\times 10^6 t}
\end{align*}
ہے-یوں امالہ گیر کے رو کا عمومی حل
\begin{align*}
i(t)=0.014+K_2 e^{-2\times 10^6 t}
\end{align*}
ہے۔ابتدائی معلومات کو اس مساوات میں پر کرتے ہوئے
\begin{align*}
0.004=0.014+K_2 e^{0}
\end{align*}
سے
\begin{align*}
K_2=\SI{-10}{\milli\ampere}
\end{align*}
حاصل ہوتا ہے۔یوں مخصوص حل درج ذیل ہے۔
\begin{align}\label{مساوات_عارضی_امالہ_گیر_مکمل_حل_ب}
i_L(t)=0.014-0.01e^{-2\times 10^6 t}
\end{align} 
اس مساوات کا وقتی مستقل \عددی{\tau=\SI{0.5}{\micro\second}} ہے۔یوں تقریباً \عددی{5\tau=\SI{2.5}{\micro\second}} میں دور پہلی برقرار حالت سے دوسری برقرار حالت اختیار کر پاتا ہے۔مساوات \حوالہ{مساوات_عارضی_امالہ_گیر_مکمل_حل_ب} کو شکل-ٹ میں دکھایا گیا ہے۔
\انتہا{مثال}
%=======================
\ابتدا{مشق}\شناخت{مشق_عارضی_برق_گیر_الف}
شکل \حوالہ{شکل_عارضی_مشق_برق_گیر_الف} میں ازل سے چالو سوئچ کو  لمحہ \عددی{t=0} پر منقطع کیا جاتا ہے۔برق گیر پر ابتدائی دباو دریافت کرتے ہوئے \عددی{v_0(t)} دریافت کریں۔ اس دور کا وقتی مستقل کیا ہے۔
\begin{figure}
\centering
\begin{tikzpicture}
\draw(0,0) to [american voltage source,l={$\SI{24}{\volt}$}]++(0,\y) to [resistor,l={$\SI{2}{\kilo\ohm}$}]++(\x,0) to [ospst,l={${t=0}$}] ++(\x,0) to [resistor,l={$\SI{6}{\kilo\ohm}$}]++(\x,0) to [resistor,l_={$\SI{4}{\kilo\ohm}$}]++(0,-\y) to [short] (0,0);
\draw(2*\x,0) to [capacitor,*-*,l={$\SI{50}{\micro\farad}$}]++(0,\y);
\draw(3*\x+\dx,\y/2)node[right]{$\begin{aligned}&+ \\ &v_0(t) \\ &- \end{aligned}$};
\end{tikzpicture}
\caption{مشق \حوالہ{مشق_عارضی_برق_گیر_الف} کا دور۔}
\label{شکل_عارضی_مشق_برق_گیر_الف}
\end{figure}

جوابات:\عددی{v_C(0_+)=\SI{20}{\volt}}، \عددی{v_0(t)=8 e^{-\frac{t}{0.5}} \, \si{\volt}}، \عددی{\tau=\SI{0.5}{\second}}
\انتہا{مشق}
%===================

\ابتدا{مشق}\شناخت{مشق_عارضی_برق_گیر_ب}
شکل \حوالہ{شکل_عارضی_مشق_برق_گیر_ب} میں ازل سے چالو سوئچ کو  لمحہ \عددی{t=0} پر منقطع کیا جاتا ہے۔برق گیر پر ابتدائی دباو دریافت کرتے ہوئے \عددی{v_0(t)} دریافت کریں۔
\begin{figure}
\centering
\begin{tikzpicture}
\draw(0,0) to [american voltage source,l={$\SI{12}{\volt}$}]++(0,\y) to [resistor,l={$\SI{2}{\kilo\ohm}$}]++(\x,0) to [ospst,l={${t=0}$}] ++(\x,0) to [short]++(2*\x,0) to [capacitor,l_={$\SI{10}{\micro\farad}$}]++(0,-\y) to [short] (0,0);
\draw(\x,0) to [resistor,*-*,l={$\SI{4}{\kilo\ohm}$}]++(0,\y);
\draw(3*\x,0) to [resistor,*-*,l={$\SI{8}{\kilo\ohm}$}]++(0,\y);
\draw(2*\x,0) to [american current source,*-*,l={$\SI{4}{\milli\ampere}$}]++(0,\y);
\draw(4*\x+2*\dx,\y/2)node[right]{$\begin{aligned}&+ \\ &v_0(t) \\ &- \end{aligned}$};
\end{tikzpicture}
\caption{مشق \حوالہ{مشق_عارضی_برق_گیر_ب} کا دور۔}
\label{شکل_عارضی_مشق_برق_گیر_ب}
\end{figure}

جوابات:\عددی{v_0(0_+)=\frac{80}{7} \, \si{\volt}}، \عددی{v_0(t)=32-\frac{144}{7}e^{-\frac{100t}{7}} \, \si{\volt}}
\انتہا{مشق}
%===================


\ابتدا{مشق}\شناخت{مشق_عارضی_برق_گیر_پ}
شکل \حوالہ{شکل_عارضی_مشق_برق_گیر_پ} میں ازل سے چالو سوئچ کو  لمحہ \عددی{t=0} پر منقطع کیا جاتا ہے۔امالہ گیر میں ابتدائی رو دریافت کرتے ہوئے \عددی{i_L(t)} دریافت کریں۔دور کا وقتی مستقل  حاصل کریں۔
\begin{figure}
\centering
\begin{tikzpicture}
\draw(0,0) to [american voltage source,l={$\SI{10}{\volt}$}]++(0,2*\y) to [ospst,l={${t=\SI{0}{\second}}$}]++(\x,0) to [short] ++(\x,0) to [short]++(\x,0) to [resistor,l_={$\SI{10}{\ohm}$}]++(0,-2*\y) to [short] (0,0);
\draw(\x,0) to [resistor,*-,l={$\SI{8}{\ohm}$}]++(0,\y) to [inductor,i<_={$i_L$},-*,l={$\SI{4}{\henry}$}]++(0,\y);
\end{tikzpicture}
\caption{مشق \حوالہ{مشق_عارضی_برق_گیر_پ} کا دور۔}
\label{شکل_عارضی_مشق_برق_گیر_پ}
\end{figure}

جوابات:\عددی{i_L(0_+)=\SI{1.25}{\ampere}}، \عددی{i_L(t)=1.25e^{-3000t} \, \si{\ampere}}، \عددی{\tau=\tfrac{1}{3} \, \si{\milli\second}}
\انتہا{مشق}
%===================
\ابتدا{مثال}\شناخت{مثال_عارضی_برق_گیر_وقت_صفر_نہیں_الف}
شکل \حوالہ{شکل_عارضی_برق_گیر_وقت_صفر_نہیں_الف} میں ازل سے چالو سوئچ لمحہ \عددی{t=\SI{2}{\second}} پر منقطع کیا جاتا ہے۔رو \عددی{i(t)} دریافت کریں۔

\begin{figure}
\centering
\begin{subfigure}{0.5\textwidth}
\centering
\begin{tikzpicture}
\draw(0,0) to [american voltage source,l={$\SI{10}{\volt}$}]++(0,\y) to [resistor,l={$\SI{2}{\kilo\ohm}$}]++(\x,0) to [resistor,i<^={$i(t)$},l={$\SI{4}{\kilo\ohm}$}]++(\x,0) to [resistor,l={$\SI{6}{\kilo\ohm}$}]++(\x,0);
\draw(0,0) to [short]++(3*\x,0) to [american voltage source,l_={$\SI{20}{\volt}$}]++(0,\y);
\draw(\x,0) to [ospst,*-*,l={${t=\SI{2}{\second}}$}]++(0,\y);
\draw(2*\x,0) node[ground]{}to [capacitor,*-*,l={$\SI{5}{\micro\farad}$}]++(0,\y)node[above]{$v(t)$};
\end{tikzpicture}
\caption*{(الف)}
\end{subfigure}
\begin{subfigure}{0.5\textwidth}
\centering
\begin{tikzpicture}
\draw(0,0) to [american voltage source,l={$\SI{10}{\volt}$}]++(0,\y) to [resistor,l={$\SI{2}{\kilo\ohm}$}]++(\x,0) to [resistor,i<^={$i(t)$},l={$\SI{4}{\kilo\ohm}$}]++(\x,0) to [resistor,l={$\SI{6}{\kilo\ohm}$}]++(\x,0);
\draw(0,0) to [short]++(3*\x,0) to [american voltage source,l_={$\SI{20}{\volt}$}]++(0,\y);
\draw(\x,0) to [short,*-*]++(0,\y);
\draw(2*\x,0)node[ground]{} to [short,*-o]++(0,\y/8);
\draw(2*\x,\y) to [short,*-o]++(0,-\y/8);
\draw(2*\x+2*\dx,\y/2)node[]{$\begin{aligned} &+ \\ &v_C(2_-) \\ &- \end{aligned}$};
\end{tikzpicture}
\caption*{(ب)}
\end{subfigure}
\begin{subfigure}{0.5\textwidth}
\centering
\pgfplotsset{scaled y ticks=false, scaled x ticks=false}
\begin{tikzpicture}
\begin{axis}[axis lines*=middle,xlabel={$t$},ylabel={$i(t)$},ytick={0.002,-0.00033,0.000833},yticklabels={$\SI{2}{\milli\ampere}$,$\SI{-0.33}{\milli\ampere}$,${\frac{5}{6}\,\si{\milli\ampere}}$},
every axis x label/.style={
    at={(ticklabel* cs:1.15)},
    anchor=east,}, 
	every axis y label/.style={
    at={(ticklabel* cs:1.05)},
    anchor=east,}
]
\addplot[domain=1.99:2,samples=10]{0.002};
\addplot[domain=2:2.075,samples=100]{5/6000-7/6000*e^(200/3*(2-x))};
\draw(axis cs:2,0.002)--(axis cs:2,-1/3000);
\end{axis}
\end{tikzpicture}
\caption*{(پ)}
\end{subfigure}
\caption{مثال \حوالہ{مثال_عارضی_برق_گیر_وقت_صفر_نہیں_الف} کے اشکال۔}
\label{شکل_عارضی_برق_گیر_وقت_صفر_نہیں_الف}
\end{figure}

حل:سوئچ منقطع کرنے سے فوراً پہلے کی صورت حال شکل-ب میں دکھائی گئی ہے۔چونکہ ازل سے سوئچ چالو تھا لہٰذا دور برقرار حالت میں ہو گا اور یوں برق گیر کو کھلا دور تصور کیا جائے گا۔شکل-ب کو دیکھ کر
\begin{align*}
i(t<\SI{2}{\second})=\frac{20}{4000+6000}=\SI{2}{\milli\ampere}
\end{align*}
اور
\begin{align*}
v_C(2_-)=v_C(2_+)=20\left(\frac{4000}{4000+6000}\right)=\SI{8}{\volt}
\end{align*}
لکھا جا سکتا ہے۔سوئچ منقطع ہونے کے بعد کی صورت حال شکل-الف میں دی گئی ہے۔جوڑ \عددی{v(t)} پر کرخوف مساوات رو لکھتے ہوئے
\begin{align*}
\frac{v(t)-10}{2000+4000}+5\times 10^{-6}\frac{\dif v(t)}{\dif t}+\frac{v(t)-20}{6000}=0
\end{align*}
ترتیب دینے سے
\begin{align*}
\frac{\dif v(t)}{\dif t}+\frac{200}{3} v(t)=1000
\end{align*}
حاصل ہوتا ہے۔اس کے جبری اور فطری  حل درج ذیل ہیں
\begin{align*}
v_J(t)&=K_1=\SI{15}{\volt}\\
v_F(t)&=K_2e^{-\frac{200}{3}t}
\end{align*}
جن کا مجموعہ عمومی حل
\begin{align*}
v(t>2)=15+K_2e^{-\frac{200}{3}t}
\end{align*}
 دیتا ہے۔ابتدائی معلومات  \عددی{v(2_+)=\SI{8}{\volt}}لمحہ \عددی{t=\SI{2}{\second}} پر ہم جانتے ہیں جنہیں درج بالا مساوات میں پُر کرتے ہوئے
\begin{align*}
8=15+K_2e^{-\frac{200}{3}\times 2}
\end{align*}
\عددی{K_2} کی قیمت درج ذیل حاصل ہوتی ہے۔
\begin{align*}
K_2=-7e^{\frac{400}{3}}
\end{align*}
یوں مخصوص حل درج ذیل ہو گا۔
\begin{align*}
v(t>2)=15-7e^{\frac{200}{3}(2-t)}
\end{align*}
اب شکل-الف کو دیکھ کر
\begin{align*}
i(t>2)&=\frac{v(t>2)-10}{6000}\\
&=\frac{5}{6}-\frac{7}{6} e^{\frac{200}{3}(2-t)} \, \si{\milli\ampere}
\end{align*}
لکھا جا سکتا ہے جو درکار مساوات ہے۔یوں سوئچ منقطع کرنے سے پہلے اور اس کے بعد کے جوابات سے درج ذیل لکھا جا سکتا ہے
\begin{align*}
i(t)=
\begin{cases}
\SI{2}{\milli\ampere} & t<\SI{2}{\second}\\
\frac{5}{6}-\frac{7}{6} e^{\frac{200}{3}(2-t)} \, \si{\milli\ampere} & t>\SI{2}{\second}
\end{cases}
\end{align*}
جسے شکل-پ میں دکھایا گیا ہے جہاں سے آپ دیکھ سکتے ہیں کہ سوئچ منقطع کرنے سے پہلے برقرار رو \عددی{\SI{2}{\milli\ampere}} تھی جبکہ سوئچ منقطع کرنے کے بعد برقرار حالت \عددی{(t \to \infty)} میں رو \عددی{\tfrac{5}{6}\,\si{\milli\ampere}} ہے۔یاد رہے کہ برق گیر کا دباو فوراً تبدیل نہیں ہو سکتا البتہ اس میں رو یک دم تبدیل ہو سکتی ہے۔

وقت \عددی{t \to \infty} پر دور برقرار حالت اختیار کر چکا ہو گا لہٰذا برق گیر کو کھلا دور کرتے ہوئے شکل \حوالہ{شکل_عارضی_برق_گیر_وقت_صفر_نہیں_الف}-الف سے برقرار حالت  رو درج ذیل لکھی جا سکتی ہے۔
\begin{align*}
i(t\to\infty)=\frac{20-10}{2000+4000+6000}=\frac{5}{6} \, \si{\milli\ampere}
\end{align*}


\انتہا{مثال}
%=======================
\ابتدا{مثال}\شناخت{مثال_عارضی_امالہ_گیر_سوئچ_چالو}
شکل \حوالہ{شکل_عارضی_امالہ_گیر_سوئچ_چالو_ب}-الف میں ازل سے منقطع سوئچ لمحہ \عددی{t=\SI{7}{\second}} پر چالو کیا جاتا ہے۔رو \عددی{i(t)} دریافت کریں۔ 
\begin{figure}
\centering
\begin{subfigure}{0.5\textwidth}
\centering
\begin{tikzpicture}
\draw(0,0) to [american current source,l={$\SI{6}{\ampere}$}]++(0,\y) to [resistor,l={$\SI{1}{\ohm}$}]++(\x,0) to [inductor,i={$i_L(t)$},l={$\SI{5}{\henry}$}]++(\x,0) to [short]++(\x,0) to [resistor,l={$\SI{2}{\ohm}$}]++(0,-\y) to [short] (0,0);
\draw(\x,0) to [resistor,i<_={$i(t)$},*-*,l={$\SI{4}{\ohm}$}]++(0,\y);
\draw(2*\x,0) to [cspst,*-*,l_={${t=\SI{7}{\second}}$}]++(0,\y);
\end{tikzpicture}%
\caption*{(الف)}
\end{subfigure}
\begin{subfigure}{0.5\textwidth}
\centering
\begin{tikzpicture}
\draw(0,0) to [american current source,l={$\SI{6}{\ampere}$}]++(0,\y) to [resistor,l={$\SI{1}{\ohm}$}]++(\x,0) to [short,i={$i_L(t)$}]++(\x,0)to [resistor,l={$\SI{2}{\ohm}$}]++(0,-\y) to [short] (0,0);
\draw(\x,0) to [resistor,i<_={$i(t)$},*-*,l={$\SI{4}{\ohm}$}]++(0,\y);
\end{tikzpicture}%
\caption*{(ب)}
\end{subfigure}%
\begin{subfigure}{0.5\textwidth}
\centering
\begin{tikzpicture}
\draw(0,0) to [american current source,l={$\SI{6}{\ampere}$}]++(0,\y) to [resistor,l={$\SI{1}{\ohm}$}]++(\x,0) to [inductor,l={$\SI{5}{\henry}$}]++(\x,0)to [short]++(0,-\y) to [short] (0,0);
\draw(\x,0) to [resistor,i<_={$i(t)$},*-*]++(0,\y);
\draw(\x,3/4*\y)node[left]{$\SI{4}{\ohm}$};
%loop currents
\draw[stealth-]([shift={(-150:\x/6)}]\x/2,\y/2) arc (-150:150:\x/6);
\draw(\x/2,\y/2)node{$i_1$};
\draw[stealth-]([shift={(-150:\x/6)}]\x+\x/2,\y/2) arc (-150:150:\x/6);
\draw(\x+\x/2,\y/2)node{$i_2$};
\end{tikzpicture}
\caption*{(پ)}
\end{subfigure}
\begin{subfigure}{0.5\textwidth}
\centering
\pgfplotsset{scaled y ticks=false, scaled x ticks=false}
\begin{tikzpicture}
\begin{axis}[axis lines*=middle,
xlabel={$t$},
ylabel={$i(t)$},
every axis x label/.style={
    at={(ticklabel* cs:1.15)},
    anchor=east,}, 
	every axis y label/.style={
    at={(ticklabel* cs:1.05)},
    anchor=east,}
]
\addplot[domain=6:7,samples=10]{2};
\addplot[domain=7:13,samples=100]{2*e^(4/5*(7-x))};
\end{axis}
\end{tikzpicture}%
\caption*{(ت)}
\end{subfigure}%
\caption{مثال \حوالہ{مثال_عارضی_امالہ_گیر_سوئچ_چالو} کے اشکال۔}
\label{شکل_عارضی_امالہ_گیر_سوئچ_چالو_ب}
\end{figure}

حل:منقطع سوئچ کی صورت میں دور برقرار حالت میں ہو گا لہٰذا امالہ گیر کو قصر دور تصور کرتے ہوئے شکل-ب حاصل کی گئی ہے۔تقسیم رو کے کلیے سے
\begin{align*}
i_L(7_-)=i_L(7_+)=6\left(\frac{4}{4+2}\right)=\SI{4}{\ampere}
\end{align*}
اور
\begin{align}\label{مساوات_عارضی_امالہ_گیر_سات_سیکنڈ_الف}
i(t)=\SI{6}{\ampere}-i_L(t)=6-4=\SI{2}{\ampere} \quad \quad (t<\SI{7}{\second})
\end{align}
لکھا جا سکتا ہے۔سوئچ چالو کرنے کے بعد کی صورت حال شکل-پ میں دکھائی گئی ہے جہاں سے درج ذیل لکھا جا سکتا ہے۔
\begin{align*}
i_1&=\SI{6}{\ampere}\\
5\frac{\dif i_2}{\dif t}+4(i_2-i_1)&=0
\end{align*}
ان مساوات کو ملاتے ہوئے
\begin{align*}
5\frac{\dif i_2}{\dif t}+4(i_2-6)&=0
\end{align*}
یعنی
\begin{align*}
\frac{\dif i_2}{\dif t}+\frac{4}{5} i_2=\frac{24}{5}
\end{align*}
حاصل ہوتا ہے جس کا عمومی حل درج ذیل ہے۔
\begin{align*}
i_2=6+K_2e^{-\frac{4}{5}t}
\end{align*}
چونکہ \عددی{i_2} درحقیقت \عددی{i_L} ہی ہے لہٰذا نا معلوم مستقل \عددی{K_2} کو ابتدائی معلومات سے حاصل کرتے ہیں۔درج بالا مساوات میں  \عددی{t=\SI{7}{\second}} پر \عددی{i_L(7_+)=\SI{4}{\ampere}} پُر کرتے ہوئے 
\begin{align*}
4=6+K_2e^{-\frac{4}{5}\times 7}
\end{align*} 
سے
\begin{align*}
K_2=-2e^{\frac{4}{5}\times 7}
\end{align*}
حاصل ہوتا ہے۔یوں سوئچ چالو کرنے کے بعد \عددی{i_2} کا مخصوص حل درج ذیل لکھا جائے گا۔
\begin{align*}
i_2=6-2e^{\frac{4}{5}(7-t)}
\end{align*}
اب شکل-پ کو دیکھتے ہوئے
\begin{align*}
i(t)&=i_1-i_2\\
&=6-\left(6-2^{\frac{4}{5}(7-t)}\right)\\
&=2e^{\frac{4}{5}(7-t)}\quad \quad (t>\SI{7}{\second})
\end{align*}
لکھا جا سکتا ہے۔یوں ازل سے ابد تک \عددی{i(t)} کو مساوات \حوالہ{مساوات_عارضی_امالہ_گیر_سات_سیکنڈ_الف} اور درج بالا مساوات  پیش کرتے ہیں۔انہیں اکٹھے لکھتے  اور شکل-ت میں پیش کرتے ہیں۔
\begin{align}
i(t)=
\begin{cases}
\SI{2}{\ampere} & t<\SI{7}{\second}\\
2e^{\frac{4}{5}(7-t)} \, \si{\ampere} & t>\SI{7}{\second}
\end{cases}
\end{align}
\انتہا{مثال}
%========================
\ابتدا{مشق}\شناخت{مشق_عارضی_امالہ_دو_منبع_الف}
شکل \حوالہ{شکل_عارضی_امالہ_دو_منبع_الف} میں ابتدائی حالت \عددی{i_L(0_+)} دریافت کریں۔دائرہ \عددی{abcfa} میں \عددی{i_1} اور \عددی{abdea} میں \عددی{i_2} لیتے ہوئے  کرخوف مساوات دباو لکھیں۔ان مساوات  سے صرف \عددی{i_1} پر مبنی مساوات حاصل کریں۔یوں ازل سے ابد تک \عددی{i_L} دریافت کریں۔
\begin{figure}
\centering
\begin{tikzpicture}
\draw(0,0)node[left]{$a$} to [american voltage source,l={$\SI{12}{\volt}$}]++(0,2*\y)node[left]{$b$} to [resistor,l={$\SI{2}{\ohm}$}]++(\x,0) node[above]{$c$} to [resistor,l={$\SI{2}{\ohm}$}] ++(\x,0)to [short]++(\x,0)node[right]{$d$} to [resistor,l_={$\SI{4}{\ohm}$}]++(0,-2*\y)node[right]{$e$} to [short] (0,0);
\draw(\x,0)node[below]{$f$} to [inductor,i<_={$i_L$},*-,l={$\SI{2}{\henry}$}]++(0,\y) to [resistor,-*,l={$\SI{3}{\ohm}$}]++(0,\y);
\draw(2*\x,0) to [american voltage source,*-,l={$\SI{16}{\volt}$}]++(0,\y) to [ospst,-*,l={${t=\SI{0}{\second}}$}]++(0,\y);
%\draw(3*\x+\dx,\y)node[right]{$\begin{aligned} &+ \\ \\ &v_0(t)  \\  \\ &- \end{aligned}$};
\end{tikzpicture}
\caption{مشق \حوالہ{مشق_عارضی_امالہ_دو_منبع_الف} کا دور۔}
\label{شکل_عارضی_امالہ_دو_منبع_الف}
\end{figure}

جوابات:\عددی{i_L(0_+)=\SI{3.5}{\ampere}}، \عددی{\tfrac{\dif i_1}{\dif t}+2.25i_1=4.5}، \عددی{i_L(t>0)=2+1.5e^{-2.25t}\,\si{\ampere}}
\انتہا{مشق}
%=======================
\ابتدا{مشق}\شناخت{مشق_عارضی_برق_گیر_دو_منبع_الف}
شکل \حوالہ{شکل_عارضی_برق_گیر_دو_منبع_الف} میں \عددی{v_0(t)} حاصل کریں۔
\begin{figure}
\centering
\begin{tikzpicture}
\draw(0,0) to [american voltage source,l={$\SI{12}{\volt}$}]++(0,2*\y) to [resistor,l={$\SI{1}{\ohm}$}]++(\x,0) to [resistor,l={$\SI{1}{\ohm}$}] ++(\x,0) to [resistor,l={$\SI{2}{\ohm}$}] ++(\x,0);
\draw(0,0) to [short]++(3*\x,0) to [ospst,l_={${t=0}$}] ++(0,\y) to [american voltage source,l_={$\SI{8}{\volt}$}]++(0,\y);
\draw(\x,0) to [capacitor,*-*,l={$\SI{2}{\farad}$}]++(0,2*\y);
\draw(2*\x,0) to [resistor,*-*,l={$\SI{2}{\ohm}$}]++(0,2*\y);
\draw(2*\x+\dx,\y)node[right]{$\begin{aligned} &+ \\ \\ &v_0(t)  \\  \\ &- \end{aligned}$};
\end{tikzpicture}
\caption{مشق \حوالہ{مشق_عارضی_برق_گیر_دو_منبع_الف} کا دور۔}
\label{شکل_عارضی_برق_گیر_دو_منبع_الف}
\end{figure}

جوابات:\عددی{v_0(t)=\tfrac{24}{5}+\tfrac{1}{5}e^{-\tfrac{5}{8}t} \, \si{\volt}}
\انتہا{مشق}
%=======================
\ابتدا{مشق}\شناخت{مشق_عارضی_برق_گیر_دو_منبع_ب}
شکل \حوالہ{شکل_عارضی_برق_گیر_دو_منبع_ب} میں سوئچ منقطع کرنے کے بعد \عددی{v_0} حاصل کریں۔
\begin{figure}
\centering
\begin{tikzpicture}
\draw(0,2*\y) to [american voltage source,l={$\SI{18}{\volt}$}]++(0,-2*\y);
\draw(0,2*\y) to [resistor,l={$\SI{4}{\ohm}$}]++(\x,0) to [resistor,l={$\SI{4}{\ohm}$}] ++(\x,0);
\draw(0,0) to [short]++(2*\x,0) to [american voltage source,l={$\SI{6}{\volt}$}]++(0,\y) to [ospst,l_={${t=0}$}] ++(0,\y);
\draw(\x,0) to [inductor,*-,l={$\SI{6}{\henry}$}]++(0,\y) to [resistor,-*,l={$\SI{8}{\ohm}$}]++(0,\y);
\draw(\x+\dx,\y+\y/2)node[right]{$\begin{aligned} &+ \\ &v_0  \\  &- \end{aligned}$};
\end{tikzpicture}
\caption{مشق \حوالہ{مشق_عارضی_برق_گیر_دو_منبع_ب} کا دور۔}
\label{شکل_عارضی_برق_گیر_دو_منبع_ب}
\end{figure}

جوابات:\عددی{v_0=-12+\tfrac{9}{2}e^{-2t} \, \si{\volt}}
\انتہا{مشق}
%=======================

\حصہ{دھڑکن}\شناخت{حصہ_عارضی_دھڑکن}
گزشتہ حصے میں سوئچ کو چالو یا منقطع کرتے ہوئے ادوار میں یکدم تبدیلی پیدا کی گئی۔فوراً تبدیلی پیدا کرنے والے دو عدد تفاعل نہایت اہم ہیں۔انہیں \اصطلاح{اکائی سیڑھی تفاعل}\فرہنگ{اکائی سیڑھی تفاعل}\حاشیہب{unit step function}\فرہنگ{unit step function} اور \اصطلاح{اکائی جھٹکا تفاعل}\فرہنگ{اکائی جھٹکا تفاعل}\حاشیہب{unit impulse function}\فرہنگ{unit impulse function} کہتے ہیں۔آئیں اکائی سیڑھی تفاعل پر غور کریں۔

\اصطلاح{اکائی سیڑھی تفاعل} \عددی{u(t)} کی الجبرائی تعریف درج ذیل ہے۔
\begin{align}
u(t)=
\begin{cases}
0 & t<0\\
1 & t>0
\end{cases}
\end{align}
اکائی سیڑھی تفاعل \اصطلاح{بے بعد}\فرہنگ{بے بعد}\حاشیہب{dimensionless}\فرہنگ{dimensionless} ہے  جو منفی\عددی{t} کی صورت میں صفر کے برابر جبکہ مثبت  \عددی{t} کی صورت میں اکائی کے برابر ہے۔شکل \حوالہ{شکل_عارضی_اکائی_سیڑھی_تفاعل_الف}-الف میں اکائی سیڑھی تفاعل کو دکھایا گیا ہے۔اکائی سیڑھی تفاعل کے متغیرہ کو \عددی{t-t_0} لکھتے ہوئے شکل \حوالہ{شکل_عارضی_اکائی_سیڑھی_تفاعل_الف}-ب حاصل ہوتا ہے جو افقی محدد پر \عددی{t_0} دائیں منتقل اکائی سیڑھی تفاعل \عددی{u(t-t_0)} ہے۔یہ تفاعل منفی \عددی{t-t_0} کی صورت میں صفر کے برابر ہے جبکہ مثبت \عددی{t-t_0} کی صورت میں یہ اکائی کے برابر ہے۔اکائی سیڑھی تفاعل کو \عددی{V_0} وولٹ کے دباو سے ضرب دینے سے \عددی{V_0} وولٹ کا سیڑھی تفاعل حاصل ہو گا جس سے بعد وولٹ \عددی{\si{\volt}} ہے۔یوں بے بعد سیڑھی تفاعل سے کسی بھی بعد کی سیڑھی تفاعل حاصل کی جا سکتی ہے۔شکل \حوالہ{شکل_عارضی_اکائی_سیڑھی_تفاعل_الف}-پ میں  \عددی{Au(t)} اور شکل \حوالہ{شکل_عارضی_اکائی_سیڑھی_تفاعل_الف}-ت میں \عددی{-Au(t-t_0)} دکھائے گئے ہیں جہاں \عددی{A} از خود مثبت عدد ہے۔
\begin{figure}
\centering
\begin{subfigure}{0.5\textwidth}
\centering
\begin{tikzpicture}
\draw[gray](0,-0.5)--(0,2)node[left]{$u(t)$};
\draw[gray](-0.5,0)--(3,0)node[right]{$t$};
\draw(-0.25,0)--(0,0)--(0,1)node[left]{$1$}--(3,1);
\end{tikzpicture}%
\caption*{(الف)}
\end{subfigure}%
\begin{subfigure}{0.5\textwidth}
\centering
\begin{tikzpicture}
\draw[gray](0,-0.5)--(0,2)node[left]{$u(t-t_0)$};
\draw[gray](-0.5,0)--(3,0)node[right]{$t$};
\draw(-0.25,0)--(1,0)node[below]{$t_0$}--(1,1)--(3,1);
\draw[gray,dashed](1,1)--(0,1)node[left,black]{$1$};
\end{tikzpicture}%
\caption*{(ب)}
\end{subfigure}
\begin{subfigure}{0.5\textwidth}
\centering
\begin{tikzpicture}
\draw[gray](0,-1.5)--(0,1.5)node[left]{$Au(t)$};
\draw[gray](-0.5,0)--(3,0)node[right]{$t$};
\draw(-0.25,0)--(0,0)--(0,1)node[left]{$A$}--(3,1);
\end{tikzpicture}%
\caption*{(پ)}
\end{subfigure}%
\begin{subfigure}{0.5\textwidth}
\centering
\begin{tikzpicture}
\draw[gray](0,-1.5)--(0,1.5)node[left]{$-Au(t-t_0)$};
\draw[gray](-0.5,0)--(3,0)node[right]{$t$};
\draw(-0.25,0)--(1,0)node[above]{$t_0$}--(1,-1)--(3,-1);
\draw[gray,dashed](1,-1)--(0,-1)node[left,black]{$-A$};
\end{tikzpicture}%
\caption*{(ت)}
\end{subfigure}%
\caption{اکائی سیڑھی تفاعل۔}
\label{شکل_عارضی_اکائی_سیڑھی_تفاعل_الف}
\end{figure}

اکائی سیڑھی تفاعل سے مستطیل تفاعل حاصل کیا جا سکتا ہے۔یہ عمل شکل \حوالہ{شکل_عارضی_اکائی_سیڑھی_تفاعل_ب} میں دکھایا گیا ہے جہاں \عددی{Au(t)} اور
 \عددی{-Au(t-t_0)} کا مجموعہ
\begin{align}
f(t)=Au(t)-Au(t-t_0)
\end{align}
 لیتے ہوئے \عددی{A} حیطے کا مستطیل تفاعل حاصل کیا گیا ہے۔ 
\begin{figure}
\centering
\begin{subfigure}{0.5\textwidth}
\centering
\begin{tikzpicture}
\draw[gray](0,-0.5)--++(0,2)node[left]{$Au(t)$};
\draw[gray](-0.5,0)--++(3.5,0)node[right]{$t$};
\draw(-0.25,0)--++(0.25,0)--++(0,1)node[left]{$A$}--++(3,0);
%
\pgfmathsetmacro{\ky}{-2}
\draw[gray](0,\ky-1.5)--++(0,2)node[left]{$-Au(t-t_0)$};
\draw[gray](-0.5,\ky)--++(3.5,0)node[right]{$t$};
\draw(-0.25,\ky)--++(1.25,0)--++(0,-1)--++(2,0);
\draw(0,\ky-1)node[left]{$-A$};
%
\pgfmathsetmacro{\ky}{-3}
\draw[gray](0,2*\ky-0.5)--++(0,2)node[left]{$f(t)$};
\draw[gray](-0.5,2*\ky)--++(3.5,0)node[right]{$t$};
\draw(-0.25,2*\ky)--++(0.25,0)--++(0,1)node[left]{$A$}--++(1,0)--++(0,-1)node[below]{$t_0$}--++(2,0);
\end{tikzpicture}%
\end{subfigure}%
\caption{اکائی سیڑھی تفاعل سے مستطیل تفاعل کا حصول۔}
\label{شکل_عارضی_اکائی_سیڑھی_تفاعل_ب}
\end{figure}
%=======================
\ابتدا{مثال}
اکائی سیڑھی تفاعل کے استعمال سے \عددی{T} طول موج اور \عددی{V_0} حیطے کی چکور موج حاصل کریں۔

حل:شکل \حوالہ{شکل_عارضی_اکائی_سیڑھی_تفاعل_ب} کی طرز پر متعدد مستطیل اشارات سے ایسی موج حاصل کی جا سکتی ہے۔ایسا کرنے کی خاطر متعدد اکائی سیڑھی تفاعل استعمال کی جائیں گی۔درکار تفاعل کو درج ذیل لکھا جا سکتا ہے
\begin{align*}
v(t)=V_0\left[u(t)-u(t-0.5T)+u(t-T)-u(t-1.5T)+u(t-2T)-\cdots\right]
\end{align*}
جسے شکل \حوالہ{شکل_عارضی_اکائی_سے_چکور}-الف میں دکھایا گیا ہے۔
\begin{figure}
\centering
\begin{subfigure}{0.5\textwidth}
\centering
\begin{tikzpicture}
\draw[gray](0,0)--(6,0)node[right]{$t$};
\draw[gray](0,-0.25)--(0,1.5)node[left]{$v(t)$};
\draw(0,0)--++(0,1)node[left]{$V_0$}--++(1,0)--++(0,-1)node[below]{$\frac{T}{2}$}--++(1,0)node[below]{$T$}--++(0,1)--++(1,0)--++(0,-1)node[below]{$\frac{3T}{2}$}--++(1,0)node[below]{$2T$}--++(0,1)--++(1,0)--++(0,-1)node[below]{$\frac{5T}{2}$}--++(0.5,0);
\end{tikzpicture}
\caption*{(الف) چکور موج۔}
\end{subfigure}
\begin{subfigure}{0.5\textwidth}
\centering
\begin{tikzpicture}
\draw[gray](0,0)--(6,0)node[right]{$t$};
\draw[gray](0,-0.25)--(0,3.5)node[left]{$v(t)$};
\draw(0,0)--++(0,0.5)node[left]{$0.5$}--++(1,0)--++(0,0.5)--++(1,0)--++(0,0.5)--++(1,0)--++(0,0.5)--++(1,0)--++(0,0.5)--++(1,0)--++(0,0.5)--++(0.5,0);
\end{tikzpicture}
\caption*{(ب) بڑھتی سیڑھی۔}
\end{subfigure}
\caption{اکائی سیڑھی تفاعل سے چکور موج کا حصول۔}
\label{شکل_عارضی_اکائی_سے_چکور}
\end{figure}
\انتہا{مثال}
%===============
\ابتدا{مثال}
اکائی سیڑھی تفاعل سے اوپر جانب بڑھتی سیڑھی تفاعل حاصل کریں۔سیڑھی کی اونچائی \عددی{0.5} رکھیں۔

حل:درج بالا مثال میں اجزاء کو بالترتیب جمع اور منفی کیا گیا۔یہاں انہیں صرف جمع کیا جاتا ہے یعنی
\begin{align*}
v(t)=0.5\left[u(t)+u(t-0.5T)+u(t-T)+u(t-1.5T)+u(t-2T)+\cdots\right]
\end{align*}
جس سے درکار سیڑھی حاصل ہو گی۔بڑھتی سیڑھی کو شکل \حوالہ{شکل_عارضی_اکائی_سے_چکور}-ب میں دکھایا گیا ہے۔   
\انتہا{مثال}
%==============
\ابتدا{مثال}\شناخت{مثال_عارضی-ایک_قطب_دو_چال_الف}
شکل \حوالہ{شکل_عارضی-ایک_قطب_دو_چال_الف}-الف میں ایک قطب دو چال کا سوئچ استعمال کیا گیا ہے جو ازل سے دور کو زمین سے ملایا ہوا ہے۔ لمحہ \عددی{t=\SI{0}{\second}} پر سوئچ کو پلٹتے ہوئے دور کو منبع دباو کے ساتھ ملایا جاتا ہے۔لمحہ \عددی{t=\SI{30}{\milli\second}} پر سوئچ کو واپس اپنی حالت میں لاتے ہوئے دور کو ایک بار پھر زمین کے ساتھ جوڑا جاتا ہے۔دباو \عددی{v_C(t)} حاصل کریں۔

\begin{figure}
\centering
\begin{subfigure}{0.5\textwidth}
\begin{tikzpicture}
\draw(0,0) to [short]++(2*\x+\x/2,0) to [capacitor,l={$\SI{2}{\micro\farad}$}]++(0,\y) to [resistor,l_={$\SI{5}{\kilo\ohm}$}]++(-\x,0)++(-\x/4,0) node[spdt,xscale=-1,yscale=-1](ksw){}++(-\x,0);
\draw(\x/2,0) to [short,*-]++(0,\y/2) |-(ksw.out 1);
\draw(0,0) to [american voltage source,l={$\SI{10}{\volt}$}]++(0,\y) |-(ksw.out 2);
\draw(2*\x+\x/2+\x/2,\y/2)node{$\begin{aligned} &+ \\ &v_C(t)  \\ &- \end{aligned}$};
\end{tikzpicture}
\caption*{(الف)}
\end{subfigure}
\begin{subfigure}{0.5\textwidth}
\begin{tikzpicture}
\draw[gray] (-0.5,0)--(3,0)node[right]{$t$};
\draw[gray](0,-0.5)--(0,1.5);
\draw(-0.5,0)--(0,0)--(0,1)node[left]{$\SI{10}{\volt}$}--(1,1)--(1,0)node[below]{$\SI{30}{\milli\second}$}--(3,0);
\end{tikzpicture}
\caption*{(ب)}
\end{subfigure}%
\begin{subfigure}{0.5\textwidth}
\begin{tikzpicture}
\draw(0,0) to [short]++(\x,0) to [capacitor,l={$\SI{2}{\micro\farad}$}]++(0,\y) to [resistor,l_={$\SI{5}{\kilo\ohm}$}]++(-\x,0);
\draw(0,0) to [square voltage source,l={$v_p$}]++(0,\y);
%\draw(0-\dx,\y/2)node[left]{$\begin{aligned} &\SI{10}{\volt} \\ & \SI{10}{\milli\second} \end{aligned}$};
\draw(\x+\x/2,\y/2-\dy)node{$\begin{aligned} &+ \\ &v_C(t)  \\ &- \end{aligned}$};
\end{tikzpicture}
\caption*{(پ)}
\end{subfigure}
\begin{subfigure}{0.5\textwidth}
\centering
\pgfplotsset{scaled y ticks=false, scaled x ticks=false}
\begin{tikzpicture}
\begin{axis}[axis lines*=middle,xlabel=$t$, ylabel=$v_C$,xtick={0.03},xticklabels={$\SI{30}{\milli\second}$},ytick={9.5021},yticklabels={$\SI{9.5021}{\volt}$},
every axis x label/.style={
    at={(ticklabel* cs:1.15)},
    anchor=east,}, 
	every axis y label/.style={
    at={(ticklabel* cs:1.05)},
    anchor=east,}
]
\addplot[domain=0:0.03,samples=100]{10-10*e^(-100*x)};
\addplot[dashed,domain=0.03:0.07,samples=100]{10-10*e^(-100*x)};
\addplot[domain=0.03:0.07,samples=100]{190.8554*e^(-100*x)};
\end{axis}
\end{tikzpicture}
\caption*{(ت)}
\end{subfigure}
\caption{مثال \حوالہ{مثال_عارضی-ایک_قطب_دو_چال_الف} کے اشکال۔}
\label{شکل_عارضی-ایک_قطب_دو_چال_الف}
\end{figure}

حل:سوئچ کو پلٹ کر واپس کرنے سے دور اور منبع \عددی{\SI{30}{\milli\second}} کے لئے جڑتے ہیں۔یوں دور کو اس دورانیے کے لئے \عددی{\SI{10}{\volt}} ملتا ہے۔شکل-ب میں اس دباو کو دکھایا گیا ہے۔شکل-الف میں سوئچ اور منبع دباو کی جگہ مستطیل  دباو پیدا کرنے والا منبع \عددی{v_p} نسب کرنے سے شکل-پ حاصل ہوتا ہے جہاں
\begin{align*}
v_p=10\left[u(t)-u(t-\SI{30}{\milli\second})\right]
\end{align*}
کے برابر ہے۔آپ دیکھ سکتے ہیں کہ شکل-پ میں بھی دور کو عین شکل-ب کا دباو مہیا کیا گیا ہے لہٰذا ان دونوں ادوار کے حل میں کوئی فرق نہیں ہو گا۔

ازل سے داخلی دباو صفر کے برابر ہونے کی بنا 
\begin{align*}
v_C(0_-)=v_C(0_+)=\SI{0}{\volt}
\end{align*}
ہو گا۔ دورانیہ \عددی{t=\SI{0}{\second}} تا \عددی{t=\SI{30}{\milli\second}}شکل-پ  میں داخلی دباو \عددی{v_p=\SI{10}{\volt}} کے برابر ہے لہٰذا کرخوف مساوات رو درج ذیل لکھی جائے گی۔
\begin{align*}
\frac{v_C-10}{5000}+2\times 10^{-6}\frac{\dif v_C}{\dif t}=0
\end{align*}
اس کو ترتیب دیتے ہوئے
\begin{align*}
\frac{\dif v_C}{\dif t}+100 v_C=1000
\end{align*}
لکھا جا سکتا ہے جس کے جبری اور فطری حل درج ذیل ہیں۔
\begin{align*}
v_{C,j}&=K_1=10\\
v_{C,f}&=K_2 e^{-100t}
\end{align*}
یوں عمومی حل درج ذیل لکھا جائے گا
\begin{align*}
v_C(t)=10+K_2e^{-100t}\quad \quad (0<t<\SI{30}{\milli\second})
\end{align*}
جس میں لمحہ \عددی{t=\SI{0}{\second}} کے معلومات پُر کرتے
\begin{align*}
0=10+K_2 e^{-100\times 0}
\end{align*}
ہوئے نا معلوم متغیر کی قیمت \عددی{K_2=-10} حاصل ہوتی ہے۔یوں عمومی حل درج ذیل ہے۔
\begin{align}\label{مساوات_عارضی_مستطیل_مثال_حل_الف}
v_C(t)=10-10e^{-100t}\quad \quad (0<t<\SI{30}{\milli\second})
\end{align}
لمحہ \عددی{t=\SI{30}{\milli\second}} پر داخلی دباو میں دوبارہ یک دم تبدیلی پائی جاتی ہے لہٰذا اس لمحے کے معلومات اگلے دورانیے کے حل کے لئے درکار ہوں گے۔مساوات \حوالہ{مساوات_عارضی_مستطیل_مثال_حل_الف} سے \عددی{t=\SI{30}{\milli\second}} پر \عددی{v_C(0.03_-)} کی قیمت حاصل کرتے ہیں۔
\begin{align*}
v_C(0.03_-)=v_C(0.03_+)=10-10e^{-100\times 0.03}=\SI{9.5021}{\volt}
\end{align*} 
اگلے دورانیے یعنی \عددی{\SI{30}{\milli\second}<t} کا حل تلاش کرتے ہیں۔اس دورانیے میں داخلی دباو \عددی{v_p=\SI{0}{\volt}} کے برابر ہے لہٰذا شکل-پ کا کرخوف مساوات رو درج ذیل ہو گا
\begin{align*}
\frac{v_C-0}{5000}+2\times10^{-6}\frac{\dif v_C}{\dif t}=0
\end{align*}
جس کا عمومی حل
\begin{align*}
v_C=K_3e^{-100t} \quad \quad (\SI{30}{\milli\second}< t)
\end{align*}
ہے۔اس میں لمحہ \عددی{t=\SI{30}{\milli\second}} پر \عددی{V_C(0.03_+)} پر کرتے  ہوئے
\begin{align*}
9.5021=K_3 e^{-100\times 0.03}
\end{align*}
نا معلوم متغیرہ \عددی{K_3=190.8554} حاصل ہوتا ہے لہٰذا عمومی حل درج ذیل لکھا جائے گا۔
\begin{align}\label{مساوات_عارضی_مستطیل_مثال_حل_ب}
v_C=190.8554e^{-100t} \quad \quad (\SI{30}{\milli\second}< t)
\end{align}
مساوات \حوالہ{مساوات_عارضی_مستطیل_مثال_حل_الف} اور مساوات \حوالہ{مساوات_عارضی_مستطیل_مثال_حل_ب} کو اکٹھے لکھتے  ہوئے اس کا خط
\begin{align}
v_C=
\begin{cases}
10-10e^{-100t}& 0<t<\SI{30}{\milli\second}\\
190.8554e^{-100t} & \SI{30}{\milli\second}< t
\end{cases}
\end{align}
شکل-ت میں کھینچتے ہیں۔

اگر لمحہ \عددی{t=\SI{30}{\milli\second}} اور اس کے بعد بھی داخلی دباو \عددی{\SI{10}{\volt}} پر برقرار رہتا تب \عددی{v_C} نقطہ دار لکیر پر چلتے ہوئے \عددی{\SI{10}{\volt}} تک جا پہنچتا۔    
\انتہا{مثال}
%====================
\ابتدا{مشق}\شناخت{مشق_عارضی_امالہ_مستطیل_دباو_الف}
شکل \حوالہ{شکل_عارضی_مشق_امالہ_مستطیل_دباو_الف}-الف کو شکل \حوالہ{شکل_عارضی_مشق_امالہ_مستطیل_دباو_الف}-ب کا داخلی دباو مہیا کیا جاتا ہے۔دباو \عددی{v_0} دریافت کریں۔

\begin{figure}
\centering
\begin{subfigure}{0.5\textwidth}
\centering
\begin{tikzpicture}
\draw (0,0) to [square voltage source,l={$v_p$}] ++(0,\y) to [resistor,l={$\SI{4}{\ohm}$}]++(\x,0) to [inductor,l={$\SI{6}{\henry}$}]++(\x,0) to [resistor,l_={$\SI{2}{\ohm}$}]++(0,-\y) to [short](0,0);
\draw(\x,0) to [resistor,*-*,l={$\SI{6}{\ohm}$}]++(0,\y);
\draw(2*\x+\x/4,\y/2)node{$\begin{aligned} &+ \\ &v_0 \\ &- \end{aligned}$};
\end{tikzpicture}%
\caption*{(الف)}
\end{subfigure}%
\begin{subfigure}{0.5\textwidth}
\centering
\begin{tikzpicture}
\draw[gray](-0.5,0)--(3,0)node[right]{$t$};
\draw[gray](0,-0.25)--(0,1.5)node[left]{$v_p$};
\draw(-0.5,0)--(0,0)--(0,1)node[left]{$\SI{20}{\volt}$}--(1,1)--(1,0)node[below]{$\SI{2}{\second}$}--(2.5,0);
\end{tikzpicture}%
\caption*{(ب)}
\end{subfigure}%
\caption{مشق \حوالہ{مشق_عارضی_امالہ_مستطیل_دباو_الف} کے اشکال۔}
\label{شکل_عارضی_مشق_امالہ_مستطیل_دباو_الف}
\end{figure}

جواب:\عددی{v_0(t<0)=\SI{0}{\volt}}، \عددی{v_0(0<t<2)=\tfrac{30}{29}\left(1-e^{-\tfrac{29}{15}t}\right)}، \\
\عددی{v_0(2<t)=8.78074e^{-\tfrac{11}{15}t}}
\انتہا{مشق}
%=====================

\حصہ{دو درجی ادوار}
شکل \حوالہ{شکل_عارضی_دو_درجی_ادوار}-الف میں \عددی{R}، \عددی{L} اور \عددی{C} متوازی منبع رو \عددی{i_S(t)} کے ساتھ جڑے ہیں جبکہ شکل-ب میں منبع دباو کے ساتھ تینوں پرزے سلسلہ وار جڑے ہیں۔شکل-الف کی کرخوف مساوات رو اور شکل-ب کی کرخوف مساوات دباو بالترتیب درج ذیل ہیں۔
\begin{align*}
\frac{v(t)}{R}+\frac{1}{L}\int_{t_0}^{t} v(t)\dif t+i_L(t_0)+ C\frac{\dif v(t)}{\dif t}&=i_S(t)\\
i(t) R+\frac{1}{C}\int_{t_0}^{t} i(t)\dif t+v_C(t_0)+L \frac{\dif i(t)}{\dif t}&=v_S(t)
\end{align*}
یہ مساوات یکساں صورت رکھتے ہیں لہٰذا ان کا حل بالکل یکساں ہو گا۔ان مساوات کا تفرق لے کر ترتیب دینے سے درج ذیل تفرقی مساوات حاصل ہوتے ہیں۔
\begin{align*}
C\frac{\dif^{\,2} v(t)}{\dif t^2}+\frac{1}{R}\frac{\dif v(t)}{\dif t}+\frac{v(t)}{L}&=\frac{\dif i_S(t)}{\dif t}\\
L\frac{\dif^{\,2} i(t)}{\dif t^2}+R\frac{\dif i(t)}{\dif t}+\frac{i(t)}{C}&=\frac{\dif v_S(t)}{\dif t}
\end{align*}
آپ نے دیکھا کہ دونوں مساوات میں تفرقی جزو کے عددی سر، مستقل مقدار ہیں۔آئیں مستقل عددی سر کے دو درجی تفرقی مساوات کو حل کرنا سیکھتے ہیں۔
 \begin{figure}
\centering
\begin{subfigure}{0.5\textwidth}
\centering
\begin{tikzpicture}
\draw(0,0) to [short]++(3*\x,0);
\draw(0,\y) to [short]++(3*\x,0);
\draw(0,0) to [american current source,l={$i_S(t)$}]++(0,\y);
\draw(\x,0) to [resistor,*-*,l={$R$}]++(0,\y);
\draw(2*\x,0)node[ground]{} to [inductor,*-*,l={$L$},i<_={$i_L(t_0)$}]++(0,\y)node[above]{$v(t)$};
\draw(3*\x,0) to [capacitor,l={$C$}]++(0,\y);
\end{tikzpicture}%
\caption*{(الف)}
\end{subfigure}
\begin{subfigure}{0.5\textwidth}
\centering
\begin{tikzpicture}
\draw(0,0) to [american voltage source,l={$v_S(t)$}]++(0,\y) to [resistor,i={$i(t)$},l={$R$}]++(\x,0) to [capacitor,l_={$C$}]++(\x,0) to [inductor,l={$L$}]++(0,-\y) to [short] (0,0);
\draw(\x+\x/2,\y+3/4*\y)node{$+ v_C(t_0) - $};
\end{tikzpicture}%
\caption*{(ب)}
\end{subfigure}
\caption{دو درجی ادوار۔}
\label{شکل_عارضی_دو_درجی_ادوار}
\end{figure}

مستقل عددی سر کے دو درجی تفرقی مساوات کی عمومی صورت درج ذیل ہے جہاں دو درجی تفرق کے عددی سر کو اکائی برابر رکھا گیا ہے۔
\begin{align}\label{مساوات_عارضی_دو_درجی_عمومی_الف}
\frac{\dif^{\,2} y(t)}{\dif t^2}+a_1\frac{\dif y(t)}{\dif t}+a_2 y(t)=f(t)
\end{align}
ایک درجی مساوات کے حل کی طرح یہاں بھی اگر مساوات \حوالہ{مساوات_عارضی_دو_درجی_عمومی_الف} کا جبری حل \عددی{y_j(t)} ہو اور درج ذیل متجانس مساوات کا فطری حل \عددی{y_f(t)} ہو
 \begin{align}\label{مساوات_عارضی_دو_درجی_عمومی_ہم_جنسی_الف}
\frac{\dif^{\,2} y(t)}{\dif t^2}+a_1\frac{\dif y(t)}{\dif t}+a_2 y(t)=0
\end{align}
تب مساوات \حوالہ{مساوات_عارضی_دو_درجی_عمومی_الف} کا عمومی حل
\begin{align}
y(t)=y_j(t)+y_f(t)
\end{align}
ہو گا۔یاد رہے کہ کسی بھی تفرقی مساوات میں جبری قوت کو صفر \عددی{(f(t)=0)} پُر کرنے سے اس کی متجانس مساوات حاصل ہوتی ہے۔مستقل جبری قوت، یعنی \عددی{f(t)=A}، کی صورت میں جبری حل بھی مستقل ہو گا جسے \عددی{K_1} تصور کرتے ہوئے مساوات \حوالہ{مساوات_عارضی_دو_درجی_عمومی_الف} میں پُر کرتے ہوئے
\begin{align}
y_j(t)=K_1=\frac{A}{a_2}
\end{align}
حاصل ہوتا ہے۔

متجانس مساوات میں \عددی{a_1=2\zeta \omega_0} اور \عددی{a_2=\omega_0^2} پُر کرنے سے
\begin{align}\label{مساوات_عارضی_دو_درجی_عمومی_ہم_جنسی_ب}
\frac{\dif^{\,2} y(t)}{\dif t^2}+2\zeta \omega_0\frac{\dif y(t)}{\dif t}+\omega_0^2 y(t)=0
\end{align}
حاصل ہوتا ہے جہاں \عددی{\omega_0} کو \اصطلاح{(بلا تقصیر) قدرتی تعدد}\فرہنگ{قدرتی تعدد!بلا تقصیر}\حاشیہب{undamped natural frequency}
\فرہنگ{natural frequency!undamped} اور \عددی{\zeta} کو \اصطلاح{تقصیری تناسب}\فرہنگ{تقصیری تناسب}\حاشیہب{damping ratio}\فرہنگ{damping ratio} کہا جاتا ہے۔ ان کی افادیت جلد سامنے آئے گی۔مساوات  \حوالہ{مساوات_عارضی_دو_درجی_عمومی_ہم_جنسی_ب} متجانس مساوات کی عمومی صورت ہے جو طبیعیات کے دیگر شعبوں میں بھی استعمال کی جاتی ہے۔اس مساوات کا فطری حل
\begin{align*}
y_f(t)=Ke^{st}
\end{align*}
تصور کرتے ہیں۔آئیں اس فطری حل کو متجانس مساوات میں پُر کرتے ہیں۔
\begin{align*}
s^2 K e^{st}+2\zeta\omega_0 s K e^{st}+\omega_0^2 K e^{st}=0
\end{align*}
اس کو \عددی{Ke^{st}} سے تقسیم کرتے ہوئے
\begin{align}
s^2+2\zeta\omega_0+\omega_0^2=0
\end{align}
حاصل ہوتا ہے جسے \اصطلاح{امتیازی مساوات}\فرہنگ{امتیازی مساوات}\حاشیہب{characteristic equation}\فرہنگ{characteristic equation} کہتے ہیں۔اس دو درجی امتیازی مساوات کو \عددی{s} کے لئے حل کرتے ہوئے
\begin{gather}
\begin{aligned}
s&=\frac{-2\zeta\omega_0\mp\sqrt{4\zeta^2\omega_0^2-4\omega_0^2}}{2}\\
&=-\zeta\omega_0\mp \omega_0\sqrt{\zeta^2-1}
\end{aligned}
\end{gather}
مساوات کے \اصطلاح{جذر}\فرہنگ{جذر}\حاشیہب{roots}\فرہنگ{roots} حاصل کرتے ہیں۔
\begin{gather}
\begin{aligned}\label{مساوات_عارضی_دو_درجی_ہم_سمتی_عمومی_حل_الف}
s_1=-\zeta\omega_0+ \omega_0\sqrt{\zeta^2-1}\\
s_2=-\zeta\omega_0- \omega_0\sqrt{\zeta^2-1}
\end{aligned}
\end{gather}
یوں دو فطری حل \عددی{K_2 e^{s_1t}} اور \عددی{K_3 e^{s_2t}} ممکن ہیں۔ایسی صورت میں عمومی فطری حل ان کا مجموعہ ہو گا
یوں عمومی فطری حل
\begin{align}
y_f(t)=K_2 e^{s_1 t}+K_3 e^{s_2 t}
\end{align}
لکھا جائے گا جہاں مستقل \عددی{K_2} اور \عددی{K_3} کو ابتدائی معلومات مثلاً \عددی{y(0)} اور \عددی{\left. \tfrac{\dif y(t)}{\dif t} \right|_{t=0}} سے حاصل کیا جاتا ہے۔عمومی حل میں مخصوص ابتدائی معلومات سے حاصل کردہ مستقل پر کرنے سے \اصطلاح{مخصوص حل}\فرہنگ{مخصوص حل}\حاشیہب{particular solution}\فرہنگ{particular solution} ملتا ہے۔

مساوات \حوالہ{مساوات_عارضی_دو_درجی_ہم_سمتی_عمومی_حل_الف} پر غور کرنے سے ظاہر ہوتا ہے کہ \عددی{s_1} اور \عددی{s_2} کی قیمتوں کا دارومدار \عددی{\zeta} کی قیمت پر ہے۔تین ممکنہ صورتیں پائی جاتی ہیں یعنی \عددی{\zeta>1}، \عددی{\zeta<1} اور \عددی{\zeta=1} جن سے بالترتیب \عددی{s_1} اور \عددی{s_2} کی قیمتیں حقیقی اور مختلف، مخلوط اور مختلف، حقیقی اور برابر حاصل ہوتی ہیں۔آئیں ان تینوں صورتوں پر تفصیلاً غور کریں۔

\حصہء{زیادہ مقصور صورت، \عددی{\zeta>1}} 
\اصطلاح{زیادہ مقصور صورت}\فرہنگ{مقصور!زیادہ}\حاشیہب{over damped condition}\فرہنگ{over damped condition} میں \عددی{s_1} اور \عددی{s_2} کی قیمتیں حقیقی اور آپس میں مختلف حاصل ہوتی ہیں۔زیادہ مقصور حالت \عددی{\zeta>1} کی صورت میں پائی جاتی ہے۔ایسی صورت میں فطری حل کو درج ذیل لکھا جا سکتا ہے
\begin{align}\label{مساوات_عارضی_فطری_زیادہ_مقصور_حل}
y_f(t)=K_2 e^{-(\zeta\omega_0- \omega_0\sqrt{\zeta^2-1})t}+K_3 e^{-(\zeta\omega_0+ \omega_0\sqrt{\zeta^2-1})t}
\end{align}
جو دو عدد، قوت نمائی انحطاطی تفاعل  کا مجموعہ ہے۔

\حصہء{کم مقصور صورت، \عددی{\zeta<1}}
 \اصطلاح{کم مقصور صورت}\فرہنگ{مقصور!کم}\حاشیہب{under damped condition}\فرہنگ{under damped condition} \عددی{\zeta<1} میں امتیازی مساوات کے حل، \عددی{s_1} اور \عددی{s_2}،  کی قیمتیں مخلوط حاصل ہوتی ہیں جنہیں درج ذیل لکھا جا سکتا ہے
\begin{gather}
\begin{aligned}
s_1=-\zeta\omega_0+ j\omega_0\sqrt{1-\zeta^2}=-\sigma+j \omega_d\\
s_2=-\zeta\omega_0- j\omega_0\sqrt{1-\zeta^2}=-\sigma-j\omega_d
\end{aligned}
\end{gather}
جہاں \عددی{\zeta\omega_0=\sigma} اور \عددی{\omega_0\sqrt{1-\zeta^2}=\omega_d} لکھے گئے ہیں جبکہ \عددی{j=\sqrt{-1}} ہے۔یوں فطری حل
\begin{align*}
y_f(t)&=K_2e^{-\sigma t+j\omega_d t}+K_3e^{-\sigma t -j \omega_d t}\\
&=e^{-\sigma t} \left[K_2 e^{j \omega_d t}+K_3 e^{-j \omega_d t}\right]\\
&=e^{-\sigma t} \left[K_2(\cos \omega_d t + j \sin \omega_d t)+K_3(\cos \omega_d t -j \sin \omega_d t)\right]\\
&=e^{-\sigma t} \left[(K_2+K_3)\cos \omega_d t+j(K_2-K_3)\sin \omega_d t\right]
\end{align*}
یعنی
\begin{gather}
\begin{aligned}\label{مساوات_عارضی_فطری_حل_کم_تقصیر_الف}
y_f(t)&=e^{-\sigma t} \left(c_1 \cos \omega_d t +c_2 \sin \omega_d t\right)\\
&=e^{-\zeta\omega_0 t}\left[c_1 \cos \omega_0\sqrt{1-\zeta^2} t+c_2 \sin \omega_0 \sqrt{1-\zeta^2} t\right]
\end{aligned}
\end{gather}
لکھا جائے گا جہاں \عددی{K_2+K_3=c_1} اور \عددی{j(K_2-K_3)=c_2} لکھے گئے ہیں۔فطری حل کے مستقل \عددی{c_1} اور \عددی{c_2} کو ابتدائی معلومات سے حاصل کیا جاتا ہے۔

مساوات \حوالہ{مساوات_عارضی_فطری_حل_کم_تقصیر_الف} میں
\begin{align*}
c_1&=A \cos \theta\\
c_2&=A\sin \theta
\end{align*}
پُر کرتے ہوئے
\begin{align*}
y_f(t)=e^{-\sigma t} \left(A \cos \theta \cos \omega_d t +A \sin \theta \sin \omega_d t\right)
\end{align*}
یعنی
\begin{gather}
\begin{aligned}\label{مساوات_عارضی_فطری_حل_کم_تقصیر_ب}
y_f(t)&=Ae^{-\sigma t} \cos(\omega_d t -\theta)\\
&=Ae^{-\zeta \omega_0 t} \cos(\omega_0 \sqrt{1-\zeta^2} t-\theta)
\end{aligned}
\end{gather}
لکھا جا سکتا ہے۔مساوات \حوالہ{مساوات_عارضی_فطری_حل_کم_تقصیر_ب} کے مستقل \عددی{A} اور \عددی{\theta} ہیں جنہیں ابتدائی معلومات سے حاصل کیا جاتا ہے۔جیسا شکل \حوالہ{شکل_عارضی_کم_قصری_فطری_حل} میں دکھایا گیا ہے، مساوات \حوالہ{مساوات_عارضی_فطری_حل_کم_تقصیر_ب} \اصطلاح{قصری ارتعاش}\فرہنگ{قصری ارتعاش}\حاشیہب{damped oscillation}\فرہنگ{damped!oscillation}  کو ظاہر کرتی ہے۔کم قصری مساوات میں \عددی{e^{-\zeta\omega_0 t}} قصری ارتعاش کے \اصطلاح{غلاف}\فرہنگ{غلاف}\حاشیہب{envelope}\فرہنگ{envelope} کو ظاہر کرتی ہے جسے شکل میں نقطہ دار لکیر سے  دکھایا گیا ہے۔
\begin{figure}
\centering
\pgfplotsset{scaled x ticks=false,scaled y ticks=false}
\begin{tikzpicture}
\pgfmathsetmacro{\zA}{0.003}
\pgfmathsetmacro{\zC}{sqrt(1-\zA*\zA)}
\begin{axis}[axis lines*=middle,xlabel=$t$, ylabel=$y_f(t)$,ticks=none,
every axis x label/.style={
    at={(ticklabel* cs:1.15)},
    anchor=east,}, 
	every axis y label/.style={
    at={(ticklabel* cs:1.05)},
    anchor=east,}
]
\addplot[domain=0:1080,samples=5000]{e^(-\zA*x)*cos(\zC*x-90)};
\addplot[gray,dashed,domain=0:1080,samples=100]{e^(-\zA*x)}node[pos=0.2,above right]{$e^{-\zeta \omega_0 t}$};
\end{axis}%
\end{tikzpicture}%
\caption{قصری ارتعاش۔}
\label{شکل_عارضی_کم_قصری_فطری_حل}
\end{figure}
\حصہء{فاصل مقصور صورت، \عددی{\zeta=1}}
\اصطلاح{فاصل مقصور صورت} \عددی{\zeta=1} میں
\begin{align}
s_1=s_2=-\zeta \omega_0
\end{align}
حاصل ہوتے ہیں۔جب \عددی{s_1} اور \عددی{s_2} کی قیمتیں ایک دونوں کے برابر \عددی{(s_1=s_2)} ہوں تب عمومی فطری حل درج ذیل لکھا جاتا ہے
\begin{align}\label{مساوات_عارضی_فطری_فاصل_مقصور_حل}
y_f(t)=K_2 e^{-\zeta \omega_0 t}+K_3 t e^{-\zeta \omega_0 t}
\end{align}
جہاں دوسرے جزو کو \عددی{t} سے ضرب دیا گیا ہے۔مساوات کے مستقل \عددی{K_2} اور \عددی{K_3} کو ابتدائی معلومات سے حاصل کرتے ہوئے مخصوص حل حاصل کیا جاتا ہے۔
%=================
\ابتدا{مشق}
سلسلہ وار \عددی{ RLC} دور میں \عددی{R=\SI{2}{\ohm}}، \عددی{L=\SI{5}{\henry}} اور \عددی{C=\SI{4}{\farad}} ہیں۔تقصیری تناسب اور غیر تقصیری قدرتی تعدد دریافت کریں۔

جوابات:\عددی{\omega_0=\SI{0.2236}{\radian\per\second}}، \عددی{\zeta=0.8944}
\انتہا{مشق}
%=================
\ابتدا{مشق}
متوازی \عددی{ RLC} دور میں \عددی{R=\SI{2}{\ohm}}، \عددی{L=\SI{5}{\henry}} اور \عددی{C=\SI{4}{\farad}} ہیں۔تقصیری تناسب اور غیر تقصیری قدرتی تعدد دریافت کریں۔

جوابات:\عددی{\omega_0=\SI{0.2236}{\radian\per\second}}، \عددی{\zeta=0.2795}
\انتہا{مشق}
%=================
\ابتدا{مشق}
سلسلہ وار \عددی{ RLC} دور میں \عددی{R=\SI{4}{\ohm}} اور \عددی{L=\SI{12}{\henry}} ہیں۔ دور کا رد عمل \عددی{C=\SI{6}{\farad}}، \عددی{C=\SI{2}{\farad}} اور \عددی{C=\SI{3}{\farad}} کی صورت میں کیا ہو گا۔

جوابات:زیادہ قصری، کم قصری اور فاصل قصری۔
\انتہا{مشق}
%=================
%=================
\ابتدا{مثال}\شناخت{مثال_عارضی_مستطیل_دباو_الف}
شکل \حوالہ{شکل_عارضی_مستطیل_دباو_الف} میں \عددی{v_C(t)} دریافت کریں جہاں لمحہ \عددی{t=0} پر ابتدائی معلومات \عددی{i_L(0)=\SI{2}{\ampere}} اور \عددی{v_C(0)=\SI{4}{\volt}} ہیں۔
\begin{figure}
\centering
\begin{tikzpicture}
\draw(0,0) to [square voltage source,l={$12 u(t)\, \si{\volt}$}]++(0,\y) to [resistor,i={$i(t)$},l={$\SI{4}{\ohm}$}]++(\x,0) to [inductor,l={$\SI{12}{\henry}$}]++(\x,0) to [capacitor,l_={$\SI{2}{\farad}$}]++(0,-\y) to [short] (0,0);
\draw(2*\x+2*\dx,\y/2)node[right]{$\begin{aligned} &+ \\& v_C(t) \\ &-\end{aligned}$};
\end{tikzpicture}
\caption{مثال \حوالہ{مثال_عارضی_مستطیل_دباو_الف} کا دور۔}
\label{شکل_عارضی_مستطیل_دباو_الف}
\end{figure}

حل:دور کی کرخوف مساوات لمحہ \عددی{t=0} کے بعد لکھتے ہیں۔
\begin{align}\label{مساوات_عارضی_مثال_سلسلہ_وار_الف}
i(t)R+L\frac{\dif i(t)}{\dif t}+\frac{1}{C}\int_{-\infty}^t i(t) \dif t=12
\end{align} 
اس میں 
\begin{align*}
i(t)&=C\frac{\dif v_C(t)}{\dif t}\\
v_C&=\frac{1}{C}\int_{-\infty}^{t} i(t)\dif t
\end{align*}
پُر کرتے ہوئے
\begin{align}\label{مساوات_عارضی_مثال_سلسلہ_وار_ب}
RC \frac{\dif v_C(t)}{\dif t}+LC \frac{\dif^{\,2} v_C(t)}{\dif t^2}+v_C(t)=12
\end{align}
ملتا ہے۔آئیں مساوات \حوالہ{مساوات_عارضی_مثال_سلسلہ_وار_ب} کو حل کریں۔

دی گئی قیمتوں کو مساوات \حوالہ{مساوات_عارضی_مثال_سلسلہ_وار_ب} میں  پُر کرتے ہوئے ترتیب دینے سے درج ذیل ملتا ہے
\begin{align}\label{مساوات_عارضی_مثال_سلسلہ_وار_پ}
\frac{\dif^{\,2} v_C(t)}{\dif t^2}+\frac{1}{3}\frac{\dif v_C(t)}{\dif t}+\frac{v_C(t)}{24}=\frac{1}{2}
\end{align}
جس میں جبری تفاعل کو صفر کے برابر پُر کرنے سے متجانس مساوات
\begin{align}\label{مساوات_مثال_ہم_جنسی_دو_درجی}
\frac{\dif^{\,2} v_C(t)}{\dif t^2}+\frac{1}{3}\frac{\dif v_C(t)}{\dif t}+\frac{v_C(t)}{24}=0
\end{align}
حاصل ہوتی ہے۔مساوات \حوالہ{مساوات_عارضی_مثال_سلسلہ_وار_پ} میں جبری تفاعل ایک مستقل مقدار ہے لہٰذا جبری حل کو مستقل \عددی{y_j(t)=K_1} تصور کرتے ہوئے مساوات \حوالہ{مساوات_عارضی_مثال_سلسلہ_وار_پ} میں پُر کرتے ہوئے
\begin{align*}
\frac{\dif^{\,2} K_1}{\dif t^2}+\frac{1}{3}\frac{\dif K_1}{\dif t}+\frac{K_1}{24}=&\frac{1}{2}\\
0+0+\frac{K_1}{24}&=\frac{1}{2}
\end{align*}
حل کرنے سے
\begin{align*}
v_{C,j}(t)=K_1=\SI{12}{\volt}
\end{align*}
ملتا ہے۔یہی جواب شکل \حوالہ{شکل_عارضی_مستطیل_دباو_الف} کو دیکھ کر بھی اخذ کیا جا سکتا ہے جہاں لمحہ \عددی{t=0} کے بہت دیر بعد، برقرار  حالت کی صورت میں برق گیر کو کھلا دور تصور کرتے ہوئے صاف ظاہر ہے کہ برق گیر کا دباو عین داخلی دباو کے برابر ہو گا۔

مساوات \حوالہ{مساوات_مثال_ہم_جنسی_دو_درجی} میں دی گئی متجانس مساوات سے درج ذیل امتیازی مساوات حاصل ہوتی ہے
\begin{align*}
s^2+\frac{s}{3}+\frac{1}{24}=0
\end{align*}
جس سے \عددی{\omega_0=\tfrac{1}{\sqrt{24}}} اور \عددی{\zeta=\tfrac{2}{\sqrt{6}}=0.333} لکھا جا سکتا ہے۔چونکہ \عددی{\zeta<1} ہے لہٰذا یہ کم قصری مساوات ہے۔امتیازی مساوات کے حل درج ذیل ہیں۔
\begin{align*}
s_1&=-\frac{1}{6}-\frac{j}{6\sqrt{2}}\\
s_2&=-\frac{1}{6}+\frac{j}{6\sqrt{2}}
\end{align*}
ان قیمتوں کو استعمال کرتے ہوئے  مساوات \حوالہ{مساوات_عارضی_فطری_حل_کم_تقصیر_الف} کے تحت فطری حل
\begin{align*}
v_{C,f}(t)&=e^{-\frac{t}{6}}\left(c_1 \cos \frac{t}{6\sqrt{2}}+c_2 \sin \frac{t}{6\sqrt{2}}\right)
\end{align*}
ہے جہاں \عددی{\sigma=\omega_0\zeta=\tfrac{1}{6}} اور \عددی{\omega_d=\omega_0\sqrt{1-\zeta^2}=\frac{1}{6\sqrt{2}}} استعمال کئے گئے۔یوں عمومی حل درج ذیل ہو گا
\begin{gather}
\begin{aligned}\label{مساوات_عارضی_سلسلہ_وار_مکمل_حل_الف}
v_C(t)&=v_{C,j}(t)+v_{C,f}(t)\\
&=12+e^{-\frac{t}{6}}\left(c_1 \cos \frac{t}{6\sqrt{2}}+c_2 \sin \frac{t}{6\sqrt{2}}\right)
\end{aligned}
\end{gather}
جس میں مستقل \عددی{c_1} اور \عددی{c_2} معلوم کرنا باقی ہے۔ابتدائی دباو \عددی{v_C(0)=\SI{4}{\volt}} کو عمومی حل میں پُر کرنے سے
\begin{align*}
4&=12+e^{-\frac{0}{6}}\left(c_1 \cos \frac{0}{6\sqrt{2}}+c_2 \sin \frac{0}{6\sqrt{2}}\right)\\
&=12+c_1
\end{align*}
یعنی
\begin{align}\label{مساوات_عارضی_مستقل_الف}
c_1=-8
\end{align}
ملتا ہے۔ابتدائی رو \عددی{i_L(0)=\SI{2}{\ampere}} کو استعمال کرنے کی خاطر مساوات \حوالہ{مساوات_عارضی_سلسلہ_وار_مکمل_حل_الف} کے دونوں اطراف کو \عددی{C} سے ضرب دیتے ہوئے تفرق لیتے ہیں۔
\begin{multline*}
C \frac{\dif v_C(t)}{\dif t}=-\frac{C}{6}e^{-\frac{t}{6}}\left(-8 \cos \frac{t}{6\sqrt{2}}+c_2 \sin \frac{t}{6\sqrt{2}}\right)\\+\frac{C}{6\sqrt{2}}e^{-\frac{t}{6}}\left(8 \sin \frac{t}{6\sqrt{2}}+c_2 \cos \frac{t}{6\sqrt{2}}\right)
\end{multline*}
اس میں \عددی{C\tfrac{\dif v_C(t)}{\dif t}=i_C(t)} کے برابر ہے لہٰذا
\begin{multline*}
i_C(t)=-\frac{1}{3}e^{-\frac{t}{6}}\left(-8 \cos \frac{t}{6\sqrt{2}}+c_2 \sin \frac{t}{6\sqrt{2}}\right)\\+\frac{\sqrt{2}}{6}e^{-\frac{t}{6}}\left(8 \sin \frac{t}{6\sqrt{2}}+c_2 \cos \frac{t}{6\sqrt{2}}\right)
\end{multline*}
لکھا جا سکتا ہے جہاں بائیں ہاتھ \عددی{i_C(t)} کے برابر ہے اور دائیں ہاتھ \عددی{C=2} پُر کیا گیا ہے۔چونکہ \عددی{L} اور \عددی{C} سلسلہ وار جڑے ہیں لہٰذا \عددی{i_C(t)=i_L(t)} ہو گا۔ درج بالا مساوات میں ابتدائی رو \عددی{i_L(0)=i_C(0)=\SI{2}{\ampere}} پُر کرتے ہوئے
\begin{multline*}
2=-\frac{1}{3}e^{-\frac{0}{6}}\left(-8 \cos \frac{0}{6\sqrt{2}}+c_2 \sin \frac{0}{6\sqrt{2}}\right)\\
+\frac{\sqrt{2}}{6}e^{-\frac{0}{6}}\left(8 \sin \frac{0}{6\sqrt{2}}+c_2 \cos \frac{0}{6\sqrt{2}}\right)
\end{multline*}
یعنی
\begin{align*}
c_2=-\sqrt{8}
\end{align*}
ملتا ہے۔مساوات کے مستقل جانتے ہوئے مخصوص حل کو درج ذیل لکھا جا سکتا ہے۔
\begin{align}
v_C(t)=12+e^{-\frac{t}{6}}\left(-8\cos \frac{t}{9\sqrt{2}}-\sqrt{8}\sin \frac{t}{9\sqrt{2}} \right)
\end{align}
اس مساوات سے \عددی{t=\SI{0}{\second}} پر \عددی{v_C=\SI{4}{\volt}} اور \عددی{t=\infty} پر \عددی{v_C=\SI{12}{\volt}} حاصل ہوتا ہے۔پہلا جواب ابتدائی دباو ہی ہے جبکہ دوسرا جواب ابدی برقرار حالت یعنی جبری حل ہے۔ 
\انتہا{مثال}
%================
\ابتدا{مثال}\شناخت{مثال_عارضی_مستطیل_دباو_ب}
شکل \حوالہ{شکل_عارضی_مستطیل_دباو_ب} میں سوئچ ازل سے دکھائے گئے حالت میں ہے۔لمحہ \عددی{t=0} پر اس کو پلٹایا جاتا ہے۔دور کا رد عمل \عددی{R_1=\SI{15}{\ohm}}، \عددی{R_2=\SI{5}{\ohm}}، \عددی{L=\SI{2}{\henry}} اور \عددی{C=\SI{0.5}{\farad}} کی صورت میں معلوم کریں۔
\begin{figure}
\centering
\begin{tikzpicture}
\draw(0,0) node[spdt,xscale=-1](ksp){};
\draw(ksp.in) to [inductor,i={$i(t)$},l={$L$}]++(\x,0) to [resistor,l={$R_1$}]++(\x,0)coordinate(kUR) to [capacitor,l_={$C$}]++(0,-\yy) to [short]++(-4*\x,0)coordinate(kBL) to [american voltage source,l={$\SI{20}{\volt}$}]++(0,\y) |-(ksp.out 1);
\draw(kBL)++(\x,0) to [american voltage source,*-,l={$\SI{10}{\volt}$}]++(0,\y) |-(ksp.out 2);
\draw(kUR) to [short,*-]++(\x,0) to [resistor,l={$R_2$}]++(0,-\yy) to [short,-*]++(-\x,0)coordinate(kCapB);
\draw(kCapB)++(2*\dx,\yy/2)node[right]{$\begin{aligned} &+ \\ & v_C(t) \\ &- \end{aligned}$};
\end{tikzpicture}
\caption{مثال \حوالہ{مثال_عارضی_مستطیل_دباو_ب} کا دور۔}
\label{شکل_عارضی_مستطیل_دباو_ب}
\end{figure}

حل:لمحہ \عددی{t=\SI{0}{\second}} کے بعد دور کے کرخوف مساوات لکھتے ہیں۔
\begin{align*}
L\frac{\dif i_(t)}{\dif t}+R_1 i_(t)+v_C(t)&=10\\
C\frac{\dif v_C(t)}{\dif t}+\frac{v_C(t)}{R_2}&=i(t)
\end{align*}
نچلی مساوات کی رو کو بالائی مساوات  میں پُر کرتے  ہوئے
\begin{align*}
L\left[C\frac{\dif^{\,2} v_C(t)}{\dif t^2}+\frac{1}{R_2}\frac{\dif v_C(t)}{\dif t}\right]+R_1 \left[C\frac{\dif v_C(t)}{\dif t}+\frac{v_C(t)}{R_2} \right]+v_C(t)&=10
\end{align*}
یعنی
\begin{align*}
\frac{\dif^{\,2} v_C(t)}{\dif t^2}+\left[\frac{1}{R_2 C}+\frac{R_1}{L}\right]\frac{\dif v_C(t)}{\dif t}+\frac{R_1}{R_2 LC}v_C(t)=\frac{10}{LC}
\end{align*}
ملتا ہے۔پرزوں کی قیمتیں پُر کرنے سے
\begin{align}\label{مساوات_عارضی_زیادہ_قصری_مثال_الف}
\frac{\dif^{\,2} v_C(t)}{\dif t^2}+7.9 \frac{\dif v_C(t)}{\dif t}+3 v_C(t)=10
\end{align}
حاصل ہوتا ہے جس سے \عددی{\omega_0=\sqrt{3}} اور \عددی{\zeta=2.28} ملتے ہیں۔چونکہ \عددی{\zeta>1} ہے لہٰذا دور زیادہ قصری ہے۔مستقل جبری قوت کی بنا \عددی{v_{C,j}(t)=K_1} متوقع ہے جسے مندرجہ بالا مساوات میں پُر کرنے سے
\begin{align*}
v_{C,j}=\frac{10}{3} \, \si{\volt}
\end{align*}
حاصل ہوتا ہے۔مساوات \حوالہ{مساوات_عارضی_زیادہ_قصری_مثال_الف} میں جبری قوت کو صفر پُر کرنے، یعنی دائیں ہاتھ کو صفر کے برابر پُر کرنے، سے درج ذیل متجانس مساوات حاصل ہو گی
\begin{align*}
\frac{\dif^{\,2} v_C(t)}{\dif t^2}+7.9 \frac{\dif v_C(t)}{\dif t}+3 v_C(t)=0
\end{align*}
جس کا متوقع حل \عددی{v_{C,f}=e^{st}} ہے۔متوقع حل کو متجانس مساوات میں پُر کرتے ہوئے
\begin{align*}
s^2 e^{st}+7.9 s e^{st}+3 e^{st}=0
\end{align*}
حاصل ہوتا ہے جس کے دونوں اطراف کو \عددی{e^{st}} سے تقسیم کرنے سے درج ذیل امتیازی مساوات حاصل ہوتی ہے
\begin{align*}
s^2+7.9s+3=0
\end{align*}
جس کے حل
\begin{align*}
s_1&=\frac{-1-\sqrt{7.9^2-4\times 3}}{2}=-7.5\\
s_2&=\frac{-1+\sqrt{7.9^2-4\times 3}}{2}=-0.4
\end{align*}
ہیں۔یوں فطری حل درج ذیل ہو گا
\begin{align*}
v_{C,f}=c_1 e^{-7.5t}+c_2e^{-0.4t}
\end{align*}
اور عمومی حل
\begin{gather}
\begin{aligned}\label{مساوات_عارضی_زیادہ_قصری_مثال_ب}
v_C(t)&=v_{C,j}(t)+v_{C,f}(t)\\
&=\frac{10}{3}+c_1 e^{-7.5t}+c_2e^{-0.4t}
\end{aligned}
\end{gather}
ہو گا۔

مساوات کے مستقل حاصل کرنے کے لئے ابتدائی معلومات درکار ہیں۔لمحہ \عددی{t=0} سے پہلے \عددی{\SI{20}{\volt}} کی منبع دور کو طاقت فراہم کر رہی تھی۔اس برقرار صورت میں برق گیر کو کھلا دور اور امالہ گیر کو قصر دور تصور کرتے ہوئے
\begin{align*}
v_C(0_-)=v_C(0_+)=20 \left(\frac{5000}{15000+5000}\right)=\SI{5}{\volt}\\
i(0_-)=i(0_+)=\frac{20-v_C}{R_1}=\frac{20-5}{15000}=\SI{1}{\milli\ampere}
\end{align*}
ملتے ہیں۔ ابتدائی دباو کو مساوات \حوالہ{مساوات_عارضی_زیادہ_قصری_مثال_ب} میں پُر کرتے ہوئے
  \begin{align*}
5&=\frac{10}{3}+c_1 e^{-7.5\times 0}+c_2e^{-0.4\times 0}
\end{align*}
یعنی
\begin{align}\label{مساوات_عارضی_مستقل_مثال_دس_الف}
c_1+c_2=\frac{5}{3}
\end{align}
ملتا ہے۔

مساوات \حوالہ{مساوات_عارضی_زیادہ_قصری_مثال_ب} کو \عددی{C} سے ضرب دے کر اس کا تفرق لیتے ہوئے
\begin{align*}
C \frac{\dif v_C(t)}{\dif t}&=0-0.5\times 7.5c_1 e^{-7.5t}-0.5\times 0.4 c_2e^{-0.4t}
\end{align*}
یعنی
\begin{align*}
i_C(t)&=-3.75c_1 e^{-7.5t}-0.2 c_2e^{-0.4t}
\end{align*}
ملتا ہے۔لمحہ \عددی{t=0_+} پر برق گیر کی رو درج بالا مساوات سے
\begin{align*}
i_C(0_+)&=-3.75c_1 e^{-7.5\times 0}-0.2 c_2e^{-0.4\times 0}\\
&=-3.75c_1-0.2 c_2
\end{align*}
حاصل ہوتی ہے جبکہ اسی لمحے پر \عددی{R_2} کی رو درج ذیل ہو گی۔
\begin{align*}
i_{R2}(0_+)&=\frac{v_C(0_+)}{R_2}=\frac{5}{5000}=\SI{1}{\milli\ampere}
\end{align*}
چونکہ \عددی{i_L(t)=i(t)} ہی ہے لہٰذا کرخوف مساوات رو کے تحت
\begin{align*}
i_L(0+)&=i_C(0+)+i_{R2}(0_+)\\
0.001&=0.001-3.75c_1-0.2c_2
\end{align*}
یعنی
\begin{align}\label{مساوات_عارضی_مستقل_مثال_دس_ب}
c_1+c_2=0
\end{align}
ہو گا۔مساوت \حوالہ{مساوات_عارضی_مستقل_مثال_دس_الف} اور مساوات \حوالہ{مساوات_عارضی_مستقل_مثال_دس_ب} ہمزاد مساوات کو حل کرتے ہوئے درج ذیل حاصل ہوتا ہے۔
\begin{align*}
c_1&=-\frac{20}{213}\\
c_2&=\frac{125}{71}
\end{align*}
یوں مخصوص حل درج ذیل ہے۔
\begin{align}
v_C(t)&=\frac{10}{3}-\frac{20}{213} e^{-7.5t}+\frac{125}{71}e^{-0.4t}
\end{align}
یہ مساوات \عددی{t=0_+} پر متوقع جوابات \عددی{v_C(0_+)=\SI{5}{\volt}} اور \عددی{t=\infty} پر \عددی{v_C(\infty)=\tfrac{10}{3}\, \si{\volt}} دیتی ہے۔
\انتہا{مثال}
%==========================
\ابتدا{مثال}\شناخت{مثال_عارضی_مستطیل_دباو_پ}
شکل \حوالہ{شکل_عارضی_مستطیل_دباو_پ} میں لمحہ \عددی{t=0} پر سوئچ کو امالہ گیر پر لے جایا جاتا ہے۔\عددی{v_0(t)} دریافت کریں۔پرزوں کی قیمتیں \عددی{R=\SI{20}{\ohm}}، \عددی{C=\SI{0.04}{\farad}} اور \عددی{L=\SI{4}{\henry}} ہیں۔
\begin{figure}
\centering
\begin{subfigure}{0.5\textwidth}
\centering
\begin{tikzpicture}
\draw(0,0) node[spdt,xscale=-1](ks){};
\draw(ks.out 1)--++(-\x,0)++(0,-2*\y)coordinate(kBL)node[ground]{} to [american voltage source,l={$\SI{20}{\volt}$}]++(0,2*\y);
\draw(kBL)++(\x,0)node[ground]{} to [inductor,l={$L$}]++(0,\y) |-(ks.out 2);
\draw(ks.in) to [resistor,i={$i(t)$},l={$R$}]++(\x,0)coordinate(kUR) to [capacitor,l_={$C$}]++(0,-\y)++(0,-\y)node[ground]{} to [american voltage source,l_={$\SI{5}{\volt}$}]++(0,\y);
\draw(kUR)++(2*\dx,-\y/2)node[right]{$\begin{aligned} &+ \\ &v_C(t) \\ &-\end{aligned}$};
\draw(kUR) to [short,*-o]++(\x/2,0)node[right]{$v_0(t)$};
\end{tikzpicture}
\caption*{(الف)}
\end{subfigure}
\begin{subfigure}{0.5\textwidth}
\centering
\begin{tikzpicture}
\draw(0,0)node[ground]{} to [american voltage source,l={$\SI{20}{\volt}$}]++(0,\y+\y/2) to [short]++(\x,0) to [resistor,l={$R$}]++(\x,0) to [short,-o]++(\x/2,0)++(0,-1.5*\y)node[ground]{} to [american voltage source,-o,l={$\SI{5}{\volt}$}]++(0,\y);
\draw(\x,0) node[ground]{} to [inductor,-o,i<_={$i_L(0_-)$},l={$L$}]++(0,\y);
\draw(2.5*\x,\y+\y/4)node[right]{$\begin{aligned}&+ \\ &v_C(0_-) \\ &-  \end{aligned}$};
\end{tikzpicture}
\caption*{(ب)}
\end{subfigure}
\caption{مثال \حوالہ{مثال_عارضی_مستطیل_دباو_پ} کا دور۔}
\label{شکل_عارضی_مستطیل_دباو_پ}
\end{figure}

حل:سوئچ امالہ پر کرنے کے بعد کرخوف مساوات لکھتے ہیں
\begin{align*}
v_C(t)+R i(t)+L\frac{\dif i(t)}{\dif t}+5=0
\end{align*}
جہاں
\begin{align*}
i(t)=C\frac{\dif v_C(t)}{\dif t}
\end{align*}
کے برابر ہے۔درج بالا دو مساوات کو ملاتے ہوئے
\begin{align*}
v_C(t)+RC\frac{\dif v_C(t)}{\dif t}+LC\frac{\dif ^{\, 2}v_C(t)}{\dif t^2}+5=0
\end{align*}
ملتا ہے  جسے ترتیب دیتے ہوئے درج ذیل لکھا جا سکتا ہے۔
\begin{align*}
\frac{\dif ^{\, 2}v_C(t)}{\dif t^2}+\frac{R}{L}\frac{\dif v_C(t)}{\dif t}+\frac{v_C(t)}{LC}=-\frac{5}{LC}
\end{align*}
پرزوں کی قیمتیں پُر کرنے سے
\begin{align*}
\frac{\dif ^{\, 2}v_C(t)}{\dif t^2}+5\frac{\dif v_C(t)}{\dif t}+6.25 v_C(t)=-31.25
\end{align*}
حاصل ہوتا ہے جس سے \عددی{\omega_0=\SI{2.5}{\radian\per\second}}، \عددی{\zeta=1}، جبری حل 
\begin{align*}
v_{C,j}=K_1=\SI{-5}{\volt}
\end{align*}
 اور متجانس مساوات درج ذیل ملتا ہے۔
\begin{align*}
\frac{\dif ^{\, 2}v_C(t)}{\dif t^2}+5\frac{\dif v_C(t)}{\dif t}+6.25 v_C(t)=0
\end{align*}
متجانس مساوات میں \عددی{e^{st}} پُر کرتے ہوئے درج ذیل حاصل کیا جا سکتا ہے
\begin{align}
s^2+5s+6.25=0
\end{align}
جس کا حل درج ذیل ہے۔
\begin{align*}
s_1=s_2=-2.5
\end{align*}
\عددی{\zeta=1} کے تحت  دور فاصل قصری ہے اور \عددی{s_1=s_2} ہی متوقع تھا۔ فاصل قصری مساوات کا فطری حل درج ذیل ہے۔
\begin{align*}
v_{C,f}(t)=c_1e^{-2.5t}+c_2 t e^{-2.5t}
\end{align*}
یوں عمومی حل
\begin{align}\label{مساوات_عارضی_مکمل_حل_مثال_دو_درجی_ب}
v_C(t)=-5+(c_1+tc_2)e^{-2.5t}
\end{align}
ہو گا۔عمومی حل کے مستقل ابتدائی معلومات سے حاصل کی جا سکتی ہیں۔ابتدائی معلومات سوئچ ہلانے سے پہلے برقرار حال سے ملتی ہیں۔لمحہ \عددی{t=0} سے پہلے برقرار صورت میں برق گیر کو کھلا دور تصور کرتے ہوئے شکل-ب ملتا ہے جہاں سے
\begin{align*}
v_C(0_-)&=v_C(0_+)=20-5=\SI{15}{\volt}\\
i_L(0_-)&=i_L(0_+)=\SI{0}{\ampere}
\end{align*}
لکھا جا سکتا ہے۔مساوات \حوالہ{مساوات_عارضی_مکمل_حل_مثال_دو_درجی_ب} میں \عددی{t=0} پر \عددی{v_C(0_+)} پُر کرنے
\begin{align*}
15=-5+(c_1+0\times c_2)e^{-2.5 \times 0}
\end{align*}
سے
\begin{align*}
c_1=20
\end{align*}
حاصل ہوتا ہے۔مساوات \حوالہ{مساوات_عارضی_مکمل_حل_مثال_دو_درجی_ب} کو استعمال کرتے ہوئے
\begin{align*}
i(t)&=C\frac{\dif v_C(t)}{\dif t}\\
&=0.04\times  (-2.5c_1+c_2-2.5tc_2)e^{-2.5 t}
\end{align*}
لکھا جا سکتا ہے۔لمحہ \عددی{t=0} کے بعد شکل-الف کو دیکھتے ہوئے \عددی{i_L(t)=-i(t)} لکھا جا سکتا ہے۔یوں امالہ گیر کی ابتدائی رو سے لمحہ \عددی{t=0_+} پر \عددی{i(0_+)=-i_L(0_+)=0} کو درج بالا مساوات میں پُر کرتے
\begin{align*}
&=0.04\times  (-2.5c_1+c_2-2.5 \times 0 \times c_2)e^{-2.5 \times 0}
\end{align*}
ہوئے
\begin{align*}
c_2=50
\end{align*}
ملتا ہے۔یوں مخصوص حل کو درج ذیل لکھا جا سکتا ہے۔
\begin{align*}
v_C(t)&=v_{C,j}(t)+v_{C,f}(t)\\
&=-5+(20+50 t)e^{-2.5t} \, \si{\volt}
\end{align*}
ہمیں \عددی{v_0(t)} درکار ہے  جسے شکل-الف سے دیکھ کر لکھتے ہیں۔
\begin{align}
v_0(t)=5+v_C(t)=(20+50 t)e^{-2.5t} \, \si{\volt}
\end{align}
\انتہا{مثال}
%===========================
\ابتدا{مشق}\شناخت{مشق_عارضی_دو_درجی_الف}
شکل \حوالہ{شکل_عارضی_دو_درجی_الف} میں \عددی{v_0(t)} دریافت کریں۔پرزوں کی قیمتیں \عددی{R_1=\SI{8}{\ohm}}، \عددی{R_2=\SI{22}{\ohm}}، {\عددی{L=\SI{4}{\henry}}} اور \عددی{C=\SI{0.04}{\farad}} ہیں۔

\begin{figure}
\centering
\begin{tikzpicture}
\draw(0,0)node[spdt,xscale=-1](ksp){};
\draw(ksp.out 1) to [resistor,l_={$R_1$}]++(-\x,0)++(0,-2.5*\y)coordinate(kBL) to [american voltage source,l={$\SI{40}{\volt}$}]++(0,2.5*\y);
\draw(kBL) to [short]++(3*\x,0) to [american voltage source,l={$\SI{20}{\volt}$}]++(0,\y) to [resistor,l={$R_2$}]++(0,\y)coordinate(kR2);
\draw(ksp.in) to [inductor,i={$i_0(t)$}]++(\x,0) -| (kR2);
\draw(kBL)++(3/4*\x,0) to [american voltage source,*-,l_={$\SI{5}{\volt}$}]++(0,\y) to [capacitor,l_={$C$}]++(0,\y) |-(ksp.out 2);
\draw(kBL)++(3*\x+\x/2,\y+\y/8)node{$\begin{aligned} &+ \\ \\ \\ \\&v_0(t)  \\ \\ \\ \\&-\end{aligned}$};
\end{tikzpicture}
\caption{مشق \حوالہ{مشق_عارضی_دو_درجی_الف} کا دور۔}
\label{شکل_عارضی_دو_درجی_الف}
\end{figure}

جوابات:\عددی{i_0(t)=2.77e^{-3.8964t}-2.103e^{-1.6043t}}، \عددی{v_0(t)=20+i_0(t)R_2}
\انتہا{مشق}
%========================
\ابتدا{مشق}\شناخت{مشق_عارضی_دو_درجی_ب}
شکل \حوالہ{شکل_عارضی_دو_درجی_ب} میں سوئچ چالو کرنے کے بعد \عددی{i(t)} دریافت کریں۔

\begin{figure}
\centering
\begin{tikzpicture}
\draw(0,0) to [short]++(\x,0) to [american voltage source,l={$\SI{30}{\volt}$}]++(0,\y) to [resistor,l={$\SI{25}{\ohm}$}]++(0,\y) to [cspst,l_={${t=0}$}]++(0,\y) to [resistor,l={$\SI{5}{\ohm}$}]++(-\x,0) to [american voltage source,l_={$\SI{10}{\volt}$}]++(0,-3*\y);
\draw(\x,0) to [short,*-]++(2*\x,0) to [capacitor,l={$\SI{0.02}{\farad}$}]++(0,3*\y) to [short,-*]++(-2*\x,0);
\draw(2*\x,0) to [inductor,*-*,i<_={$i(t)$},l={$\SI{4}{\henry}$}]++(0,3*\y);
\end{tikzpicture}
\caption{مشق \حوالہ{مشق_عارضی_دو_درجی_ب} کا دور۔}
\label{شکل_عارضی_دو_درجی_ب}
\end{figure}

جواب:\عددی{i(t)=3+1.3035e^{-17.071t}-6.035e^{-2.9289t}}
\انتہا{مشق}
%=========================
آئیں عارضی ردعمل کے چند دلچسپ مثال دیکھیں۔

%========================
\ابتدا{مثال}\شناخت{مثال_عارضی_مزاحمت_برق_گیر_الف}
صفحہ \حوالہصفحہ{مثال_عارضی_یک_درجی_دور_الف} پر مثال \حوالہ{مثال_عارضی_یک_درجی_دور_الف} میں سلسلہ وار جڑے مزاحمت اور بے بار برق گیر کو لمحہ \عددی{t=0} پر \عددی{V_I} وولٹ کے منبع دباو کے ساتھ جوڑا گیا۔برق گیر پر دباو صفر وولٹ سے بڑھتے بڑھتے آخر کار \عددی{V_I} تک پہنچتی ہے۔اس دور میں مزاحمت کی قیمت کم کرنے سے ابتدائی رو کی قیمت بڑھتی ہے حتٰی کہ \عددی{R=\SI{0}{\ohm}} کی صورت میں، توقع کے عین مطابق، لامحدود قیمت کی ابتدائی رو حاصل ہوتی ہے۔حقیقی ادوار میں مزاحمت کو بالکل صفر اوہم کرنا ناممکن ہوتا ہے لہٰذا حقیقت میں لامحدود رو کی بجائے انتہائی زیادہ رو پائی جائے گی جو یا تو سوئچ کو اور یا برق گیر کو تباہ کر دے گی۔

آئیں مزاحمت کی جگہ امالہ گیر نسب کرتے ہوئے صورت حال دیکھیں۔شکل \حوالہ{شکل_عارضی_مزاحمت_برق_گیر_الف}-الف میں بے بار برق گیر کے ساتھ امالہ گیر سلسلہ وار جڑا ہے۔لمحہ \عددی{t=0} پر انہیں مستقل منبع دباو \عددی{V_I} کے ساتھ جوڑا جاتا ہے۔\عددی{v_0(t)} دریافت کریں۔
\begin{figure}
\centering
\begin{subfigure}{0.5\textwidth}
\begin{tikzpicture}
\draw(0,0) to [american voltage source,l={$V_I$}]++(0,\y) to [cspst,l={${t=\SI{0}{\second}}$}]++(\x,0) to [inductor,i>_={$i(t)$},l={$L$}]++(\x,0)node[above]{$v_0(t)$} to [capacitor,l={$C$}]++(0,-\y) to [short,-*] (0,0)node[ground]{};
\end{tikzpicture}%
\caption*{(الف)}
\end{subfigure}
\begin{subfigure}{0.5\textwidth}
\pgfplotsset{scaled y ticks=false, scaled x ticks=false}
\begin{tikzpicture}
\begin{axis}[axis lines*=middle,xlabel=$\frac{t}{\sqrt{LC}}$,ylabel=$v_0(t)$,ytick={1,2},yticklabels={$V_I$,$2V_I$},xtick={180,360,720},xticklabels={$\pi$,$2\pi$,$4\pi$},
every axis x label/.style={
    at={(ticklabel* cs:1.15)},
    anchor=east,}, 
	every axis y label/.style={
    at={(ticklabel* cs:1.05)},
    anchor=east,}
]
\addplot[domain=0:360*2.2,samples=100]{1-cos(x)}node[pos=0.3,above right]{$v_0(t)$};
\addplot[gray,domain=0:360*2.2,samples=100]{0.7*sin(x)}node[pos=0.15, right]{$i(t)$};
\end{axis}
\end{tikzpicture}%
\caption*{(ب)}
\end{subfigure}%
\caption{مثال \حوالہ{مثال_عارضی_مزاحمت_برق_گیر_الف} کا اشکال۔}
\label{شکل_عارضی_مزاحمت_برق_گیر_الف}
\end{figure}

حل:سوئچ چالو کرنے سے پہلے برق گیر بے بار ہے لہٰذا اس پر دباو بھی صفر وولٹ ہو گا۔اسی طرح امالہ گیر کی ابتدائی رو صفر ہے۔
\begin{gather}
\begin{aligned}\label{مساوات_عارضی_ابتدائی_معلومات_امالہ_برق_گیر_الف}
v_C(0_+)&=\SI{0}{\volt}\\
i_L(0_+)&=\SI{0}{\ampere}
\end{aligned}
\end{gather}
سوئچ چالو کرنے کے بعد کی کرخوف مساوات لکھتے ہیں۔
\begin{align}\label{مساوات_عارضی_امالہ_برق_گیر_الف}
L\frac{\dif i(t)}{\dif t}+\frac{1}{C}\int_{0}^{t} i(t)\dif t+v_C(0_+)=V_I
\end{align}
مساوات \حوالہ{مساوات_عارضی_ابتدائی_معلومات_امالہ_برق_گیر_الف} کے ابتدائی معلومات کو استعمال کرتے ہوئے ہم دیگر ابتدائی معلومات درج بالا مساوات سے حاصل کر سکتے ہیں۔ لمحہ \عددی{t=0_+} یعنی سوئچ چالو کرنے کے فوراً بعد، درج بالا مساوات میں ابتدائی معلومات پُر کرتے ہوئے  حل کرنے سے
\begin{align*}
L\frac{\dif i(0_+)}{\dif t}+\frac{1}{C}\int_{0}^{0_+} i(t)\dif t+v_C(0_+)&=V_I\\
L\frac{\dif i(0_+)}{\dif t}+0+0+0&=V_I
\end{align*}
یعنی
\begin{align}
\frac{\dif i(0_+)}{\dif t}=\frac{V_I}{L}
\end{align}
حاصل ہوتا ہے جو ابتدائی شرح رو ہے۔یہی جواب، \عددی{v_C(0_+)=\SI{0}{\volt}} تصور کرتے ہوئے، شکل \حوالہ{شکل_عارضی_مزاحمت_برق_گیر_الف} کو دیکھ کر لکھا جا سکتا ہے۔

مساوات \حوالہ{مساوات_عارضی_امالہ_برق_گیر_الف} میں تکمل کا نشان ختم کرنے کی خاطر تفرق لیتے ہوئے
\begin{align*}
\frac{\dif^{\, 2} i(t)}{\dif t^2}+\frac{i}{LC}&=0
\end{align*}
تفرقی مساوات حاصل ہوتی ہے جس سے جبری حل
\begin{align*}
i_J(t)=K_1=0
\end{align*}
حاصل ہوتا ہے۔تفرقی مساوات سے درج ذیل امتیازی مساوات حاصل ہوتی ہے
\begin{align*}
s^2+\frac{1}{LC}=0
\end{align*}
جس کے حل درج ذیل ہیں۔
\begin{align*}
s_1&=\frac{j}{\sqrt{LC}}\\
s_2&=-\frac{j}{\sqrt{LC}}
\end{align*}
یوں فطری حل درج ذیل لکھا جا سکتا ہے۔
\begin{align*}
i_F(t)&=A e^{j\frac{t}{\sqrt{LC}}}+Be^{-j\frac{t}{\sqrt{LC}}}\\
&=(A+B) \cos \frac{t}{\sqrt{LC}}+j(A-B) \sin \frac{t}{\sqrt{LC}}\\
&=c_1\cos \frac{t}{\sqrt{LC}}+c_2\sin \frac{t}{\sqrt{LC}}
\end{align*}
عمومی حل
\begin{align*}
i(t)=i_J(t)+i_F(t)=c_1\cos \frac{t}{\sqrt{LC}}+c_2\sin \frac{t}{\sqrt{LC}}
\end{align*}
ہو گا۔مساوات کے مستقل ابتدائی معلومات سے حاصل کئے جاتے ہیں۔لمحہ \عددی{t=0_+} پر \عددی{i(0_+)=i_L(0_+)=0} پرُ کرتے ہوئے \عددی{c_1=0} حاصل ہوتا ہے۔درج بالا مساوات میں \عددی{c_1=0} پُر کرتے ہوئے تفرق لیتے ہوئے
\begin{align*}
\frac{\dif i(t)}{\dif t}=\frac{c_2}{\sqrt{LC}}\cos \frac{t}{\sqrt{LC}}
\end{align*}
ابتدائی \عددی{\tfrac{\dif i(0_+)}{\dif t}} پُر کرنے سے
\begin{align*}
\frac{V_I}{L}=\frac{c_2}{\sqrt{LC}}\cos \frac{0}{\sqrt{LC}}
\end{align*}
مستقل کی قیمت \عددی{c_2=V_1\sqrt{\tfrac{C}{L}}} حاصل ہوتی ہے۔یوں مخصوص حل درج ذیل ہے۔
\begin{align}\label{مساوات_عارضی_امالہ_برق_گیر_رو}
i(t)=V_I\sqrt{\frac{C}{L}} \sin \frac{t}{\sqrt{LC}}
\end{align}
اس مساوات کو استعمال کرتے ہوئے برق گیر پر دباو  \عددی{v_0(t)} درج ذیل مساوات
\begin{align*}
v_0&=\frac{1}{C}\int_{0}^{t} i(t) \dif t+v_C(0_+)
\end{align*}
سے
\begin{align}\label{مساوات_عارضی_امالہ_برق_گیر_دگنا_دباو}
v_0=V_I\left(1-\cos \frac{t}{\sqrt{LC}} \right)
\end{align}
حاصل کرتے ہیں جسے شکل \حوالہ{شکل_عارضی_مزاحمت_برق_گیر_الف}-ب میں دکھایا گیا ہے۔

مساوات \حوالہ{مساوات_عارضی_امالہ_برق_گیر_دگنا_دباو} میں حاصل نتیجہ جسے شکل \حوالہ{شکل_عارضی_مزاحمت_برق_گیر_الف}-ب میں دکھایا گیا  ہے غور طلب ہے۔اس مساوات کے تحت جب بھی برق گیر کو سوئچ کے ذریعے منبع دباو کے ساتھ جوڑا جائے، برق گیر پر منبع دباو کی دگنی چوٹی حاصل ہو گی۔اسی شکل میں ہلکی سیاہی سے مساوات \حوالہ{مساوات_عارضی_امالہ_برق_گیر_رو} کو بھی دکھایا گیا ہے۔دباو کی چوٹی عین اس وقت پائی جاتی ہے جب رو کی قیمت صفر ہو۔

قوی برقیات میں \اصطلاح{بدلتی رو}\فرہنگ{بدلتی رو}\حاشیہب{alternating current, AC}\فرہنگ{alternating current, AC}\فرہنگ{AC} سے \اصطلاح{یک سمتی رو}\فرہنگ{یک سمتی رو}\حاشیہب{direct current, DC}\فرہنگ{direct current, DC}\فرہنگ{DC} بذریعہ \اصطلاح{سمت کار}\فرہنگ{سمت کار}\حاشیہب{rectifier}\فرہنگ{rectifier} حاصل کی جاتی ہے۔سمت کار صرف ایک سمت میں رو گزارتا ہے۔یوں عین اس لمحہ جب دور میں رو کی قیمت منفی ہونے کی کوشش کرے، سمت کار رو گزارنا روک دیتا ہے اور برق گیر دگنی دباو پر رہ جاتا ہے۔قوی برقیات کے میدان میں اس حقیقت کا خاص خیال رکھنا ضروری ہے اور جہاں اس دگنی دباو کی پہنچ ہو، وہاں استعمال کئے گئے پرزوں کی استعداد دگنی دباو سے زیادہ ہونی لازمی ہے۔یوں \عددی{\SI{100}{\volt}} کی یک سمتی منبع کے ساتھ کم از کم \عددی{\SI{200}{\volt}} پر کام کرنے والا برق گیر استعمال کیا جائے گا۔ یہاں یہ بتلانا بھی ضروری ہے آپ کسی صورت یہ نہ فرض کر لیں کہ چونکہ آپ نے دور میں امالہ نسب نہیں کیا لہٰذا آپ کو اس مسئلے سے واسطہ نہیں ہے چونکہ منبع اور برق گیر کو آپس میں جوڑنے والی تار اذ خود بطور امالہ گیر کردار ادا کرتی ہے۔منبع دباو اور برق گیر کو بغیر تار کے آپس میں جوڑنے سے بھی منبع دباو اور برق گیر کی اندرونی لمبائی جس سے رو گزرتی ہے بطور امالہ گیر کردار ادا کرے گی۔مساوات \حوالہ{مساوات_عارضی_امالہ_برق_گیر_دگنا_دباو} سے ظاہر ہے کہ امالہ کی قیمت کم سے کم کرنے سے دباو کی پہلی چوٹی جلد سے جلد حاصل ہوتی ہے اور مساوات \حوالہ{مساوات_عارضی_امالہ_برق_گیر_رو} کے تحت  رو کی چوٹی زیادہ سے زیادہ ہوتی ہے۔امالہ گیر کے استعمال سے ابتدائی رو کو قابل قبول حد تک رکھا جاتا ہے۔قوی برقیات میں ابتدائی رو قابو کرنے کی خاطر امالہ گیر کی جگہ مزاحمت اس لئے استعمال نہیں کیا جاتا کہ مزاحمت طاقت ضائع کرتی ہے جبکہ امالہ گیر طاقت ضائع نہیں کرتی۔
\انتہا{مثال}
%=========================
\ابتدا{مثال}\شناخت{مثال_عارضی_دباو_پکڑ}
\اصطلاح{قوی برقیات}\فرہنگ{قوی برقیات}\حاشیہب{power electronics}\فرہنگ{power electronics} کے میدان میں برقی طاقت کو قابو کیا جاتا ہے۔یہ طاقت چند واٹ \عددی{\si{\watt}} سے کئی سو میگا واٹ \عددی{\si{\mega\watt}} تک ہو سکتی ہے۔شکل \حوالہ{شکل_عارضی_دباو_پکڑ}-الف میں مزاحمت \عددی{R_L} کو سوئچ کے ذریعہ منبع دباو سے طاقت فراہم کی گئی ہے۔سوئچ کو چالو اور منقطع کرتے ہوئے مزاحمت کو منتقل طاقت قابو کی جاتی ہے۔منبع اور مزاحمت کے درمیان امالہ گیر بھی موجود ہے۔

\begin{figure}
\centering
\begin{subfigure}{0.5\textwidth}
\centering
\begin{tikzpicture}
\draw(0,0)node[ground]{} to [american voltage source,l={$V_i$}]++(0,\y) to [resistor,l={$R_L$}]++(\x,0) to [inductor,i={$i_L(t)$},l={$L$}]++(\x,0) to [ospst,l={${t=0}$}]++(0,-\y) node[ground]{};
\end{tikzpicture}
\caption*{(الف)}
\end{subfigure}%
\begin{subfigure}{0.5\textwidth}
\centering
\begin{tikzpicture}
\draw(0,0) to [short,i>={${i_L(t)=I_0}$}] ++(\x/2,0) to [inductor,l={$L$}]++(\x,0) to [short]++(0,-\y) to [resistor,l={$R_m$}]++(-\x-\x/2,0) to [short]++(0,\y);
\draw(\x/2+\x/2,-\dy)node[below]{$+ \, v_L(t) \, -$};
\end{tikzpicture}
\caption*{(ب)}
\end{subfigure}
\begin{subfigure}{0.5\textwidth}
\centering
\begin{tikzpicture}
\draw(0,0)node[ground]{} to [american voltage source,l={$\SI{5}{\volt}$}]++(0,\y) to [resistor,l={$\SI{2}{\ohm}$}]++(\x,0) to [inductor,i>_={$i(t)$},l={$\SI{1}{\milli\henry}$}]++(\x,0)node[above]{$v_0(t)$} to [ospst,l_={${t=0}$}]++(0,-\y) node[ground]{};
\draw(2*\x,\y) to [resistor,*-,l={$R$}]++(\x,0) to [capacitor,l_={$C$}]++(0,-\y) node[ground]{};
\draw(3*\x+2*\dx,\y/2)node[right]{$\begin{aligned} &+ \\& v_C(t) \\ &-\end{aligned}$};
\end{tikzpicture}
\caption*{(پ)}
\end{subfigure}
\begin{subfigure}{0.5\textwidth}
\centering
\pgfplotsset{scaled x ticks=false,scaled y ticks=false}
\begin{tikzpicture}
\begin{axis}[axis x line*=middle,axis y line*=left,xlabel=$t(\si{\micro\second})$,ylabel=$v_0(t)$,xtick={-0.00001,0,0.00001,0.00002,0.00003,0.00004,0.00005},xticklabels={$-10$,$0$,$10$,$20$,$30$,$40$,$50$},ytick={495},yticklabels={$\SI{495}{\volt}$},
every axis x label/.style={
    at={(ticklabel* cs:1.15)},
    anchor=east,}, 
	every axis y label/.style={
    at={(ticklabel* cs:1.05)},
    anchor=east,}
]
\addplot[domain=0:0.00005,samples=200]{5+490*e^(-100000*x)-2.4*10^(7)*x*e^(-100000*x)};
\addplot[domain=-0.00001:0.00003,samples=2]{0};
\draw(axis cs:0,0)--(axis cs:0,495);
\end{axis}
\end{tikzpicture}
\end{subfigure}
\caption{مثال \حوالہ{مثال_عارضی_دباو_پکڑ} کے اشکال۔}
\label{شکل_عارضی_دباو_پکڑ}
\end{figure}
فرض کریں کہ سوئچ اتنی دیر سے چالو ہے کہ دور برقرار صورت اختیار کئے ہوئے ہے۔یوں امالہ گیر کو قصر دور تصور کرتے ہوئے
\begin{align*}
i_L=\frac{V_I}{R_L}=I_0
\end{align*}
لکھا جا سکتا ہے۔شکل-ب میں  امالہ گیر اور \عددی{R_m} متوازی جڑے دکھائے گئے ہیں جہاں امالہ گیر کی ابتدائی رو \عددی{I_0} ہے۔ آپ جانتے ہیں کہ ایسی صورت میں امالہ گیر کی رو درج ذیل مساوات کے تحت آخرکار صفر ہو جائے گی
\begin{align*}
i_L(t)=I_0e^{-\frac{R}{L}t}
\end{align*} 
اور اس دوران اس پر برقی دباو
\begin{align*}
v_L(t)=L\frac{\dif i_L(t)}{\dif t}=-\frac{R}{L} I_0e^{-\frac{R}{L}t}
\end{align*}
پایا جائے گا۔آپ دیکھ سکتے ہیں کہ امالہ گیر پر دباو منفی ہو گا یعنی برقی دباو شکل-ب میں دکھائے گئے \عددی{v_L(t)} کے الٹ ہو گا۔  اب شکل-الف پر دوبارہ غور کریں جہاں سوئچ منقطع ہونے کے بعد امالہ گیر کے متوازی لامحدود قیمت کی مزاحمت پائی جائے گی۔یوں درج بالا مساوات میں دباو کی قیمت منفی اور لامحدود ہو گی۔
\begin{align*}
v_L(t)=-\frac{\infty}{L} I_0e^{-\frac{\infty}{L}t}
\end{align*}
امالہ گیر کی رو جلدی سے منقطع کرنے سے پیدا دباو کو  \اصطلاح{امالی لات}\فرہنگ{امالی لات}\حاشیہب{inductive kick}\فرہنگ{inductive kick}\فرہنگ{kickback} کہتے\حاشیہد{ایسا معلوم ہوتا ہے جیسے امالہ گیر غصے میں آ کر لات مارتا ہے۔} ہیں۔لامحدود دباو سوئچ پر شعلہ پیدا کرتا ہے جس سے سوئچ جھلس سکتا ہے۔ قوی برقیات کے میدان میں کام کرنے والوں کے لئے امالی لات ایک مسلسل درد سر ثابت ہوتا ہے۔

سوئچ پر دباو کی قیمت قابو کرنے سے شعلہ روکا جا سکتا ہے۔دباو کی قیمت تبدیلی رو کی شرح پر منحصر ہے لہٰذا اس شرح کو کم کرتے ہوئے دباو پر قابو پایا جا سکتا ہے۔شکل-پ میں سوئچ کے متوازی \عددی{RC} جوڑے گئے ہیں۔شکل-پ میں سوئچ منقطع کرنے سے رو یک دم صفر نہیں ہو جاتی بلکہ اس کی سمت \عددی{RC} کی طرف مُڑ جاتی ہے لہٰذا امالہ گیر میں رو برقرار رہتی ہے اور لامحدود دباو پیدا ہونے کا جواز ہی نہیں رہتا۔آئیں \عددی{L}، \عددی{R} اور \عددی{C} کی قیمتیں حاصل کرنا سیکھیں۔

تصور کریں کہ \عددی{V_I=\SI{5}{\volt}}، \عددی{L=\SI{1}{\milli\henry}} اور \عددی{R_L=\SI{2}{\ohm}} ہیں۔یوں برقرار چالو سوئچ میں امالہ گیر کی رو درج ذیل ہو گی جسے سوئچ منقطع کرتے وقت کی ابتدائی رو لیا جاتا ہے۔ 
\begin{align*}
i_L(0_-)=i_L(0_+)=\frac{\SI{5}{\volt}}{\SI{2}{\ampere}}=\SI{2.5}{\ampere}
\end{align*} 
برقرار چالو سوئچ کی صورت میں برق گیر پر دباو صفر ہو گا۔
\begin{align*}
v_C(0_-)=v_C(0_+)=\SI{0}{\volt}
\end{align*} 
سوئچ منقطع کرنے کے بعد دور سلسلہ وار \عددی{RLC} صورت اختیار کر لیتا ہے جس کی تفرقی مساوات درج ذیل ہے۔
\begin{align}\label{مساوات_عارضی_تفرقی_دو_درجی_مثال}
L\frac{\dif i(t)}{\dif t}+(R_L+R) i(t)+\frac{1}{C}\int_{0}^{t} i(t) \dif t+v_C(0_+)&=5
\end{align}
اس سے امتیازی مساوات حاصل کرتے ہیں۔
\begin{align*}
s^2+\left(\frac{2+R}{L}\right)s+\frac{1}{LC}=s^2+2\zeta\omega_0 s+\omega_0^2=0
\end{align*}
ہم \عددی{\zeta=1} اور \عددی{\omega_0=\SI{100}{\kilo\radian\per\second}} چنتے ہوئے آگے بڑھتے ہیں۔یوں
\begin{align*}
C&=\SI{0.1}{\micro\farad}\\
R&=\SI{198}{\ohm}
\end{align*}
حاصل ہوتا ہے۔

اب سوئچ منقطع کرتے وقت کے دباو پر غور کرتے ہیں۔چونکہ برق گیر کی ابتدائی دباو صفر وولٹ ہے لہٰذا سوئچ منقطع کرنے کے فوراً بعد اس پر \عددی{\SI{0}{\volt}} ہی ہو گا۔اس لمحہ سوئچ پر دباو
\begin{align*}
v_0(0_+)=i(0_+)R+v_C(0_+)=2.5\times 198+0=\SI{495}{\volt}
\end{align*}
ہو گا۔سوئچ کے متوازی \عددی{RC} نسب کرنے سے بے قابو بڑھتے ہوئے دباو پر قابو پاتے ہوئے دباو کو قابل قبول حد تک محدود کیا جاتا ہے۔قوی برقیات کے میدان میں سوئچ کے متوازی \عددی{RC} نسب کرنا لازمی ثابت ہوتا ہے۔دباو کی روک تھام کی خاطر سوئچ کے متوازی \عددی{RC} دور کو \اصطلاح{دباو پکڑ}\فرہنگ{دباو پکڑ}\حاشیہب{snubber}\فرہنگ{snubber} کہتے ہیں۔

پرزوں کی قیمتیں پُر کرتے ہوئے امتیازی مساوات درج ذیل لکھا جائے گا
\begin{align*}
s^2+2\times 10^5 s+10^{10}=0
\end{align*}
جس کے حل
\begin{align*}
s_1=s_2=\num{100000}
\end{align*}
سے فطری حل درج ذیل لکھا جاتا ہے۔
\begin{align*}
i_F(t)=c_1e^{-100000t}+t c_2e^{-100000t}=i(t)
\end{align*}
چونکہ جبری حل صفر کے برابر ہے لہٰذا فطری حل ہی عمومی حل \عددی{i(t)} ہے۔عمومی حل کے مستقل دریافت کرنے کی خاطر ابتدائی \عددی{\tfrac{\dif i(0_+)}{\dif t}} درکار  ہے جسے مساوات \حوالہ{مساوات_عارضی_تفرقی_دو_درجی_مثال}  میں لمحہ \عددی{t=0_+} کے معلومات پُر کرنے
\begin{align*}
\left. 10^{-3}\frac{\dif i(t)}{\dif t}\right|_{t=0_+}+(2+198)\times 2.5+0+0&=5
\end{align*}
سے
\begin{align*}
\left. \frac{\dif i(t)}{\dif t}\right|_{t=0_+}=\SI{-495000}{\volt\per\second}
\end{align*}
حاصل کیا جا سکتا ہے۔عمومی حل میں \عددی{i(0_+)} پُر کرنے سے  \عددی{c_1} کی قیمت حاصل ہوتی ہے۔
\begin{align*}
c_1=2.5
\end{align*}
اسی طرح عمومی حل کے تفرق میں ابتدائی \عددیء{\tfrac{\dif i(t)}{\dif t}} پُر کرنے سے
\begin{align*}
c_2=\num{-245000}
\end{align*}
ملتا ہے۔یوں عمومی حل درج ذیل لکھا جاتا ہے۔
\begin{align*}
i(t)=2.5e^{-100000t}-245000 t e^{-100000t}
\end{align*}
یوں سوئچ پر دباو درج ذیل ہو گا
\begin{align*}
v_0(t)&=Ri(t)+\frac{1}{C}\int_{0}^{t} i(t) \dif t+v_C(0_+)\\
&=5+490e^{-100000t}-2.4\times 10^7 t e^{-100000 t}
\end{align*}
جسے شکل \حوالہ{شکل_عارضی_دباو_پکڑ}-ت میں دکھایا گیا ہے۔ درج بالا مساوات سے \عددی{t=\infty} پر \عددی{v_0=\SI{5}{\volt}} ملتا ہے۔شکل-ت میں اتنی کم مقدار دکھایا ممکن نہیں ہے۔
\انتہا{مثال}
%=========================================
\ابتدا{مثال}\شناخت{مثال_عارضی_منبع_چالو_بند}
شکل \حوالہ{شکل_عارضی_منبع_چالو_بند} میں منبع دباو\حاشیہب{switching supply}\فرہنگ{switching supply} کا نہایت مقبول دور دکھایا گیا ہے۔آپ یقین کے ساتھ کہہ سکتے ہیں کہ آپ کے \اصطلاح{کمپیوٹر}\فرہنگ{کمپیوٹر}\حاشیہب{computer}\فرہنگ{computer} اور گھر میں موجود \اصطلاح{ٹیلیویژن}\فرہنگ{ٹیلیویژن}\حاشیہب{television, TV}\فرہنگ{television, TV}\فرہنگ{TV} کو یہی برقی طاقت مہیا کرتا ہے۔آئیں اس کی کارکردگی پر غور کریں۔
\begin{figure}
\centering
\begin{subfigure}{1\textwidth}
\centering
\begin{tikzpicture}
\draw(0,0)node[spdt](ksw){};
\draw(ksw.in)  to [inductor,l={$+ \, v_L(t) \, -$}]++(-\x,0) to [short,i<_={$i_L(t)$}]++(-\x/4,0) ++(0,-\y-\y/4)node[ground]{}coordinate(kBL) to [american voltage source,*-,l={$V_I$}]++(0,\y+\y/4);
%path
\path[name path=kBOT](kBL)--++(4*\x+\x/4,0);
\path[name path=kcap](ksw.out 1)++(\x,0)--++(0,-\y-\y/2);
\path[name path=kres](ksw.out 1)++(2*\x,0)--++(0,-\y-\y/2);
\path[name path=ksout](ksw.out 2)--++(0,-\y-\y/8);
%
\draw[name intersections={of={kBOT and kcap}}](ksw.out 1) to [short]++(\x,0)coordinate(kCT) to [capacitor,*-*,l_={$C$}] (intersection-1)coordinate(kBotC)-- (kBL);
\draw[name intersections={of={kBOT and kres}}](kCT) to [short]++(\x,0)coordinate(kresT) to [resistor,l_={$R$}](intersection-1)coordinate(kresB)--(kBotC);
\draw($(kresT)!0.5!(kresB)$)++(\dx,0)node[right]{$\begin{aligned}&+ \\& v_0(t) \\ &- \end{aligned}$};
\draw[name intersections={of={kBOT and ksout}}] (intersection-1) to [short,*-]++(0,\y) |-(ksw.out 2);
\end{tikzpicture}
\caption*{(الف)}
\end{subfigure}
\begin{subfigure}{0.5\textwidth}
\centering
\begin{tikzpicture}
\draw(0,0) to [capacitor,l={$C$}]++(0,\y) to [short,i>={$i(t)$}]++(\x,0) to [resistor,l_={$R$}]++(0,-\y) to [short] (0,0);
\draw(\x+\dx,\y/2)node[right]{$\begin{aligned} &+ \\& v_0(t) \\ &-\end{aligned}$};
\end{tikzpicture}
\caption*{(ب)}
\end{subfigure}%
\begin{subfigure}{0.5\textwidth}
\centering
\begin{tikzpicture}
\draw(0,0) to [american voltage source,l={$V_I$}]++(0,\y) to [short,i>={$i_L(t)$}]++(\x,0) to [inductor,l_={$L$}]++(0,-\y) to [short] (0,0);
\draw(\x+\dx,\y/2)node[right]{$\begin{aligned} &+ \\& v_L(t) \\ &-\end{aligned}$};
\end{tikzpicture}
\caption*{(پ)}
\end{subfigure}
\caption{مثال \حوالہ{مثال_عارضی_منبع_چالو_بند} کے اشکال۔}
\label{شکل_عارضی_منبع_چالو_بند}
\end{figure}

منبع میں ایک قطب اور دو چال والا سوئچ استعمال کیا گیا ہے۔یہ سوئچ امالہ گیر کو زمین کے ساتھ \عددی{t_1} دورانیے کے لئے اور برق گیر کے ساتھ \عددی{t_2} دورانیے کے لئے جوڑتا ہے۔یوں سوئچ کا دوری عرصہ \عددی{T=t_1+t_2} ہے۔فرض کریں کہ سوئچ صفر دورانیے\حاشیہب{دکھایا گیا دور حقیقی دور کی سادہ شکل ہے۔} میں جوڑ تبدیل کرتا ہے لہٰذا ایسا کبھی بھی نہیں ہو گا کہ امالہ گیر کی رو یک دم روکی جائے۔دوران \عددی{t_1} منبع کو دو علیحدہ علیحدہ ادوار تصور کیا جا سکتا ہے جنہیں شکل-ب اور شکل-پ میں دکھایا گیا ہے۔

دوران \عددی{t_1} امالہ گیر کی رو مسلسل بڑھتی ہے جس سے امالہ گیر میں ذخیرہ توانائی \عددی{W=\tfrac{Li_L^2}{2}} بڑھتی ہے۔اس دوران مزاحمت کو برق گیر طاقت فراہم کرتا ہے لہٰذا برق گیر کا دباو مسلسل گھٹتا ہے۔دوران \عددی{t_2} امالہ گیر کی رو کا کچھ حصہ برق گیر میں بار بھرتا ہے جبکہ بقایا حصہ مزاحمت سے گزرتا ہے۔امالہ گیر کی رو یک دم تبدیل نہیں ہو سکتی لہٰذا اس دوران امالہ گیر کی رو بتدریج گھٹتی ہے اور امالہ گیر میں ذخیرہ توانائی برق گیر اور مزاحمت کو منتقل ہوتا ہے۔ دور یہ سلسلہ لگاتار دہراتا ہے۔یوں \عددی{t_1} کے دوران امالہ گیر توانائی حاصل کرتے ہوئے \عددی{t_2} کے دوران اسے برق گیر اور مزاحمت کو منتقل کرتا ہے۔یوں آپ دیکھ سکتے ہیں کہ \عددی{t_1} کے ابتدا اور \عددی{t_2} کے اختتام پر امالہ گیر میں رو کی قیمت یکساں طور پر \عددی{I_0} ہو گی۔شکل \حوالہ{شکل_عارضی_منبع_چالو_بند_رو}-الف میں \عددی{i_L} کو دکھایا گیا ہے۔
\begin{figure}
\begin{subfigure}{1\textwidth}
\centering
\begin{tikzpicture}
\draw[gray](0,0)--(9.5,0)node[right]{$t$};
\draw[gray](0,0)--(0,2.5)node[left]{$i_L(t)$};
\draw(0,1.5)node[left]{$I_0$}--++(2,0.5)--++(1,-0.5)--++(2,0.5)--++(1,-0.5)--++(2,0.5)--++(1,-0.5);
\draw[gray,dashed](0,1.5)--++(7,0);
\draw[gray,dashed](2,2)--(2,0)node[below]{$t_1$};
\draw[gray,dashed](3,1.5)--(3,0)node[below]{$T$};
\end{tikzpicture}
\caption*{(الف)}
\end{subfigure}
%
\begin{subfigure}{1\textwidth}
\centering
\begin{tikzpicture}
\draw[gray](0,0)--(9.5,0)node[right]{$t$};
\draw[gray](0,-1.5)--(0,1)node[left]{$v_L(t)$};
\draw(0,0.5)node[left]{$V_I$}--++(2,0)--++(0,-1.5)--++(1,0)--++(0,1.5)coordinate(ks)--++(2,0)coordinate(ke)--++(0,-1.5)--++(1,0)--++(0,1.5)--++(2,0)--++(0,-1.5)--++(1,0)--++(0,1.5);
\draw[gray,dashed](2,-1)--(0,-1)node[left,black]{$V_I-V_0$};
\draw(2,0)node[below left]{$t_1$};
\draw(3,0)node[below right]{$T$};
%
\draw[gray](ks)++(0,0.1)--++(0,0.4);
\draw[gray](ke)++(0,0.1)--++(0,0.4);
\draw[gray](ke)++(1,0)++(0,0.1)--++(0,0.4);
\draw[stealth-stealth,gray](ks)++(0,0.3)--++(2,0)node[pos=0.5,fill=white]{$t_1$};
\draw[stealth-stealth,gray](ke)++(0,0.3)--++(1,0)node[pos=0.5,fill=white]{$t_2$};
\end{tikzpicture}
\caption*{(ب)}
\end{subfigure}
\caption{مثال \حوالہ{مثال_عارضی_منبع_چالو_بند} کے اشکال۔}
\label{شکل_عارضی_منبع_چالو_بند_رو}
\end{figure}
آئیں دوران \عددی{t_1} شکل \حوالہ{شکل_عارضی_منبع_چالو_بند}-ب اور شکل \حوالہ{شکل_عارضی_منبع_چالو_بند}-پ پر تفصیلاً غور کریں۔

دوران \عددی{t_1} امالہ گیر کے لئے  شکل-پ کو دیکھتے ہوئے 
\begin{align*}
\frac{\dif i_L(t)}{\dif t}=\frac{V_I}{L}
\end{align*}
یا
\begin{gather}
\begin{aligned}\label{مساوات_عارضی_بڑھاتا_منبع_دباو_الف}
i_L(t)&=\frac{1}{L} \int_{0}^{t_1} v_L(t) \dif t+i_L(0_+)\\
&=\frac{1}{L}\int_{0}^{t_1} V_I \dif t+I_0\\
&=\frac{V_I}{L} t_1+I_0\quad \quad \quad \quad 0<t<t_1
\end{aligned}
\end{gather}
لکھا جا سکتا ہے لہٰذا اس دورانیے میں امالہ گیر کی رو بڑھتی ہے۔چونکہ  رو اور امالہ گیر میں مقناطیسی توانائی کا تعلق \عددی{W=\tfrac{L i_L(t)^2}{2}} ہے لہٰذا امالہ گیر کی رو بڑھنے سے اس میں ذخیرہ توانائی بڑھتی ہے۔اسی دوران مزاحمت \عددی{R} کو برق گیر توانائی فراہم کرتا ہے لہٰذا برق گیر کی توانائی بتدریج گھٹتی ہے۔برق گیر کی ابتدائی دباو \عددی{V_0} لیتے ہوئے 
\begin{align*}
v_0(t)=V_0 e^{-\frac{t}{RC}}
\end{align*}
لکھا جا سکتا ہے۔حقیقت میں \عددی{RC} وقتی مستقل کی قیمت سوئچ کے دوری عرصہ \عددی{T} سے  بہت کم \عددی{(RC \ll T)} ہوتی ہے  لہٰذا \عددی{t_1} کے دوران برق گیر کے دباو میں تبدیلی قابل نظر انداز ہوتی ہے۔یوں برق گیر کے دباو کو مستقل تصور کیا جا سکتا ہے۔

آئیں اب \عددی{t_2} کے دوران صورت حال پر غور کریں۔ سادہ مساوات کے حصول کی خاطر برق گیر کے دباو کو مستقل مقدار \عددی{V_0} تصور کرتے  ہوئے شکل-ب  سے
\begin{align*}
\frac{\dif i_L(t)}{\dif t}=\frac{V_I-V_0}{L}
\end{align*}
لکھا جا سکتا ہے۔دورانیہ \عددی{t_2} کی ابتدائی رو مساوات \حوالہ{مساوات_عارضی_بڑھاتا_منبع_دباو_الف} کی اختتامی رو ہو گی۔یوں درج ذیل لکھا جا سکتا ہے 
\begin{gather}
\begin{aligned}\label{مساوات_عارضی_بڑھاتا_منبع_دباو_ب}
i_L(t)&=\frac{1}{L}\int_{t_1}^{t_1+t_2} v_L(t) \dif t+\left[\frac{V_I}{L} t_1+I_0\right]\\
&=\frac{1}{L}\int_{t_1}^{t_1+t_2}(V_I-V_0)\dif t+\left[\frac{V_I}{L} t_1+I_0\right]\\
&=\frac{V_I}{L}(t_1+t_2)-\frac{V_0}{L}t_2+I_0 \quad \quad \quad \quad t_1<t<(t_1+t_2)
\end{aligned}
\end{gather}
جہاں ابتدائی رو کو چکور قوسین میں بند لکھا گیا ہے اور آخری قدم پر نتائج کو ترتیب دیتے ہوئے پیش کیا گیا ہے۔

جیسے شکل \حوالہ{شکل_عارضی_منبع_چالو_بند_رو}-الف میں دکھایا گیا ہے، لمحہ \عددی{t_2} کے اختتام پر امالہ گیر کی رو وہی ہو گی جو \عددی{t_1} کی ابتدا پر ہے۔اگر \عددی{t_2} کے اختتام پر رو کی قیمت \عددی{I_0} سے زیادہ ہو تب امالہ گیر کی رو ہر چکر میں بتدریج بڑھتی رہے گی حتٰی کہ آخرکار یہ امالہ گیر کو تباہ کر دے گی۔اسی طرح اگر \عددی{t_2} کے اختتام پر رو کی قیمت \عددی{I_0} سے کم ہو تب ہر چکر میں رو کی قیمت بتدریج کم ہوتے ہوئے صفر ہو جائے گی۔منبع دباو کی صحیح کارکردگی کے لئے ضروری ہے کہ \عددی{t_1} کی ابتدا پر اور \عددی{t_2} کی اختتام پر رو کی قیمت یک برابر رہے۔ان حقائق کو مد نظر رکھتے ہوئے مساوات \حوالہ{مساوات_عارضی_بڑھاتا_منبع_دباو_ب} کی اختتامی رو کو \عددی{I_0} کے برابر پُر کرتے ہوئے حل کرتے ہیں
\begin{align*}
\frac{V_I}{L}(t_1+t_2)-\frac{V_0}{L}t_2+I_0 &=I_0\\
\frac{V_I}{L}T=\frac{V_0}{L}t_2
\end{align*}
جہاں دوسری قدم پر \عددی{t_1+t_2=T} لکھا گیا ہے۔یوں درج ذیل حاصل ہوتا ہے
\begin{align*}
V_0=V_I \frac{T}{t_2}=V_0 \frac{T}{T-t_1}=\frac{V_0}{1-\frac{t_1}{T}}
\end{align*}
جس میں
\begin{align}
D=\frac{t_1}{T}\quad \quad \quad (0< D < 1)
\end{align}
لکھتے ہوئے
\begin{align}
V_0=\frac{V_I}{1-D}
\end{align}
ملتا ہے۔وقت \عددی{t_1} اور دوری عرصہ \عددی{T} کی شرح \عددی{D} کو \اصطلاح{فعال عرصہ}\فرہنگ{فعال عرصہ}\حاشیہب{duty cycle}\فرہنگ{duty cycle} کہتے ہیں جسے عموماً فی صد کی صورت میں بیان کیا جاتا ہے لہٰذا \عددی{\SI{40}{\percent}} فعال عرصے کی مراد \عددی{t_1=0.4T} ہے۔

یہاں غور کریں کہ \عددی{t_1< T} ہے لہٰذا \عددی{D}  مثبت ہو گا جبکہ اس کی قیمت صفر تا اکائی \عددی{(0<D<1)} ممکن ہے۔یوں درج بالا مساوات کے تحت \عددی{V_0\ge V_I} ہو گا یعنی خارجی دباو کی قیمت داخلی دباو سے زیادہ ہو گی۔اسی لئے اس منبع کو \اصطلاح{اٹھان منبع}\فرہنگ{اٹھان منبع}\حاشیہب{boost converter}\فرہنگ{boost converter} کہتے ہیں۔دوری عرصہ \عددی{T} کو مستقل رکھتے ہوئے \عددی{V_0} کی قیمت کو \عددی{D} کی مدد سے تبدیل کیا جاتا ہے۔
\انتہا{مثال}
%============================

\حصہء{سوالات}

%==================
\ابتدا{سوال}\شناخت{سوال_عارضی_الف}
شکل \حوالہ{شکل_سوال_عارضی_الف}-الف میں  سوئچ منقطع کرنے کے بعد \عددی{i(t)} دریافت کریں۔
\begin{figure}
\centering
\begin{subfigure}{0.3\textwidth}
\centering
\begin{tikzpicture}
\draw(0,0) to [american voltage source,l_={$\SI{20}{\volt}$}]++(0,2*\y) to [ospst,l={${t=0}$}]++(\x,0) to [resistor,l={$\SI{10}{\ohm}$}]++(0,-\y) to [inductor,l_={$\SI{4}{\henry}$}]++(0,-\y) to [short]++(-\x,0);
\draw(\x,\y) to [short,*-]++(\x/2,0) to [resistor,l={$\SI{12}{\ohm}$},i<={$i(t)$}]++(0,-\y) to [short,-*]++(-\x/2,0); 
\end{tikzpicture}
\caption*{(الف)}
\end{subfigure}%
\begin{subfigure}{0.7\textwidth}
\centering
\begin{tikzpicture}
\draw(0,0) to [american current source,l={$\SI{8}{\milli\ampere}$}]++(0,\y) to [ospst,l={${t=0}$}]++(\x,0) to [resistor,l_={$\SI{4}{\kilo\ohm}$},i={$i(t)$}]++(0,-\y)  to [short]++(-\x,0);
\draw(\x,\y) to [short,*-]++(\x,0) to [capacitor,l_={$\SI{100}{\micro\farad}$}]++(0,-\y) to [short,-*]++(-\x,0); 
\draw(0,0) to [short,*-]++(-\x,0) to [resistor,l={$\SI{4}{\kilo\ohm}$}]++(0,\y) to [short,-*]++(\x,0);
\end{tikzpicture}
\caption*{(ب)}
\end{subfigure}%
\caption{سوال \حوالہ{سوال_عارضی_الف} اور سوال \حوالہ{سوال_عارضی_ب} کے ادوار۔}
\label{شکل_سوال_عارضی_الف}
\end{figure}

جواب:\عددی{i(t)=2e^{-3t}\,\si{\ampere}}
\انتہا{سوال}
%====================
\ابتدا{سوال}\شناخت{سوال_عارضی_ب}
شکل \حوالہ{شکل_سوال_عارضی_الف}-ب میں  سوئچ منقطع کرنے کے بعد \عددی{i(t)} دریافت کریں۔

جواب:\عددی{i(t)=4e^{-\tfrac{5t}{2}}\,\si{\milli\ampere}}
\انتہا{سوال}
%====================
\ابتدا{سوال}\شناخت{سوال_عارضی_پ}
شکل \حوالہ{شکل_سوال_عارضی_پ} میں  \عددی{t=0} پر سوئچ کو منبع کی جانب کر دیا جاتا ہے۔اس لمحے  کے بعد \عددی{v_0(t)} دریافت کریں۔
\begin{figure}
\centering
\begin{tikzpicture}[american voltages]
\draw(0,0) node[spdt,xscale=-1,yscale=-1](sw1){};
\draw[-stealth,thick]([shift={(200:0.7)}]sw1.in) arc (200:140:0.7)node[above]{$t=0$};
\draw(sw1.out 2)--++(-\x/2,0) ++(0,-\y)coordinate(kBL) to [american voltage source,l={$\SI{8}{\volt}$}]++(0,\y);
\draw[name path=kbot](sw1.in) to [resistor,l={$\SI{10}{\kilo\ohm}$}]++(\x,0)coordinate(kT) to [resistor,l={$\SI{4}{\kilo\ohm}$}]++(\x,0) to [resistor,l={$\SI{6}{\kilo\ohm}$},v={$v_0(t)$}]++(0,-\y)coordinate(kBR) -|(kBL);
\path[name path=kvrt](sw1.out 1)--++(0,-\y);
\draw[name intersections={of=kvrt and kbot}] (sw1.out 1)--(intersection-1)node[circ]{};
\draw(kT) to [capacitor,*-*,l_={$\SI{200}{\micro\farad}$}]++(0,-\y);
\end{tikzpicture}
\caption{سوال \حوالہ{سوال_عارضی_پ} کا دور۔}
\label{شکل_سوال_عارضی_پ}
\end{figure}

جواب:\عددی{v_0(t)=\tfrac{12}{5}(1-e^{-t})\,\si{\volt}}
\انتہا{سوال}
%====================
\ابتدا{سوال}\شناخت{سوال_عارضی_ت}
شکل \حوالہ{شکل_سوال_عارضی_ت}-الف میں \عددی{v_0(t)} کو \عددی{t>0} کے لئے حاصل کریں۔ 
\begin{figure}
\centering
\begin{subfigure}{0.5\textwidth}
\centering
\begin{tikzpicture}[american voltages]
\draw(0,0) to [american voltage source,l={$\SI{20}{\volt}$}]++(0,\y) to [ospst,l={${t=0}$}]++(\x,0) to [resistor,l={$\SI{8}{\kilo\ohm}$}]++(\x,0) to [resistor,l={$\SI{8}{\kilo\ohm}$},v={$v_0(t)$}]++(0,-\y) to [short]++(-2*\x,0);
\draw(\x,0) to [resistor,*-*,l={$\SI{8}{\kilo\ohm}$}]++(0,\y) to [short]++(0,3/4*\y) to [capacitor,l={$\SI{200}{\micro\farad}$}]++(\x,0) to [short,-*]++(0,-3/4*\y);
\end{tikzpicture}
\caption*{(الف)}
\end{subfigure}%
\begin{subfigure}{0.5\textwidth}
\centering
\begin{tikzpicture}[american voltages]
\draw(0,0) to [resistor,*-,l={$\SI{4}{\kilo\ohm}$}]++(0,\y) to [american voltage source,l={$\SI{10}{\volt}$}]++(0,\y);
\draw(\x,0) to [resistor,l={$\SI{4}{\kilo\ohm}$},v_>={$v_0(t)$}]++(0,2*\y) to [short,*-]++(0,3/4*\y) to [capacitor,l={$\SI{100}{\micro\farad}$}]++(-\x,0) to [short,-*]++(0,-3/4*\y);
\draw(\x,0) to [short]++(-2*\x,0) to [ospst,l={${t=0}$}]++(0,\y) to [american voltage source,l={$\SI{6}{\volt}$}]++(0,\y) to [short,-*]++(\x,0) to [resistor,l_={$\SI{8}{\kilo\ohm}$}]++(\x,0);
\end{tikzpicture}
\caption*{(ب)}
\end{subfigure}%
\caption{سوال \حوالہ{سوال_عارضی_ت} اور سوال \حوالہ{سوال_عارضی_ٹ} کے ادوار۔}
\label{شکل_سوال_عارضی_ت}
\end{figure}

جواب:\عددی{v_0(t)=-5e^{-\tfrac{15t}{16}}\,\si{\volt}}
\انتہا{سوال}
%=====================
\ابتدا{سوال}\شناخت{سوال_عارضی_ٹ}
شکل \حوالہ{شکل_سوال_عارضی_ت}-ب میں \عددی{v_0(t)} کو \عددی{t>0} کے لئے حاصل کریں۔ 

جواب:\عددی{v_0(t)=\tfrac{5}{2}+\tfrac{1}{2}e^{-\tfrac{5t}{2}}\,\si{\volt}}
\انتہا{سوال}
%=====================
\ابتدا{سوال}\شناخت{سوال_عارضی_ث}
شکل \حوالہ{شکل_سوال_عارضی_ث} میں \عددی{i_0(t)} کو \عددی{t>0} کے لئے حاصل کریں۔ 
\begin{figure}
\centering
\begin{tikzpicture}
\draw(0,0) to [american voltage source,l={$\SI{10}{\volt}$}]++(0,\y);
\draw(\x,0) to [capacitor,*-*,l={$\SI{100}{\micro\farad}$}]++(0,\y);
\draw(2*\x,0) to [cspst,*-*,l={${t=0}$}]++(0,\y);
\draw(3*\x,0) to [resistor,l={$\SI{2}{\kilo\ohm}$}]++(0,\y);
\draw(0,0) to [short]++(3*\x,0);
\draw(0,\y) to [resistor,l={$\SI{2}{\kilo\ohm}$}]++(\x,0) to [resistor,l={$\SI{6}{\kilo\ohm}$},i={$i(t)$}]++(\x,0) to [short]++(\x,0);
\end{tikzpicture}
\caption{سوال \حوالہ{سوال_عارضی_ث} کا دور۔}
\label{شکل_سوال_عارضی_ث}
\end{figure}

جواب:\عددی{i(t)=\tfrac{5}{4}+\tfrac{1}{12}e^{-\tfrac{20t}{3}}\,\si{\milli\ampere}}
\انتہا{سوال}
%=====================
\ابتدا{سوال}\شناخت{سوال_عارضی_ج}
شکل \حوالہ{شکل_سوال_عارضی_ج} میں \عددی{i_0(t)} کو \عددی{t>0} کے لئے حاصل کریں۔ 
\begin{figure}
\centering
\begin{tikzpicture}
\draw(0,0) to [resistor,l={$\SI{4}{\kilo\ohm}$},i<={$i(t)$}]++(0,\y);
\draw(\x,0) to [inductor,*-*,l={$\SI{4}{\henry}$}]++(0,\y);
\draw(2*\x,0) to [american voltage source,l={$\SI{8}{\volt}$}]++(0,\y);
\draw(0,0) to [short] ++(\x,0) to [ospst,l={${t=0}$}]++(\x,0);
\draw(0,\y) to [short]++(\x,0) to [resistor,l={$\SI{1}{\kilo\ohm}$},]++(\x,0);
\end{tikzpicture}
\caption{سوال \حوالہ{سوال_عارضی_ج} کا دور۔}
\label{شکل_سوال_عارضی_ج}
\end{figure}

جواب:\عددی{i(t)=-8e^{-1000t}\,\si{\milli\ampere}}
\انتہا{سوال}
%=====================
\ابتدا{سوال}\شناخت{سوال_عارضی_چ}
شکل \حوالہ{شکل_سوال_عارضی_چ} میں \عددی{i_0(t)} کو \عددی{t>0} کے لئے حاصل کریں۔ 
\begin{figure}
\centering
\begin{tikzpicture}
\draw(0,0) to [resistor,l={$\SI{4}{\kilo\ohm}$}]++(0,\y);
\draw(\x,0) to [american current source,*-*,l={$\SI{6}{\milli\henry}$}]++(0,\y);
\draw(2*\x,0) to [cspst,*-*,l={${t=0}$}]++(0,\y);
\draw(3*\x,0) to [capacitor,*-*,l={$\SI{200}{\micro\farad}$}] ++(0,\y);
\draw(4*\x,0) to [resistor,l={$\SI{2}{\kilo\ohm}$},i_<={$i(t)$}]++(0,\y);
\draw(0,0) to [short]++(4*\x,0);
\draw(0,\y) to [resistor,l={$\SI{2}{\kilo\ohm}$}]++(\x,0) to [resistor,l={$\SI{4}{\kilo\ohm}$}]++(\x,0) to [resistor,l={$\SI{6}{\kilo\ohm}$}]++(\x,0) to [short]++(\x,0);
\end{tikzpicture}
\caption{سوال \حوالہ{سوال_عارضی_چ} کا دور۔}
\label{شکل_سوال_عارضی_چ}
\end{figure}

جواب:\عددی{i(t)=2e^{-\tfrac{10t}{3}}\,\si{\milli\ampere}}
\انتہا{سوال}
%=====================
\ابتدا{سوال}\شناخت{سوال_عارضی_ح}
شکل \حوالہ{شکل_سوال_عارضی_ح} میں \عددی{v_0(t)} کو \عددی{t>0} کے لئے حاصل کریں۔ 
\begin{figure}
\centering
\begin{tikzpicture}[american voltages]
\draw(0,0) to [resistor,l={$\SI{4}{\kilo\ohm}$}]++(0,\y);
\draw(2*\x,0) to [american current source,*-*,l={$\SI{10}{\milli\ampere}$}]++(0,\y);
\draw(3*\x,0) to [capacitor,*-*,l={$\SI{40}{\micro\farad}$}]++(0,\y);
\draw(4*\x,0) to [resistor,l={$\SI{4}{\kilo\ohm}$},v_>={$v(t)$}]++(0,\y);
\draw(0,0) to [short]++(4*\x,0);
\draw(0,\y) to [resistor,l={$\SI{4}{\kilo\ohm}$}]++(\x,0) to [ospst,l={${t=0}$}]++(\x,0) to [resistor,l={$\SI{2}{\kilo\ohm}$}]++(\x,0) to [resistor,l={$\SI{2}{\kilo\ohm}$}]++(\x,0);
\end{tikzpicture}
\caption{سوال \حوالہ{سوال_عارضی_ح} کا دور۔}
\label{شکل_سوال_عارضی_ح}
\end{figure}

جواب:\عددی{v(t)=40-20e^{-\tfrac{25t}{6}}\,\si{\volt}}
\انتہا{سوال}
%=====================
\ابتدا{سوال}\شناخت{سوال_عارضی_خ}
شکل \حوالہ{شکل_سوال_عارضی_خ} میں \عددی{v_0(t)} کو \عددی{t>0} کے لئے حاصل کریں۔ 
\begin{figure}
\centering
\begin{tikzpicture}[american voltages]
\draw(0,0) to [cspst,l={${t=0}$}]++(0,\y);
\draw(\x,0) to [american current source,*-*,l={$\SI{6}{\milli\ampere}$}]++(0,\y);
\draw(2*\x,0) to [resistor,*-*,l={$\SI{4}{\kilo\ohm}$}]++(0,\y);
\draw(3*\x,0) to [resistor,l={$\SI{4}{\kilo\ohm}$},v_>={$v(t)$}]++(0,\y);
\draw(0,0) to [short]++(3*\x,0);
\draw(0,\y) to [short]++(\x,0) to [resistor,l={$\SI{2}{\kilo\ohm}$}]++(\x,0)to [capacitor,l={$\SI{60}{\micro\farad}$}]++(\x,0);
\end{tikzpicture}
\caption{سوال \حوالہ{سوال_عارضی_خ} کا دور۔}
\label{شکل_سوال_عارضی_خ}
\end{figure}

جواب:\عددی{v_0(t)=-18e^{-\tfrac{25t}{8}}\,\si{\volt}}
\انتہا{سوال}
%=====================
\ابتدا{سوال}\شناخت{سوال_عارضی_د}
شکل \حوالہ{شکل_سوال_عارضی_د}-الف میں \عددی{i_0(t)} کو \عددی{t>0} کے لئے حاصل کریں۔ 
\begin{figure}
\centering
\begin{subfigure}{0.5\textwidth}
\centering
\begin{tikzpicture}[american voltages]
\draw(0,0) to [resistor,l={$\SI{2}{\kilo\ohm}$}]++(0,\y);
\draw(\x,0) to [resistor,*-*,l={$\SI{2}{\kilo\ohm}$},i<={$i_0(t)$}]++(0,\y);
\draw(2*\x,0) to [capacitor,l={$\SI{80}{\micro\farad}$}]++(0,\y);
\draw(0,0) to [short]++(2*\x,0);
\draw(0,\y) to [cspst,l={${t=0}$}]++(\x,0) to [resistor,l={$\SI{4}{\kilo\ohm}$}]++(\x,0);
\draw(0,\y) to [short,*-]++(0,3/4*\y) to [american voltage source,l={$\SI{8}{\volt}$}]++(2*\x,0) to [short,-*]++(0,-3/4*\y);
\end{tikzpicture}
\caption*{(الف)}
\end{subfigure}%
\begin{subfigure}{0.5\textwidth}
\centering
\begin{tikzpicture}[american voltages]
\draw(0,0) to [american voltage source,l={$\SI{12}{\volt}$}]++(0,2*\y);
\draw(\x,0) to [american voltage source,*-,l={$\SI{10}{\volt}$}]++(0,\y) to [resistor,-*,l={$\SI{2}{\ohm}$}]++(0,\y);
\draw(2*\x,0) to [resistor,l={$\SI{4}{\ohm}$},v_>={$v_0(t)$}]++(0,2*\y);
\draw(0,0) to [short]++(2*\x,0);
\draw(0,2*\y) to [ospst,l={${t=0}$}]++(\x,0) to [resistor,l={$\SI{4}{\ohm}$}]++(\x,0);
\draw(\x,2*\y) to [short]++(0,3/4*\y) to [inductor,l={$\SI{2}{\henry}$}]++(\x,0) to [short,-*]++(0,-3/4*\y);
\end{tikzpicture}
\caption*{(ب)}
\end{subfigure}%
\caption{سوال \حوالہ{سوال_عارضی_د} اور سوال \حوالہ{سوال_عارضی_امالہ_الف} کے ادوار۔ }
\label{شکل_سوال_عارضی_د}
\end{figure}

جواب:\عددی{i_0(t)=5e^{-\tfrac{25t}{2}}\,\si{\milli\ampere}}
\انتہا{سوال}
%=====================
\ابتدا{سوال}\شناخت{سوال_عارضی_امالہ_الف}
شکل \حوالہ{شکل_سوال_عارضی_د}-ب میں \عددی{v_0(t)} کو \عددی{t>0} کے لئے حاصل کریں۔ 

جواب:\عددی{v_0(t)=\tfrac{112}{15}e^{-\tfrac{6t}{5}}-\tfrac{20}{3}\,\si{\volt}}
\انتہا{سوال}
%=====================
\ابتدا{سوال}\شناخت{سوال_عارضی_امالہ_ب}
شکل \حوالہ{شکل_سوال_عارضی_امالہ_ب} میں \عددی{v_0(t)} کو \عددی{t>0} کے لئے حاصل کریں۔ 
\begin{figure}
\centering
\begin{tikzpicture}[american voltages]
\draw(0,0) to [american current source,l={$\SI{12}{\ampere}$}]++(0,2*\y);
\draw(\x,0) to [inductor,*-,l={$\SI{6}{\henry}$}]++(0,\y) to [resistor,-*,l={$\SI{2}{\ohm}$}]++(0,\y);
\draw(2*\x,0) to [resistor,l={$\SI{4}{\ohm}$},v_>={$v_0(t)$}]++(0,2*\y);
\draw(0,0) to [short]++(2*\x,0);
\draw(0,2*\y) to [resistor,l={$\SI{4}{\ohm}$}]++(\x,0) to [resistor,l={$\SI{6}{\ohm}$}]++(\x,0);
\draw(0,2*\y) to [short,*-]++(0,3/4*\y) to [cspst,l={${t=0}$}]++(2*\x,0) to [short,-*]++(0,-3/4*\y);
\end{tikzpicture}
\caption{سوال \حوالہ{سوال_عارضی_امالہ_ب} کا دور۔}
\label{شکل_سوال_عارضی_امالہ_ب}
\end{figure}

جواب:\عددی{v_0(t)=\tfrac{176}{7}-\tfrac{120}{7}e^{-\tfrac{7t}{5}}\,\si{\volt}}
\انتہا{سوال}
%=====================
\ابتدا{سوال}\شناخت{سوال_عارضی_امالہ_پ}
شکل \حوالہ{شکل_سوال_عارضی_امالہ_پ} میں \عددی{i_0(t)} کو \عددی{t>0} کے لئے حاصل کریں۔ 
\begin{figure}
\centering
\begin{tikzpicture}[american voltages]
\draw(0,0) to [american voltage source,l={$\SI{22}{\volt}$}]++(0,2*\y);
\draw(\x,0) to [american voltage source,*-,l={$\SI{12}{\volt}$}]++(0,\y) to [ospst,-*,l={${t=0}$}]++(0,\y);
\draw(2*\x,0) to [resistor,*-,l={$\SI{4}{\kilo\ohm}$},i<={$i_0(t)$}]++(0,\y) to [inductor,-*,l={$\SI{4}{\henry}$}]++(0,\y);
\draw(3*\x,0) to [resistor,l={$\SI{4}{\kilo\ohm}$}]++(0,2*\y);
\draw(0,0) to [short]++(3*\x,0);
\draw(0,2*\y) to [resistor,l={$\SI{1}{\kilo\ohm}$}]++(\x,0) to [resistor,l={$\SI{2}{\kilo\ohm}$}]++(\x,0) to [short]++(\x,0);
\end{tikzpicture}
\caption{سوال \حوالہ{سوال_عارضی_امالہ_پ} کا دور۔}
\label{شکل_سوال_عارضی_امالہ_پ}
\end{figure}

جواب:\عددی{i_0(t)=\tfrac{11}{5}-\tfrac{7}{10}e^{-\tfrac{10000t}{7}}\,\si{\milli\ampere}}
\انتہا{سوال}
%=====================
\ابتدا{سوال}\شناخت{سوال_عارضی_امالہ_ت}
شکل \حوالہ{شکل_سوال_عارضی_امالہ_ت} میں \عددی{i_0(t)} کو \عددی{t>0} کے لئے حاصل کریں۔ 
\begin{figure}
\centering
\begin{tikzpicture}[american voltages]
\draw(0,0) to [american voltage source,l={$\SI{20}{\volt}$}]++(0,\y);
\draw(\x,0) to [resistor,*-*,l={$\SI{4}{\ohm}$}]++(0,\y);
\draw(2*\x,0) to [inductor,*-*,l={$\SI{0.4}{\henry}$}]++(0,\y);
\draw(4*\x,\y) to [american voltage source,l={$\SI{8}{\volt}$}]++(0,-\y);
\draw(0,0) to [short]++(4*\x,0);
\draw(0,\y) to [resistor,l={$\SI{2}{\ohm}$}]++(\x,0) to [resistor,l={$\SI{4}{\ohm}$},i={$i_0(t)$}]++(\x,0) to [ospst,l={${t=0}$}]++(\x,0) to [resistor,l={$\SI{4}{\ohm}$}]++(\x,0);
\end{tikzpicture}
\caption{سوال \حوالہ{سوال_عارضی_امالہ_ت} کا دور۔}
\label{شکل_سوال_عارضی_امالہ_ت}
\end{figure}

جواب:\عددی{i_0(t)=3e^{-\tfrac{40t}{3}}-2.5\,\si{\ampere}}
\انتہا{سوال}
%=====================
\ابتدا{سوال}\شناخت{سوال_عارضی_امالہ_ٹ}
شکل \حوالہ{شکل_سوال_عارضی_امالہ_ٹ} میں \عددی{i_0(t)} کو \عددی{t>0} کے لئے حاصل کریں۔ 
\begin{figure}
\centering
\begin{tikzpicture}[american voltages]
\draw(0,0) to [american voltage source,l={$\SI{8}{\volt}$}]++(0,\y);
\draw(\x,0) to [resistor,*-*,l={$\SI{4}{\ohm}$}]++(0,\y);
\draw(2*\x,0) to [cspst,*-*,l={${t=0}$}]++(0,\y);
\draw(3*\x,0) to [american current source,*-*,l={$\SI{2}{\ampere}$}]++(0,\y);
\draw(4*\x,0) to [resistor,l={$\SI{2}{\ohm}$},i<={$i_0(t)$}]++(0,\y);
\draw(0,0) to [short]++(4*\x,0);
\draw(0,\y) to [resistor,l={$\SI{4}{\ohm}$}]++(\x,0) to [resistor,l={$\SI{2}{\ohm}$}]++(\x,0) to [inductor,l={$\SI{2}{\henry}$}]++(\x,0) to [short]++(\x,0);
\end{tikzpicture}
\caption{سوال \حوالہ{سوال_عارضی_امالہ_ٹ} کا دور۔}
\label{شکل_سوال_عارضی_امالہ_ٹ}
\end{figure}

جواب:\عددی{i_0(t)=2e^{-t}\,\si{\ampere}}
\انتہا{سوال}
%=====================
\ابتدا{سوال}\شناخت{سوال_عارضی_امالہ_ث}
شکل \حوالہ{شکل_سوال_عارضی_امالہ_ث}-الف میں \عددی{v_0(t)} کو \عددی{t>0} کے لئے حاصل کریں۔ داخلی اشارہ شکل-ب میں دیا گیا ہے۔
\begin{figure}
\centering
\begin{subfigure}{0.4\textwidth}
\centering
\begin{tikzpicture}[american voltages]
\draw(0,0) to [american voltage source,l={$v_s(t)$}]++(0,\y);
\draw(\x,0) to [resistor,*-*,l={$\SI{4}{\kilo\ohm}$}]++(0,\y);
\draw(2*\x,0) to [capacitor,l={$\SI{100}{\micro\farad}$},v_>={$v_0(t)$}]++(0,\y);
\draw(0,0) to [short]++(2*\x,0);
\draw(0,\y) to [resistor,l={$\SI{2}{\ohm}$}]++(\x,0) to [resistor,l={$\SI{1}{\ohm}$}]++(\x,0);
\end{tikzpicture}
\caption*{(الف)}
\end{subfigure}%
\begin{subfigure}{0.6\textwidth}
\centering
\begin{tikzpicture}[american voltages]
\begin{axis}[small,xlabel={$t \, (\si{\second})$},ylabel={$v_s(t) \, (\si{\volt})$},ylabel style={rotate=-90},ylabel style={at={(axis description cs:0,1.05)}},xtick={0,2}, xticklabels={$0$,$2$},ytick={0,6},yticklabels={$0$,$6$}]
\addplot[] plot coordinates {(-0.5,0) (0,0) (0,6) (2,6) (2,0) (4,0)};
\end{axis}
\end{tikzpicture}
\caption*{(ب)}
\end{subfigure}%
\caption{سوال \حوالہ{سوال_عارضی_امالہ_ث} کا دور۔}
\label{شکل_سوال_عارضی_امالہ_ث}
\end{figure}

جواب:
\begin{align*}
v_0(t)=
\begin{cases}
4(1-e^{-\frac{30t}{7}}) & 0<t<2\\
4(1-e^{-\frac{60}{7}})e^{-\frac{30}{7}(t-2)} & 2<t
\end{cases}
\end{align*}
\انتہا{سوال}
%=====================
\ابتدا{سوال}\شناخت{سوال_عارضی_امالہ_ج}
شکل \حوالہ{شکل_سوال_عارضی_امالہ_ج}-الف میں لمحہ \عددی{t=0} پر امالہ کی رو صفر کے برابر ہے۔امالہ کی رو کو \عددی{t=\SI{3}{\second}}، \عددی{t=\SI{5}{\second}} اور \عددی{t=\SI{6}{\second}} پر دریافت کریں۔ داخلی اشارہ شکل-ب میں دیا گیا ہے۔امالہ کو کامل تصور کریں۔
\begin{figure}
\centering
\begin{subfigure}{0.4\textwidth}
\centering
\begin{tikzpicture}[american voltages]
\draw(0,0) to [american voltage source,l={$v_s(t)$}]++(0,\y);
\draw(\x,0) to [resistor,*-*,l={$\SI{6}{\ohm}$}]++(0,\y);
\draw(2*\x,0) to [inductor,l={$\SI{4}{\henry}$},i<={$i_L$}]++(0,\y);
\draw(0,0) to [short]++(2*\x,0);
\draw(0,\y) to [short]++(2*\x,0);
\end{tikzpicture}
\caption*{(الف)}
\end{subfigure}%
\begin{subfigure}{0.6\textwidth}
\centering
\begin{tikzpicture}[american voltages]
\begin{axis}[small,xlabel={$t \, (\si{\second})$},ylabel={$v_s(t) \, (\si{\volt})$},ylabel style={rotate=-90},ylabel style={at={(axis description cs:0,1.05)}},xtick={0,3,5}, xticklabels={$0$,$3$,$5$},ytick={0,8,-8},yticklabels={$0$,$8$,$-8$}]
\addplot[] plot coordinates {(-0.5,0) (0,0) (0,8) (3,8) (3,-8) (5,-8) (5,0) (5.5,0)};
\end{axis}
\end{tikzpicture}
\caption*{(ب)}
\end{subfigure}%
\caption{سوال \حوالہ{سوال_عارضی_امالہ_ج} کا دور۔}
\label{شکل_سوال_عارضی_امالہ_ج}
\end{figure}

جواب:\عددی{\SI{6}{\ampere}}، \عددی{\SI{2}{\ampere}}، \عددی{\SI{2}{\ampere}}
\انتہا{سوال}
%=====================
\ابتدا{سوال}\شناخت{سوال_عارضی_دو_درجی_الف}
ایک دور کی تفرقی مساوات درج ذیل ہے۔ \عددی{i(t)} کی مساوات حاصل کریں۔
\begin{align*}
\frac{\dif^{\,2} i}{\dif t^2 }+5\frac{\dif i}{\dif t}+6i=0
\end{align*}

جوابات:\عددی{i(t)=c_1e^{-2t}+c_2e^{-3t}}
\انتہا{سوال}
%========================
\ابتدا{سوال}\شناخت{سوال_عارضی_دو_درجی_ب}
ایک دور کی تفرقی مساوات درج ذیل ہے۔ \عددی{i(t)} کی مساوات حاصل کریں۔
\begin{align*}
\frac{\dif^{\,2} v}{\dif t^2 }+7\frac{\dif v}{\dif t}+6v=0
\end{align*}

جوابات:\عددی{v(t)=c_1e^{-t}+c_2e^{-6t}}
\انتہا{سوال}
%========================
\ابتدا{سوال}\شناخت{سوال_عارضی_دو_درجی_پ}
شکل \حوالہ{شکل_سوال_عارضی_دو_درجی_پ} میں امالہ کی ابتدائی رو \عددی{i_L(0)=\SI{2}{\ampere}} ہے  اور برق گیر کا ابتدائی دباو \عددی{v_C(0)=\SI{15}{\volt}} ہے۔ ابتدائی \عددی{\tfrac{\dif v(t)}{\dif t}} دریافت کریں۔دباو \عددی{v(t)} کی مساوات حاصل کریں۔
\begin{figure}
\centering
\begin{tikzpicture}[american voltages]
\draw(0,0) to [capacitor,l={$\SI{2}{\farad}$},v_>={$v_C(t)$}]++(0,\yy);
\draw(\xx,0) to [resistor,*-*,l={$\SI{10}{\ohm}$},v_>={$v(t)$}]++(0,\yy);
\draw(2*\xx,0) to [inductor,l={$\SI{4}{\henry}$},i<={$i_L(t)$}]++(0,\yy);
\draw(0,0) to [short]++(2*\xx,0);
\draw(0,\yy) to [short]++(2*\xx,0);
\end{tikzpicture}
\caption{سوال \حوالہ{سوال_عارضی_دو_درجی_پ} کا دور۔}
\label{شکل_سوال_عارضی_دو_درجی_پ}
\end{figure}


جوابات:\عددی{\tfrac{\dif v(t)}{\dif t}=\SI{-1.75}{\volt\per\second}}، \عددی{v(t)=e^{-\tfrac{t}{40}}[15\cos \tfrac{\sqrt{199} t}{40}-\tfrac{55}{\sqrt{199}}\sin \tfrac{\sqrt{199} t}{40}]\,\si{\volt}}
\انتہا{سوال}
%========================
\ابتدا{سوال}\شناخت{سوال_عارضی_دو_درجی_ت}
شکل \حوالہ{شکل_سوال_عارضی_دو_درجی_ت} میں ازل سے منقطع سوئچ کو \عددی{t=0} پر چالو کیا جاتا ہے۔سوئچ چالو کرنے کے فوراً بعد \عددی{\tfrac{\dif v_C(t)}{\dif t}} کی قیمت دریافت کریں۔\عددی{v_C(t)} کی مساوات \عددی{t>0} کے لئے حاصل کریں۔
\begin{figure}
\centering
\begin{tikzpicture}[american voltages]
\draw(0,0) to [american current source,l={$\SI{2}{\ampere}$}]++(0,\y);
\draw(\x,0) to [resistor,*-*,l={$\SI{4}{\ohm}$}]++(0,\y);
\draw(2*\x,0) to [capacitor,*-*,l={$\SI{0.4}{\farad}$},v_>={$v_C(t)$}]++(0,\y);
\draw(3*\x,0) to [inductor,l_={$\SI{2}{\henry}$}]++(0,\y);
\draw(0,0) to [short]++(3*\x,0);
\draw(0,\y) to [short]++(2*\x,0) to [cspst,l={${t=0}$}]++(\x,0);
\end{tikzpicture}
\caption{سوال \حوالہ{سوال_عارضی_دو_درجی_ت} کا دور۔}
\label{شکل_سوال_عارضی_دو_درجی_ت}
\end{figure}


جوابات:\عددی{\tfrac{\dif v_c(t)}{\dif t}=\SI{0}{\volt\per\second}}، \عددی{v_C(t)=e^{-\tfrac{5t}{10}}[8\cos\tfrac{\sqrt{295}t}{16}+\tfrac{40}{\sqrt{295}}\sin\tfrac{\sqrt{295}t}{16}]\,\si{\volt}}
\انتہا{سوال}
%========================
\ابتدا{سوال}\شناخت{سوال_عارضی_دو_درجی_ٹ}
شکل \حوالہ{شکل_سوال_عارضی_دو_درجی_ٹ} میں لمحہ \عددی{t=0_+} پر \عددی{\tfrac{\dif i(t)}{\dif t}} کی قیمت دریافت کریں۔  \عددی{v_C(t)} کی مساوات \عددی{t>0} کے لئے حاصل کریں۔
\begin{figure}
\centering
\begin{tikzpicture}[american voltages]
\draw(0,0) to [american voltage source,l={$\SI{20}{\volt}$}]++(0,\y) to [cspst,l={${t=0}$}]++(\x,0) to [resistor,l={$\SI{1}{\kilo\ohm}$}]++(\x,0) to [inductor,l={$\SI{120}{\milli\henry}$}]++(\x,0) to [capacitor,l={$\SI{1}{\micro\farad}$},v={$v_C(t)$}]++(0,-\y) to [short] (0,0);
\end{tikzpicture}
\caption{سوال \حوالہ{سوال_عارضی_دو_درجی_ٹ} کا دور۔}
\label{شکل_سوال_عارضی_دو_درجی_ٹ}
\end{figure}

جوابات:
\begin{align*}
\left. \frac{\dif i(t)}{\dif t}\right|_{t=0_+}&=\frac{500}{3}\,\si{\ampere\per\second}\\
v_C(t)&=20+3.8675e^{-\tfrac{2500}{3}(\sqrt{13}+5)t}-23.8675e^{\tfrac{2500}{3}(\sqrt{13}-5)t}
\end{align*}
\انتہا{سوال}
%=======================
\ابتدا{سوال}\شناخت{سوال_عارضی_دو_درجی_ث}
شکل \حوالہ{شکل_سوال_عارضی_دو_درجی_ث} میں لمحہ \عددی{t=0_+} پر سوئچ کو دوسری جانب کر دیا جاتا ہے۔اس لمحے \عددی{\tfrac{\dif v(t)}{\dif t}} کی قیمت دریافت کریں۔  \عددی{v(t)} کی مساوات \عددی{t>0} کے لئے حاصل کریں۔
\begin{figure}
\centering
\begin{tikzpicture}[american voltages]
\draw(0,0) node[spdt,rotate=90](sw){};
\draw(sw.out 1) to [resistor,l={$\SI{10}{\ohm}$}]++(-\x,0)++(0,-\y)coordinate(kBL) to [american voltage source,l={$\SI{20}{\volt}$}]++(0,\y);
\draw(sw.in)coordinate(kTT) to [capacitor,l={$\SI{0.25}{\farad}$}]++(0,-\y)coordinate(kB)-|(kBL);
\draw[name path={kbot}](sw.out 2) to [short]++(2*\x,0) to [resistor,l={$\SI{10}{\ohm}$},v={$v(t)$}]++(0,-\y)|-(kB)node[circ]{}; 
\path[name path=kvert] (sw.out 2)++(\x,-0.5)--++(0,-1.5*\y);
\path[name intersections={of=kvert and kbot}];
\draw(sw.out 2)++(\x,0) to [inductor,*-*,l={$\SI{2}{\henry}$}] (intersection-1);
\draw[thick,-stealth]([shift={(120:0.7)}]kTT) arc (120:60:0.7)node[right]{${t=0}$};
\end{tikzpicture}
\caption{سوال \حوالہ{سوال_عارضی_دو_درجی_ث} کا دور۔}
\label{شکل_سوال_عارضی_دو_درجی_ث}
\end{figure}

جوابات:\عددی{\tfrac{\dif v(0_+)}{\dif t}=\SI{-8}{\volt\per\second}}،
 \عددی{v(t)=e^{-\tfrac{t}{5}}(20\cos \tfrac{7t}{5}-\tfrac{20}{7}\sin \tfrac{7t}{5})\,\si{\volt}}

\انتہا{سوال}
%========================
\ابتدا{سوال}\شناخت{سوال_عارضی_دو_درجی_ج}
شکل \حوالہ{شکل_سوال_عارضی_دو_درجی_ج} میں \عددی{v(t)} دریافت کریں۔
\begin{figure}
\centering
\begin{tikzpicture}[american voltages]
\draw(0,0) to [american voltage source,l={$20e^{-t} u(t) \, \si{\volt}$}]++(0,\y)  to [resistor,l={$\SI{2}{\ohm}$}]++(\x,0) to [inductor,l={$\SI{4}{\henry}$},v={$v(t)$}]++(\x,0) to [short]++(0,-\y) to [short] (0,0);
\end{tikzpicture}
\caption{سوال \حوالہ{سوال_عارضی_دو_درجی_ج} کا دور۔}
\label{شکل_سوال_عارضی_دو_درجی_ج}
\end{figure}

جواب:\عددی{v(t)=[30e^{-t}-10e^{-\tfrac{t}{2}}]u(t)\,\si{\volt}}
\انتہا{سوال}
%==================
\ابتدا{سوال}\شناخت{سوال_عارضی_دو_درجی_چ}
شکل \حوالہ{شکل_سوال_عارضی_دو_درجی_چ}-الف میں  \عددی{t=0} کے بعد \عددی{i(t)=\tfrac{5}{3}(1+2e^{-2t})\,\si{\ampere}} ہے۔دور میں \عددی{R_1}، \عددی{R_2} اور \عددی{L} کی قیمتیں دریافت کریں۔
\begin{figure}
\centering
\begin{subfigure}{0.5\textwidth}
\centering
\begin{tikzpicture}[american voltages]
\draw(0,0) to [american voltage source,l={$\SI{10}{\volt}$}]++(0,\y)  to [resistor,l={$R_1$}]++(\x,0) to [resistor,l={$R_2$}]++(\x,0)to
 [inductor,l={$L$},i={$i(t)$}]++(0,-\y)  to [short] (0,0);
\draw(0,\y) to [short,*-]++(0,3/4*\y) to [ospst,l={${t=0}$}]++(\x,0) to [short,-*]++(0,-3/4*\y);
\end{tikzpicture}
\caption*{(الف)}
\end{subfigure}%
\begin{subfigure}{0.5\textwidth}
\centering
\begin{tikzpicture}[american voltages]
\draw(0,0) to [inductor,l={$\SI{4}{\henry}$}]++(0,\y);
\draw(\x,0) to [resistor,*-*,l={$\SI{6}{\ohm}$},v_>={$v(t)$}]++(0,\y);
\draw(2*\x,0) to [inductor,l={$\SI{2}{\henry}$},i_<={$i_2(t)$}]++(0,\y);
\draw(0,0) to [short]++(2*\x,0);
\draw(0,\y) to [short,i<={$i_1(t)$}]++(\x,0) to [cspst,l={${t=0}$}]++(\x,0);
\end{tikzpicture}
\caption*{(ب)}
\end{subfigure}%
\caption{سوال \حوالہ{سوال_عارضی_دو_درجی_چ} کا دور۔}
\label{شکل_سوال_عارضی_دو_درجی_چ}
\end{figure}

جواب:\عددی{R_1=\SI{4}{\ohm}}، \عددی{R_2=\SI{2}{\ohm}}، \عددی{L=\SI{3}{\henry}}
\انتہا{سوال}
%==================
\ابتدا{سوال}\شناخت{سوال_عارضی_دو_درجی_ح} 
شکل \حوالہ{شکل_سوال_عارضی_دو_درجی_چ}-ب میں \عددی{i_1(0_-)=\SI{6}{\ampere}} ہے۔لمحہ \عددی{t=0} پر سوئچ کو چالو کیا جاتا ہے۔\عددی{i_2(0_+)}، \عددی{v(0_+)} اور \عددی{i_1(t \to \infty)} دریافت کریں۔

جوابات:\عددی{i_2(0_+)=\SI{0}{\ampere}}، \عددی{v(0_+)=\SI{-36}{\volt}}، \عددی{i_1(t \to \infty)=\SI{0}{\ampere}}
\انتہا{سوال}
%=====================
\ابتدا{سوال}\شناخت{سوال_عارضی_دو_درجی_خ}
شکل \حوالہ{شکل_سوال_عارضی_دو_درجی_خ} میں  \عددی{t=0} پر \عددی{R_2} کو منقطع کیا جاتا ہے جس کے بعد \عددی{v_C(t)=120-80e^{-\tfrac{t}{2}}\,\si{\volt}} حاصل ہوتا ہے۔دور میں \عددی{R_1}، \عددی{R_2} اور \عددی{C} کی قیمتیں دریافت کریں۔
\begin{figure}
\centering
\begin{tikzpicture}[american voltages]
\draw(0,0) to [american current source,l={$\SI{4}{\ampere}$}]++(0,2*\y);
\draw(\x,0)  to [resistor,*-*,l={$R_1$}]++(0,2*\y);
\draw(2*\x,0) to [ospst,*-,l={${t=0}$}]++(0,\y) to [resistor,-*,l={$R_2$}]++(0,\y);
\draw(3*\x,0) to [capacitor,l={$C$},v_>={$v_C(t)$}]++(0,2*\y);
\draw(0,0) to [short]++(3*\x,0);
\draw(0,2*\y) to [short]++(3*\x,0);
\end{tikzpicture}
\caption{سوال \حوالہ{سوال_عارضی_دو_درجی_خ} کا دور۔}
\label{شکل_سوال_عارضی_دو_درجی_خ}
\end{figure}

جواب:\عددی{R_1=\SI{20}{\ohm}}، \عددی{R_2=\SI{10}{\ohm}}، \عددی{C=\SI{0.1}{\farad}}
\انتہا{سوال}
%==================
\ابتدا{سوال}\شناخت{سوال_عارضی_دو_درجی_د}
شکل \حوالہ{شکل_سوال_عارضی_دو_درجی_د} میں \عددی{v_C(t)=\tfrac{20}{7}[e^(-\tfrac{t}{4})-e^(-2t)]\,\si{\volt}} ہے۔دور میں \عددی{C} کی قیمتیں دریافت کریں۔
\begin{figure}
\centering
\begin{tikzpicture}[american voltages]
\draw(0,0) to [american voltage source,l={$20e^{-2t}u(t)\,\si{\volt}$}]++(0,\y) to [resistor,l={$\SI{6}{\ohm}$}]++(\x,0) to [capacitor,l={$C$},v={$v_C(t)$}]++(\x,0) to [resistor,l={$\SI{4}{\ohm}$}]++(0,-\y) to [short] (0,0);
\end{tikzpicture}
\caption{سوال \حوالہ{سوال_عارضی_دو_درجی_د} کا دور۔}
\label{شکل_سوال_عارضی_دو_درجی_د}
\end{figure}

جواب:\عددی{C=\SI{0.4}{\farad}}
\انتہا{سوال}
%==================
\ابتدا{سوال}\شناخت{سوال_عارضی_دو_درجی_ڈ}
شکل \حوالہ{شکل_سوال_عارضی_دو_درجی_ڈ} میں ازل سے منبع دور کے ساتھ منسلک ہے  جس کو \عددی{t=0} پر دور سے منقطع کیا جاتا ہے۔دور میں  \عددی{i_1(t)} اور \عددی{i_2(t)} دریافت کریں۔
\begin{figure}
\centering
\begin{tikzpicture}[american voltages]
\draw(0,0) node[spdt,rotate=90](sw){};
\draw(sw.out 1) to [resistor,l={$\SI{2}{\ohm}$}]++(-\x,0)++(0,-\y)coordinate(kBL) to [american voltage source,l={$\SI{10}{\volt}$}]++(0,\y);
\draw(sw.in)coordinate(kTT) to [inductor,l_={$\SI{3}{\henry}$}]++(0,-\y)coordinate(kB)-|(kBL);
\draw[name path={kbot}](sw.out 2) to [short]++(2*\x,0) to [resistor,l={$\SI{10}{\ohm}$}]++(0,-\y)|-(kB)node[circ]{}; 
\path[name path=kvert] (sw.out 2)++(\x,-0.5)--++(0,-1.5*\y);
\path[name intersections={of=kvert and kbot}];
\draw(sw.out 2)++(\x,0) to [inductor,*-*,l_={$\SI{2}{\henry}$}] (intersection-1);
\draw[thick,-stealth]([shift={(120:0.7)}]kTT) arc (120:60:0.7)node[right]{${t=0}$};
%currents
\path(kB)++(\x/2,\y/2)coordinate(kC)++(\x+\x/4,0)coordinate(kD);
\draw[stealth-]([shift={(150:\x/4)}]kC) arc (150:-150:\x/4);
\draw[stealth-]([shift={(150:\x/4)}]kD) arc (150:-150:\x/4);
\draw(kC)node{$i_1(t)$};
\draw(kD)node{$i_2(t)$};
\end{tikzpicture}
\caption{سوال \حوالہ{سوال_عارضی_دو_درجی_ڈ} کا دور۔}
\label{شکل_سوال_عارضی_دو_درجی_ڈ}
\end{figure}


جواب:
\begin{align*}
i_1(t)&=-2e^{-\frac{25t}{3}}-3 \,\si{\ampere}\\
i_2(t)&=-3e^{-\frac{25t}{3}}-3\,\si{\ampere}
\end{align*}
\انتہا{سوال}
%==================


\باب{عارضی رد عمل}
\حصہ{تعارف}
ایسے ادوار جن میں امالہ گیر اور (یا) برق گیر پائے جاتے ہوں میں توانائی ذخیرہ کرنے کی صلاحیت ہوتی ہے۔توانائی ذخیرہ کرنے والے ادوار کا رد عمل منبع طاقت کے علاوہ ذخیرہ توانائی پر بھی منحصر ہوتا ہے۔ایسے ادوار میں کسی بھی طرح کی تبدیلی سے ذخیرہ توانائی میں تبدیلی رونما ہو سکتی ہے۔دور میں تبدیلی مثلاً  کسی سوئچ کے چالو یا غیر چالو کرنے سے پیدا ہو سکتی ہے۔ایسی صورت جہاں دور یکساں ایک ہی حالت میں رہے کو \اصطلاح{برقرار حالت}\فرہنگ{برقرار حالت}\حاشیہب{steady state}\فرہنگ{steady state} کہتے ہیں۔تبدیلی کے بعد دور متبادل برقرار حالت اختیار کرتا ہے۔ایک برقرار حالت سے دوسری برقرار حالت تک پہنچنے کے دوران، دور  \اصطلاح{عارضی حالت}\فرہنگ{عارضی حالت}\حاشیہب{transient state}\فرہنگ{transient state} میں ہوتا ہے۔

\حصہ{ایک درجی ادوار}
وہ ادوار جن میں صرف امالہ گیر توانائی ذخیرہ کرتے ہوں کی کرخوف مساوات \اصطلاح{ایک درجی تفرقی مساوات}\فرہنگ{تفرقی مساوات!ایک درجی}\فرہنگ{ایک درجی!تفرقی مساوات}\حاشیہب{first order differential equation}\فرہنگ{differential equation!first order} ہوتی ہے۔اسی طرح وہ ادوار جن میں صرف برق گیر توانائی ذخیرہ کرتے ہوں بھی ایک درجی کرخوف مساوات  دیتے ہیں۔اسی لئے انہیں \اصطلاح{یک درجی ادوار}\فرہنگ{یک درجی ادوار}\فرہنگ{دور!یک درجی}\حاشیہب{first order circuits}\فرہنگ{first order!circuits} کہتے ہیں۔اس کے برعکس ایسے ادوار جن میں امالہ گیر اور برق گیر دونوں پائے جاتے ہوں \اصطلاح{دو درجی تفرقی مساوات}\فرہنگ{دو درجی!تفرقی مساوات}\فرہنگ{تفرقی مساوات!دو درجی}\حاشیہب{second order differential equations}\فرہنگ{differential equation!second order} دیتے ہیں اور انہیں \اصطلاح{دو درجی ادوار}\فرہنگ{دو درجی!ادوار}\فرہنگ{دور!دو درجی}\حاشیہب{second order circuits}\فرہنگ{second order!circuits} کہا جاتا ہے۔

\begin{figure}
\centering
\begin{subfigure}{0.5\textwidth}
\centering
\begin{tikzpicture}
\draw(0,0) to [american voltage source,l={$v_i(t)$}]++(0,\y) to [resistor,i>^={$i(t)$},l={$R$}]++(\x,0) to [inductor,l={$L$}]++(0,-\y) to [short]++(-\x,0);
\end{tikzpicture}
\caption*{(الف)}
\end{subfigure}%
\begin{subfigure}{0.5\textwidth}
\centering
\begin{tikzpicture}
\draw(0,0) to [american current source,l={$i_i(t)$}]++(0,\y) to [short]++(2*\x,0) to [capacitor,i={$i_C(t)$},l_={$C$}]++(0,-\y) to [short]++(-2*\x,0);
\draw(\x,0)node[ground]{} to [resistor,*-*,i<_={$i_R(t)$},l={$R$}]++(0,\y)node[above]{$v(t)$};
\end{tikzpicture}
\caption*{(ب)}
\end{subfigure}%
\caption{ایک درجی ادوار کی مثالیں۔}
\label{شکل_عارضی_ایک_درجی_ادوار_الف}
\end{figure}

شکل \حوالہ{شکل_عارضی_ایک_درجی_ادوار_الف} میں ایک درجی ادوار کی مثالیں دی گئی ہیں۔آئیں ان کی کرخوف مساوات لکھ کر دیکھیں۔شکل-الف کی مساوات درج ذیل ہے۔
\begin{align}
v(t)=i(t)R+L \frac{\dif i(t)}{\dif t}
\end{align}
اسی طرح شکل-ب کی کرخوف مساوات درج ذیل ہے۔
\begin{align}
i_i(t)=\frac{v(t)}{R}+C\frac{\dif v(t)}{\dif t}
\end{align}
آپ دیکھ سکتے ہیں کہ درج بالا دونوں مساوات ایک درجی تفرقی مساوات ہیں۔

\begin{figure}
\centering
\begin{tikzpicture}
\draw(0,0) to [american voltage source,l={$v_i(t)$}]++(0,\y) to [resistor,i={$i(t)$},l={$R$}]++(\x,0) to [inductor,l={$L$}]++(\x,0) to [capacitor,l={$C$}]++(0,-\y) to [short]++(-2*\x,0);
\end{tikzpicture}
\caption{دو درجی دور۔}
\label{شکل_عارضی_دور_درجی_دور_الف}
\end{figure}

شکل \حوالہ{شکل_عارضی_دور_درجی_دور_الف} میں دو درجی دور دکھایا گیا ہے جس کی کرخوف مساوات درج ذیل ہے۔
\begin{align*}
v_i(t)=R i(t)+L\frac{\dif i(t)}{\dif t}+\frac{1}{C} \int_{-\infty}^{t}i(t) \dif t
\end{align*}
اس مساوات میں تکمل کی علامت ختم کرنے سے تفرقی مساوات حاصل ہو گی۔تکمل کی علامت ختم کرنے کی خاطر اس کا تفرق لیتے ہیں۔
\begin{align}
\frac{\dif v_i(t)}{\dif t}=R \frac{\dif i(t)}{\dif t}+L\frac{\dif^{\,2} i(t)}{\dif t^2}+\frac{i(t)}{C}
\end{align} 
آپ دیکھ سکتے ہیں کہ امالہ گیر اور برق گیر دونوں کی موجودگی سے دو درجی تفرقی مساوات حاصل ہوتی ہے۔

\جزوحصہ{رد عمل کی عمومی مساوات}
ایک درجی ادوار کے رد عمل جاننے کی خاطر ان کی تفرقی مساوات حل کی جاتی ہے جس سے دور کے مختلف مقامات پر دباو اور رو حاصل کی جاتی ہے۔ان یک درجی مساوات کی عمومی صورت درج ذیل ہوتی ہے
\begin{align}\label{مساوات_عارضی_یک_درجی_عمومی_مساوات}
\frac{\dif y(t)}{\dif t}+a y(t)= g(t)
\end{align}
جہاں \عددی{y(t)} دباو یا رو کو ظاہر کرتی ہے، \عددی{a} مستقل ہے اور \عددی{g(t)} \اصطلاح{تفاعل عملی}\فرہنگ{تفاعل عملی}\حاشیہب{forcing function}\فرہنگ{forcing function} ہے۔اس مساوات کا آزاد متغیرہ وقت \عددی{t} ہے۔تفرقی مساوات کا ایک بنیادی مسئلہ کہتا ہے کہ مساوات \حوالہ{مساوات_عارضی_یک_درجی_عمومی_مساوات} کا مکمل حل اس کے \اصطلاح{فطری رد عمل}\فرہنگ{فطری رد عمل}\فرہنگ{رد عمل!فطری}\حاشیہب{natural response, complementary solution}\فرہنگ{complementary solution}\فرہنگ{natural response}\فرہنگ{response!natural} \عددی{y_f(t)} اور \اصطلاح{جبری رد عمل}\فرہنگ{جبری رد عمل}\فرہنگ{رد عمل!جبری}\حاشیہب{forced response, particular solution}\فرہنگ{particular solution}\فرہنگ{forced response} 
\عددی{y_j(t)} کا مجموعہ ہے۔مساوات \حوالہ{مساوات_عارضی_یک_درجی_عمومی_مساوات} کے کسی بھی حل کو بطور جبری رد عمل لیا جا سکتا ہے جبکہ درج ذیل \اصطلاح{ہم جنسی مساوات}\فرہنگ{ہم جنسی مساوات}\فرہنگ{مساوات!ہم جنسی}\حاشیہب{homogenous equation}\فرہنگ{homogenous equation}
\begin{align}\label{مساوات_عارضی_یک_درجی_عمومی_مساوات_ب}
\frac{\dif y(t)}{\dif t}+a y(t)=0
\end{align}
 کے کسی بھی حل کو فطری رد عمل تصور کیا جا سکتا ہے۔مساوات \حوالہ{مساوات_عارضی_یک_درجی_عمومی_مساوات} میں \عددی{g(t)=0} پُر کرنے سے ہم جنسی مساوات  حاصل ہوتی ہے۔

آئیں \عددی{g(t)=A} کی صورت میں مساوات \حوالہ{مساوات_عارضی_یک_درجی_عمومی_مساوات} کا حل حاصل کریں جہاں \عددی{A} ایک مستقل ہے۔یوں ہمیں درج ذیل دو مساوات کے حل درکار ہیں۔
\begin{align}
\frac{\dif y_j(t)}{\dif t}+a y_j(t)&=A \label{مساوات_عارضی_یک_درجی_عمومی_مساوات_پ}\\
\frac{\dif y_f(t)}{\dif t}+a y_f(t)&=0\label{مساوات_عارضی_یک_درجی_عمومی_مساوات_ت}
\end{align}
جبری حل کو قیاس کے ذریعہ حاصل کیا جائے گا۔  جبری حل کو تفاعل عملی اور اس کے تمام ممکنہ تفرق کے مجموعے کے برابر تصور کرتے ہوئے آگے بڑھتے ہیں۔چونکہ مستقل کا تفرق \عددی{(\tfrac{\dif A}{\dif t}=0)} صفر کے برابر ہے لہٰذا جبری حل کو مستقل \عددی{K_1} تصور کرتے ہیں۔
\begin{align}
y_j(t)=K_1
\end{align}
اس قیمت کو مساوات \حوالہ{مساوات_عارضی_یک_درجی_عمومی_مساوات_پ} میں پُر کرتے ہوئے حل کرنے سے
\begin{align*}
\frac{\dif K_1}{\dif t}+a K_1&=A \\
0+a K_1&=A
\end{align*}
یعنی
\begin{align}\label{مساوات_عارضی_یک_درجی_عمومی_مساوات_ٹ}
K_1=\frac{A}{a}
\end{align}
حاصل ہوتا ہے۔مساوات \حوالہ{مساوات_عارضی_یک_درجی_عمومی_مساوات_ت} کو ترتیب دیتے ہوئے
\begin{align*}
\frac{\dif y_f(t)}{y_f(t)}=-a \dif t
\end{align*}
لکھا جا سکتا ہے  جس کا تکمل
\begin{align*}
\ln y_f(t)=-a t +c
\end{align*}
یعنی
\begin{align}\label{مساوات_عارضی_یک_درجی_عمومی_مساوات_ث}
y_f(t)=K_2e^{-at}
\end{align}
کے برابر ہے جہاں \عددی{c} تکمل کا مستقل ہے اور \عددی{K_2=e^{c}} کے برابر ہے۔مساوات \حوالہ{مساوات_عارضی_یک_درجی_عمومی_مساوات_ٹ} اور مساوات \حوالہ{مساوات_عارضی_یک_درجی_عمومی_مساوات_ث} سے مکمل حل درج ذیل حاصل ہوتا ہے۔
\begin{align}
y(t)=\frac{A}{a}+K_2 e^{-at}
\end{align}
کسی بھی لمحے پر \عددی{y(t)} جاننے سے درج بالا مساوات میں نا معلوم مستقل \عددی{K_2} دریافت کیا جا سکتا ہے۔درج بالا مساوات کو درج ذیل عمومی حل کی صورت میں لکھا جا سکتا ہے
\begin{align}\label{مساوات_عارضی_یک_درجی_عمومی_مساوات_ج}
y(t)=K_1+K_2 e^{-\frac{t}{\tau}}
\end{align}
جہاں \عددی{\tau=\tfrac{1}{a}} کے برابر ہے۔


مساوات \حوالہ{مساوات_عارضی_یک_درجی_عمومی_مساوات_ج} کے مختلف اجزاء کو نام دیے گئے ہیں۔یوں \عددی{\tau} \اصطلاح{وقتی مستقل}\فرہنگ{وقتی مستقل}\حاشیہب{time constant}\فرہنگ{time constant} کہلاتا ہے جبکہ \عددی{K_1} \اصطلاح{برقرار حالت حل}\فرہنگ{برقرار حالت حل}\فرہنگ{حل:برقرار حالت}\فرہنگ{برقرار حالت:حل}\حاشیہب{steady state solution}\فرہنگ{steady state solution} کہلاتا ہے۔مساوات \حوالہ{مساوات_عارضی_یک_درجی_عمومی_مساوات_ج} میں \عددی{t=\infty} پُر کرنے سے برقرار حالت حل حاصل ہوتا ہے۔یوں کسی بھی تبدیلی کے بہت دیر بعد دور برقرار حالت میں ہو گا یعنی ابدی صورت کو برقرار حالت کہا جاتا ہے۔

\begin{figure}
\begin{tikzpicture}
\begin{axis}[name=ka,axis lines*=middle,
	 every axis x label/.style={
    at={(ticklabel* cs:1.05)},
    anchor=east,}, 
	every axis y label/.style={
    at={(ticklabel* cs:1.05)},
    anchor=east,}
,xlabel=$t$,ylabel=$K_2 e^{-\frac{t}{\tau}}$,ytick={1,0.368},yticklabel style={/pgf/number format/precision=3},yticklabels={$K_2$,$0.368 K_2$},xtick={0.5,1,1.5,2,2.5},xticklabels={$\tau$,$2\tau$,$3\tau$,$4\tau$,$5\tau$}]
\addplot[width=4cm,domain=0:3,samples=100]{e^(-x/0.5)};
\draw[dashed](axis cs:0,1)--(axis cs:0.5,0);
\draw[dashed](axis cs:0,0.368)--(axis cs:0.5,0.368)--(axis cs:0.5,0);
\end{axis}
\node [anchor=north] at (ka.south){(الف)};
\end{tikzpicture}%
\begin{tikzpicture}
\begin{axis}[name=kb,axis lines*=middle,
 every axis x label/.style={
    at={(ticklabel* cs:1.05)},
    anchor=east,}, 
	every axis y label/.style={
    at={(ticklabel* cs:1.05)},
    anchor=east,},
 xlabel=$t$,ylabel=$e^{-\frac{t}{\tau}}$]
\addplot[width=4cm,domain=0:3,samples=100]{e^(-x/0.5)}node[pos=0.25,above right]{$\tau=0.5$};
\addplot[width=4cm,domain=0:3,samples=100]{e^(-x/2)}node[pos=0.25,above right]{$\tau=2$};
\end{axis}
\node[anchor=north] at (ka.south){(ب)};
\end{tikzpicture}
\caption{وقتی مستقل}
\label{شکل_عارضی_وقتی_مستقل_الف}
\end{figure}

شکل \حوالہ{شکل_عارضی_وقتی_مستقل_الف}-الف میں مثبت \عددی{a} کی صورت میں جبری حل دکھایا گیا ہے۔ابتدائی لمحہ \عددی{t=0} پر \عددی{y_j(0)=K_2} کے برابر ہے جبکہ ایک وقتی مستقل برابر وقت بعد اس کی قیمت \عددی{y_j(\tau)=0.368K_2} رہ گئی ہے یعنی \عددی{\tau} دورانیے میں جبری حل کی قیمت میں \عددی{\SI{63.2}{\percent}} کمی واقع ہوئی ہے۔اسی طرح دو وقتی مستقل وقفے کے بعد \عددی{y_j(2\tau)=0.135K_2} ہے جو \عددی{y_p(\tau)} کے \عددی{0.368} گنا ہے۔حقیقت میں کسی بھی لمحہ \عددی{t_1} پر \عددی{y_j} کی قیمت میں لمحہ \عددی{t_1+\tau} پر \عددی{\SI{63.2}{\percent}} کمی واقع ہو گی۔پانچ وقتی مستقل وقفے کے بعد \عددی{y_j(5\tau)=0.0067K_2} رہ جاتا ہے جو ابتدائی قیمت کے \عددی{\SI{0.67}{\percent}} ہے۔

مساوات \حوالہ{مساوات_عارضی_یک_درجی_عمومی_مساوات_ث}  \اصطلاح{قوت نمائی انحطاطی}\فرہنگ{قوت نمائی!انحطاط}\حاشیہب{exponential decaying}\فرہنگ{exponential decay} خط ہے۔قوت نمائی انحطاطی خط کی ایک خصوصیت یہ ہے کہ ابتدائی لمحے  پر اس کا مماس افقی محور کو \عددی{\tau} پر کاٹتا ہے۔اس مماس کو شکل \حوالہ{شکل_عارضی_وقتی_مستقل_الف}-الف میں \عددی{(0,K_2)} تا \عددی{(\tau,0)} نقطہ دار لکیر سے دکھایا گیا ہے۔ شکل \حوالہ{شکل_عارضی_وقتی_مستقل_الف}-ب میں مختلف \عددی{\tau} کی قیمتوں کے لئے مساوات \حوالہ{مساوات_عارضی_یک_درجی_عمومی_مساوات_ث}  کو کھینچا گیا ہے۔آپ دیکھ سکتے ہیں کہ کم وقتی مستقل کا خط جلد اختتامی قیمت تک پہنچتا ہے۔یوں وقتی مستقل کسی بھی دور کے رد عمل کے دورانیے کی ناپ ہے۔
%=======================

\ابتدا{مثال}\شناخت{مثال_عارضی_یک_درجی_دور_الف}
شکل \حوالہ{شکل_عارضی_سلسلہ_وار_مزاحمت_برق_گیر_الف} میں مزاحمت اور بے بار برق گیر سلسلہ وار جڑے ہیں۔لمحہ \عددی{t=0} پر \اصطلاح{سوئچ}\فرہنگ{سوئچ}\حاشیہد{اس طرز کے سوئچ کا پورا نام ایک قطب ایک چال سوئچ ہے۔}\حاشیہب{switch, spst, single pole single throw}\فرہنگ{switch} چالو کرتے ہوئے انہیں مستقل منبع دباو \عددی{V_I} کے ساتھ جوڑا جاتا ہے۔برق گیر کا دباو \عددی{v(t)} اور رو \عددی{i(t)} دریافت کریں۔

\begin{figure}
\centering
\begin{tikzpicture}
\draw(0,0) to [american voltage source,l={$V_I$}]++(0,\y) to [cspst,l={${t=0}$}]++(\x,0) to [resistor,l={$R$}]++(\x,0)node[above]{$v(t)$} to [capacitor,-*,l={$C$}]++(0,-\y) node[ground]{} to [short] (0,0);
\draw(\x/2+\dx,\y-\dy)node[below]{سوئچ};
\draw(\x,-0.5)node{(الف)};
\end{tikzpicture}%
\begin{tikzpicture}
\begin{axis}[name=kb,axis lines*=middle,
	 every axis x label/.style={
    at={(ticklabel* cs:1.05)},
    anchor=east,}, 
	every axis y label/.style={
    at={(ticklabel* cs:1.05)},
    anchor=east,}
,xlabel=$t$,ylabel=$v(t)$,ytick={1,0.5},yticklabel style={/pgf/number format/precision=3},yticklabels={$V_I$,$0.5 V_I$},xtick={1,2,3,4,5},xticklabels={$\tau$,$2\tau$,$3\tau$,$4\tau$,$5\tau$}]
\addplot[width=4cm,domain=0:5,samples=100]{1-e^(-x)}node[pos=0.3,below right]{$v(t)=V_I \left(1-e^{-\frac{t}{RC}}\right)$};
\end{axis}%
\node [anchor=north] at (kb.south){(ب)};
\end{tikzpicture}%
\begin{tikzpicture}
\begin{axis}[name=kc,axis lines*=middle,
	 every axis x label/.style={
    at={(ticklabel* cs:1.05)},
    anchor=east,}, 
	every axis y label/.style={
    at={(ticklabel* cs:1.05)},
    anchor=east,}
,xlabel=$t$,ylabel=$i(t)$,ytick={1,0.5},yticklabel style={/pgf/number format/precision=3},yticklabels={$\frac{V_I}{R}$,$0.5\frac{V_I}{R}$},xtick={1,2,3,4,5},xticklabels={$\tau$,$2\tau$,$3\tau$,$4\tau$,$5\tau$}]
\addplot[width=4cm,domain=0:5,samples=100]{e^(-x)}node[pos=0.3,above right]{$i(t)=\frac{V_I}{R}e^{-\frac{t}{RC}}$};
\end{axis}%
\node [anchor=north] at (kc.south){(پ)};
\end{tikzpicture}%
\caption{مثال \حوالہ{مثال_عارضی_یک_درجی_دور_الف} کا دور، دباو اور رو۔}
\label{شکل_عارضی_سلسلہ_وار_مزاحمت_برق_گیر_الف}
\end{figure} 

حل:سوئچ چالو کرنے سے پہلے برق گیر بے بار ہے لہٰذا اس پر دباو صفر کے برابر ہے۔صفحہ \حوالہصفحہ{مساوات_امالہ_برق_گیر_دباو_بلا_جوڑ_ہے} پر مساوات \حوالہ{مساوات_امالہ_برق_گیر_دباو_بلا_جوڑ_ہے} کے تحت \عددی{v_C(0_+)=v_C(0_-)} ہو گا یعنی یوں سوئچ چالو کرنے کے فوراً بعد برق گیر کا دباو صفر ہی ہو گا۔سوئچ چالو کرنے کے بعد  دباو جوڑ \عددی{v(t)} کے استعمال سے کرخوف مساوات رو لکھتے ہیں
\begin{align*}
\frac{v(t)-V_I}{R}+C\frac{\dif v(t)}{\dif t}=0
\end{align*}
جسے ترتیب دیتے ہوئے
\begin{align}\label{مساوات_عارضی_برق_گیر_عارضی_حل_الف}
\frac{\dif v(t)}{\dif t}+\frac{v(t)}{RC}=\frac{V_I}{RC}
\end{align}
لکھا  جا سکتا ہے جو عمومی مساوات \حوالہ{مساوات_عارضی_یک_درجی_عمومی_مساوات} کی طرح ہے۔چونکہ \عددی{V_I} مستقل قیمت ہے لہٰذا اس مساوات کا جبری حل
\begin{align*}
v_j(t)=K_1
\end{align*}
 تصور کیا جا سکتا ہے جسے  مساوات \حوالہ{مساوات_عارضی_برق_گیر_عارضی_حل_الف} میں پُر کرتے ہوئے حل کرنے سے
\begin{align*}
\frac{\dif  K_1}{\dif t}+\frac{K_1}{RC}&=\frac{V_I}{RC}\\
0+\frac{K_1}{RC}&=\frac{V_I}{RC}
\end{align*}
یعنی
\begin{align*}
K_1=V_I
\end{align*}
حاصل ہوتا ہے۔یوں جبری حل درج ذیل حاصل ہوتا ہے۔
\begin{align*}
v_j(t)=V_I
\end{align*}
اس نتیجے کے تحت سوئچ چالو کرنے کے بہت دیر بعد برق گیر پر دباو عین منبع دباو کے برابر ہو گا۔شکل کو دیکھتے ہوئے اسی نتیجے تک یوں پہنچا جا سکتا ہے کہ سوئچ چالو کرنے کے بعد دور میں رو کی وجہ سے برق گیر پر بار جمع ہونا شروع ہو جائے گا۔جب تک برق گیر کا دباو منبع کے دباو سے کم ہو، مزاحمت پر دباو پایا جائے گا لہٰذا اس میں رو پائی جائے گی۔یہ رو برق گیر پر جمع بار میں اضافہ کرتی رہے گی۔عین اس وقت جب برق گیر اور منبع کے دباو برابر ہو جائیں، رو کی قیمت صفر ہو جائے گی اور برق گیر کا دباو اسی قیمت پر ابد تک برقرار رہے گا۔ 

آئیں اب فطری حل دریافت کریں۔فطری حل ہم جنسی مساوات سے حاصل ہوتا ہے۔مساوات \حوالہ{مساوات_عارضی_برق_گیر_عارضی_حل_الف} کے دائیں بازو کو صفر کے برابر پُر کرنے سے ہم جنسی مساوات
\begin{align}\label{مساوات_عارضی_برق_گیر_ہم_جنسی_الف}
\frac{\dif v(t)}{\dif t}+\frac{v(t)}{RC}=0
\end{align}
 حاصل ہوتی ہے۔اس کو
\begin{align*}
\frac{\dif v(t)}{v(t)}=-\frac{\dif t}{RC}
\end{align*}
لکھتے ہوئے تکمل لینے سے
\begin{align*}
\ln v(t)=-\frac{t}{RC}+c
\end{align*}
یعنی
\begin{align*}
v_f(t)=K_2 e^{-\frac{t}{RC}}
\end{align*}
فطری حل حاصل ہوتا ہے۔ جبری اور فطری حل کا مجموعہ مکمل حل ہو گا۔
\begin{align*}
v(t)=V_I+K_2 e^{-\frac{t}{RC}}
\end{align*}
مکمل حل میں نا معلوم مستقل کو \اصطلاح{ابتدائی شرائط}\فرہنگ{ابتدائی شرائط}\حاشیہب{initial conditions}\فرہنگ{initial conditions} سے حاصل کرتے ہیں جس کے تحت \عددی{t=0_+} پر \عددی{v_C(0_+)=0} کی قیمت معلوم ہے۔ان قیمتوں کو درج بالا مساوات میں پُر کرتے ہوئے حل کرنے سے
\begin{align*}
0&=V_I+K_2 e^{-\frac{0}{RC}}\\
0&=V_I+K_2
\end{align*}
یعنی
\begin{align*}
K_2=-V_I
\end{align*}
حاصل ہوتا ہے۔ 

جبری حل اور فطری حل کا مجموعہ مکمل حل دیتا ہے
\begin{gather}
\begin{aligned}
v(t)&=v_j(t)+v_f(t)\\
&=V_I\left(1-e^{-\frac{t}{RC}}\right)\\
&=V_I\left(1-e^{-\frac{t}{\tau}}\right)
\end{aligned}
\end{gather}
درج بالا مساوات میں وقتی مستقل درج ذیل ہے۔
\begin{align}
\tau=RC
\end{align}
یوں \عددی{R} یا (اور) \عددی{C} بڑھانے سے  وقتی مستقل بڑھے گا جس سے دور برقرار صورت زیادہ دیر کے بعد اختیار کرے گا۔ 

رو \عددی{i(t)} کو درج بالا مساوات سے حاصل کرتے ہیں۔
\begin{align*}
i(t)&=C\frac{\dif v(t)}{\dif t}\\
&=C V_I \left(0+\frac{1}{RC}e^{-\frac{t}{RC}}\right)\\
&=\frac{V_I}{R}e^{-\frac{t}{RC}}
\end{align*}
یہی رو مزاحمت پر اوہم کے قانون کی مدد سے بھی حاصل کی جا سکتی ہے یعنی
\begin{align*}
i(t)&=\frac{V_I-v(t)}{R}\\
&=\frac{V_I}{R}e^{-\frac{t}{RC}}
\end{align*} 
\انتہا{مثال}
%===============
\ابتدا{مثال}\شناخت{مثال_عارضی_مزاحمت_امالہ_عارضی_الف}
شکل \حوالہ{شکل_عارضی_مزاحمت_امالہ_الف} میں لمحہ \عددی{t=0} پر سوئچ چالو کیا جاتا ہے۔رو کا خط کھینچیں۔
\begin{figure}
\centering
\begin{tikzpicture}
\draw(0,0) to [american voltage source,l={$V_I$}]++(0,\y) to [cspst,l={${t=0}$}]++(\x,0) to [resistor,i>_={$i(t)$},l={$R$}]++(\x,0) to [inductor,-*,l={$L$}]++(0,-\y)node[ground]{} to [short] (0,0);
\end{tikzpicture}%
\begin{tikzpicture}
\begin{axis}[name=kb,xlabel=$t$,ylabel=$i(t)$,ytick={1,0.5},yticklabels={$\frac{V_I}{R}$,$0.5\frac{V_I}{R}$},xtick={1,2,3,4,5},xlabel={$t$},xticklabels={$\tau$,$2\tau$,$3\tau$,$4\tau$,$5\tau$},
axis lines*=middle,
 every axis x label/.style={
    at={(ticklabel* cs:1.05)},
    anchor=east,}, 
	every axis y label/.style={
    at={(ticklabel* cs:1.05)},
    anchor=east,}
]
\addplot[domain=0:5]{1-e^(-x)};
\end{axis}%
\end{tikzpicture}%
\caption{مثال \حوالہ{مثال_عارضی_مزاحمت_امالہ_عارضی_الف} کے اشکال۔}
\label{شکل_عارضی_مزاحمت_امالہ_الف}
\end{figure}

حل:کرخوف مساوات دباو
\begin{align*}
V_I=i(t)R+L\frac{\dif i(t)}{\dif t}
\end{align*}
کو ترتیب دیتے ہوئے عمومی شکل میں لاتے ہیں
\begin{align}\label{مساوات_عارضی_امالہ_گیر_عمومی_الف}
\frac{\dif i(t)}{\dif t}+\frac{R}{L} i(t)=\frac{V_I}{L}
\end{align}
جس کا جبری حل
\begin{align*}
i_j(t)=K_1
\end{align*}
ہو گا۔جبری حل کو عمومی مساوات میں پُر کرتے ہوئے حل کرنے سے
\begin{align*}
\frac{\dif K_1}{\dif t}+\frac{R}{L} K_1&=\frac{V_I}{L}\\
0+\frac{R}{L} K_1&=\frac{V_I}{L}
\end{align*}
یعنی
\begin{align*}
K_1=\frac{V_I}{R}
\end{align*}
حاصل ہوتا ہے جس سے جبری حل درج ذیل لکھا جائے گا۔
\begin{align*}
i_j(t)=\frac{V_I}{R}
\end{align*}
یہی جواب منطق سے بھی حاصل کیا جا سکتا ہے۔چونکہ  یک سمتی رو کے لئے امالہ گیر بطور قصر دور کردار ادا کرتا ہے لہٰذا عارضی دورانیہ گزر جانے کے بعد ہم امالہ گیر کو قصر دور تصور کر سکتے ہیں۔شکل \حوالہ{شکل_عارضی_مزاحمت_امالہ_الف} میں امالہ گیر کو قصر دور کرتے ہوئے اوہم کے قانون سے \عددی{i_j(t)=\tfrac{V_I}{R}} لکھا جا سکتا ہے۔

فطری حل حاصل کرنے کی خاطر مساوات \حوالہ{مساوات_عارضی_امالہ_گیر_عمومی_الف} میں دیے گئے عمومی مساوات کا دایاں ہاتھ صفر کے برابر پُر کرتے ہوئے درج ذیل ہم جنسی مساوات حاصل کرتے ہیں۔
\begin{align*}
\frac{\dif i(t)}{\dif t}+\frac{R}{L} i(t)=0
\end{align*}
اس کو ترتیب دیتے ہوئے
\begin{align*}
\frac{\dif i(t)}{i(t)}=-\frac{R}{L} \dif t
\end{align*}
تکمل لینے سے
\begin{align*}
\ln i(t)=-\frac{R}{L}t +c
\end{align*}
یعنی
\begin{align*}
i_f(t)=K_2e^{-\frac{R}{L}t}
\end{align*}
حاصل ہوتا ہے۔

جبری اور فطری حل کا مجموعہ مکمل حل دیتا ہے
\begin{gather}
\begin{aligned}\label{مساوات_عارضی_مزاحمت_امالہ_مکمل_حل_الف}
i(t)&=i_j(t)+i_f(t)\\
&=\frac{V_I}{R}+K_2e^{-\frac{R}{L}t}\\
&=\frac{V_I}{R}+K_2e^{-\frac{t}{\tau}}
\end{aligned}
\end{gather}
جہاں وقتی مستقل درج ذیل ہے۔
\begin{align}
\tau=\frac{R}{L}
\end{align}
مکمل حل میں نا معلوم مستقل \عددی{K_2} کو ابتدائی معلومات سے حاصل کیا جا سکتا ہے۔سوئچ چالو کرنے سے پہلے دور میں رو صفر کے برابر ہے۔صفحہ \حوالہصفحہ{مساوات_امالہ_امالہ_گیر_رو_بلا_جوڑ_ہے} پر مساوات \حوالہ{مساوات_امالہ_امالہ_گیر_رو_بلا_جوڑ_ہے} کے تحت امالہ کی رو بلا جوڑ تفاعل
\begin{align*}
i_L(t_+)=i_L(t_-)
\end{align*} 
ہے لہٰذا سوئچ چالو کرنے کے فوراً بعد امالہ کی رو وہی ہو گی جو سوئچ چالو کرنے کے فوراً پہلے تھی یعنی لمحہ \عددی{t=0_+} پر \عددی{i_L(0_+)=i_L(0_-)=0} ہو گی۔ان معلومات کو مساوات \حوالہ{مساوات_عارضی_مزاحمت_امالہ_مکمل_حل_الف} میں دیے مکمل حل میں پُر کرنے سے
\begin{align*}
0&=\frac{V_I}{R}+K_2e^{-\frac{0}{\tau}}
\end{align*}
یعنی
\begin{align*}
K_2=-\frac{V_I}{R}
\end{align*}
حاصل ہوتا ہے۔یوں مکمل حل درج ذیل لکھا جائے گا۔
\begin{align}
i(t)=\frac{V_I}{R} \left(1-e^{-\frac{t}{\tau}}\right)
\end{align}
رو کے خط کو شکل \حوالہ{شکل_عارضی_مزاحمت_امالہ_الف}-ب میں دکھایا گیا ہے۔
\انتہا{مثال}
%=====================
\ابتدا{مثال}\شناخت{مثال_عارضی_دو_عدد_مزاحمت_برق_گیر_الف}
ازل سے شکل \حوالہ{شکل_عارضی_دو_عدد_مزاحمت_برق_گیر_الف} میں \اصطلاح{ایک قطب دو چال سوئچ}\فرہنگ{سوئچ!ایک قطب دو چال}\حاشیہب{single pole double throw switch, spdt}\فرہنگ{switch!spdt} اسی جگہ پر ہے۔لمحہ \عددی{t=0} پر اس کی جگہ تبدیل کرتے ہوئے \عددی{\SI{5}{\kilo\ohm}} مزاحمت کو زمین کے ساتھ جوڑا جاتا ہے۔برق گیر پر دباو دریافت کریں۔
\begin{figure}
\centering
\begin{tikzpicture}
\draw(0,0) to [american voltage source,l={$\SI{20}{\volt}$}]++(0,\y)coordinate(kP)++(\x,0) node[spdt,xscale=-1](kspdt){};
\draw[name path=kkk](kspdt.in) to [resistor,l={$\SI{5}{\kilo\ohm}$}]++(\x,0) to [short]++(\x,0) to [capacitor,l={$\SI{200}{\micro\farad}$}]++(0,-\y)coordinate(kbot)-|(0,0);
\draw(kP)|-(kspdt.out 1);
\draw(kbot)++(-\x,0)node[ground]{} to [resistor,*-*,l={$\SI{15}{\kilo\ohm}$}]++(0,\y);
\path[name path=kvert](kspdt.out 2)--++(0,-\y);
\draw[name intersections={of=kvert and kkk}] (kspdt.out 2) to [short,-*](intersection-1);
\draw(\x,-\y/4)node{(الف)};
\end{tikzpicture}%
\begin{tikzpicture}
\draw(0,0) to [american voltage source,l={$\SI{20}{\volt}$}]++(0,\y)coordinate(kP)++(\x,0) node[spdt,xscale=-1](kspdt){};
\draw[](kspdt.in) to [resistor,l={$\SI{5}{\kilo\ohm}$}]++(\x,0) to [short]++(\x,0) to [short,-o]++(0,-\y/8)++(0,-6/8*\y) to [short,o-]++(0,-\y/8)coordinate(kbot);
\draw[name path=kkk](0,0)--++(kbot);
\draw(kP)|-(kspdt.out 1);
\draw(kbot)++(-\x,0)node[ground]{} to [resistor,*-*,l={$\SI{15}{\kilo\ohm}$}]++(0,\y);
\path[name path=kvert](kspdt.out 2)--++(0,-\y);
\draw[name intersections={of=kvert and kkk}] (kspdt.out 2) to [short,-*](intersection-1);
\draw(kbot)++(0,\y/2)node{$\begin{aligned}&+ \\ &v_C \\ &-  \end{aligned}$};
\draw(\x,-\y/4)node{(ب)};
\end{tikzpicture}%
\begin{tikzpicture}
\draw(0,0) to [american voltage source,l={$\SI{20}{\volt}$}]++(0,\y)coordinate(kP)++(\x,0) node[spdt,xscale=-1,yscale=-1](kspdt){};
\draw[name path=kkk](kspdt.in) to [resistor,l={$\SI{5}{\kilo\ohm}$}]++(\x,0) to [short]++(\x,0)node[above]{$v_C(t)$} to [capacitor,l={$\SI{200}{\micro\farad}$}]++(0,-\y)coordinate(kbot)-|(0,0);
\draw(kP)|-(kspdt.out 2);
\draw(kbot)++(-\x,0)node[ground]{} to [resistor,*-*,l={$\SI{15}{\kilo\ohm}$}]++(0,\y);
\path[name path=kvert](kspdt.out 1)--++(0,-\y);
\draw[name intersections={of=kvert and kkk}] (kspdt.out 1) to [short,-*](intersection-1);
\draw(\x,-\y/4)node{(پ)};
\end{tikzpicture}%
\begin{tikzpicture}
\begin{axis}[axis lines*=middle,ytick={5,10},xlabel=$t$,ylabel=$v_C(t)$,
every axis x label/.style={
    at={(ticklabel* cs:1.05)},
    anchor=east,}, 
	every axis y label/.style={
    at={(ticklabel* cs:1.05)},
    anchor=east,}
]
\addplot[domain=0:6,samples=100]{15*e^(-4*x/3)};
\addplot[domain=-1:0,samples=10]{15}node[right]{$\SI{15}{\volt}$};
\end{axis}
\end{tikzpicture}
\caption{مثال \حوالہ{مثال_عارضی_دو_عدد_مزاحمت_برق_گیر_الف} کے اشکال۔}
\label{شکل_عارضی_دو_عدد_مزاحمت_برق_گیر_الف}
\end{figure}

حل:ازل سے دور منبع کے ساتھ جڑا رہا ہے۔یوں دور برقرار حالت میں ہو گا اور برق گیر کو کھلا دور تصور کیا جاتا ہے۔ایسا کرنے سے شکل-ب حاصل ہوتی ہے جہاں سے تقسیم دباو کے کلیے سے برق گیر کا ابتدائی دباو درج ذیل حاصل ہوتا ہے۔
\begin{align*}
v_C(0_-)=20\left(\frac{\SI{15}{\kilo\ohm}}{\SI{5}{\kilo\ohm}+\SI{15}{\kilo\ohm}}\right)=\SI{15}{\volt}
\end{align*}
برق گیر کا دباو بلا جوڑ ہے لہٰذا
\begin{align*}
v_C(0_+)=v_C(0_-)=\SI{15}{\volt} \quad \quad \text{\RL{ابتدائی حالت}}
\end{align*}
ہو گا۔لمحہ \عددیء{t=0} کے بعد کی صورت شکل-پ میں دکھائی گئی ہے۔ہمیں اس شکل میں \عددی{v(t)} درکار ہے جسے کرخوف مساوات رو کی مدد سے حاصل کرتے ہیں۔
\begin{align*}
\frac{v_C(t)}{5000}+\frac{v_C(t)}{15000}+200\times 10^{-6}\frac{\dif v_C(t)}{\dif t}=0
\end{align*}
اس ہم جنسی مساوات کو ترتیب دیتے ہوئے
\begin{align*}
\frac{\dif v_C(t)}{v_C(t)}=-\frac{4}{3}\dif t
\end{align*}
لکھا جا سکتا ہے جس کا تکمل
\begin{align*}
\ln v_C(t)=-\frac{4}{3}t+c
\end{align*}
یا
\begin{align*}
v_C(t)=Ke^{-\frac{4}{3}t}
\end{align*}
کے برابر ہے جہاں تکمل کے مستقل کو \عددی{c} یا \عددی{K} لکھا گیا ہے۔ابتدائی حالت کی معلومات اس مساوات میں پُر کرتے ہوئے 
\begin{align*}
15=Ke^{0}
\end{align*}
سے \عددی{K} کی قیمت درج ذیل
\begin{align*}
K=15
\end{align*}
حاصل ہوتی ہے۔یوں 
\begin{align*}
v_C(t)=15 e^{-\frac{4}{3}t}
\end{align*}
حاصل ہوتا ہے جس میں وقتی مستقل \عددی{\tau=\tfrac{3}{4}} کے برابر ہے۔یوں سوئچ چالو کرنے کے \عددی{\SI{0.75}{\second}} بعد برق گیر کا دباو ابتدائی قیمت کے \عددی{\SI{36.8}{\percent}} یعنی \عددی{0.368 \times 15=\SI{5.52}{\volt}} ہو گا۔ 
\انتہا{مثال}
%======================
\ابتدا{مثال}\شناخت{مثال_عارضی_دو_عدد_مزاحمت_امالہ_گیر_الف}
ازل سے شکل \حوالہ{شکل_عارضی_دو_عدد_مزاحمت_امالہ_گیر_الف} میں سوئچ غیر چالو تھا جسے \عددی{t=0} پر چالو کیا جاتا ہے۔امالہ گیر کی رو \عددی{i_L(t)} دریافت کریں۔

\begin{figure}
\centering
\begin{subfigure}{0.5\textwidth}
\centering
\begin{tikzpicture}
\draw(0,0) to [american voltage source,l={$\SI{24}{\volt}$}]++(0,\y) to [resistor,l={$\SI{1}{\kilo\ohm}$}]++(\x,0) to [cspst,l={${t=0}$}]++(\x,0) to [resistor,l={$\SI{1}{\kilo\ohm}$}]++(\x,0) to [inductor,i={$i_L(t)$},l={$\SI{1}{\milli\henry}$}]++(0,-\y) to [short] (0,0);
\draw(2*\x,0) to [american current source,*-*,l={$\SI{4}{\milli\ampere}$}]++(0,\y);
\end{tikzpicture}
\caption*{(الف)}
\end{subfigure}
\begin{subfigure}{0.5\textwidth}
\centering
\begin{tikzpicture}
\draw(0,0) to [american voltage source,l={$\SI{24}{\volt}$}]++(0,\y) to [resistor,l={$\SI{1}{\kilo\ohm}$}]++(\x,0) to [resistor,-o,l={$\SI{1}{\kilo\ohm}$}]++(\x,0);
\draw(0,0) to [short,-o]++(2*\x,0);
% to [inductor,i={$i_L(t)$},l={$\SI{1}{\milli\henry}$}]++(0,-\y) to [short] (0,0);
\draw(\x,0) to [american current source,*-*,l={$\SI{4}{\milli\ampere}$}]++(0,\y)node[above]{$v_t$};
\draw(2*\x,\y/2)node{$\begin{aligned} &+ \\ &v_{\text{تھونن}}\\ &- \end{aligned}$};
\end{tikzpicture}%
\caption*{(ب)}
\end{subfigure}%
\begin{subfigure}{0.5\textwidth}
\centering
\begin{tikzpicture}
\draw(0,0) to [short]++(0,\y) to [resistor,l={$\SI{1}{\kilo\ohm}$}]++(\x,0) to [resistor,-o,l={$\SI{1}{\kilo\ohm}$}]++(\x,0);
\draw(0,0) to [short,-o]++(2*\x,0);
% to [inductor,i={$i_L(t)$},l={$\SI{1}{\milli\henry}$}]++(0,-\y) to [short] (0,0);
%\draw(\x,0) to [american current source,*-*,l={$\SI{4}{\milli\ampere}$}]++(0,\y);
\draw[stealth-](2*\x,\y/2)--++(\x/4,0)--++(0,-\y/8)node[below]{$R_{\text{تھونن}}$};
\end{tikzpicture}%
\caption*{(پ)}
\end{subfigure}
\begin{subfigure}{0.5\textwidth}
\centering
\begin{tikzpicture}
\draw(0,0) to [american voltage source,l={$\SI{28}{\volt}$}]++(0,\y) to [resistor,l={$\SI{2}{\kilo\ohm}$}]++(\x,0) to [inductor,i={$i_L(t)$},l={$\SI{1}{\milli\henry}$}]++(0,-\y) --(0,0);
\end{tikzpicture}%
\caption*{(ت)}
\end{subfigure}%
\begin{subfigure}{0.5\textwidth}
\centering
\pgfplotsset{scaled y ticks=false, scaled x ticks=false}
\begin{tikzpicture}
\begin{axis}[axis lines*=middle,
ytick={0.014},yticklabels={$\SI{14}{\milli\ampere}$},xlabel=$t(\si{\micro\second})$,ylabel=$i_L(t)$,
xtick={-0.000001,0.000001,0.000002,0.000003,0.000004,0.000005}, xticklabels={$-1$,$1$,$2$,$3$,$4$,$5$},
every axis x label/.style={
    at={(ticklabel* cs:1.15)},
    anchor=east,}, 
	every axis y label/.style={
    at={(ticklabel* cs:1.05)},
    anchor=east,}
]
\addplot[domain=0:3*10^(-6),samples=20]{0.014-0.01*e^(-2000000*x)};
\addplot[domain=-1*10^(-6):0,samples=10]{0.004}node[right]{$\SI{4}{\milli\ampere}$};
\end{axis}
\end{tikzpicture}%
\caption*{(ٹ)}
\end{subfigure}%
\caption{مثال \حوالہ{مثال_عارضی_دو_عدد_مزاحمت_امالہ_گیر_الف} کے اشکال۔}
\label{شکل_عارضی_دو_عدد_مزاحمت_امالہ_گیر_الف}
\end{figure}

حل:غیر چالو سوئچ کی صورت میں منبع رو کی تمام رو امالہ گیر سے گزرتی ہے لہٰذا
\begin{align*}
i_L(0_-)=i_L(0_+)=\SI{4}{\milli\ampere}
\end{align*}
ہو گا۔اس دور کو مسئلہ تھونن کی مدد سے حل کرتے ہیں۔یوں امالہ کو بوجھ تصور کرتے ہوئے بقایا دور کا تھونن مساوی حاصل کرتے ہیں۔تھونن دباو حاصل کرنے کی خاطر بوجھ کو کھلے دور کیا جاتا ہے جس سے شکل \حوالہ{شکل_عارضی_دو_عدد_مزاحمت_امالہ_گیر_الف}-ب حاصل ہوتی ہے۔اس شکل میں منبع رو کی تمام رو بائیں مزاحمت اور منبع دباو سے گزرے گی لہٰذا مزاحمت پر \عددی{\SI{4}{\volt}} کا دباو ہو گا۔یوں
\begin{align*}
v_t=v_{\text{تھونن}}=\SI{24}{\volt}+\SI{4}{\volt}=\SI{28}{\volt}
\end{align*}
لکھا جا سکتا ہے۔یاد رہے کہ بالائی دائیں مزاحمت میں رو صفر کے برابر ہے لہٰذا اس پر دباو بھی صفر ہو گا اور یوں \عددی{v_t} اور \عددی{v_{\text{تھونن}}} برابر ہوں گے۔

منبع دباو کو قصر دور اور منبع رو کو کھلے دور کرتے ہوئے شکل-پ حاصل ہوتی ہے جسے دیکھتے ہوئے تھونن مزاحمت
\begin{align*}
R_{\text{تھونن}}=\SI{2}{\kilo\ohm}
\end{align*}
لکھی جا سکتی ہے۔

تھونن مساوی دور استعمال کرتے ہوئے شکل-الف کو شکل-ت کی طرز پر بنایا جا سکتا ہے۔شکل-ت کی کرخوف مساوات
\begin{align*}
28=2000 i(t)+0.001 \frac{\dif i(t)}{\dif t}
\end{align*}
کو عمومی صورت میں لکھتے ہیں۔
\begin{align*}
\frac{\dif i(t)}{\dif t}+2\times 10^6 i(t)=28000
\end{align*}
اس مساوات کا جبری حل 
\begin{align*}
i_j(t)=K_1=\SI{14}{\milli\ampere}
\end{align*}
حاصل ہوتا ہے اور اس کا فطری حل
\begin{align*}
i_f(t)=K_2 e^{-2\times 10^6 t}
\end{align*}
ہے-یوں امالہ گیر کے رو کا مکمل حل
\begin{align*}
i(t)=0.014+K_2 e^{-2\times 10^6 t}
\end{align*}
ہے۔ابتدائی معلومات کو اس مساوات میں حل کرتے ہوئے
\begin{align*}
0.004=0.014+K_2 e^{0}
\end{align*}
سے
\begin{align*}
K_2=\SI{-10}{\milli\ampere}
\end{align*}
حاصل ہوتا ہے۔یوں مکمل حل درج ذیل ہے۔
\begin{align}\label{مساوات_عارضی_امالہ_گیر_مکمل_حل_ب}
i_L(t)=0.014-0.01e^{-2\times 10^6 t}
\end{align} 
اس مساوات کا وقتی مستقل \عددی{\tau=\SI{0.5}{\micro\second}} ہے۔یوں تقریباً \عددی{5\tau=\SI{2.5}{\micro\second}} میں دور پہلی برقرار حالت سے دوسری برقرار حالت اختیار کر پاتا ہے۔مساوات \حوالہ{مساوات_عارضی_امالہ_گیر_مکمل_حل_ب} کو شکل-ٹ میں دکھایا گیا ہے۔
\انتہا{مثال}
%=======================
\ابتدا{مشق}\شناخت{مشق_عارضی_برق_گیر_الف}
شکل \حوالہ{شکل_عارضی_مشق_برق_گیر_الف} میں ازل سے چالو سوئچ کو  لمحہ \عددی{t=0} پر منقطع کیا جاتا ہے۔برق گیر پر ابتدائی دباو دریافت کرتے ہوئے \عددی{v_0(t)} دریافت کریں۔ اس دور کا وقتی مستقل کیا ہے۔
\begin{figure}
\centering
\begin{tikzpicture}
\draw(0,0) to [american voltage source,l={$\SI{24}{\volt}$}]++(0,\y) to [resistor,l={$\SI{2}{\kilo\ohm}$}]++(\x,0) to [ospst,l={${t=0}$}] ++(\x,0) to [resistor,l={$\SI{6}{\kilo\ohm}$}]++(\x,0) to [resistor,l_={$\SI{4}{\kilo\ohm}$}]++(0,-\y) to [short] (0,0);
\draw(2*\x,0) to [capacitor,*-*,l={$\SI{50}{\micro\farad}$}]++(0,\y);
\draw(3*\x+\dx,\y/2)node[right]{$\begin{aligned}&+ \\ &v_0(t) \\ &- \end{aligned}$};
\end{tikzpicture}
\caption{مشق \حوالہ{مشق_عارضی_برق_گیر_الف} کا دور۔}
\label{شکل_عارضی_مشق_برق_گیر_الف}
\end{figure}

جوابات:\عددی{v_C(0_+)=\SI{20}{\volt}}، \عددی{v_0(t)=8 e^{-\frac{t}{0.5}} \, \si{\volt}}، \عددی{\tau=\SI{0.5}{\second}}
\انتہا{مشق}
%===================

\ابتدا{مشق}\شناخت{مشق_عارضی_برق_گیر_ب}
شکل \حوالہ{شکل_عارضی_مشق_برق_گیر_ب} میں ازل سے چالو سوئچ کو  لمحہ \عددی{t=0} پر منقطع کیا جاتا ہے۔برق گیر پر ابتدائی دباو دریافت کرتے ہوئے \عددی{v_0(t)} دریافت کریں۔
\begin{figure}
\centering
\begin{tikzpicture}
\draw(0,0) to [american voltage source,l={$\SI{12}{\volt}$}]++(0,\y) to [resistor,l={$\SI{2}{\kilo\ohm}$}]++(\x,0) to [ospst,l={${t=0}$}] ++(\x,0) to [short]++(2*\x,0) to [capacitor,l_={$\SI{10}{\micro\farad}$}]++(0,-\y) to [short] (0,0);
\draw(\x,0) to [resistor,*-*,l={$\SI{4}{\kilo\ohm}$}]++(0,\y);
\draw(3*\x,0) to [resistor,*-*,l={$\SI{8}{\kilo\ohm}$}]++(0,\y);
\draw(2*\x,0) to [american current source,*-*,l={$\SI{4}{\milli\ampere}$}]++(0,\y);
\draw(4*\x+2*\dx,\y/2)node[right]{$\begin{aligned}&+ \\ &v_0(t) \\ &- \end{aligned}$};
\end{tikzpicture}
\caption{مشق \حوالہ{مشق_عارضی_برق_گیر_ب} کا دور۔}
\label{شکل_عارضی_مشق_برق_گیر_ب}
\end{figure}

جوابات:\عددی{v_0(0_+)=\frac{80}{7} \, \si{\volt}}، \عددی{v_0(t)=32-\frac{144}{7}e^{-\frac{100t}{7}} \, \si{\volt}}
\انتہا{مشق}
%===================


\ابتدا{مشق}\شناخت{مشق_عارضی_برق_گیر_پ}
شکل \حوالہ{شکل_عارضی_مشق_برق_گیر_پ} میں ازل سے چالو سوئچ کو  لمحہ \عددی{t=0} پر منقطع کیا جاتا ہے۔امالہ گیر میں ابتدائی رو دریافت کرتے ہوئے \عددی{i_L(t)} دریافت کریں۔دور کا وقتی مستقل  حاصل کریں۔
\begin{figure}
\centering
\begin{tikzpicture}
\draw(0,0) to [american voltage source,l={$\SI{10}{\volt}$}]++(0,2*\y) to [ospst,l={${t=\SI{0}{\second}}$}]++(\x,0) to [short] ++(\x,0) to [short]++(\x,0) to [resistor,l_={$\SI{10}{\ohm}$}]++(0,-2*\y) to [short] (0,0);
\draw(\x,0) to [resistor,*-,l={$\SI{8}{\ohm}$}]++(0,\y) to [inductor,i<_={$i_L$},-*,l={$\SI{4}{\henry}$}]++(0,\y);
\end{tikzpicture}
\caption{مشق \حوالہ{مشق_عارضی_برق_گیر_پ} کا دور۔}
\label{شکل_عارضی_مشق_برق_گیر_پ}
\end{figure}

جوابات:\عددی{i_L(0_+)=\SI{1.25}{\ampere}}، \عددی{i_L(t)=1.25e^{-3000t} \, \si{\ampere}}، \عددی{\tau=\tfrac{1}{3} \, \si{\milli\second}}
\انتہا{مشق}
%===================
\ابتدا{مثال}\شناخت{مثال_عارضی_برق_گیر_وقت_صفر_نہیں_الف}
شکل \حوالہ{شکل_عارضی_برق_گیر_وقت_صفر_نہیں_الف} میں ازل سے چالو سوئچ لمحہ \عددی{t=\SI{2}{\second}} پر منقطع کیا جاتا ہے۔رو \عددی{i(t)} دریافت کریں۔

\begin{figure}
\centering
\begin{subfigure}{0.5\textwidth}
\centering
\begin{tikzpicture}
\draw(0,0) to [american voltage source,l={$\SI{10}{\volt}$}]++(0,\y) to [resistor,l={$\SI{2}{\kilo\ohm}$}]++(\x,0) to [resistor,i<^={$i(t)$},l={$\SI{4}{\kilo\ohm}$}]++(\x,0) to [resistor,l={$\SI{6}{\kilo\ohm}$}]++(\x,0);
\draw(0,0) to [short]++(3*\x,0) to [american voltage source,l_={$\SI{20}{\volt}$}]++(0,\y);
\draw(\x,0) to [ospst,*-*,l={${t=\SI{2}{\second}}$}]++(0,\y);
\draw(2*\x,0) node[ground]{}to [capacitor,*-*,l={$\SI{5}{\micro\farad}$}]++(0,\y)node[above]{$v(t)$};
\end{tikzpicture}
\caption*{(الف)}
\end{subfigure}
\begin{subfigure}{0.5\textwidth}
\centering
\begin{tikzpicture}
\draw(0,0) to [american voltage source,l={$\SI{10}{\volt}$}]++(0,\y) to [resistor,l={$\SI{2}{\kilo\ohm}$}]++(\x,0) to [resistor,i<^={$i(t)$},l={$\SI{4}{\kilo\ohm}$}]++(\x,0) to [resistor,l={$\SI{6}{\kilo\ohm}$}]++(\x,0);
\draw(0,0) to [short]++(3*\x,0) to [american voltage source,l_={$\SI{20}{\volt}$}]++(0,\y);
\draw(\x,0) to [short,*-*]++(0,\y);
\draw(2*\x,0)node[ground]{} to [short,*-o]++(0,\y/8);
\draw(2*\x,\y) to [short,*-o]++(0,-\y/8);
\draw(2*\x+2*\dx,\y/2)node[]{$\begin{aligned} &+ \\ &v_C(2_-) \\ &- \end{aligned}$};
\end{tikzpicture}
\caption*{(ب)}
\end{subfigure}
\begin{subfigure}{0.5\textwidth}
\pgfplotsset{scaled y ticks=false, scaled x ticks=false}
\begin{tikzpicture}
\begin{axis}[axis lines*=middle,xlabel={$t$},ylabel={$i(t)$},ytick={0.002,-0.00033,0.000833},yticklabels={$\SI{2}{\milli\ampere}$,$\SI{-0.33}{\milli\ampere}$,${\frac{5}{6}\,\si{\milli\ampere}}$},
every axis x label/.style={
    at={(ticklabel* cs:1.15)},
    anchor=east,}, 
	every axis y label/.style={
    at={(ticklabel* cs:1.05)},
    anchor=east,}
]
\addplot[domain=1.99:2,samples=10]{0.002};
\addplot[domain=2:2.075,samples=100]{5/6000-7/6000*e^(200/3*(2-x))};
\draw(axis cs:2,0.002)--(axis cs:2,-1/3000);
\end{axis}
\end{tikzpicture}
\caption*{(پ)}
\end{subfigure}
\caption{مثال \حوالہ{مثال_عارضی_برق_گیر_وقت_صفر_نہیں_الف} کے اشکال۔}
\label{شکل_عارضی_برق_گیر_وقت_صفر_نہیں_الف}
\end{figure}

حل:سوئچ منقطع کرنے سے فوراً پہلے کی صورت حال شکل-ب میں دکھائی گئی ہے۔چونکہ ازل سے سوئچ چالو تھا لہٰذا دور برقرار حالت میں ہو گا اور یوں برق گیر کو کھلا دور تصور کیا جائے گا۔شکل-ب کو دیکھ کر
\begin{align*}
i(t<\SI{2}{\second})=\frac{20}{4000+6000}=\SI{2}{\milli\ampere}
\end{align*}
اور
\begin{align*}
v_C(2_-)=v_C(2_+)=20\left(\frac{4000}{4000+6000}\right)=\SI{8}{\volt}
\end{align*}
لکھا جا سکتا ہے۔سوئچ منقطع ہونے کے بعد کی صورت حال شکل-الف میں دی گئی ہے۔جوڑ \عددی{v(t)} پر کرخوف مساوات رو لکھتے ہوئے
\begin{align*}
\frac{v(t)-10}{2000+4000}+5\times 10^{-6}\frac{\dif v(t)}{\dif t}+\frac{v(t)-20}{6000}=0
\end{align*}
ترتیب دینے سے
\begin{align*}
\frac{\dif v(t)}{\dif t}+\frac{200}{3} v(t)=1000
\end{align*}
حاصل ہوتا ہے۔اس کے جبری اور فطری  حل درج ذیل ہیں
\begin{align*}
v_j(t)&=K_1=\SI{15}{\volt}\\
v_f(t)&=K_2e^{-\frac{200}{3}t}
\end{align*}
جن کا مجموعہ مکمل حل
\begin{align*}
v(t>2)=15+K_2e^{-\frac{200}{3}t}
\end{align*}
 دیتا ہے۔ابتدائی معلومات  \عددی{v(2_+)=\SI{8}{\volt}}لمحہ \عددی{t=\SI{2}{\second}} پر ہم جانتے ہیں جنہیں درج بالا مساوات میں پُر کرتے ہوئے
\begin{align*}
8=15+K_2e^{-\frac{200}{3}\times 2}
\end{align*}
\عددی{K_2} کی قیمت درج ذیل حاصل ہوتی ہے۔
\begin{align*}
K_2=-7e^{\frac{400}{3}}
\end{align*}
یوں مکمل حل درج ذیل ہو گا۔
\begin{align*}
v(t>2)=15-7e^{\frac{200}{3}(2-t)}
\end{align*}
اب شکل-الف کو دیکھ کر
\begin{align*}
i(t>2)&=\frac{v(t>2)-10}{6000}\\
&=\frac{5}{6}-\frac{7}{6} e^{\frac{200}{3}(2-t)} \, \si{\milli\ampere}
\end{align*}
لکھا جا سکتا ہے جو درکار مساوات ہے۔یوں سوئچ منقطع کرنے سے پہلے اور اس کے بعد کے جوابات سے درج ذیل لکھا جا سکتا ہے
\begin{align*}
i(t)=
\begin{cases}
\SI{2}{\milli\ampere} & t<\SI{2}{\second}\\
\frac{5}{6}-\frac{7}{6} e^{\frac{200}{3}(2-t)} \, \si{\milli\ampere} & t>\SI{2}{\second}
\end{cases}
\end{align*}
جسے شکل-پ میں دکھایا گیا ہے جہاں سے آپ دیکھ سکتے ہیں کہ سوئچ منقطع کرنے سے پہلے برقرار رو \عددی{\SI{2}{\milli\ampere}} تھی جبکہ سوئچ منقطع کرنے کے بعد برقرار حالت \عددی{(t \to \infty)} میں رو \عددی{\tfrac{5}{6}\,\si{\milli\ampere}} ہے۔یاد رہے کہ برق گیر کا دباو فوراً تبدیل نہیں ہو سکتا البتہ اس میں رو یک دم تبدیل ہو سکتی ہے۔

وقت \عددی{t \to \infty} پر دور برقرار حالت اختیار کر چکا ہو گا لہٰذا برق گیر کو کھلا دور کرتے ہوئے شکل \حوالہ{شکل_عارضی_برق_گیر_وقت_صفر_نہیں_الف}-الف سے برقرار حالت  رو درج ذیل لکھی جا سکتی ہے۔
\begin{align*}
i(t\to\infty)=\frac{20-10}{2000+4000+6000}=\frac{5}{6} \, \si{\milli\ampere}
\end{align*}


\انتہا{مثال}
%=======================
\ابتدا{مثال}\شناخت{مثال_عارضی_امالہ_گیر_سوئچ_چالو}
شکل \حوالہ{شکل_عارضی_امالہ_گیر_سوئچ_چالو_ب}-الف میں ازل سے منقطع سوئچ لمحہ \عددی{t=\SI{7}{\second}} پر چالو کیا جاتا ہے۔رو \عددی{i(t)} دریافت کریں۔ 
\begin{figure}
\centering
\begin{subfigure}{0.5\textwidth}
\centering
\begin{tikzpicture}
\draw(0,0) to [american current source,l={$\SI{6}{\ampere}$}]++(0,\y) to [resistor,l={$\SI{1}{\ohm}$}]++(\x,0) to [inductor,i={$i_L(t)$},l={$\SI{5}{\henry}$}]++(\x,0) to [short]++(\x,0) to [resistor,l={$\SI{2}{\ohm}$}]++(0,-\y) to [short] (0,0);
\draw(\x,0) to [resistor,i<_={$i(t)$},*-*,l={$\SI{4}{\ohm}$}]++(0,\y);
\draw(2*\x,0) to [cspst,*-*,l_={${t=\SI{7}{\second}}$}]++(0,\y);
\end{tikzpicture}%
\caption*{(الف)}
\end{subfigure}
\begin{subfigure}{0.5\textwidth}
\centering
\begin{tikzpicture}
\draw(0,0) to [american current source,l={$\SI{6}{\ampere}$}]++(0,\y) to [resistor,l={$\SI{1}{\ohm}$}]++(\x,0) to [short,i={$i_L(t)$}]++(\x,0)to [resistor,l={$\SI{2}{\ohm}$}]++(0,-\y) to [short] (0,0);
\draw(\x,0) to [resistor,i<_={$i(t)$},*-*,l={$\SI{4}{\ohm}$}]++(0,\y);
\end{tikzpicture}%
\caption*{(ب)}
\end{subfigure}%
\begin{subfigure}{0.5\textwidth}
\centering
\begin{tikzpicture}
\draw(0,0) to [american current source,l={$\SI{6}{\ampere}$}]++(0,\y) to [resistor,l={$\SI{1}{\ohm}$}]++(\x,0) to [inductor,l={$\SI{5}{\henry}$}]++(\x,0)to [short]++(0,-\y) to [short] (0,0);
\draw(\x,0) to [resistor,i<_={$i(t)$},*-*]++(0,\y);
\draw(\x,3/4*\y)node[left]{$\SI{4}{\ohm}$};
%loop currents
\draw[stealth-]([shift={(-150:\x/6)}]\x/2,\y/2) arc (-150:150:\x/6);
\draw(\x/2,\y/2)node{$i_1$};
\draw[stealth-]([shift={(-150:\x/6)}]\x+\x/2,\y/2) arc (-150:150:\x/6);
\draw(\x+\x/2,\y/2)node{$i_2$};
\end{tikzpicture}
\caption*{(پ)}
\end{subfigure}
\begin{subfigure}{0.5\textwidth}
\centering
\pgfplotsset{scaled y ticks=false, scaled x ticks=false}
\begin{tikzpicture}
\begin{axis}[axis lines*=middle,
xlabel={$t$},
ylabel={$i(t)$},
every axis x label/.style={
    at={(ticklabel* cs:1.15)},
    anchor=east,}, 
	every axis y label/.style={
    at={(ticklabel* cs:1.05)},
    anchor=east,}
]
\addplot[domain=6:7,samples=10]{2};
\addplot[domain=7:13,samples=100]{2*e^(4/5*(7-x))};
\end{axis}
\end{tikzpicture}%
\caption*{(ت)}
\end{subfigure}%
\caption{مثال \حوالہ{مثال_عارضی_امالہ_گیر_سوئچ_چالو} کا اشکال۔}
\label{شکل_عارضی_امالہ_گیر_سوئچ_چالو_ب}
\end{figure}

حل:منقطع سوئچ کی صورت میں دور برقرار حالت میں ہو گا لہٰذا امالہ گیر کو قصر دور تصور کرتے ہوئے شکل-ب حاصل کی گئی ہے۔تقسیم رو کے کلیے سے
\begin{align*}
i_L(7_-)=i_L(7_+)=6\left(\frac{4}{4+2}\right)=\SI{4}{\ampere}
\end{align*}
اور
\begin{align}\label{مساوات_عارضی_امالہ_گیر_سات_سیکنڈ_الف}
i(t)=\SI{6}{\ampere}-i_L(t)=6-4=\SI{2}{\ampere} \quad \quad (t<\SI{7}{\second})
\end{align}
لکھا جا سکتا ہے۔سوئچ چالو کرنے کے بعد کی صورت حال شکل-پ میں دکھائی گئی ہے جہاں سے درج ذیل لکھا جا سکتا ہے۔
\begin{align*}
i_1&=\SI{6}{\ampere}\\
5\frac{\dif i_2}{\dif t}+4(i_2-i_1)&=0
\end{align*}
ان مساوات کو ملاتے ہوئے
\begin{align*}
5\frac{\dif i_2}{\dif t}+4(i_2-6)&=0
\end{align*}
یعنی
\begin{align*}
\frac{\dif i_2}{\dif t}+\frac{4}{5} i_2=\frac{24}{5}
\end{align*}
حاصل ہوتا ہے جس کا مکمل حل درج ذیل ہے۔
\begin{align*}
i_2=6+K_2e^{-\frac{4}{5}t}
\end{align*}
چونکہ \عددی{i_2} درحقیقت \عددی{i_L} ہی ہے لہٰذا نا معلوم مستقل \عددی{K_2} کو ابتدائی معلومات سے حاصل کرتے ہیں۔درج بالا مساوات میں  \عددی{t=\SI{7}{\second}} پر \عددی{i_L(7_+)=\SI{4}{\ampere}} پُر کرتے ہوئے 
\begin{align*}
4=6+K_2e^{-\frac{4}{5}\times 7}
\end{align*} 
سے
\begin{align*}
K_2=-2e^{\frac{4}{5}\times 7}
\end{align*}
حاصل ہوتا ہے۔یوں سوئچ چالو کرنے کے بعد \عددی{i_2} کا مکمل حل درج ذیل لکھا جائے گا۔
\begin{align*}
i_2=6-2e^{\frac{4}{5}(7-t)}
\end{align*}
اب شکل-پ کو دیکھتے ہوئے
\begin{align*}
i(t)&=i_1-i_2\\
&=6-\left(6-2^{\frac{4}{5}(7-t)}\right)\\
&=2e^{\frac{4}{5}(7-t)}\quad \quad (t>\SI{7}{\second})
\end{align*}
لکھا جا سکتا ہے۔یوں ازل سے ابد تک \عددی{i(t)} کو مساوات \حوالہ{مساوات_عارضی_امالہ_گیر_سات_سیکنڈ_الف} اور درج بالا مساوات  پیش کرتے ہیں۔انہیں اکٹھے لکھتے  اور شکل-ت میں پیش کرتے ہیں۔
\begin{align}
i(t)=
\begin{cases}
\SI{2}{\ampere} & t<\SI{7}{\second}\\
2e^{\frac{4}{5}(7-t)} \, \si{\ampere} & t>\SI{7}{\second}
\end{cases}
\end{align}
\انتہا{مثال}
%========================
\ابتدا{مشق}\شناخت{مشق_عارضی_امالہ_دو_منبع_الف}
شکل \حوالہ{شکل_عارضی_امالہ_دو_منبع_الف} میں ابتدائی حالت \عددی{i_L(0_+)} دریافت کریں۔دائرہ \عددی{abcfa} میں \عددی{i_1} اور \عددی{abdea} میں \عددی{i_2} لیتے ہوئے  کرخوف مساوات دباو لکھیں۔ان مساوات  سے صرف \عددی{i_1} پر مبنی مساوات حاصل کریں۔یوں ازل سے ابد تک \عددی{i_L} دریافت کریں۔
\begin{figure}
\centering
\begin{tikzpicture}
\draw(0,0)node[left]{$a$} to [american voltage source,l={$\SI{12}{\volt}$}]++(0,2*\y)node[left]{$b$} to [resistor,l={$\SI{2}{\ohm}$}]++(\x,0) node[above]{$c$} to [resistor,l={$\SI{2}{\ohm}$}] ++(\x,0)to [short]++(\x,0)node[right]{$d$} to [resistor,l_={$\SI{4}{\ohm}$}]++(0,-2*\y)node[right]{$e$} to [short] (0,0);
\draw(\x,0)node[below]{$f$} to [inductor,i<_={$i_L$},*-,l={$\SI{2}{\henry}$}]++(0,\y) to [resistor,-*,l={$\SI{3}{\ohm}$}]++(0,\y);
\draw(2*\x,0) to [american voltage source,*-,l={$\SI{16}{\volt}$}]++(0,\y) to [ospst,-*,l={${t=\SI{0}{\second}}$}]++(0,\y);
%\draw(3*\x+\dx,\y)node[right]{$\begin{aligned} &+ \\ \\ &v_0(t)  \\  \\ &- \end{aligned}$};
\end{tikzpicture}
\caption{مشق \حوالہ{مشق_عارضی_امالہ_دو_منبع_الف} کا دور۔}
\label{شکل_عارضی_امالہ_دو_منبع_الف}
\end{figure}

جوابات:\عددی{i_L(0_+)=\SI{3.5}{\ampere}}، \عددی{\tfrac{\dif i_1}{\dif t}+2.25i_1=4.5}، \عددی{i_L(t>0)=2+1.5e^{-2.25t}\,\si{\ampere}}
\انتہا{مشق}
%=======================
\ابتدا{مشق}\شناخت{مشق_عارضی_برق_گیر_دو_منبع_الف}
شکل \حوالہ{شکل_عارضی_برق_گیر_دو_منبع_الف} میں \عددی{v_0(t)} حاصل کریں۔
\begin{figure}
\centering
\begin{tikzpicture}
\draw(0,0) to [american voltage source,l={$\SI{12}{\volt}$}]++(0,2*\y) to [resistor,l={$\SI{1}{\ohm}$}]++(\x,0) to [resistor,l={$\SI{1}{\ohm}$}] ++(\x,0) to [resistor,l={$\SI{2}{\ohm}$}] ++(\x,0);
\draw(0,0) to [short]++(3*\x,0) to [ospst,l_={${t=0}$}] ++(0,\y) to [american voltage source,l_={$\SI{8}{\volt}$}]++(0,\y);
\draw(\x,0) to [capacitor,*-*,l={$\SI{2}{\farad}$}]++(0,2*\y);
\draw(2*\x,0) to [resistor,*-*,l={$\SI{2}{\ohm}$}]++(0,2*\y);
\draw(2*\x+\dx,\y)node[right]{$\begin{aligned} &+ \\ \\ &v_0(t)  \\  \\ &- \end{aligned}$};
\end{tikzpicture}
\caption{مشق \حوالہ{مشق_عارضی_برق_گیر_دو_منبع_الف} کا دور۔}
\label{شکل_عارضی_برق_گیر_دو_منبع_الف}
\end{figure}

جوابات:\عددی{v_0(t)=\tfrac{24}{5}+\tfrac{1}{5}e^{-\tfrac{5}{8}t} \, \si{\volt}}
\انتہا{مشق}
%=======================
\ابتدا{مشق}\شناخت{مشق_عارضی_برق_گیر_دو_منبع_ب}
شکل \حوالہ{شکل_عارضی_برق_گیر_دو_منبع_ب} میں سوئچ منقطع کرنے کے بعد \عددی{v_0} حاصل کریں۔
\begin{figure}
\centering
\begin{tikzpicture}
\draw(0,2*\y) to [american voltage source,l={$\SI{18}{\volt}$}]++(0,-2*\y);
\draw(0,2*\y) to [resistor,l={$\SI{4}{\ohm}$}]++(\x,0) to [resistor,l={$\SI{4}{\ohm}$}] ++(\x,0);
\draw(0,0) to [short]++(2*\x,0) to [american voltage source,l={$\SI{6}{\volt}$}]++(0,\y) to [ospst,l_={${t=0}$}] ++(0,\y);
\draw(\x,0) to [inductor,*-,l={$\SI{6}{\henry}$}]++(0,\y) to [resistor,-*,l={$\SI{8}{\ohm}$}]++(0,\y);
\draw(\x+\dx,\y+\y/2)node[right]{$\begin{aligned} &+ \\ &v_0  \\  &- \end{aligned}$};
\end{tikzpicture}
\caption{مشق \حوالہ{مشق_عارضی_برق_گیر_دو_منبع_ب} کا دور۔}
\label{شکل_عارضی_برق_گیر_دو_منبع_ب}
\end{figure}

جوابات:\عددی{v_0=-12+\tfrac{9}{2}e^{-2t} \, \si{\volt}}
\انتہا{مشق}
%=======================

\حصہ{دھڑکن}
گزشتہ حصے میں سوئچ کو چالو یا منقطع کرتے ہوئے ادوار میں یکدم تبدیلی پیدا کی گئی۔فوراً تبدیلی پیدا کرنے والے دو عدد تفاعل نہایت اہم ہیں۔انہیں \اصطلاح{اکائی سیڑھی تفاعل}\فرہنگ{اکائی سیڑھی تفاعل}\حاشیہب{unit step function}\فرہنگ{unit step function} اور \اصطلاح{اکائی جھٹکا تفاعل}\فرہنگ{اکائی جھٹکا تفاعل}\حاشیہب{unit impulse function}\فرہنگ{unit impulse function} کہتے ہیں۔آئیں اکائی سیڑھی تفاعل پر غور کریں۔

\اصطلاح{اکائی سیڑھی تفاعل} \عددی{u(t)} کی الجبرائی تعریف درج ذیل ہے۔
\begin{align}
u(t)=
\begin{cases}
0 & t<0\\
1 & t>0
\end{cases}
\end{align}
یوں یہ تفاعل \اصطلاح{بے بعد}\فرہنگ{بے بعد}\حاشیہب{dimensionless}\فرہنگ{dimensionless} ہے  جو منفی\عددی{t} کی صورت میں صفر کے برابر جبکہ مثبت  \عددی{t} کی صورت میں اکائی کے برابر ہے۔شکل \حوالہ{شکل_عارضی_اکائی_سیڑھی_تفاعل_الف}-الف میں اکائی سیڑھی تفاعل کو دکھایا گیا ہے۔اکائی سیڑھی تفاعل کے متغیرہ کو \عددی{t-t_0} لکھتے ہوئے شکل \حوالہ{شکل_عارضی_اکائی_سیڑھی_تفاعل_الف}-ب حاصل ہوتا ہے جو افقی محدد پر \عددی{t_0} دائیں منتقل اکائی سیڑھی تفاعل \عددی{u(t-t_0)} ہے۔یہ تفاعل منفی \عددی{t-t_0} کی صورت میں صفر کے برابر ہے جبکہ مثبت \عددی{t-t_0} کی صورت میں یہ اکائی کے برابر ہے۔اکائی سیڑھی تفاعل کو \عددی{A} سے ضرب دینے سے \عددی{A} گنا اونچی سیڑھی حاصل ہو گی۔شکل \حوالہ{شکل_عارضی_اکائی_سیڑھی_تفاعل_الف}-پ میں مثبت \عددی{A} کی صورت میں \عددی{Au(t)} اور شکل \حوالہ{شکل_عارضی_اکائی_سیڑھی_تفاعل_الف}-ت میں \عددی{-Au(t-t_0)} دکھائے گئے ہیں۔ 
\begin{figure}
\centering
\begin{subfigure}{0.5\textwidth}
\centering
\begin{tikzpicture}
\draw[gray](0,-0.5)--(0,2)node[left]{$u(t)$};
\draw[gray](-0.5,0)--(3,0)node[right]{$t$};
\draw(-0.25,0)--(0,0)--(0,1)node[left]{$1$}--(3,1);
\end{tikzpicture}%
\caption*{(الف)}
\end{subfigure}%
\begin{subfigure}{0.5\textwidth}
\centering
\begin{tikzpicture}
\draw[gray](0,-0.5)--(0,2)node[left]{$u(t-t_0)$};
\draw[gray](-0.5,0)--(3,0)node[right]{$t$};
\draw(-0.25,0)--(1,0)node[below]{$t_0$}--(1,1)--(3,1);
\draw[gray,dashed](1,1)--(0,1)node[left,black]{$1$};
\end{tikzpicture}%
\caption*{(ب)}
\end{subfigure}
\begin{subfigure}{0.5\textwidth}
\centering
\begin{tikzpicture}
\draw[gray](0,-1.5)--(0,1.5)node[left]{$Au(t)$};
\draw[gray](-0.5,0)--(3,0)node[right]{$t$};
\draw(-0.25,0)--(0,0)--(0,1)node[left]{$A$}--(3,1);
\end{tikzpicture}%
\caption*{(پ)}
\end{subfigure}%
\begin{subfigure}{0.5\textwidth}
\centering
\begin{tikzpicture}
\draw[gray](0,-1.5)--(0,1.5)node[left]{$-Au(t-t_0)$};
\draw[gray](-0.5,0)--(3,0)node[right]{$t$};
\draw(-0.25,0)--(1,0)node[above]{$t_0$}--(1,-1)--(3,-1);
\draw[gray,dashed](1,-1)--(0,-1)node[left,black]{$-A$};
\end{tikzpicture}%
\caption*{(ت)}
\end{subfigure}%
\caption{اکائی سیڑھی تفاعل۔}
\label{شکل_عارضی_اکائی_سیڑھی_تفاعل_الف}
\end{figure}

اکائی سیڑھی تفاعل سے مستطیل تفاعل حاصل کیا جا سکتا ہے۔یہ عمل شکل \حوالہ{شکل_عارضی_اکائی_سیڑھی_تفاعل_ب} میں دکھایا گیا ہے جہاں \عددی{Au(t)} اور
 \عددی{-Au(t-t_0)} کا مجموعہ
\begin{align}
v(t)=Au(t)-Au(t-t_0)
\end{align}
 لیتے ہوئے \عددی{A} حیطے کا مستطیل تفاعل حاصل کیا گیا ہے۔ 
\begin{figure}
\centering
\begin{subfigure}{0.5\textwidth}
\centering
\begin{tikzpicture}
\draw[gray](0,-0.5)--++(0,2)node[left]{$Au(t)$};
\draw[gray](-0.5,0)--++(3.5,0)node[right]{$t$};
\draw(-0.25,0)--++(0.25,0)--++(0,1)node[left]{$A$}--++(3,0);
%
\pgfmathsetmacro{\ky}{-2}
\draw[gray](0,\ky-1.5)--++(0,2)node[left]{$-Au(t-t_0)$};
\draw[gray](-0.5,\ky)--++(3.5,0)node[right]{$t$};
\draw(-0.25,\ky)--++(1.25,0)--++(0,-1)--++(2,0);
\draw(0,\ky-1)node[left]{$-A$};
%
\pgfmathsetmacro{\ky}{-3}
\draw[gray](0,2*\ky-0.5)--++(0,2)node[left]{$v(t)$};
\draw[gray](-0.5,2*\ky)--++(3.5,0)node[right]{$t$};
\draw(-0.25,2*\ky)--++(0.25,0)--++(0,1)node[left]{$A$}--++(1,0)--++(0,-1)node[below]{$t_0$}--++(2,0);
\end{tikzpicture}%
\end{subfigure}%
\caption{اکائی سیڑھی تفاعل کے استعمال سے دیگر تفاعل کا حصول۔}
\label{شکل_عارضی_اکائی_سیڑھی_تفاعل_ب}
\end{figure}


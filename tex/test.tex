\begin{align*}
v_{R1}&=i(t) R_1\\
&=\left[\frac{v(t)}{R_1+R_2}\right] R_1
\end{align*}
یا
\begin{align}\label{مساوات_مزاحمتی_تقسیم_دباو_ٹ}
v_{R1}=\frac{R_1}{R_1+R_2} v(t)
\end{align}


\ابتدا{مثال}
شکل \حوالہ{شکل_مزاحمتی_دباو_تقسیم} میں \عددی{v(t)=\SI{15}{\volt}} ہے جبکہ مزاحمت \عددی{R_1=\SI{1}{\kilo\ohm}} اور \عددی{R_2=\SI{2}{\kilo\ohm}} ہیں۔دونوں مزاحمت کے دباو حاصل کریں۔منبع اور مزاحمتوں کی طاقت دریافت کریں۔

مساوات \حوالہ{مساوات_مزاحمتی_تقسیم_دباو_ٹ} سے
\begin{align*}
v_{R1}=\frac{15 \times 1000}{1000+2000}=\SI{5}{\volt} 
\end{align*}
اور مساوات \حوالہ{مساوات_مزاحمتی_تقسیم_دباو_ث} سے
\begin{align*}
v_{R2}=\frac{15 \times 2000}{1000+2000}=\SI{10}{\volt} 
\end{align*}
حاصل ہوتا ہے۔یہی جوابات یوں بھی حاصل کئے جا سکتے ہیں کہ پہلے مساوات \حوالہ{مساوات_مزاحمتی_تقسیم_دباو_ت} سے رو
\begin{align*}
i(t)=\frac{15}{1000+2000}=\SI{5}{\milli\ampere}
\end{align*}
حاصل کریں اور پھر قانون اوہم سے
\begin{align*}
v_{R1}&=i(t) R_1 =5 \times 10^{-3} \times 1000=\SI{5}{\volt}\\
v_{R2}&=i(t) R_2 =5 \times 10^{-3} \times 2000=\SI{10}{\volt}
\end{align*}
لکھیں۔منبع کی طاقت
\begin{align*}
p_{\text{منبع}}=15 \times (-5\times 10^{-3})=\SI{-75}{\milli\watt}
\end{align*}
جبکہ \عددی{R_1} کی طاقت
\begin{align*}
p_{R1}=5\times 5\times 10^{-3}= \SI{25}{\milli\watt}
\end{align*}
اور \عددی{R_2} کی طاقت
\begin{align*}
p_{R2}=10\times 5\times 10^{-3}= \SI{50}{\milli\watt}
\end{align*}
حاصل ہوتی ہے۔آپ دیکھ سکتے ہیں کہ طاقت کی پیداوار اور ضیاع برابر ہیں۔

مزاحمت کی طاقت مساوات  \حوالہ{مساوات_مزاحمتی_کلیات_طاقت} میں دئے دیگر کلیات سے بھی حاصل کر کے دیکھتے ہیں۔
\begin{align*}
p_{R1}&=i^2(t) R_1=(5\times 10^{-3})^2 \times 1000=\SI{25}{\milli\watt}\\
p_{R1}&=\frac{v^2_{R1}}{R_1}=\frac{5^2}{1000}=\SI{25}{\milli\watt}\\
p_{R2}&=i^2(t) R_2=(5\times 10^{-3})^2 \times 2000=\SI{50}{\milli\watt}\\
p_{R2}&=\frac{v^2_{R2}}{R_2}=\frac{10^2}{2000}=\SI{50}{\milli\watt}
\end{align*}
\انتہا{مثال}
%====================

آپ دیکھ سکتے ہیں کہ سلسلہ وار مزاحمت جوڑنے سے داخلی دباو کو مختلف قیمتوں میں تقسیم کیا جا سکتا ہے۔دو سے زیادہ مزاحمت سلسلہ وار جوڑتے ہوئے داخلی دباو کو زیادہ حصوں میں تقسیم کیا جا سکتا ہے۔تقسیم دباو کے مساوات کے تحت داخلی دباو سلسلہ وار جڑے مزاحمت پر مزاحمت کی قیمت کے نسبت سے تقسیم ہوتے ہیں۔مندرجہ بالا مثال میں آپ نے دیکھا کہ تقسیم دباو کی مساوات سے مزاحمت کا دباو حاصل کرتے ہوئے  برقی رو  کا حصول درکار نہیں ہوتا۔آپ نے یہ بھی دیکھ لیا ہو گا کہ زیادہ قیمت کی مزاحمت پر زیادہ دباو پیدا ہوتی ہے اور اس میں طاقت کا ضیاع بھی زیادہ ہوتا ہے۔
%===================
\ابتدا{مشق}
شکل \حوالہ{شکل_مزاحمتی_دباو_تقسیم} میں \عددی{v(t)=\SI{10}{\volt}} ہے جبکہ مزاحمت \عددی{R_1=\SI{2}{\kilo\ohm}} ہے۔ مزاحمت \عددی{R_2} پر \عددی{\SI{6}{\volt}} درکار ہیں۔اس مزاحمت کی قیمت حاصل کریں اور اس میں طاقت کا ضیاع دریافت کریں۔منبع کی پیدا کردہ طاقت بھی دریافت کریں۔ اگر \عددی{R_2} کی قیمت \عددی{\SI{2}{\kilo\ohm}} ہوتی تب \عددی{R_2} کی دباو اور  طاقت کے علاوہ منبع کی پیدا کردہ طاقت کیا ہوتی۔

جواب: \عددی{R_2=\SI{3}{\kilo\ohm}}، \عددی{\SI{12}{\milli\watt}}، \عددی{\SI{-20}{\milli\watt}}، \عددی{\SI{5}{\volt}}، \عددی{\SI{12.5}{\milli\watt}}، \عددی{\SI{-25}{\milli\watt}}
\انتہا{مشق}
%=======================

 اس مشق سے ظاہر ہے کہ کل سلسلہ وار مزاحمت کی قیمت کم کرنے سے پیدا کردہ طاقت اور مزاحمت میں طاقت کا ضیاع بڑھتا ہے۔

\حصہ{متعدد سلسلہ وار مزاحمت}
شکل \حوالہ{شکل_مزاحمتی_متعدد_مزاحمت_تقسیم_دباو}-الف میں متعدد مزاحمت سلسلہ وار جڑے ہیں۔تمام سلسلہ وار جڑے پرزوں میں یکساں رو \عددی{i(t)} پائی جاتی ہے۔کرخوف قانون دباو سے
\begin{align}
v(t)=v_{R1}+v_{R2}+v_{R3}+\cdots + v_{Rn}
\end{align}
لکھتے ہیں جہاں قانون اوہم سے
\begin{align*}
v_{R1}&=i(t) R_1\\
v_{R2}&=i(t) R_2\\
\vdots \\
v_{Rn}&=i(t) R_n
\end{align*}
لکھا جا سکتا ہے۔یوں
\begin{align*}
v(t)&=i(t) R_1+i(t) R_2+\cdots+ i(t)R_n
\end{align*}
یا
\begin{align}\label{مساوات_مزاحمتی_متعدد_تقسیم_دباو_مساوی-الف}
v(t)&=i(t)\left[R_1+R_2+\cdots+R_n \right]
\end{align}
حاصل ہوتا ہے جس میں
\begin{align}\label{مساوات_مزاحمتی_متعدد_سلسلہ_مساوی_مزاحمت}
R_s=R_1+R_2+R_3+\cdots+R_n \quad \text{\RL{متعدد سلسلہ وار جڑے مزاحمتوں کا مساوی مزاحمت}}
\end{align}
لکھتے ہوئے
\begin{align}\label{مساوات_مزاحمتی_متعدد_تقسیم_دباو_مساوی_ب}
v(t)=i(t) R_s
\end{align}
حاصل ہوتا ہے۔مساوات \حوالہ{مساوات_مزاحمتی_متعدد_تقسیم_دباو_مساوی-الف} اور مساوات \حوالہ{مساوات_مزاحمتی_متعدد_تقسیم_دباو_مساوی_ب} شکل \حوالہ{شکل_مزاحمتی_متعدد_مزاحمت_تقسیم_دباو}-ب پر بھی پوری اترتے ہیں۔یوں شکل \حوالہ{شکل_مزاحمتی_متعدد_مزاحمت_تقسیم_دباو}-الف اور شکل \حوالہ{شکل_مزاحمتی_متعدد_مزاحمت_تقسیم_دباو}-ب مساوی اشکال ہیں۔آپ دیکھ سکتے ہیں کہ متعدد سلسلہ وار جڑے مزاحمت کی جگہ ان کا مجموعی مزاحمت نسب کیا جا سکتا ہے۔مساوات \حوالہ{مساوات_مزاحمتی_متعدد_سلسلہ_مساوی_مزاحمت} متعدد سلسلہ وار جڑے مزاحمتوں کا مساوی مزاحمت \عددی{R_s} دیتی ہے۔ 
\begin{figure}
\centering
\begin{subfigure}{0.5\textwidth}
\centering
\includegraphics[scale=0.8]{figResistanceVoltageDividerMultipleResistors}
\caption*{(الف)}
\end{subfigure}%
%
\begin{subfigure}{0.5\textwidth}
\centering
\includegraphics[scale=0.8]{figResistanceVoltageDividerMultipleResistorsEquivalent}
\caption*{(ب)}
\end{subfigure}%
\caption{متعدد سلسلہ وار مزاحمت اور تقسیم دباو۔}
\label{شکل_مزاحمتی_متعدد_مزاحمت_تقسیم_دباو}
\end{figure}
%==================

\ابتدا{مثال}
شکل \حوالہ{شکل_مزاحمتی_متعدد_مزاحمت_تقسیم_دباو}-الف میں چار عدد مزاحمت نسب ہیں جن کی قیمتیں \عددی{\SI{100}{\ohm}}، \عددی{\SI{50}{\ohm}}، \عددی{\SI{120}{\ohm}} اور \عددی{\SI{30}{\ohm}} ہیں۔منبع دباو \عددی{\SI{9}{\volt}} پیدا کرتا ہے۔دور میں رو دریافت کریں۔پچاس اوہم مزاحمت پر دباو بھی حاصل کریں۔

حل:مجموعی مزاحمت کی قیمت
\begin{align*}
R_S=100+50+120+30=\SI{300}{\ohm}
\end{align*}
ہے۔یوں قانون اوہم اور شکل-ب سے
\begin{align*}
i(t)=\frac{v(t)}{R_S}=\frac{9}{300}=\SI{30}{\milli\ampere}
\end{align*}
حاصل ہوتا ہے۔پچاس اوہم مزاحمت پر دباو قانون اوہم سے درج ذیل حاصل ہوتا ہے۔
\begin{align*}
v_{\SI{50}{\ohm}}=i(t) R = 30\times 10^{-3} \times 50=\SI{1.5}{\volt}
\end{align*}
\انتہا{مثال}
\FloatBarrier
%====================
\ابتدا{مثال}
ایک ملی میٹر قطر کے المونیم تار کی مزاحمت \عددی{\SI{33.33}{\ohm}} فی کلومیٹر ہے۔ اس تار کو استعمال کرتے ہوئے \عددی{\SI{220}{\volt}} منبع دباو سے \عددی{\SI{50}{\ohm}} کے مزاحمتی بوجھ کو طاقت فراہم کی جاتی ہے۔منبع اور بوجھ کے درمیان \عددی{\SI{5}{\meter}} کا فاصلہ ہونے کی صورت میں مزاحمت میں طاقت کا ضیاع دریافت کریں۔ اگر یہ فاصلہ \عددی{\SI{1}{\kilo\meter}} ہوتا تب جواب کیا ہوتا؟ 

\begin{figure}
\centering
\begin{subfigure}{0.33\textwidth}
\centering
\includegraphics[scale=0.8]{figResistancePowerFedViaLongWireA}
\caption*{(الف)}
\end{subfigure}%
\begin{subfigure}{0.33\textwidth}
\centering
\includegraphics[scale=0.8]{figResistancePowerFedViaLongWireB}
\caption*{(ب)}
\end{subfigure}%
\begin{subfigure}{0.33\textwidth}
\centering
\includegraphics[scale=0.8]{figResistancePowerFedViaLongWireC}
\caption*{(پ)}
\end{subfigure}%
\caption{برقی بوجھ کو بذریعہ تار طاقت فراہم کی جا رہی ہے۔}
\label{شکل_مزاحمتی_فراہمی_طاقت_بذریعہ_تار}
\end{figure}

حل:منبع کے مثبت اور منفی سروں کو بوجھ کے دو سروں کے ساتھ جوڑا جاتا ہے۔چونکہ ایک کلومیٹر تار کی مزاحمت \عددی{\SI{33.33}{\ohm}} ہے لہٰذا پانچ میٹر تار کی مزاحمت \عددی{\SI{166.65}{\milli\ohm}} ہو گی۔صورت حال شکل \حوالہ{شکل_مزاحمتی_فراہمی_طاقت_بذریعہ_تار}-الف میں دکھائی گئی ہے۔بالائی اور نچلی تار سلسلہ وار جڑے ہیں لہٰذا ان کے مزاحمت آپس میں جمع کئے جا سکتے ہیں۔ایسا کرتے ہوئے مسئلے کو شکل  \حوالہ{شکل_مزاحمتی_فراہمی_طاقت_بذریعہ_تار}-ب کے طرز پر ظاہر کیا جا سکتا ہے۔ادوار کے اشکال بناتے ہوئے عموماً ایسا ہی کرتے ہوئے تار کی مجموعی مزاحمت کو بالائی تار پر ظاہر کیا جاتا ہے جبکہ نچلی تار کی مزاحمت صفر تصور کی جاتی ہے۔دور میں رو
\begin{align*}
i=\frac{220}{50+0.16665}=\SI{4.3854}{\ampere}
\end{align*}
اور  بوجھ میں طاقت کا ضیاع
\begin{align*}
p=i^2 R=4.3854^2 \times 50=\SI{962}{\watt}
\end{align*}
ہے۔یہاں غور کریں کہ تار کی مزاحمت بوجھ کی مزاحمت سے بہت کم ہے۔ایسی صورت میں تار کی مزاحمت کو رد کیا جا سکتا ہے اور تار کو کامل موصل تصور کیا جا سکتا ہے۔ایسا کرتے ہوئے تار کی مزاحمت کو \عددی{\SI{0}{\ohm}} تصور کرتے ہوئے جوابات
\begin{align*}
i&=\frac{220}{50+0}=\SI{4.4}{\ampere}\\
p&=4.4^2 \times 50=\SI{968}{\watt}
\end{align*}
حاصل ہوتے ہیں۔ان دو جوابات میں صرف
\begin{align*}
\abs{\frac{962-968}{962}}\times 100=\SI{0.62}{\percent}
\end{align*}
فرق پایا جاتا ہے جسے رد کیا جا سکتا ہے۔اس کے برعکس منبع اور تار کے درمیان ایک کلومیٹر فاصلے کی صورت میں صورت حال شکل-پ ظاہر کرتی ہے جہاں سے
\begin{align*}
i&=\frac{220}{50+66.66}=\SI{1.8858}{\ampere}\\
p&=1.8858^2 \times 50=\SI{179}{\watt}
\end{align*}
حاصل ہوتے ہیں۔یہاں تار کی مزاحمت کو رد نہیں کیا جا سکتا اور اس کے اثرات کو مد نظر رکھنا ضروری ہے۔
\انتہا{مثال}
%=====================

\حصہ{سلسلہ وار متعدد منبع دباو اور مزاحمت}
شکل \حوالہ{شکل_مزاحمتی_متعدد_سلسلہ_وار_دور}-الف میں متعدد منبع دباو اور متعدد مزاحمت سلسلہ وار جڑے ہیں۔سلسلہ وار دور میں یکساں رو \عددی{i(t)} پائی جائے گی۔دور میں گھڑی کی سمت گھومتے اور گھٹتے دباو کو مثبت لکھتے ہوئے
\begin{align}
v_1(t)-v_2(t)+v_{R1}+v_{R2}-v_3(t)+v_{R3}+v_{R4}+\cdots+v_k(t)+v_{Rn}=0
\end{align}
لکھا جا سکتا ہے۔منبع دباو کو ایک جانب اور مزاحمتی دباو کو دوسری جانب لکھتے ہوئے اسے درج ذیل صورت میں لکھا جا سکتا ہے۔
\begin{align}
-v_1(t)+v_2(t)+v_3(t)+\cdots-v_k(t)=v_{R1}+v_{R2}+v_{R3}+v_{R4}+\cdots+v_{Rn}
\end{align}
قانون اوہم کی مدد سے \عددی{v_{R1}=i(t) R_1} وغیرہ لکھتے ہوئے 
\begin{gather}
\begin{aligned}
-v_1(t)+v_2(t)+v_3(t)+\cdots-v_k(t)&=i(t) R_1+i(t) R_2+i(t) R_3+i(t) R_4+\cdots+i(t) R_n\\
&=i(t)\left[R_1+R_2+\cdots +R_n \right]
\end{aligned}
\end{gather}
حاصل ہوتا ہے۔اس مساوات میں 
\begin{align}
-v_1(t)+v_2(t)+v_3(t)+\cdots-v_k(t)&=v_s(t)\\
R_1+R_2+\cdots +R_n&=R_s
\end{align}
لکھنے سے
\begin{align}\label{مساوات_مزاحمتی_سلسلہ_وار_متعدد_مزاحمت_منبع_مساوی}
v_s(t)=i(t) R_s
\end{align}
حاصل ہوتا ہے۔اس مساوات سے حاصل دور کو شکل \حوالہ{شکل_مزاحمتی_متعدد_سلسلہ_وار_دور}-ب میں دکھایا گیا ہے۔آپ دیکھ سکتے ہیں کہ تمام سلسلہ وار جڑے مزاحمت کی جگہ ان کا مجموعہ نسب کیا جا سکتا ہے اور اسی طرح تمام سلسلہ وار جڑے منبع کی جگہ ان کا مجموعہ نسب کیا جا سکتا ہے۔جیسا شکل \حوالہ{شکل_مزاحمتی_متعدد_سلسلہ_وار_دور}-ب میں دکھایا گیا ہے،منبع کا مجموعہ حاصل کرتے وقت بڑھتے دباو کو مثبت اور گھٹتے دباو کو منفی لیا جاتا ہے۔یوں مساوات \حوالہ{مساوات_مزاحمتی_سلسلہ_وار_متعدد_مزاحمت_منبع_مساوی} میں مساوی نشان \عددی{(=)} کے بائیں جانب بڑھتے دباو کا مجموعہ اور نشان کے دائیں جانب گھٹتے دباو کا مجموعہ ہے۔اس مساوات سے دور کی رو \عددی{i(t)} حاصل کی جا سکتی ہے۔
\begin{figure}
\centering
\begin{subfigure}{\textwidth}
\centering
\includegraphics{figResistanceVoltageDividerMultipleSuppliesAndResistors}
\caption*{(الف)}
\end{subfigure}
%
\begin{subfigure}{\textwidth}
\centering
\includegraphics{figResistanceVoltageDividerMultipleSourcesAndResistorsEquivalent}
\caption*{(ب)}
\end{subfigure}
\caption{متعدد منبع اور متعدد مزاحمت سلسلہ وار جڑے ہیں۔}
\label{شکل_مزاحمتی_متعدد_سلسلہ_وار_دور}
\end{figure}

\حصہ{متوازی جڑے مزاحمت پر یکساں دباو پایا جاتا ہے}
شکل \حوالہ{شکل_مزاحمتی_متوازی_جڑے_پرزے}-الف میں منبع دباو کے متوازی دو عدد برقی پرزے جڑے دکھائے گئے ہیں۔بند دائرہ \عددی{abcda} پر کرخوف قانون دباو سے
\begin{align}
v(t)=v_{cd}
\end{align}
حاصل ہوتا ہے جبکہ بند دائرہ \عددی{abefa} پر کرخوف  قانون دباو سے
\begin{align}
v(t)=v_{ef}
\end{align}
حاصل ہوتا ہے۔یوں دونوں برقی پرزوں پر \عددی{v(t)} دباو پایا جاتا ہے۔ اس مثال میں مزید پرزے متوازی جوڑتے ہوئے آپ دیکھ سکتے ہیں کہ تمام متوازی جڑے پرزوں پر یکساں دباو پایا جاتا ہے۔
\begin{figure}
\centering
\includegraphics{figResistancesInParallel}
\caption{متوازی جڑے پرزوں پر یکساں دباو پایا جاتا ہے}
\label{شکل_مزاحمتی_متوازی_جڑے_پرزے}
\end{figure}

\حصہ{تقسیم رو}
شکل \حوالہ{شکل_مزاحمتی_متوازی_جڑے_پرزوں_میں_تقسیم_رو}-الف میں منبع رو \عددی{i(t)} کے متوازی دو عدد مزاحمت جڑے ہیں۔رو \عددی{i(t)} متوازی جڑے مزاحمت سے گزرتی ہے جس سے اوہم کے قانون کے تحت مزاحمت پر دباو \عددی{v(t)} پیدا ہو گا۔مزاحمت \عددی{R_1} میں رو \عددی{i_1(t)} اور مزاحمت \عددی{R_2} میں رو \عددی{i_2(t)} پائی جائے گی۔جوڑ \عددی{b} پر کرخوف قانون رو لکھتے ہیں۔
\begin{align}\label{مساوات_مزاحمتی_تقسیم_رو_الف}
i(t)=i_1(t)+i_2(t)
\end{align}
مزاحمتوں کے لئے قانون اوہم سے
\begin{align}\label{مساوات_مزاحمتی_تقسیم_رو_ب}
i_1(t)&=\frac{v(t)}{R_1}\\
i_2(t)&=\frac{v(t)}{R_2}
\end{align}
لکھا جا سکتا ہے۔درج بالا تین مساوات کے ملاپ سے
\begin{gather}
\begin{aligned}\label{مساوات_مزاحمتی_تقسیم_رو_پ}
i(t)&=\frac{v(t)}{R_1}+\frac{v(t)}{R_2}\\
&=\left(\frac{1}{R_1}+\frac{1}{R_2}\right) v(t)
\end{aligned}
\end{gather}
لکھا جا سکتا ہے۔اس مساوات میں قوسین میں بند قیمت کو
\begin{align}\label{مساوات_مزاحمتی_متوازی_مساوی}
\frac{1}{R_m}=\frac{1}{R_1}+\frac{1}{R_2}
\end{align}
لکھتے ہوئے
\begin{align}\label{مساوات_مزاحمتی_دباو_اور_مساوی_مزاحمت_برابر_دباو}
i(t)=\frac{v(t)}{R_m}
\end{align}
لکھا جا سکتا ہے۔شکل \حوالہ{شکل_مزاحمتی_متوازی_جڑے_پرزوں_میں_تقسیم_رو}-ب سے یہی مساوات لکھی جا سکتی ہے۔متوازی جڑے مزاحمتوں کی مساوی مزاحمت مساوات \حوالہ{مساوات_مزاحمتی_متوازی_مساوی} سے حاصل ہوتی ہے۔
\begin{figure}
\centering
\begin{subfigure}{0.5\textwidth}
\centering
\includegraphics{figResistancesCurrentDivider}
\caption*{(الف)}
\end{subfigure}%
\begin{subfigure}{0.5\textwidth}
\centering
\includegraphics{figResistancesParallelResisitorsEquivalent}
\caption*{(ب)}
\end{subfigure}%
\caption{متوازی جڑے مزاحمت کا مساوی مزاحمت۔}
\label{شکل_مزاحمتی_متوازی_جڑے_پرزوں_میں_تقسیم_رو}
\end{figure}

مساوات \حوالہ{مساوات_مزاحمتی_تقسیم_رو_ب} کے پہلی مساوات  کو مساوات \حوالہ{مساوات_مزاحمتی_تقسیم_رو_ب} سے تقسیم کرتے ہوئے
\begin{align*}
\frac{i_1(t)}{i(t)}=\frac{R_2}{R_1+R_2}
\end{align*}
یا
\begin{align}\label{مساوات_مزاحمتی_تقسیم_رو_کی_مساوات_الف}
i_1(t)=\frac{R_2}{R_1+R_2} i(t)
\end{align}
لکھا جا سکتا ہے۔اسی طرح مساوات \حوالہ{مساوات_مزاحمتی_تقسیم_رو_ب} کے دوسری مساوات  کو مساوات \حوالہ{مساوات_مزاحمتی_تقسیم_رو_ب} سے تقسیم کرتے ہوئے
\begin{align}\label{مساوات_مزاحمتی_تقسیم_رو_کی_مساوات_ب}
i_2(t)=\frac{R_1}{R_1+R_2} i(t)
\end{align}
حاصل ہوتا ہے۔مساوات \حوالہ{مساوات_مزاحمتی_تقسیم_رو_کی_مساوات_الف} اور مساوات \حوالہ{مساوات_مزاحمتی_تقسیم_رو_کی_مساوات_ب} تقسیم رو کے مساوات ہیں۔

مساوات \حوالہ{مساوات_مزاحمتی_متوازی_مساوی} سے دو عدد متوازی مزاحمتوں کا مساوی مزاحمت 
\begin{align}
R_m=\frac{R_1 R_2}{R_1+R_2}
\end{align}
حاصل کیا جا سکتا ہے۔ 
%=======================
\ابتدا{مثال}
شکل \حوالہ{شکل_مزاحمتی_متوازی_جڑے_پرزوں_میں_تقسیم_رو} میں \عددی{R_1=\SI{2}{\kilo\ohm}}، \عددی{R_2=\SI{6}{\kilo\ohm}} اور \عددی{i(t)=\SI{8}{\milli\ampere}} ہیں۔مزاحمت \عددی{R_1} اور مزاحمت \عددی{R_2} میں رو دریافت کریں۔کل متوازی مزاحمت دریافت کریں۔مزاحمت \عددی{R_1} اور \عددی{R_2} میں طاقت کا ضیاع دریافت کریں۔منبع کی طاقت بھی حاصل کریں۔

حل:مساوات \حوالہ{مساوات_مزاحمتی_تقسیم_رو_کی_مساوات_الف} سے
\begin{align*}
i_1(t)=\left(\frac{6000}{2000+6000}\right)  \times 8\times 10^{-3} =\SI{6}{\milli\ampere}
\end{align*}
حاصل ہوتا ہے جبکہ مساوات \حوالہ{مساوات_مزاحمتی_تقسیم_رو_کی_مساوات_ب} سے
\begin{align*}
i_2(t)=\left(\frac{2000}{2000+6000}\right) \times 8\times 10^{-3}=\SI{2}{\milli\ampere}
\end{align*}
حاصل ہوتا ہے۔یہی جواب بالائی جوڑ پر کرخوف قانون رو
\begin{align*}
\SI{8}{\milli\ampere}=\SI{6}{\milli\ampere}+i_2(t)
\end{align*}
یعنی
\begin{align*}
i_2(2)=\SI{8}{\milli\ampere}-\SI{6}{\milli\ampere}=\SI{2}{\milli\ampere}
\end{align*}
سے بھی حاصل کیا جا سکتا ہے۔کل متوازی مزاحمت
\begin{align*}
\frac{1}{R_m}=\frac{1}{2000}+\frac{1}{6000}=\frac{1}{1500}
\end{align*}
سے 
\begin{align*}
R_m=\SI{1.5}{\kilo\ohm}
\end{align*}
حاصل ہوتا ہے۔مزاحمت \عددی{R_1} میں طاقت کا ضیاع
\begin{align*}
p_{R1}=i_1(t)^2 R_1=(6\times 10^{-3})^2  \times 2000=\SI{72}{\milli\watt}
\end{align*}
ہے۔اسی طرح مزاحمت \عددی{R_2} کی طاقت
\begin{align*}
p_{R2}=i_2(t)^2 R_2=(2\times 10^{-3})^2  \times 6000=\SI{24}{\milli\watt}
\end{align*}
ہے۔منبع کی طاقت حاصل کرنے کے لئے منبع کا دباو جاننا ضروری ہے۔مساوات \حوالہ{مساوات_مزاحمتی_دباو_اور_مساوی_مزاحمت_برابر_دباو} سے  منبع کا دباو
\begin{align*}
v(t)=i(t) R_m=8\times 10^{-3}\times 1500=\SI{12}{\volt} 
\end{align*}
حاصل ہوتا ہے۔یوں منبع کی طاقت درج ذیل ہو گی جو دونوں مزاحمت کے مجموعی طاقت کے  عین برابر ہے۔
\begin{align*}
p_{\text{منبع}}=v(t) i(t)=12\times 8\times 10^{-3}=\SI{96}{\milli\watt}
\end{align*}
\انتہا{مثال}
%========================

اس مثال سے آپ دیکھ سکتے ہیں کہ متوازی جڑے مزاحمتوں میں کم قیمت کے مزاحمت میں زیادہ رو پائی جاتی ہے۔آپ کو یاد ہو گا کہ سلسلہ وار جڑے مزاحمتوں میں تقسیم دباو کے تحت زیادہ قیمت کے مزاحمت پر زیادہ دباو پایا جاتا ہے۔

دو سے زیادہ تعداد میں متوازی جڑے مزاحمتوں کو بالکل اسی طرح حل کیا جا سکتا ہے۔یوں شکل \حوالہ{شکل_مزاحمتی_متعدد_متوازی_جڑے_پرزوں_میں_تقسیم_رو}-الف سے
\begin{align*}
i(t)&=i_1(t)+i_2(t)+i_3(t)+\cdots+i_n(t)\\
i_1(t)&=\frac{v(t)}{R_1}\\
i_2(t)&=\frac{v(t)}{R_2}\\
i_3(t)&=\frac{v(t)}{R_3}\\
\vdots\\
i_N(t)&=\frac{v(t)}{R_N}\\
\end{align*}
یا
\begin{align}
i(t)&=\left(\frac{1}{R_1}+\frac{1}{R_2}+\frac{1}{R_3}+\cdots +\frac{1}{R_N}\right) v(t)
\end{align}
حاصل ہوتا ہے جس میں
\begin{gather}
\begin{aligned}\label{مساوات_مزاحمتی_متعدد_متوازی_کا_مساوی}
\frac{1}{R_m}&=\frac{1}{R_1}+\frac{1}{R_2}+\frac{1}{R_3}+\cdots +\frac{1}{R_N} \quad \text{\RL{متوازی مزاحمتوں کا مساوی مزاحمت}} \\
&=\sum \limits_{n=1}^{N} \frac{1}{R_n}
\end{aligned}
\end{gather}
پر کرنے سے
\begin{align}
i(t)=\frac{v(t)}{R_m}
\end{align}
لکھا جا سکتا ہے۔ شکل \حوالہ{شکل_مزاحمتی_متعدد_متوازی_جڑے_پرزوں_میں_تقسیم_رو}-ب سے بھی یہی مساوات حاصل ہوتی ہے لہٰذا شکل-الف اور شکل-ب مساوی ادوار ہیں۔مساوات \حوالہ{مساوات_مزاحمتی_متعدد_متوازی_کا_مساوی} متعدد متوازی جڑے مزاحمتوں کا مساوی مزاحمت \عددی{R_m} دیتی ہے۔
\begin{figure}
\centering
\begin{subfigure}{0.6\textwidth}
\centering
\includegraphics{figResistancesMultipleInParallel}
\caption*{(الف)}
\end{subfigure}%
\begin{subfigure}{0.4\textwidth}
\centering
\includegraphics{figResistancesMultipleParallelResisitorsEquivalent}
\caption*{(ب)}
\end{subfigure}%
\caption{متعدد متوازی جڑے مزاحمت کا مساوی مزاحمت۔}
\label{شکل_مزاحمتی_متعدد_متوازی_جڑے_پرزوں_میں_تقسیم_رو}
\end{figure}

%=====================
\ابتدا{مثال}
شکل \حوالہ{شکل_مزاحمتی_متعدد_متوازی_جڑے_پرزوں_میں_تقسیم_رو}-الف میں تین عدد مزاحمت استعمال ہوتے ہیں۔ان کی قیمتیں \عددی{\SI{2}{\kilo\ohm}}، \عددی{\SI{4}{\kilo\ohm}} اور \عددی{\SI{5}{\kilo\ohm}} ہیں۔منبع رو \عددی{\SI{15}{\milli\ampere}} ہے۔مساوی متوازی مزاحمت \عددی{R_m} حاصل کریں۔دباو \عددی{v(t)} حاصل کرتے ہوئے تمام مزاحمتوں میں رو حاصل کریں۔منبع کی طاقت اور مزاحمتوں میں طاقت کا ضیاع بھی دریافت کریں۔

جوابات:مساوی مزاحمت پہلے حاصل کرتے ہیں۔مساوات \حوالہ{مساوات_مزاحمتی_متعدد_متوازی_کا_مساوی} سے
\begin{align*}
\frac{1}{R_m}=\frac{1}{2000}+\frac{1}{4000}+\frac{1}{5000}=\frac{19}{20000}
\end{align*}
یعنی
\begin{align*}
R_m=\frac{20}{19} \, \si{\kilo\ohm}
\end{align*}
حاصل ہوتا ہے۔شکل \حوالہ{شکل_مزاحمتی_متعدد_متوازی_جڑے_پرزوں_میں_تقسیم_رو}-ب سے 
\begin{align*}
v(t)=15 \times 10^{-3} \times \frac{20000}{19} \approx \SI{15.7895}{\volt}
\end{align*}
حاصل ہوتا ہے۔یوں شکل-الف سے رو درج ذیل حاصل ہوتے ہیں جہاں سے آپ دیکھ سکتے ہیں کہ \عددی{i_1(t)+i_2(t)+i_3(t)} عین منبع کی رو کے برابر ہے۔
\begin{align*}
i_1(t)&=\frac{15.7895}{2000}=\SI{7.89}{\milli\ampere}\\
i_2(t)&=\frac{15.7895}{4000}=\SI{3.95}{\milli\ampere}\\
i_3(t)&=\frac{15.7895}{5000}=\SI{3.16}{\milli\ampere}
\end{align*}
منبع کی طاقت
\begin{align*}
p_{\text{منبع}}=15.7895 \times (-15\times 10^{-3})=\SI{-236.8}{\milli\watt}
\end{align*}
جبکہ مزاحمتوں کی طاقت
\begin{align*}
p_{\SI{2}{\kilo\ohm}}&=15.7895\times 7.89\times 10^{-3}=\SI{124.58}{\milli\watt}\\
p_{\SI{4}{\kilo\ohm}}&=15.7895\times 3.95\times 10^{-3}=\SI{62.37}{\milli\watt}\\
p_{\SI{5}{\kilo\ohm}}&=15.7895\times 3.16\times 10^{-3}=\SI{49.89}{\milli\watt}
\end{align*}
حاصل ہوتے ہیں۔آپ دیکھ سکتے ہیں کہ پیدا کردہ طاقت اور طاقت کا ضیاع برابر ہیں۔متوازی جڑے مزاحمتوں میں زیادہ قیمت کے مزاحمت میں کم برقی رو پائی جاتی ہے اور اس میں طاقت کا ضیاع بھی کم ہوتا ہے۔
\انتہا{مثال}
\FloatBarrier
%========================
\ابتدا{مشق}
شکل \حوالہ{مشق_مزاحمتی_تقسیم_رو_مشق} میں \عددی{i_1}، \عددی{i_2}، \عددی{R_m} اور \عددی{v_m} دریافت کریں۔
\begin{figure}
\centering
\includegraphics{figResistanceQuizParallelResistorsCurrent}
\caption{تقسیم رو کی مشق۔}
\label{مشق_مزاحمتی_تقسیم_رو_مشق}
\end{figure}

جوابات:\عددی{i_1=\SI{-400}{\milli\ampere}}، \عددی{i_2=\SI{1200}{\milli\ampere}}، \عددی{R_m=\SI{75}{\ohm}}، \عددی{v_m=\SI{120}{\volt}}
\انتہا{مشق}
%==================
\FloatBarrier
%========================
\ابتدا{مشق}
شکل \حوالہ{مشق_مزاحمتی_تقسیم_دباو_مشق} میں \عددی{R_s}، \عددی{i}، \عددی{v_1}، \عددی{v_2}،  اور \عددی{v_3} دریافت کریں۔تین وولٹ اور پانچ وولٹ منبع کی طاقت دریافت کریں۔
\begin{figure}
\centering
\includegraphics{figResistanceQuizSeriesResistorsVoltage}
\caption{تقسیم دباو کی مشق۔}
\label{مشق_مزاحمتی_تقسیم_دباو_مشق}
\end{figure}

جوابات:\عددی{R_s=\SI{100}{\ohm}}، \عددی{i=\SI{-90}{\milli\ampere}}، \عددی{v_1=\SI{4.5}{\volt}}، \عددی{v_2=\SI{-0.9}{\volt}}، \عددی{v_3=\SI{-3.6}{\volt}}، \عددی{p_{\SI{3}{\volt}}=\SI{-0.27}{\watt}}، \عددی{p_{\SI{5}{\volt}}=\SI{0.45}{\watt}}
\انتہا{مشق}
%=================
\FloatBarrier


\حصہ{سلسلہ وار اور متوازی مزاحمت}
ہم جانتے ہیں کہ سلسلہ وار مزاحمتوں کا مساوی مزاحمت
\begin{align}\label{مساوات_مزاحمتی_سلسلہ_وار_دوبارہ}
R_s=R_1+R_2+R_3+\cdots+R_N
\end{align}
ہوتا ہے جبکہ متوازی مزاحمتوں کا مساوی مزاحمت
\begin{align}\label{مساوات_مزاحمتی_متوازی_دوبارہ}
\frac{1}{R_m}=\frac{1}{R_1}+\frac{1}{R_2}+\frac{1}{R_3}+\cdots+\frac{1}{R_N}
\end{align}
ہے۔آئیں ان کلیات کو استعمال کرتے ہوئے مختلف انداز میں جڑے مزاحمتوں کا مساوی مزاحمت حاصل کریں۔ایسا کرنے کی خاطر شکل \حوالہ{مشق_مزاحمتی_سلسلہ_وار_متوازی_مزاحمت} میں کو مثال بناتے ہوئے \عددی{A} اور \عددی{B} کے مابین مزاحمت \عددی{R_{AB}} حاصل کرتے ہیں۔
\begin{figure}
\centering
\includegraphics{figResistancesMixedSeriesParallelA}
\caption{سلسلہ وار اور متوازی مزاحمت۔}
\label{مشق_مزاحمتی_سلسلہ_وار_متوازی_مزاحمت}
\end{figure}

اگر آپ \عددی{FGH} کو دیکھیں تو  یہاں \عددی{\SI{3}{\kilo\ohm}} اور \عددی{\SI{1}{\kilo\ohm}} سلسلہ وار جڑے ہیں۔دو مزاحمت تب سلسلہ وار جڑے ہوتے ہیں جب دوسری مزاحمت میں وہی رو گزرے جو پہلی میں گزرتی ہو۔ایسے مزاحمتوں کا ایک سرا آپس میں جڑا ہوتا ہے جبکہ ان کا دوسرا سرا آپس میں نہیں جڑا ہوتا۔یوں \عددی{\SI{1}{\kilo\ohm}} کا دایاں سرا اور \عددی{\SI{3}{\kilo\ohm}} کا نچلا سرا \عددی{H} پر آپس میں جڑے ہیں جبکہ \عددی{\SI{1}{\kilo\ohm}} کا بایاں سرا اور \عددی{\SI{3}{\kilo\ohm}} کا بالائی سرا آپس میں نہیں جڑے ہیں۔یوں ان مزاحمتوں کا مجموعی مزاحمت مساوات \حوالہ{مساوات_مزاحمتی_سلسلہ_وار_دوبارہ} سے درج ذیل حاصل ہوتا ہے۔
\begin{align*}
R_{FGH}=3000+1000=\SI{4}{\kilo\ohm}
\end{align*}
ان سلسلہ وار جڑے مزاحمتوں کی جگہ ان کا مساوی مزاحمت \عددی{R_{FGH}} نسب کرتے ہوئے شکل \حوالہ{مشق_مزاحمتی_سلسلہ_وار_متوازی_مزاحمت_الف} حاصل ہوتا ہے۔اس شکل میں \عددی{F} اور \عددی{G} نقطوں کے مابین \عددی{\SI{6}{\kilo\ohm}} اور \عددی{\SI{4}{\kilo\ohm}} متوازی جڑے ہیں۔متوازی جڑے مزاحمتوں پر یکساں دباو پایا جاتا ہے۔ یوں ان متوازی جڑے مزاحمتوں کا مساوی مزاحمت مساوات \حوالہ{مساوات_مزاحمتی_متوازی_دوبارہ} سے حاصل ہو گا یعنی 
\begin{align*}
\frac{1}{R_{FG}}=\frac{1}{6000}+\frac{1}{4000}=\frac{1}{2400}
\end{align*}
یا
\begin{align*}
R_{FG}=\SI{2.4}{\kilo\ohm}
\end{align*}

\begin{figure}
\centering
\includegraphics{figResistancesMixedSeriesParallelB}
\caption{}
\label{مشق_مزاحمتی_سلسلہ_وار_متوازی_مزاحمت_الف}
\end{figure}

نقطہ \عددی{F} اور نقطہ \عددی{G} کے درمیان مساوی مزاحمت نسب کرنے سے شکل \حوالہ{مشق_مزاحمتی_سلسلہ_وار_متوازی_مزاحمت_ب} حاصل ہوتا ہے۔اب آپ دیکھ سکتے ہیں کہ \عددی{EGF} پر \عددی{\SI{0.6}{\kilo\ohm}} اور \عددی{\SI{2.4}{\kilo\ohm}} سلسلہ وار جڑے ہیں جن کا مساوی مزاحمت
\begin{align*}
R_{EGF}=600+2400=\SI{3}{\kilo\ohm}
\end{align*}
ہو گا۔
\begin{figure}
\centering
\includegraphics{figResistancesMixedSeriesParallelC}
\caption{}
\label{مشق_مزاحمتی_سلسلہ_وار_متوازی_مزاحمت_ب}
\end{figure}

\عددی{R_{EGF}} کے استعمال سے شکل  \حوالہ{مشق_مزاحمتی_سلسلہ_وار_متوازی_مزاحمت_پ} حاصل ہوتا ہے جس میں \عددی{E} اور \عددی{F} کے درمیان دو عدد \عددی{\SI{3}{\kilo\ohm}} مزاحمت متوازی جڑے ہیں جن کا مساوی مزاحمت
\begin{align*}
\frac{1}{R_{EF}}=\frac{1}{3000}+\frac{1}{3000}=\frac{1}{1500}
\end{align*}
یعنی
\begin{align*}
R_{EF}=\SI{1.5}{\kilo\ohm}
\end{align*}
حاصل ہوتا ہے۔
\begin{figure}
\centering
\includegraphics{figResistancesMixedSeriesParallelD}
\caption{}
\label{مشق_مزاحمتی_سلسلہ_وار_متوازی_مزاحمت_پ}
\end{figure}
یوں شکل \حوالہ{مشق_مزاحمتی_سلسلہ_وار_متوازی_مزاحمت_ت}-الف حاصل ہوتا ہے۔
\begin{figure}
\centering
\begin{subfigure}{0.5\textwidth}
\centering
\includegraphics{figResistancesMixedSeriesParallelE}
\caption*{(الف)}
\end{subfigure}%
\begin{subfigure}{0.5\textwidth}
\centering
\includegraphics{figResistancesMixedSeriesParallelF}
\caption*{(ب)}
\end{subfigure}
%
\begin{subfigure}{0.5\textwidth}
\centering
\includegraphics{figResistancesMixedSeriesParallelFb}
\caption*{(پ)}
\end{subfigure}%
\begin{subfigure}{0.5\textwidth}
\centering
\includegraphics{figResistancesMixedSeriesParallelG}
\caption*{(ت)}
\end{subfigure}
%
\begin{subfigure}{0.5\textwidth}
\centering
\includegraphics{figResistancesMixedSeriesParallelH}
\caption*{(ٹ)}
\end{subfigure}%
\begin{subfigure}{0.5\textwidth}
\centering
\includegraphics{figResistancesMixedSeriesParallelI}
\caption*{(ث)}
\end{subfigure}%
\caption{}
\label{مشق_مزاحمتی_سلسلہ_وار_متوازی_مزاحمت_ت}
\end{figure}
%============================
اسی طریقے سے آگے بڑھتے ہوئے آخر کار شکل \حوالہ{مشق_مزاحمتی_سلسلہ_وار_متوازی_مزاحمت_ت}-ٹ حاصل ہوتا ہے جس سے \عددی{R_{AB}} درج ذیل حاصل ہوتا ہے
\begin{align*}
R_{AB}=2\frac{2}{3}\, \si{\kilo\ohm}
\end{align*}
یوں شکل \حوالہ{مشق_مزاحمتی_سلسلہ_وار_متوازی_مزاحمت}  کو حل کرتے کرتے آخر کار  شکل \حوالہ{مشق_مزاحمتی_سلسلہ_وار_متوازی_مزاحمت_ت}-ث حاصل کیا گیا جو مساوی مزاحمت  دیتا  ہے۔

%====================
\ابتدا{مشق}
شکل \حوالہ{شکل_مزاحمتی_مشق_متعدد_مزاحمت} میں \عددی{R_{AB}} دریافت کریں۔
\begin{figure}
\centering
\includegraphics{figResistancesQuizMixedSeriesParallelA}
\caption{متعدد سلسلہ وار اور متوازی مزاحمت کا  دور۔}
\label{شکل_مزاحمتی_مشق_متعدد_مزاحمت}
\end{figure}

جواب:\عددی{R_{AB}=\SI{5}{\kilo\ohm}}
\انتہا{مشق}
%====================

متعدد سلسلہ وار اور متوازی مزاحمتوں کا مساوی مزاحمت حاصل کرتے وقت درج ذیل طریقہ کار اختیار کیا جاتا ہے۔
\begin{itemize}
\item
داخلی برقی سروں سے دور ترین  مزاحمت سے شروع کریں۔
\item
دو عدد سلسلہ وار مزاحمت کی جگہ ان کا مساوی مزاحمت \عددی{R_s=R_1+R_2} نسب کریں۔جس جوڑ پر سلسلہ وار مزاحمت آپس میں جڑے ہوں اس جوڑ پر کوئی تیسرا پرزہ نہیں جڑا ہو سکتا۔یوں پہلے مزاحمت سے گزرتی رو دوسری مزاحمت سے بھی گزرتی ہے۔اگر جوڑ پر تیسرا پرزہ بھی نسب ہو تب مزاحمتوں کو سلسلہ وار جڑا تصور نہیں کیا جا سکتا۔
\item
دو عدد متوازی جڑے مزاحمتوں کی جگہ ان کا مساوی مزاحمت \عددی{R_m=\tfrac{R_1R_2}{R_1+R_2}} نسب کریں۔جن دو جوڑ کے ساتھ پہلا مزاحمت جڑا ہوتا ہے اگر انہیں جوڑ کے ساتھ دوسرا مزاحمت بھی جڑا ہو تب ان مزاحمتوں کو متوازی جڑا تصور کیا جاتا ہے۔متوازی مزاحمتوں پر برابر دباو پایا جاتا ہے۔
\item
متواتر سلسلہ وار اور متوازی مزاحمتوں کی جگہ ان کا مساوی مزاحمت نسب کرتے ہوئے دور کے داخلی سروں تک پہنچ کر پورے دور کا مساوی مزاحمت حاصل کریں۔
\end{itemize}
%==================

\حصہ{تخصیص مزاحمت}
جدول \حوالہ{جدول_مزاحمتی_معیاری_مزاحمت_قیمتیں} مزاحمت کی وہ مخصوص  قیمتیں دیتا ہے جو عام دستیاب ہیں۔مزاحمت کی قیمت کے علاوہ اس کی \اصطلاح{طاقتی استعداد}\فرہنگ{طاقت!استعداد}\فرہنگ{استعداد!طاقت}\حاشیہب{power rating}\فرہنگ{power!rating}\فرہنگ{rating!power} اور قیمت میں \اصطلاح{خلل}\فرہنگ{خلل}\حاشیہب{tolerance}\فرہنگ{tolerance} بھی جاننا ضروری ہے۔اس جدول میں دئے تمام مزاحمتوں کی قیمتوں میں \عددی{\SI{5}{\percent}} مزاحمتی خلل ممکن ہے۔یوں انہیں \عددی{\SI{5}{\percent}} مزاحمت کہتے ہیں۔مزاحمت کی طاقتی استعداد عموماً \عددی{\SI{0.25}{\watt}}، \عددی{\SI{0.5}{\watt}}، \عددی{\SI{1}{\watt}}، \عددی{\SI{2}{\watt}} وغیرہ ہوتی ہے۔اس کے علاوہ زیادہ طاقت کے  مخصوص مزاحمت بھی دستیاب ہیں۔

مزاحمت میں طاقت کا ضیاع حرارتی توانائی میں تبدیل ہوتا ہے جس سے مزاحمت کی درجہ حرارت بڑھتی ہے۔دو اجسام  کے مابین \اصطلاح{ایصال حرارت}\فرہنگ{ایصال حرارت}\فرہنگ{حرارت!ایصال}\حاشیہب{heat conduction}\فرہنگ{conduction!heat} یا \اصطلاح{اتصال حرارت}\فرہنگ{اتصال حرارت}\فرہنگ{حرارت!اتصال}\حاشیہب{heat convection}\فرہنگ{heat convection} کا دارومدار ان کے درجہ حرارت میں فرق پر منحصر ہے۔دو اجسام کے درجہ حرارت میں فرق بڑھانے سے ان کے مابین ایصال حرارت یا اتصال حرارت بڑھتی ہے۔  مزاحمت میں طاقت کے ضیاع سے مزاحمت کا درجہ حرارت ارد گرد کے ماحول  سے بڑھ جاتا ہے۔ایصال حرارت اور اتصال حرارت سے مزاحمت کی حرارتی توانائی ارد گرد کے ماحول کو منتقل ہوتی ہے۔جس درجہ حرارت پر مزاحمت کی طاقتی ضیاع اور مزاحمت سے انتقال حرارت برابر ہوں، مزاحمت کا درجہ حرارت اسی حتمی قیمت پر جا رکھتا ہے۔ ہر شے کسی مخصوص درجہ حرارت پر تباہ ہوتا ہے۔یہی مزاحمت کے لئے بھی درست ہے لہٰذا یہ ضروری ہے کہ اس کا درجہ حرارت اتنا نہ بڑھ جائے کہ مزاحمت جل کر راکھ ہو جائے۔طاقتی استعداد سے مراد وہ طاقت ہے جس پر مزاحمت محفوظ رہ سکتا ہے۔اگر طاقتی ضیاع مزاحمت کے طاقتی استعداد سے بڑھ جائے تو مزاحمت جل کر تباہ ہو جاتا ہے۔

\begin{table}
\caption{مزاحمت کے معیاری قیمتیں۔ قیمتوں میں \عددیء{\SI{5}{\percent}} خلل ممکن ہے۔}
\centering
\begin{tabular}{l l l l l l l}
$\SI{1.0}{\ohm} $& $\SI{10}{\ohm}$ & $\SI{100}{\ohm}$ & $\SI{1.0}{\kilo\ohm}$ &$\SI{10}{\kilo\ohm} $& $\SI{100}{\kilo\ohm}$ & $\SI{1.0}{\mega\ohm}$\\
$\SI{1.1}{\ohm} $& $\SI{11}{\ohm}$ & $\SI{110}{\ohm}$ & $\SI{1.1}{\kilo\ohm}$ &$\SI{11}{\kilo\ohm} $& $\SI{110}{\kilo\ohm}$ & $\SI{1.1}{\mega\ohm}$\\
$\SI{1.2}{\ohm} $& $\SI{12}{\ohm}$ & $\SI{120}{\ohm}$ & $\SI{1.2}{\kilo\ohm}$ &$\SI{12}{\kilo\ohm} $& $\SI{120}{\kilo\ohm}$ & $\SI{1.2}{\mega\ohm}$\\
$\SI{1.3}{\ohm} $& $\SI{13}{\ohm}$ & $\SI{130}{\ohm}$ & $\SI{1.3}{\kilo\ohm}$ &$\SI{13}{\kilo\ohm} $& $\SI{130}{\kilo\ohm}$ & $\SI{1.3}{\mega\ohm}$\\
$\SI{1.5}{\ohm} $& $\SI{15}{\ohm}$ & $\SI{150}{\ohm}$ & $\SI{1.5}{\kilo\ohm}$ &$\SI{15}{\kilo\ohm} $& $\SI{150}{\kilo\ohm}$ & $\SI{1.5}{\mega\ohm}$\\
$\SI{1.6}{\ohm} $& $\SI{16}{\ohm}$ & $\SI{160}{\ohm}$ & $\SI{1.6}{\kilo\ohm}$ &$\SI{16}{\kilo\ohm} $& $\SI{160}{\kilo\ohm}$ & $\SI{1.6}{\mega\ohm}$\\
$\SI{1.8}{\ohm} $& $\SI{18}{\ohm}$ & $\SI{180}{\ohm}$ & $\SI{1.8}{\kilo\ohm}$ &$\SI{18}{\kilo\ohm} $& $\SI{180}{\kilo\ohm}$ & $\SI{1.8}{\mega\ohm}$\\
$\SI{2.0}{\ohm} $& $\SI{20}{\ohm}$ & $\SI{200}{\ohm}$ & $\SI{2.0}{\kilo\ohm}$ &$\SI{20}{\kilo\ohm} $& $\SI{200}{\kilo\ohm}$ & $\SI{2.0}{\mega\ohm}$\\
$\SI{2.2}{\ohm} $& $\SI{22}{\ohm}$ & $\SI{220}{\ohm}$ & $\SI{2.2}{\kilo\ohm}$ &$\SI{22}{\kilo\ohm} $& $\SI{220}{\kilo\ohm}$ & $\SI{2.2}{\mega\ohm}$\\
$\SI{2.4}{\ohm} $& $\SI{24}{\ohm}$ & $\SI{240}{\ohm}$ & $\SI{2.4}{\kilo\ohm}$ &$\SI{24}{\kilo\ohm} $& $\SI{240}{\kilo\ohm}$ & $\SI{2.4}{\mega\ohm}$\\
$\SI{2.7}{\ohm} $& $\SI{27}{\ohm}$ & $\SI{270}{\ohm}$ & $\SI{2.7}{\kilo\ohm}$ &$\SI{27}{\kilo\ohm} $& $\SI{270}{\kilo\ohm}$ & $\SI{2.7}{\mega\ohm}$\\
$\SI{3.0}{\ohm} $& $\SI{30}{\ohm}$ & $\SI{300}{\ohm}$ & $\SI{3.0}{\kilo\ohm}$ &$\SI{30}{\kilo\ohm} $& $\SI{300}{\kilo\ohm}$ & $\SI{3.0}{\mega\ohm}$\\
$\SI{3.3}{\ohm} $& $\SI{33}{\ohm}$ & $\SI{330}{\ohm}$ & $\SI{3.3}{\kilo\ohm}$ &$\SI{33}{\kilo\ohm} $& $\SI{330}{\kilo\ohm}$ & $\SI{3.3}{\mega\ohm}$\\
$\SI{3.6}{\ohm} $& $\SI{36}{\ohm}$ & $\SI{360}{\ohm}$ & $\SI{3.6}{\kilo\ohm}$ &$\SI{36}{\kilo\ohm} $& $\SI{360}{\kilo\ohm}$ & $\SI{3.6}{\mega\ohm}$\\
$\SI{3.9}{\ohm} $& $\SI{39}{\ohm}$ & $\SI{390}{\ohm}$ & $\SI{3.9}{\kilo\ohm}$ &$\SI{39}{\kilo\ohm} $& $\SI{390}{\kilo\ohm}$ & $\SI{3.9}{\mega\ohm}$\\
$\SI{4.3}{\ohm} $& $\SI{43}{\ohm}$ & $\SI{430}{\ohm}$ & $\SI{4.3}{\kilo\ohm}$ &$\SI{43}{\kilo\ohm} $& $\SI{430}{\kilo\ohm}$ & $\SI{4.3}{\mega\ohm}$\\
$\SI{4.7}{\ohm} $& $\SI{47}{\ohm}$ & $\SI{470}{\ohm}$ & $\SI{4.7}{\kilo\ohm}$ &$\SI{47}{\kilo\ohm} $& $\SI{470}{\kilo\ohm}$ & $\SI{4.7}{\mega\ohm}$\\
$\SI{5.1}{\ohm} $& $\SI{51}{\ohm}$ & $\SI{510}{\ohm}$ & $\SI{5.1}{\kilo\ohm}$ &$\SI{51}{\kilo\ohm} $& $\SI{510}{\kilo\ohm}$ & $\SI{5.1}{\mega\ohm}$\\
$\SI{5.6}{\ohm} $& $\SI{56}{\ohm}$ & $\SI{560}{\ohm}$ & $\SI{5.6}{\kilo\ohm}$ &$\SI{56}{\kilo\ohm} $& $\SI{560}{\kilo\ohm}$ & $\SI{5.6}{\mega\ohm}$\\
$\SI{6.2}{\ohm} $& $\SI{62}{\ohm}$ & $\SI{620}{\ohm}$ & $\SI{6.2}{\kilo\ohm}$ &$\SI{62}{\kilo\ohm} $& $\SI{620}{\kilo\ohm}$ & $\SI{6.2}{\mega\ohm}$\\
$\SI{6.8}{\ohm} $& $\SI{68}{\ohm}$ & $\SI{680}{\ohm}$ & $\SI{6.8}{\kilo\ohm}$ &$\SI{68}{\kilo\ohm} $& $\SI{680}{\kilo\ohm}$ & $\SI{6.8}{\mega\ohm}$\\
$\SI{7.5}{\ohm} $& $\SI{75}{\ohm}$ & $\SI{750}{\ohm}$ & $\SI{7.5}{\kilo\ohm}$ &$\SI{75}{\kilo\ohm} $& $\SI{750}{\kilo\ohm}$ & $\SI{7.5}{\mega\ohm}$\\
$\SI{8.2}{\ohm} $& $\SI{82}{\ohm}$ & $\SI{820}{\ohm}$ & $\SI{8.2}{\kilo\ohm}$ &$\SI{82}{\kilo\ohm} $& $\SI{820}{\kilo\ohm}$ & $\SI{8.2}{\mega\ohm}$\\
$\SI{9.1}{\ohm} $& $\SI{91}{\ohm}$ & $\SI{910}{\ohm}$ & $\SI{9.1}{\kilo\ohm}$ &$\SI{91}{\kilo\ohm} $& $\SI{910}{\kilo\ohm}$ & $\SI{9.1}{\mega\ohm}$
\end{tabular}
\label{جدول_مزاحمتی_معیاری_مزاحمت_قیمتیں}
\end{table}
%===========

\FloatBarrier
\ابتدا{مثال}
شکل \حوالہ{مثال_مزاحمتی_خلل_طاقتی_ضیاع} میں \عددی{\SI{5}{\percent}} مزاحمت استعمال کیا گیا ہے۔دور میں کم سے کم اور زیادہ سے زیادہ رو دریافت کریں۔دونوں صورتوں میں مزاحمتی ضیاع بھی حاصل کریں۔
\begin{figure}
\centering
\includegraphics{figResistancesRating}
\caption{مزاحمت کی قیمت میں خلل اور طاقت کے ضیاع کی مثال۔}
\label{مثال_مزاحمتی_خلل_طاقتی_ضیاع}
\end{figure}

حل:مزاحمت کی  قیمت \عددی{\SI{9.1}{\kilo\ohm}} ہے۔اس قیمت کو \اصطلاح{علامتی قیمت}\فرہنگ{علامتی قیمت}\فرہنگ{قیمت!علامتی}\حاشیہب{typical value}\فرہنگ{value!typical} کہتے ہیں۔مزاحمت کی حقیقی قیمت اس سے \عددیء{\SI{5}{\percent}} کم یا زیادہ ممکن ہے۔یوں اس مزاحمت کی کم سے کم قیمت
\begin{align*}
R_{\text{کمتر}}=(1-0.05)\times 9100=\SI{8.645}{\kilo\ohm}
\end{align*}
اور زیادہ سے زیادہ قیمت
\begin{align*}
R_{\text{بلندتر}}=(1+0.05)\times 9100=\SI{9.555}{\kilo\ohm}
\end{align*}
ہو سکتی ہے۔مزاحمت کی اصل قیمت ان حدود کے درمیان رہے گی۔یوں کمتر اور بلند تر  رو درج ذیل ہوں گے۔
\begin{align*}
i_{\text{\RL{کمتر}}}&=\frac{20}{9555}=\SI{2.093}{\milli\ampere}\\
i_{\text{\RL{بلندتر}}}&=\frac{20}{8645}=\SI{2.313}{\milli\ampere}
\end{align*}
مزاحمت میں کمتر اور بلند تر طاقت کا ضیاع درج ذیل ہو گا۔
\begin{align*}
p_{\text{\RL{کمتر}}}&=20 \times 2.093 \times 10^{-3}=\SI{41.86}{\milli\watt}\\
p_{\text{\RL{بلندتر}}}&=20 \times 2.313 \times 10^{-3}=\SI{46.26}{\milli\watt}
\end{align*} 
مزاحمت میں طاقت کا ضیاع \عددی{\SI{42}{\milli\watt}} تا \عددی{\SI{46}{\milli\watt}} ممکن ہے۔یوں \عددی{\SI{0.25}{\watt}} کی مزاحمت یہاں استعمال کی جا سکتی ہے جو \عددی{\SI{250}{\milli\watt}} کی طاقتی ضیاع کو برداشت کرنے کی صلاحیت رکھتی ہے۔
\انتہا{مثال}
%============================

مندرجہ بالا مثال میں اگر مزاحمت کی قیمت \عددی{\SI{100}{\ohm}} ہوتی تب رو کی علامتی قیمت \عددی{\tfrac{20}{100}=\SI{0.2}{\ampere}} ہوتی اور مزاحمت ضیاع \عددی{\SI{4}{\watt}} ہوتا۔مزاحمت کی استعداد \عددیء{\SI{0.25}{\watt}} ہونے کی صورت میں مزاحمت تاب نہ لاتے ہوئے جل کر راکھ ہو جائے گا۔یوں ایسی صورت میں \عددی{\SI{4}{\watt}} سے زیادہ طاقتی استعداد\حاشیہد{میں متوقع طاقتی ضیاع کی دگنا قیمت کے طاقتی استعداد کا مزاحمت استعمال کرتا ہوں۔} کا مزاحمت استعمال کرنا ضروری ہے۔

\حصہ{سلسلہ وار اور متوازی مزاحمتوں کے ادوار کا حل}
قانون اوہم اور کرخوف کے قوانین کو بطور تجزیاتی آلات استعمال کرتے ہوئے برقی ادوار حل کئے جاتے ہیں۔اب تک ہم سادہ ترین ادوار حل کرتے رہے ہیں۔اس حصے میں سلسلہ وار اور متوازی مزاحمتوں پر مبنی بڑے ادوار حل کرنا دیکھتے ہیں۔ 

\FloatBarrier

\ابتدا{مثال}
شکل \حوالہ{شکل_مزاحمتی_سلسلہ_وار_متوازی_دور_حل_مثال_الف}-الف کے دور میں تمام نا معلوم دباو اور رو دریافت کریں۔
\begin{figure}
\centering
\begin{subfigure}{\textwidth}
\centering
\includegraphics{figResistancesMixedSolutionExampleA}
\caption*{(الف)}
\end{subfigure}
%
\begin{subfigure}{0.6\textwidth}
\centering
\includegraphics{figResistancesMixedSolutionExampleB}
\caption*{(ب)}
\end{subfigure}%
\begin{subfigure}{0.4\textwidth}
\centering
\includegraphics{figResistancesMixedSolutionExampleC}
\caption*{(پ)}
\end{subfigure}%
\caption{سلسلہ وار اور متوازی مزاحمتوں کے دور کی مثال۔}
\label{شکل_مزاحمتی_سلسلہ_وار_متوازی_دور_حل_مثال_الف}
\end{figure}

حل:ہم منبع سے دور ترین مزاحمت سے شروع کرتے ہوئے  سلسلہ وار اور متوازی مزاحمتوں کی جگہ ان کا مساوی مزاحمت پر کرتے ہوئے آخر کار شکل \حوالہ{شکل_مزاحمتی_سلسلہ_وار_متوازی_دور_حل_مثال_الف}-پ تک پہنچتے ہیں جہاں سے \عددی{i_1} اور \عددی{v_a} کیا جا سکتا ہے۔ان قیمتوں کو کرخوف کے قوانین اور قانون اوہم کے ساتھ استعمال کرتے ہوئے مزید نا معلوم متغیرات حاصل کئے جائیں گے۔آئیں یہ عمل قدم با قدم دیکھیں۔ 

شکل-الف میں منبع سے دور ترین \عددی{\SI{9}{\kilo\ohm}} اور \عددی{\SI{3}{\kilo\ohm}} سلسلہ وار جڑے ہیں۔ان کا مساوی
 مزاحمت \عددی{\SI{9}{\kilo\ohm}+\SI{3}{\kilo\ohm}=\SI{12}{\kilo\ohm}} ہے جو \عددی{\SI{4}{\kilo\ohm}} کے متوازی ہے۔یوں ان کا مساوی مزاحمت \عددی{\tfrac{\SI{4}{\kilo\ohm} \times \SI{12}{\kilo\ohm}}{\SI{4}{\kilo\ohm} + \SI{12}{\kilo\ohm}}=\SI{3}{\kilo\ohm}} ہو گا جسے شکل-ب میں استعمال کیا گیا ہے۔شکل-ب میں \عددی{\SI{6}{\kilo\ohm}} اور \عددی{\SI{3}{\kilo\ohm}} کا مساوی \عددی{\SI{9}{\kilo\ohm}} ہے جو از خود \عددی{\SI{18}{\kilo\ohm}} کے متوازی ہے۔یوں ان کا مساوی \عددی{\SI{6}{\kilo\ohm}} ہو گا جس کے استعمال سے شکل-پ حاصل ہوتا ہے۔

شکل \حوالہ{شکل_مزاحمتی_سلسلہ_وار_متوازی_دور_حل_مثال_الف}-پ میں
\begin{align*}
i_1&=\frac{20}{4000+6000}=\SI{2}{\milli\ampere}
\end{align*}
حاصل ہوتا ہے جس سے قانون اوہم کے تحت
\begin{align*}
v_a=i_1 \times \SI{6}{\kilo\ohm}=\SI{12}{\volt}
\end{align*}
حاصل ہوتا ہے۔شکل-ب میں ان قیمتوں کو دکھایا گیا ہے جہاں سے ظاہر ہے کہ \عددی{\SI{18}{\kilo\ohm}} مزاحمت پر \عددی{\SI{12}{\volt}} دباو ہے لہٰذا اس کی رو
\begin{align*}
i_2=\frac{v_a}{\SI{18}{\kilo\ohm}}=\frac{12}{18000}=\frac{2}{3} \, \si{\milli\ampere}
\end{align*}
ہو گی۔شکل-الف میں قانون رو سے 
\begin{align*}
i_1=i_2+i_3
\end{align*}
لکھتے ہوئے
\begin{align*}
i_3&=i_1-i_2\\
&=\SI{2}{\milli\ampere}-\frac{2}{3}\, \si{\milli\ampere}\\
&=\frac{4}{3}\, \si{\milli\ampere}
\end{align*}
حاصل ہوتا ہے۔شکل-ب میں \عددی{i_3} کے استعمال سے
\begin{align*}
v_b&=i_3 \times \SI{3}{\kilo\ohm}\\
&=\frac{4}{3}\times 10^{-3} \times 3000\\
&=\SI{4}{\volt}
\end{align*}
حاصل ہوتا ہے۔اب شکل-الف میں \عددی{v_b} جانتے ہوئے \عددی{i_4} حاصل کرتے ہیں۔
\begin{align*}
i_4&=\frac{v_b}{\SI{4}{\kilo\ohm}}\\
&=\frac{4}{4000}\\
&=\SI{1}{\milli\ampere}
\end{align*}
قانون رو سے
\begin{align*}
i_3&=i_4+i_5
\end{align*}
لکھتے ہوئے
\begin{align*}
i_5&=i_3-i_4\\
&=\frac{4}{3}\, \si{\milli\ampere}-\SI{1}{\milli\ampere}\\
&=\frac{1}{3} \, \si{\milli\ampere}
\end{align*}
حاصل ہوتا ہے جسے استعمال کرتے ہوئے قانون اوہم سے
\begin{align*}
v_c&=i_5 \times \SI{9}{\kilo\ohm}\\
&=\frac{1}{3} \times 10^{-3} \times 9000\\
&=\SI{3}{\volt}
\end{align*}
لکھا جا سکتا ہے۔

\انتہا{مثال}
%===================

\ابتدا{مثال}
شکل \حوالہ{شکل_مزاحمتی_دور_کا_حل} میں \عددی{i_5=\SI{2}{\milli\ampere}} ہونے کی صورت میں تمام نا معلوم متغیرات دریافت کریں۔

\begin{figure}
\centering
\includegraphics{figResistancesMixedSolutionExampleD}
\caption{سلسلہ وار اور متوازی مزاحمتوں کا دور۔}
\label{شکل_مزاحمتی_دور_کا_حل}
\end{figure}

حل:یہ مثال گزشتہ مثال کے الٹ ہے۔یہاں دور میں کسی ایک مقام کے رو (یا دباو) سے منبع کی دباو اور دیگر متغیرات دریافت کیے جائیں گے۔دی معلومات سے قانون اوہم کے ذریعہ
\begin{align*}
v_{cd}&=i_5 \times \SI{4}{\kilo\ohm}\\
&=2\times 10^{-3}\times 4000\\
&=\SI{8}{\volt}
\end{align*}
لکھا جا سکتا ہے جسے استعمال کرتے ہوئے قانون اوہم کی مدد سے
\begin{align*}
i_4&=\frac{v_{cd}}{\SI{2}{\kilo\ohm}}\\
&=\frac{8}{2000}\\
&=\SI{4}{\milli\ampere}
\end{align*}
حاصل ہوتا ہے۔کرخوف قانون رو 
\begin{align*}
i_3&=i_4+i_5\\
&=\SI{4}{\milli\ampere}+\SI{2}{\milli\ampere}\\
&=\SI{6}{\milli\ampere}
\end{align*}
حاصل ہوتا ہے۔یوں قانون اوہم سے
\begin{align*}
v_{bc}&=i_3 \times \SI{1}{\kilo\ohm}\\
&=6\times 10^{-3} \times 1000\\
&=\SI{6}{\volt}
\end{align*}
حاصل ہوتا ہے۔دائرہ \عددی{dcbfg} پر کرخوف قانون دباو
\begin{align*}
V_{cd}+V_{bc}=i_2 \times \SI{3}{\kilo\ohm}+i_2\times \SI{4}{\kilo\ohm}
\end{align*}
لکھا جائے گا جس سے
\begin{align*}
i_2&=\frac{V_{cd}+V_{bc}}{\SI{3}{\kilo\ohm}+\SI{4}{\kilo\ohm}}\\
&=\frac{8+6}{3000+4000}\\
&=\SI{2}{\milli\ampere}
\end{align*}
حاصل ہوتا ہے۔کرخوف قانون رو سے
\begin{align*}
i_1&=i_2+i_3\\
&=\SI{2}{\milli\ampere}+\SI{6}{\milli\ampere}\\
&=\SI{8}{\milli\ampere}
\end{align*}
حاصل ہوتا ہے جسے قانون اوہم میں استعمال کرتے ہوئے
\begin{align*}
V_{ab}&=i_1 \times \SI{5}{\kilo\ohm}\\
&=8\times 10^{-3}\times 5000\\
&=\SI{40}{\volt}
\end{align*}
حاصل ہوتا ہے جہاں \عددی{V_{ab}} نقطہ \عددی{b} کے حوالے سے نقطہ \عددی{a} پر دباو ہے۔دائرہ \عددی{eabcd} پر کرخوف قانون دباو
\begin{align*}
V_0&=V_{ab}+V_{bc}+V_{cd}
\end{align*}
لکھا جائے گا جس سے منبع کا دباو
\begin{align*}
V_0&=40+6+8\\
&=\SI{54}{\volt}
\end{align*}
حاصل ہوتا ہے۔
\انتہا{مثال}
%=====================
\FloatBarrier

\ابتدا{مشق}\شناخت{مشق_مزاحمتی-متعدد_الف}
شکل \حوالہ{شکل_مزاحمتی_متعدد_مشق_الف}-الف میں \عددی{V_0} دریافت کریں۔
\begin{figure}
\centering
\begin{subfigure}{0.5\textwidth}
\centering
\includegraphics{figResistancesMixedSolutionQuizA}
\caption*{(الف)}
\end{subfigure}%
\begin{subfigure}{0.5\textwidth}
\centering
\includegraphics{figResistancesMixedSolutionQuizB}
\caption*{(ب)}
\end{subfigure}%
\caption{دور برائے مشق \حوالہ{مشق_مزاحمتی-متعدد_الف} اور مشق \حوالہ{مشق_مزاحمتی-متعدد_ب}}
\label{شکل_مزاحمتی_متعدد_مشق_الف}
\end{figure}

جواب:\عددی{\SI{3.667}{\volt}}
\انتہا{مشق}
%=====================

\ابتدا{مشق}\شناخت{مشق_مزاحمتی-متعدد_ب}
شکل \حوالہ{شکل_مزاحمتی_متعدد_مشق_الف}-ب میں \عددی{V_m} دریافت کریں۔  

جواب: \عددی{\SI{60}{\volt}}
\انتہا{مشق}
%=====================

\ابتدا{مشق}\شناخت{مشق_مزاحمتی-متعدد_پ}
شکل \حوالہ{شکل_مزاحمتی_متعدد_مشق_پ} میں \عددی{V_0} دریافت کریں۔
\begin{figure}
\centering
\includegraphics{figResistancesMixedSolutionQuizD}
\caption{دور برائے مشق \حوالہ{مشق_مزاحمتی-متعدد_پ}}
\label{شکل_مزاحمتی_متعدد_مشق_پ}
\end{figure}

جواب:\عددی{\SI{14.19}{\volt}}
\انتہا{مشق}
%=====================
\FloatBarrier

\ابتدا{مشق}\شناخت{مشق_مزاحمتی-متعدد_ت}
شکل \حوالہ{شکل_مزاحمتی_متعدد_مشق_ت} میں \عددی{V_0}  اور \عددی{I_0} دریافت کریں۔
\begin{figure}
\centering
\includegraphics{figResistancesMixedSolutionQuizE}
\caption{دور برائے مشق \حوالہ{مشق_مزاحمتی-متعدد_ت}}
\label{شکل_مزاحمتی_متعدد_مشق_ت}
\end{figure}

جوابات:\عددی{\SI{8.05}{\volt}}، \عددی{\SI{2.93}{\milli\ampere}}
\انتہا{مشق}
%=====================

\ابتدا{مشق}\شناخت{مشق_مزاحمتی-متعدد_ٹ}
شکل \حوالہ{شکل_مزاحمتی_متعدد_مشق_ٹ} میں \عددی{V_0}  اور \عددی{I_0} دریافت کریں۔
\begin{figure}
\centering
\includegraphics{figResistancesMixedSolutionQuizF}
\caption{دور برائے مشق \حوالہ{مشق_مزاحمتی-متعدد_ٹ}}
\label{شکل_مزاحمتی_متعدد_مشق_ٹ}
\end{figure}

جوابات:\عددی{\SI{-6.906}{\volt}}، \عددی{\SI{2.94}{\milli\ampere}}
\انتہا{مشق}
%=====================
\FloatBarrier
\حصہ{ستارہ-تکون تبادلہ}
ہم نے اب تک ایسے ادوار دیکھے جن میں سلسلہ وار مزاحمتوں اور متوازی مزاحمتوں کی جگہ مساوی مزاحمت نسب کرتے ہوئے سادہ دور حاصل کیا گیا۔اس حصے میں جس ترکیب پر غور کیا جائے گا، اس کی اہمیت شکل \حوالہ{شکل_مزاحمتی_ستارہ_تکون_الف} سے واضح ہو گی۔آپ اس دور میں \عددی{i} حاصل کرنے کی کوشش کریں۔آپ دیکھ سکتے ہیں کہ اس میں کوئی بھی دو مزاحمت سلسلہ وار یا متوازی نہیں جڑے لہٰذا اس دور کی سادہ صورت گزشتہ ترکیب سے حاصل نہیں کی جا سکتی۔کیا اچھا ہوتا اگر ایسی صورت میں دور کے کچھ حصے کی جگہ متبادل دور نسب کرتے ہوئے اسے قابل حل بنانا ممکن ہوتا۔خوش قسمتی سے ایسا کرنا ممکن ہے۔ اس ترکیب کو \اصطلاح{ستارہ-تکونی تبادلہ}\فرہنگ{ستارہ-تکون تبادلہ}\فرہنگ{تبادلہ!ستارہ-تکون}\حاشیہب{wye-delta transformation}\فرہنگ{wye-delta transformation} یا \عددی{Y-\Delta} تبادلہ کہتے ہیں۔آئیں ستارہ-تکون تبادلہ کے  ترکیب پر غور کریں۔ 
\begin{figure}
\centering
\includegraphics{figResistancesYDeltaA}
\caption{اس دور کو سلسلہ وار اور متوازی مزاحمتوں کی طرح حل نہیں کیا جا سکتا۔}
\label{شکل_مزاحمتی_ستارہ_تکون_الف}
\end{figure}


شکل \حوالہ{شکل_مزاحمتی_ستارہ_تکون_مبدل}-الف میں تین مزاحمت تکونی کی شکل \عددی{\Delta} میں جڑے ہیں جبکہ شکل-ب میں تین مزاحمت ستارہ کی شکل \عددی{Y} میں جڑے ہیں۔ہم ستارہ مزاحمت کی جگہ تکونی مزاحمت یا تکونی مزاحمت کی جگہ ستارہ مزاحمت اس صورت نسب کر سکتے ہیں جب اس تبدیلی سے بقایا دور پر کوئی اثر نہ پڑے۔یوں اگر کسی دور میں تین نقطوں \عددی{a}، \عددی{b} اور \عددی{c} کے درمیان تکونی مزاحمت (ستارہ مزاحمت) جڑے ہوں تب انہیں تین نقطوں پر مبدل ستارہ مزاحمت (تکونی مزاحمت) نسب کرنے سے بقایا دور میں کسی بھی مقام پر دباو اور رو میں تبدیلی رو نما نہیں ہونی چاہیے۔ایسا تب ممکن ہو گا کہ ان تین نقطوں پر بھی دباو اور رو میں کوئی تبدیلی نہ پیدا ہو یعنی \عددی{a-b} اور \عددی{a-c} اور \عددی{b-c}  کے درمیان مزاحمت میں تبدیلی نہیں پیدا ہونی چاہیے۔

\begin{figure}
\centering
\begin{subfigure}{0.5\textwidth}
\centering
\includegraphics{figResistancesYDeltaB}
\caption*{(الف) تکونی مزاحمت}
\end{subfigure}%
\begin{subfigure}{0.5\textwidth}
\centering
\includegraphics{figResistancesYDeltaC}
\caption*{(ب) ستارہ مزاحمت}
\end{subfigure}%
\caption{ستارہ-تکون مبدل}
\label{شکل_مزاحمتی_ستارہ_تکون_مبدل}
\end{figure}
ستارہ-تکونی تبادلہ \عددی{abc} کے ساتھ کسی بھی دور کے لئے کارآمد ہونا چاہیے۔یوں یہ تبادلہ اس صورت بھی کارآمد ہونا ضروری ہے جب \عددی{a} اور \عددی{b} دور کے ساتھ منسلک ہوں جبکہ \عددی{c} آزاد ہو اور کہیں نہ جڑا ہو۔ایسی صورت میں شکل  \حوالہ{شکل_مزاحمتی_ستارہ_تکون_مبدل}-الف سے \عددی{a-c} کی مزاحمت درج ذیل حاصل ہوتی ہے
\begin{align*}
R_{ab}=\frac{R_2(R_1+R_3)}{R_1+R_2+R_3}
\end{align*}
جبکہ شکل \حوالہ{شکل_مزاحمتی_ستارہ_تکون_مبدل}-ب سے \عددی{a-c} کی مزاحمت
\begin{align*}
R_{ab}=R_a+R_b
\end{align*}
حاصل ہوتی ہے۔مندرجہ بالا دونوں قیمت برابر ہونا ضروری ہے یعنی
\begin{align}\label{مساوات_مزاحمتی_تکونی_ستارہ_الف}
R_{ab}=\frac{R_2(R_1+R_3)}{R_1+R_2+R_3}=R_a+R_b
\end{align}
اسی طرح اگر \عددی{b} کہیں بھی نہ جڑا ہو تب دونوں اشکال سے \عددی{a-c} کی مزاحمت برابر پر کرنے سے
\begin{align}\label{مساوات_مزاحمتی_تکونی_ستارہ_ب}
R_{ac}=\frac{R_1(R_2+R_3)}{R_1+R_2+R_3}=R_a+R_c
\end{align}
حاصل ہوتا ہے۔اگر  \عددی{a} کہیں بھی نہ جڑا ہو تب دونوں اشکال سے \عددی{b-c} کی مزاحمت برابر پر کرنے سے
\begin{align}\label{مساوات_مزاحمتی_تکونی_ستارہ_پ}
R_{bc}=\frac{R_3(R_1+R_2)}{R_1+R_2+R_3}=R_b+R_c
\end{align}
حاصل ہوتا ہے۔مساوات \حوالہ{مساوات_مزاحمتی_تکونی_ستارہ_الف}، مساوات \حوالہ{مساوات_مزاحمتی_تکونی_ستارہ_ب} اور مساوات \حوالہ{مساوات_مزاحمتی_تکونی_ستارہ_پ} تین عدد مساوات ہیں جنہیں \عددی{R_a}، \عددی{R_b} اور \عددی{R_c} کے لئے حل کرنے سے درج ذیل حاصل ہوتے ہیں۔
\begin{gather}
\begin{aligned}\label{مساوات_مزاحمتی_تکون_سے_ستارہ}
R_a&=\frac{R_1 R_2}{R_1+R_2+R_3}\\
R_b&=\frac{R_2 R_3}{R_1+R_2+R_3}\\
R_c&=\frac{R_1 R_3}{R_1+R_2+R_3}
\end{aligned}
\end{gather}


اسی طرح مساوات \حوالہ{مساوات_مزاحمتی_تکونی_ستارہ_الف} تا مساوات \حوالہ{مساوات_مزاحمتی_تکونی_ستارہ_پ} کو \عددی{R_1}، \عددی{R_2} اور \عددی{R_3} کے لئے حل کرنے سے درج ذیل حاصل ہوتے ہیں۔
\begin{gather}
\begin{aligned}\label{مساوات_مزاحمتی_ستارہ_سے_تکون}
R_1&=\frac{R_a R_b + R_b R_c R_c R_a}{R_b}\\
R_2&=\frac{R_a R_b + R_b R_c R_c R_a}{R_c}\\
R_3&=\frac{R_a R_b + R_b R_c R_c R_a}{R_a}
\end{aligned}
\end{gather}
مساوات \حوالہ{مساوات_مزاحمتی_تکون_سے_ستارہ}  تکونی مزاحمت سے ستارہ مزاحمت کی قیمتیں  دیتا ہے جبکہ مساوات \حوالہ{مساوات_مزاحمتی_ستارہ_سے_تکون} ستارہ مزاحمت سے تکونی مزاحمت کی قیمتیں دیتا ہے۔
%==============

\ابتدا{مشق}
مساوات  \حوالہ{مساوات_مزاحمتی_تکون_سے_ستارہ} حاصل کریں۔
\انتہا{مشق}

%=====================

\ابتدا{مشق}
مساوات  \حوالہ{مساوات_مزاحمتی_ستارہ_سے_تکون} حاصل کریں۔
\انتہا{مشق}
%==========================

شکل \حوالہ{شکل_مزاحمتی_ستارہ_تکون_الف} کی مثال آگے بڑھاتے ہیں۔اسے شکل \حوالہ{شکل_مزاحمتی_تکون_کا_مبدل}-الف میں دوبارہ دکھایا گیا ہے جہاں تکون \عددی{abc} کی نشاندہی کرتے ہوئے \عددی{R_1}، \عددی{R_2} اور \عددی{R_3} کا مبدل ستارہ ہلکی سیاہی میں دکھایا گیا ہے۔ شکل \حوالہ{شکل_مزاحمتی_تکون_کا_مبدل}-ب میں تکون کی جگہ ستارہ نسب دکھایا گیا ہے جہاں سے واضح ہے کہ نیا دور قابل حل ہے۔نئی شکل میں مزاحمت \عددی{R_a}، \عددی{R_b} اور \عددی{R_c} ستارہ جڑے ہیں۔ 
\begin{figure}
\centering
\begin{subfigure}{0.5\textwidth}
\centering
\includegraphics{figResistancesDeltaReplacedWithY}
\caption*{(الف) بالائی تکون  کی جگہ ستارہ نسب کیا جا رہا ہے۔ستارہ کو ہلکی سیاہی میں دکھایا گیا ہے۔}
\end{subfigure}%
\begin{subfigure}{0.5\textwidth}
\centering
\includegraphics{figResistancesDeltaReplacedWithYb}
\caption*{(ب) تکون کی جگہ ستارہ نسب کرنے سے دور قابل حل ہو گیا ہے۔}
\end{subfigure}%
\caption{تکون-ستارہ تبادلہ۔}
\label{شکل_مزاحمتی_تکون_کا_مبدل}
\end{figure}

%=====================
\ابتدا{مثال}
شکل \حوالہ{شکل_مزاحمتی_ستارہ_تکون_الف} میں \عددی{i} حاصل کریں۔دیگر معلومات درج ذیل ہیں۔
\begin{align*}
&V_1=\SI{16}{\volt}, \quad R_1=\SI{10}{\kilo\ohm}, \quad  R_2=\SI{15}{\kilo\ohm}, \quad R_3=\SI{5}{\kilo\ohm}, \quad R_4=\frac{1}{3}\,\si{\kilo\ohm}, \quad R_5=\frac{1}{2}\, \si{\kilo\ohm}, \quad R_6=\SI{1.8}{\kilo\ohm}
\end{align*}

حل:مساوات \حوالہ{مساوات_مزاحمتی_تکون_سے_ستارہ} کی مدد سے ستارہ مزاحمت حاصل کرتے ہیں۔
\begin{align*}
R_a&=\frac{10000 \times 15000 }{10000+15000+5000}=\SI{5}{\kilo\ohm}\\
R_b&=\frac{15000\times 5000}{10000+15000+5000}=\frac{5}{2} \, \si{\kilo\ohm}\\
R_c&=\frac{10000\times 5000}{10000+15000+5000}=\frac{5}{3} \, \si{\kilo\ohm}
\end{align*}
ان قیمتوں کو شکل \حوالہ{شکل_مزاحمتی_تکون_کا_مبدل}-ب میں پُر کرتے ہیں۔ اب \عددی{R_c} اور \عددی{R_4} سلسلہ وار جڑے ہیں لہٰذا ان کا مساوی مزاحمت
\begin{align*}
R_{c4}=\frac{5}{3} \, \si{\kilo\ohm}+\frac{1}{3}\, \si{\kilo\ohm}=\SI{2}{\kilo\ohm}
\end{align*}
ہو گا۔اسی طرح \عددی{R_b} اور \عددی{R_5} سلسلہ وار جڑے ہیں جن کا مساوی مزاحمت 
\begin{align*}
R_{b5}=\frac{5}{2} \, \si{\kilo\ohm}+\frac{1}{2} \, \si{\kilo\ohm}=\SI{3}{\kilo\ohm}
\end{align*}
ہے۔یہ دو عدد مساوی مزاحمت آپس میں متوازی جڑے ہیں لہٰذا ان کا مساوی مزاحمت
\begin{align*}
R_m=\frac{2000 \times 3000}{2000+3000}=\SI{1.2}{\kilo\ohm}
\end{align*}
ہو گا جو \عددی{R_a} اور \عددی{R_6} کے ساتھ سلسلہ وار ہے لہٰذا برقی رو درج ذیل حاصل ہوتی ہے۔
\begin{align*}
i&=\frac{V_1}{R_6+R_a+R_m}\\
&=\frac{16}{1800+5000+1200}\\
&=\SI{2}{\milli\ampere}
\end{align*}

\انتہا{مثال}
%=======================

مساوات \حوالہ{مساوات_مزاحمتی_تکون_سے_ستارہ} اور مساوات \حوالہ{مساوات_مزاحمتی_ستارہ_سے_تکون} عمومی مساوات ہیں۔متوازن تکون  میں \عددی{R_1=R_2=R_3} ہو گا۔ایسی صورت میں مساوات \حوالہ{مساوات_مزاحمتی_تکون_سے_ستارہ} درج ذیل صورت اختیار کرتی ہیں۔
\begin{align}
R_Y=\frac{R_{\Delta}}{3}
\end{align}
اسی طرح متوازن ستارے میں \عددی{R_a=R_b=R_c} ہو گا۔ایسی صورت میں مساوات \حوالہ{مساوات_مزاحمتی_ستارہ_سے_تکون} درج ذیل صورت اختیار کرتی ہیں۔
\begin{align}
R_{\Delta}=3 R_Y
\end{align}
%=======================

\FloatBarrier
\ابتدا{مشق}\شناخت{مشق_مزاحمتی_تکونی_ستارہ_الف}
شکل \حوالہ{شکل_مزاحمتی_مشق_تکون_ستارہ_الف} میں تکون-ستارہ مبدل کی مدد سے \عددی{i} دریافت کریں۔
\begin{figure}
\centering
\includegraphics{figResistancesQuizYDeltaA}
\caption{مشق \حوالہ{مشق_مزاحمتی_تکونی_ستارہ_الف} کا دور۔}
\label{شکل_مزاحمتی_مشق_تکون_ستارہ_الف}
\end{figure}

جواب: \عددی{\SI{1.05778}{\milli\ampere}}
\انتہا{مشق}
%=======================
\FloatBarrier
\ابتدا{مشق}\شناخت{مشق_مزاحمتی_تکونی_ستارہ_ب}
شکل \حوالہ{شکل_مزاحمتی_مشق_تکون_ستارہ_ب}-الف میں تکون-ستارہ مبدل کی مدد سے \عددی{V_1} دریافت کریں۔
\begin{figure}
\centering
\begin{subfigure}{0.5\textwidth}
\centering
\includegraphics{figResistancesQuizYDeltaB}
\caption*{(الف)}
\end{subfigure}%
\begin{subfigure}{0.5\textwidth}
\centering
\includegraphics{figResistancesQuizYDeltaC}
\caption*{(ب)}
\end{subfigure}%
\caption{مشق \حوالہ{مشق_مزاحمتی_تکونی_ستارہ_ب} اور مشق \حوالہ{مشق_مزاحمتی_تکونی_ستارہ_پ} کا دور۔}
\label{شکل_مزاحمتی_مشق_تکون_ستارہ_ب}
\end{figure}

جواب: \عددی{\SI{5.103}{\volt}}
\انتہا{مشق}
%=======================
%=======================
\FloatBarrier
\ابتدا{مشق}\شناخت{مشق_مزاحمتی_تکونی_ستارہ_پ}
شکل \حوالہ{شکل_مزاحمتی_مشق_تکون_ستارہ_ب}-ب میں تکون-ستارہ مبدل کی مدد سے \عددی{V_1} دریافت کریں۔


جواب: \عددی{\SI{6.609}{\volt}}
\انتہا{مشق}
%=======================

\حصہ{تابع منبع استعمال کرتے ادوار}
تابع منبع استعمال کرتے ادوار \اصطلاح{برقیات}\فرہنگ{برقیات}\حاشیہب{electronics}\فرہنگ{electronics} کے میدان میں اہم کردار ادا کرتے ہیں جہاں دو \اصطلاح{جوڑ ٹرانزسٹر}\فرہنگ{ٹرانزسٹر!دو جوڑ}\حاشیہب{Bipolar Junction Transistor, BJT}\فرہنگ{transistor!BJT}،\اصطلاح{میدانی ٹرانزسٹر}\فرہنگ{ٹرانزسٹر!میدانی}\حاشیہب{Field Effect Transistor, FET}\فرہنگ{transistor!FET}،  \اصطلاح{ماسفیٹ}\فرہنگ{ماسفیٹ}\حاشیہب{Metal Oxide Semiconductor Field Effect Transistor, MOSFET}\فرہنگ{MOSFET} وغیرہ کے ریاضی نمونے تابع منبع کو استعمال کرتے ہوئے بنائے جاتے ہیں۔اس حصے میں تابع منبع استعمال کرنے والے سادہ ترین ادوار پر مثالوں کی مدد سے غور کیا جائے گا۔ تابع منبع استعمال کرتے ادوار حل کرنے کی ترکیب مندرجہ ذیل ہے۔

\begin{itemize}
\item
تابع منبع کو غیر تابع منبع تصور کرتے ہوئے درکار کرخوف مساوات لکھیں۔
\item
تابع منبع کی قابو مساوات لکھیں۔
\item
ان ہمزاد مساوات کو حل کریں۔یاد رہے کہ مساوات کی تعداد نا معلوم متغیرات کے برابر ہونا ضروری ہے۔  
\end{itemize}

%=========================
\ابتدا{مثال}\شناخت{مثال_مزاحمتی_تابع_مثال_الف}
شکل \حوالہ{شکل_مزاحمتی_مثال_تابع_منبع_الف} میں دباو سے قابو منبع دباو استعمال کیا گیا ہے۔ایسی تابع منبع کو \اصطلاح{دباو تابع منبع دباو}\فرہنگ{منبع دباو!دباو تابع}\حاشیہب{voltage controlled voltage source}\فرہنگ{voltage source!voltage controlled} کہتے ہیں۔اس دور میں  \عددی{i} اور \عددی{V_2} دریافت کریں۔
\begin{figure}
\centering
\includegraphics{figResistancesExampleControlledSourceA}
\caption{مثال \حوالہ{مثال_مزاحمتی_تابع_مثال_الف} کا دور۔}
\label{شکل_مزاحمتی_مثال_تابع_منبع_الف}
\end{figure}

حل:کرخوف قانون دباو سے
\begin{align*}
-20+1200 i-V_0+2000 i=0
\end{align*}
لکھتے ہیں۔تابع منبع کی قابو مساوات درج ذیل ہے۔
\begin{align*}
V_0=3V_1 = 3\times 1200 i
\end{align*}
مندرجہ بالا دو ہمزاد مساوات کو حل کرنے سے
\begin{align*}
i&=\SI{-50}{\milli\ampere}
\end{align*}
حاصل ہوتا ہے جسے استعمال کرتے ہوئے
\begin{align*}
V_2&=2000 \times (-50\times 10^{-3})\\
&=\SI{-100}{\volt}
\end{align*}
ملتا ہے۔
\انتہا{مثال}
%=====================
 \ابتدا{مثال}\شناخت{مثال_مزاحمتی_تابع_مثال_ب}
شکل \حوالہ{شکل_مزاحمتی_مثال_تابع_منبع_ب} میں \اصطلاح{رو تابع منبع رو}\فرہنگ{منبع رو!رو تابع}\حاشیہب{current controlled current source}\فرہنگ{current source!current controlled} استعمال کیا گیا ہے۔اس دور میں \عددی{V_1} دریافت کریں۔
\begin{figure}
\centering
\includegraphics{figResistancesExampleControlledSourceB}
\caption{مثال \حوالہ{مثال_مزاحمتی_تابع_مثال_ب} کا دور۔}
\label{شکل_مزاحمتی_مثال_تابع_منبع_ب}
\end{figure}

حل:دباو \عددی{V_2} استعمال کرتے ہوئے بالائی جوڑ پر کرخوف قانون رو لکھتے ہیں۔
\begin{align*}
-2.2\times 10^{-3}+\frac{V_2}{5000+15000}+\frac{V_2}{12000}+I_0=0
\end{align*}
منبع کی قابو مساوات بھی لکھتے ہیں۔
\begin{align*}
I_0=12 I_1 = \frac{12 \times V_2}{12000} 
\end{align*}
مندرجہ بالا دونوں مساواتوں سے درج ذیل
\begin{align*}
-2.2\times 10^{-3}+\frac{V_2}{5000+15000}+\frac{V_2}{12000}+ \frac{12 \times V_2}{12000} =0
\end{align*}
لکھتے ہوئے
\begin{align*}
V_2=\frac{33}{17} \, \si{\volt}
\end{align*}
حاصل ہوتا ہے جس سے درکار دباو حاصل کرتے ہیں۔
\begin{align*}
V_1&=\left(\frac{15000}{5000+15000} \right) V_2\\
&=\left(\frac{15000}{5000+15000} \right) \times \frac{33}{17}\\
&=\SI{1.456}{\volt}
\end{align*}
\انتہا{مثال}
%===================
\FloatBarrier

\ابتدا{مثال}\شناخت{مثال_مزاحمتی_تابع_مثال_پ}
دو جوڑ ٹرانزسٹر کے استعمال سے بنائے گئے \اصطلاح{ایمٹر مشترک ایمپلیفائر}\فرہنگ{ایمپلیفائر!ایمٹر مشترک}\حاشیہب{common emitter amplifier}\فرہنگ{amplifier!common emitter} کا مساوی دور شکل \حوالہ{شکل_مزاحمتی_مثال_تابع_منبع_پ}-الف میں دکھایا گیا ہے۔مساوی دور کے حصول میں  \اصطلاح{دباو تابع منبع رو}\فرہنگ{منبع رو!دباو تابع}\حاشیہب{voltage controlled current source}\فرہنگ{current source!voltage controlled} کا استعمال کیا گیا ہے۔ خارجی اشارہ \عددی{v_0} اور داخلی اشارہ \عددی{v_s} کی شرح کو  \اصطلاح{افزائش دباو}\فرہنگ{افزائش دباو}\حاشیہب{voltage gain}\فرہنگ{gain!voltage}  \عددی{A_v} کہتے ہیں۔آئیں \عددی{A_v=\tfrac{v_0}{v_s}} حاصل کریں۔
\begin{figure}
\centering
\centering
\begin{subfigure}{\textwidth}
\centering
\includegraphics{figResistancesExampleControlledSourceC}
\caption*{(الف)}
\end{subfigure}
\begin{subfigure}{\textwidth}
\centering
\includegraphics{figResistancesExampleControlledSourceCequivalent}
\caption*{(ب)}
\end{subfigure}
\caption{ایمٹر مشترک ایمپلیفائر کا مساوی دور۔}
\label{شکل_مزاحمتی_مثال_تابع_منبع_پ}
\end{figure}

حل:خارجی جانب \عددی{R_C} اور \عددی{R_L} متوازی جڑے ہیں جن کی جگہ مساوی مزاحمت \عددی{R_M} نسب کیا جا سکتا ہے۔
\begin{align*}
R_M=\frac{R_C R_L}{R_C+R_L}
\end{align*} 
ایسا کرنے سے شکل-ب حاصل ہوتا ہے جہاں  بائیں دائرے کے لئے درج ذیل لکھا جا سکتا ہے۔
\begin{align*}
v_{be}=\frac{r_{be} v_s}{R_S+r_{be}}
\end{align*} 
شکل-ب میں دائیں دائرے سے درج ذیل لکھا جا سکتا ہے۔
\begin{align*}
v_0=-g_m v_{be} R_M
\end{align*}
درج بالا دونوں مساواتوں کو ملاتے ہوئے
\begin{align*}
v_0=-g_m \left(\frac{r_{be} v_s}{R_S+r_{be}}\right) R_M
\end{align*}
حاصل ہوتا ہے جہاں سے افزائش دباو حاصل ہوتی ہے۔
\begin{align*}
A_v=\frac{v_0}{v_s}= -\frac{g_m r_{be} R_M}{R_S+r_{be}}
\end{align*}
\انتہا{مثال}

%=================================
\ابتدا{مثال}\شناخت{مثال_مزاحمتی_تابع_مثال_ت}
شکل \حوالہ{شکل_مزاحمتی_مثال_تابع_منبع_ت} میں \اصطلاح{رو تابع منبع دباو}\فرہنگ{منبع دباو!رو تابع}\حاشیہب{current controlled voltage source}\فرہنگ{voltage source!current controlled} کا استعمال دکھایا گیا ہے۔اس دور میں خارجی دباو \عددی{V_0} حاصل کریں۔
\begin{figure}
\centering
\includegraphics{figResistancesExampleControlledSourceD}
\caption{رو تابع منبع دباو کے استعمال کی مثال۔}
\label{شکل_مزاحمتی_مثال_تابع_منبع_ت}
\end{figure}

حل: کرخوف قانون دباو سے درج ذیل لکھا جا سکتا ہے۔
\begin{align*}
-16+5000 I_1-V_m+2000 I_1=0
\end{align*}
منبع کی قابو مساوات درج ذیل ہے۔
\begin{align*}
V_m=3000I_1
\end{align*}
مندرجہ بالا دو مساواتوں کو ملاتے ہوئے
\begin{align*}
-16+5000 I_1-3000I_1+2000 I_1=0
\end{align*}
لکھا جا سکتا ہے جس سے
\begin{align*}
I_1&=\frac{16}{4000}\\
&=\SI{4}{\milli\ampere}
\end{align*}
حاصل ہوتا ہے۔یوں قانون اوہم کی مدد سے خارجی دباو
\begin{align*}
V_0&=4\times 10^{-3} \times 2000\\
&=\SI{8}{\volt}
\end{align*}
حاصل ہوتا ہے۔
\انتہا{مثال}
%=======================

\ابتدا{مشق}\شناخت{مشق_مزاحمتی_تابع_منبع_الف}
شکل \حوالہ{شکل_مزاحمتی-مشق_تابع_الف} میں \اصطلاح{موصلیت نما ایمپلیفائر}\فرہنگ{ایمپلیفائر!موصلیت نما}\حاشیہب{transconductance amplifier}\فرہنگ{amplifier!transconductance} کا مساوی دور دکھایا گیا ہے۔\اصطلاح{افزائش موصلیت نما}\فرہنگ{افزائش موصلیت نما}\حاشیہب{transconductance gain}\فرہنگ{gain!transconductance} \عددی{A_g=\tfrac{i_L}{v_s}} کی مساوات حاصل کریں اور \عددی{R_S=\SI{100}{\ohm}}، \عددی{r_{be}=\SI{400}{\ohm}}، \عددی{R_C=\SI{18}{\kilo\ohm}}، \عددی{R_L=\SI{2}{\kilo\ohm}} اور \عددی{\beta=180} کی صورت میں افزائش کی قیمت دریافت کریں۔
\begin{figure}
\centering
\includegraphics{figResistancesQuizControlledSourceA}
\caption{موصلیت نما ایمپلیفائر کا مساوی دور۔}
\label{شکل_مزاحمتی-مشق_تابع_الف}
\end{figure}

جواب:\عددی{A_g=-\beta\left(\frac{1}{R_S+r_{be}}\right)\left(\frac{R_C}{R_C+R_L}\right)}، \عددی{\SI{-0.324}{\ampere\per\volt}}
\انتہا{مشق}
%========================

\ابتدا{مشق}\شناخت{مشق_مزاحمتی-تابع_ب}
شکل \حوالہ{شکل_مزاحمتی-مشق_تابع_ب} میں \عددی{V_0} کی قیمت حاصل کریں۔
\begin{figure}
\centering
\includegraphics{figResistancesQuizControlledSourceB}
\caption{مشق \حوالہ{مشق_مزاحمتی-تابع_ب} کا دور۔}
\label{شکل_مزاحمتی-مشق_تابع_ب}
\end{figure}
\انتہا{مشق}
%================

\ابتدا{مشق}\شناخت{مشق_مزاحمتی-تابع_پ}
شکل \حوالہ{شکل_مزاحمتی-مشق_تابع_پ} میں \عددی{V_{AB}} دریافت کریں۔
\begin{figure}
\centering
\includegraphics{figResistancesQuizControlledSourceC}
\caption{مشق \حوالہ{مشق_مزاحمتی-تابع_پ} کا دور۔}
\label{شکل_مزاحمتی-مشق_تابع_پ}
\end{figure}
\انتہا{مشق}
%==================


\ابتدا{مشق}\شناخت{مشق_مزاحمتی-تابع_ت}
شکل \حوالہ{شکل_مزاحمتی-مشق_تابع_ت} میں \عددی{V_{mn}} دریافت کریں۔
\begin{figure}
\centering
\includegraphics{figResistancesQuizControlledSourceD}
\caption{مشق \حوالہ{مشق_مزاحمتی-تابع_ت} کا دور۔}
\label{شکل_مزاحمتی-مشق_تابع_ت}
\end{figure}
\انتہا{مشق}
%==================
\ابتدا{مشق}\شناخت{مشق_مزاحمتی-تابع_ٹ}
شکل \حوالہ{شکل_مزاحمتی-مشق_تابع_ٹ} میں \عددی{I_x} دریافت کریں۔
\begin{figure}
\centering
\includegraphics{figResistancesQuizControlledSourceE}
\caption{مشق \حوالہ{مشق_مزاحمتی-تابع_ٹ} کا دور۔}
\label{شکل_مزاحمتی-مشق_تابع_ٹ}
\end{figure}
\انتہا{مشق}
%==================

\باب{برقرار سوالات}

%======================
\ابتدا{سوال}
برقی دباو \عددی{v(t)=45\cos(100t-42^{\circ})\,\si{\volt}} کی تعدد، دوری عرصہ، زاویہ ہٹاو اور موثر دباو دریافت کریں۔

جوابات:\عددی{\SI{15.92}{\hertz}}، \عددی{\SI{62.8}{\milli\second}}، \عددی{-42^{\circ}}، \عددی{\SI{31.82}{\volt}}
\انتہا{سوال}
%=======================
\ابتدا{سوال}
درج ذیل امواج کا زاویائی فرق بیان کریں۔
\begin{align*}
v_1&=310\cos(100t+32^{\circ})\,\si{\volt}\\
v_2&=202\cos(100t-14^{\circ})\,\si{\volt}
\end{align*}

جواب:\عددی{v_2} سے \عددی{v_1} \عددی{46^{\circ}} آگے ہے۔
\انتہا{سوال}
%=====================
\ابتدا{سوال}
درج ذیل امواج کا زاویائی فرق بیان کریں۔
\begin{align*}
i&=2\cos(55t-80^{\circ})\,\si{\ampere}\\
v&=202\sin(55t-30^{\circ})\,\si{\volt}
\end{align*}

جواب:رو \عددی{40^{\circ}} آگے ہے۔
\انتہا{سوال}
%=====================

\ابتدا{سوال}
درج ذیل امواج کا زاویائی فرق بیان کریں۔
\begin{align*}
i&=2\cos(314t-80^{\circ})\,\si{\ampere}\\
v&=-54\sin(314t-30^{\circ})\,\si{\volt}
\end{align*}

جواب:رو \عددی{140^{\circ}} پیچے ہے۔
\انتہا{سوال}
%=====================
\ابتدا{سوال}\شناخت{سوال_برقرار_برق_گیر_الف}
شکل \حوالہ{شکل_سوال_برقرار_برق_گیر_الف}-الف میں درج ذیل دباو کی صورت میں رو کو وقتی دائرہ کار اور تعددی دائرہ کار میں لکھیں۔
\begin{align*}
v(t)&=10\cos (314t-30^{\circ})\\
v(t)&=15\sin(314t+60^{\circ})
\end{align*}
%
\begin{figure}
\centering
\begin{subfigure}{0.5\textwidth}
\centering
\begin{tikzpicture}[american voltages]
\draw(0,-0.25) rectangle ++(-0.5,\y+0.5);
\draw(0,0) to [short,-o]++(\x/2,0) to [short]++(\x,0) to [capacitor,l_={$\SI{0.1}{\micro\farad}$}]++(0,\y) to [short,i_<={$i(t)$},-o]++(-\x,0) to [short]++(-\x/2,0);
\draw(\x/2,\y/2) node{$\begin{aligned}&+ \\ &v(t) \\ &-  \end{aligned}$};
\end{tikzpicture}
\caption*{(الف)}
\end{subfigure}%
\begin{subfigure}{0.5\textwidth}
\centering
\begin{tikzpicture}[american voltages]
\draw(0,0) to [short,o-]++(2*\x,0) to [inductor,l_={$j6\,\si{\ohm}$}]++(0,\y) to [short,-o]++(-2*\x,0);
\draw(\x,0) to [resistor,*-*,l_={$\SI{2}{\ohm}$}]++(0,\y);
\draw[stealth-] (\x/4,\y/2)--++(-\x/4,0)--++(0,-\y/8)node[below]{$\bZ$};
\end{tikzpicture}
\caption*{(ب)}
\end{subfigure}%
\caption{سوال \حوالہ{سوال_برقرار_برق_گیر_الف} اور سوال \حوالہ{سوال_برقرار_رکاوٹ} کے ادوار۔}
\label{شکل_سوال_برقرار_برق_گیر_الف}
\end{figure}
جوابات: (الف)  \عددی{-0.314\sin(314t-30^{\circ}) \, \si{\milli\ampere}}، \عددی{0.314\phase{60^{\circ}}\,\si{\milli\ampere}}؛\\
 (ب) \عددی{0.511\cos(314t+60^{\circ})\,\si{\milli\ampere}}، \عددی{0.511\phase{60^{\circ}}\,\si{\milli\ampere}}
\انتہا{سوال}
%========================
\ابتدا{سوال}\شناخت{سوال_برقرار_رکاوٹ}
شکل \حوالہ{شکل_سوال_برقرار_برق_گیر_الف}-ب میں \عددی{\bZ} دریافت کریں۔

جواب:\عددی{\bZ=\tfrac{9}{5}+j\tfrac{3}{5}\,\si{\ohm}}
\انتہا{سوال}
%===========================
\ابتدا{سوال}\شناخت{سوال_برقرار_رکاوٹ_الف}
تعدد \عددی{\SI{50}{\hertz}} پر شکل \حوالہ{شکل_سوال_برقرار_رکاوٹ_الف}-الف کی رکاوٹ \عددی{\bZ} دریافت کریں۔
\begin{figure}
\centering
\begin{subfigure}{0.4\textwidth}
\centering
\begin{tikzpicture}
\draw(0,0) to [short,o-] ++(2*\x,0) to [capacitor,l={$\SI{0.8}{\milli\farad}$}]++(0,\y) to [resistor,l_={$\SI{2}{\ohm}$}]++(-\x,0) to [inductor,-o,l_={$\SI{20}{\milli\henry}$}]++(-\x,0); 
\draw(\x,0) to [resistor,*-*,l={$\SI{10}{\ohm}$}]++(0,\y);
\draw[stealth-](\x/4,\y/2)--++(-\x/4,0)--++(0,-\y/8)node[below]{$\bZ$};
\end{tikzpicture}
\caption*{(الف)}
\end{subfigure}%
\begin{subfigure}{0.6\textwidth}
\centering
\begin{tikzpicture}
\draw(0,0) to [short,o-]++(3*\x,0) to [capacitor]++(0,\y) to [inductor,l_={$j6\,\si{\ohm}$}]++(-\x,0) to [resistor,l_={$\SI{4}{\ohm}$}]++(-\x,0) to [resistor,-o,l_={$\SI{2}{\ohm}$}]++(-\x,0);
\draw(3*\x,3/4*\y)node[right]{$-j2\,\si{\ohm}$};
\draw(\x,0) to [capacitor,*-*,l={$-j2\,\si{\ohm}$}]++(0,\y);
\draw(\x,0) to [inductor,*-*]++(\x,\y);
\draw(\x+\x/2+\x/8,\y/4)node[right,rotate=45]{$j4\,\si{\ohm}$};
\draw(3*\x,0) to [resistor,*-*,l={$2\,\si{\ohm}$}]++(-\x,\y);
\draw[stealth-](\x/4,3/4*\y)--++(-\x/4,0)--++(0,-\y/8)node[below]{$\bZ$};
\end{tikzpicture}
\caption*{(ب)}
\end{subfigure}%
\caption{سوال \حوالہ{سوال_برقرار_رکاوٹ_الف} اور سوال \حوالہ{سوال_برقرار_رکاوٹ_ب} کے ادوار۔}
\label{شکل_سوال_برقرار_رکاوٹ_الف}
\end{figure}

جواب:\عددی{2.492+j3.794\,\si{\ohm}}
\انتہا{سوال}
%==============================
\ابتدا{سوال}\شناخت{سوال_برقرار_رکاوٹ_ب}
شکل \حوالہ{شکل_سوال_برقرار_رکاوٹ_الف}-ب کی رکاوٹ \عددی{\bZ} دریافت کریں۔

جواب:\عددی{\bZ=2.769-j1.846\,\si{\ohm}}
\انتہا{سوال}
%==================================
\ابتدا{سوال}\شناخت{سوال_برقرار_رکاوٹ_پ}
شکل \حوالہ{شکل_سوال_برقرار_رکاوٹ_پ} میں امالہ \عددی{L} کی وہ قیمت دریافت کریں جس پر دباو اور رو ہم قدم ہوں۔ایسی صورت میں منبع کو کیا رکاوٹ نظر آئے گی۔
\begin{figure}
\centering
\begin{tikzpicture}
\draw(0,0) to [american voltage source,l={$32\cos(500t+40^{\circ})\,\si{\volt}$}]++(0,\y) to [resistor,l={$\SI{6}{\ohm}$}]++(\x,0) to [inductor,l={$L$}]++(0,-\y) to [capacitor,l={$\SI{20}{\micro\farad}$}]++(-\x,0);
\end{tikzpicture}
\caption{سوال \حوالہ{سوال_برقرار_رکاوٹ_پ} کا دور۔}
\label{شکل_سوال_برقرار_رکاوٹ_پ}
\end{figure}

جواب:\عددی{\SI{5.066}{\milli\henry}}، \عددی{\SI{6}{\ohm}}
\انتہا{سوال}
%==================================
\ابتدا{سوال}\شناخت{سوال_برقرار_رکاوٹ_ت}
شکل \حوالہ{شکل_سوال_برقرار_رکاوٹ_ت} میں برق گیر \عددی{C} کی وہ قیمت دریافت کریں جس پر دباو منبع اور رو \عددی{i(t)} ہم قدم ہوں۔ایسی صورت میں منبع کو کیا رکاوٹ نظر آئے گی۔
\begin{figure}
\centering
\begin{tikzpicture}
\draw(0,0) to [american voltage source,l={$40\cos(1000t-20^{\circ})\,\si{\volt}$}]++(0,\y) to [resistor,i={$i(t)$},l={$\SI{8}{\ohm}$}]++(\x,0)
 to [resistor,l={$\SI{5}{\ohm}$}]++(\x,0) to [inductor,l={$\SI{6}{\milli\henry}$}]++(0,-\y) to [short](0,0);
\draw(\x,0) to [capacitor,*-*,l={$C$}]++(0,\y);
\end{tikzpicture}
\caption{سوال \حوالہ{سوال_برقرار_رکاوٹ_ت} کا دور۔}
\label{شکل_سوال_برقرار_رکاوٹ_ت}
\end{figure}

جواب:\عددی{\SI{4.149}{\micro\farad}}
\انتہا{سوال}
%==================================
\ابتدا{سوال}\شناخت{سوال_برقرار_رکاوٹ_ٹ}
شکل \حوالہ{شکل_سوال_برقرار_رکاوٹ_ٹ} میں وہ تعدد دریافت کریں جس پر رو کی چوٹی \عددی{\SI{5}{\ampere}} ہو۔
\begin{figure}
\centering
\begin{tikzpicture}
\draw(0,0) to [american voltage source,l={$60\cos(\omega t)\,\si{\volt}$}]++(0,\y) to [resistor,i={$i(t)$},l={$\SI{6}{\ohm}$}]++(\x,0)
  to [capacitor,l={$\SI{1}{\milli\farad}$}]++(0,-\y) to [short](0,0);
\end{tikzpicture}
\caption{سوال \حوالہ{سوال_برقرار_رکاوٹ_ٹ} کا دور۔}
\label{شکل_سوال_برقرار_رکاوٹ_ٹ}
\end{figure}

جواب:\عددی{\SI{15.31}{\hertz}}
\انتہا{سوال}
%==================================
\ابتدا{سوال}\شناخت{سوال_برقرار_رکاوٹ_ث}
شکل \حوالہ{شکل_سوال_برقرار_رکاوٹ_ث} میں منبع درج ذیل ہیں۔
\begin{align*}
i_1(t)&=20\cos(10^8 t+20^{\circ})\,\si{\milli\ampere}\\
i_2(t)&=15\sin(10^8 t+40^{\circ})\,\si{\milli\ampere}\\
v_s(t)&=10\cos 10^8 t \,\si{\volt}
\end{align*}
اس دور کو تعددی دائرہ کار میں بنائیں اور \عددی{v_R(t)} کے لئے حل کریں۔حاصل جواب اور دوری سمتیات کے استعمال سے \عددی{v_C(t)} حاصل کریں۔ 
\begin{figure}
\centering
\begin{tikzpicture}[american voltages]
\draw(0,0) to [american current source,l={$i_1(t)$}]++(0,\yy);
\draw(\xx,\yy) to [american current source,*-*,l_={$i_2(t)$}]++(0,-\yy);
\draw(2*\xx,0) to [resistor,*-*,l_={$\SI{40}{\ohm}$},v^>={$v_R(t)$}]++(0,\yy);
\draw(3*\xx,0) to [american voltage source,l_={$v_s(t)$}]++(0,\yy);
\draw(0,0) to [short]++(3*\xx,0);
\draw(0,\yy) to [short]++(2*\xx,0) to [capacitor,l_={$\SI{0.5}{\nano\farad}$},v^>={$v_C(t)$}]++(\xx,0);
\end{tikzpicture}
\caption{سوال \حوالہ{سوال_برقرار_رکاوٹ_ث} کا دور۔}
\label{شکل_سوال_برقرار_رکاوٹ_ث}
\end{figure}

جواب:\عددی{v_R(t)=10.431\cos(10^8 t-0.84^{\circ})\,\si{\volt}}، \عددی{v_C(t)=0.456\cos(10^8 t+160.3^{\circ})\,\si{\volt}}
\انتہا{سوال}
%==================================
\ابتدا{سوال}\شناخت{سوال_برقرار_کرخوف_الف}
شکل \حوالہ{شکل_سوال_برقرار_کرخوف_الف} میں \عددی{v_R(t)} اور \عددی{v_C(t)} حاصل کریں۔منبع کا دباو \عددی{v_s(t)=40\cos 15t \, \si{\volt}} ہے۔
\begin{figure}
\centering
\begin{tikzpicture}[american voltages]
\draw(0,0) to [inductor,l={$\SI{4}{\henry}$}]++(0,2*\y);
\draw(\x,0) to [american voltage source,*-,l_={$v_s(t)$}]++(0,\y) to [resistor,-*,l={$\SI{5}{\ohm}$},v_>={$v_R(t)$}]++(0,\y);
\draw(2*\x,0) to [resistor,*-*,l={$\SI{20}{\ohm}$}]++(0,2*\y);
\draw(3*\x,0) to [capacitor,l_={$\SI{0.02}{\farad}$},v^>={$v_C(t)$}]++(0,2*\y);
\draw(0,0) to [short]++(3*\x,0);
\draw(0,2*\y) to [resistor,l={$\SI{6}{\ohm}$}]++(\x,0) to [inductor,l={$\SI{0.3}{\henry}$}]++(\x,0) to [short]++(\x,0);
\end{tikzpicture}
\caption{سوال \حوالہ{سوال_برقرار_کرخوف_الف} کا دور۔}
\label{شکل_سوال_برقرار_کرخوف_الف}
\end{figure}

جواب:\عددی{v_R=35.35\cos(15t+167.4^{\circ})\,\si{\volt}}، \عددی{v_C(t)=22.76\cos(15t-92.7^{\circ})\,\si{\volt}}
\انتہا{سوال}
%==================================
\ابتدا{سوال}\شناخت{سوال_برقرار_کرخوف_ب}
شکل \حوالہ{شکل_سوال_برقرار_کرخوف_ب} میں \عددی{v_L(t)} حاصل کریں۔
\begin{figure}
\centering
\begin{tikzpicture}[american voltages]
\draw(0,0) to [american voltage source,l={$80\cos 100t \,\si{\volt}$}]++(0,\y) to [resistor,l={$\SI{2}{\ohm}$}]++(0,\y);
\draw(\x,0) to [capacitor,*-,l_={$\SI{2}{\milli\farad}$}]++(0,\y) to [resistor,-*,l={$\SI{6}{\ohm}$}]++(0,\y);
\draw(2*\x,0) to [inductor,*-*,l={$\SI{0.04}{\henry}$},v_>={$v_L(t)$}]++(0,2*\y);
\draw(3*\x,0) to [resistor,l_={$\SI{10}{\ohm}$}]++(0,2*\y);
\draw(0,0) to [short]++(3*\x,0);
\draw(0,2*\y) to [inductor,l={$\SI{0.02}{\henry}$}]++(\x,0) to [resistor,l={$\SI{4}{\ohm}$}]++(\x,0) to [capacitor,l={$\SI{1}{\milli\farad}$}]++(\x,0);
\end{tikzpicture}
\caption{سوال \حوالہ{سوال_برقرار_کرخوف_ب} کا دور۔}
\label{شکل_سوال_برقرار_کرخوف_ب}
\end{figure}

جواب:\عددی{v_L=37.3\cos(100t+18.9^{\circ})\,\si{\volt}}
\انتہا{سوال}
%==================================
\ابتدا{سوال}\شناخت{سوال_برقرار_کرخوف_پ}
شکل \حوالہ{شکل_سوال_برقرار_کرخوف_پ} میں \عددی{v_C(t)} اور \عددی{i(t)} حاصل کریں۔
\begin{figure}
\centering
\begin{tikzpicture}[american voltages]
\draw(0,0) to [capacitor,l={$\SI{0.02}{\farad}$},v_>={$v_C(t)$}]++(0,\y) to [resistor,l={$\SI{6}{\ohm}$}]++(0,\y);
\draw(2*\x,0) to [american voltage source,*-*,l={$60\cos 3t \,\si{\volt}$}]++(0,2*\y);
\draw(3*\x,0) to [resistor,*-,l={$\SI{6}{\ohm}$}]++(0,\y) to [inductor,-*,l={$\SI{1}{\henry}$},i<_={$i(t)$}]++(0,\y);
\draw(4*\x,0) to [resistor,l_={$\SI{4}{\ohm}$}]++(0,\y) to [capacitor,l={$\SI{0.2}{\farad}$}]++(0,\y);
\draw(0,0) to [short]++(4*\x,0);
\draw(0,2*\y) to [inductor,l={$\SI{2}{\henry}$}]++(2*\x,0) to [short]++(2*\x,0);
\end{tikzpicture}
\caption{سوال \حوالہ{سوال_برقرار_کرخوف_پ} کا دور۔}
\label{شکل_سوال_برقرار_کرخوف_پ}
\end{figure}

جواب:\عددی{vC(t)=81.7\cos(3t-29.4^{\circ})\,\si{\volt}}، \عددی{i(t)=4\sqrt{5}\cos(3t-26.6^{\circ})\,\si{\ampere}}
\انتہا{سوال}
%==================================
\ابتدا{سوال}\شناخت{سوال_برقرار_کرخوف_ت}
شکل \حوالہ{شکل_سوال_برقرار_کرخوف_ت} میں رو درج ذیل ہیں۔دباو \عددی{v(t)} دریافت کریں۔
\begin{align*}
i_1(t)&=6\cos(100t+22^{\circ})\,\si{\ampere}\\
i_2(t)&=4\cos(100t-30^{\circ})\,\si{\ampere}
\end{align*}
%
\begin{figure}
\centering
\begin{tikzpicture}[american voltages]
\draw(0,0) to [resistor,l={$\SI{4}{\ohm}$}]++(0,\y);
\draw(\x,0) to [american current source,*-*,l={$i_1(t)$}]++(0,\y);
\draw(2*\x,0) to [inductor,*-*,l={$\SI{40}{\milli\henry}$}]++(0,\y);
\draw(3*\x,\y) to [american current source,*-*,l_={$i_2(t)$}]++(0,-\y);
\draw(4*\x,0) to [resistor,*-*,l={$\SI{6}{\ohm}$}]++(0,\y);
\draw(5*\x,0) to [capacitor,l={$\SI{2}{\milli\farad}$},v_>={$v(t)$}]++(0,\y);
\draw(0,0) to [short]++(5*\x,0);
\draw(0,\y) to [short]++(5*\x,0);
\end{tikzpicture}
\caption{سوال \حوالہ{سوال_برقرار_کرخوف_ت} کا دور۔}
\label{شکل_سوال_برقرار_کرخوف_ت}
\end{figure}

جواب:\عددی{v(t)=11.29\cos(100t+70.5^{\circ})\,\si{\volt}}
\انتہا{سوال}
%==================================
\ابتدا{سوال}\شناخت{سوال_برقرار_کرخوف_ٹ}
شکل \حوالہ{شکل_سوال_برقرار_کرخوف_ٹ} میں  \عددی{v(t)} اور \عددی{i(t)} دریافت کریں۔
%
\begin{figure}
\centering
\begin{tikzpicture}[american voltages]
\draw(0,0) to [american voltage source,l={$30\cos 10t \,\si{\volt}$}]++(0,2*\y);
\draw(\x,0) to [capacitor,*-,l={$\SI{0.04}{\farad}$}]++(0,\y) to [resistor,-*,l={$\SI{4}{\ohm}$}]++(0,\y);
\draw(2*\x,0) to [inductor,*-*,l={$\SI{0.6}{\henry}$}]++(0,2*\y);
\draw(3*\x,0) to [resistor,*-*,l={$\SI{8}{\ohm}$}]++(0,2*\y);
\draw(4*\x,0) to [capacitor,*-*,l={$\SI{0.05}{\farad}$}]++(0,2*\y);
\draw(5*\x,0) to [capacitor,l={$\SI{0.02}{\farad}$},v_>={$v(t)$}]++(0,2*\y) to [resistor,l={$\SI{6}{\ohm}$}]++(-\x,0);
\draw(0,0) to [short]++(2*\x,0) to [short,i<={$i(t)$}]++(\x,0) to [short]++(2*\x,0);
\draw(0,2*\y) to [short]++(4*\x,0);
\end{tikzpicture}
\caption{سوال \حوالہ{سوال_برقرار_کرخوف_ٹ} کا دور۔}
\label{شکل_سوال_برقرار_کرخوف_ٹ}
\end{figure}

جواب:\عددی{v=19.2\cos(10t-50.2^{\circ})\,\si{\volt}}، \عددی{i(t)=17.7\cos(10t+80.4^{\circ})\,\si{\ampere}}
\انتہا{سوال}
%==================================
\ابتدا{سوال}\شناخت{سوال_برقرار_کرخوف_ث}
شکل \حوالہ{شکل_سوال_برقرار_کرخوف_ث} میں \عددی{v(t)}  دریافت کریں۔
%
\begin{figure}
\centering
\begin{tikzpicture}[american voltages]
\draw(0,0) to [american voltage source,l={$60\cos 60t \,\si{\volt}$}]++(0,\y) to [resistor,l={$\SI{8}{\ohm}$}]++(\x,0) to [inductor,-o,l={$\SI{0.2}{\henry}$}]++(\x,0)coordinate(kT) to [capacitor,l={$\SI{0.002}{\farad}$}]++(\x,0);
\draw(0,0) to [resistor,l_={$\SI{2}{\ohm}$}]++(\x,0) to [short,-o]++(\x,0) to [inductor,l_={$\SI{0.4}{\henry}$}]++(\x,0) to [american voltage source,l_={$70\cos(60t-45^{\circ})\,\si{\volt}$}]++(0,\y);
\draw(kT)++(0,-\y/2)node{$\begin{aligned} &+ \\ &v(t) \\ &- \end{aligned}$};
\end{tikzpicture}
\caption{سوال \حوالہ{سوال_برقرار_کرخوف_ث} کا دور۔}
\label{شکل_سوال_برقرار_کرخوف_ث}
\end{figure}

جواب:\عددی{v(t)=47.1\cos(60t-22.5^{\circ})\,\si{\volt}}
\انتہا{سوال}
%==================================
\ابتدا{سوال}\شناخت{سوال_برقرار_دوری_سمتیہ_الف}
شکل \حوالہ{شکل_سوال_برقرار_دوری_سمتیہ_الف}-الف میں \عددی{\bV}  دریافت کریں۔
%
\begin{figure}
\centering
\begin{subfigure}{0.4\textwidth}
\centering
\begin{tikzpicture}[american voltages]
\draw(0,0) to [american voltage source,l={$160\phase{0^{\circ}}\,\si{\volt}$}]++(0,\y) to [resistor,l={$\SI{2}{\ohm}$}]++(\x,0)
 to [inductor,l_={$j4\,\si{\ohm}$},v^<={$\bV$}]++(0,-\y) to [short](0,0);
\end{tikzpicture}
\caption*{(الف)}
\end{subfigure}%
\begin{subfigure}{0.6\textwidth}
\centering
\begin{tikzpicture}[american voltages]
\draw(0,0) to [american voltage source,l={$10\phase{60^{\circ}}\,\si{\volt}$}]++(0,\y) to [resistor,l={$\SI{4}{\ohm}$}]++(\x,0) to [inductor,l={$j8\,\si{\ohm}$}]++(\x,0) to [capacitor,l_={$-j6\,\si{\ohm}$},v^<={$\bV$}]++(0,-\y) to [short](0,0);
\end{tikzpicture}
\caption*{(ب)}
\end{subfigure}%
\caption{سوال \حوالہ{سوال_برقرار_دوری_سمتیہ_الف} اور سوال \حوالہ{سوال_برقرار_دوری_سمتیہ_ب} کے ادوار۔}
\label{شکل_سوال_برقرار_دوری_سمتیہ_الف}
\end{figure}

جواب:\عددی{\bV=113.1\phase{45^{\circ}}\,\si{\volt}}
\انتہا{سوال}
%==================================
\ابتدا{سوال}\شناخت{سوال_برقرار_دوری_سمتیہ_ب}
شکل \حوالہ{شکل_سوال_برقرار_دوری_سمتیہ_الف}-ب میں \عددی{\bV}  دریافت کریں۔

جواب:\عددی{\bV=13.4\phase{-56.6^{\circ}}\,\si{\volt}}
\انتہا{سوال}
%==================================

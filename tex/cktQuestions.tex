\باب{سوالات تعددی ردعمل}
%===================
\ابتدا{سوال}\شناخت{سوال_تعددی_رکاوٹ_الف}
شکل \حوالہ{شکل_سوال_تعددی_رکاوٹ_الف} میں داخلی رکاوٹ \عددی{\bZ(s)} حاصل کریں۔
\begin{figure}
\centering
\begin{tikzpicture}
\draw(0,0) to [resistor,o-,l={$R_1$}]++(\x,0) to [short]++(2*\x,0) to [capacitor,l={$C$}]++(0,-\y) to [short,-o]++(-3*\x,0);
\draw(\x,0) to [resistor,*-*,l={$R_2$}]++(0,-\y);
\draw(2*\x,0) to [inductor,*-*,l={$L$}]++(0,-\y);
\draw[stealth-](\x/4,-\y/2)--++(-\x/4,0)--++(0,-\y/8)node[below]{$\bZ(s)$};
\end{tikzpicture}
\caption{سوال \حوالہ{سوال_تعددی_رکاوٹ_الف} کا دور۔}
\label{شکل_سوال_تعددی_رکاوٹ_الف}
\end{figure}

جواب:\عددی{\bZ(s)=R_1\tfrac{sR_2L}{S^2R_2LC+SL+R_2}}
\انتہا{سوال}
%====================
\ابتدا{سوال}\شناخت{سوال_تعددی_رکاوٹ_ب}
شکل \حوالہ{شکل_سوال_تعددی_رکاوٹ_ب} میں داخلی رکاوٹ \عددی{\bZ(s)} حاصل کریں۔
\begin{figure}
\centering
\begin{tikzpicture}[american voltages]
\draw(0,0) to [short,o-]++(\x/2,0) to [capacitor,l={$C_1$}]++(\x,0) to [short,-o]++(\x,0);
\draw(0,-2*\y) to [short,o-o]++(2*\x+\x/2,0);
\draw(\x/2,0) to [short,*-]++(0,3/4*\y) to [inductor,l={$L_1$}]++(\x,0) to [short,-*]++(0,-3/4*\y);
\draw(\x+\x/2,0) to [inductor,l={$L_2$}]++(0,-\y) to [capacitor,-*,l={$C_2$}]++(0,-\y);
\draw(0,0) to [open,v={$v_d(t)$}]++(0,-2*\y);
\draw(2*\x+\x/2,0) to [open,v^<={$v_0(t)$}]++(0,-2*\y);
\end{tikzpicture}
\caption{سوال \حوالہ{سوال_تعددی_رکاوٹ_ب} اور سوال \حوالہ{سوال_تعددی_رکاوٹ_پ} کا دور۔}
\label{شکل_سوال_تعددی_رکاوٹ_ب}
\end{figure}

جواب:\عددی{\bZ(s)=\tfrac{sL_1}{s^2L_1C_1+1}+\tfrac{s^2L_2C_2+1}{sC_2}}
\انتہا{سوال}
%====================
\ابتدا{سوال}\شناخت{سوال_تعددی_رکاوٹ_پ}
شکل \حوالہ{شکل_سوال_تعددی_رکاوٹ_ب} میں تبادلی تفاعل \عددی{\tfrac{\bV_0(s)}{\bV_d(s)}} لکھیں۔

جواب:
$\tfrac{\bV_0(s)}{\bV_d(s)}=\tfrac{s^4L_1L_2C_1C_2+s^2(L_1C_1+L_2C_2)+1}{s^4L_1L_2C_1C_2+s^2*(L_1C_1+L_2C_2+L_1C_2)+1}$
\انتہا{سوال}
%=====================
\ابتدا{سوال}\شناخت{سوال_تعددی_رکاوٹ_ت}
شکل \حوالہ{شکل_سوال_تعددی_رکاوٹ_ت} کی داخلی رکاوٹ \عددی{\bZ(s)} دریافت کریں۔
\begin{figure}
\centering
\begin{tikzpicture}
\draw(0,0) to [short,o-]++(3*\x,0) to [resistor,l={$\SI{2}{\ohm}$}]++(0,2*\y) to [short]++(-\x,0) to [resistor,l={$\SI{6}{\ohm}$}]++(-\x,0) to [inductor,-o,l={$\SI{1}{\henry}$}]++(-\x,0);
\draw(\x,2*\y) to [resistor,*-,l_={$\SI{6}{\ohm}$}]++(0,-\y) to [short]++(\x,0) to [resistor,-*,l_={$\SI{6}{\ohm}$}]++(0,\y);
\draw(\x+\x/2,0) to [capacitor,*-*,l={$\SI{1}{\farad}$}]++(0,\y);
\draw[stealth-](\x/4,\y)--++(-\x/4,0)--++(0,-\y/8)node[below]{$\bZ(s)$};
\end{tikzpicture}
\caption{سوال \حوالہ{سوال_تعددی_رکاوٹ_ت} کا دور۔}
\label{شکل_سوال_تعددی_رکاوٹ_ت}
\end{figure}

جواب:\عددی{\bZ(s)=\tfrac{6s^2+21s+6}{6s+1}}
\انتہا{سوال}
%====================
\ابتدا{سوال}\شناخت{سوال_تعددی_رکاوٹ_ٹ}
شکل \حوالہ{شکل_سوال_تعددی_رکاوٹ_ٹ} میں \عددی{c} اور \عددی{d} کو کھلے سر رکھتے ہوئے \عددی{a} اور \عددی{b} کے مابین رکاوٹ دریافت کریں۔
\begin{figure}
\centering
\begin{tikzpicture}
\pgfmathsetmacro{\ang}{atan(\yy/(\xx+\xx/2))}
\pgfmathsetmacro{\len}{\yy/sin(\ang)}
\draw(0,0)node[left]{$a$} to [short,o-]++(\xx/2,0) to [resistor,l={$\SI{1}{\ohm}$}]++(\xx+\xx/2,0) to [short,-o]++(\xx/2,0)node[right]{$c$};
\draw(0,0-\yy)node[left]{$b$} to [short,o-]++(\xx/2,0) to [capacitor,l_={$\SI{0.5}{\farad}$}]++(\xx+\xx/2,0) to [short,-o]++(\xx/2,0)node[right]{$d$};
\draw(\x/2,0) to [resistor,*-,l_={$\SI{2}{\ohm}$}]++(-\ang:\xx) to [short,-*]++(-\ang:\len-\xx);
\draw(\xx+\xx,0) to [inductor,*-,l={$\SI{1}{\henry}$}]++(-180+\ang:\xx) to [short,-*]++(-180+\ang:\len-\xx);
\end{tikzpicture}
\caption{سوال \حوالہ{سوال_تعددی_رکاوٹ_ٹ} کا دور۔}
\label{شکل_سوال_تعددی_رکاوٹ_ٹ}
\end{figure}

جواب:\عددی{\bZ=\tfrac{2s+2}{s+2}}
\انتہا{سوال}
%====================
\ابتدا{سوال}\شناخت{سوال_تعددی_رکاوٹ_ث}
شکل \حوالہ{شکل_سوال_تعددی_رکاوٹ_ٹ} میں \عددی{c} اور \عددی{d} کو آپس میں قصر دور کرتے ہوئے \عددی{a} اور \عددی{b} کے مابین رکاوٹ دریافت کریں۔

جواب:\عددی{\bZ:\tfrac{2s^2+6s+4}{3s^2+6}}
\انتہا{سوال}
%===========================
\ابتدا{سوال}\شناخت{سوال_تعددی_رکاوٹ_ج}
شکل \حوالہ{شکل_سوال_تعددی_رکاوٹ_ٹ} میں \عددی{c} اور \عددی{d} کے مابین \عددی{\SI{1}{\ohm}} مزاحمت نسب کرتے ہوئے \عددی{a} اور \عددی{b} کے مابین رکاوٹ دریافت کریں۔

جواب:\عددی{\bZ(s)=\tfrac{4s^2+10s+6}{4s^2+3s+8}}
\انتہا{سوال}
%==========================
\ابتدا{سوال}\شناخت{سوال_تعددی_رکاوٹ_چ}
شکل \حوالہ{شکل_سوال_تعددی_رکاوٹ_چ} میں داخلی رکاوٹ \عددی{\bZ(s)} دریافت کریں۔
\begin{figure}
\centering
\begin{tikzpicture}
\draw(0,0) to [short,o-]++(\x/2,0) to [capacitor,l_={$\SI{1}{\farad}$}]++(\x,0);
\draw(0,2*\y) to [short,o-]++(\x/2,0) to [capacitor,l={$\SI{0.5}{\farad}$}]++(\x,0);
\draw(\x/2,\y)  to [resistor,*-*,l={$\SI{1}{\ohm}$}]++(\x,0);
\draw(\x/2,0) to [resistor,*-,l={$\SI{2}{\ohm}$}]++(0,\y) to [inductor,-*,l={$\SI{2}{\henry}$}]++(0,\y);
\draw(\x+\x/2,0) to [resistor,l={$\SI{1}{\ohm}$}]++(0,\y) to [inductor,l={$\SI{1}{\henry}$}]++(0,\y);
\draw[stealth-](\x/8,\y+\y/4)--++(-\x/8,0)--++(0,-\y/8)node[below]{$\bZ(s)$};
\end{tikzpicture}
\caption{سوال \حوالہ{سوال_تعددی_رکاوٹ_چ} کا دور۔}
\label{شکل_سوال_تعددی_رکاوٹ_چ}
\end{figure}

جواب:
$\bZ(s)=\tfrac{8s^4+12s^3+26s^2+14s+4}{12s^3+6s^2+9s+2}$
\انتہا{سوال}
%====================
\ابتدا{سوال}\شناخت{سوال_تعددی_بوڈا_الف}
تبادلی تفاعل \عددی{\bH(j\omega)=\tfrac{1}{(j\omega+1)(0.1j\omega+1)}} کا بوڈا خط کھینچیں۔
\انتہا{سوال}
%======================================
\ابتدا{سوال}\شناخت{سوال_تعددی_بوڈا_ب}
تبادلی تفاعل \عددی{\bH(j\omega)=\tfrac{100j\omega}{(j\omega+1)(j\omega+10)(j\omega+50)}} کا بوڈا خط کھینچیں۔
\انتہا{سوال}
%=====================
\ابتدا{سوال}\شناخت{سوال_تعددی_بوڈا_پ}
تبادلی تفاعل \عددی{\bH(j\omega)=\tfrac{100}{(j\omega)^2(j\omega+100)}} کا بوڈا خط کھینچیں۔
\انتہا{سوال}
%===============================
\ابتدا{سوال}\شناخت{سوال_تعددی_بوڈا_ت}
تبادلی تفاعل \عددی{\bH(j\omega)=\tfrac{500(j\omega+2)(j\omega+100)}{-\omega^2(j\omega+1000)^2}} کا بوڈا خط کھینچیں۔
\انتہا{سوال}
%==========================
\ابتدا{سوال}\شناخت{سوال_تعددی_ردعمل_الف}
شکل \حوالہ{شکل_سوال_تعددی_ردعمل_الف}-الف میں منبع کی تعدد \عددی{\omega} قابل تبدیل ہے۔دور کی قدرتی گمکی تعدد \عددی{\SI{500}{\radian\per\second}} ہونے کی صورت میں \عددی{C} کی قیمت کا تخمینہ لگائیں۔قدرتی تعدد \عددی{\omega_0} پر دور میں رو \عددی{i(t)} دریافت کریں۔تعدد \عددی{2\omega_0} اور \عددی{\tfrac{\omega_0}{2}} پر بھی رو دریافت کریں۔ 
\begin{figure}
\centering
\begin{subfigure}{0.6\textwidth}
\centering
\begin{tikzpicture}
\draw(0,0) to [american voltage source,l={$40\cos(\omega t+30^{\circ})\,\si{\volt}$}]++(0,\y) to [inductor,l={$\SI{20}{\milli\henry}$}]++(\x,0) to [resistor,l={$\SI{2}{\ohm}$},i={$i(t)$}]++(0,-\y) to [capacitor,l={$C$}]++(-\x,0);
\end{tikzpicture}
\caption*{(الف)}
\end{subfigure}%
\begin{subfigure}{0.4\textwidth}
\centering
\begin{tikzpicture}
\draw(0,0) to [american voltage source,l={$30\phase{0^{\circ}}\,\si{\volt}$}]++(0,\y) to [inductor,l={$\SI{10}{\milli\henry}$}]++(\x,0)
 to [variable european resistor,l_={$\SI{10}{\ohm}$}]++(0,-\y) to [capacitor,l={$\SI{400}{\micro\farad}$}]++(-\x,0);
\end{tikzpicture}
\caption*{(ب)}
\end{subfigure}%
\caption{سوال \حوالہ{سوال_تعددی_ردعمل_الف} کا دور۔}
\label{شکل_سوال_تعددی_ردعمل_الف}
\end{figure}

جوابات:\عددی{20\cos(500t+30^{\circ})\,\si{\ampere}}، \عددی{2.640\cos(1000t-52.4^{\circ})\,\si{\ampere}}،  \عددی{2.640\cos(250t+112.4^{\circ})\,\si{\ampere}}
\انتہا{سوال}
%=========================
\ابتدا{سوال}\شناخت{سوال_تعددی_ردعمل_ب}
شکل \حوالہ{شکل_سوال_تعددی_ردعمل_الف}-ب میں عرض پٹی دریافت کریں۔متغیر مزاحمت کی قیمت تبدیل کرتے ہوئے عرض پٹی آدھی کریں۔مزاحمت کی قیمت کیا ہو گی؟

جوابات:\عددی{\BW=\SI{1000}{\radian\per\second}}، \عددی{R=\SI{5}{\ohm}}
\انتہا{سوال}
%======================
\ابتدا{سوال}
ایک سلسلہ وار \عددی{RLC} دور کی گمکی تعدد \عددی{\omega_0=\SI{2}{\kilo\radian\per\second}} ہے جبکہ \عددی{C=\SI{40}{\micro\farad}} اور گمکی تعدد پر کل رکاوٹ \عددی{\SI{2.2}{\ohm}} ہے۔ مزاحمت اور امالہ کی قیمت دریافت کریں۔دور کی عرض پٹی اور معیاری مستقل بھی حاصل کریں۔

جوابات:\عددی{R=\SI{2.2}{\ohm}}، \عددی{L=\SI{6.25}{\milli\henry}}، \عددی{\BW=\SI{352}{\radian\per\second}}، \عددی{Q=5.682}  
\انتہا{سوال}
%=====================
\ابتدا{سوال}
سلسلہ وار \عددی{RLC} دور کا معیاری مستقل \عددی{120} اور گمکی تعدد \عددی{\SI{15000}{\radian\per\second}} ہے۔ دور کی عرض پٹی، بلند انقطاعی تعدد اور پست انقطاعی تعدد دریافت کریں۔

جوابات:\عددی{\BW=\SI{125}{\radian\per\second}}، \عددی{\omega_H=\SI{15063}{\radian\per\second}}، \عددی{\omega_L=\SI{14938}{\radian\per\second}}
\انتہا{سوال}
%==================
\ابتدا{سوال}\شناخت{سوال_تعددی_ردعمل_دباو_الف}
شکل \حوالہ{شکل_سوال_تعددی_ردعمل_دباو_الف}-الف میں گمکی تعدد \عددی{\omega_0}، معیاری مستقل \عددی{Q}، عرض پٹی \عددی{\BW} اور بلند انقطاعی تعدد \عددی{\omega_H} حاصل کریں۔زیادہ سے زیادہ \عددی{v_0(t)} بھی دریافت کریں۔
\begin{figure}
\centering
\begin{subfigure}{0.6\textwidth}
\centering
\begin{tikzpicture}[american voltages]
\draw(0,0) to [american voltage source,l={$10\cos \omega t\,\si{\volt}$}]++(0,\y) to [inductor,l={$\SI{4}{\milli\henry}$}]++(\x,0) to [resistor,l={$\SI{1}{\ohm}$}]++(0,-\y) to [capacitor,l={$\SI{20}{\micro\farad}$},v={$v_0(t)$}]++(-\x,0);
\end{tikzpicture}
\caption*{(الف)}
\end{subfigure}%
\begin{subfigure}{0.4\textwidth}
\centering
\begin{tikzpicture}
\draw(0,0) to [american voltage source,l={$v_d(t)$}]++(0,\y) to [inductor,l={$\SI{40}{\milli\henry}$}]++(\x,0)
 to [variable european resistor,l_={$\SI{100}{\ohm}$}]++(0,-\y) to [capacitor,l={$\SI{60}{\micro\farad}$}]++(-\x,0);
\end{tikzpicture}
\caption*{(ب)}
\end{subfigure}%
\caption{سوال \حوالہ{سوال_تعددی_ردعمل_دباو_الف} کا دور۔}
\label{شکل_سوال_تعددی_ردعمل_دباو_الف}
\end{figure}

جوابات:\عددی{\omega_0=\SI{3536}{\radian\per\second}}، \عددی{Q=14.1}، \عددی{\BW=\SI{250}{\radian\per\second}}، \\ \عددی{\omega_H=\SI{3663}{\radian\per\second}}،  \عددی{v_{0\text{بلندتر}}=\SI{141.51}{\volt}}

\انتہا{سوال}
%=========================
\ابتدا{سوال}
شکل \حوالہ{شکل_سوال_تعددی_ردعمل_دباو_الف}-ب میں \عددی{v_d(t)=20\cos \omega t \,\si{\volt}} ہے۔قدرتی تعدد، معیاری مستقل، عرض پٹی اور گمکی تعدد پر دور میں طاقت کا ضیاع حاصل کریں۔

جوابات:\عددی{\omega_0=\SI{645}{\radian\per\second}}، \عددی{Q=0.26}، \عددی{\BW=\SI{2500}{\radian\per\second}}، \عددی{p=\SI{2}{\watt}}
\انتہا{سوال}
%==================

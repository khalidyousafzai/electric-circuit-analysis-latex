\باب{سوالات تین دوری}
%=======================
\ابتدا{سوال}
تین دوری \عددی{abc} نظام میں \عددی{\bV_{an}=220\phase{90^{\circ}}\,\volt\,\rms} ہے۔تینوں دباو تار دریافت کریں۔

جوابات:\عددی{\bV_{ab}=381\phase{120^{\circ}}\,\si{\volt}\,\rms}، \عددی{\bV_{bc}=381\phase{0^{\circ}}\,\si{\volt}\,\rms}، 
\عددی{\bV_{ca}=381\phase{-120^{\circ}}\,\si{\volt}\,\rms}
\انتہا{سوال}
%=======================
\ابتدا{سوال}
تین دوری \عددی{abc} نظام میں \عددی{\bV_{an}=100\phase{30^{\circ}}\,\volt\,\rms} ہے۔تینوں دباو تار دریافت کریں۔

جوابات:\عددی{\bV_{ab}=173\phase{60^{\circ}}\,\si{\volt}\,\rms}، \عددی{\bV_{bc}=173\phase{0^{\circ}}\,\si{\volt}\,\rms}، 
\عددی{\bV_{ca}=173\phase{-60^{\circ}}\,\si{\volt}\,\rms}
\انتہا{سوال}
%=======================
\ابتدا{سوال}
تین دوری \عددی{abc} نظام میں \عددی{\bV_{ab}=200\phase{60^{\circ}}\,\volt\,\rms} ہے۔تینوں دباو دور دریافت کریں۔

جوابات:\عددی{\bV_{an}=115\phase{30^{\circ}}\,\si{\volt}\,\rms}، \عددی{\bV_{bn}=115\phase{-90^{\circ}}\,\si{\volt}\,\rms}، 
\عددی{\bV_{cn}=115\phase{150^{\circ}}\,\si{\volt}\,\rms}
\انتہا{سوال}
%=======================
\ابتدا{سوال}
تین دوری \عددی{abc} نظام میں \عددی{\bV_{an}=240\phase{45^{\circ}}\,\volt\,\rms} ہے۔تینوں دباو تار دریافت کریں۔

جوابات:\عددی{\bV_{ab}=416\phase{75^{\circ}}\,\si{\volt}\,\rms}، \عددی{\bV_{bc}=416\phase{-45^{\circ}}\,\si{\volt}\,\rms}، 
\عددی{\bV_{ca}=416\phase{-165^{\circ}}\,\si{\volt}\,\rms}
\انتہا{سوال}
%=======================
\ابتدا{سوال}\شناخت{سوال_تین_دوری_رکاوٹ_الف}
شکل \حوالہ{شکل_سوال_تین_دوری_رکاوٹ_الف}-الف میں مساوی تکونی رکاوٹ \عددی{\bZ_{ab}}، \عددی{\bZ_{bc}} اور \عددی{\bZ_{ca}} حاصل کریں۔
\begin{figure}
\centering
\begin{subfigure}{0.5\textwidth}
\centering
\begin{tikzpicture}
\draw(0,0) to [european resistor,*-o,l={$1+j2\,\si{\ohm}$}]++(90:\x)node[right]{$A$};
\draw(0,0) to [european resistor,-o,l={$1+j2\,\si{\ohm}$}]++(-30:\x)node[right]{$B$};
\draw(0,0) to [european resistor,-o,l={$1+j2\,\si{\ohm}$}]++(-150:\x)node[left]{$C$};
\end{tikzpicture}
\caption*{(الف)}
\end{subfigure}%
\begin{subfigure}{0.5\textwidth}
\centering
\begin{tikzpicture}
\draw(0,0) to [european resistor,*-o,l={$1+j2\,\si{\ohm}$}]++(90:\x)node[right]{$A$};
\draw(0,0) to [european resistor,-o,l={$2-j2\,\si{\ohm}$}]++(-30:\x)node[right]{$B$};
\draw(0,0) to [european resistor,-o,l={$2+j4\,\si{\ohm}$}]++(-150:\x)node[left]{$C$};
\end{tikzpicture}
\caption*{(ب)}
\end{subfigure}%
\caption{سوال \حوالہ{سوال_تین_دوری_رکاوٹ_الف} اور سوال \حوالہ{سوال_تین_دوری_رکاوٹ_ب} کے ادوار۔}
\label{شکل_سوال_تین_دوری_رکاوٹ_الف}
\end{figure}

جوابات:\عددی{\bZ_{ab}=\bZ_{bc}=\bZ_{ca}=3+j6\,\si{\ohm}}
\انتہا{سوال}
%==========================
\ابتدا{سوال}\شناخت{سوال_تین_دوری_رکاوٹ_ب}
شکل \حوالہ{شکل_سوال_تین_دوری_رکاوٹ_الف}-ب میں مساوی تکونی رکاوٹ \عددی{\bZ_{ab}}، \عددی{\bZ_{bc}} اور \عددی{\bZ_{ca}} حاصل کریں۔

جوابات:\عددی{\bZ_{ab}=4-j1\,\si{\ohm}}، \عددی{\bZ_{bc}=8-j2\,\si{\ohm}}، \عددی{\bZ_{ca}=-0.5+j6.5\,\si{\ohm}}
\انتہا{سوال}
%=============================
\ابتدا{سوال}\شناخت{سوال_تین_دوری_رکاوٹ_پ}
شکل \حوالہ{شکل_سوال_تین_دوری_رکاوٹ_پ}-الف میں مساوی ستارہ رکاوٹ \عددی{\bZ_{a}}، \عددی{\bZ_{b}} اور \عددی{\bZ_{c}} حاصل کریں۔
\begin{figure}
\centering
\begin{subfigure}{0.5\textwidth}
\centering
\begin{tikzpicture}
\pgfmathsetmacro{\len}{\x*sqrt(3)}
\draw(0,0)coordinate(kB) to [european resistor,l={$0.3-j1.2\,\si{\ohm}$}]++(180:\len)coordinate(kC) to [european resistor,l={$0.3-j1.2\,\si{\ohm}$}]++(60:\len) coordinate(kA)to [european resistor,l={$0.3-j1.2\,\si{\ohm}$}]++(-60:\len);
\draw(kA) to [short,*-o]++(90:0.3)node[left]{$A$};
\draw(kB) to [short,*-o]++(-30:0.3)node[right]{$B$};
\draw(kC) to [short,*-o]++(-150:0.3)node[left]{$C$};
\end{tikzpicture}
\caption*{(الف)}
\end{subfigure}%
\begin{subfigure}{0.5\textwidth}
\centering
\begin{tikzpicture}
\pgfmathsetmacro{\len}{\x*sqrt(3)}
\draw(0,0)coordinate(kB) to [european resistor,l={$4+j3\,\si{\ohm}$}]++(180:\len)coordinate(kC) to [european resistor,l={$2-j4\,\si{\ohm}$}]++(60:\len) coordinate(kA)to [european resistor,l={$1+j1\,\si{\ohm}$}]++(-60:\len);
\draw(kA) to [short,*-o]++(90:0.3)node[left]{$A$};
\draw(kB) to [short,*-o]++(-30:0.3)node[right]{$B$};
\draw(kC) to [short,*-o]++(-150:0.3)node[left]{$C$};
\end{tikzpicture}
\caption*{(ب)}
\end{subfigure}%
\caption{سوال \حوالہ{سوال_تین_دوری_رکاوٹ_پ} اور سوال \حوالہ{سوال_تین_دوری_رکاوٹ_ت} کے ادوار۔}
\label{شکل_سوال_تین_دوری_رکاوٹ_پ}
\end{figure}

جوابات:\عددی{\bZ_a=\bZ_b=\bZ_c=0.1-j0.4\,\si{\ohm}}
\انتہا{سوال}
%==========================
\ابتدا{سوال}\شناخت{سوال_تین_دوری_رکاوٹ_ت}
شکل \حوالہ{شکل_سوال_تین_دوری_رکاوٹ_پ}-ب میں مساوی ستارہ رکاوٹ \عددی{\bZ_{a}}، \عددی{\bZ_{b}} اور \عددی{\bZ_{c}} حاصل کریں۔

جوابات:\عددی{\bZ_a=0.86-j0.29\,\si{\ohm}}، \عددی{\bZ_b=0.14-j1\,\si{\ohm}}، \عددی{\bZ_c=2.86-j1.43\,\si{\ohm}}
\انتہا{سوال}
%=============================
\ابتدا{سوال}\شناخت{سوال_تین_دوری_رکاوٹ_ٹ}
شکل \حوالہ{شکل_سوال_تین_دوری_رکاوٹ_ٹ} کا مساوی رکاوٹ \عددی{\bZ} دریافت کریں۔
\begin{figure}
\centering
\begin{tikzpicture}
\pgfmathsetmacro{\len}{\x*sqrt(3)}
\draw(0,0)coordinate(kB) to [european resistor,l={$-j2\,\si{\ohm}$}]++(180:\len)coordinate(kC) to [european resistor,l={$1+j1\,\si{\ohm}$}]++(60:\len) coordinate(kA)to [european resistor,l={$1+j1\,\si{\ohm}$}]++(-60:\len);
\draw(kA) to [capacitor,*-,l_={$-j1\,\si{\ohm}$}]++(0,\y) to [resistor,-o,l_={$\SI{2}{\ohm}$}]++(-\x-\len/2,0)coordinate(kT);
\draw(kB) to [capacitor,*-,l={$-j2\,\si{\ohm}$}]++(0,-\y) to [capacitor,l={$-j4\,\si{\ohm}$}]++(-\len,0)coordinate(kD) to [resistor,*-o,l={$\SI{1}{\ohm}$}]++(-\x,0)coordinate(kB);
\draw(kC) to [capacitor,*-,l_={$-j2\,\si{\ohm}$}]++(0,-\y);
\draw[stealth-]($(kT)!0.5!(kB)$)++(\x/4,0)--++(-\x/4,0)--++(0,-\y/8)node[below]{$\bZ$};
\end{tikzpicture}
\caption{سوال \حوالہ{سوال_تین_دوری_رکاوٹ_ٹ} کا دور۔}
\label{شکل_سوال_تین_دوری_رکاوٹ_ٹ}
\end{figure}

جوابات:\عددی{\bZ=3.58-j2.12\,\si{\ohm}}
\انتہا{سوال}
%=============================
\ابتدا{سوال}\شناخت{سوال_تین_دوری_رکاوٹ_ث}
شکل \حوالہ{شکل_سوال_تین_دوری_رکاوٹ_ث} کا مساوی رکاوٹ \عددی{\bZ} دریافت کریں۔
\begin{figure}
\centering
\begin{tikzpicture}
\pgfmathsetmacro{\len}{\x*sqrt(2)}
\pgfmathsetmacro{\lenA}{\x/sqrt(2)}
\draw(0,0)coordinate(kBot) to [european resistor,l={$1-j1\,\si{\ohm}$}]++(135:\len)coordinate(kC) to [european resistor,l={$1+j1\,\si{\ohm}$}]++(45:\len) coordinate(kA)to [european resistor,l={$1+j1\,\si{\ohm}$}]++(-45:\len)coordinate(kB) to [european resistor,l={$1-j1\,\si{\ohm}$}]++(-135:\len);
\draw(kA) to [short,*-]++(0,\y/8) to [resistor,-o,l_={$\SI{1}{\ohm}$}]++(-\x-\lenA,0)coordinate(kT);
\draw(kBot) to [short,*-]++(0,-\y/8) to [inductor,-o,l={$j1\,\si{\ohm}$}]++(-\x-\lenA,0)coordinate(kD);
\draw(kC) to [resistor,*-*,l_={$\SI{1}{\ohm}$}](kB);
\draw[stealth-]($(kT)!0.5!(kD)$)++(\x/4,0)--++(-\x/4,0)--++(0,-\y/8)node[below]{$\bZ$};
\end{tikzpicture}
\caption{سوال \حوالہ{سوال_تین_دوری_رکاوٹ_ث} کا دور۔}
\label{شکل_سوال_تین_دوری_رکاوٹ_ث}
\end{figure}

جوابات:\عددی{\bZ=2+j1\,\si{\ohm}}
\انتہا{سوال}
%=============================
\ابتدا{سوال}
متوازن ستارہ بوجھ کو ستارہ منبع \عددی{abc} سے طاقت مہیا کیا جاتا ہے۔دباو تار \عددی{\SI{215}{\volt}\,\rms} ہے جبکہ ستارہ بوجھ \عددی{12+j8\,\si{\ohm}} ہے۔\عددی{\phase{\hat{V}_{an}}=0^{\circ}} لیتے ہوئے تینوں تار کی رو دریافت کریں۔

جوابات:\عددی{\hat{I}_a=8.61\phase{-33.7^{\circ}}\,\si{\ampere}\,\rms}، \عددی{\hat{I}_b=8.61\phase{-153.7^{\circ}}\,\si{\ampere}\,\rms}، \عددی{\hat{I}_c=8.61\phase{86.31^{\circ}}\,\si{\ampere}\,\rms}
\انتہا{سوال}
%==========================
\ابتدا{سوال}
متوازن ستارہ بوجھ کو ستارہ منبع \عددی{abc} سے طاقت مہیا کیا جاتا ہے۔تار کی رکاوٹ \عددی{0.5+j0.8\,\si{\ohm}}، منبع پر دباو
 \عددی{\hat{V}_{an}=\tfrac{240}{\sqrt{3}}\phase{30^{\circ}}\,\rms} ہے جبکہ ستارہ بوجھ \عددی{6+j4\,\si{\ohm}} ہے۔تینوں تار کی رو دریافت کریں۔

جوابات:\عددی{\hat{I}_a=17.1\phase{-6.4^{\circ}}\,\si{\ampere}\,\rms}، \عددی{\hat{I}_b=17.1\phase{-126.4^{\circ}}\,\si{\ampere}\,\rms}، \عددی{\hat{I}_c=17.1\phase{113.6^{\circ}}\,\si{\ampere}\,\rms}
\انتہا{سوال}
%==========================
\ابتدا{سوال}
متوازن ستارہ بوجھ کو ستارہ منبع \عددی{abc} سے طاقت مہیا کیا جاتا ہے۔تار کی رکاوٹ \عددی{0.2+j0.6\,\si{\ohm}}، منبع پر دباو  \عددی{\hat{V}_{ab}=460\phase{45^{\circ}}\,\rms} ہے جبکہ تار کی رو \عددی{\hat{I}_a=78\phase{34^{\circ}}\,\si{\ampere}\,\rms} ہے۔ستارہ بوجھ کی رکاوٹ دریافت کریں۔

جوابات:\عددی{\bZ_Y=3.22-j1.11\,\si{\ohm}}
\انتہا{سوال}
%==========================
\ابتدا{سوال}
متوازن تکون بوجھ کو ستارہ منبع \عددی{abc} سے طاقت مہیا کیا جاتا ہے۔منبع پر دباو  \عددی{\hat{V}_{ab}=440\phase{20^{\circ}}\,\rms} ہے جبکہ تکونی بوجھ
 \عددی{\bZ_{\Delta}=15+j12\,\si{\ohm}} ہے۔رو تار دریافت کریں۔

جوابات:\عددی{\hat{I}_a=39.7\phase{-48.7^{\circ}}\,\si{\ampere}\,\rms}، \عددی{\hat{I}_b=39.7\phase{-168.7^{\circ}}\,\si{\ampere}\,\rms}، \عددی{\hat{I}_c=39.7\phase{71.3^{\circ}}\,\si{\ampere}\,\rms}
\انتہا{سوال}
%==========================
\ابتدا{سوال}
متوازن تکون بوجھ کو ستارہ منبع \عددی{abc} سے طاقت مہیا کیا جاتا ہے۔منبع پر دباو  \عددی{\hat{V}_{ab}=380\phase{80^{\circ}}\,\rms} ہے، تکونی بوجھ
 \عددی{\bZ_{\Delta}=6+j9\,\si{\ohm}} اور تار کی رکاوٹ \عددی{0.1+j0.2\,\si{\ohm}} ہے۔ستارہ منبع کی رو \عددی{\hat{I}_{an}} اور بوجھ کی رو \عددی{\hat{I}_{AB}} دریافت کریں۔

جوابات:\عددی{\hat{I}_{an}=57.3\phase{-6.7^{\circ}}\,\si{\ampere}\,\rms}، \عددی{\hat{I}_{AB}=99.3\phase{23.3^{\circ}}\,\si{\ampere}\,\rms}
\انتہا{سوال}
%==========================
\ابتدا{سوال}
متوازن ستارہ بوجھ کو ستارہ منبع \عددی{abc} سے طاقت مہیا کیا جاتا ہے۔بوجھ پر دباو  \عددی{\hat{V}_{AN}=215\phase{17^{\circ}}\,\rms} ہے،  بوجھ
 \عددی{\bZ_{Y}=8+j8\,\si{\ohm}} اور تار کی رکاوٹ \عددی{1+j2\,\si{\ohm}} ہے۔ستارہ منبع کا دباو \عددی{\hat{V}_{an}} دریافت کریں۔

جواب: \عددی{\hat{V}_{an}=256\phase{20^{\circ}}\,\si{\volt}\,\rms}
\انتہا{سوال}
%==========================
\ابتدا{سوال}
متوازن ستارہ بوجھ کو ستارہ منبع \عددی{abc} سے طاقت مہیا کیا جاتا ہے۔بوجھ پر دباو  \عددی{\hat{V}_{AN}=120\phase{33^{\circ}}\,\rms} ہے، بوجھ
 \عددی{\bZ_{Y}=2+j3\,\si{\ohm}} اور تار کی رکاوٹ \عددی{0.8+j1\,\si{\ohm}} ہے۔ستارہ منبع پر دباو \عددی{\hat{V}_{ab}} دریافت کریں۔

جواب: \عددی{\hat{V}_{ab}=281\phase{61.7^{\circ}}\,\si{\volt}\,\rms}
\انتہا{سوال}
%==========================
\ابتدا{سوال}
متوازن تکون بوجھ کو ستارہ منبع \عددی{abc} سے طاقت مہیا کیا جاتا ہے۔منبع دباو  \عددی{\hat{V}_{an}=120\phase{40^{\circ}}\,\rms} ہے، بوجھ
 \عددی{\bZ_{\Delta}=24+j18\,\si{\ohm}} اور تار کی رکاوٹ \عددی{0.5+j0.4\,\si{\ohm}} ہے۔تکونی بوجھ کی رو دریافت کریں۔

جوابات: \عددی{\hat{I}_{AB}=19.5\phase{33^{\circ}}\,\si{\ampere}\,\rms}،  \عددی{\hat{I}_{BC}=19.5\phase{-87^{\circ}}\,\si{\ampere}\,\rms}،  \عددی{\hat{I}_{CA}=19.5\phase{153^{\circ}}\,\si{\ampere}\,\rms}
\انتہا{سوال}
%==========================
\ابتدا{سوال}
متوازن ستارہ بوجھ کو ستارہ منبع \عددی{abc} سے طاقت مہیا کیا جاتا ہے۔منبع دباو  \عددی{\hat{V}_{an}=120\phase{0^{\circ}}\,\rms}، بوجھ پر دباو
 \عددی{\hat{V}_{AN}=111.62\phase{-1.33^{\circ}}\,\si{\volt}\,\rms} اور بوجھ  \عددی{\bZ_{Y}=8+j4\,\si{\ohm}} ہے۔اور تار کی رکاوٹ  دریافت کریں۔

جواب: \عددی{0.499+j0.499\,\si{\ohm}}
\انتہا{سوال}
%==========================
\ابتدا{سوال}
متوازن ستارہ بوجھ کو ستارہ منبع \عددی{abc} سے طاقت مہیا کیا جاتا ہے۔جزو بوجھ \عددی{0.8} امالی  جبکہ دباو  بوجھ \عددی{\hat{V}_{AN}=210\phase{0^{\circ}}\,\rms} ہے۔کل تار کا ضیاع \عددی{\SI{300}{\watt}} ہے۔تار کی رکاوٹ \عددی{0.8+j1.2\,\si{\ohm}} ہے۔بوجھ کی رکاوٹ دریافت کریں۔

جواب:\عددی{15-j11.3\,\si{\ohm}}
\انتہا{سوال}
%==========================
\ابتدا{سوال}
ستارہ بوجھ \عددی{10+j16\,\si{\ohm}} پر دباو \عددی{\hat{V}_{an}=220\phase{0^{\circ}}\,\rms} ہے۔منبع 
دباو\عددی{\hat{V}_{AN}=235\phase{7^{\circ}}\,\rms} ہے۔متوازن بوجھ قصر دور ہونے پر رو تار کی مقدار حاصل کریں۔

جواب:\عددی{\hat{I}_a=86.8\phase{-116^{\circ}}\,\si{\ampere}\,\rms}
\انتہا{سوال}
%==========================
\ابتدا{سوال}
ستارہ بوجھ \عددی{10+j8\,\si{\ohm}} کو \عددی{1.4+0.6j\,\si{\ohm}} رکاوٹ کے تار سے طاقت مہیا کیا جاتا ہے۔بوجھ پر دباو کا زاویہ \عددی{\phase{\hat{V}_{AN}}=45^{\circ}} ہے۔تاروں میں کل طاقت کا ضیاع \عددی{\SI{450}{\watt}} ہے۔ دباو بوجھ اور دباو منبع حاصل کریں۔

جواب:\عددی{\hat{V}_{AN}=132.6\phase{45^{\circ}}\,\si{\volt}\,\rms}، \عددی{\hat{V}_{an}=147.8\phase{43.4^{\circ}}\,\si{\volt}\,\rms}
\انتہا{سوال}
%==========================
\ابتدا{سوال}
ستارہ بوجھ \عددی{10+j8\,\si{\ohm}} کو \عددی{1.4+0.6j\,\si{\ohm}} رکاوٹ کے تار سے طاقت مہیا کیا جاتا ہے۔بوجھ کا کل طاقت \عددی{\SI{15}{\kilo\watt}} ہے۔تاروں میں کل طاقت کا ضیاع دریافت کریں۔

جواب:\عددی{\SI{2.1}{\kilo\watt}}
\انتہا{سوال}
%==========================

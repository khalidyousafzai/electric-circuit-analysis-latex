\باب{سوالات مسئلے}
%==================
\ابتدا{سوال}\شناخت{سوال_مسئلے_خطیت_الف}
شکل \حوالہ{شکل_سوال_مسئلے_خطیت_الف} میں \عددی{V_0=\SI{2}{\volt}} فرض کرتے ہوئے مسئلہ خطیت کے استعمال سے \عددی{V_0} دریافت کریں۔
\begin{figure}
\centering
\begin{tikzpicture}[american voltages]
\draw(0,0)to [resistor,l={$\SI{2}{\kilo\ohm}$}]++(0,2*\y);
\draw(\x,0)to [resistor,*-,l={$\SI{1}{\kilo\ohm}$}]++(0,\y) to [short,-*]++(0,\y);
\draw(2*\x,0) to [short,*-]++(0,\y) to [resistor,-*,l={$\SI{1}{\kilo\ohm}$}]++(0,\y);
\draw(3*\x,0)to [resistor,l={$\SI{2}{\kilo\ohm}$},v_>={$V_0$}]++(0,2*\y);
\draw(0,0) to [short]++(3*\x,0);
\draw(0,2*\y) to [short]++(\x,0) to [resistor,l_={$\SI{1}{\kilo\ohm}$}]++(\x,0) to [short]++(\x,0);
\draw(\x,\y) to [american current source,*-*,l={$\SI{4}{\milli\ampere}$}]++(\x,0);
\end{tikzpicture}
\caption{سوال \حوالہ{سوال_مسئلے_خطیت_الف} کا دور۔}
\label{شکل_سوال_مسئلے_خطیت_الف}
\end{figure}

جواب:\عددی{V_0=-\tfrac{16}{21}\,\si{\volt}}
\انتہا{سوال}
%======================
\ابتدا{سوال}\شناخت{سوال_مسئلے_خطیت_ب}
شکل \حوالہ{شکل_سوال_مسئلے_خطیت_ب} میں \عددی{I_0=\SI{1}{\milli\ampere}} فرض کرتے ہوئے مسئلہ خطیت کے استعمال سے \عددی{I_0} دریافت کریں۔
\begin{figure}
\centering
\begin{tikzpicture}[american voltages]
\draw(0,0)to [resistor,l={$\SI{10}{\kilo\ohm}$}]++(0,\y);
\draw(\x,\y)to [american current source,*-*,l_={$\SI{6}{\milli\ampere}$}]++(0,-\y);
\draw(2*\x,0) to [resistor,*-*,l={$\SI{6}{\kilo\ohm}$}]++(0,\y);
\draw(3*\x,0)to [resistor,l_={$\SI{2}{\kilo\ohm}$},i_<={$I_0$}]++(0,\y);
\draw(0,0) to [short]++(3*\x,0);
\draw(0,\y) to [resistor,l={$\SI{2}{\kilo\ohm}$}]++(\x,0) to [resistor,l={$\SI{4}{\kilo\ohm}$}]++(\x,0) to [resistor,l={$\SI{4}{\kilo\ohm}$}]++(\x,0);
\end{tikzpicture}
\caption{سوال \حوالہ{سوال_مسئلے_خطیت_ب} کا دور۔}
\label{شکل_سوال_مسئلے_خطیت_ب}
\end{figure}

جواب:\عددی{I_0=\SI{-1.895}{\milli\ampere}}
\انتہا{سوال}
%======================

\باب{سوالات لاپلاس}

%==================
\ابتدا{سوال}\شناخت{سوال_لاپلاس_حل_رکاوٹ_الف}
شکل \حوالہ{شکل_سوال_لاپلاس_حل_رکاوٹ_الف} کی داخلی رکاوٹ \عددی{\bZ(s)} حاصل کریں۔
\begin{figure}
\centering
\begin{tikzpicture}
\draw(0,0) to [short,o-]++(2*\x,0) to [resistor,l={$\SI{2}{\ohm}$}]++(\x,0) to [capacitor,l={$\SI{0.5}{\farad}$}]++(0,-\y) to [resistor,l={$\SI{2}{\ohm}$}]++(-\x,0) to [short,-o]++(-2*\x,0);
\draw(\x,0) to [resistor,*-*,l_={$\SI{1}{\ohm}$}]++(0,-\y);
\draw(\x+\x/2,0) to [inductor,*-*,l={$\SI{1}{\henry}$}]++(0,-\y);
\draw(3*\x,0) to [resistor,*-*,l={$\SI{1}{\ohm}$}]++(-\x-\x/2,-\y);
\draw[stealth-](\x/4,-\y/2)--++(-\x/4,0)--++(0,-\y/8)node[below]{$\bZ(s)$};
\end{tikzpicture}
\caption{سوال \حوالہ{سوال_لاپلاس_حل_رکاوٹ_الف} کا دور۔}
\label{شکل_سوال_لاپلاس_حل_رکاوٹ_الف}
\end{figure}

جواب:\عددی{\bZ(s)=\tfrac{2s(4s+3)}{11s^2+16s+6}}
\انتہا{سوال}
%=====================
\ابتدا{سوال}\شناخت{سوال_لاپلاس_حل_رکاوٹ_ب}
شکل \حوالہ{شکل_سوال_لاپلاس_حل_رکاوٹ_ب} میں \عددی{c} اور \عددی{d} کو کھلے سر رکھتے ہوئے \عددی{a} اور \عددی{b} کے مابین رکاوٹ دریافت کریں۔
\begin{figure}
\centering
\begin{tikzpicture}
\pgfmathsetmacro{\ang}{atan(\yy/(\xx+\xx/2))}
\pgfmathsetmacro{\len}{\yy/sin(\ang)}
\draw(0,0)node[left]{$a$} to [short,o-]++(\xx/2,0) to [resistor,l={$\SI{1}{\ohm}$}]++(\xx+\xx/2,0) to [short,-o]++(\xx/2,0)node[right]{$c$};
\draw(0,0-\yy)node[left]{$b$} to [short,o-]++(\xx/2,0) to [capacitor,l_={$\SI{0.5}{\farad}$}]++(\xx+\xx/2,0) to [short,-o]++(\xx/2,0)node[right]{$d$};
\draw(\x/2,0) to [resistor,*-,l_={$\SI{2}{\ohm}$}]++(-\ang:\xx) to [short,-*]++(-\ang:\len-\xx);
\draw(\xx+\xx,0) to [inductor,*-,l={$\SI{1}{\henry}$}]++(-180+\ang:\xx) to [short,-*]++(-180+\ang:\len-\xx);
\end{tikzpicture}
\caption{سوال \حوالہ{سوال_لاپلاس_حل_رکاوٹ_ب} اور سوال \حوالہ{سوال_لاپلاس_حل_رکاوٹ_پ} کا دور۔}
\label{شکل_سوال_لاپلاس_حل_رکاوٹ_ب}
\end{figure}

جواب:\عددی{\bZ(s)=\tfrac{2s+2}{s+2}}
\انتہا{سوال}
%====================
\ابتدا{سوال}\شناخت{سوال_لاپلاس_حل_رکاوٹ_پ}
شکل \حوالہ{شکل_سوال_لاپلاس_حل_رکاوٹ_ب} میں \عددی{c} اور \عددی{d} کو آپس میں قصر دور کرتے ہوئے \عددی{a} اور \عددی{b} کے مابین رکاوٹ دریافت کریں۔

جواب:\عددی{\bZ(s):\tfrac{2s^2+6s+4}{3s^2+6}}
\انتہا{سوال}
%===========================
\ابتدا{سوال}\شناخت{سوال_لاپلاس_حل_رکاوٹ_ت}
شکل \حوالہ{شکل_سوال_لاپلاس_حل_رکاوٹ_ت}-الف میں \عددی{v_C(t)} حاصل کریں۔
\begin{figure}
\centering
\begin{subfigure}{0.5\textwidth}
\centering
\begin{tikzpicture}[american voltages]
\draw(0,0) to [american voltage source,l={$10\,u(t)\,\si{\volt}$}]++(0,2*\y) to [resistor,l={$\SI{2}{\ohm}$}]++(\x,0) to [resistor,l={$\SI{2}{\ohm}$}]++(\x,0) to [inductor,l={$\SI{1}{\henry}$},v={$v_L(t)$}]++(0,-2*\y) to [short](0,0);
\draw(\x,0) to[capacitor,*-,l={$\SI{0.5}{\farad}$},v_>={$v_C(t)$}]++(0,\y) to  [resistor,-*,l={$\SI{1}{\ohm}$}]++(0,\y);
\end{tikzpicture}
\caption*{(الف)}
\end{subfigure}%
\begin{subfigure}{0.5\textwidth}
\centering
\begin{tikzpicture}[american voltages]
\draw(0,0) to [american voltage source,l_={$6\,u(t)\,\si{\volt}$}]++(0,2*\y) to [resistor,l={$\SI{2}{\ohm}$}]++(\x,0) to [resistor,l={$\SI{2}{\ohm}$}]++(\x,0) to [capacitor,l={$\SI{1}{\farad}$},v={$v_0(t)$}]++(0,-2*\y) to [short](0,0);
\draw(\x,0) to [resistor,*-,l={$\SI{2}{\ohm}$}]++(0,\y) to [inductor,-*,l={$\SI{2}{\henry}$}]++(0,\y);
\end{tikzpicture}
\caption*{(ب)}
\end{subfigure}%
\caption{سوال \حوالہ{سوال_لاپلاس_حل_رکاوٹ_ت} تا سوال \حوالہ{سوال_لاپلاس_حل_رکاوٹ_ث} کے ادوار۔}
\label{شکل_سوال_لاپلاس_حل_رکاوٹ_ت}
\end{figure}

جواب:\عددی{v_C(t)=[5-5e^{-\tfrac{4}{3}t}]\,u(t)\,\si{\volt}}

\انتہا{سوال}
%================
\ابتدا{سوال}\شناخت{سوال_لاپلاس_حل_رکاوٹ_ٹ}
شکل \حوالہ{شکل_سوال_لاپلاس_حل_رکاوٹ_ت}-الف میں \عددی{v_L(t)} حاصل کریں۔

جواب:\عددی{v_L(t)=\tfrac{10}{3}e^{-\tfrac{4}{3}t}\, u(t)\,\si{\volt}}
\انتہا{سوال}
%=========================
\ابتدا{سوال}\شناخت{سوال_لاپلاس_حل_رکاوٹ_ث}
شکل \حوالہ{شکل_سوال_لاپلاس_حل_رکاوٹ_ت}-ب میں \عددی{v_0(t)} حاصل کریں۔

جواب:
$v_0(t)=\frac{1}{2\sqrt{17}}\left[6\sqrt{17}-(9+3\sqrt{17})e^{-\tfrac{7}{8}t}+(9-3\sqrt{17})e^{-\big(\tfrac{7+\sqrt{17}}{8}\big)t}\right]\,u(t)$
\انتہا{سوال}
%====================
\ابتدا{سوال}\شناخت{سوال_لاپلاس_حل_رکاوٹ_ج}
شکل \حوالہ{شکل_سوال_لاپلاس_حل_رکاوٹ_ج} میں \عددی{v_0(t)} حاصل کریں۔
\begin{figure}
\centering
\begin{tikzpicture}[american voltages]
\draw(0,0) to [resistor,l={$\SI{6}{\ohm}$}]++(0,\y);
\draw(\x+\x/2,0) to [american current source,*-*,l={$10\,u(t)\,\si{\ampere}$}]++(0,\y);
\draw(2*\x+\x/2,0) to [capacitor,*-*,l={$\SI{1}{\farad}$}]++(0,\y);
\draw(3*\x+\x/2,0) to [resistor,l={$\SI{1}{\ohm}$},v_>={$v_0(t)$}]++(0,\y);
\draw(0,0) to [short]++(3*\x+\x/2,0);
\draw(0,\y) to [resistor,l={$\SI{4}{\ohm}$}]++(\x+\x/2,0) to [inductor,l={$\SI{1}{\henry}$}]++(\x,0) to [resistor,l={$\SI{2}{\ohm}$}]++(\x,0);
\end{tikzpicture}
\caption{سوال \حوالہ{سوال_لاپلاس_حل_رکاوٹ_ج} کا دور۔}
\label{شکل_سوال_لاپلاس_حل_رکاوٹ_ج}
\end{figure}

جواب:
$v_0(t)=\tfrac{100}{13}[1-e^{-\tfrac{31}{6}t}(\cosh \tfrac{\sqrt{805}t}{6}+\tfrac{31}{\sqrt{805}}\sinh \tfrac{\sqrt{805}t}{6})]\, u(t)\,\si{\volt}$
\انتہا{سوال}
%================
\ابتدا{سوال}\شناخت{سوال_لاپلاس_حل_رکاوٹ_چ}
شکل \حوالہ{شکل_سوال_لاپلاس_حل_رکاوٹ_چ} میں \عددی{v_C(t)} حاصل کریں۔
\begin{figure}
\centering
\begin{tikzpicture}[american voltages]
\draw(0,0) to [american voltage source,l={$3e^{-t}\,u(t)\,\si{\volt}$}]++(0,\y) to [resistor,l={$\SI{2}{\ohm}$}]++(\x,0) to [inductor,l={$\SI{1}{\henry}$}]++(\x,0) to [capacitor,l={$\SI{0.25}{\farad}$},v={$v_C(t)$}]++(0,-\y) to [short] (0,0);
\end{tikzpicture}
\caption{سوال \حوالہ{سوال_لاپلاس_حل_رکاوٹ_چ} کا دور۔}
\label{شکل_سوال_لاپلاس_حل_رکاوٹ_چ}
\end{figure}

جواب:$v_C(t)=4e^{-t}(1-\cos \sqrt{3}t)\,u(t)\,\si{\volt}$
\انتہا{سوال}
%================
\ابتدا{سوال}\شناخت{سوال_لاپلاس_حل_رکاوٹ_ح}
شکل \حوالہ{شکل_سوال_لاپلاس_حل_رکاوٹ_ح} میں \عددی{i_x(t)} حاصل کریں۔
\begin{figure}
\centering
\begin{tikzpicture}[american voltages]
\draw(0,0) to [american voltage source,l={$12\,u(t)\,\si{\volt}$}]++(0,\y) to [resistor,l={$\SI{1}{\ohm}$},i={$i_x(t)$}]++(\x,0) to [capacitor,l={$\SI{1}{\farad}$}]++(\x,0) to [american voltage source,l={$6\,u(t)\,\si{\volt}$}]++(0,-\y) to [short] (0,0);
\draw(\x,0) to [inductor,*-*,l={$\SI{1}{\henry}$}]++(0,\y);
\end{tikzpicture}
\caption{سوال \حوالہ{سوال_لاپلاس_حل_رکاوٹ_ح} کا دور۔}
\label{شکل_سوال_لاپلاس_حل_رکاوٹ_ح}
\end{figure}

جواب:
$i_x(t)=[12-e^{-\tfrac{t}{2}}(10\sqrt{3} \sin \tfrac{\sqrt{3}t}{2}-6\cos \tfrac{\sqrt{3}t}{2})]\,u(t)\,\si{\ampere}$
\انتہا{سوال}
%================
\ابتدا{سوال}\شناخت{سوال_لاپلاس_حل_رکاوٹ_خ}
شکل \حوالہ{شکل_سوال_لاپلاس_حل_رکاوٹ_خ} میں \عددی{v_x=8\,u(t)\,\si{\volt}} ہے۔آپ سے گزارش ہے کہ \عددی{v_C(t)} حاصل کریں۔
\begin{figure}
\centering
\begin{tikzpicture}[american voltages]
\draw(0,0) to [american voltage source,l={$v_x(t)$}]++(0,\y) to [resistor,l={$\SI{1}{\ohm}$}]++(\x,0) to [resistor,l={$\SI{2}{\ohm}$}]++(\x,0) to [american voltage source,l={$4\,u(t)\,\si{\volt}$}]++(0,-\y) to [short] (0,0);
\draw(\x,0) to [capacitor,*-*,l={$\SI{1}{\farad}$},v_>={$v_C(t)$}]++(0,\y);
\end{tikzpicture}
\caption{سوال \حوالہ{سوال_لاپلاس_حل_رکاوٹ_خ} کا دور۔}
\label{شکل_سوال_لاپلاس_حل_رکاوٹ_خ}
\end{figure}

جواب:
$v_C(t)=4(1-e^{-\tfrac{3}{2}t})\,u(t)\,\si{\volt}$
\انتہا{سوال}
%================
\ابتدا{سوال}\شناخت{سوال_لاپلاس_حل_رکاوٹ_د}
شکل \حوالہ{شکل_سوال_لاپلاس_حل_رکاوٹ_خ} میں \عددی{v_x=8e^{-t}\,u(t)\,\si{\volt}} ہے۔ \عددی{v_C(t)} حاصل کریں۔

جواب:
$v_C(t)=\left(16e^{-t}-\tfrac{44}{3}e^{-\tfrac{3}{2}t}-\tfrac{4}{3}\right)\,u(t) \,\si{\volt}$
\انتہا{سوال}
%==========================
\ابتدا{سوال}\شناخت{سوال_لاپلاس_حل_رکاوٹ_ڈ}
شکل \حوالہ{شکل_سوال_لاپلاس_حل_رکاوٹ_ڈ} میں  \عددی{v_C(t)} حاصل کریں۔
\begin{figure}
\centering
\begin{tikzpicture}[american voltages]
\draw(0,0) to [american voltage source,l={$17\cos t \,u(t)\,\si{\volt}$}]++(0,\y) to [resistor,l={$\SI{1}{\ohm}$}]++(\x,0) to [resistor,l={$\SI{1}{\ohm}$}]++(\x,0) to [american voltage source,l={$1\,u(t)\,\si{\volt}$}]++(0,-\y) to [short] (0,0);
\draw(\x,0) to [capacitor,*-*,l={$\SI{0.5}{\farad}$},v_>={$v_C(t)$}]++(0,\y);
\end{tikzpicture}
\caption{سوال \حوالہ{سوال_لاپلاس_حل_رکاوٹ_ڈ} کا دور۔}
\label{شکل_سوال_لاپلاس_حل_رکاوٹ_ڈ}
\end{figure}

جواب:
$v_C(t)=\left(8\cos t+2\sin t-7.5e^{-4t}-0.5\right)\,u(t)\,\si{\volt}$
\انتہا{سوال}
%================
\ابتدا{سوال}\شناخت{سوال_لاپلاس_حل_دور_الف}
شکل \حوالہ{شکل_سوال_لاپلاس_حل_دور_الف} میں \عددی{v_s(t)=4 \,u(t)\,\si{\volt}} ہے۔آپ سے گزارش ہے کہ  \عددی{v_x(t)} حاصل کریں۔
\begin{figure}
\centering
\begin{tikzpicture}[american voltages]
\draw(0,0) to [american voltage source,l={$v_s(t)$}]++(0,\y) to [inductor,l={$\SI{1}{\henry}$}]++(\x,0) to [resistor,l={$\SI{1}{\ohm}$}]++(\x,0) to [american voltage source,l={$e^{-t}\,u(t)\,\si{\volt}$}]++(0,-\y) to [short] (0,0);
\draw(\x,0) to [resistor,*-*,l={$\SI{2}{\ohm}$},v_>={$v_x(t)$}]++(0,\y);
\draw(0,\y) to [short,*-]++(0,3/4*\y) to [resistor,l={$\SI{1}{\ohm}$}]++(\x,0) to [capacitor,l={$\SI{1}{\farad}$}]++(\x,0) to [short,-*]++(0,-3/4*\y);
\end{tikzpicture}
\caption{سوال \حوالہ{سوال_لاپلاس_حل_دور_الف} کا دور۔}
\label{شکل_سوال_لاپلاس_حل_دور_الف}
\end{figure}

جواب:
$v_x(t)=[4-2e^{-t}-\tfrac{8}{3}e^{-\tfrac{2}{3}t}]\,u(t)\,\si{\volt}$
\انتہا{سوال}
%================
\ابتدا{سوال}\شناخت{سوال_لاپلاس_حل_دور_ب}
شکل \حوالہ{شکل_سوال_لاپلاس_حل_دور_الف} میں \عددی{v_s(t)=4e^{-2t} \,u(t)\,\si{\volt}} ہے۔آپ سے گزارش ہے کہ  \عددی{v_x(t)} حاصل کریں۔

جواب:
$v_x(t)=[\tfrac{10}{3}e^{-\tfrac{2}{3}t}-2e^{-t}-2e^{-2t}]\,u(t)\,\si{\volt}$
\انتہا{سوال}
%===============================
\ابتدا{سوال}\شناخت{سوال_لاپلاس_حل_دور_پ}
شکل \حوالہ{شکل_سوال_لاپلاس_حل_دور_پ} میں \عددی{v_0(t)} حاصل کریں۔
\begin{figure}
\centering
\begin{tikzpicture}[american voltages]
\draw(0,0) to [american voltage source,l={$6\,u(t)\,\si{\volt}$}]++(0,\y) to [inductor,l={$\SI{1}{\henry}$}]++(\x+\x/2,0) to [capacitor,l={$\SI{0.5}{\farad}$}]++(\x,0) to [resistor,l={$\SI{2}{\ohm}$},v={$v_0(t)$}]++(0,-\y) to [short] (0,0);
\draw(\x+\x/2,0) to [american current source,*-*,l={$2\,u(t)\,\si{\ampere}$}]++(0,\y);
\end{tikzpicture}
\caption{سوال \حوالہ{سوال_لاپلاس_حل_دور_پ} کا دور۔}
\label{شکل_سوال_لاپلاس_حل_دور_پ}
\end{figure}

جواب:
$v_0(t)=e^{-t}(4\cos t+8\sin t)\,u(t)\,\si{\volt}$
\انتہا{سوال}
%============================
\ابتدا{سوال}\شناخت{سوال_لاپلاس_حل_دور_ت}
شکل \حوالہ{شکل_سوال_لاپلاس_حل_دور_ت} میں \عددی{v_0(t)} حاصل کریں۔
\begin{figure}
\centering
\begin{tikzpicture}[american voltages]
\draw(0,0) to [american voltage source,l={$6\,u(t)\,\si{\volt}$}]++(0,\y) to [capacitor,l={$\SI{0.5}{\farad}$}]++(\x+\x/2,0) to [inductor,l={$\SI{1}{\henry}$}]++(\x,0)  to [resistor,l={$\SI{2}{\ohm}$},v={$v_0(t)$}]++(0,-\y) to [short] (0,0);
\draw(\x+\x/2,0) to [american current source,*-*,l={$2\,u(t)\,\si{\ampere}$}]++(0,\y);
\end{tikzpicture}
\caption{سوال \حوالہ{سوال_لاپلاس_حل_دور_ت} کا دور۔}
\label{شکل_سوال_لاپلاس_حل_دور_ت}
\end{figure}

جواب:
$v_0(t)=[4-e^{-t}(4\cos t-8\sin t)]\,u(t)\,\si{\volt}$
\انتہا{سوال}
%=============================
\ابتدا{سوال}\شناخت{سوال_لاپلاس_حل_دور_ٹ}
شکل \حوالہ{شکل_سوال_لاپلاس_حل_دور_ٹ} میں \عددی{v_0(t)} حاصل کریں۔
\begin{figure}
\centering
\begin{tikzpicture}[american voltages]
\draw(0,0) to [american current source,l={$2\,u(t)\,\si{\ampere}$}]++(0,2*\y) to [resistor,l={$\SI{1}{\ohm}$}]++(\x,0) to [capacitor,l={$\SI{0.5}{\farad}$}]++(\x,0) to [inductor,l={$\SI{1}{\henry}$}]++(\x,0)  to [resistor,l={$\SI{1}{\ohm}$},v={$v_0(t)$}]++(0,-2*\y) to [short] (0,0);
\draw(\x,0) to [american voltage source,*-,l={$4\,u(t)\,\si{\volt}$}]++(0,\y) to [resistor,-*,l={$\SI{1}{\ohm}$}]++(0,\y);
\draw(2*\x,0) to [resistor,*-*,l={$\SI{1}{\ohm}$}]++(0,2*\y);
\end{tikzpicture}
\caption{سوال \حوالہ{سوال_لاپلاس_حل_دور_ٹ} کا دور۔}
\label{شکل_سوال_لاپلاس_حل_دور_ٹ}
\end{figure}

جواب:
$v_0(t)=\tfrac{12}{\sqrt{15}}e^{-\tfrac{7}{4}t}\sin \tfrac{\sqrt{15} t}{4} \, u(t)\,\si{\volt}$
\انتہا{سوال}
%=======================
\ابتدا{سوال}\شناخت{سوال_لاپلاس_حل_دور_ث}
شکل \حوالہ{شکل_سوال_لاپلاس_حل_دور_ث} میں \عددی{v_0(t)} حاصل کریں۔
\begin{figure}
\centering
\begin{tikzpicture}[american voltages]
\draw(0,0) to [resistor,l={$\SI{1}{\ohm}$}]++(0,\y) to [american current source,l={$4e^{-t}\,u(t)\,\si{\ampere}$}]++(0,\y);
\draw(\x+\x/2,2*\y) to [capacitor,*-,l_={$\SI{1}{\farad}$}]++(0,-\y) to [american current source,-*,l_={$2\,u(t)\,\si{\ampere}$}]++(0,-\y);
\draw(2*\x+\x/2,0) to [resistor,l_={$\SI{1}{\ohm}$},v^>={$v_0(t)$}]++(0,\y) to [american current source,l_={$6\,u(t)\,\si{\ampere}$}]++(0,\y);
\draw(0,0) to [short]++(2*\x+\x/2,0);
\draw(0,2*\y) to [short]++(2*\x+\x/2,0);
\draw(0,\y) to [resistor,*-*,l={$\SI{1}{\ohm}$}]++(\x+\x/2,0) to [inductor,-*,l={$\SI{1}{\henry}$}]++(\x,0);
\end{tikzpicture}
\caption{سوال \حوالہ{سوال_لاپلاس_حل_دور_ث} کا دور۔}
\label{شکل_سوال_لاپلاس_حل_دور_ث}
\end{figure}

جواب:
$v_0(t)=[2e^{-t}-\tfrac{22}{3}e^{-3t}-\tfrac{2}{3}]\,u(t)\,\si{\volt}$
\انتہا{سوال}

\باب{سوالات عارضی}

%==================
\ابتدا{سوال}\شناخت{سوال_عارضی_الف}
شکل \حوالہ{شکل_سوال_عارضی_الف}-الف میں  سوئچ منقطع کرنے کے بعد \عددی{i(t)} دریافت کریں۔
\begin{figure}
\centering
\begin{subfigure}{0.3\textwidth}
\centering
\begin{tikzpicture}
\draw(0,0) to [american voltage source,l_={$\SI{20}{\volt}$}]++(0,2*\y) to [ospst,l={${t=0}$}]++(\x,0) to [resistor,l={$\SI{10}{\ohm}$}]++(0,-\y) to [inductor,l_={$\SI{4}{\henry}$}]++(0,-\y) to [short]++(-\x,0);
\draw(\x,\y) to [short,*-]++(\x/2,0) to [resistor,l={$\SI{12}{\ohm}$},i<={$i(t)$}]++(0,-\y) to [short,-*]++(-\x/2,0); 
\end{tikzpicture}
\caption*{(الف)}
\end{subfigure}%
\begin{subfigure}{0.7\textwidth}
\centering
\begin{tikzpicture}
\draw(0,0) to [american current source,l={$\SI{8}{\milli\ampere}$}]++(0,\y) to [ospst,l={${t=0}$}]++(\x,0) to [resistor,l_={$\SI{4}{\kilo\ohm}$},i={$i(t)$}]++(0,-\y)  to [short]++(-\x,0);
\draw(\x,\y) to [short,*-]++(\x,0) to [capacitor,l_={$\SI{100}{\micro\farad}$}]++(0,-\y) to [short,-*]++(-\x,0); 
\draw(0,0) to [short,*-]++(-\x,0) to [resistor,l={$\SI{4}{\kilo\ohm}$}]++(0,\y) to [short,-*]++(\x,0);
\end{tikzpicture}
\caption*{(ب)}
\end{subfigure}%
\caption{سوال \حوالہ{سوال_عارضی_الف} اور سوال \حوالہ{سوال_عارضی_ب} کے ادوار۔}
\label{شکل_سوال_عارضی_الف}
\end{figure}

جواب:\عددی{i(t)=2e^{-3t}\,\si{\ampere}}
\انتہا{سوال}
%====================
\ابتدا{سوال}\شناخت{سوال_عارضی_ب}
شکل \حوالہ{شکل_سوال_عارضی_الف}-ب میں  سوئچ منقطع کرنے کے بعد \عددی{i(t)} دریافت کریں۔

جواب:\عددی{i(t)=4e^{-\tfrac{5t}{2}}\,\si{\milli\ampere}}
\انتہا{سوال}
%====================
\ابتدا{سوال}\شناخت{سوال_عارضی_پ}
شکل \حوالہ{شکل_سوال_عارضی_پ} میں  \عددی{t=0} پر سوئچ کو منبع کی جانب کر دیا جاتا ہے۔اس لمحے  کے بعد \عددی{v_0(t)} دریافت کریں۔
\begin{figure}
\centering
\begin{tikzpicture}[american voltages]
\draw(0,0) node[spdt,xscale=-1,yscale=-1](sw1){};
\draw[-stealth,thick]([shift={(200:0.7)}]sw1.in) arc (200:140:0.7)node[above]{$t=0$};
\draw(sw1.out 2)--++(-\x/2,0) ++(0,-\y)coordinate(kBL) to [american voltage source,l={$\SI{8}{\volt}$}]++(0,\y);
\draw[name path=kbot](sw1.in) to [resistor,l={$\SI{10}{\kilo\ohm}$}]++(\x,0)coordinate(kT) to [resistor,l={$\SI{4}{\kilo\ohm}$}]++(\x,0) to [resistor,l={$\SI{6}{\kilo\ohm}$},v={$v_0(t)$}]++(0,-\y)coordinate(kBR) -|(kBL);
\path[name path=kvrt](sw1.out 1)--++(0,-\y);
\draw[name intersections={of=kvrt and kbot}] (sw1.out 1)--(intersection-1)node[circ]{};
\draw(kT) to [capacitor,*-*,l_={$\SI{200}{\micro\farad}$}]++(0,-\y);
\end{tikzpicture}
\caption{سوال \حوالہ{سوال_عارضی_پ} کا دور۔}
\label{شکل_سوال_عارضی_پ}
\end{figure}

جواب:\عددی{v_0(t)=\tfrac{12}{5}(1-e^{-t})\,\si{\volt}}
\انتہا{سوال}
%====================
\ابتدا{سوال}\شناخت{سوال_عارضی_ت}
شکل \حوالہ{شکل_سوال_عارضی_ت}-الف میں \عددی{v_0(t)} کو \عددی{t>0} کے لئے حاصل کریں۔ 
\begin{figure}
\centering
\begin{subfigure}{0.5\textwidth}
\centering
\begin{tikzpicture}[american voltages]
\draw(0,0) to [american voltage source,l={$\SI{20}{\volt}$}]++(0,\y) to [ospst,l={${t=0}$}]++(\x,0) to [resistor,l={$\SI{8}{\kilo\ohm}$}]++(\x,0) to [resistor,l={$\SI{8}{\kilo\ohm}$},v={$v_0(t)$}]++(0,-\y) to [short]++(-2*\x,0);
\draw(\x,0) to [resistor,*-*,l={$\SI{8}{\kilo\ohm}$}]++(0,\y) to [short]++(0,3/4*\y) to [capacitor,l={$\SI{200}{\micro\farad}$}]++(\x,0) to [short,-*]++(0,-3/4*\y);
\end{tikzpicture}
\caption*{(الف)}
\end{subfigure}%
\begin{subfigure}{0.5\textwidth}
\centering
\begin{tikzpicture}[american voltages]
\draw(0,0) to [resistor,*-,l={$\SI{4}{\kilo\ohm}$}]++(0,\y) to [american voltage source,l={$\SI{10}{\volt}$}]++(0,\y);
\draw(\x,0) to [resistor,l={$\SI{4}{\kilo\ohm}$},v_>={$v_0(t)$}]++(0,2*\y) to [short,*-]++(0,3/4*\y) to [capacitor,l={$\SI{100}{\micro\farad}$}]++(-\x,0) to [short,-*]++(0,-3/4*\y);
\draw(\x,0) to [short]++(-2*\x,0) to [ospst,l={${t=0}$}]++(0,\y) to [american voltage source,l={$\SI{6}{\volt}$}]++(0,\y) to [short,-*]++(\x,0) to [resistor,l_={$\SI{8}{\kilo\ohm}$}]++(\x,0);
\end{tikzpicture}
\caption*{(ب)}
\end{subfigure}%
\caption{سوال \حوالہ{سوال_عارضی_ت} اور سوال \حوالہ{سوال_عارضی_ٹ} کے ادوار۔}
\label{شکل_سوال_عارضی_ت}
\end{figure}

جواب:\عددی{v_0(t)=-5e^{-\tfrac{15t}{16}}\,\si{\volt}}
\انتہا{سوال}
%=====================
\ابتدا{سوال}\شناخت{سوال_عارضی_ٹ}
شکل \حوالہ{شکل_سوال_عارضی_ت}-ب میں \عددی{v_0(t)} کو \عددی{t>0} کے لئے حاصل کریں۔ 

جواب:\عددی{v_0(t)=\tfrac{5}{2}+\tfrac{1}{2}e^{-\tfrac{5t}{2}}\,\si{\volt}}
\انتہا{سوال}
%=====================
\ابتدا{سوال}\شناخت{سوال_عارضی_ث}
شکل \حوالہ{شکل_سوال_عارضی_ث} میں \عددی{i_0(t)} کو \عددی{t>0} کے لئے حاصل کریں۔ 
\begin{figure}
\centering
\begin{tikzpicture}
\draw(0,0) to [american voltage source,l={$\SI{10}{\volt}$}]++(0,\y);
\draw(\x,0) to [capacitor,*-*,l={$\SI{100}{\micro\farad}$}]++(0,\y);
\draw(2*\x,0) to [cspst,*-*,l={${t=0}$}]++(0,\y);
\draw(3*\x,0) to [resistor,l={$\SI{2}{\kilo\ohm}$}]++(0,\y);
\draw(0,0) to [short]++(3*\x,0);
\draw(0,\y) to [resistor,l={$\SI{2}{\kilo\ohm}$}]++(\x,0) to [resistor,l={$\SI{6}{\kilo\ohm}$},i={$i(t)$}]++(\x,0) to [short]++(\x,0);
\end{tikzpicture}
\caption{سوال \حوالہ{سوال_عارضی_ث} کا دور۔}
\label{شکل_سوال_عارضی_ث}
\end{figure}

جواب:\عددی{i(t)=\tfrac{5}{4}+\tfrac{1}{12}e^{-\tfrac{20t}{3}}\,\si{\milli\ampere}}
\انتہا{سوال}
%=====================
\ابتدا{سوال}\شناخت{سوال_عارضی_ج}
شکل \حوالہ{شکل_سوال_عارضی_ج} میں \عددی{i_0(t)} کو \عددی{t>0} کے لئے حاصل کریں۔ 
\begin{figure}
\centering
\begin{tikzpicture}
\draw(0,0) to [resistor,l={$\SI{4}{\kilo\ohm}$},i<={$i(t)$}]++(0,\y);
\draw(\x,0) to [inductor,*-*,l={$\SI{4}{\henry}$}]++(0,\y);
\draw(2*\x,0) to [american voltage source,l={$\SI{8}{\volt}$}]++(0,\y);
\draw(0,0) to [short] ++(\x,0) to [ospst,l={${t=0}$}]++(\x,0);
\draw(0,\y) to [short]++(\x,0) to [resistor,l={$\SI{1}{\kilo\ohm}$},]++(\x,0);
\end{tikzpicture}
\caption{سوال \حوالہ{سوال_عارضی_ج} کا دور۔}
\label{شکل_سوال_عارضی_ج}
\end{figure}

جواب:\عددی{i(t)=-8e^{-1000t}\,\si{\milli\ampere}}
\انتہا{سوال}
%=====================
\ابتدا{سوال}\شناخت{سوال_عارضی_چ}
شکل \حوالہ{شکل_سوال_عارضی_چ} میں \عددی{i_0(t)} کو \عددی{t>0} کے لئے حاصل کریں۔ 
\begin{figure}
\centering
\begin{tikzpicture}
\draw(0,0) to [resistor,l={$\SI{4}{\kilo\ohm}$}]++(0,\y);
\draw(\x,0) to [american current source,*-*,l={$\SI{6}{\milli\henry}$}]++(0,\y);
\draw(2*\x,0) to [cspst,*-*,l={${t=0}$}]++(0,\y);
\draw(3*\x,0) to [capacitor,*-*,l={$\SI{200}{\micro\farad}$}] ++(0,\y);
\draw(4*\x,0) to [resistor,l={$\SI{2}{\kilo\ohm}$},i_<={$i(t)$}]++(0,\y);
\draw(0,0) to [short]++(4*\x,0);
\draw(0,\y) to [resistor,l={$\SI{2}{\kilo\ohm}$}]++(\x,0) to [resistor,l={$\SI{4}{\kilo\ohm}$}]++(\x,0) to [resistor,l={$\SI{6}{\kilo\ohm}$}]++(\x,0) to [short]++(\x,0);
\end{tikzpicture}
\caption{سوال \حوالہ{سوال_عارضی_چ} کا دور۔}
\label{شکل_سوال_عارضی_چ}
\end{figure}

جواب:\عددی{i(t)=2e^{-\tfrac{10t}{3}}\,\si{\milli\ampere}}
\انتہا{سوال}
%=====================
\ابتدا{سوال}\شناخت{سوال_عارضی_ح}
شکل \حوالہ{شکل_سوال_عارضی_ح} میں \عددی{v_0(t)} کو \عددی{t>0} کے لئے حاصل کریں۔ 
\begin{figure}
\centering
\begin{tikzpicture}[american voltages]
\draw(0,0) to [resistor,l={$\SI{4}{\kilo\ohm}$}]++(0,\y);
\draw(2*\x,0) to [american current source,*-*,l={$\SI{10}{\milli\ampere}$}]++(0,\y);
\draw(3*\x,0) to [capacitor,*-*,l={$\SI{40}{\micro\farad}$}]++(0,\y);
\draw(4*\x,0) to [resistor,l={$\SI{4}{\kilo\ohm}$},v_>={$v(t)$}]++(0,\y);
\draw(0,0) to [short]++(4*\x,0);
\draw(0,\y) to [resistor,l={$\SI{4}{\kilo\ohm}$}]++(\x,0) to [ospst,l={${t=0}$}]++(\x,0) to [resistor,l={$\SI{2}{\kilo\ohm}$}]++(\x,0) to [resistor,l={$\SI{2}{\kilo\ohm}$}]++(\x,0);
\end{tikzpicture}
\caption{سوال \حوالہ{سوال_عارضی_ح} کا دور۔}
\label{شکل_سوال_عارضی_ح}
\end{figure}

جواب:\عددی{v(t)=40-20e^{-\tfrac{25t}{6}}\,\si{\volt}}
\انتہا{سوال}
%=====================

\باب{سوالات مسئلے}
%==================
\ابتدا{سوال}\شناخت{سوال_مسئلے_خطیت_الف}
شکل \حوالہ{شکل_سوال_مسئلے_خطیت_الف} میں \عددی{V_0=\SI{2}{\volt}} فرض کرتے ہوئے مسئلہ خطیت کے استعمال سے اصل \عددی{V_0} دریافت کریں۔
\begin{figure}
\centering
\begin{tikzpicture}[american voltages]
\draw(0,0)to [resistor,l={$\SI{2}{\kilo\ohm}$}]++(0,2*\y);
\draw(\x,0)to [resistor,*-,l={$\SI{1}{\kilo\ohm}$}]++(0,\y) to [short,-*]++(0,\y);
\draw(2*\x,0) to [short,*-]++(0,\y) to [resistor,-*,l={$\SI{1}{\kilo\ohm}$}]++(0,\y);
\draw(3*\x,0)to [resistor,l={$\SI{2}{\kilo\ohm}$},v_>={$V_0$}]++(0,2*\y);
\draw(0,0) to [short]++(3*\x,0);
\draw(0,2*\y) to [short]++(\x,0) to [resistor,l_={$\SI{1}{\kilo\ohm}$}]++(\x,0) to [short]++(\x,0);
\draw(\x,\y) to [american current source,*-*,l={$\SI{4}{\milli\ampere}$}]++(\x,0);
\end{tikzpicture}
\caption{سوال \حوالہ{سوال_مسئلے_خطیت_الف} کا دور۔}
\label{شکل_سوال_مسئلے_خطیت_الف}
\end{figure}

جواب:\عددی{V_0=-\tfrac{16}{21}\,\si{\volt}}
\انتہا{سوال}
%======================
\ابتدا{سوال}\شناخت{سوال_مسئلے_خطیت_ب}
شکل \حوالہ{شکل_سوال_مسئلے_خطیت_ب} میں \عددی{I_0=\SI{1}{\milli\ampere}} فرض کرتے ہوئے مسئلہ خطیت کے استعمال سے اصل \عددی{I_0} دریافت کریں۔
\begin{figure}
\centering
\begin{tikzpicture}[american voltages]
\draw(0,0)to [resistor,l={$\SI{10}{\kilo\ohm}$}]++(0,\y);
\draw(\x,\y)to [american current source,*-*,l_={$\SI{6}{\milli\ampere}$}]++(0,-\y);
\draw(2*\x,0) to [resistor,*-*,l={$\SI{6}{\kilo\ohm}$}]++(0,\y);
\draw(3*\x,0)to [resistor,l_={$\SI{2}{\kilo\ohm}$},i_<={$I_0$}]++(0,\y);
\draw(0,0) to [short]++(3*\x,0);
\draw(0,\y) to [resistor,l={$\SI{2}{\kilo\ohm}$}]++(\x,0) to [resistor,l={$\SI{4}{\kilo\ohm}$}]++(\x,0) to [resistor,l={$\SI{4}{\kilo\ohm}$}]++(\x,0);
\end{tikzpicture}
\caption{سوال \حوالہ{سوال_مسئلے_خطیت_ب} کا دور۔}
\label{شکل_سوال_مسئلے_خطیت_ب}
\end{figure}

جواب:\عددی{I_0=\SI{-1.895}{\milli\ampere}}
\انتہا{سوال}
%======================
\ابتدا{سوال}\شناخت{سوال_مسئلے_خطیت_پ}
شکل \حوالہ{شکل_سوال_مسئلے_خطیت_پ} میں \عددی{I_0=\SI{1}{\milli\ampere}} فرض کرتے ہوئے مسئلہ خطیت کے استعمال سے اصل \عددی{I_0} دریافت کریں۔
\begin{figure}
\centering
\begin{tikzpicture}[american voltages]
\draw(0,0)to [american current source,l={$\SI{20}{\milli\ampere}$}]++(0,\yy);
\draw(\xx,\yy)to [resistor,*-*,l_={$\SI{2}{\kilo\ohm}$}]++(0,-\yy);
\draw(2*\xx,0) to [resistor,*-*,l={$\SI{2}{\kilo\ohm}$}]++(0,\yy);
\draw(3*\xx,0)to [resistor,l_={$\SI{2}{\kilo\ohm}$}]++(0,\yy);
\draw(0,0) to [short]++(3*\xx,0);
\draw(0,\yy) to [short]++(\xx,0) to [resistor,l={$\SI{2}{\kilo\ohm}$}]++(\xx,0) to [resistor,l={$\SI{2}{\kilo\ohm}$}]++(\xx,0);
\draw(2*\xx,\yy) to [resistor,l={$\SI{4}{\kilo\ohm}$},i_>={$I_0$}]++(\xx,-\yy);
\end{tikzpicture}
\caption{سوال \حوالہ{سوال_مسئلے_خطیت_پ} کا دور۔}
\label{شکل_سوال_مسئلے_خطیت_پ}
\end{figure}

جواب:\عددی{I_0=\SI{2}{\milli\ampere}}
\انتہا{سوال}
%======================
\ابتدا{سوال}\شناخت{سوال_مسئلے_نفاذ_الف}
شکل \حوالہ{شکل_سوال_مسئلے_نفاذ_الف} میں مسئلہ نفاذ کے استعمال سے \عددی{I_0} دریافت کریں۔
\begin{figure}
\centering
\begin{tikzpicture}[american voltages]
\draw(0,0)to [american current source,l={$\SI{8}{\volt}$}]++(0,\y);
\draw(\x,0)to [resistor,*-*,l={$\SI{4}{\kilo\ohm}$}]++(0,\y);
\draw(2*\x,0) to [american current source,*-*,l={$\SI{2}{\milli\ampere}$}]++(0,\y);
\draw(3*\x,0)to [resistor,l_={$\SI{4}{\kilo\ohm}$},i_<={$I_0$}]++(0,\y);
\draw(0,0) to [short]++(3*\x,0);
\draw(0,\y) to [resistor,l={$\SI{4}{\kilo\ohm}$}]++(\x,0) to [resistor,l={$\SI{4}{\kilo\ohm}$}]++(\x,0) to [short]++(\x,0);
\end{tikzpicture}
\caption{سوال \حوالہ{سوال_مسئلے_نفاذ_الف} کا دور۔}
\label{شکل_سوال_مسئلے_نفاذ_الف}
\end{figure}

جواب:\عددی{I_0=\tfrac{8}{5}\,\si{\milli\ampere}}
\انتہا{سوال}
%======================
\ابتدا{سوال}\شناخت{سوال_مسئلے_نفاذ_ب}
شکل \حوالہ{شکل_سوال_مسئلے_نفاذ_ب} میں مسئلہ نفاذ کے استعمال سے \عددی{I_0} دریافت کریں۔
\begin{figure}
\centering
\begin{tikzpicture}[american voltages]
\draw(0,0)to [american voltage source,l={$\SI{6}{\volt}$}]++(0,\y);
\draw(\x,0)to [resistor,*-*,l={$\SI{6}{\kilo\ohm}$},i<_={$I_0$}]++(0,\y);
\draw(2*\x,0) to [american current source,*-*,l={$\SI{4}{\milli\ampere}$}]++(0,\y);
\draw(3*\x,0)to [resistor,l_={$\SI{3}{\kilo\ohm}$}]++(0,\y);
\draw(0,0) to [short]++(3*\x,0);
\draw(0,\y) to [resistor,l={$\SI{3}{\kilo\ohm}$}]++(\x,0) to [resistor,l={$\SI{3}{\kilo\ohm}$}]++(\x,0) to [short]++(\x,0);
\end{tikzpicture}
\caption{سوال \حوالہ{سوال_مسئلے_نفاذ_ب} کا دور۔}
\label{شکل_سوال_مسئلے_نفاذ_ب}
\end{figure}

جواب:\عددی{I_0=\SI{1}{\milli\ampere}}
\انتہا{سوال}
%======================
\ابتدا{سوال}\شناخت{سوال_مسئلے_نفاذ_پ}
شکل \حوالہ{شکل_سوال_مسئلے_نفاذ_پ} میں مسئلہ نفاذ کے استعمال سے \عددی{V_0} دریافت کریں۔
\begin{figure}
\centering
\begin{tikzpicture}[american voltages]
\draw(0,0)to [american voltage source,l={$\SI{12}{\volt}$}]++(0,\y);
\draw(\x,0)to [resistor,*-*,l={$\SI{8}{\kilo\ohm}$}]++(0,\y);
\draw(2*\x,\y) to [american current source,*-*,l={$\SI{8}{\milli\ampere}$}]++(0,-\y);
\draw(3*\x,0)to [resistor,l_={$\SI{2}{\kilo\ohm}$}]++(0,\y);
\draw(0,0) to [short]++(3*\x,0);
\draw(0,\y) to [resistor,l={$\SI{4}{\kilo\ohm}$}]++(\x,0) to [resistor,l={$\SI{6}{\kilo\ohm}$},v={$V_0$}]++(\x,0) to [short]++(\x,0);
\end{tikzpicture}
\caption{سوال \حوالہ{سوال_مسئلے_نفاذ_پ} کا دور۔}
\label{شکل_سوال_مسئلے_نفاذ_پ}
\end{figure}

جواب:\عددی{V_0=\SI{13.5}{\volt}}
\انتہا{سوال}
%======================
\ابتدا{سوال}\شناخت{سوال_مسئلے_نفاذ_ت}
شکل \حوالہ{شکل_سوال_مسئلے_نفاذ_ت}-الف میں مسئلہ نفاذ کے استعمال سے \عددی{V_0} دریافت کریں۔
\begin{figure}
\centering
\begin{subfigure}{0.5\textwidth}
\centering
\begin{tikzpicture}[american voltages]
\draw(0,0)to [resistor,l={$\SI{4}{\kilo\ohm}$}]++(0,\y);
\draw(\x,0)to [resistor,*-*,l={$\SI{2}{\kilo\ohm}$}]++(0,\y);
\draw(2*\x,\y) to [resistor,l={$\SI{4}{\kilo\ohm}$}]++(0,-\y);
%\draw(3*\x,0)to [resistor,l_={$\SI{2}{\kilo\ohm}$}]++(0,\y);
\draw(0,0) to [short]++(2*\x,0);
\draw(0,\y) to [american voltage source,l={$\SI{8}{\volt}$}]++(\x,0);
\draw(2*\x,\y) to [american current source,l={$\SI{4}{\milli\ampere}$}]++(-\x,0);
\draw(0,\y) to [short,*-]++(0,3/4*\y) to [resistor,l={$\SI{8}{\kilo\ohm}$},v={$V_0$}]++(2*\x,0) to [short,-*]++(0,-3/4*\y);
\end{tikzpicture}
\caption*{(الف)}
\end{subfigure}%
\begin{subfigure}{0.5\textwidth}
\centering
\begin{tikzpicture}[american voltages]
\draw(0,\y)to [american voltage source,l_={$\SI{8}{\volt}$}]++(0,-\y);
\draw(\x,0)to [resistor,*-*,l_={$\SI{4}{\kilo\ohm}$}]++(0,\y) to [resistor,-*,l_={$\SI{4}{\kilo\ohm}$}]++(0,\y);
\draw(2*\x,0) to [resistor,l={$\SI{6}{\kilo\ohm}$},v_>={$V_0$}]++(0,2*\y);
%\draw(3*\x,0)to [resistor,l_={$\SI{2}{\kilo\ohm}$}]++(0,\y);
\draw(0,0) to [short]++(2*\x,0);
\draw(0,\y) to [american current source,l={$\SI{4}{\milli\ampere}$}]++(0,\y);
\draw(0,2*\y) to [short]++(2*\x,0);
\draw(0,\y) to [resistor,*-*,l={$\SI{8}{\kilo\ohm}$}]++(\x,0);
\end{tikzpicture}
\caption*{(ب)}
\end{subfigure}%
\caption{سوال \حوالہ{سوال_مسئلے_نفاذ_ت} اور سوال \حوالہ{سوال_مسئلے_نفاذ_ٹ} کے ادوار۔}
\label{شکل_سوال_مسئلے_نفاذ_ت}
\end{figure}

جواب:\عددی{V_0=\SI{9.6}{\volt}}
\انتہا{سوال}
%======================
\ابتدا{سوال}\شناخت{سوال_مسئلے_نفاذ_ٹ}
شکل \حوالہ{شکل_سوال_مسئلے_نفاذ_ت}-ب میں مسئلہ نفاذ کے استعمال سے \عددی{V_0} دریافت کریں۔

جواب:\عددی{V_0=\tfrac{56}{19}\,\si{\volt}}
\انتہا{سوال}
%======================
\ابتدا{سوال}\شناخت{سوال_مسئلے_تھونن_الف}
شکل \حوالہ{شکل_سوال_مسئلے_تھونن_الف}-الف میں مسئلہ تھونن کی مدد سے \عددی{V_0} دریافت کریں۔

\begin{figure}
\centering
\begin{subfigure}{0.5\textwidth}
\centering
\begin{tikzpicture}[american voltages]
\draw(0,0) to [resistor,l={$\SI{6}{\kilo\ohm}$}]++(0,\y);
\draw(\x,0) to [resistor,*-,l={$\SI{8}{\kilo\ohm}$}]++(0,\y);
\draw(2*\x,0) to [resistor,l={$\SI{4}{\kilo\ohm}$},v_>={$V_0$}]++(0,\y);
\draw(0,0) to [short]++(2*\x,0);
\draw(\x,\y) to [american voltage source,*-,l_={$\SI{12}{\volt}$}]++(-\x,0);
\draw(\x,\y) to [american voltage source,l={$\SI{8}{\volt}$}]++(\x,0);
\end{tikzpicture}
\caption*{(الف)}
\end{subfigure}%
\begin{subfigure}{0.5\textwidth}
\centering
\begin{tikzpicture}[american voltages]
\draw(0,\y) to [american current source,l_={$\SI{2}{\milli\ampere}$}]++(0,-\y);
\draw(\x,0) to [resistor,*-*,l={$\SI{6}{\kilo\ohm}$}]++(0,\y);
\draw(2*\x,0) to [american voltage source,l={$\SI{8}{\volt}$}]++(0,\y);
\draw(0,\y) to [short]++(2*\x,0);
\draw(0,0) to [resistor,l_={$\SI{4}{\kilo\ohm}$}]++(\x,0) to [resistor,l_={$\SI{4}{\kilo\ohm}$}]++(\x,0);
\draw(0,0) to [short,*-]++(0,-3/4*\y) to [resistor,l={$\SI{8}{\kilo\ohm}$},i={$I_0$}]++(2*\x,0) to [short,-*]++(0,3/4*\y);
\end{tikzpicture}
\caption*{(ب)}
\end{subfigure}%
\caption{سوال \حوالہ{سوال_مسئلے_تھونن_الف} اور سوال \حوالہ{سوال_مسئلے_تھونن_ب} کے ادوار۔}
\label{شکل_سوال_مسئلے_تھونن_الف}
\end{figure}

جواب:\عددی{V_0=\tfrac{8}{13}\,\si{\volt}}
\انتہا{سوال}
%====================
\ابتدا{سوال}\شناخت{سوال_مسئلے_تھونن_ب}
شکل \حوالہ{شکل_سوال_مسئلے_تھونن_الف}-ب میں مسئلہ تھونن کی مدد سے \عددی{I_0} دریافت کریں۔

جواب:\عددی{I_0=\tfrac{26}{27}\,\si{\milli\ampere}}
\انتہا{سوال}
%====================
\ابتدا{سوال}\شناخت{سوال_مسئلے_تھونن_پ}
شکل \حوالہ{شکل_سوال_مسئلے_تھونن_پ}-الف میں مسئلہ تھونن کی مدد سے \عددی{V_0} دریافت کریں۔

\begin{figure}
\centering
\begin{subfigure}{0.5\textwidth}
\centering
\begin{tikzpicture}[american voltages]
\draw(0,0) to [resistor,l={$\SI{4}{\kilo\ohm}$}]++(0,\y);
\draw(\x,0) to [american current source,*-*,l={$\SI{6}{\milli\ampere}$}]++(0,\y);
\draw(2*\x,0) to [resistor,l={$\SI{8}{\kilo\ohm}$},v_>={$V_0$}]++(0,\y);
\draw(0,0) to [short]++(2*\x,0);
\draw(0,\y) to [resistor,l={$\SI{2}{\kilo\ohm}$}]++(\x,0) to [resistor,l={$\SI{4}{\kilo\ohm}$}]++(\x,0);
\draw(0,\y) to [short,*-]++(0,3/4*\y) to [american voltage source,l={$\SI{4}{\volt}$}]++(2*\x,0) to [short,-*]++(0,-3/4*\y);
\end{tikzpicture}
\caption*{(الف)}
\end{subfigure}%
\begin{subfigure}{0.5\textwidth}
\centering
\begin{tikzpicture}[american voltages]
\draw(0,0) to [resistor,l={$\SI{2}{\kilo\ohm}$}]++(0,\y) to [american voltage source,l={$\SI{12}{\volt}$}]++(0,\y);
\draw(\x,0) to [american current source,*-*,l_={$\SI{4}{\milli\ampere}$}]++(0,\y) to [resistor,-*,l_={$\SI{4}{\kilo\ohm}$}]++(0,\y);
\draw(2*\x,0) to [resistor,l={$\SI{2}{\kilo\ohm}$}]++(0,2*\y);
\draw(0,0) to [short]++(2*\x,0);
\draw(0,2*\y) to [short]++(2*\x,0);
\draw(0,\y) to [resistor,*-*,l={$\SI{1}{\kilo\ohm}$},i={$I_0$}]++(\x,0);
\end{tikzpicture}
\caption*{(ب)}
\end{subfigure}%
\caption{سوال \حوالہ{سوال_مسئلے_تھونن_پ} اور سوال \حوالہ{سوال_مسئلے_تھونن_ت} کے ادوار۔}
\label{شکل_سوال_مسئلے_تھونن_پ}
\end{figure}

جواب:\عددی{V_0=\tfrac{56}{3}\,\si{\volt}}
\انتہا{سوال}
%====================
\ابتدا{سوال}\شناخت{سوال_مسئلے_تھونن_ت}
شکل \حوالہ{شکل_سوال_مسئلے_تھونن_پ}-ب میں مسئلہ تھونن کی مدد سے \عددی{I_0} دریافت کریں۔

جواب:\عددی{I_0=-\tfrac{28}{5}\,\si{\milli\ampere}}
\انتہا{سوال}
%====================
\ابتدا{سوال}\شناخت{سوال_مسئلے_تھونن_ٹ}
شکل \حوالہ{شکل_سوال_مسئلے_تھونن_ٹ}-الف میں \عددی{AB} سروں پر \عددی{\SI{2}{\kilo\ohm}} نسب کرنے سے مزاحمت میں \عددی{\tfrac{5}{2}\,\si{\milli\ampere}} پیدا ہوتی ہے  جبکہ ان سروں پر  \عددی{\SI{6}{\kilo\ohm}} نسب کرنے سے مزاحمت میں \عددی{\tfrac{5}{4}\,\si{\milli\ampere}} پیدا ہوتی ہے۔ دور کے متغیرات \عددی{V_{\text{تھونن}}} اور \عددی{R_{\text{تھونن}}} دریافت کریں۔ 

\begin{figure}
\centering
\begin{subfigure}{0.5\textwidth}
\centering
\begin{tikzpicture}[american voltages]
\draw(0,0)node[right]{$B$} to [short,o-]++(-\x,0) to [american voltage source,l={$V_{\text{تھونن}}$}]++(0,\y) to [resistor,-o,l={$R_{\text{تھونن}}$}]++(\x,0)node[right]{$A$};
\end{tikzpicture}
\caption*{(الف)}
\end{subfigure}%
\begin{subfigure}{0.5\textwidth}
\centering
\begin{tikzpicture}[american voltages]
\draw(0,0) rectangle ++(-0.5,\y);
\draw(0,\y-0.25) to [short,-o]++(\x,0)node[right]{$A$};
\draw(0,0.25) to [short,-o]++(\x,0)node[right]{$B$};
\draw(-0.25,\y/2)node[rotate=90]{\RL{خطی دور}};
\end{tikzpicture}
\caption*{(ب)}
\end{subfigure}%
\caption{سوال \حوالہ{سوال_مسئلے_تھونن_ٹ} اور سوال \حوالہ{سوال_مسئلے_تھونن_ث} کے ادوار۔}
\label{شکل_سوال_مسئلے_تھونن_ٹ}
\end{figure}

جواب:\عددی{\SI{10}{\volt}}، \عددی{\SI{2}{\kilo\ohm}}
\انتہا{سوال}
%====================
\ابتدا{سوال}\شناخت{سوال_مسئلے_تھونن_ث}
شکل \حوالہ{شکل_سوال_مسئلے_تھونن_ٹ}-ب میں \عددی{AB} سروں پر \عددی{\SI{6}{\kilo\ohm}} نسب کرنے سے  \عددی{V_{AB}=\SI{6}{\volt}} حاصل ہوتا ہے جبکہ  \عددی{\SI{3}{\kilo\ohm}} نسب کرنے سے  \عددی{V_{AB}=\SI{4}{\volt}} حاصل ہوتا ہے۔خطی دور کے تھونن متغیرات \عددی{V_{\text{تھونن}}} اور \عددی{R_{\text{تھونن}}} دریافت کریں۔ 

جواب:\عددی{\SI{12}{\volt}}، \عددی{\SI{6}{\kilo\ohm}}
\انتہا{سوال}
%====================
\ابتدا{سوال}\شناخت{سوال_مسئلے_نارٹن_الف}
شکل \حوالہ{شکل_سوال_مسئلے_نارٹن_الف} میں مسئلہ نارٹن استعمال کرتے ہوئے \عددی{I_0} دریافت کریں۔

\begin{figure}
\centering
\begin{tikzpicture}[american voltages]
\draw(0,0) to [american voltage source,l={$\SI{6}{\volt}$}]++(0,\y);
\draw(\x,0) to [resistor,*-*,l={$\SI{2}{\kilo\ohm}$},i<_={$I_0$}]++(0,\y);
\draw(2*\x,\y) to [american current source,*-*,l={$\SI{4}{\milli\ampere}$}]++(0,-\y);
\draw(3*\x,0) to [resistor,l_={$\SI{4}{\kilo\ohm}$}]++(0,\y);
\draw(0,0) to [short]++(3*\x,0);
\draw(0,\y) to [resistor,l={$\SI{2}{\kilo\ohm}$}]++(\x,0) to [resistor,l={$\SI{2}{\kilo\ohm}$}]++(\x,0) to [short]++(\x,0);
\end{tikzpicture}
\caption{سوال \حوالہ{سوال_مسئلے_نارٹن_الف} کا دور۔}
\label{شکل_سوال_مسئلے_نارٹن_الف}
\end{figure}

جواب:\عددی{I_0=\tfrac{1}{7}\,\si{\milli\ampere}}
\انتہا{سوال}
%====================
\ابتدا{سوال}\شناخت{سوال_مسئلے_نارٹن_ب}
شکل \حوالہ{شکل_سوال_مسئلے_نارٹن_ب}-الف میں مسئلہ نارٹن استعمال کرتے ہوئے \عددی{I_0} دریافت کریں۔

\begin{figure}
\centering
\begin{subfigure}{0.5\textwidth}
\centering
\begin{tikzpicture}[american voltages]
\draw(0,0) to [american voltage source,l={$\SI{18}{\volt}$}]++(0,\y);
\draw(\x,0) to [resistor,*-*,l={$\SI{6}{\kilo\ohm}$}]++(0,\y);
\draw(2*\x,\y) to [resistor,l={$\SI{6}{\kilo\ohm}$}]++(0,-\y);
\draw(0,\y) to [short,*-] ++(0,3/4*\y) to [resistor,l_={$\SI{4}{\kilo\ohm}$},i={$I_0$}]++(2*\x,0) to [short,-*]++(0,-3/4*\y);
\draw(0,0) to [short]++(2*\x,0);
\draw(0,\y) to [resistor,l={$\SI{2}{\kilo\ohm}$}]++(\x,0) to [resistor,l={$\SI{4}{\kilo\ohm}$}]++(\x,0);
\end{tikzpicture}
\caption*{(الف)}
\end{subfigure}%
\begin{subfigure}{0.5\textwidth}
\centering
\begin{tikzpicture}[american voltages]
\draw(0,0) to [american voltage source,l={$\SI{8}{\volt}$}]++(0,\y);
\draw(\x,0) to [resistor,*-*,l={$\SI{6}{\kilo\ohm}$}]++(0,\y);
\draw(2*\x,0) to [resistor,l={$\SI{8}{\kilo\ohm}$},v_>={$V_0$}]++(0,\y);
%\draw(0,\y) to [short,*-] ++(0,3/4*\y) to [resistor,l_={$\SI{4}{\kilo\ohm}$},i={$I_0$}]++(2*\x,0) to [short,-*]++(0,-3/4*\y);
\draw(0,0) to [short]++(2*\x,0);
\draw(0,\y) to [resistor,l={$\SI{2}{\kilo\ohm}$},v={$V_x$}]++(\x,0);
\draw(2*\x,\y) to [american controlled voltage source,l_={$4V_x$}]++(-\x,0);
\end{tikzpicture}
\caption*{(ب)}
\end{subfigure}%
\caption{سوال \حوالہ{سوال_مسئلے_نارٹن_ب} اور سوال \حوالہ{سوال_مسئلے_نارٹن_پ} کے ادوار۔}
\label{شکل_سوال_مسئلے_نارٹن_ب}
\end{figure}

جواب:\عددی{I_0=\tfrac{90}{47}\,\si{\milli\ampere}}
\انتہا{سوال}
%====================
\ابتدا{سوال}\شناخت{سوال_مسئلے_نارٹن_پ}
شکل \حوالہ{شکل_سوال_مسئلے_نارٹن_ب} -ب میں مسئلہ نارٹن استعمال کرتے ہوئے \عددی{V_0} دریافت کریں۔

جواب:\عددی{V_0=-\tfrac{32}{31}\,\si{\volt}}
\انتہا{سوال}
%========================
\ابتدا{سوال}\شناخت{سوال_مسئلے_نارٹن_ت}
شکل \حوالہ{شکل_سوال_مسئلے_نارٹن_ت} میں مسئلہ نارٹن استعمال کرتے ہوئے \عددی{I_0} دریافت کریں۔
\begin{figure}
\centering
\begin{tikzpicture}
\draw(0,0) to [american current source,l={$\SI{4}{\milli\ampere}$}]++(0,2*\y);
\draw(2*\x,\y) to [american voltage source,*-,l={$\SI{4}{\volt}$}]++(-\x,0) to [resistor,-*,l={$\SI{2}{\kilo\ohm}$}]++(0,\y);
\draw(2*\x,2*\y) to [resistor,*-,l={$\SI{4}{\kilo\ohm}$}]++(0,-\y) to [resistor,-*,l={$\SI{2}{\kilo\ohm}$},i={$I_0$}]++(0,-\y);
\draw(0,2*\y) to [short]++(3*\x,0);
\draw(0,0) to [short]++(3*\x,0) to [american voltage source,l_={$\SI{6}{\volt}$}]++(0,\y) to [resistor,l_={$\SI{2}{\kilo\ohm}$}]++(0,\y);
\end{tikzpicture}
\caption{سوال \حوالہ{سوال_مسئلے_نارٹن_ت} کا دور۔}
\label{شکل_سوال_مسئلے_نارٹن_ت}
\end{figure}

جواب:\عددی{I_0=\tfrac{17}{8}\,\si{\milli\ampere}}
\انتہا{سوال}
%========================
\ابتدا{سوال}\شناخت{سوال_مسئلے_نارٹن_ٹ}
شکل \حوالہ{شکل_سوال_مسئلے_نارٹن_ٹ} میں مسئلہ تھونن استعمال کرتے ہوئے \عددی{V_0} دریافت کریں۔
\begin{figure}
\centering
\begin{tikzpicture}[american voltages]
\draw(0,0) to [american controlled voltage source,l={$2V_x$}]++(0,\y);
\draw(\x,0) to [resistor,*-*,l={$\SI{1}{\kilo\ohm}$}]++(0,\y);
\draw(2*\x,0) to [american current source,*-*,l={$\SI{2}{\milli\ampere}$}]++(0,\y);
\draw(3*\x,0) to [resistor,*-*,l={$\SI{1}{\kilo\ohm}$},v_>={$V_x$}]++(0,\y);
\draw(4*\x,0) to [resistor,l={$\SI{1}{\kilo\ohm}$},v_>={$V_0$}]++(0,\y) to [resistor,l_={$\SI{1}{\kilo\ohm}$}]++(-\x,0);
\draw(0,0) to [short]++(4*\x,0);
\draw(0,\y) to [resistor,l={$\SI{1}{\kilo\ohm}$}]++(\x,0) to [american voltage source,l={$\SI{2}{\volt}$}]++(\x,0) to [resistor,l={$\SI{1}{\kilo\ohm}$}]++(\x,0);
\draw(0,\y) to [short,*-]++(0,3/4*\y) to [american current source,l={$\SI{4}{\milli\ampere}$}]++(3*\x,0) to [short]++(0,-3/4*\y);
\end{tikzpicture}
\caption{سوال \حوالہ{سوال_مسئلے_نارٹن_ٹ} کا دور۔}
\label{شکل_سوال_مسئلے_نارٹن_ٹ}
\end{figure}

جواب:\عددی{V_0=\tfrac{2}{3}\,\si{\volt}}
\انتہا{سوال}
%========================
\ابتدا{سوال}\شناخت{سوال_مسئلے_نارٹن_ث}
شکل \حوالہ{شکل_سوال_مسئلے_نارٹن_ث} میں مسئلہ نارٹن استعمال کرتے ہوئے \عددی{V_0} دریافت کریں۔
\begin{figure}
\centering
\begin{tikzpicture}[american voltages]
\draw(0,0) to [american current source,l={$\SI{4}{\milli\ampere}$}]++(0,\y);
\draw(\x,0) to [resistor,*-*,l={$\SI{1}{\kilo\ohm}$}]++(0,\y);
\draw(2*\x,\y) to [american current source,*-*,l_={$\SI{6}{\milli\ampere}$}]++(0,-\y);
\draw(3*\x,0) to [resistor,*-*,l={$\SI{1}{\kilo\ohm}$},v_>={$V_x$}]++(0,\y);
\draw(4*\x,0) to [resistor,l={$\SI{1}{\kilo\ohm}$},v_>={$V_0$}]++(0,\y) to [resistor,l_={$\SI{1}{\kilo\ohm}$}]++(-\x,0);
\draw(0,0) to [short]++(4*\x,0);
\draw(0,\y) to [resistor,l={$\SI{1}{\kilo\ohm}$}]++(\x,0) to [american voltage source,l={$\SI{2}{\volt}$}]++(\x,0) to [resistor,l={$\SI{1}{\kilo\ohm}$}]++(\x,0);
\draw(0,\y) to [short,*-]++(0,3/4*\y) to [american controlled voltage source,l={$4V_x$}]++(3*\x,0) to [short]++(0,-3/4*\y);
\end{tikzpicture}
\caption{سوال \حوالہ{سوال_مسئلے_نارٹن_ث} کا دور۔}
\label{شکل_سوال_مسئلے_نارٹن_ث}
\end{figure}

جواب:\عددی{V_0=\SI{-2}{\volt}}
\انتہا{سوال}
%========================
\ابتدا{سوال}\شناخت{سوال_مسئلے_تبادلہ_منبع_الف}
شکل \حوالہ{شکل_سوال_مسئلے_تبادلہ_منبع_الف} کو تبادلہ منبع سے حل کرتے ہوئے \عددی{I_0}  معلوم کریں۔
\begin{figure}
\centering
\begin{tikzpicture}
\draw(0,0) to [american voltage source,l={$\SI{12}{\volt}$}]++(0,\y);
\draw(\x,0) to [resistor,*-*,l={$\SI{12}{\kilo\ohm}$}]++(0,\y);
\draw(2*\x,0) to [resistor,*-*,l={$\SI{4}{\kilo\ohm}$}]++(0,\y);
\draw(3*\x,0) to [resistor,l_={$\SI{3}{\kilo\ohm}$}]++(0,\y) to [resistor,l_={$\SI{9}{\kilo\ohm}$}]++(-\x,0);
\draw(0,0) to [short]++(3*\x,0);
\draw(2*\x,\y) to [american current source,l_={$\SI{8}{\milli\ampere}$}]++(-\x,0) to [resistor,l_={$\SI{6}{\kilo\ohm}$},i<={$I_0$}]+(-\x,0);
\draw(\x,\y) to [short]++(0,3/4*\y) to [resistor,l={$\SI{3}{\kilo\ohm}$}]++(\x,0) to [short]++(0,-3/4*\y);
\end{tikzpicture}
\caption{سوال \حوالہ{سوال_مسئلے_تبادلہ_منبع_الف} کا دور۔}
\label{شکل_سوال_مسئلے_تبادلہ_منبع_الف}
\end{figure}

جواب:\عددی{I_0=-\tfrac{2}{5}\,\si{\milli\ampere}}
\انتہا{سوال}
%===========================
\ابتدا{سوال}\شناخت{سوال_مسئلے_تبادلہ_منبع_ب}
شکل \حوالہ{شکل_سوال_مسئلے_تبادلہ_منبع_ب} کو تبادلہ منبع سے حل کرتے ہوئے \عددی{V_0}  معلوم کریں۔
\begin{figure}
\centering
\begin{tikzpicture}[american voltages]
\draw(0,0) to [resistor,l={$\SI{2}{\kilo\ohm}$}]++(0,\y);
\draw(\x,0) to [american current source,*-*,l={$\SI{4}{\milli\ampere}$}]++(0,\y);
\draw(2*\x,0) to [resistor,*-*,l={$\SI{12}{\kilo\ohm}$}]++(0,\y);
\draw(3*\x,0) to [resistor,*-*,l_={$\SI{12}{\kilo\ohm}$}]++(0,\y);
 \draw(4*\x,0) to [american voltage source,l_={$\SI{12}{\volt}$}]++(0,\y) to [resistor,l_={$\SI{3}{\kilo\ohm}$}]++(-\x,0);
\draw(0,0) to [short]++(4*\x,0);
\draw(3*\x,\y) to [resistor,l={$\SI{3}{\kilo\ohm}$},v={$V_0$}]++(-\x,0) to [short]++(-\x,0) to [resistor,l_={$\SI{2}{\kilo\ohm}$}]++(-\x,0);
\end{tikzpicture}
\caption{سوال \حوالہ{سوال_مسئلے_تبادلہ_منبع_ب} کا دور۔}
\label{شکل_سوال_مسئلے_تبادلہ_منبع_ب}
\end{figure}

جواب:\عددی{V_0=-\tfrac{6}{7}\,\si{\volt}}
\انتہا{سوال}
%===========================
\ابتدا{سوال}\شناخت{سوال_مسئلے_تبادلہ_منبع_پ}
شکل \حوالہ{شکل_سوال_مسئلے_تبادلہ_منبع_پ} کو تبادلہ منبع سے حل کرتے ہوئے \عددی{I_0}  معلوم کریں۔
\begin{figure}
\centering
\begin{tikzpicture}[american voltages]
\draw(0,0) to [resistor,l={$\SI{1}{\kilo\ohm}$}]++(0,\y);
\draw(\x,0) to [american current source,*-*,l={$\SI{6}{\milli\ampere}$}]++(0,\y);
\draw(2*\x,0) to [resistor,*-*,l_={$\SI{4}{\kilo\ohm}$}]++(0,\y);
\draw(3*\x,0) to [american current source,*-*,l_={$\SI{4}{\milli\ampere}$}]++(0,\y);
 \draw(4*\x,0) to [american voltage source,l_={$\SI{8}{\volt}$},i_<={$I_0$}]++(0,\y) to [resistor,l_={$\SI{4}{\kilo\ohm}$}]++(-\x,0);
\draw(0,0) to [short]++(4*\x,0);
\draw(3*\x,\y) to [short]++(-\x,0) to  [resistor,l={$\SI{2}{\kilo\ohm}$}]++(-\x,0) to [short]++(-\x,0);
\draw(\x,\y)to [short]++(0,3/4*\y) to [american current source,l_={$\SI{2}{\milli\ampere}$}]++(\x,0) to [short]++(0,-3/4*\y);
\end{tikzpicture}
\caption{سوال \حوالہ{سوال_مسئلے_تبادلہ_منبع_پ} کا دور۔}
\label{شکل_سوال_مسئلے_تبادلہ_منبع_پ}
\end{figure}

جواب:\عددی{I_0=\SI{2}{\milli\ampere}}
\انتہا{سوال}
%===========================
\ابتدا{سوال}\شناخت{سوال_مسئلے_تبادلہ_منبع_ت}
شکل \حوالہ{شکل_سوال_مسئلے_تبادلہ_منبع_ت} کو تبادلہ منبع سے حل کرتے ہوئے \عددی{I_0}  معلوم کریں۔
\begin{figure}
\centering
\begin{tikzpicture}[american voltages]
\draw(2*\x,0) to [resistor,l={$\SI{6}{\kilo\ohm}$}]++(-\x,0) to [american voltage source,l={$\SI{8}{\volt}$}]++(-\x,0);
\draw(0,\y) to [resistor,l={$\SI{12}{\kilo\ohm}$}]++(\x,0) to [short]++(\x,0);
\draw(0,2*\y) to [short]++(\x,0) to [resistor,l={$\SI{6}{\kilo\ohm}$}]++(\x,0);
\draw(0,3*\y) to [short]++(\x,0) to [american voltage source,l={$\SI{6}{\volt}$}]++(\x,0);
\draw(0,0) to [short,-*]++(0,\y) to [short,-*]++(0,\y) to [american current source,l={$\SI{12}{\milli\ampere}$}]++(0,\y);
\draw(2*\x,0) to [short,-*]++(0,\y) to [resistor,-*,l={$\SI{2}{\kilo\ohm}$}]++(0,\y) to [short]++(0,\y);
\draw(\x,2*\y) to [resistor,*-*,l={$\SI{6}{\kilo\ohm}$}]++(0,\y);
\draw(\x,0) to [resistor,*-*,l_={$\SI{6}{\kilo\ohm}$},i_<={$I_0$}]++(0,\y);
\end{tikzpicture}
\caption{سوال \حوالہ{سوال_مسئلے_تبادلہ_منبع_ت} کا دور۔}
\label{شکل_سوال_مسئلے_تبادلہ_منبع_ت}
\end{figure}

جواب:\عددی{I_0=\tfrac{302}{111}\,\si{\milli\ampere}}
\انتہا{سوال}
%===========================
\ابتدا{سوال}\شناخت{سوال_مسئلے_بلند_تر_طاقت_الف}
شکل \حوالہ{شکل_سوال_مسئلے_بلند_تر_طاقت_الف} میں بوجھ \عددی{R_L} کی وہ قیمت دریافت کریں جس پر اس کو زیادہ سے زیادہ طاقت منتقل ہو گا۔اس طاقت کا تخمینہ لگائیں۔
\begin{figure}
\centering
\begin{tikzpicture}
\draw(0,0) to [resistor,l={$\SI{4}{\kilo\ohm}$}]++(0,\y) to [resistor,l={$\SI{4}{\kilo\ohm}$}]++(\x,0) to [resistor,l={$\SI{2}{\kilo\ohm}$}]++(\x,0) to [short]++(\x,0) to [resistor,l={$R_L$}]++(0,-\y) to [short](0,0);
\draw(\x,0) to [american current source,*-*,l={$\SI{2}{\milli\ampere}$}]++(0,\y);
\draw(2*\x,0) to [resistor,*-*,l_={$\SI{2}{\kilo\ohm}$}]++(0,\y);
\end{tikzpicture}
\caption{سوال \حوالہ{سوال_مسئلے_بلند_تر_طاقت_الف} کا دور۔}
\label{شکل_سوال_مسئلے_بلند_تر_طاقت_الف}
\end{figure}

جوابات:\عددی{R_L=\tfrac{8}{3}\,\si{\kilo\ohm}}، \عددی{p=\tfrac{512}{507}\,\si{\milli\watt}}
\انتہا{سوال}
%======================
\ابتدا{سوال}\شناخت{سوال_مسئلے_بلند_تر_طاقت_ب}
شکل \حوالہ{شکل_سوال_مسئلے_بلند_تر_طاقت_ب} میں بوجھ \عددی{R_L} کی وہ قیمت دریافت کریں جس پر اس کو زیادہ سے زیادہ طاقت منتقل ہو گا۔اس طاقت کا تخمینہ لگائیں۔
\begin{figure}
\centering
\begin{tikzpicture}
\draw(0,0) to [american voltage source,l={$\SI{3}{\volt}$}]++(0,\y);
\draw(\x,0) to [resistor,*-*,l={$\SI{6}{\kilo\ohm}$}]++(0,\y);
\draw(2*\x,0) to [resistor,l_={$R_L$}]++(0,\y);
\draw(0,0) to [short]++(2*\x,0);
\draw(0,\y) to [resistor,l={$\SI{3}{\kilo\ohm}$}]++(\x,0) to [resistor,l={$\SI{2}{\kilo\ohm}$}]++(\x,0);
\draw(\x,\y) to [short]++(0,3/4*\y) to [american current source,l={$\SI{1}{\milli\ampere}$}]++(\x,0) to [short,-*]++(0,-3/4*\y);
\end{tikzpicture}
\caption{سوال \حوالہ{سوال_مسئلے_بلند_تر_طاقت_ب} کا دور۔}
\label{شکل_سوال_مسئلے_بلند_تر_طاقت_ب}
\end{figure}

جوابات:\عددی{R_L=\SI{4}{\kilo\ohm}}، \عددی{p=\SI{1}{\milli\watt}}
\انتہا{سوال}
%======================

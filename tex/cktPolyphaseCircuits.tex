\باب{تین دوری نظام}
\حصہ{تین دوری ستارہ دباو}
اب تک بدلتی رو طاقت کی بات کرتے ہوئے  ایک عدد منبع دباو کی بات کی جاتی رہی۔حقیقت میں بدلتی رو طاقت کی پیدا وار اور ترسیل تین دوری نظام سے کی جاتی ہے۔شکل \حوالہ{شکل_تین_دوری_تین_دوری_نظام} میں تین دوری نظام دکھایا گیا ہے جہاں تین عدد منبع استعمال کئے گئے ہیں جو آپس میں \عددی{120^{\circ}} زاویائی فاصلہ رکھتے ہیں۔تمام دباو کے حیطے یک برابر ہونے کی صورت میں اس کو \اصطلاح{متوازن تین دوری نظام}\فرہنگ{متوازن!تین دوری نظام}\فرہنگ{تین دور!متوازن}\حاشیہب{balanced three phase system}\فرہنگ{three phase balanced system} کہا جاتا ہے۔دکھائے گئے متوازن نظام کے دباو درج ذیل ہیں جن کے دوری سمتیات کو شکل-ب میں دکھایا گیا ہے۔
\begin{gather}
\begin{aligned}
\hat{V}_{an}&=230 \phase{0^{\circ}}\,\si{\volt}\, \rms\\
\hat{V}_{bn}&=230 \phase{-120^{\circ}}\,\si{\volt}\, \rms\\
\hat{V}_{cn}&=230 \phase{-240^{\circ}}\,\si{\volt}\, \rms\\
&=230 \phase{120^{\circ}}\,\si{\volt}\, \rms
\end{aligned}
\end{gather}
انہیں کو وقتی دائرہ کار میں درج ذیل لکھا جائے گا۔شکل-پ میں انہیں دکھایا گیا ہے۔
\begin{gather}
\begin{aligned}\label{مساوات_تین_دوری_ستارہ_الف}
v_{an}(t)&=230\sqrt{2} \cos\omega t \,\si{\volt}\\
v_{bn}(t)&=230\sqrt{2} \cos(\omega t-120^{\circ})\,\si{\volt}\\
v_{cn}(t)&=230\sqrt{2} \cos(\omega t +120^{\circ})\,\si{\volt}
\end{aligned}
\end{gather}
متوازن بوجھ کی صورت میں تینوں رو کے حیطے اور زاویے بھی برابر ہوں گے لہٰذا انہیں درج ذیل لکھا جائے گا۔
\begin{gather}
\begin{aligned}
i_{an}(t)&=I_0 \cos(\omega t -\theta)\,\si{\ampere}\\
i_{bn}(t)&=I_0 \cos(\omega t-120^{\circ}-\theta)\,\si{\ampere}\\
i_{cn}(t)&=I_0 \cos(\omega t +120^{\circ}-\theta)\,\si{\ampere}
\end{aligned}
\end{gather}
%
\begin{figure}
\centering
\begin{subfigure}{1\textwidth}
\centering
\begin{tikzpicture}
\draw(0,0)to [short]++(0,2*\y)  to [american voltage source,l_={${\hat{V}_{an}=230\phase{0^{\circ}}\,\si{\volt}\,\rms}$}]++(0,\y)  to [short,-o]++(4*\x,0)node[right]{$a$};
\draw(1*\x,0) to [short]++(0,\y) to [american voltage source,l_={${\hat{V}_{bn}=230\phase{-120^{\circ}}\,\si{\volt}\,\rms}$}]++(0,\y) to [short,-o]++(3*\x,0)node[right]{$b$};
\draw(2*\x,0) to [american voltage source,l_={${\hat{V}_{cn}=230\phase{120^{\circ}}\,\si{\volt}\,\rms}$}] ++(0,\y)to [short,-o]++(2*\x,0)node[right]{$c$};
\draw(0,0) to [short,-*]++(1*\x,0) to [short,-*]++(1*\x,0) to [short,-o]++(2*\x,0)node[right]{$n$};
\end{tikzpicture}
\caption*{(الف)}
\end{subfigure}
\begin{subfigure}{0.4\textwidth}
\centering
\begin{tikzpicture}
\pgfmathsetmacro{\len}{\x}
\draw[-latex](0,0)--++(0:\len)coordinate(kva)node[right]{$\hat{V}_{an}$};
\draw[-latex](0,0)--++(-120:\len)coordinate(kvb)node[left]{$\hat{V}_{bn}$};
\draw[-latex](0,0)--++(120:\len)coordinate(kvc)node[left]{$\hat{V}_{cn}$};
\draw[stealth-stealth]([shift={(0:0.3)}]0,0) arc (0:120:0.3);
\draw[stealth-stealth]([shift={(-120:0.3)}]0,0) arc (-120:0:0.3);
\draw[stealth-stealth]([shift={(120:0.3)}]0,0) arc (120:240:0.3);
\draw(60:0.8)node{$120^{\circ}$};
\draw(-60:0.8)node{$120^{\circ}$};
\draw(180:0.8)node{$120^{\circ}$};
\end{tikzpicture}
\caption*{(ب)}
\end{subfigure}%
\begin{subfigure}{0.6\textwidth}
\centering
\begin{tikzpicture}
\begin{axis}[kStyleCircuitsA,small,xlabel=$\omega t$, xtick={90,180,270,360},xticklabels={$90^{\circ}$,$180^{\circ}$,$270^{\circ}$,$360^{\circ}$},ytick={10},yticklabels={$230\sqrt{2}\,\si{\volt}$},]
\addplot[domain=0:370,samples=100]{10*cos(1*x+0)}node[pos=0,above right]{$v_{an}$};
\addplot[domain=0:370,samples=100]{10*cos(1*x-120)}node[pos=0.35,above right]{$v_{bn}$};
\addplot[domain=0:370,samples=100]{10*cos(1*x+120)}node[pos=0.65,above right]{$v_{cn}$};
\end{axis}%
\end{tikzpicture}
\caption*{(پ)}
\end{subfigure}%
\caption{تین دوری نظام۔}
\label{شکل_تین_دوری_تین_دوری_نظام}
\end{figure} 

مساوات \حوالہ{مساوات_تین_دوری_ستارہ_الف} کے تینوں دباو کو عمومی شکل میں لکھتے ہیں۔
\begin{gather}
\begin{aligned}
v_{an}(t)&=V_0 \cos\omega t \,\si{\volt}\\
v_{bn}(t)&=V_0 \cos(\omega t-120^{\circ})\,\si{\volt}\\
v_{cn}(t)&=V_0 \cos(\omega t +120^{\circ})\,\si{\volt}
\end{aligned}
\end{gather}
شکل \حوالہ{شکل_تین_دوری_تین_دوری_نظام} میں \عددی{n} تا \عددی{a} کے دباو \عددی{\hat{V}_{an}} کو شاخ کا دباو یا \اصطلاح{دوری دباو}\فرہنگ{دوری دباو}\فرہنگ{دباو!دوری}\حاشیہب{phase voltage}\فرہنگ{voltage!phase}\فرہنگ{phase!voltage} کہا جاتا ہے۔اسی طرح  \عددی{n} تا \عددی{b} کے دباو \عددی{\hat{V}_{bn}} اور  \عددی{n} تا \عددی{c} کے دباو \عددی{\hat{V}_{cn}} بھی دوری دباو ہیں۔آئیں اس شکل سے \عددی{b} تا \عددی{a} دباو دریافت کریں جسے \اصطلاح{دباو تار}\فرہنگ{دباو!تار}\فرہنگ{تار!دباو}\حاشیہب{line to line voltage}\فرہنگ{line to line!voltage}\فرہنگ{voltage!line to line} کہا جاتا ہے۔
\begin{align*}
\hat{V}_{ab}&=\hat{V}_{an}-\hat{V}_{bn}\\
&=V_0\phase{0^{\circ}}-V_0\phase{-120^{\circ}}\\
&=V_0-V_0\left(-\frac{1}{2}-j\frac{\sqrt{3}}{2}\right)\\
&=V_0\left(\frac{3}{2}-j\frac{\sqrt{3}}{2}\right)\\
&=\sqrt{3}V_0\phase{30^{\circ}}
\end{align*}
یہی جواب شکل \حوالہ{شکل_تین_دوری_تار_تعلق}-الف میں ترسیمی طریقے سے حاصل کیا جا سکتا ہے جہاں تکون سے درج ذیل لکھتے
\begin{align*}
V^2_{ab}=V^2_0+V^2_0-2V^2_0\cos 120^{\circ}
\end{align*}
ہوئے
\begin{align}
V_{ab}=\sqrt{3}V_0
\end{align}
ملتا ہے اور زاویہ شکل سے \عددی{30^{\circ}} پڑھا جا سکتا ہے لہٰذا \عددی{\hat{V}_{ab}=\sqrt{3}V_0\phase{30^{\circ}}} ہو گا۔

چونکہ \عددی{V_0} دور کا دباو ہے جبکہ \عددی{\sqrt{3}V_0} تار کا دباو ہے لہٰذا درج بالا مساوات کو درج ذیل لکھا جا سکتا ہے۔
\begin{align}\label{مساوات_تین_ستارہ_دور_تار_دباو_تعلق}
V_{\text{تار}}=\sqrt{3}V_{\text{دور}}
\end{align} 
%
\begin{figure}
\centering
\begin{subfigure}{0.4\textwidth}
\centering
\begin{tikzpicture}
\pgfmathsetmacro{\len}{\x}
\draw[-latex,gray](0,0)--++(0:\len)coordinate(kva)node[shift={(0,0.5)}]{$\hat{V}_{an}$}node[above,pos=0.4]{$V_0$};
\draw[-latex,gray](0,0)--++(-120:\len)coordinate(kvb)node[left]{$\hat{V}_{bn}$}node[left,pos=0.4]{$V_0$};
\draw[-latex,gray](0,0)--++(120:\len)coordinate(kvc)node[left]{$\hat{V}_{cn}$};
\draw[gray]([shift={(-120:0.3)}]0,0) arc (-120:0:0.3);
%
\draw[gray](-60:0.6)node{$120^{\circ}$};
\draw([shift={(180:0.5)}]kva) arc (180:210:0.5);
\draw[gray](kva)++(198:0.8)node{$30^{\circ}$};
\draw([shift={(30:0.5)}]kvb) arc (30:60:0.5);
\draw[gray](kvb)++(40:0.95)node{$30^{\circ}$};
\draw[gray,dashed](kva)--++(1,0);
\draw[gray,dashed](kva)--++(30:1);
\draw[gray]([shift={(0:0.3)}]kva) arc (0:30:0.3);
\draw[gray](kva)++(15:0.9)node{$30^{\circ}$};
%
\draw[-latex](kvb)--(kva)node[pos=0.5,below right]{$\hat{V}_{ab}$};
\draw[-latex](kva)--(kvc)node[pos=0.5,above]{$\hat{V}_{ca}$};
\draw[-latex](kvc)--(kvb)node[pos=0.5,left]{$\hat{V}_{bc}$};
\end{tikzpicture}
\caption*{(الف) تکونی یا \عددی{\Delta} دباو۔}
\end{subfigure}%
\begin{subfigure}{0.6\textwidth}
\centering
\begin{tikzpicture}
\pgfmathsetmacro{\len}{\x*sqrt(3)}
%
\draw[-latex](0,0)--++(30:\len)coordinate(kva)node[right]{$\hat{V}_{ab}$}node[above left,pos=0.5]{$\sqrt{3}V_0$};
\draw[-latex](0,0)--++(150:\len)coordinate(kvb)node[left]{$\hat{V}_{ca}$}node[above right,pos=0.5]{$\sqrt{3}V_0$};
\draw[-latex](0,0)--++(-90:\len)coordinate(kvc)node[left]{$\hat{V}_{bc}$}node[left,pos=0.5]{$\sqrt{3}V_0$};
\draw[gray](0,0)--++(2,0);
\draw[-stealth]([shift={(0:0.8)}]0,0) arc (0:30:0.8);
\draw(15:0.8)node[right]{$30^{\circ}$};
\draw([shift={(150:0.4)}]0,0)arc (150:270:0.4);
\draw(210:0.8)node{$120^{\circ}$};
\end{tikzpicture}
\caption*{(ب) ستارہ یا \عددی{Y} دباو۔}
\end{subfigure}%
\caption{دوری دباو اور دباو تار کا تعلق۔}
\label{شکل_تین_دوری_تار_تعلق}
\end{figure}%

یوں ہم تین دوری دباو تار لکھ سکتے ہیں جنہیں شکل \حوالہ{شکل_تین_دوری_تار_تعلق}-ب میں دکھایا گیا ہے۔
\begin{gather}
\begin{aligned}
\hat{V}_{ab}&=\sqrt{3}V_0\phase{30^{\circ}}\\
\hat{V}_{ca}&=\sqrt{3}V_0\phase{150^{\circ}}\\
\hat{V}_{bc}&=\sqrt{3}V_0\phase{-90^{\circ}}
\end{aligned}
\end{gather}
تین دوری دباو تار بھی آپس میں \عددی{120^{\circ}} زاویے پر پائے جاتے ہیں۔

شکل \حوالہ{شکل_تین_دوری_تین_دوری_نظام}-ب میں  \عددی{v_{bn}} کو \عددی{v_{an}} سے \عددی{120^{\circ}} پیچھے اور \عددی{v_{cn}} کو \عددی{v_{bn}} سے \عددی{120^{\circ}}  پیچھے دکھایا گیا ہے لہٰذا اس نظام کی ترتیب \عددی{abc}\فرہنگ{abc} ہے۔تین دوری نظام میں عموماً \عددی{a}، \عددی{b}، \عددی{c} اور \عددی{n}  لئے مختلف رنگ کے تار استعمال کئے جاتے ہیں۔ہمارے ہاں لال، پیلا، نیلا اور کالا رنگ استعمال ہوتا ہے۔

%=================
\ابتدا{مثال}\شناخت{مثال_تین_دوری_تکونی_صفر_برابر_ہے}
درج ذیل مساوات کو ثابت کریں۔
\begin{align}
\cos \alpha+\cos(\alpha+120^{\circ})+\cos(\alpha-120^{\circ})&=0\label{مساوات_تین_دوری_تکونی_صفر_برابر_ہے}\\
\cos \alpha+\cos(\alpha-240^{\circ})+\cos(\alpha+240^{\circ})&=0\label{مساوات_تین_دوری_تکونی_صفر_برابر_ب}
\end{align}

حل:مساوات \حوالہ{مساوات_تین_دوری_تکونی_صفر_برابر_ہے} میں دوسرے اور تیسرے اجزاء کو درج ذیل لکھا جا سکتا ہے۔
\begin{align*}
\cos(\alpha+120^{\circ})&=\cos \alpha \cos 120^{\circ}-\sin \alpha \sin 120^{\circ}=-\frac{1}{2}\cos \alpha-\frac{\sqrt{3}}{2}\sin\alpha\\
\cos(\alpha-120^{\circ})&=\cos \alpha \cos 120^{\circ}+\sin \alpha \sin 120^{\circ}=-\frac{1}{2}\cos \alpha+\frac{\sqrt{3}}{2}\sin\alpha
\end{align*}
یوں تینوں اجزاء کا مجموعہ درج ذیل ہے۔
\begin{align*}
(\cos \alpha)+(-\frac{1}{2}\cos \alpha-\frac{\sqrt{3}}{2}\sin\alpha)+(-\frac{1}{2}\cos \alpha+\frac{\sqrt{3}}{2}\sin\alpha)=0
\end{align*}
آئیں اب مساوات \حوالہ{مساوات_تین_دوری_تکونی_صفر_برابر_ب} کو ثابت کریں۔مساوات کے دوسرے جزو میں \عددی{\cos(\alpha-240^{\circ})=\cos(\alpha+120^{\circ})} استعمال کرتے ہوئے اور تیسرے جزو میں \عددی{\cos(\alpha+240^{\circ})=\cos(\alpha-120^{\circ})} استعمال کرتے ہوئے مساوات \حوالہ{مساوات_تین_دوری_تکونی_صفر_برابر_ہے} ملتا ہے جسے ہم ثابت کر چکے ہیں۔
\انتہا{مثال}
%=================
\ابتدا{مشق}
متوازن \عددی{abc} ترتیب کے تین دوری ستارہ دباو  میں \عددی{\hat{V}_{an}=230\phase{30^{\circ}}\,\si{\volt}\,\rms} ہے۔باقی دو موثر ستارہ دباو حاصل کرتے ہوئے موثر دباو تار بھی حاصل کریں۔

جوابات: \عددی{\hat{V}_{bn}=230\phase{150^{\circ}}\,\si{\volt}\,\rms}، \عددی{\hat{V}_{cn}=-90\phase{30^{\circ}}\,\si{\volt}\,\rms}، \عددی{\hat{V}_{ab}=398.4\phase{60^{\circ}}\,\si{\volt}\,\rms}، \\
\عددی{\hat{V}_{ca}=398.4\phase{180^{\circ}}\,\si{\volt}\,\rms}، \عددی{\hat{V}_{bc}=398.4\phase{-60^{\circ}}\,\si{\volt}\,\rms}
\انتہا{مشق}
%==================
\ابتدا{مشق}
متوازن تین دوری \عددی{abc} ستارہ نظام میں \عددی{\hat{V}_{ab}=415\phase{0^{\circ}}\,\si{\volt}\,\rms} ہے۔ دباو تار کا تکون شکل \حوالہ{شکل_تین_دوری_تار_تعلق}-الف کے طرز پر کھینچیں۔ترسیمی طریقے سے  موثر ستارہ دوری دباو حاصل کریں۔

جوابات:\عددی{\hat{V}_{an}=239.4\phase{-30^{\circ}}\,\si{\volt}\,\rms}، \عددی{\hat{V}_{bn}=239.4\phase{-150^{\circ}}\,\si{\volt}\,\rms}، \\ \عددی{\hat{V}_{cn}=239.4\phase{90^{\circ}}\,\si{\volt}\,\rms}
\انتہا{مشق}
%=================
تین دوری نظام میں علیحدہ علیحدہ دور کے لمحاتی طاقت لکھتے ہیں 
\begin{align*}
p_a(t)&=v_{an}i_{an}\\
&=V_0 I_0 \cos \omega t \cos(\omega t -\theta)\\
&=\frac{V_0 I_0}{2}[\cos \theta +\cos(2\omega t -\theta)]\\
p_b(t)&=v_{bn}i_{bn}\\
&=V_0 I_0 \cos(\omega t -120^{\circ})\cos(\omega t-120^{\circ} -\theta)\\
&=\frac{V_0 I_0}{2}[\cos \theta +\cos(2\omega t -\theta-240^{\circ})]\\
p_c(t)&=v_{cn}i_{cn}\\
&=V_0 I_0 \cos (\omega t +120^{\circ})\cos(\omega t+120^{\circ} -\theta)\\
&=\frac{V_0 I_0}{2}[\cos \theta +\cos(2\omega t -\theta+240^{\circ})]
\end{align*}
جہاں \عددی{\cos \alpha \cos \beta=\tfrac{1}{2} [\cos(\alpha-\beta)+\cos(\alpha+\beta)]} کا استعمال کیا گیا ہے۔یوں مکمل نظام کا لمحاتی طاقت \عددی{p(t)} درج بالا کا مجموعہ ہو گا۔
\begin{align*}
p(t)&=p_a(t)+p_b(t)+p_c(t)\\
&=\frac{V_0 I_0}{2}[3\cos \theta +\cos(2\omega t-\theta)+\cos(2\omega t -\theta-240^{\circ})+\cos(2\omega t -\theta+240^{\circ})]
\end{align*}
درج بالا مساوات میں \عددی{2\omega t -\theta=\alpha} لکھتے ہوئے اور مساوات \حوالہ{مساوات_تین_دوری_تکونی_صفر_برابر_ب} استعمال کرتے ہوئے آخری تین اجزاء کے مجموعے کو صفر کے برابر لکھا جا سکتا ہے۔یوں لمحاتی طاقت درج ذیل حاصل ہوتی ہے۔
\begin{align}\label{مساوات_تین_دوری_لمحاتی_طاقت_برقرار}
p(t)=\frac{3V_0 I_0}{2}\cos \theta =3 \Vrms \Irms \cos \theta \,\si{\watt}
\end{align}
آپ مساوات \حوالہ{مساوات_تین_دوری_لمحاتی_طاقت_برقرار} کا \عددی{p_a(t)=\frac{V_0 I_0}{2}[\cos \theta +\cos(2\omega t -\theta)]} کے ساتھ موازنہ کریں جو دگنی تعدد یعنی \عددی{2\omega} کے ساتھ تبدیل ہوتا ہے۔آپ دیکھ سکتے ہیں کہ تین دوری نظام میں لمحاتی طاقت برقرار رہتا ہے۔یہ انتہائی اہم نتیجہ ہے۔تین دور کا موٹر برقرار میکانی قوت پیدا کرے گا لہٰذا اس میں ترتراہٹ کم سے کم ہو گی جو میکانی خرابی کی وجہ بنتی ہے۔

\حصہ{ستارہ ستارہ \,(\عددیء{YY}) \, جوڑ}
مساوات \حوالہ{مساوات_تین_دوری_ستارہ_الف} میں لمحہ \عددی{t=0} پر \عددی{v_{an}} کی چوٹی پائی جاتی ہے۔ہم کہتے ہیں کہ \عددی{v_{an}} کا زاویائی ہٹاو صفر کے برابر ہے۔اگر \عددی{v_{an}} کا زاویائی ہٹاو \عددی{\theta} ہو تب تین دوری نظام کے دوری سمتیات درج ذیل ہوں گے۔
 \begin{gather}
\begin{aligned}
\hat{V}_{an}&=230 \phase{\theta}\,\si{\volt} \,\rms\\
\hat{V}_{bn}&=230 \phase{\theta-120^{\circ}}\,\si{\volt}\, \rms\\
\hat{V}_{cn}&=230 \phase{\theta-240^{\circ}}\,\si{\volt} \,\rms
\end{aligned}
\end{gather}
ایسی صورت میں شکل \حوالہ{شکل_تین_دوری_تین_دوری_نظام}-ب کے تینوں دوری سمتیات \عددی{\theta} زاویہ گھوم جائیں گے۔تین دوری \عددی{abc} نظام کی بات کرتے ہوئے ہم \عددی{v_{an}} کا زاویہ ہٹاو صفر کے برابر لیں گے تا کہ بار بار زاویہ ہٹاو کا تعین نہ کرنا پڑے۔
\begin{figure}
\centering
\begin{subfigure}{0.5\textwidth}
\centering
\begin{tikzpicture}
\draw(0,0)node[left]{$n$} to [american voltage source,*-o,l={$\hat{V}_{an}$}]++(0:\x)node[right]{$a$};
\draw(0,0) to [american voltage source,-o,l={$\hat{V}_{bn}$}]++(-120:\x)node[left]{$b$};
\draw(0,0) to [american voltage source,-o,l={$\hat{V}_{cn}$}]++(120:\x)node[left]{$c$};
\end{tikzpicture}
\caption*{(الف)}
\end{subfigure}%
\begin{subfigure}{0.5\textwidth}
\centering
\begin{tikzpicture}
\draw[-latex](0,0)--++(0:\x)node[right]{$\hat{V}_{an}$};
\draw[-latex](0,0)--++(-120:\x)node[left]{$\hat{V}_{bn}$};
\draw[-latex](0,0)--++(120:\x)node[left]{$\hat{V}_{cn}$};
\draw[stealth-stealth]([shift={(0:0.3)}]0,0) arc (0:120:0.3);
\draw[stealth-stealth]([shift={(-120:0.3)}]0,0) arc (-120:0:0.3);
\draw[stealth-stealth]([shift={(120:0.3)}]0,0) arc (120:240:0.3);
\draw(60:0.8)node{$120^{\circ}$};
\draw(-60:0.8)node{$120^{\circ}$};
\draw(180:0.8)node{$120^{\circ}$};
\end{tikzpicture}
\caption*{(ب)}
\end{subfigure}%
\caption{ستارہ (\عددی{Y}) جوڑ۔}
\label{شکل_تین_دوری_ستارہ_نظام}
\end{figure}


شکل \حوالہ{شکل_تین_دوری_تین_دوری_نظام}-الف کے تین دوری \عددی{abc} نظام کو شکل \حوالہ{شکل_تین_دوری_ستارہ_نظام}-الف میں \اصطلاح{ستارہ جڑا}\فرہنگ{ستارہ جوڑ}\حاشیہب{star connected, Y connected}\فرہنگ{star connected}\فرہنگ{Y connected} دکھایا گیا ہے۔ساتھ ہی شکل-ب میں دوری سمتیات دکھائے گئے ہیں جو ستارہ شکل بناتے ہیں۔تین دوری نظام کو اس طرح کاغذ پر بناتے ہوئے مکمل معلومات بغیر لکھے دی جاتی ہے۔یوں شکل \حوالہ{شکل_تین_دوری_تین_دوری_نظام}-الف سے ظاہر ہے کہ \عددی{v_{an}} کا زاویہ ہٹاو صفر کے برابر ہے اور \عددی{v_{bn}} اس سے \عددی{120^{\circ}} پیچھے ہے۔یوں ظاہر ہے کہ اس نظام کی ترتیب \عددی{abc} ہے۔ساتھ ہی آپ دیکھ سکتے ہیں کہ تینوں دباو کے حیطے برابر ہیں۔تینوں دباو کو نقطہ \عددی{n} سے ناپا جاتا ہے۔ستارہ جوڑ کو \عددی{Y}\حاشیہد{ستارہ جوڑ کی شکل حرف \عددی{Y} سے مشابہت رکھتا ہے۔اسی لئے اس کو \عددی{Y} جوڑ بھی کہتے ہیں۔} جوڑ بھی کہتے ہیں۔

دوری سمتیات کا مجموعہ حاصل کرتے وقت ایک دوری سمتیہ کی نوک کے ساتھ دوسری دوری سمتیہ کی دم ملائی جاتی ہے۔اس ترکیب کو استعمال کرتے ہوئے شکل \حوالہ{شکل_تین_دوری_دباو_مجموعہ_صفر} میں ترسیمی طریقے سے درج ذیل مساوات ثابت کی گئی ہے۔
\begin{align}
\hat{V}_{an}+\hat{V}_{bn}+\hat{V}_{cn}=0
\end{align} 

\begin{figure}
\centering
\begin{tikzpicture}
\draw[-latex](0,0)--++(0:\x)coordinate(ka)node[above,pos=0.7]{$\hat{V}_{an}$};
\draw[-latex](ka)--++(-120:\x)coordinate(kb)node[right,pos=0.7]{$\hat{V}_{bn}$};
\draw[-latex](kb)--++(120:\x)node[left,pos=0.7]{$\hat{V}_{cn}$};
\draw(-\x/2,-\y/2)node[left]{$\hat{V}_{an}+\hat{V}_{bn}+\hat{V}_{cn}=0$};
\end{tikzpicture}
\caption{تین دوری نظام کے تینوں دباو کا مجموعہ صفر کے برابر ہے۔}
\label{شکل_تین_دوری_دباو_مجموعہ_صفر}
\end{figure}

شکل \حوالہ{شکل_تین_دوری_ستارہ_ستارہ_الف}-الف میں تین دوری نظام  کے تینوں منبع پر بوجھ لدا دکھایا گیا ہے۔اسی کو شکل-ب میں ستارہ صورت میں دکھایا گیا ہے۔منبع اور بوجھ دونوں ستارہ جڑے ہیں اور انہیں جوڑنے میں چار عدد تار استعمال کئے گئے ہیں لہٰذا اس نظام کو \اصطلاح{چار تار، ستارہ ستارہ}\فرہنگ{چار تار!ستارہ ستارہ}\فرہنگ{ستارہ ستارہ!چار تار}\حاشیہب{four-wire, star-star}\فرہنگ{star-star!four-wire}\فرہنگ{YY} نظام یا \عددیء{YY} نظام کہا جاتا ہے۔شاخ \عددی{a} پر نظر ڈالتے ہوئے معلوم ہوتا ہے کہ منبع \عددی{\hat{V}_{an}} کی دوری رو \عددی{\hat{I}_{a}} ہی منبع سے بوجھ تک تار میں پائے جانے والی رو تار \عددی{\hat{I}_a} ہے۔یوں ستارہ ستارہ نظام کے لئے درج ذیل لکھا جا سکتا ہے جہاں مساوات \حوالہ{مساوات_تین_ستارہ_دور_تار_دباو_تعلق} کو دوبارہ پیش کیا گیا ہے۔
\begin{gather}
\begin{aligned}
I_{\text{تار}}&=I_{\text{دوری}}\\
V_{\text{تار}}&=\sqrt{3}V_{\text{دوری}} \quad \quad \text{\RL{ستارہ ستارہ نظام میں دوری اور تار کے متغیرات کے تعلق}}
\end{aligned}
\end{gather}
%
 \begin{figure}
\centering
\begin{subfigure}{1\textwidth}
\centering
\begin{tikzpicture}
\draw(0,0) to [american voltage source,l_={${\hat{V}_{an}}$},i={$\hat{I}_a$}]++(0,\y) to [short]++(0,\y) to [short]++(2*\x,0)  to [short,-o,i={$\hat{I}_a$}]++(1*\x,0)node[above]{$a$}coordinate(ka);
\draw(1*\x,0)to [american voltage source,l_={${\hat{V}_{bn}}$},i={$\hat{I}_b$}]++(0,\y)  to [short]++(0,\y/2) to [short]++(\x,0)  to [short,-o,i={$\hat{I}_b$}]++(\x,0)node[above]{$b$}coordinate(kb);
\draw(2*\x,0) to [american voltage source,l_={${\hat{V}_{cn}}$},i={$\hat{I}_c$}] ++(0,\y)to [short,-o,i={$\hat{I}_c$}]++(1*\x,0)node[above]{$c$}coordinate(kc);
\draw(0,0) to [short,-*]++(1*\x,0) to [short,-*]++(1*\x,0) to [short,-o,i<_={$\hat{I}_n$}]++(1*\x,0)node[below]{$n$} to [short,o-]++(\x/2+2*\x,0);
%load
\draw(ka) to [short,o-]++(2*\x+\x/2,0) to [short]++(0,-\y) to [european resistor,l={$\bZ_a$}]++(0,-\y);
\draw(kb) to [short,o-]++(\x+\x/2,0) to [short] ++(0,-\y/2) to [european resistor,-*,l={$\bZ_b$}]++(0,-\y);
\draw(kc) to [short,o-]++(\x/2,0) to [european resistor,-*,l={$\bZ_c$}]++(0,-\y);
%text
\draw(0,\y+0.15)node[left]{\RL{دوری رو}};
\draw(2*\x+\x/2,2*\y)node[shift={(0,0.8)}]{\RL{رو تار}};
\end{tikzpicture}
\caption*{(الف)}
\end{subfigure}
\begin{subfigure}{1\textwidth}
\centering
\begin{tikzpicture}
%star voltages
\draw(0,0)node[left]{$n$} to [american voltage source,l={$\hat{V}_{an}$}]++(0:\x)coordinate(ka);
\draw(0,0) to [american voltage source,l={$\hat{V}_{bn}$}]++(-120:\x)coordinate(kb);
\draw(0,0) to [american voltage source,l_={$\hat{V}_{cn}$}]++(120:\x)coordinate(kc);
%star load
\draw(3*\x,0)node[right]{$n$} to [european resistor,l_={$\bZ_Y$}]++(180:\x)coordinate(kaa);
\draw(3*\x,0) to [european resistor,l_={$\bZ_Y$}]++(-60:\x)coordinate(kbb);
\draw(3*\x,0) to [european resistor,l_={$\bZ_Y$}]++(60:\x)coordinate(kcc);
%connections
\draw(ka)--(kaa);
\draw(kb)--(kbb);
\draw(kc)--(kcc);
\draw(0,0) to [short,*-]++(-45:3/4*\x)coordinate(nL);
\draw (3*\x,0) to [short,*-]++(-135:3/4*\x)--(nL)node[above,pos=0.5]{\RL{تعدیلی تار}};
\end{tikzpicture}
\caption*{(ب)}
\end{subfigure}%
\caption{متوازن چار تار، ستارہ ستارہ (\عددیء{YY}) نظام۔}
\label{شکل_تین_دوری_ستارہ_ستارہ_الف}
\end{figure} 

متوازن ستارہ بوجھ کی صورت میں \عددی{\bZ_a=\bZ_b=\bZ_c=\bZ_Y} ہو گا۔ایسی صورت میں شکل \حوالہ{شکل_تین_دوری_ستارہ_ستارہ_الف}-الف میں تین دوری رو درج ذیل ہوں گی جہاں \عددی{\hat{V}_a} کا زاویہ ہٹاو صفر لیا گیا ہے اور \عددی{\tfrac{V_0}{Z_Y}} کو \عددی{I_0} لکھا گیا ہے۔
\begin{gather}
\begin{aligned}\label{مساوات_تین_دوری_ستارہ_رو}
\hat{I}_a&=\frac{\hat{V}_a}{\bZ_Y}=\frac{V_0\phase{0^{\circ}}}{Z_Y\phase{\theta_{z}}}=\frac{V_0}{Z_Y}\phase{-\theta_z}=I_0\phase{-\theta_z}\\
\hat{I}_b&=\frac{\hat{V}_b}{\bZ_Y}=\frac{V_0\phase{-120^{\circ}}}{Z_Y\phase{\theta_{z}}}=\frac{V_0}{Z_Y}\phase{-120^{\circ}-\theta_z}=I_0\phase{-120^{\circ}-\theta_z}\\
\hat{I}_c&=\frac{\hat{V}_c}{\bZ_Y}=\frac{V_0\phase{120^{\circ}}}{Z_Y\phase{\theta_{z}}}=\frac{V_0}{Z_Y}\phase{120^{\circ}-\theta_z}=I_0\phase{120^{\circ}-\theta_z}
\end{aligned}
\end{gather} 
شکل \حوالہ{شکل_تین_دوری_ستارہ_ستارہ_الف}-الف میں منبعوں کے جوڑ پر کرخوف قانون رو کی مدد سے  تعدیلی تار میں رو \عددی{\hat{I}_n} کی مساوات لکھتے ہیں
\begin{align*}
\hat{I}_n=\hat{I}_a+\hat{I}_b+\hat{I}_c
\end{align*}
جس میں مساوات \حوالہ{مساوات_تین_دوری_ستارہ_رو} پر کرتے ہوئے ثابت ہوتا ہے کہ \عددی{\hat{I}_n} صفر کے برابر ہے۔
\begin{align}
\hat{I}_n=\hat{I}_a+\hat{I}_b+\hat{I}_c=0\quad \quad \text{\RL{متوازن ستارہ ستارہ میں تعدیلی رو صفر ہے}}
\end{align}
شکل \حوالہ{شکل_تین_دوری_تعدیلی_رو_صفر} میں پیچھے جزو طاقت کی صورت میں ستارہ رو اور ان کا مجموعہ دکھایا گیا ہے۔آپ دیکھ سکتے ہیں کہ متوازن ستارہ منبع اور متوازن ستارہ بوجھ کی صورت میں تعدیلی رو صفر ہو گی لہٰذا تعدیلی تار اتارنے  سے نظام پر کوئی اثر نہیں ہو گا۔ ہاں اگر ایک بوجھ یا ایک منبع کی قیمت تبدیل کر دی جائے تب اس شاخ کی رو تبدیل ہو جائے گی اور یوں تینوں شاخوں کی رو کا مجموعہ صفر نہ رہ پائے گا لہٰذا غیر متوازن صورت میں تعدیلی رو پائی جائے گی۔ تعدیلی تار نہ استعمال کرنے سے \اصطلاح{تین تار، ستارہ ستارہ}\فرہنگ{تین تار!ستارہ ستارہ}\فرہنگ{ستارہ ستارہ!تین تار}\حاشیہب{three-wire, star-star}\فرہنگ{star-star!three-wire} نظام حاصل ہوتا ہے۔جیسا مثال \حوالہ{مثال_تین_دوری_تعدیلی_تار_غیر_موجود} میں دکھایا گیا ہے، غیر متوازن تین تار، ستارہ ستارہ نظام میں بوجھ کے تینوں شاخوں پر مختلف دباو پائے جائیں گے لہٰذا غیر متوازن نظام میں تعدیلی تار کا استعمال کرنا نہایت اہم ہے۔
\begin{figure}
\centering
\begin{subfigure}{0.5\textwidth}
\centering
\begin{tikzpicture}
\pgfmathsetmacro{\ang}{-30}
\draw[dashed](0,0)--++(\x,0);
\draw([shift={(0:0.5)}]0,0) arc (0:\ang:0.5);
\draw(1/2*\ang:0.8)node{$\theta$};
%
\draw[-latex](0,0)--++(\ang:\x)node[right]{$\hat{I}_a$};
\draw[-latex](0,0)--++(\ang-120:\x)node[left]{$\hat{I}_b$};
\draw[-latex](0,0)--++(\ang+120:\x)node[left]{$\hat{I}_c$};
\end{tikzpicture}
\caption*{(الف)}
\end{subfigure}%
\begin{subfigure}{0.5\textwidth}
\centering
\begin{tikzpicture}
\pgfmathsetmacro{\ang}{-30}
\draw[-latex](0,0)--++(\ang:\x)node[pos=0.7,above]{$\hat{I}_a$};
\draw[-latex](0,0)++(\ang:\x)--++(\ang-120:\x)node[pos=0.7,below]{$\hat{I}_b$};
\draw[-latex](0,0)++(\ang:\x)++(\ang-120:\x)--++(\ang+120:\x)node[pos=0.7,left]{$\hat{I}_c$};
\end{tikzpicture}
\caption*{(ب)}
\end{subfigure}
\caption{متوازن منبع اور متوازن بوجھ کی صورت میں تعدیلی رو صفر  ہو گی۔}
\label{شکل_تین_دوری_تعدیلی_رو_صفر}
\end{figure}

متوازن ستارہ ستارہ نظام میں تینوں رو کی قیمت برابر ہوتی ہے جبکہ ان میں زاویائی فاصلہ \عددی{120^{\circ}} پایا جاتا ہے۔یوں ہم صرف ایک منبع اور اس کے بوجھ کو حل کرتے ہوئے تمام جوابات اخذ کر سکتے ہیں۔اس نظام میں تینوں تار کی رکاوٹ بھی برابر ہوتی ہے لہٰذا تار کی رکاوٹ کے اثرات شامل کرتے ہوئے بھی صرف ایک دور حل کرنا پڑتا ہے۔چونکہ متوازن ستارہ ستارہ نظام کے تعدیلی تار میں رو صفر رہتی ہے لہٰذا اس تار کی رکاوٹ کا نظام میں دباو اور رو پر کوئی اثر نہیں ہوتا لہٰذا تعدیلی تار کی رکاوٹ غیر اہم ہے۔یوں تعدیلی تار کی رکاوٹ کچھ بھی تصور کی جا سکتی ہے۔ہم تعدیلی تار کی رکاوٹ صفر تصور کریں گے۔
%================
\ابتدا{مثال}\شناخت{مثال_تین_دوری_تعدیلی_تار_غیر_موجود}
گھریلو صارفین کو تعدیلی تار اور ایک زندہ تار کے ذریعہ طاقت فراہم کیا جاتا ہے۔یوں ایک ہی محلے میں کچھ گھرانوں کو \عددی{\hat{V}_{an}} فراہم کیا جائے گا، کچھ کو
 \عددی{\hat{V}_{bn}} اور کچھ گھرانوں کو \عددی{\hat{V}_{cn}} فراہم کیا جائے گا۔یوں اوسطاً ترسیلی نظام کو متوازن صورت حال نظر آتی ہے۔حقیقت میں گھریلو بوجھ غیر متوازن بوجھ ہے لہٰذا اس کو تعدیلی تار سے جوڑنا ضروری ہے۔آئیں دیکھیں کہ ایسا نہ کرنے کے کیا نتائج  ہو سکتے ہیں۔

ٹرانسفارمر کے قریب تعدیلی تار کو زمین میں نمی کی گہرائی تک دھنسا جاتا ہے لہٰذا تعدیلی تار \اصطلاح{ٹھنڈی تار}\فرہنگ{ٹھنڈی تار}\فرہنگ{تار!ٹھنڈی} بھی کہلاتی ہے۔بعض اوقات تعدیلی تار کہیں سے ٹوٹ جاتی ہے۔شکل \حوالہ{شکل_تین_دوری_تعدیلی_تار_غیر_موجود} میں ایسا ہی دکھایا گیا ہے جہاں ایک گھرانے نے \عددی{\SI{1}{\kilo\watt}} کا پمپ چالو کیا ہوا ہے جبکہ بقایا دو گھرانوں نے ایک ایک عدد \عددی{\SI{100}{\watt}} کا بلب روشن کیا ہو ہے۔پمپ کو \عددی{\SI{52.9}{\ohm}} سے اور بلب کو \عددی{\SI{529}{\ohm}} سے ظاہر کیا گیا ہے۔دوری موثر دباو \عددی{\SI{230}{\volt}\,\rms} ہے۔گھروں میں تعدیلی تار یعنی نقطہ \عددی{n'} پر دباو دریافت کرتے ہوئے بلب پر دباو حاصل کریں۔
 \begin{figure}
\centering
\begin{tikzpicture}
%star voltages
\draw(0,0)node[above right]{$n$} to [american voltage source,l={$\hat{V}_{an}$}]++(0:\x)coordinate(ka);
\draw(0,0) to [american voltage source,l={$\hat{V}_{bn}$}]++(-120:\x)coordinate(kb);
\draw(0,0) to [american voltage source,l_={$\hat{V}_{cn}$}]++(120:\x)coordinate(kc);
\draw(0,0) to [short]++(-3/4*\x,0) node[ground]{};
%star load
\draw(3*\x,0)node[right]{$n'$} to [european resistor,l_={$\SI{529}{\ohm}$}]++(180:\x)coordinate(kaa);
\draw(3*\x,0) to [european resistor,l_={$\SI{52.9}{\ohm}$}]++(-60:\x)coordinate(kbb);
\draw(3*\x,0) to [european resistor,l_={$\SI{529}{\ohm}$}]++(60:\x)coordinate(kcc);
%connections
\draw(ka)--(kaa);
\draw(kb)--(kbb);
\draw(kc)--(kcc);
\draw(0,0)node[circ]{}++(-45:3/4*\x)coordinate(nL);
\draw[gray,dashed] (3*\x,0) to [short,*-]++(-135:3/4*\x)--(nL)node[above,pos=0.5]{\RL{تعدیلی تار}};
\end{tikzpicture}
\caption{مثال \حوالہ{مثال_تین_دوری_تعدیلی_تار_غیر_موجود} کا دور۔}
\label{شکل_تین_دوری_تعدیلی_تار_غیر_موجود}
\end{figure}

حل:نقطہ \عددی{n'} پر کرخوف مساوات رو لکھتے ہیں۔اس نقطے سے رو کو نکلتا ہوا لیا گیا ہے۔ 
\begin{align*}
\frac{\hat{V}_n'-\hat{V}_{an}}{52.9}+\frac{\hat{V}_n'-\hat{V}_{bn}}{529}+\frac{\hat{V}_n'-\hat{V}_{cn}}{529}&=0
\end{align*} 
اس میں منبع کے دباو پر کرتے ہوئے
\begin{align*}
\frac{\hat{V}_n'-230\phase{0^{\circ}}}{52.9}+\frac{\hat{V}_n'-230\phase{-120^{\circ}}}{529}+\frac{\hat{V}_n-230\phase{120^{\circ}}}{529}&=0
\end{align*} 
حل کرتے ہیں۔
\begin{align*}
\hat{V}_n'=172.5\phase{-120^{\circ}}\,\si{\volt}\,\rms
\end{align*}
یہاں غور کریں۔عام حالت میں تعدیلی تار پر صفر وولٹ کا دباو ہوتا ہے۔اسی لئے اس کو ٹھنڈی تار کہتے ہیں۔یہاں تعدیلی نقطے پر \عددی{\SI{172.5}{\volt}\,\rms} کا خطرناک دباو پایا جاتا ہے۔آئیں اب بلب پر دباو دیکھیں۔

منبع \عددی{\hat{V}_{an}} کے ساتھ جڑے بلب پر درج ذیل دباو ہو گا۔
\begin{align*}
230\phase{0^{\circ}}-172.5\phase{-120^{\circ}}=349.8\phase{25^{\circ}}\,\si{\volt}\,\rms
\end{align*}
آپ دیکھ سکتے ہیں کہ \عددی{\SI{230}{\volt}\,\rms} پر چلنے والا بلب \عددی{\SI{349.8}{\volt}\,\rms} کا تاب نہ لاتے ہوئے جھلس\حاشیہد{یہ گھرانے خوش قسمت ہیں۔بلب کی جگہ قیمتی کمپیوٹر\فرہنگ{کمپیوٹر} یا ٹیلی ویژن\فرہنگ{ٹیلی ویژن} بھی ہو سکتا تھا۔} جائے گا۔
\انتہا{مثال}
%================
\ابتدا{مثال}\شناخت{مثال_تین_دوری_ستارہ_ستارہ_بوجھ_رکاوٹ_الف}
متوازن تین دوری ستارہ ستارہ \عددی{abc} نظام میں موثر دوری دباو \عددی{\SI{230}{\volt}\,\rms} ہے جبکہ تار اور بوجھ کے رکاوٹ بالترتیب \عددی{0.5+j1\,\si{\ohm}} اور  \عددی{15+j12\,\si{\ohm}} ہیں۔تمام دباو بوجھ اور تار کی رو دریافت کریں۔

حل:شاخ \عددی{a} کو صفر زاویے پر رکھتے ہوئے  تین منبع کے دباو لکھتے ہیں۔
\begin{align*}
\hat{V}_{an}&=230\phase{0^{\circ}}\,\si{\volt}\,\rms\\
\hat{V}_{bn}&=230\phase{-120^{\circ}}\,\si{\volt}\,\rms\\
\hat{V}_{acn}&=230\phase{120^{\circ}}\,\si{\volt}\,\rms
\end{align*}
ستارہ ستارہ نظام کے ایک شاخ کو شکل \حوالہ{شکل_تین_دوری_ستارہ_ستارہ_بوجھ_رکاوٹ_الف} میں دکھایا گیا ہے جہاں سے درج ذیل لکھا جا سکتا ہے۔
\begin{align*}
\hat{I}_a&=\frac{230\phase{0^{\circ}}}{0.5+j1+15+j12}=11.37\phase{-40^{\circ}}\,\si{\ampere}\,\rms\\
\hat{V}_{AN}&=\left(\frac{15+j12}{0.5+j1+15+j12}\right) 230\phase{0^{\circ}}=218.4\phase{-1.3^{\circ}}\,\si{\volt}\,\rms
\end{align*}
ان جوابات کو \عددی{120^{\circ}} ہٹاو دیتے ہوئے بقایا جوابات لکھتے ہیں۔
\begin{align*}
\hat{I}_b&=11.37\phase{-120^{\circ}-40^{\circ}}=11.37\phase{-160^{\circ}}\,\si{\ampere}\,\rms\\
\hat{I}_c&=11.37\phase{+120^{\circ}-40^{\circ}}=11.37\phase{80^{\circ}}\,\si{\ampere}\,\rms\\
\hat{V}_{BN}&=218.4\phase{-120^{\circ}-1.3^{\circ}}=218.4\phase{-121.3^{\circ}}\,\si{\volt}\,\rms\\
\hat{V}_{CN}&=218.4\phase{+120^{\circ}-1.3^{\circ}}=218.4\phase{118.7^{\circ}}\,\si{\volt}\,\rms
\end{align*} 
%
\begin{figure}
\centering
\begin{tikzpicture}[american voltages]
\draw(0,0)node[below]{$n$} to [american voltage source,l={${230\phase{0^{\circ}}}\,\si{\volt}\,\rms$}]++(0,\y)node[above]{$a$} to [short,i={$\hat{I}_a$}]++(\x/2,0) to [european resistor,l={$0.5+j1\,\si{\ohm}$}]++(\x,0)node[above]{$A$} to [european resistor,l={$15+j12\,\si{\ohm}$},v={$\hat{V}_{AN}$}]++(0,-\y)node[below]{$N$} to [short](0,0);
\draw(0.3,0) to [open,v_>={$\hat{V}_{an}$}]++(0,\y);
\end{tikzpicture}
\caption{مثال \حوالہ{مثال_تین_دوری_ستارہ_ستارہ_بوجھ_رکاوٹ_الف} کا دور۔}
\label{شکل_تین_دوری_ستارہ_ستارہ_بوجھ_رکاوٹ_الف}
\end{figure}
\انتہا{مثال}
%==================
\ابتدا{مشق}
متوازن \عددی{abc} ستارہ جڑے منبع میں \عددی{\hat{V}_{an}=100\phase{180^{\circ}}\,\si{\volt}} ہے۔دباو تار حاصل کریں۔

جوابات:\عددی{\hat{V}_{ab}=173.2\phase{-150^{\circ}}\,\si{\volt}}، \عددی{\hat{V}_{ca}=173.2\phase{-30^{\circ}}\,\si{\volt}}، \عددی{\hat{V}_{bc}=173.2\phase{90^{\circ}}\,\si{\volt}}
\انتہا{مشق}
%================

\ابتدا{مشق}
متوازن \عددی{abc} ستارہ جڑے منبع میں \عددی{\hat{V}_{ab}=180\phase{150^{\circ}}\,\si{\volt}} ہے۔دوری دباو حاصل کریں۔

جوابات:\عددی{\hat{V}_{an}=86.6\phase{90^{\circ}}\,\si{\volt}}، \عددی{\hat{V}_{bn}=86.6\phase{-30^{\circ}}\,\si{\volt}}، \عددی{\hat{V}_{cn}=86.6\phase{-150^{\circ}}\,\si{\volt}}
\انتہا{مشق}
%================
\ابتدا{مشق}
ستارہ ستارہ \عددی{abc} ترتیب کے نظام میں بوجھ پر دباو \عددی{\hat{V}_{AN}=220\phase{-15.6^{\circ}}\,\si{\volt}\,\rms} ہے۔ستارہ بوجھ کے ایک دور کی رکاوٹ \عددی{4+j2\,\si{\ohm}} اور تار کی رکاوٹ \عددی{1+j1.5\,\si{\ohm}} ہے۔ستارہ منبع کی دوری دباو حاصل کریں۔

جوابات:\عددی{\hat{V}_{an}=300\phase{-7.2^{\circ}}\,\si{\volt}\,\rms}، \عددی{\hat{V}_{bn}=300\phase{-127.2^{\circ}}\,\si{\volt}\,\rms}، \\ \عددی{\hat{V}_{cn}=300\phase{112.8^{\circ}}\,\si{\volt}\,\rms}
\انتہا{مشق}
%================
\ابتدا{مشق}
متوازن ستارہ بوجھ کے ایک دور کی رکاوٹ \عددی{0.2-j0.12\,\si{\ohm}} ہے۔اس کو متوازن ستارہ منبع سے طاقت فراہم کی جاتی ہے جس کا دباو دور \عددی{\SI{110}{\volt}\,\rms} ہے۔نظام کی ترتیب \عددی{abc} ہے۔ دور \عددی{a} کا زاویہ ہٹاو صفر لیتے ہوئے تار کی رو دریافت کریں۔

جوابات:\عددی{\hat{I}_{a}=471\phase{31^{\circ}}\,\si{\ampere}\,\rms}، \عددی{\hat{I}_{b}=471\phase{-89^{\circ}}\,\si{\ampere}\,\rms}، 
\عددی{\hat{I}_{c}=471\phase{151^{\circ}}\,\si{\ampere}\,\rms}
\انتہا{مشق}
%==================
\ابتدا{مشق}
متوازن ستارہ ستارہ نظام میں تاروں میں کل ضیاع \عددی{\SI{962}{\watt}} ہے۔بوجھ کا دوری دباو \عددی{v_{AN}=240\phase{38^{\circ}}\,\si{\volt}\,\rms} جبکہ اس کا آگے جزو طاقت \عددی{0.69} ہے۔تار کی رکاوٹ \عددی{1.2+j1.5\,\si{\ohm}} ہے۔بوجھ کی دوری رکاوٹ دریافت کریں۔

جواب:\عددی{10.13-j10.63\,\si{\ohm}}
\انتہا{مشق}
%===================

\حصہ{تین دوری تکونی (\عددیء{\Delta}) دباو}
شکل \حوالہ{شکل_تین_ستارہ_اور_تکونی_دباو}-الف میں تین عدد منبع کو تین تاروں کے مابین \اصطلاح{تکونی}\فرہنگ{تکونی جوڑ}\حاشیہب{delta connected, $\Delta$}\فرہنگ{delta connection}\فرہنگ{$\Delta$} جوڑا گیا ہے۔مساوات \حوالہ{مساوات_تین_ستارہ_دور_تار_دباو_تعلق} دباو تار اور دوری دباو کا تعلق دیتا ہے۔یوں اگر شکل-الف کے تکونی جڑے منبع کے دباو
\begin{gather}
\begin{aligned}
\hat{V}_{ab}&=V_L\phase{0^{\circ}}\\
\hat{V}_{bc}&=V_L\phase{-120^{\circ}}\\
\hat{V}_{ca}&=V_L\phase{+120^{\circ}}
\end{aligned}
\end{gather}
ہوں جہاں \عددی{V_L} دباو تار کا حیطہ  ہے تب شکل-ب میں دکھائے گئے ان کے مساوی ستارہ منبع درج ذیل ہوں گے جہاں ستارہ جڑے منبع کے دباو کا حیطہ \عددی{V_p} لکھا گیا ہے۔
 \begin{gather}
\begin{aligned}
\hat{V}_{an}&=\frac{V_L}{\sqrt{3}}\phase{-30^{\circ}}=V_p\phase{-30^{\circ}}\\
\hat{V}_{bn}&=\frac{V_L}{\sqrt{3}}\phase{-150^{\circ}}=V_p\phase{-150^{\circ}}\\
\hat{V}_{cn}&=\frac{V_L}{\sqrt{3}}\phase{-270^{\circ}}=V_p\phase{90^{\circ}}
\end{aligned}
\end{gather}
یوں جہاں بھی تکونی منبع نسب ہو، اس کی جگہ مساوی ستارہ منبع نسب کرتے ہوئے دور کو ستارہ منبع کے تمام طریقوں سے حل کیا جا سکتا ہے۔آئیں اس پر ایک مثال دیکھیں۔ 
\begin{figure}
\centering
\begin{subfigure}{0.5\textwidth}
\centering
\begin{tikzpicture}
\pgfmathsetmacro{\len}{2*\x*sin(60)}
\pgfmathsetmacro{\kh}{\len*cos(30)}
%
\draw(0,0) to [american voltage source,*-*,l_={$\hat{V}_{bc}$}]++(0:\len)coordinate(kb);
\draw(0,0)++(0:\len)  to [american voltage source,*-*]++(120:\len)coordinate(ka);
\draw(0,0)++(0:\len)++(120:\len)  to [american voltage source,*-*]++(-120:\len)coordinate(kc);
%text
\draw(0:\len)++(120:\len/2)node[shift={(30:0.8)}]{$\hat{V}_{ab}$};
\draw(0,0)++(0:\len)++(120:\len)++(-120:\len/2)node[shift={(150:0.8)}]{$\hat{V}_{ca}$};
%leads
\draw(ka) to [short,-o]++(\len/2+\x/4,0) to [short]++(\x/4,0);
\draw(kb) to [short,-o]++(\x/4,0) to [short]++(\x/4,0);
\draw(kc) to [short]++(0,-3/4*\y) to [short,-o]++(\len+\x/4,0) to [short]++(\x/4,0)coordinate(kB);
\draw(kB)++(0,-0.5) rectangle ++(0.5,\kh+1+3/4*\y);
\end{tikzpicture}
\caption*{(الف) تکون \عددی{\Delta}}
\end{subfigure}%
\begin{subfigure}{0.5\textwidth}
\centering
\begin{tikzpicture}
\pgfmathsetmacro{\kl}{\x*sin(60)}
\pgfmathsetmacro{\kh}{\x+\x*cos(60)}
\draw(0,0) to [american voltage source,*-*,l={${\hat{V}_{an}}$}]++(90:\x) to [short]++(\kl,0) to [short,-o]++(\x/4,0) to [short]++(\x/4,0);
\draw(0,0) to [american voltage source,-*,l={${\hat{V}_{bn}}$}]++(-30:\x) to [short,-o]++(\x/4,0)to [short]++(\x/4,0);
\draw(0,0) to [american voltage source,-*,l={${\hat{V}_{cn}}$}]++(-150:\x) to [short]++(0,-3/4*\y)to [short,-o]++(2*\kl+\x/4,0)to [short]++(\x/4,0)coordinate(kB);
\draw(kB)++(0,-0.5) rectangle ++(0.5,\kh+1+3/4*\y);
\end{tikzpicture} 
\caption*{(ب) ستارہ \عددی{Y}}
\end{subfigure}%
\caption{ستارہ اور تکونی دباو۔}
\label{شکل_تین_ستارہ_اور_تکونی_دباو}
\end{figure}

کسی بھی تین عدد منبع کے منفی سر آپس میں جوڑنے سے ستارہ منبع حاصل ہو گا۔تین عدد منبع کو تکونی جوڑتے وقت چوکس رہنا ضروری ہے۔تکونی جوڑ میں ایک منبع کا منفی سر دوسرے منبع کے مثبت سر سے جڑتا ہے۔شکل \حوالہ{شکل_تین_صفر_متوازن_بدلتی_منبع_تکونی_ممکن}-الف میں تین متوازن بدلتی رو منبع کو تکونی جوڑا گیا ہے۔یہاں رک کر تسلی کر لیں کہ تین متوازی بدلتی رو منبع کو سلسلہ وار جوڑتے ہوئے ابتدائی سر \عددی{a'} اور اختتامی سر \عددی{a} کے مابین صفر وولٹ دباو پایا جاتا ہے لہٰذا انہیں آپس میں جوڑا کر تکونی منبع حاصل کیا جاتا ہے۔یہاں یہ تسلی بھی کر لیں کہ \عددی{V_L} دباو کے  یک سمتی منبع  کی صورت میں \عددی{a'} اور \عددی{a} کے مابین تین گنا دباو \عددی{(3V_L)} پایا جائے گا لہٰذا انہیں کسی بھی صورت آپس میں نہیں جوڑا جا سکتا ہے۔یہاں یہ بھی تسلی کر لیں کہ تینوں منبع کے دباو کی حتمی قیمت بالکل برابر ہونا ضروری ہے اور ان میں \عددی{120^{\circ}} زاویائی فرق بھی لازم ہے۔آپ یہ بھی دیکھ سکتے ہیں کہ تکونی منبع میں کسی ایک منبع کے الٹ جڑنے سے خطرناک نتائج رو نما ہو سکتے ہیں۔

تکونی منبع میں ایک دلچسپ بات یہ ہے کہ اس میں سے کسی ایک منبع کے ہٹانے سے دباو تار تبدیل نہیں ہوتے۔شکل \حوالہ{شکل_تین_صفر_متوازن_بدلتی_منبع_تکونی_ممکن}-ب میں نقطہ دار لکیر سے دکھائے گئے منبع کو ہٹاتے ہوئے تسلی کر لیں کہ تینوں دباو تار تبدیل نہیں ہوتے۔شکل-ب میں \اصطلاح{کھلا تکون}\فرہنگ{کھلا تکون}\فرہنگ{تکون!کھلا}\حاشیہب{open delta}\فرہنگ{open delta} دکھایا گیا ہے۔چونکہ طاقت کی فراہمی منبع سے ہوتی ہے لہٰذا کھلا تکون پورے تکونی منبع کے \عددی{\tfrac{2}{3}} گنا طاقت فراہم کرے گا۔ 
\begin{figure}
\centering
\begin{subfigure}{0.6\textwidth}
\centering
\begin{tikzpicture}
\draw(0,0)node[above]{$a'$} to [american voltage source,o-o,l={$V_L\phase{120^{\circ}}$}]++(\x,0)node[above]{$c$} to [american voltage source,o-o,l={$V_L\phase{-120^{\circ}}$}]++(\x,0)node[above]{$b$} to [american voltage source,l={$V_L\phase{0^{\circ}}$}] ++(\x,0)node[above]{$a$} to [short]++(0,-\y) to [short]++(-3*\x,0) to [short]++(0,\y);
\end{tikzpicture}
\caption*{(الف) صرف متوازن بدلتی رو منبع کو تکونی جوڑا جا سکتا ہے۔}
\end{subfigure}%
\begin{subfigure}{0.4\textwidth}
\centering
\begin{tikzpicture}
\draw(0,0)node[below]{$b$} to [american voltage source,*-*,l_={$V_L\phase{0^{\circ}}$}]++(2*\x,0)node[below]{$a$} to [american voltage source,*-*]++(120:2*\x)node[above]{$c$};
\draw[gray,dashed](60:2*\x) to [american voltage source](0,0);
%text
\draw(2*\x,0)++(120:\x)node[shift={(1.3,0)}]{$V_L\phase{120^{\circ}}$};
\end{tikzpicture}
\caption*{(ب) کھلا تکون۔}
\end{subfigure}%
\caption{تکونی منبع دباو۔}
\label{شکل_تین_صفر_متوازن_بدلتی_منبع_تکونی_ممکن}
\end{figure}
%==============
\ابتدا{مشق}
شکل \حوالہ{شکل_تین_صفر_متوازن_بدلتی_منبع_تکونی_ممکن}-الف میں ثابت کریں کہ \عددی{a'} تا \عددی{a} دباو صفر کے برابر ہے۔
\انتہا{مشق}
%================
\ابتدا{مشق}
شکل \حوالہ{شکل_تین_صفر_متوازن_بدلتی_منبع_تکونی_ممکن}-ب میں منبع \عددی{\hat{V}_{bc}} نہیں پایا جاتا ہے۔بقایا دو منبع کے دباو سے  نقطہ \عددی{c} تا \عددی{b} دباو \عددی{\hat{V}_{bc}} حاصل کریں۔
\انتہا{مشق}
%===============
\ابتدا{مثال}\شناخت{مثال_تین_تکون_سے_ستارہ_الف}
شکل \حوالہ{شکل_تین_تکون_سے_ستارہ_الف}-الف میں تکونی منبع کی جگہ مساوی ستارہ منبع نسب کرتے ہوئے  تار کی رو اور بوجھ پر دباو  حاصل کریں۔
\begin{figure}
\centering
\begin{subfigure}{1\textwidth}
\centering
\begin{tikzpicture}
\draw(0,0)node[circ]{}node[below]{$c$}coordinate(kc) to [american voltage source,l_={$415\phase{-120^{\circ}}\,\si{\volt}$}]++(0,\yy)node[circ]{}node[left]{$b$}coordinate(kb) to [american voltage source,l_={$415\phase{0^{\circ}}\,\si{\volt}$}]++(0,\yy)node[circ]{}node[above]{$a$}coordinate(ka) to [short]++(-\xx/2,0) to [american voltage source,l_={$415\phase{120^{\circ}}\,\si{\volt}$}]++(0,-2*\yy) to [short]++(\xx/2,0);
\draw(ka) to [european resistor,-*,l={$0.1+j0.15\,\si{\ohm}$}]++(\xx,0)node[above]{$A$} to [european resistor,l={$6+j2\,\si{\ohm}$}]++(\xx,0)coordinate(kT);
\draw(kb) to [european resistor,-*,l={$0.1+j0.15\,\si{\ohm}$}]++(\xx,0)node[above]{$B$} to [european resistor,l={$6+j2\,\si{\ohm}$}]++(\xx,0);
\draw(kc) to [european resistor,-*,l={$0.1+j0.15\,\si{\ohm}$}]++(\xx,0)node[above]{$B$} to [european resistor,l={$6+j2\,\si{\ohm}$}]++(\xx,0);
\draw(kT) to [short,-*]++(0,-\yy)node[right]{$N$} to [short]++(0,-\yy);
\end{tikzpicture}
\caption*{(الف) تکونی منبع پر ستارہ بوجھ لدا ہے۔}
\end{subfigure}
\begin{subfigure}{1\textwidth}
\centering
\begin{tikzpicture}[american voltages]
\draw(0,0)node[below]{$n$} to [american voltage source,l={${\frac{415}{\sqrt{3}}\phase{-30^{\circ}}}$}]++(0,\yy)node[above]{$a$} to [european resistor,-*,l={$0.1+j0.15\,\si{\ohm}$}]++(\xx,0)node[above]{$A$} to [european resistor,l={$6+j2\,\si{\ohm}$},i>^={$\hat{I}_a$},v={$\hat{V}_{AN}$}]++(0,-\yy)node[below]{$N$} to [short] (0,0);
\end{tikzpicture}
\caption*{(ب) تکونی منبع کا مساوی ستارہ منبع نسب کرتے ہوئے ایک شاخ دکھایا گیا ہے۔}
\end{subfigure}%
\caption{مثال \حوالہ{مثال_تین_تکون_سے_ستارہ_الف} کا دور۔}
\label{شکل_تین_تکون_سے_ستارہ_الف}
\end{figure}

حل:شکل-ب میں تکونی منبع کا مساوی ستارہ منبع استعمال کرتے ہوئے ایک شاخ دکھایا گیا ہے جس سے درج ذیل حاصل کیا جا سکتا ہے۔
\begin{align*}
\hat{I}_a&=\frac{\frac{415}{\sqrt{3}}\phase{-30^{\circ}}}{0.1+j0.15+6+j2}=37\phase{-49^{\circ}}\,\si{\ampere}\\
\hat{V}_{AN}&=\frac{415}{\sqrt{3}}\phase{-30^{\circ}}\left(\frac{6+j2}{0.1+j0.15+6+j2}\right)=234\phase{-31^{\circ}}\,\si{\volt}
\end{align*}
یوں بوجھ پر دباو تار \عددی{\sqrt{3}(234)=\SI{405}{\volt}} ہو گا۔آپ دیکھ سکتے ہیں کہ منبع پر دباو تار کی قیمت بوجھ پر دباو تار سے زیادہ ہے لہٰذا دباو کی بات کرتے وقت مقام کی  وضاحت ضروری ہے۔
\انتہا{مثال}
%==================
\ابتدا{مشق}
شکل \حوالہ{شکل_تین_تکون_سے_ستارہ_الف} میں تار کی رکاوٹ \عددی{0.8+j1\,\si{\ohm}} اور بوجھ کی دوری رکاوٹ \عددی{14-j6\,\si{\ohm}} جبکہ تکونی منبع کا دباو \عددی{\hat{V}_{ab}=66\phase{0^{\circ}}\,\si{\volt}} لیتے ہوئے بوجھ کی رو اور دوری دباو حاصل کریں۔

جوابات:\عددی{\hat{I}_a=2.4\phase{-11^{\circ}}\,\si{\ampere}}، \عددی{\hat{V}_{AN}=37\phase{-35^{\circ}}\,\si{\volt}}
\انتہا{مشق}
%================
\ابتدا{مشق}\شناخت{مشق_تین_ستارہ_تکونی_الف}
شکل \حوالہ{شکل_تین_ستارہ_تکونی_الف} میں \عددی{\b_Z_{\text{تار}}=0.4+j0.2\,\si{\ohm}} اور \عددی{\b_Z_{\text{بوجھ}}=12+j4\,\si{\ohm}} ہیں۔ بوجھ پر موثر دباو تار دریافت کریں۔
\begin{figure}
\centering
\begin{tikzpicture}
%delta supplies
\draw(0,0)node[circ]{}node[below]{$c$}coordinate(kc) to [american voltage source,l_={$440\phase{-120^{\circ}}\,\si{\volt}\,\rms$}]++(0,\yy)node[circ]{}node[left]{$b$}coordinate(kb) to [american voltage source,l_={$440\phase{0^{\circ}}\,\si{\volt}$\,\rms}]++(0,\yy)node[circ]{}node[above]{$a$}coordinate(ka) to [short]++(-\xx/2,0) to [american voltage source,l_={$440\phase{120^{\circ}}\,\si{\volt}\,\rms$}]++(0,-2*\yy) to [short]++(\xx/2,0);
%wire impedances
\draw(ka) to [european resistor,-*,l={$\bZ_{\text{تار}}$}]++(2*\xx,0)node[above]{$A$}coordinate(ka) to [european resistor,l={$\bZ_{\text{بوجھ}}$}]++(0,-\yy);
\draw(kb) to [european resistor,-*,l={$\bZ_{\text{تار}}$}]++(2*\xx,0)node[right]{$B$} to [european resistor,l={$\bZ_{\text{بوجھ}}$}]++(0,-\yy);
\draw(kc) to [european resistor,-*,l={$\bZ_{\text{تار}}$}]++(2*\xx,0)node[below]{$C$}coordinate(kc) to [short]++(\xx/2,0) to [european resistor,l_={$\bZ_{\text{بوجھ}}$}]++(0,2*\yy) to [short]++(-\xx/2,0);
\end{tikzpicture}
\caption{مشق \حوالہ{مشق_تین_ستارہ_تکونی_الف} کا تکون تکون \عددیء{\Delta \Delta} دور۔}
\label{شکل_تین_ستارہ_تکونی_الف}
\end{figure}

جواب:\عددی{V_L=\SI{398}{\volt}\,\rms}
\انتہا{مشق}
%================

\حصہ{تکونی بوجھ}
شکل \حوالہ{شکل_تین_دوری_ستارہ_تکونی_الف} میں ستارہ منبع کے ساتھ \اصطلاح{تکونی بوجھ}\فرہنگ{تکونی!بوجھ}\فرہنگ{بوجھ!تکونی} جوڑا گیا ہے۔آپ دیکھ سکتے ہیں کہ تکونی بوجھ یا تکونی منبع کی صورت میں تعدیلی تار استعمال نہیں کیا جا سکتا ہے لہٰذا یہ \اصطلاح{تین تار}\فرہنگ{تین تار!نظام} کا نظام ہو گا۔بوجھ کے ایک شاخ پر دباو تار پایا جاتا ہے۔یوں اگر ستارہ دباو درج ذیل ہوں
\begin{align*}
\hat{V}_{an}&=V_p\phase{0^{\circ}}\\
\hat{V}_{bn}&=V_p\phase{-120^{\circ}}\\
\hat{V}_{cn}&=V_p\phase{+120^{\circ}}
\end{align*}
تب دباو تار درج ذیل ہوں گے جہاں تار کی رکاوٹ نہ ہونے کی وجہ سے منبع اور بوجھ پر برابر دباو تار پایا جاتا ہے۔
\begin{align*}
\hat{V}_{ab}&=\sqrt{3}V_p\phase{30^{\circ}}=V_L\phase{30^{\circ}}=\hat{V}_{AB}\\
\hat{V}_{bc}&=\sqrt{3}V_p\phase{-90^{\circ}}=V_L\phase{-90^{\circ}}=\hat{V}_{BC}\\
\hat{V}_{ca}&=\sqrt{3}V_p\phase{-210^{\circ}}=V_L\phase{150^{\circ}}=\hat{V}_{CA}
\end{align*}
شکل \حوالہ{شکل_تین_دوری_ستارہ_تکونی_الف} کو دیکھ کر 
\begin{align*}
\hat{I}_{AB}&=\frac{\hat{V}_{AB}}{\bZ_{\Delta}}=\frac{V_L\phase{30^{\circ}}}{Z\phase{\theta_z}}=I_0\phase{30^{\circ}-\theta_z}\\
\hat{I}_{BC}&=\frac{\hat{V}_{BC}}{\bZ_{\Delta}}=\frac{V_L\phase{-90^{\circ}}}{Z\phase{\theta_z}}=I_0\phase{-90^{\circ}-\theta_z}\\
\hat{I}_{CA}&=\frac{\hat{V}_{CA}}{\bZ_{\Delta}}=\frac{V_L\phase{150^{\circ}}}{Z\phase{\theta_z}}=I_0\phase{150^{\circ}-\theta_z}
\end{align*}
لکھا جا سکتا ہے جہاں \عددی{\bZ_{\Delta}=Z\phase{\theta_z}} ہے اور مساوات میں \عددی{\tfrac{V_L}{Z}=I_0} لکھا گیا ہے۔درج بالا رو آپس میں \عددی{120^{\circ}} زاویائی فاصلے پر پائے جاتے ہیں جبکہ تینوں رو کی حتمی قیمت برابر ہے۔
\begin{figure}
\centering
\begin{tikzpicture}
%star source
\draw(0,0) to [american voltage source,-o,l={$\hat{V}_{cn}$}]++(\xx,0)node[above]{$c$} to [short,-*,i={$\hat{I}_c$}]++(\xx,0)node[below]{$C$}coordinate(kc);
\draw(0,\yy) to [american voltage source,-o,l={$\hat{V}_{bn}$}]++(\xx,0)node[above]{$b$} to [short,-*,i={$\hat{I}_b$}]++(\xx,0)node[right]{$B$} coordinate(kb);
\draw(0,2*\yy) to [american voltage source,-o,l={$\hat{V}_{an}$}]++(\xx,0)node[above]{$a$} to [short,-*,i={$\hat{I}_a$}]++(\xx,0)node[above]{$A$}coordinate(ka);
\draw(0,0) to [short,-*]++(0,\yy)node[left]{$n$} to [short]++(0,\yy);
%delta load
\draw(kc) to [short]++(\xx,0) to [european resistor,l_={$\bZ_{\Delta}$},i>_={$\hat{I}_{CA}$}]++(0,2*\yy) to [short]++(-\xx,0) to [european resistor,l={$\bZ_{\Delta}$},i>_={$\hat{I}_{AB}$}]++(0,-\yy) to [european resistor,l={$\bZ_{\Delta}$},i>_={$\hat{I}_{BC}$}]++(0,-\yy);
\end{tikzpicture}
\caption{تین تار، ستارہ تکونی نظام۔}
\label{شکل_تین_دوری_ستارہ_تکونی_الف}
\end{figure}
تکونی بوجھ کی رو کو شکل \حوالہ{شکل_تین_تکونی_بوجھ_تار_رو} میں دکھایا گیا ہے۔
\begin{figure}
\centering
\begin{tikzpicture}
\pgfmathsetmacro{\ang}{25}
\draw[dashed](0,0)--++(\x/2,0);
\draw([shift={(0:0.5)}]0,0) arc (0:\ang:0.5);
\draw(\ang/2:0.8)node[right]{$30^{\circ}-\theta_z$};
\draw[-latex](0,0)--++(\ang:\x)node[above]{$\hat{I}_{AB}$}coordinate(ka);
\draw[-latex](0,0)--++(\ang-120:\x)node[below]{$\hat{I}_{BC}$};
\draw[-latex](0,0)--++(\ang+120:\x)node[left]{$\hat{I}_{CA}$}node[below left,pos=0.5]{$I_0$}coordinate(kc);
%
\draw[-latex](kc)--(ka)node[pos=0.5,above,sloped]{$\hat{I}_{a}$};
\draw([shift={(\ang:0.3)}]0,0) arc (\ang:\ang+120:0.3);
\draw(\ang+60:0.5)node{$120^{\circ}$};
\end{tikzpicture}
\caption{تکونی بوجھ کی رو سے تار کی رو کا حصول۔}
\label{شکل_تین_تکونی_بوجھ_تار_رو}
\end{figure}

درج بالا حاصل کردہ بوجھ کی رو استعمال کرتے ہوئے تار کی رو حاصل کرتے ہیں۔شکل \حوالہ{شکل_تین_دوری_ستارہ_تکونی_الف} کو دیکھ کر کرخوف مساوات رو سے درج ذیل لکھتے ہیں
\begin{align*}
\hat{I}_a&=\hat{I}_{AB}-\hat{I}_{CA}
\end{align*}
جسے شکل \حوالہ{شکل_تین_تکونی_بوجھ_تار_رو} میں ترسیمی طریقے سے حل کرنا دکھایا گیا ہے۔اس شکل میں دکھائے گئے تکون کا زاویہ \عددی{120^{\circ}} اور اس کے دونوں  اطراف \عددی{I_0} کے برابر ہیں۔یوں تار کے رو کا حیطہ کوسائن کے کلیے سے درج ذیل حاصل ہوتا ہے
\begin{align*}
I_a=\sqrt{I_0^2+I_0^2-2I_0^2\cos 120^{\circ}}=\sqrt{3}I_0
\end{align*}
جبکہ اس کا زاویہ \عددی{-\theta_z} ہے لہٰذا تار کی رو \عددی{\hat{I}_a=\sqrt{3}I_0 \phase{-\theta_z}}  ہے۔بقایا دو تاروں کی رو بھی اسی طرح حاصل کی جا سکتا ہے۔
\begin{gather}
\begin{aligned}
\hat{I}_a&=\sqrt{3}I_0 \phase{-\theta_z}\\
\hat{I}_b&=\sqrt{3}I_0 \phase{-120^{\circ}-\theta_z}\\
\hat{I}_c&=\sqrt{3}I_0 \phase{+120^{\circ}-\theta_z}
\end{aligned}
\end{gather}

شکل \حوالہ{شکل_تین_دوری_ستارہ_تکونی_الف} میں تکونی بوجھ کی جگہ اس کا مساوی ستارہ بوجھ نسب کرنے سے شکل \حوالہ{شکل_تین_دوری_ستارہ_تکونی_سے_ستارہ_ستارہ}-الف  حاصل ہوتا ہے۔صفحہ \حوالہصفحہ{حصہ_مزاحمتی_ستارہ_تکون_تبادلہ} پر ستارہ-تکون تبادلہ پر غور کیا گیا ہے جہاں مساوات \حوالہ{مساوات_مزاحمتی_متوازن_تکون_سے_ستارہ} متوازن تکونی مزاحمتی بوجھ کا مساوی ستارہ بوجھ دیتا ہے۔یہی مساوات متوازن رکاوٹی بوجھ کے لئے بھی قابل استعمال ہے لہٰذا متوازن تکونی بوجھ کا مساوی ستارہ بوجھ \عددی{\tfrac{\bZ_{\Delta}}{3}} ہے۔تکونی جوڑ میں تعدیلی نقطہ \عددی{N} نہیں پایا جاتا ہے۔شکل \حوالہ{شکل_تین_دوری_ستارہ_تکونی_سے_ستارہ_ستارہ}-الف میں مساوی ستارہ بوجھ کا تعدیلی نقطہ \عددی{N} دکھایا گیا ہے جسے ستارہ منبع کے تعدیلی نقطہ \عددی{n} کے ساتھ سے جوڑا گیا ہے۔تعدیلی تار کو نقطہ دار لکیر سے دکھایا گیا ہے۔ہم جانتے ہیں کہ متوازن نظام میں تعدیلی تار میں رو صفر کے برابر ہوتی ہے اور اس کو استعمال کرنے یا نہ کرنے سے جوابات پر کوئی اثر نہیں ہوتا۔موجودہ دور میں تعدیلی تار کے استعمال سے دور کو حل کرنے میں مدد ملتی ہے لہٰذا اس کو استعمال کیا گیا ہے۔شکل-ب میں ستارہ ستارہ دور کی ایک شاخ دکھائی گئی ہے جس سے تار کی رو لکھتے ہیں
\begin{align*}
\hat{I}_a&=\frac{\hat{V}_{an}}{\frac{\bZ_{\Delta}}{3}}\\
&=\frac{3V_p\phase{0^{\circ}}}{Z\phase{\theta_z}}\\
&=\frac{\sqrt{3}V_L\phase{-\theta_z}}{Z}\\
&=\sqrt{3}I_0\phase{-\theta_z}
\end{align*}
جہاں \عددی{V_p=\tfrac{V_L}{\sqrt{3}}} اور \عددی{\tfrac{V_L}{Z}=I_0} کا استعمال کیا گیا ہے۔

آپ دیکھ سکتے ہیں کہ تکونی بوجھ کا مساوی ستارہ بوجھ استعمال کرنے سے دور حل کرنے میں مدد ملتی ہے۔یہی وجہ ہے کہ تین دوری نظام کو حل کرتے ہوئے پہلے ستارہ ستارہ دور حاصل کیا جاتا ہے۔اس ستارہ ستارہ دور کو حل کیا جاتا ہے اور آخر میں درکار جوابات ستارہ تکونی تبادلہ سے حاصل کئے جاتے ہیں۔ 
%
\begin{figure}
\centering
\begin{subfigure}{0.6\textwidth}
\centering
\begin{tikzpicture}
%star source
\draw(0,0) to [american voltage source,-o,l={$\hat{V}_{cn}$}]++(\xx,0)node[above]{$c$} to [short,-*,i={$\hat{I}_c$}]++(\xx/2,0)node[above]{$C$}coordinate(kc);
\draw(0,\yy) to [american voltage source,-o,l={$\hat{V}_{bn}$}]++(\xx,0)node[above]{$b$} to [short,-*,i={$\hat{I}_b$}]++(\xx/2,0)node[above]{$B$} coordinate(kb);
\draw(0,2*\yy) to [american voltage source,-o,l={$\hat{V}_{an}$}]++(\xx,0)node[above]{$a$} to [short,-*,i={$\hat{I}_a$}]++(\xx/2,0)node[above]{$A$}coordinate(ka);
\draw(0,0) to [short,-*]++(0,\yy)node[left]{$n$}coordinate(nL) to [short]++(0,\yy);
%star load
\draw(kc) to [european resistor,l={$\frac{\bZ_{\Delta}}{3}$}]++(\xx,0)coordinate(kB);
\draw(kb) to [european resistor,l={$\frac{\bZ_{\Delta}}{3}$}]++(\xx,0);
\draw(ka) to [european resistor,l={$\frac{\bZ_{\Delta}}{3}$}]++(\xx,0);
\draw(kB) to [short,-*]++(0,\yy)coordinate(nR)node[right]{$N$} to [short]++(0,\yy);
%neutral
\draw[dashed](nL)--++(-45:\x/2)coordinate(knL);
\draw[dashed](nR)--++(-135:\x/2)--(knL);
\end{tikzpicture}
\caption*{(الف)}
\end{subfigure}%
\begin{subfigure}{0.4\textwidth}
\centering
\begin{tikzpicture}
\draw(0,0)node[below]{$n$} to [american voltage source,l={$\hat{V}_{an}$}]++(0,\yy)node[above]{$a$} to [short,i={$\hat{I}_a$}]++(\xx,0)node[above]{$A$} to [european resistor,l={$\frac{\b_Z_{\Delta}}{3}$}]++(0,-\yy)node[below]{$N$} to [short] (0,0);
\end{tikzpicture}
\caption*{(ب)}
\end{subfigure}%
\caption{تکونی بوجھ کی جگہ مساوی ستارہ بوجھ نسب کیا گیا ہے۔}
\label{شکل_تین_دوری_ستارہ_تکونی_سے_ستارہ_ستارہ}
\end{figure}
%=================
\ابتدا{مثال}
متوازی تکونی بوجھ کے ایک شاخ کی رکاوٹ \عددی{5+j3\,\si{\ohm}} ہے۔اس پر متوازن دباو تار لاگو کی جاتی ہے۔بوجھ کے تمام شاخوں کی رو اور تمام تاروں کی رو دریافت کریں۔ستارہ منبع کے ایک شاخ کا دباو \عددی{\hat{V}_{an}=240\phase{42^{\circ}}\,\si{\volt}} ہے۔

حل:دباو تار درج ذیل ہیں جہاں تار کی رکاوٹ نہ ہونے کی وجہ سے منبع اور بوجھ کے دباو تار برابر ہیں۔
\begin{align*}
\hat{V}_{ab}=240\sqrt{3}\phase{72^{\circ}}=\hat{V}_{AB}\\
\hat{V}_{bc}=240\sqrt{3}\phase{-48^{\circ}}=\hat{V}_{BC}\\
\hat{V}_{ca}=240\sqrt{3}\phase{192^{\circ}}=\hat{V}_{CA}
\end{align*}
یوں بوجھ کے شاخوں کی رو درج ذیل ہو گی۔
\begin{align*}
\hat{I}_{AB}&=\frac{\hat{V}_{AB}}{5+j3}=71.3\phase{41^{\circ}}\,\si{\ampere}
\end{align*}
بقایا دو شاخوں کی رو \عددی{\mp120^{\circ}} زاویائی فاصلے پر ہو گی یعنی
\begin{align*}
\hat{I}_{BA}&=71.3\phase{-79^{\circ}}\,\si{\ampere}\\
\hat{I}_{CA}&=71.3\phase{161^{\circ}}\,\si{\ampere}
\end{align*}
تار کی رو حاصل کرنے کی خاطر ستارہ بوجھ استعمال کرتے ہیں۔
\begin{align*}
\bZ_{Y}=\frac{\bZ_{\Delta}}{3}=\frac{5+j3}{3}=\frac{5}{3}+j1\,\si{\ohm}
\end{align*}
یوں تار کی رو درج ذیل ہو گی۔
\begin{align*}
\hat{I}_a=\frac{\hat{V}_{an}}{\bZ_{Y}}=\frac{240\phase{42^{\circ}}}{\frac{5}{3}+j1}=123.5\phase{11^{\circ}}\,\si{\ampere}
\end{align*}
بقایا دو تاروں کی رو \عددی{\mp120^{\circ}} زاویائی فاصلے پر ہوں گی۔
\begin{align*}
\hat{I}_b&=123.5\phase{-109^{\circ}}\,\si{\ampere}\\
\hat{I}_c&=123.5\phase{131^{\circ}}\,\si{\ampere}\\
\end{align*}
\انتہا{مثال}
%==========================
\ابتدا{مشق}\شناخت{مشق_تین_دوری_تکون_ستارہ_تکون}
شکل \حوالہ{شکل_تین_دوری_تکون_ستارہ_تکون} میں تکونی منبع کے ساتھ \عددی{4-j3\,\si{\ohm}} کا تکونی بوجھ اور \عددی{2+j1.5\,\si{\ohm}} کا ستارہ بوجھ متوازی جڑے ہیں۔تار کی رو دریافت کریں۔
\begin{figure}
\centering
\begin{tikzpicture}
%delta source
\draw(0,0)node[below]{$c$}coordinate(kc) to [american voltage source,*-*,l_={${415\phase{-120^{\circ}}\,\si{\volt}\,\rms}$}]++(0,\yy) node[left]{$b$}coordinate(kb) to [american voltage source,*-*,l_={${415\phase{0^{\circ}}\,\si{\volt}\,\rms}$}]++(0,\yy) node[above]{$a$}coordinate(ka) to [short]++(-\xx/2,0) to [american voltage source,l_={${415\phase{120^{\circ}}\,\si{\volt}\,\rms}$}]++(0,-2*\yy) to [short]++(\xx/2,0);
%wires and star load
\draw(ka) to [european resistor,l={$0.2+j0.4\,\si{\ohm}$}]++(\xx,0) to [short]++(1*\xx,0) to [european resistor,l={$2+j1.5\,\si{\ohm}$}]++(\xx,0);
\draw(kb) to [european resistor,l={$0.2+j0.4\,\si{\ohm}$}]++(\xx,0) to [short]++(1*\xx,0) to [european resistor,l={$2+j1.5\,\si{\ohm}$}]++(\xx,0);
\draw(kc) to [european resistor,l_={$0.2+j0.4\,\si{\ohm}$}]++(\xx,0) to [short]++(\xx/2,0)coordinate(kB) to [short]++(\xx/2,0,0) to [european resistor,l_={$2+j1.5\,\si{\ohm}$}]++(\xx,0) to [short,-*]++(0,\yy) to [short]++(0,\yy);
%delta load
\draw(kB) to [european resistor,*-*]++(0,\yy) to [european resistor,-*]++(0,\yy)++(\xx/2,0) to [short,*-]++(0,-\yy) to [european resistor,-*]++(0,-\yy);
\draw(kB)++(0.5,\yy/2)node[rotate=90]{$4-j3\,\si{\ohm}$};
\draw(kB)++(0.5,\yy+\yy/2)node[rotate=90]{$4-j3\,\si{\ohm}$};
\draw(kB)++(\xx/2+0.5,\yy/2)node[rotate=90]{$4-j3\,\si{\ohm}$};
\end{tikzpicture}
\caption{مشق \حوالہ{مشق_تین_دوری_تکون_ستارہ_تکون} کا دور۔}
\label{شکل_تین_دوری_تکون_ستارہ_تکون}
\end{figure}

جوابات:\عددی{\hat{I}_a=166.5\phase{-38.7^{\circ}}\,\si{\ampere}\,\rms}، \عددی{\hat{I}_b=166.5\phase{-158.7^{\circ}}\,\si{\ampere}\,\rms}،\\ \عددی{\hat{I}_c=166.5\phase{81.3^{\circ}}\,\si{\ampere}\,\rms}
\انتہا{مشق}
%=========================

\حصہ{طاقت کے کلیات}
چاہے بوجھ ستارہ جڑا ہو یا تکونی، فی دور حقیقی طاقت اور  متعاملی طاقت درج ذیل ہوں گے جہاں \عددی{V_p} موثر دوری دباو، \عددی{I_p} موثر دوری رو اور \عددی{\theta} ان کے مابین زاویہ یعنی زاویہ رکاوٹ \عددی{\theta_z} ہیں۔
\begin{gather}
\begin{aligned}\label{مساوات_تین_طاقت_حقیقی-متعاملی_الف}
P_p&=V_p I_p \cos \theta\\
Q_p&=V_p I_p \sin \theta
\end{aligned}
\end{gather} 
ستارہ جڑے نظام میں \عددی{V_p=\tfrac{V_L}{\sqrt{3}}} اور \عددی{I_p=I_L} جبکہ تکونی نظام میں \عددی{V_p=V_L} اور \عددی{I_p=\frac{I_L}{\sqrt{3}}} لکھے جا سکتے ہیں جہاں \عددی{V_L} دباو تار اور \عددی{I_L} رو تار ہیں۔ اس طرح  مساوات \حوالہ{مساوات_تین_طاقت_حقیقی-متعاملی_ب} کو درج ذیل لکھا جا سکتا ہے
\begin{gather}
\begin{aligned}\label{مساوات_تین_طاقت_حقیقی-متعاملی_ب}
P_p&=\frac{V_L I_L}{\sqrt{3}} \cos \theta\\
Q_p&=\frac{V_L I_L}{\sqrt{3}} \sin \theta
\end{aligned}
\end{gather}
جس سے تینوں دور کی کل طاقت درج ذیل حاصل ہوتی ہے جہاں یاد رہے کہ \عددی{\theta} درحقیقت کسی ایک شاخ کے  بوجھ کا زاویہ \عددی{\theta_z} ہے۔
\begin{gather}
\begin{aligned}\label{مساوات_تین_طاقت_حقیقی-متعاملی_پ}
P_T&=3P_p=\sqrt{3}V_L I_L \cos \theta\\
Q_T&=3Q_p=\sqrt{3}V_L I_L \sin \theta
\end{aligned}
\end{gather}
یوں مخلوط طاقت کی حتمی قیمت اور زاویہ  درج ذیل ہوں گے۔
\begin{gather}
\begin{aligned}
S_T&=\sqrt{P^2_T+Q^2_Y}\\
&=\sqrt{3}V_L I_L\\
\phase{\bf{S}_T}=\theta
\end{aligned}
\end{gather}
%======================
\ابتدا{مثال}
شکل \حوالہ{شکل_تین_دوری_جزو_طاقت_تکونی_بوجھ}-الف میں تکونی بوجھ دکھایا گیا ہے۔جزو طاقت کے لئے دباو اور رو کے مابین زاویائی فرق جاننا ضروری ہے۔رو \عددی{\hat{I}} کا زاویہ  دباو \عددی{\hat{V}_1} سے ناپا جائے گا کہ دباو \عددی{\hat{V}_2} سے ناپا جائے گا؟
\begin{figure}
\centering
\begin{subfigure}{0.5\textwidth}
\centering
\begin{tikzpicture}[american voltages]
\draw(0,0)coordinate(ka) to [european resistor]++(\xx,0)coordinate(kb) to [european resistor]++(120:\xx)coordinate(kc) to [european resistor]++(-120:\xx);
\draw(ka) to [short,*-]++(0,-\y/4) to [short,-o]++(-\xx/2,0);
\draw(kb) to [short,*-]++(0,-\y/2) to [short,-o]++(-1*\xx-\xx/2,0);
\draw(kc) to [short,*-]++(0,\y/4) to [short,-o,i<_={$\hat{I}$}]++(-\xx,0);
\draw(ka) to [open,v^<={$\hat{V}_1$}](kc);
\draw(kc) to [open,v^<={$\hat{V}_2$}](kb);
\end{tikzpicture}
\caption*{(الف)}
\end{subfigure}%
\begin{subfigure}{0.5\textwidth}
\centering
\begin{tikzpicture}[american voltages]
\pgfmathsetmacro{\kl}{\x*sin(60)}
%
\draw(0,0) to [european resistor,*-,v_>={$\hat{V}$}]++(90:\x)coordinate(ka);
\draw(0,0) to [european resistor]++(-30:\x)coordinate(kb);
\draw(0,0) to [european resistor]++(-150:\x)coordinate(kc);
%
\draw(ka)  to [short,-o,i<_={$\hat{I}$}]++(-\kl-\x/2,0);
\draw(kb) to [short]++(0,-\y/4) to [short,-o]++(-2*\kl-\x/2,0);
\draw(kc)  to [short,-o]++(-\x/2,0);
\end{tikzpicture}
\caption*{(ب) دباو اور رو کا زاویہ فرضی ستارہ دباو اور تار کی رو کے مابین ناپا جاتا ہے۔}
\end{subfigure}%
\caption{تین دوری نظام میں جزو طاقت۔}
\label{شکل_تین_دوری_جزو_طاقت_تکونی_بوجھ}
\end{figure} 

حل:تار کی رو کا زاویہ ان دونوں دباو سے نہیں ناپا جاتا بلکہ ستارہ دباو کے ساتھ ناپا جاتا ہے۔شکل-ب میں درست ستارہ شاخ کی نشاندہی کی گئی ہے۔تکونی بوجھ کی صورت میں فرضی ستارہ دباو دریافت کرتے ہوئے صحیح زاویہ ناپا جاتا ہے۔یاد رہے کہ جزو طاقت کا زاویہ حقیقت میں بوجھ کے رکاوٹ کا زاویہ ہے۔
\انتہا{مثال}
%====================
\ابتدا{مثال}
ایک ستارہ تکونی نظام میں بوجھ کا امالی زاویہ \عددی{34^{\circ}} اور دباو تار \عددی{\SI{400}{\volt}\,\rms} ہیں۔بوجھ کا حقیقی طاقت \عددی{\SI{3}{\kilo\watt}} ہے۔ہمیں تار کی رو اور تکونی بوجھ کی رکاوٹ درکار ہے۔

حل:مساوات \حوالہ{مساوات_تین_طاقت_حقیقی-متعاملی_پ} سے رو تار حاصل کرتے ہیں۔
\begin{align*}
I_L=\frac{P_T}{\sqrt{3}V_L \cos \theta}=\frac{3000}{\sqrt{3} 400 \cos 34^{\circ}}=\SI{5.2231}{\ampere}\,\rms
\end{align*}
یوں تکونی بوجھ کی شاخ میں درج ذیل رو پائی جائے گی۔
\begin{align*}
I_{\Delta}=\frac{I_L}{\sqrt{3}}=\frac{5.2231}{\sqrt{3}}=\SI{3.0156}{\ampere}\,\rms
\end{align*}
اس طرح تکونی بوجھ کی ایک شاخ کے رکاوٹ کی حتمی قیمت درج ذیل ہو گی۔
\begin{align*}
\abs{\bZ_{\Delta}}=\frac{V_L}{I_{\Delta}}=\frac{400}{3.0156}=\SI{132.64}{\ohm}
\end{align*}
چونکہ امالی بوجھ کا زاویہ \عددی{34^{\circ}} ہے لہٰذا بوجھ کی رکاوٹ درج ذیل ہو گی۔
\begin{align*}
\bZ_{\Delta}=132.64\phase{34^{\circ}}=110+j74\,\si{\ohm}
\end{align*}
\انتہا{مثال}
%===========================
\ابتدا{مثال}
ستارہ ستارہ نظام میں منبع کا دوری دباو \عددی{\SI{200}{\volt}\,\rms} ہے۔تار کی رکاوٹ \عددی{0.5+j0.8\,\si{\ohm}} اور بوجھ کے ایک شاخ کی رکاوٹ \عددی{10+j4\,\si{\ohm}} ہے۔بوجھ کے ایک شاخ پر حقیقی اور متعاملی طاقت دریافت کریں۔منبع کی کل حقیقی، متعاملی اور مخلوط طاقت دریافت کریں۔

حل:ہم حزب معمول \عددی{\hat{V}_{an}=200\phase{0^{\circ}}\,\rms} لیتے  ہیں۔تار کی رو اور بوجھ کا دوری دباو حاصل کرتے ہیں۔
\begin{align*}
\hat{I}_a&=\frac{200\phase{0^{\circ}}}{0.5+j0.8+10+j4}=17.323\phase{-24.567^{\circ}}\,\rms\\
\hat{V}_{AN}&=200\phase{0^{\circ}}\left(\frac{10+j4}{0.5+j0.8+10+j4}\right)=186.578\phase{-2.766^{\circ}}\,\rms
\end{align*}
یوں بوجھ کے ایک شاخ کی مخلوط طاقت
\begin{align*}
\bS_{\text{بوجھ}}=\hat{V}_{AN}\hat{I}^*_a=(186.578\phase{-2.766^{\circ}})(17.323\phase{-24.567^{\circ}})=3000+j1200 \,\si{\volt\ampere}
\end{align*}
ہے یعنی بوجھ کے ایک شاخ کا حقیقی طاقت \عددی{\SI{3}{\kilo\watt}} اور متعاملی طاقت \عددی{\SI{1.2}{\kilo\var}} ہیں۔

منبع کے ایک شاخ پر مخلوط طاقت حاصل کرتے ہیں۔
\begin{align*}
\bS_{\text{منبع}}=\hat{V}_{an}\hat{I}^*_a=(200\phase{0^{\circ}})(17.323\phase{-24.567^{\circ}})=3151+j1400 \,\si{\volt\ampere}
\end{align*}
اس طرح منبع کا کل حقیقی طاقت \عددی{\SI{9.453}{\kilo\watt}}، کل متعاملی طاقت \عددی{\SI{4.2}{\kilo\var}} اور کل ظاہری طاقت \عددی{\SI{10.344}{\kilo\volt\ampere}} ہے۔
\انتہا{مثال}
%========================
\ابتدا{مثال}\شناخت{مثال_تین_متعدد_بوجھ_الف}
تین دوری \عددی{abc} منبع سے درج ذیل بوجھ کو طاقت فراہم کی جاتی ہے۔
\begin{itemize}
\item
پہلا بوجھ: \عددی{\SI{15}{\kilo\watt}} امالی طاقت جس کا جزو طاقت \عددی{0.83} ہے۔
\item
دوسرا بوجھ: \عددی{\SI{6}{\kilo\watt}} مزاحمت بوجھ۔
\item
تیسرا بوجھ: \عددی{\SI{10}{\kilo\volt\ampere}} برق گیر بوجھ جس کا جزو طاقت \عددی{0.92} ہے۔
\end{itemize}  
بوجھ پر دباو تار \عددی{\SI{425}{\volt}\,\rms} ہے۔تار کی رو دریافت کریں اور تمام بوجھ کا مجموعی جزو طاقت حاصل کریں۔

حل:دی گئی معلومات سے مخلوط طاقت لکھتے ہیں۔
\begin{align*}
\bS_1&=15000+j1080\\
\bS_2&=6000\\
\bS_3&=9200-j3919
\end{align*} 
اس سے کل مخلوط طاقت حاصل کرتے ہیں۔
\begin{align*}
\bS&=30200+j6161\\
&=30822\phase{11.53^{\circ}}\,\si{\volt\ampere}
\end{align*}
یوں کل بوجھ کا امالی جزو طاقت اور رو تار درج ذیل ہوں گے۔
\begin{align*}
\pf&=\cos(11.53^{\circ})=0.98\\
I_L&=\frac{\abs{\bS}}{\sqrt{3}V_L}\\
&=\frac{30822}{425\sqrt{3}}\\
&=\SI{41.87}{\ampere}\,\rms
\end{align*}
\انتہا{مثال}
%=======================
\ابتدا{مثال}
مثال \حوالہ{مثال_تین_متعدد_بوجھ_الف} میں تار کی رکاوٹ \عددی{0.06+j0.08\,\si{\ohm}} لیتے ہوئے منبع  پر دباو تار اور جزو طاقت حاصل کریں۔

حل:تینوں ترسیلی تار کی کل مخلوط طاقت دریافت کرتے ہیں۔
\begin{align*}
\bS_{\text{تار}}&=3I_L^2 \bZ_{\text{تار}}\\
&=3 (41.87^2)(0.06+j0.08)\\
&=315.557+j420.743
\end{align*}
یوں منبع کی مخلوط طاقت حاصل کی جا سکتی ہے۔
\begin{align*}
\bS_{\text{منبع}}&=\bS_{\text{بوجھ}}+\bS_{\text{تار}}\\
&=30200+j6161+315.557+j420.743\\
&=\num{31217}\phase{12.17^{\circ}}
\end{align*}
منبع پر دباو تار حاصل کرتے ہیں۔
\begin{align*}
V_{L\text{منبع}}&=\frac{S_{\text{منبع}}}{\sqrt{3} I_L}\\
&=\frac{\num{31217}}{\sqrt{3}(41.87)}\\
&=\SI{430}{\volt}\,\rms
\end{align*}
منبع کے مخلوط طاقت کے زاویے سے امالی جزو طاقت لکھتے ہیں۔
\begin{align*}
\pf=\cos 12.17^{\circ}=0.977
\end{align*}
\انتہا{مثال}
%=====================
\ابتدا{مثال}\شناخت{مثال_تین_متوازن_نظام_بڑا_بوجھ_مثال_الف}
شکل \حوالہ{شکل_تین_متوازن_نظام_بڑا_بوجھ_مثال_الف} میں متوازن تین دوری نظام دکھایا گیا ہے۔تار میں کل ضیاع کو بوجھ پر \عددی{\SI{11}{\kilo\volt}\,\rms} دباو تار اور \عددی{\SI{133}{\kilo\volt}\,\rms} دباو تار کی صورت میں دریافت کریں۔
\begin{figure}
\centering
\begin{tikzpicture}
\draw(0,2*\y) to [short,-o]++(\x/4,0)node[above]{$a$} to [european resistor,-o,l={$0.12+j0.15\,\si{\ohm}$}]++(\x+\x/2,0)coordinate(kA)node[above]{$A$} to [short]++(\x/4,0);
\draw(0,\y) to [short,-o]++(\x/4,0)node[above]{$b$} to [european resistor,-o,l={$0.12+j0.15\,\si{\ohm}$}]++(\x+\x/2,0)coordinate(kB)node[above]{$B$} to [short]++(\x/4,0);
\draw(0,0) to [short,-o]++(\x/4,0)node[above]{$c$} to [european resistor,-o,l={$0.12+j0.15\,\si{\ohm}$}]++(\x+\x/2,0)node[above]{$C$} to [short]++(\x/4,0)coordinate(kBR);
%boxes
\draw(0,-\y/4) rectangle ++(-1.4,2*\y+\y/2);
\draw(0,-\y/4)++(-0.7,\y+\y/4)node[]{\RL{متوازن منبع}};
\draw(kBR)++(0,-\y/4) rectangle ++(2,2*\y+\y/2);
\draw(kBR)++(0,-\y/4)++(1,\y+\y/4)node[]{$\begin{aligned} &\text{\RL{متوازن امالی بوجھ}} \\& \SI{100}{\mega\volt\ampere} \\ & \pf=0.92\end{aligned}$};
%text
\draw($(kA)!0.5!(kB)$) node{$\begin{aligned} &+ \\ &V_{\text{تار}} \\ &- \end{aligned}$};
\end{tikzpicture}
\caption{مثال \حوالہ{مثال_تین_متوازن_نظام_بڑا_بوجھ_مثال_الف} کا دور۔}
\label{شکل_تین_متوازن_نظام_بڑا_بوجھ_مثال_الف}
\end{figure}

حل:پہلے \عددی{\SI{11}{\kilo\volt}\,\rms} پر رو تار اور تار میں ضیاع دریافت کرتے ہیں۔
\begin{align*}
I_L&=\frac{S}{\sqrt{3}V_L}=\frac{\num{100e6}}{\sqrt{3}(\num{11000})}=\SI{5248}{\ampere}\,\rms\\
P_{\text{تار}}&=3I_L^2 R_{\text{تار}}=3 (5248^2)(0.12)=\SI{9.91}{\mega\watt}
\end{align*}
اب \عددی{\SI{133}{\kilo\volt}\,\rms} پر رو تار اور تار میں ضیاع دریافت کرتے ہیں۔
\begin{align*}
I_L&=\frac{S}{\sqrt{3}V_L}=\frac{\num{100e6}}{\sqrt{3}(\num{133000})}=\SI{434}{\ampere}\,\rms\\
P_{\text{تار}}&=3I_L^2 R_{\text{تار}}=3 (434^2)(0.12)=\SI{68}{\kilo\watt}
\end{align*}
آپ دیکھ سکتے ہیں کہ زیادہ دباو پر طاقت کی ترسیل انتہائی زیادہ سودمند ثابت ہوتی ہے۔یہی وجہ ہے کہ طاقت کی ترسیل زیادہ سے زیادہ ممکنہ دباو پر کی جاتی ہے۔

ہمارے ملک پاکستان میں برقی طاقت کا بیشتر حصہ پانی کے ڈیم سے حاصل کیا جاتا ہے۔یہ ڈیم عموماً شہروں سے دور پائے جاتے ہیں۔ڈیم پر نسب \اصطلاح{دباو بڑھاتا ٹرانسفارمر}\فرہنگ{ٹرانسفارمر!دباو بڑھاتا}\حاشیہب{step up transformer}\فرہنگ{transformer!step up} پیدا کردہ طاقت کے دباو تار کو بڑھا کر \عددی{\SI{133}{\kilo\volt}\,\rms} یا اس سے بھی زیادہ کرتا ہے۔ترسیل کے بعد شہر میں \اصطلاح{دباو گھٹاتا ٹرانسفارمر}\فرہنگ{ٹرانسفارمر!دباو گھٹاتا}\حاشیہب{step down transformer}\فرہنگ{transformer!step down} دباو تار کو گھٹا کر \عددی{\SI{11}{\kilo\volt}\,\rms} کرتا ہے۔شہر کے اندر طاقت کی ترسیل \عددی{\SI{11}{\kilo\volt}\,\rms} کے نسبتاً کم دباو پر ہوتی ہے۔آپ کے گھر کے قریب دباو گھٹاتا ٹرانسفارمر \عددی{\SI{400}{\volt}\,\rms} دباو تار پیدا کرتا ہے جو آپ کو مہیا کا جاتا ہے۔
\انتہا{مثال}
%====================
\ابتدا{مشق}
ستارہ ستارہ نظام میں بوجھ کو کل \عددی{\SI{42}{\kilo\watt}} طاقت \عددی{0.86} امالی جزو طاقت پر فراہم کی جا رہی ہے۔بوجھ پر دباو تار \عددی{\SI{440}{\volt}\,\rms} ہے۔ بوجھ کے ایک شاخ کی رکاوٹ دریافت کریں۔

جواب:\عددی{3.96\phase{30.68^{\circ}}\,\si{\ohm}}
\انتہا{مشق}
%====================
\ابتدا{مشق}
ستارہ ستارہ نظام \عددی{\SI{55}{\kilo\volt\ampere}}، امالی جزو طاقت \عددی{0.64} اور \عددی{\SI{22}{\kilo\volt\ampere}}، امالی جزو طاقت \عددی{0.78} کے  بوجھوں کو طاقت فراہم کرتا ہے۔بوجھ پر دباو تار \عددی{\SI{560}{\volt}\,\rms} ہے۔ تار کی رو دریافت کریں۔

جواب:\عددی{\SI{79}{\ampere}\,\rms}
\انتہا{مشق}
%=====================
\ابتدا{مشق}\شناخت{مشق_تین_متوازن_نظام_بڑا_بوجھ_مثال_ب}
شکل \حوالہ{شکل_تین_متوازن_نظام_بڑا_بوجھ_مثال_ب} میں رو تار اور  طاقت منبع دریافت کریں۔
\begin{figure}
\centering
\begin{tikzpicture}
\draw(0,2*\y) to [short,-o]++(\x/4,0)node[below]{$a$} to [european resistor,-o,l={$0.12+j0.15\,\si{\ohm}$}]++(\x,0)coordinate(kAL) to [short,-o]++(3*\x,0)coordinate(kA)node[right]{$A$};
\draw(0,\y) to [short,-o]++(\x/4,0)node[below]{$b$} to [european resistor,-o,l={$0.12+j0.15\,\si{\ohm}$}]++(\x,0)coordinate(kBL) to [short,-o]++(3*\x,0)coordinate(kB)node[right]{$B$};
\draw(0,0) to [short,-o]++(\x/4,0)node[below]{$c$} to [european resistor,-o,l={$0.12+j0.15\,\si{\ohm}$}]++(\x,0)coordinate(kCL) to [short,-o]++(3*\x,0)node[right]{$C$} coordinate(kBR);
%source box
\draw(0,-\y/4) rectangle ++(-0.5,2*\y+\y/2);
\draw(0,-\y/4)++(-0.25,\y+\y/4)node[rotate=90]{\RL{متوازن منبع}};
%load boxes left
\draw(kAL)++(\x/4+\x/2-\x/3,0) to [short,*-]++(0,-2*\y-\y/2)coordinate(kTL);
\draw(kBL)++(\x/4+\x/2,0) to [short,*-]++(0,-1*\y-\y/2);
\draw(kCL)++(\x/4+\x/2+\x/3,0) to [short,*-]++(0,-\y/2);
\draw(kTL)++(-0.5,0) rectangle ++(2/3*\x+1,-2);
\draw(kTL)++(\x/3,-1)node{$\begin{aligned}\text{\RL{بوجھ-ب}}\\ \SI{100}{\kilo\volt\ampere} \\ 0.95 \text{\RL{آگے جزو طاقت}} \end{aligned}$};
%load boxes right
\draw(kAL)++(2*\x+\x/4-\x/3,0) to [short,*-]++(0,-2*\y-\y/2)coordinate(kTL);
\draw(kBL)++(2*\x+\x/4,0) to [short,*-]++(0,-1*\y-\y/2);
\draw(kCL)++(2*\x+\x/4+\x/3,0) to [short,*-]++(0,-\y/2);
\draw(kTL)++(-0.5,0) rectangle ++(2/3*\x+1,-2);
\draw(kTL)++(\x/3,-1)node{$\begin{aligned}\text{\RL{بوجھ-الف}}\\ \SI{56}{\kilo\watt} \\ 0.82 \text{\RL{پیچھے جزو طاقت}} \end{aligned}$};
%line voltages
\draw($(kA)!0.5!(kB)$)node{$\begin{aligned} &+ \\ &\SI{420}{\volt}\,\rms \\ &- \end{aligned}$};
\end{tikzpicture}
\caption{مشق \حوالہ{مشق_تین_متوازن_نظام_بڑا_بوجھ_مثال_ب} کا دور۔}
\label{شکل_تین_متوازن_نظام_بڑا_بوجھ_مثال_ب}
\end{figure}

جوابات:\عددی{\SI{468.8}{\volt}\,\rms}، \عددی{0.987} پیچھے۔
\انتہا{مشق}
%=================
\ابتدا{مثال}\شناخت{مثال_تین_دوری_طاقت_کی_سمت_کا_تعین}
کسی بھی ملک میں متعدد جنریٹر متوازی جوڑتے ہوئے پورے ملک کو طاقت فراہم کی جاتی ہے۔ان جنریٹروں کی تعداد سیکڑوں یا ہزاروں میں ہو سکتی ہے اور ان کے درمیان فاصلہ سیکڑوں کلومیٹر ہو سکتا ہے۔پاکستان میں تمام ڈیم اور دیگر جنریٹر قومی ترسیلی نظام سے جڑے ہیں۔تمام جنریٹروں کی تعدد ٹھیک \عددی{\SI{50}{\hertz}} ہونا لازم ہے۔اس قومی نظام اور کسی ایک جنریٹر کے مابین زاویائی فرق سے طاقت کے بہاو کی سمت قابو کی جاتی ہے۔

شکل \حوالہ{شکل_تین_دوری_طاقت_کی_سمت_کا_تعین} میں \عددی{\hat{V}_{ab}=11000\phase{0^{\circ}}\,\si{\volt}\,\rms} اور \عددی{\hat{V}_{mn}=11000\phase{6^{\circ}}\,\si{\volt}\,\rms} ہیں۔طاقت کا بہاو کس جانب کو ہے؟
\begin{figure}
\centering
\begin{subfigure}{1\textwidth}
\centering
\begin{tikzpicture}
\draw(0,2*\y)node[above right]{$a$} to [european resistor,l={$0.15+j0.28\,\si{\ohm}$},i_={$\hat{I}_{au}$}]++(2*\x,0)node[above left]{$u$};
\draw(0,\y)node[above right]{$b$} to [european resistor,l={$0.15+j0.28\,\si{\ohm}$},i_={$\hat{I}_{bv}$}]++(2*\x,0)node[above left]{$v$};
\draw(0,0)node[above right]{$c$} to [european resistor,l={$0.15+j0.28\,\si{\ohm}$},i_={$\hat{I}_{cw}$}]++(2*\x,0)node[above left]{$w$}coordinate(BR);
%boxes
\draw(0,0)++(0,-0.5) rectangle ++(-1,2*\y+1);
\draw(BR)++(0,-0.5) rectangle ++(1,2*\y+1);
\draw(0,0)++(-0.5,\y)node{\RL{نظام الف}};
\draw(2*\x,0)++(0.5,\y)node{\RL{نظام ب}};
\end{tikzpicture}
\caption*{(الف)}
\end{subfigure}
\begin{subfigure}{1\textwidth}
\centering
\begin{tikzpicture}
\draw(0,0)node[below]{$n$} to [american voltage source,l={${\frac{11}{\sqrt{3}}\phase{-30^{\circ}}\,\si{\kilo\volt}\,\rms}$}]++(0,\y)node[above]{$a$} to [european resistor,l={$0.15+j0.28\,\si{\ohm}$},i={$\hat{I}_{au}$}]++(2*\x,0);
\draw(0,0) to [short]++(2*\x,0)node[below]{$n'$} to [american voltage source,l_={$\frac{11}{\sqrt{3}}\phase{-24^{\circ}}\,\si{\kilo\volt}\,\rms$}]++(0,\y)node[above]{$u$};
\end{tikzpicture}
\caption*{(ب)}
\end{subfigure}
\caption{مثال \حوالہ{مثال_تین_دوری_طاقت_کی_سمت_کا_تعین} کا دور۔}
\label{شکل_تین_دوری_طاقت_کی_سمت_کا_تعین}
\end{figure}

حل:آپ دیکھ سکتے ہیں کہ دونوں نظام کے دباو کے حیطے برابر ہیں۔شکل-ب میں ستارہ ستارہ کا ایک شاخ دکھایا گیا ہے جس سے  رو لکھتے ہیں۔
\begin{align*}
\hat{I}_{au}&=\frac{\hat{V}_{an}-\hat{V}_{un'}}{0.01+j0.02}\\
&=\frac{\frac{11000}{\sqrt{3}}\phase{-30^{\circ}}-\frac{11000}{\sqrt{3}}\phase{-24^{\circ}}}{0.15+j0.28}\\
&=2093\phase{181.18^{\circ}}\,\si{\ampere}
\end{align*}
یوں نظام-ب کو درج ذیل کل اوسط طاقت فراہم کی جا رہی ہے۔
\begin{align*}
P_{\text{ب}}&=\sqrt{3} V_{uv} I_{au} \cos(\theta_{V_{un'}-\theta_{I_{au}}})\\
&=\sqrt{3} (11000)(2093)\cos(-24^{\circ}-181.18^{\circ})\\
&=\SI{-36.1}{\mega\watt}
\end{align*}
منفی جواب کا مطلب ہے کہ نظام-ب درحقیقت طاقت پیدا کر رہا ہے اور نظام-الف طاقت جذب کر رہا ہے۔نظام-الف کو فراہم طاقت حاصل کرنے کی خاطر رو کی سمت الٹاتے ہیں۔
\begin{align*}
\hat{I}_{ua}=-\hat{I}_{au}=2093\phase{1.18^{\circ}}\,\si{\ampere}
\end{align*}
یوں طاقت درج ذیل ہو گا۔
\begin{align*}
P_{\text{الف}}&=\sqrt{3} V_{ab} I_{ua} \cos(\theta_{V_{an}-\theta_{I_{ua}}})\\
&=\sqrt{3} (11000)(2093)\cos(-30^{\circ}-1.18^{\circ})\\
&=\SI{34.11}{\mega\watt}
\end{align*}
دونوں نظام کے طاقت میں فرق ترسیلی تاروں کے ضیاع کی بدولت ہے۔ 
\انتہا{مثال}
%==============
\حصہ{جزو طاقت کی درستگی}
یک دوری نظام کا جزو طاقت بہتر کرنے پر  حصہ \حوالہ{حصہ_طاقت_جزو_طاقت_پر_قابو} میں غور کیا گیا۔تین دوری نظام کا جزو طاقت بالکل اسی طرح درست کیا جاتا ہے البتہ تین دوری نظام میں تین عدد برق گیر استعمال کئے جائیں گے۔جزو طاقت درست کرنے والے برق گیر کو تکونی یا ستارہ نما بوجھ کے متوازی جوڑا جا سکتا ہے۔

صفحہ \حوالہصفحہ{مساوات_برقرار_طاقت_جزو_طاقت_درستگی_برق_گیر} پر مساوات \حوالہ{مساوات_برقرار_طاقت_جزو_طاقت_درستگی_برق_گیر} جزو طاقت درست کرنے کے لئے درکار برق گیر دیتا ہے جہاں \عددی{\bS_C} کو شکل \حوالہ{شکل_طاقت_درستگی_جزو_طاقت_صنعتی_بہتر}-پ سے حاصل کیا جاتا ہے۔
\begin{align*}
\bS_C=-j \omega C \VrmsS
\end{align*}
درج بالا مساوات میں \عددی{\bVrms} انفرادی برق گیر پر لاگو دباو ہے۔

جزو طاقت کے درستگی کے لئے دستیاب برق گیر کی گنجائش \عددی{\kilo\var} میں بتلائی جاتی ہے۔ساتھ ہی  استعمال کی تعدد اور موثر دباو بھی بتلایا جاتا ہے۔ہمارے ہاں \عددی{\SI{50}{\hertz}} درکار تعدد ہے۔

جزو طاقت بہتر بنانے والے برق گیر کو بوجھ کے قریب نسب کیا جاتا ہے نہ کہ منبع کے قریب۔جزو طاقت بہتر کرنے سے درکار مخلوط طاقت کی قیمت گھٹتی ہے۔یوں تار میں رو کی قیمت گھٹتی ہے لہٰذا تار میں طاقت کا ضیاع بھی کم ہوتا ہے۔اسی طرح جنریٹر سے بوجھ تک ترسیل کے راستے میں آنے والے ٹرانسفارمروں میں بھی رو گھٹنے سے طاقت کا ضیاع کم ہوتا ہے۔جنریٹر میں بھی رو کی قیمت گھٹنے سے  طاقت کا ضیاع کم ہوتا ہے۔
%==============
\ابتدا{مثال}\شناخت{مثال_تین_دوری_جزو_طاقت_الف}
شکل \حوالہ{شکل_تین_دوری_جزو_طاقت_الف} میں متوازن، \عددی{abc} نظام دکھایا گیا ہے جس میں موثر دباو تار \عددی{\SI{11}{\kilo\volt}\,\rms} اور تعدد \عددی{\SI{50}{\hertz}} ہے۔جزو طاقت \عددی{0.97} آگے  کرنے کی خاطر درکار برق گیر \عددی{C} کی گنجائش دریافت کریں۔
\begin{figure}
\centering
\begin{tikzpicture}
\draw(0,\y)node[above right]{$a$} to [short]++(3*\x,0);
\draw(0,\y/2)node[above right]{$b$} to [short]++(3*\x,0);
\draw(0,0)node[above right]{$c$} to [short]++(3*\x,0)coordinate(kBR);
%capacitors
\draw(3*\x-\x/4,0) to [capacitor,*-*]++(0,-\y)node[shift={(0.3,0.5)}]{$C$};
\draw(3*\x-\x/4-3/4*\x,\y/2) to [short,*-]++(0,-\y/2) to [capacitor,-*]++(0,-\y)node[shift={(0.3,0.5)}]{$C$};
\draw(3*\x-\x/4-2*3/4*\x,\y) to [short,*-]++(0,-\y) to [capacitor,-*]++(0,-\y)node[shift={(0.3,0.5)}]{$C$}coordinate(kCB);
%boxes
\draw(0,-0.5) rectangle ++(-1,\y+1);
\draw(kBR)++(0,-0.5) rectangle ++(2.4,\y+1);
\draw(-0.5,-0.5)coordinate(kL);
\draw(kBR)++(1.2,-0.5)coordinate(kR);
%neutral
\draw(kCB)-|(kL)node[below,pos=0.4]{\RL{تعدیلی تار}};
\draw(kCB)-|(kR);
%text
\draw(-0.5,\y/2)node{\RL{متوازن منبع}};
\draw(kBR)++(1.2,\y/2)node{$\begin{aligned} \text{\RL{متوازن تین دوری بوجھ}} \\\SI{50}{\mega \volt\ampere} \\ \pf=0.69 \text{امالی}\end{aligned}$};
\end{tikzpicture}
\caption{مثال \حوالہ{مثال_تین_دوری_جزو_طاقت_الف} کا دور۔}
\label{شکل_تین_دوری_جزو_طاقت_الف}
\end{figure}

حل:یک دوری نظام کی طرح حل کرتے ہوئے پہلے \عددی{\bS_{\text{پرانا}}} اور \عددی{\bS_{\text{نیا}}} حاصل کرتے ہیں۔
\begin{align*}
\bS_{\text{پرانا}}&=50\phase{\cos^{-1} 0.69}\,\si{\mega\volt\ampere}\\
&=50\phase{46.37^{\circ}}\, \si{\mega\volt\ampere}\\
&=34.5+j36.19\,\si{\mega\volt\ampere}
\end{align*}
نیا زاویہ \عددی{-acos 0.97=-14.07^{\circ}} کے برابر ہے لہٰذا
\begin{align*}
\bS_{\text{نیا}}&=34.5-j34.5 \tan(-14.07^{\circ})\,\si{\mega\volt\ampere}\\
&=34.5-j8.66\,\si{\mega\volt\ampere}
\end{align*}
ہو گا۔اس طرح درکار برق گیر کی مخلوط طاقت درج ذیل ہو گی
\begin{align*}
\bS_{\text{نیا}}-\bS_{\text{پرانا}}=-j44.84\,\si{\mega\volt\ampere}
\end{align*}
جو \عددی{-j\omega C' \VrmsS} کے برابر ہو گا جہاں \عددی{C'} کل برق گیر ہے اور \عددی{V_{\textup{rms}}=\tfrac{\SI{11}{\kilo\volt}\,\rms}{\sqrt{3}}} ہے  لہٰذا
\begin{align*}
C'&=\frac{-j44.84\,\si{\mega\volt\ampere}}{-j2\pi 50 \left(\frac{11000}{\sqrt{3}}\right)^2}\\
&=\SI{3.54}{\milli\farad}
\end{align*}
ہو گا۔یوں شکل \حوالہ{شکل_تین_دوری_جزو_طاقت_الف} میں برق گیر کی قیمت درج ذیل ہو گی
\begin{align*}
C=\frac{C'}{3}=\SI{1.18}{\milli\farad}
\end{align*}
لہٰذا تین عدد برق گیر ستارہ جوڑے جائیں گے جہاں ایک کی متعاملی استعداد تقریباً \عددی{\SI{15}{\mega\var}}ہو گی۔
\انتہا{مثال}
%==================
\ابتدا{مشق}\شناخت{مشق_تین_بہتر_جزو_طاقت_تکونی_الف}
مثال \حوالہ{مثال_تین_دوری_جزو_طاقت_الف} میں \عددی{0.97} امالی جزو طاقت حاصل کرنے کی خاطر \عددی{C} کی قیمت دریافت کریں۔

جواب:\عددی{C=\SI{725}{\micro\farad}}
\انتہا{مشق}
%==================
\ابتدا{مشق}
مثال \حوالہ{مثال_تین_دوری_جزو_طاقت_الف} میں برق گیر کو ستارہ کی بجائے تکونی نسب کرتے ہوئے  \عددی{0.97} امالی جزو طاقت حاصل کیا جاتا ہے۔برق گیر \عددی{C} کی گنجائش دریافت کریں۔

جواب:تکونی جڑے برق گیر کا ایک شاخ اب بھی تقریباً \عددی{\SI{15}{\mega\var}} کا ہو گا البتہ \عددی{C=\SI{242}{\micro\farad}} ہو گا۔
\انتہا{مشق}
%=================


\حصہء{سوالات}
%=======================
\ابتدا{سوال}
تین دوری \عددی{abc} نظام میں \عددی{\bV_{an}=220\phase{90^{\circ}}\,\volt\,\rms} ہے۔تینوں دباو تار دریافت کریں۔

جوابات:\عددی{\bV_{ab}=381\phase{120^{\circ}}\,\si{\volt}\,\rms}، \عددی{\bV_{bc}=381\phase{0^{\circ}}\,\si{\volt}\,\rms}، 
\عددی{\bV_{ca}=381\phase{-120^{\circ}}\,\si{\volt}\,\rms}
\انتہا{سوال}
%=======================
\ابتدا{سوال}
تین دوری \عددی{abc} نظام میں \عددی{\bV_{an}=100\phase{30^{\circ}}\,\volt\,\rms} ہے۔تینوں دباو تار دریافت کریں۔

جوابات:\عددی{\bV_{ab}=173\phase{60^{\circ}}\,\si{\volt}\,\rms}، \عددی{\bV_{bc}=173\phase{0^{\circ}}\,\si{\volt}\,\rms}، 
\عددی{\bV_{ca}=173\phase{-60^{\circ}}\,\si{\volt}\,\rms}
\انتہا{سوال}
%=======================
\ابتدا{سوال}
تین دوری \عددی{abc} نظام میں \عددی{\bV_{ab}=200\phase{60^{\circ}}\,\volt\,\rms} ہے۔تینوں دباو دور دریافت کریں۔

جوابات:\عددی{\bV_{an}=115\phase{30^{\circ}}\,\si{\volt}\,\rms}، \عددی{\bV_{bn}=115\phase{-90^{\circ}}\,\si{\volt}\,\rms}، 
\عددی{\bV_{cn}=115\phase{150^{\circ}}\,\si{\volt}\,\rms}
\انتہا{سوال}
%=======================
\ابتدا{سوال}
تین دوری \عددی{abc} نظام میں \عددی{\bV_{an}=240\phase{45^{\circ}}\,\volt\,\rms} ہے۔تینوں دباو تار دریافت کریں۔

جوابات:\عددی{\bV_{ab}=416\phase{75^{\circ}}\,\si{\volt}\,\rms}، \عددی{\bV_{bc}=416\phase{-45^{\circ}}\,\si{\volt}\,\rms}، 
\عددی{\bV_{ca}=416\phase{-165^{\circ}}\,\si{\volt}\,\rms}
\انتہا{سوال}
%=======================
\ابتدا{سوال}\شناخت{سوال_تین_دوری_رکاوٹ_الف}
شکل \حوالہ{شکل_سوال_تین_دوری_رکاوٹ_الف}-الف میں مساوی تکونی رکاوٹ \عددی{\bZ_{ab}}، \عددی{\bZ_{bc}} اور \عددی{\bZ_{ca}} حاصل کریں۔
\begin{figure}
\centering
\begin{subfigure}{0.5\textwidth}
\centering
\begin{tikzpicture}
\draw(0,0) to [european resistor,*-o,l={$1+j2\,\si{\ohm}$}]++(90:\x)node[right]{$A$};
\draw(0,0) to [european resistor,-o,l={$1+j2\,\si{\ohm}$}]++(-30:\x)node[right]{$B$};
\draw(0,0) to [european resistor,-o,l={$1+j2\,\si{\ohm}$}]++(-150:\x)node[left]{$C$};
\end{tikzpicture}
\caption*{(الف)}
\end{subfigure}%
\begin{subfigure}{0.5\textwidth}
\centering
\begin{tikzpicture}
\draw(0,0) to [european resistor,*-o,l={$1+j2\,\si{\ohm}$}]++(90:\x)node[right]{$A$};
\draw(0,0) to [european resistor,-o,l={$2-j2\,\si{\ohm}$}]++(-30:\x)node[right]{$B$};
\draw(0,0) to [european resistor,-o,l={$2+j4\,\si{\ohm}$}]++(-150:\x)node[left]{$C$};
\end{tikzpicture}
\caption*{(ب)}
\end{subfigure}%
\caption{سوال \حوالہ{سوال_تین_دوری_رکاوٹ_الف} اور سوال \حوالہ{سوال_تین_دوری_رکاوٹ_ب} کے ادوار۔}
\label{شکل_سوال_تین_دوری_رکاوٹ_الف}
\end{figure}

جوابات:\عددی{\bZ_{ab}=\bZ_{bc}=\bZ_{ca}=3+j6\,\si{\ohm}}
\انتہا{سوال}
%==========================
\ابتدا{سوال}\شناخت{سوال_تین_دوری_رکاوٹ_ب}
شکل \حوالہ{شکل_سوال_تین_دوری_رکاوٹ_الف}-ب میں مساوی تکونی رکاوٹ \عددی{\bZ_{ab}}، \عددی{\bZ_{bc}} اور \عددی{\bZ_{ca}} حاصل کریں۔

جوابات:\عددی{\bZ_{ab}=4-j1\,\si{\ohm}}، \عددی{\bZ_{bc}=8-j2\,\si{\ohm}}، \عددی{\bZ_{ca}=-0.5+j6.5\,\si{\ohm}}
\انتہا{سوال}
%=============================
\ابتدا{سوال}\شناخت{سوال_تین_دوری_رکاوٹ_پ}
شکل \حوالہ{شکل_سوال_تین_دوری_رکاوٹ_پ}-الف میں مساوی ستارہ رکاوٹ \عددی{\bZ_{a}}، \عددی{\bZ_{b}} اور \عددی{\bZ_{c}} حاصل کریں۔
\begin{figure}
\centering
\begin{subfigure}{0.5\textwidth}
\centering
\begin{tikzpicture}
\pgfmathsetmacro{\len}{\x*sqrt(3)}
\draw(0,0)coordinate(kB) to [european resistor,l={$0.3-j1.2\,\si{\ohm}$}]++(180:\len)coordinate(kC) to [european resistor,l={$0.3-j1.2\,\si{\ohm}$}]++(60:\len) coordinate(kA)to [european resistor,l={$0.3-j1.2\,\si{\ohm}$}]++(-60:\len);
\draw(kA) to [short,*-o]++(90:0.3)node[left]{$A$};
\draw(kB) to [short,*-o]++(-30:0.3)node[right]{$B$};
\draw(kC) to [short,*-o]++(-150:0.3)node[left]{$C$};
\end{tikzpicture}
\caption*{(الف)}
\end{subfigure}%
\begin{subfigure}{0.5\textwidth}
\centering
\begin{tikzpicture}
\pgfmathsetmacro{\len}{\x*sqrt(3)}
\draw(0,0)coordinate(kB) to [european resistor,l={$4+j3\,\si{\ohm}$}]++(180:\len)coordinate(kC) to [european resistor,l={$2-j4\,\si{\ohm}$}]++(60:\len) coordinate(kA)to [european resistor,l={$1+j1\,\si{\ohm}$}]++(-60:\len);
\draw(kA) to [short,*-o]++(90:0.3)node[left]{$A$};
\draw(kB) to [short,*-o]++(-30:0.3)node[right]{$B$};
\draw(kC) to [short,*-o]++(-150:0.3)node[left]{$C$};
\end{tikzpicture}
\caption*{(ب)}
\end{subfigure}%
\caption{سوال \حوالہ{سوال_تین_دوری_رکاوٹ_پ} اور سوال \حوالہ{سوال_تین_دوری_رکاوٹ_ت} کے ادوار۔}
\label{شکل_سوال_تین_دوری_رکاوٹ_پ}
\end{figure}

جوابات:\عددی{\bZ_a=\bZ_b=\bZ_c=0.1-j0.4\,\si{\ohm}}
\انتہا{سوال}
%==========================
\ابتدا{سوال}\شناخت{سوال_تین_دوری_رکاوٹ_ت}
شکل \حوالہ{شکل_سوال_تین_دوری_رکاوٹ_پ}-ب میں مساوی ستارہ رکاوٹ \عددی{\bZ_{a}}، \عددی{\bZ_{b}} اور \عددی{\bZ_{c}} حاصل کریں۔

جوابات:\عددی{\bZ_a=0.86-j0.29\,\si{\ohm}}، \عددی{\bZ_b=0.14-j1\,\si{\ohm}}، \عددی{\bZ_c=2.86-j1.43\,\si{\ohm}}
\انتہا{سوال}
%=============================
\ابتدا{سوال}\شناخت{سوال_تین_دوری_رکاوٹ_ٹ}
شکل \حوالہ{شکل_سوال_تین_دوری_رکاوٹ_ٹ} کا مساوی رکاوٹ \عددی{\bZ} دریافت کریں۔
\begin{figure}
\centering
\begin{tikzpicture}
\pgfmathsetmacro{\len}{\x*sqrt(3)}
\draw(0,0)coordinate(kB) to [european resistor,l={$-j2\,\si{\ohm}$}]++(180:\len)coordinate(kC) to [european resistor,l={$1+j1\,\si{\ohm}$}]++(60:\len) coordinate(kA)to [european resistor,l={$1+j1\,\si{\ohm}$}]++(-60:\len);
\draw(kA) to [capacitor,*-,l_={$-j1\,\si{\ohm}$}]++(0,\y) to [resistor,-o,l_={$\SI{2}{\ohm}$}]++(-\x-\len/2,0)coordinate(kT);
\draw(kB) to [capacitor,*-,l={$-j2\,\si{\ohm}$}]++(0,-\y) to [capacitor,l={$-j4\,\si{\ohm}$}]++(-\len,0)coordinate(kD) to [resistor,*-o,l={$\SI{1}{\ohm}$}]++(-\x,0)coordinate(kB);
\draw(kC) to [capacitor,*-,l_={$-j2\,\si{\ohm}$}]++(0,-\y);
\draw[stealth-]($(kT)!0.5!(kB)$)++(\x/4,0)--++(-\x/4,0)--++(0,-\y/8)node[below]{$\bZ$};
\end{tikzpicture}
\caption{سوال \حوالہ{سوال_تین_دوری_رکاوٹ_ٹ} کا دور۔}
\label{شکل_سوال_تین_دوری_رکاوٹ_ٹ}
\end{figure}

جوابات:\عددی{\bZ=3.58-j2.12\,\si{\ohm}}
\انتہا{سوال}
%=============================
\ابتدا{سوال}\شناخت{سوال_تین_دوری_رکاوٹ_ث}
شکل \حوالہ{شکل_سوال_تین_دوری_رکاوٹ_ث} کا مساوی رکاوٹ \عددی{\bZ} دریافت کریں۔
\begin{figure}
\centering
\begin{tikzpicture}
\pgfmathsetmacro{\len}{\x*sqrt(2)}
\pgfmathsetmacro{\lenA}{\x/sqrt(2)}
\draw(0,0)coordinate(kBot) to [european resistor,l={$1-j1\,\si{\ohm}$}]++(135:\len)coordinate(kC) to [european resistor,l={$1+j1\,\si{\ohm}$}]++(45:\len) coordinate(kA)to [european resistor,l={$1+j1\,\si{\ohm}$}]++(-45:\len)coordinate(kB) to [european resistor,l={$1-j1\,\si{\ohm}$}]++(-135:\len);
\draw(kA) to [short,*-]++(0,\y/8) to [resistor,-o,l_={$\SI{1}{\ohm}$}]++(-\x-\lenA,0)coordinate(kT);
\draw(kBot) to [short,*-]++(0,-\y/8) to [inductor,-o,l={$j1\,\si{\ohm}$}]++(-\x-\lenA,0)coordinate(kD);
\draw(kC) to [resistor,*-*,l_={$\SI{1}{\ohm}$}](kB);
\draw[stealth-]($(kT)!0.5!(kD)$)++(\x/4,0)--++(-\x/4,0)--++(0,-\y/8)node[below]{$\bZ$};
\end{tikzpicture}
\caption{سوال \حوالہ{سوال_تین_دوری_رکاوٹ_ث} کا دور۔}
\label{شکل_سوال_تین_دوری_رکاوٹ_ث}
\end{figure}

جوابات:\عددی{\bZ=2+j1\,\si{\ohm}}
\انتہا{سوال}
%=============================
\ابتدا{سوال}
متوازن ستارہ بوجھ کو ستارہ منبع \عددی{abc} سے طاقت مہیا کیا جاتا ہے۔دباو تار \عددی{\SI{215}{\volt}\,\rms} ہے جبکہ ستارہ بوجھ \عددی{12+j8\,\si{\ohm}} ہے۔\عددی{\phase{\hat{V}_{an}}=0^{\circ}} لیتے ہوئے تینوں تار کی رو دریافت کریں۔

جوابات:\عددی{\hat{I}_a=8.61\phase{-33.7^{\circ}}\,\si{\ampere}\,\rms}، \عددی{\hat{I}_b=8.61\phase{-153.7^{\circ}}\,\si{\ampere}\,\rms}، \عددی{\hat{I}_c=8.61\phase{86.31^{\circ}}\,\si{\ampere}\,\rms}
\انتہا{سوال}
%==========================
\ابتدا{سوال}
متوازن ستارہ بوجھ کو ستارہ منبع \عددی{abc} سے طاقت مہیا کیا جاتا ہے۔تار کی رکاوٹ \عددی{0.5+j0.8\,\si{\ohm}}، منبع پر دباو
 \عددی{\hat{V}_{an}=\tfrac{240}{\sqrt{3}}\phase{30^{\circ}}\,\rms} ہے جبکہ ستارہ بوجھ \عددی{6+j4\,\si{\ohm}} ہے۔تینوں تار کی رو دریافت کریں۔

جوابات:\عددی{\hat{I}_a=17.1\phase{-6.4^{\circ}}\,\si{\ampere}\,\rms}، \عددی{\hat{I}_b=17.1\phase{-126.4^{\circ}}\,\si{\ampere}\,\rms}، \عددی{\hat{I}_c=17.1\phase{113.6^{\circ}}\,\si{\ampere}\,\rms}
\انتہا{سوال}
%==========================
\ابتدا{سوال}
متوازن ستارہ بوجھ کو ستارہ منبع \عددی{abc} سے طاقت مہیا کیا جاتا ہے۔تار کی رکاوٹ \عددی{0.2+j0.6\,\si{\ohm}}، منبع پر دباو  \عددی{\hat{V}_{ab}=460\phase{45^{\circ}}\,\rms} ہے جبکہ تار کی رو \عددی{\hat{I}_a=78\phase{34^{\circ}}\,\si{\ampere}\,\rms} ہے۔ستارہ بوجھ کی رکاوٹ دریافت کریں۔

جوابات:\عددی{\bZ_Y=3.22-j1.11\,\si{\ohm}}
\انتہا{سوال}
%==========================
\ابتدا{سوال}
متوازن تکون بوجھ کو ستارہ منبع \عددی{abc} سے طاقت مہیا کیا جاتا ہے۔منبع پر دباو  \عددی{\hat{V}_{ab}=440\phase{20^{\circ}}\,\rms} ہے جبکہ تکونی بوجھ
 \عددی{\bZ_{\Delta}=15+j12\,\si{\ohm}} ہے۔رو تار دریافت کریں۔

جوابات:\عددی{\hat{I}_a=39.7\phase{-48.7^{\circ}}\,\si{\ampere}\,\rms}، \عددی{\hat{I}_b=39.7\phase{-168.7^{\circ}}\,\si{\ampere}\,\rms}، \عددی{\hat{I}_c=39.7\phase{71.3^{\circ}}\,\si{\ampere}\,\rms}
\انتہا{سوال}
%==========================
\ابتدا{سوال}
متوازن تکون بوجھ کو ستارہ منبع \عددی{abc} سے طاقت مہیا کیا جاتا ہے۔منبع پر دباو  \عددی{\hat{V}_{ab}=380\phase{80^{\circ}}\,\rms} ہے، تکونی بوجھ
 \عددی{\bZ_{\Delta}=6+j9\,\si{\ohm}} اور تار کی رکاوٹ \عددی{0.1+j0.2\,\si{\ohm}} ہے۔ستارہ منبع کی رو \عددی{\hat{I}_{an}} اور بوجھ کی رو \عددی{\hat{I}_{AB}} دریافت کریں۔

جوابات:\عددی{\hat{I}_{an}=57.3\phase{-6.7^{\circ}}\,\si{\ampere}\,\rms}، \عددی{\hat{I}_{AB}=33.1\phase{23.3^{\circ}}\,\si{\ampere}\,\rms}
\انتہا{سوال}
%==========================
\ابتدا{سوال}
متوازن ستارہ بوجھ کو ستارہ منبع \عددی{abc} سے طاقت مہیا کیا جاتا ہے۔بوجھ پر دباو  \عددی{\hat{V}_{AN}=215\phase{17^{\circ}}\,\rms} ہے،  بوجھ
 \عددی{\bZ_{Y}=8+j8\,\si{\ohm}} اور تار کی رکاوٹ \عددی{1+j2\,\si{\ohm}} ہے۔ستارہ منبع کا دباو \عددی{\hat{V}_{an}} دریافت کریں۔

جواب: \عددی{\hat{V}_{an}=256\phase{20^{\circ}}\,\si{\volt}\,\rms}
\انتہا{سوال}
%==========================
\ابتدا{سوال}
متوازن ستارہ بوجھ کو ستارہ منبع \عددی{abc} سے طاقت مہیا کیا جاتا ہے۔بوجھ پر دباو  \عددی{\hat{V}_{AN}=120\phase{33^{\circ}}\,\rms} ہے، بوجھ
 \عددی{\bZ_{Y}=2+j3\,\si{\ohm}} اور تار کی رکاوٹ \عددی{0.8+j1\,\si{\ohm}} ہے۔ستارہ منبع پر دباو \عددی{\hat{V}_{ab}} دریافت کریں۔

جواب: \عددی{\hat{V}_{ab}=281\phase{61.7^{\circ}}\,\si{\volt}\,\rms}
\انتہا{سوال}
%==========================
\ابتدا{سوال}
متوازن تکون بوجھ کو ستارہ منبع \عددی{abc} سے طاقت مہیا کیا جاتا ہے۔منبع دباو  \عددی{\hat{V}_{an}=120\phase{40^{\circ}}\,\rms} ہے، بوجھ
 \عددی{\bZ_{\Delta}=24+j18\,\si{\ohm}} اور تار کی رکاوٹ \عددی{0.5+j0.4\,\si{\ohm}} ہے۔تکونی بوجھ کی رو دریافت کریں۔

جوابات: \عددی{\hat{I}_{AB}=6.5\phase{33^{\circ}}\,\si{\ampere}\,\rms}،  \عددی{\hat{I}_{BC}=6.5\phase{-87^{\circ}}\,\si{\ampere}\,\rms}،  \عددی{\hat{I}_{CA}=6.5\phase{153^{\circ}}\,\si{\ampere}\,\rms}
\انتہا{سوال}
%==========================
\ابتدا{سوال}
متوازن ستارہ بوجھ کو ستارہ منبع \عددی{abc} سے طاقت مہیا کیا جاتا ہے۔منبع دباو  \عددی{\hat{V}_{an}=120\phase{0^{\circ}}\,\rms}، بوجھ پر دباو
 \عددی{\hat{V}_{AN}=111.62\phase{-1.33^{\circ}}\,\si{\volt}\,\rms} اور بوجھ  \عددی{\bZ_{Y}=8+j4\,\si{\ohm}} ہے۔اور تار کی رکاوٹ  دریافت کریں۔

جواب: \عددی{0.499+j0.499\,\si{\ohm}}
\انتہا{سوال}
%==========================
\ابتدا{سوال}
متوازن ستارہ بوجھ کو ستارہ منبع \عددی{abc} سے طاقت مہیا کیا جاتا ہے۔جزو بوجھ \عددی{0.8} امالی  جبکہ دباو  بوجھ \عددی{\hat{V}_{AN}=210\phase{0^{\circ}}\,\rms} ہے۔کل تار کا ضیاع \عددی{\SI{300}{\watt}} ہے۔تار کی رکاوٹ \عددی{0.8+j1.2\,\si{\ohm}} ہے۔بوجھ کی رکاوٹ دریافت کریں۔

جواب:\عددی{15-j11.3\,\si{\ohm}}
\انتہا{سوال}
%==========================
\ابتدا{سوال}
ستارہ بوجھ \عددی{10+j16\,\si{\ohm}} پر دباو \عددی{\hat{V}_{an}=220\phase{0^{\circ}}\,\rms} ہے۔منبع 
دباو\عددی{\hat{V}_{AN}=235\phase{7^{\circ}}\,\rms} ہے۔متوازن بوجھ قصر دور ہونے پر رو تار کی مقدار حاصل کریں۔

جواب:\عددی{\hat{I}_a=86.8\phase{-116^{\circ}}\,\si{\ampere}\,\rms}
\انتہا{سوال}
%==========================
\ابتدا{سوال}
ستارہ بوجھ \عددی{10+j8\,\si{\ohm}} کو \عددی{1.4+0.6j\,\si{\ohm}} رکاوٹ کے تار سے طاقت مہیا کیا جاتا ہے۔بوجھ پر دباو کا زاویہ \عددی{\phase{\hat{V}_{AN}}=45^{\circ}} ہے۔تاروں میں کل طاقت کا ضیاع \عددی{\SI{450}{\watt}} ہے۔ دباو بوجھ اور دباو منبع حاصل کریں۔

جواب:\عددی{\hat{V}_{AN}=132.6\phase{45^{\circ}}\,\si{\volt}\,\rms}، \عددی{\hat{V}_{an}=147.8\phase{43.4^{\circ}}\,\si{\volt}\,\rms}
\انتہا{سوال}
%==========================
\ابتدا{سوال}
ستارہ بوجھ \عددی{10+j8\,\si{\ohm}} کو \عددی{1.4+0.6j\,\si{\ohm}} رکاوٹ کے تار سے طاقت مہیا کیا جاتا ہے۔بوجھ کا کل طاقت \عددی{\SI{15}{\kilo\watt}} ہے۔تاروں میں کل طاقت کا ضیاع دریافت کریں۔

جواب:\عددی{\SI{2.1}{\kilo\watt}}
\انتہا{سوال}
%==========================
\ابتدا{سوال}
سلسلہ وار \عددی{\SI{10}{\ohm}} اور \عددی{\SI{20}{\milli\henry}} تکونی بوجھ کو \عددی{\SI{50}{\hertz}} کے ستارہ \عددی{abc} منبع سے طاقت فراہم کیا جاتا ہے۔تار کی رکاوٹ کو نظر انداز کرتے ہوئے تمام رو دریافت کریں۔منبع دباو \عددی{\hat{V}_{an}=230\phase{0^{\circ}}\,\si{\volt}\,\rms} ہے۔

جوابات:\عددی{\hat{I}_a=58.4\phase{-32.1^{\circ}}\,\si{\ampere}\,\rms}، \عددی{\hat{I}_b=58.4\phase{-152.1^{\circ}}\,\si{\ampere}\,\rms}، \عددی{\hat{I}_c=58.4\phase{87.9^{\circ}}\,\si{\ampere}\,\rms}، \عددی{\hat{I}_{ab}=33.7\phase{-2.1^{\circ}}\,\si{\ampere}\,\rms}، \عددی{\hat{I}_{bc}=33.7\phase{-122.1^{\circ}}\,\si{\ampere}\,\rms}، \عددی{\hat{I}_{ca}=33.7\phase{117.9^{\circ}}\,\si{\ampere}\,\rms}
\انتہا{سوال}
%==========================
\ابتدا{سوال}\شناخت{سوال_تین_دوری_بوجھ_اور_دباو_الف}
ستارہ بوجھ \عددی{4+j2\,\si{\ohm}} کو تکونی \عددی{abc} منبع \عددی{\hat{V}_{ab}=420\phase{10^{\circ}}\,\si{\volt}\,\rms} طاقت فراہم  کرتا ہے۔تار کی رکاوٹ \عددی{0.2+j0.4\,\si{\ohm}} ہے۔بوجھ پر دباو \عددی{\hat{V}_{AB}} حاصل کریں۔ سوال \حوالہ{سوال_تین_دوری_بوجھ_اور_دباو_الف}، سوال \حوالہ{سوال_تین_دوری_بوجھ_اور_دباو_ب} اور سوال \حوالہ{سوال_تین_دوری_بوجھ_اور_دباو_پ} میں صرف بوجھ تبدیل کیا گیا ہے۔بوجھ کی تبدیلی کا بوجھ کے دباو پر اثر آپ دیکھ سکتے ہیں۔

جوابات:\عددی{335\phase{1.7^{\circ}}\,\si{\volt}\,\rms}
\انتہا{سوال}
%==========================
\ابتدا{سوال}\شناخت{سوال_تین_دوری_بوجھ_اور_دباو_ب}
ستارہ بوجھ \عددی{4+j8\,\si{\ohm}} کو تکونی \عددی{abc} منبع \عددی{\hat{V}_{ab}=420\phase{10^{\circ}}\,\si{\volt}\,\rms} طاقت فراہم  کرتا ہے۔تار کی رکاوٹ \عددی{0.2+j0.4\,\si{\ohm}} ہے۔بوجھ پر دباو \عددی{\hat{V}_{AB}} حاصل کریں۔

جوابات:\عددی{365\phase{10^{\circ}}\,\si{\volt}\,\rms}
\انتہا{سوال}
%==========================
\ابتدا{سوال}\شناخت{سوال_تین_دوری_بوجھ_اور_دباو_پ}
ستارہ بوجھ \عددی{4-j8\,\si{\ohm}} کو تکونی \عددی{abc} منبع \عددی{\hat{V}_{ab}=420\phase{10^{\circ}}\,\si{\volt}\,\rms} طاقت فراہم  کرتا ہے۔تار کی رکاوٹ \عددی{0.2+j0.4\,\si{\ohm}} ہے۔بوجھ پر دباو \عددی{\hat{V}_{AB}} حاصل کریں۔

جوابات:\عددی{458\phase{2.5^{\circ}}\,\si{\volt}\,\rms}
\انتہا{سوال}
%==========================
\ابتدا{سوال}
ستارہ بوجھ کو ستارہ \عددی{abc} منبع \عددی{\hat{V}_{an}=235\phase{50^{\circ}}\,\si{\volt}\,\rms} طاقت فراہم  کرتا ہے۔تار کی
 رکاوٹ \عددی{1+j1\,\si{\ohm}} اور \عددی{\hat{I}_a=10\phase{20^{\circ}}\,\si{\ampere}\,\rms} ہے۔بوجھ کی رکاوٹ حاصل کریں۔

جوابات:\عددی{\bZ_Y=19.4+j10.7\,\si{\ohm}}
\انتہا{سوال}
%==========================
\ابتدا{سوال}
ستارہ بوجھ \عددی{10+j8\,\si{\ohm}} کو تکونی \عددی{abc} منبع طاقت فراہم  کرتا ہے۔تار کی رکاوٹ \عددی{0.6+j0.8\,\si{\ohm}} اور \عددی{\hat{I}_a=8\phase{-40^{\circ}}\,\si{\ampere}\,\rms} ہے۔منبع کا دوری دباو حاصل کریں۔

جوابات:\عددی{\hat{V}_{ab}=191\phase{29.7^{\circ}}\,\si{\volt}\,\rms}
\انتہا{سوال}
%==========================
\ابتدا{سوال}
تکونی بوجھ \عددی{9+j12\,\si{\ohm}} کو تکونی \عددی{abc} منبع طاقت فراہم  کرتا ہے۔تار کی رکاوٹ \عددی{0.4+j0.2\,\si{\ohm}} اور بوجھ کی رو 
 \عددی{\hat{I}_{AB}=15\phase{40^{\circ}}\,\si{\ampere}\,\rms} ہے۔منبع کا دوری دباو حاصل کریں۔

جوابات:\عددی{\hat{V}_{ab}=243\phase{91^{\circ}}\,\si{\volt}\,\rms}
\انتہا{سوال}
%==========================
\ابتدا{سوال}
تکونی بوجھ \عددی{24+j12\,\si{\ohm}}  اور ستارہ بوجھ \عددی{12+j8\,\si{\ohm}} متوازی جڑے ہیں۔انہیں تکونی منبع \عددی{\hat{V}_{ab}=440\phase{60^{\circ}}\,\si{\volt}\,\rms} سے طاقت فراہم کیا جاتا ہے۔تار کی رکاوٹ \عددی{1+j0.8\,\si{\ohm}} ہے۔تار کی رو اور تکونی بوجھ کا کل طاقت دریافت کریں۔

جوابات:\عددی{\hat{I}_a=37.4\phase{-1^{\circ}}\,\si{\ampere}\,\rms}، \عددی{p_{\Delta}=\SI{12.816}{\kilo\watt}}
\انتہا{سوال}
%==========================
\ابتدا{سوال}
تین دوری \عددی{\SI{50}{\hertz}} منبع درج ذیل بوجھ کو طاقت فراہم کرتا ہے۔
\begin{itemize}
\item
\عددی{0.8} امالی جزو طاقت کا \عددی{\SI{60}{\kilo\volt\ampere}} بوجھ اور
\item
\عددی{0.7} پیچھے جزو طاقت کا \عددی{\SI{40}{\kilo\volt\ampere}} بوجھ۔
\end{itemize}
بوجھ پر موثر دباو تار \عددی{\SI{440}{\volt}\,\rms} ہے۔رو تار اور بوجھ پر کل جزو طاقت دریافت کریں۔

جوابات:\عددی{I_a=\SI{131}{\ampere}\,\rms}، امالی جزو طاقت \عددی{0.762} ہے۔
\انتہا{سوال}
%===========================
\ابتدا{سوال}
تین دوری \عددی{\SI{50}{\hertz}} منبع درج ذیل بوجھ کو طاقت فراہم کرتا ہے۔
\begin{itemize}
\item
\عددی{0.8} امالی جزو طاقت کا \عددی{\SI{20}{\kilo\volt\ampere}} بوجھ،
\item
\عددی{0.8} آگے جزو طاقت کا \عددی{\SI{10}{\kilo\volt\ampere}} بوجھ اور
\item
\عددی{0.75} آگے جزو طاقت کا \عددی{\SI{12}{\kilo\watt}} بوجھ۔
\end{itemize}
بوجھ پر موثر دباو تار \عددی{\SI{440}{\volt}\,\rms} ہے۔تار کی رکاوٹ کو نظر انداز کرتے ہوئے منبع پر موثر دباو تار اور جزو طاقت دریافت کریں۔

جوابات:\عددی{\SI{440}{\volt}\,\rms}، آگے جزو طاقت \عددی{0.992} ہے۔
\انتہا{سوال}
%===========================
\ابتدا{سوال}
تین دوری \عددی{\SI{50}{\hertz}} منبع درج ذیل بوجھ کو طاقت فراہم کرتا ہے۔
\begin{itemize}
\item
\عددی{0.8} امالی جزو طاقت کا \عددی{\SI{15}{\kilo\volt\ampere}} بوجھ،
\item
\عددی{0.6} آگے جزو طاقت کا \عددی{\SI{10}{\kilo\volt\ampere}} بوجھ،
\item
اکائی جزو طاقت کا \عددی{\SI{10}{\kilo\watt}} بوجھ اور
\item
\عددی{0.7} امالی جزو طاقت کا \عددی{\SI{15}{\kilo\watt}} بوجھ
\end{itemize}
تار کی رکاوٹ \عددی{0.2+j0.6\,\si{\ohm}} ہے جبکہ بوجھ پر موثر دباو تار \عددی{\SI{415}{\volt}\,\rms} ہے۔منبع پر دباو تار اور جزو طاقت دریافت کریں۔

جوابات:\عددی{V_{ab}=\SI{462}{\volt}\,\rms}، امالی جزو طاقت \عددی{0.887} ہے۔
\انتہا{سوال}
%===========================

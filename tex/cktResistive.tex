\باب{مزاحمتی ادوار}

\حصہ{قانون اوہم}
شکل \حوالہ{شکل_مزاحمتی_سیدھے_خطوط}-الف میں \اصطلاح{کارتیسی محدد}\فرہنگ{کارتیسی محدد}\حاشیہب{Cartesian coordinates}\فرہنگ{Cartesian coordinates} پر سیدھے خطوط دکھائے گئے ہیں۔بالائی خط کی مساوات \عددی{y=m_1 x+c_1} ہے جہاں خط کی \اصطلاح{ڈھلوان}\فرہنگ{ڈھلوان}\حاشیہب{slope}\فرہنگ{slope} \عددی{m_1} ہے جبکہ خط  \عددی{y} محدد کو \عددی{c_1} پر کاٹتا ہے۔نچلی خط کی ڈھلوان \عددی{m_2} ہے جبکہ یہ محدد کے مرکز \عددی{(0,0)} سے گزرتی ہے لہٰذا یہ خط \عددی{y} محدد کو \عددی{0} پر کاٹتی ہے اور یوں اس کی مساوات \عددی{y=m_2 x} ہے۔

\begin{figure}
\centering
\begin{subfigure}{0.5\textwidth}
\includegraphics{figResistiveStraightLines}
\caption{سیدھے خطوط اور ان کی ریاضی مساوات۔}
\end{subfigure}%
%
\begin{subfigure}{0.5\textwidth}
\includegraphics{figResistiveOhmLaw}
\caption{مزاحمت کے برقی دباو بالمقابل رو خط اور اوہم کا قانون۔}
\end{subfigure}%
\caption{قانون اوہم دراصل سیدھے خط کی مساوات ہے۔}
\label{شکل_مزاحمتی_سیدھے_خطوط}
\end{figure}


مزاحمت کے دو سروں کے مابین مختلف برقی دباو \عددی{v} لاگو کرتے ہوئے برقی رو \عددی{i} ناپی گئی۔برقی دباو کو عمودی محدد اور برقی رو کو افقی محدد پر رکھتے ہوئے ان کے تعلق کو شکل \حوالہ{شکل_مزاحمتی_سیدھے_خطوط}-ب میں دکھایا گیا ہے۔اس خط کو مزاحمت کی \اصطلاح{دباو بالمقابل رو خط} کہا جاتا ہے۔شکل-ب کا شکل-الف کی نچلی خط کے ساتھ موازنہ کرتے ہوئے اس خط کو 
\begin{align}
v=R i  \quad \quad \text{\RL{قانون اوہم}}
\end{align}
لکھا جا سکتا ہے جہاں خط کی ڈھلوان کو \عددی{R} لکھا اور \اصطلاح{برقی مزاحمت}\فرہنگ{مزاحمت}\حاشیہب{electrical resistance}\فرہنگ{resistance}  یا صرف \اصطلاح{مزاحمت} پکارا جاتا ہے۔اس مساوات کو \اصطلاح{قانون اوہم}\فرہنگ{اوہم!قانون}\فرہنگ{قانون!اوہم}\حاشیہب{Ohm's law}\فرہنگ{Ohm's law} کہتے ہیں۔شکل-ب میں \اصطلاح{مزاحمت} \عددیء{R} کو بطور ڈھلوان دکھایا گیا ہے۔
\begin{align}
R=\frac{v_2-v_1}{i_2-i_1}=\frac{\Delta v}{\Delta i} \quad \quad \text{\RL{مزاحمت کی تعریف}}
\end{align}

شکل \حوالہ{شکل_مزاحمتی_سیدھے_خطوط}-ب میں دباو اور رو راست تناسب کا تعلق رکھتے ہیں۔راست تناسبی تعلق کو \اصطلاح{خطی}\فرہنگ{خطی}\حاشیہب{linear}\فرہنگ{linear} تعلق کہا جاتا ہے۔اگرچہ اس کتاب میں مزاحمت کو \اصطلاح{خطی پرزہ}\فرہنگ{خطی پرزہ}\حاشیہب{linear component}\فرہنگ{linear component} ہی تصور کیا جائے گا، یہ جاننا ضروری ہے کہ کئی نہایت اہم اقسام کے پرزے  غیر خطی مزاحمت کی خاصیت رکھتے ہیں۔عام استعمال میں \عددی{\SI{220}{\volt}} پر جلنے والا بلب غیر خطی مزاحمت کی مثال ہے۔اس بلب کے \عددی{v-i} تعلق کو شکل \حوالہ{شکل_مزاحمتی_غیر_خطی_تعلق} میں دکھایا گیا ہے۔

\begin{figure}
\centering
\includegraphics{figResistiveIncadecentBulb}
\caption{غیر خطی دباو بالمقابل رو کی تعلق۔}
\label{شکل_مزاحمتی_غیر_خطی_تعلق}
\end{figure} 

وقت کے ساتھ بدلتا دباو اور بدلتی رو کی صورت میں قانون اوہم
\begin{align}\label{مساوات_مزاحمت_قانون_اوہم}
v(t) =Ri(t) 
\end{align}
لکھا جائے گا جہاں وقت \عددی{t} کے ساتھ بدلتے برقی دباو اور بدلتی برقی رو کو چھوٹے حروف میں لکھا گیا ہے۔مساوات \حوالہ{مساوات_مزاحمت_قانون_اوہم} سے مزاحمت کا بُعد \عددی{\si{\volt\per\ampere}} حاصل ہوتا ہے جسے \اصطلاح{اوہم}\فرہنگ{اوہم}\حاشیہب{Ohm}\فرہنگ{Ohm} پکارا اور \عددی{\si{\ohm}} سے ظاہر کیا جاتا ہے۔یوں اگر کسی مزاحمت پر \عددی{\SI{10}{\volt}} کا برقی دباو لاگو کرنے سے مزاحمت میں \عددی{\SI{5}{\ampere}} کی رو گزرے تب مزاحمت کی قیمت \عددی{R=\tfrac{10}{5}=\SI{2}{\ohm}} ہو گی۔

\begin{figure}
\centering
\includegraphics{figResistanceOhmLawPower}
\caption{اوہم کا قانون اور مزاحمتی ضیاع۔}
\label{شکل_مزاحمت_اوہم_قانون_مزاحمتی_ضیاع}
\end{figure}

شکل \حوالہ{شکل_مزاحمت_اوہم_قانون_مزاحمتی_ضیاع} میں برقی دور کے ساتھ مزاحمت \عددی{R} جڑی ہے۔مزاحمت کی دباو \عددی{v(t)} اور  رو \عددی{i(t)} ہیں۔  صفحہ \حوالہصفحہ{مساوات_بنیادی_طاقت_مساوی_دباو_ضرب_رو} پر مساوات \حوالہ{مساوات_بنیادی_طاقت_مساوی_دباو_ضرب_رو} کے تحت اس مزاحمت میں طاقت کا ضیاع
\begin{align*}
p(t)=v(t) i(t)
\end{align*}
ہو گا۔ اس مساوات میں برقی دباو \عددی{v(t)} میں قانون اوہم  پُر کرتے ہوئے
\begin{align*}
p(t)=R i(t) \times i(t)=R i^2(t)
\end{align*}
حاصل ہوتا ہے۔ اسی طرح طاقتی ضیاع کی مساوات  میں \عددی{i(t)} کی جگہ قانون اوہم استعمال کرتے ہوئے
\begin{align*}
p(t)=v(t) \times \frac{v(t)}{R}= \frac{v^2(t)}{R}
\end{align*}
حاصل ہوتا ہے۔مندرجہ بالا تین مساوات کو اکٹھے لکھتے ہیں۔
\begin{align}
p(t)=v(t) i(t)=R i^2(t)=\frac{v^2(t)}{R}  \quad \quad \text{\RL{مزاحمتی ضیاع}}
\end{align}
درج بالا مساوات مزاحمت کی طاقت دیتی ہے۔یہ طاقت حرارتی توانائی میں تبدیل ہوتی ہے جس سے مزاحمت کا درجہ حرارت بڑھتا ہے۔

مزاحمت کے علاوہ \اصطلاح{موصلیت}\فرہنگ{موصلیت}\حاشیہب{conductance}\فرہنگ{conductance} \عددی{G} بھی بہت مقبول ہے جہاں
\begin{align}\label{مساوات_مزاحمتی_موصلیت_اور_مزاحمت}
G=\frac{1}{R}
\end{align}
کے برابر ہے۔موصلیت کی اکائی \اصطلاح{سیمنز}\فرہنگ{سیمنز}\حاشیہب{Siemens}\فرہنگ{Siemens} \عددی{\si{\siemens}} ہے جہاں
\begin{align}
\SI{1}{\siemens}=\SI{1}{\ampere\per\volt}
\end{align}
کے برابر ہے۔مساوات \حوالہ{مساوات_مزاحمتی_موصلیت_اور_مزاحمت} کے استعمال سے  اوہم کے قانون کو
\begin{align}\label{مساوات_مزاحمتی_موصلیت_تعریف}
i(t)=G v(t)
\end{align}
اور مزاحمت کی طاقت کو
\begin{align}
p(t)=G v^2(t)=\frac{i^2(t)}{G}
\end{align}
لکھا جا سکتا ہے۔
%====================
\ابتدا{مثال}
ایک عدد مزاحمت پر \عددی{\SI{20}{\volt}} لاگو کرنے سے  مزاحمت میں \عددی{\SI{4}{\ampere}} پیدا ہوتی ہے۔ اس کی موصلیت دریافت کریں۔

حل:مساوات \حوالہ{مساوات_مزاحمتی_موصلیت_تعریف} کی مدد سے
\begin{align*}
G=\frac{i}{v}=\frac{4}{20}=\SI{0.2}{\siemens}
\end{align*}
حاصل ہوتا ہے۔یہی جواب، اوہم کے قانون سے  \عددی{R=\tfrac{20}{4}=\SI{5}{\ohm}} لکھتے اور \عددی{G=\tfrac{1}{R}=\SI{0.2}{\siemens}} سے بھی حاصل ہوتا ہے۔
\انتہا{مثال}
%=================

\begin{figure}
\centering
\begin{subfigure}{0.3\textwidth}
\centering
\includegraphics{figResistanceVariableResistance}
\caption*{(الف) متغیر مزاحمت}
\end{subfigure}%
%
\begin{subfigure}{0.3\textwidth}
\centering
\includegraphics{figResistanceShortCircuit}
\caption*{(ب) قصرِ دور}
\end{subfigure}%
%
\begin{subfigure}{0.3\textwidth}
\centering
\includegraphics{figResistanceOpenCircuit}
\caption*{(پ) کھلا دور }
\end{subfigure}%
\caption{قصر دور اور کھلا دور۔}
\label{شکل_مزاحمت_قصر_اور_کھلا_دور}
\end{figure}

شکل \حوالہ{شکل_مزاحمت_قصر_اور_کھلا_دور}-الف میں برقی دور کے ساتھ \اصطلاح{متغیر مزاحمت}\فرہنگ{متغیر مزاحمت}\فرہنگ{مزاحمت!متغیر}\حاشیہب{variable resistor}\فرہنگ{resistor!variable} جڑا دکھایا گیا ہے۔مزاحمت پر ترچھا تیر کھینچ کر متغیر مزاحمت کو ظاہر کیا جاتا ہے۔اگر متغیر مزاحمت کی قیمت کم کرتے کرتے صفر کر دی جائے تو کسی بھی رو \عددی{i(t)} کی صورت میں مزاحمت پر لاگو برقی دباو، قانون اوہم کے تحت \عددی{v= i(t) \times 0 =\SI{0}{\volt}} ہو گا۔یہ صورت حال شکل-ب میں دکھائی گئی ہے اور اس صورت کو \اصطلاح{قصر دور}\فرہنگ{ْقصر دور}\حاشیہب{short circuit}\فرہنگ{short circuit} کہتے ہیں۔ دو نقطوں کو موصل تار سے جوڑ کر قصر دور کیا جاتا ہے۔اس کے برعکس اگر متغیر مزاحمت کی قیمت لامحدود کر دی جائے تب کسی بھی دباو \عددی{v(t)} پر، قانون اوہم کے تحت \عددی{i=\tfrac{v(t)}{\infty}=\SI{0}{\ampere}} ہو گی۔ایسی صورت، جسے \اصطلاح{کھلا دور}\فرہنگ{کھلا دور}\حاشیہب{open circuit}\فرہنگ{open circuit} کہتے ہیں کو شکل-پ میں دکھائی گئی ہے۔کسی بھی دو نقطوں کو کھلا دور کرنے کا مطلب یہ ہے کہ ان  نقطوں کے مابین مزاحمت لامحدود کر دی جائے۔قصر دور پر ہر صورت صفر دباو پایا جاتا ہے جبکہ کھلا دور پر ہر صورت صفر رو پائی جاتی ہے۔
%===============
\ابتدا{مثال}\شناخت{مثال_مزاحمتی_مثال_طاقت_اکلتوتا_مزاحمت_الف}
شکل \حوالہ{شکل_مزاحمتی_اکلوتا_مزاحمت_کی_طاقت}-الف میں رو اور مزاحمتی طاقت دریافت کریں۔

حل:قانون اوہم سے مزاحمت میں رو
\begin{align*}
i=\frac{12}{3}=\SI{4}{\ampere}
\end{align*}
حاصل ہوتی ہے اور یوں مزاحمتی طاقت درج ذیل ہو گا۔
\begin{align*}
p=v \times i =12 \times 4=\SI{48}{\watt}
\end{align*}
\انتہا{مثال}
%======================

\begin{figure}
\centering
\begin{subfigure}{0.3\textwidth}
\includegraphics{figResistanceExampleA}
\caption*{(الف)}
\end{subfigure}%
%
\begin{subfigure}{0.3\textwidth}
\includegraphics{figResistanceExampleB}
\caption*{(ب)}
\end{subfigure}%
%
\begin{subfigure}{0.3\textwidth}
\includegraphics{figResistanceExampleC}
\caption*{(پ)}
\end{subfigure}
\caption{مزاحمتی ادوار مثال \حوالہ{مثال_مزاحمتی_مثال_طاقت_اکلتوتا_مزاحمت_الف} تا مثال \حوالہ{مثال_مزاحمتی_مثال_طاقت_اکلتوتا_مزاحمت_پ}}
\label{شکل_مزاحمتی_اکلوتا_مزاحمت_کی_طاقت}
\end{figure}

%========================
\ابتدا{مثال}\شناخت{مثال_مزاحمتی_مثال_طاقت_اکلتوتا_مزاحمت_ب}
شکل \حوالہ{شکل_مزاحمتی_اکلوتا_مزاحمت_کی_طاقت}-ب میں رو اور مزاحمتی طاقت دریافت کریں۔

حل:مزاحمت کا بالائی سرا مثبت ہے لہٰذا اس میں رو کی سمت اوپر سے نیچے ہو گی جو دکھلائے گئی سمت کے الٹ ہے۔اس طرح دی گئی سمت میں رو کی قیمت منفی ہو گی یعنی
\begin{align*}
i=-\frac{10}{5}=\SI{-2}{\ampere}
\end{align*}
جبکہ مزاحمت طاقت درج ذیل ہو گا۔
\begin{align*}
p=i^2 R=\SI{20}{\watt}
\end{align*}
\انتہا{مثال}
%========================

\ابتدا{مثال}\شناخت{مثال_مزاحمتی_مثال_طاقت_اکلتوتا_مزاحمت_پ}
شکل \حوالہ{شکل_مزاحمتی_اکلوتا_مزاحمت_کی_طاقت}-پ میں رو اور مزاحمتی دریافت کریں۔

حل:دور میں طاقت کی پیداوار اور ضیاع برابر لیتے ہوئے طاقت کی مساوات \عددی{p=vi} سے منبع کی رو حاصل کرتے ہیں۔
\begin{align*}
i=\frac{p}{v}=\frac{2.5}{5}=\SI{0.5}{\ampere}
\end{align*}
اوہم کے قانون سے مزاحمت کی قیمت درج ذیل حاصل ہوتی ہے۔
\begin{align*}
R=\frac{v}{i}=\frac{5}{0.5}=\SI{10}{\ohm}
\end{align*} 
\انتہا{مثال}
%========================

\begin{figure}
\centering
\begin{subfigure}{0.5\textwidth}
\includegraphics{figResistanceExampleD}
\caption*{(الف)}
\end{subfigure}%
%
\begin{subfigure}{0.5\textwidth}
\includegraphics{figResistanceExampleE}
\caption*{(ب)}
\end{subfigure}%
\caption{مزاحمتی ادوار مثال \حوالہ{مثال_مزاحمتی_دو_منبع_الف} تا مثال \حوالہ{مثال_مزاحمتی_دو_منبع_ب}}
\label{شکل_مزاحمتی_اکلوتا_مزاحمت_کئی_منبع_کی_طاقت}
\end{figure}

%================
\ابتدا{مثال}\شناخت{مثال_مزاحمتی_دو_منبع_الف}
شکل \حوالہ{شکل_مزاحمتی_اکلوتا_مزاحمت_کئی_منبع_کی_طاقت}-الف میں مزاحمت کی رو اور طاقت دریافت کریں۔

حل:قانون اوہم میں مزاحمت کی دباو \عددی{\SI{15}{\volt}-\SI{3}{\volt}=\SI{12}{\volt}} لیتے ہوئے رو حاصل کرتے ہیں۔
\begin{align*}
i=\frac{12}{10}=\SI{1.2}{\ampere}
\end{align*}
اسی طرح مزاحمت کی دباو \عددی{\SI{12}{\volt}} لیتے ہوئے اس کی طاقت درج ذیل حاصل ہوتی ہے۔یہی جواب \عددی{p=i^2 R} سے بھی حاصل ہو گا۔
\begin{align*}
p=v i =12 \times 1.2=\SI{14.4}{\watt}
\end{align*}
\انتہا{مثال}
%========================
\ابتدا{مثال}\شناخت{مثال_مزاحمتی_دو_منبع_ب}
شکل \حوالہ{شکل_مزاحمتی_اکلوتا_مزاحمت_کئی_منبع_کی_طاقت}-ب میں مزاحمت میں رو اور طاقت دریافت کریں۔دائیں منبع کی طاقت بھی دریافت کریں۔

حل:بائیں منبع کی طاقت اور دباو دیے گئے جس سے منبع کی مثبت سر سے خارج ہوتی رو کی قیمت \عددی{\SI{12}{\ampere}} حاصل ہوتی ہے۔مزاحمت کی دباو \عددی{\SI{8}{\volt}} ہے  لہٰذا اس کی مزاحمت
\begin{align*}
R=\frac{8}{12}=\frac{2}{3} \, \si{\ohm}
\end{align*}
ہو گی۔اس طرح مزاحمت کی طاقت
\begin{align*}
p=v i=8 \times 12=\SI{96}{\watt}
\end{align*}
ہو گا۔دائیں منبع کو طاقت فراہم کی جا رہی ہے جس کی قیمت درج ذیل ہے۔
\begin{align*}
p=v i =2 \times 12=\SI{24}{\watt}
\end{align*}
آپ دیکھ سکتے ہیں کہ طاقت کی پیدا وار اور ضیاع برابر ہیں۔
\انتہا{مثال}
%========================
\begin{figure}
\centering
\begin{subfigure}{0.33\textwidth}
\includegraphics{figResistanceResistorsPowerAndCurrentQuizA}
\caption*{(الف)}
\end{subfigure}%
%
\begin{subfigure}{0.33\textwidth}
\includegraphics{figResistanceResistorsPowerAndCurrentQuizB}
\caption*{(ب)}
\end{subfigure}%
\begin{subfigure}{0.33\textwidth}
\includegraphics{figResistanceResistorsPowerAndCurrentQuizC}
\caption*{(پ)}
\end{subfigure}%
\caption{مزاحمتی ادوار مشق \حوالہ{مثال_مزاحمتی_منبع_رو_طاقت_مشق_الف} تا مشق \حوالہ{مثال_مزاحمتی_منبع_رو_طاقت_مشق_پ}}
\label{شکل_مزاحمتی_اکلوتا_مزاحمت_طاقت_رو_مشق}
\end{figure}
%===================
\ابتدا{مشق}\شناخت{مثال_مزاحمتی_منبع_رو_طاقت_مشق_الف}
شکل \حوالہ{شکل_مزاحمتی_اکلوتا_مزاحمت_طاقت_رو_مشق}-الف میں مزاحمت کی رو اور طاقت حاصل کریں۔منبع کی طاقت بھی حاصل کریں۔

جوابات:\عددی{i=\SI{7}{\ampere}}، \عددی{p=\SI{127}{\watt}}، \عددی{p=\SI{-127}{\watt}}
\انتہا{مشق}
%======================
\ابتدا{مشق}
شکل \حوالہ{شکل_مزاحمتی_اکلوتا_مزاحمت_طاقت_رو_مشق}-ب میں مزاحمت کا دباو اور طاقت حاصل کریں۔منبع کی طاقت بھی دریافت کریں۔

جوابات:\عددی{v=\SI{24}{\volt}}، \عددی{p=\SI{48}{\watt}}، \عددی{p=\SI{-48}{\watt}}
\انتہا{مشق}
%======================
\ابتدا{مشق}\شناخت{مثال_مزاحمتی_منبع_رو_طاقت_مشق_پ}
شکل \حوالہ{شکل_مزاحمتی_اکلوتا_مزاحمت_طاقت_رو_مشق}-پ میں مزاحمت کی رو اور دباو حاصل کریں۔منبع کی طاقت دریافت کریں۔

جوابات:\عددی{i=\SI{2}{\ampere}}، \عددی{v=\SI{18}{\volt}}، \عددی{p=\SI{-36}{\watt}}
\انتہا{مشق}
%======================
%=======================================================================

\حصہ{قوانین کرچاف}
اوہم کے قانون سے ایک مزاحمت اور ایک منبع پر مبنی دور آسانی سے حل ہوتا ہے البتہ زیادہ  پرزوں پر مبنی دور حل کرتے ہوئے اس کا استعمال قدر مشکل ہوتا ہے۔زیادہ پرزہ جات کے ادوار \اصطلاح{قوانین کرچاف}\فرہنگ{قانون کرچاف}\حاشیہب{Kirchoff's laws}\فرہنگ{Kirchoff's laws} کی مدد سے نہایت آسانی کے ساتھ حل ہوتے ہیں۔برقی دور میں برقی پرزوں کو موصل تاروں سے آپس میں جوڑا جاتا ہے۔موصل تار کی مزاحمت کو صفر اوہم تصور کیا جاتا ہے لہٰذا ان میں طاقت کا ضیاع صفر ہو گا۔یوں طاقت کی  پیداوار اور ضیاع صرف برقی پرزوں میں ممکن ہے۔

\begin{figure}
\centering
\begin{subfigure}{0.5\textwidth}
\centering
\includegraphics{figResistanceNodesAndLoopsA}
\caption*{(الف)}
\end{subfigure}%
%
\begin{subfigure}{0.5\textwidth}
\centering
\includegraphics{figResistanceNodesAndLoopsB}
\caption*{(ب)}%
\end{subfigure}
\caption{جوڑ اور دائرے۔}
\label{شکل_مزاحمتی_جوڑ_دائرہ}
\end{figure}

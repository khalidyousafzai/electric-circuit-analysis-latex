\باب{مزاحمتی ادوار}

\حصہ{قانون اوہم}
شکل \حوالہ{شکل_مزاحمتی_سیدھے_خطوط}-الف میں \اصطلاح{کارتیسی محدد}\فرہنگ{کارتیسی محدد}\حاشیہب{Cartesian coordinates}\فرہنگ{Cartesian coordinates} پر سیدھے خطوط دکھائے گئے ہیں۔بالائی خط کی مساوات \عددی{y=m_1 x+c_1} ہے جہاں خط کی \اصطلاح{ڈھلوان}\فرہنگ{ڈھلوان}\حاشیہب{slope}\فرہنگ{slope} \عددی{m_1} ہے جبکہ خط  \عددی{y} محدد کو \عددی{c_1} پر کاٹتا ہے۔نچلی خط کی ڈھلوان \عددی{m_2} ہے جبکہ یہ محدد کے مرکز \عددی{(0,0)} سے گزرتی ہے لہٰذا یہ خط \عددی{y} محدد کو \عددی{0} پر کاٹتی ہے اور یوں اس کی مساوات \عددی{y=m_2 x} ہے۔

\begin{figure}
\centering
\begin{subfigure}{0.5\textwidth}
\includegraphics{figResistiveStraightLines}
\caption{سیدھے خطوط اور ان کی ریاضی مساوات۔}
\end{subfigure}%
%
\begin{subfigure}{0.5\textwidth}
\includegraphics{figResistiveOhmLaw}
\caption{مزاحمت کے برقی دباو بالمقابل رو خط اور اوہم کا قانون۔}
\end{subfigure}%
\caption{قانون اوہم دراصل سیدھے خط کی مساوات ہے۔}
\label{شکل_مزاحمتی_سیدھے_خطوط}
\end{figure}


مزاحمت کے دو سروں کے مابین مختلف برقی دباو \عددی{v} لاگو کرتے ہوئے برقی رو \عددی{i} ناپی گئی۔برقی دباو کو عمودی محدد اور برقی رو کو افقی محدد پر رکھتے ہوئے ان کے تعلق کو شکل \حوالہ{شکل_مزاحمتی_سیدھے_خطوط}-ب میں دکھایا گیا ہے۔اس خط کو مزاحمت کی \اصطلاح{دباو بالمقابل رو خط} کہا جاتا ہے۔شکل-ب کا شکل-الف کی نچلی خط کے ساتھ موازنہ کرتے ہوئے اس خط کو 
\begin{align}
v=R i  \quad \quad \text{\RL{قانون اوہم}}
\end{align}
لکھا جا سکتا ہے جہاں خط کی ڈھلوان کو \عددی{R} لکھا اور \اصطلاح{برقی مزاحمت}\فرہنگ{مزاحمت}\حاشیہب{electrical resistance}\فرہنگ{resistance}  یا صرف \اصطلاح{مزاحمت} پکارا جاتا ہے۔اس مساوات کو \اصطلاح{قانون اوہم}\فرہنگ{اوہم!قانون}\فرہنگ{قانون!اوہم}\حاشیہب{Ohm's law}\فرہنگ{Ohm's law} کہتے ہیں۔شکل-ب میں \اصطلاح{مزاحمت} \عددیء{R} کو بطور ڈھلوان دکھایا گیا ہے۔
\begin{align}
R=\frac{v_2-v_1}{i_2-i_1}=\frac{\Delta v}{\Delta i} \quad \quad \text{\RL{مزاحمت کی تعریف}}
\end{align}

شکل \حوالہ{شکل_مزاحمتی_سیدھے_خطوط}-ب میں دباو اور رو راست تناسب کا تعلق رکھتے ہیں۔راست تناسبی تعلق کو \اصطلاح{خطی}\فرہنگ{خطی}\حاشیہب{linear}\فرہنگ{linear} تعلق کہا جاتا ہے۔اگرچہ اس کتاب میں مزاحمت کو \اصطلاح{خطی پرزہ}\فرہنگ{خطی پرزہ}\حاشیہب{linear component}\فرہنگ{linear component} ہی تصور کیا جائے گا، یہ جاننا ضروری ہے کہ کئی نہایت اہم اقسام کے پرزے  غیر خطی مزاحمت کی خاصیت رکھتے ہیں۔عام استعمال میں \عددی{\SI{220}{\volt}} پر جلنے والا بلب غیر خطی مزاحمت کی مثال ہے۔اس بلب کے \عددی{v-i} تعلق کو شکل \حوالہ{شکل_مزاحمتی_غیر_خطی_تعلق} میں دکھایا گیا ہے۔

\begin{figure}
\centering
\includegraphics{figResistiveIncadecentBulb}
\caption{غیر خطی دباو بالمقابل رو کی تعلق۔}
\label{شکل_مزاحمتی_غیر_خطی_تعلق}
\end{figure} 

وقت کے ساتھ بدلتا دباو اور بدلتی رو کی صورت میں قانون اوہم
\begin{align}\label{مساوات_مزاحمت_قانون_اوہم}
v(t) =Ri(t) 
\end{align}
لکھا جائے گا جہاں وقت \عددی{t} کے ساتھ بدلتے برقی دباو اور بدلتی برقی رو کو چھوٹے حروف میں لکھا گیا ہے۔مساوات \حوالہ{مساوات_مزاحمت_قانون_اوہم} سے مزاحمت کا بُعد \عددی{\si{\volt\per\ampere}} حاصل ہوتا ہے جسے \اصطلاح{اوہم}\فرہنگ{اوہم}\حاشیہب{Ohm}\فرہنگ{Ohm} پکارا اور \عددی{\si{\ohm}} سے ظاہر کیا جاتا ہے۔یوں اگر کسی مزاحمت پر \عددی{\SI{10}{\volt}} کا برقی دباو لاگو کرنے سے مزاحمت میں \عددی{\SI{5}{\ampere}} کی رو گزرے تب مزاحمت کی قیمت \عددی{R=\tfrac{10}{5}=\SI{2}{\ohm}} ہو گی۔

\begin{figure}
\centering
\includegraphics{figResistanceOhmLawPower}
\caption{اوہم کا قانون اور مزاحمتی ضیاع۔}
\label{شکل_مزاحمت_اوہم_قانون_مزاحمتی_ضیاع}
\end{figure}

شکل \حوالہ{شکل_مزاحمت_اوہم_قانون_مزاحمتی_ضیاع} میں برقی دور کے ساتھ مزاحمت \عددی{R} جڑی ہے۔مزاحمت کی دباو \عددی{v(t)} اور  رو \عددی{i(t)} ہیں۔  صفحہ \حوالہصفحہ{مساوات_بنیادی_طاقت_مساوی_دباو_ضرب_رو} پر مساوات \حوالہ{مساوات_بنیادی_طاقت_مساوی_دباو_ضرب_رو} کے تحت اس مزاحمت میں طاقت کا ضیاع
\begin{align*}
p(t)=v(t) i(t)
\end{align*}
ہو گا۔ اس مساوات میں برقی دباو \عددی{v(t)} میں قانون اوہم  پُر کرتے ہوئے
\begin{align*}
p(t)=R i(t) \times i(t)=R i^2(t)
\end{align*}
حاصل ہوتا ہے۔ اسی طرح طاقتی ضیاع کی مساوات  میں \عددی{i(t)} کی جگہ قانون اوہم استعمال کرتے ہوئے
\begin{align*}
p(t)=v(t) \times \frac{v(t)}{R}= \frac{v^2(t)}{R}
\end{align*}
حاصل ہوتا ہے۔مندرجہ بالا تین مساوات کو اکٹھے لکھتے ہیں۔
\begin{align}
p(t)=v(t) i(t)=R i^2(t)=\frac{v^2(t)}{R}  \quad \quad \text{\RL{مزاحمتی ضیاع}}
\end{align}
درج بالا مساوات مزاحمت کی طاقت دیتی ہے۔یہ طاقت حرارتی توانائی میں تبدیل ہوتی ہے جس سے مزاحمت کا درجہ حرارت بڑھتا ہے۔

مزاحمت کے علاوہ \اصطلاح{موصلیت}\فرہنگ{موصلیت}\حاشیہب{conductance}\فرہنگ{conductance} \عددی{G} بھی بہت مقبول ہے جہاں
\begin{align}\label{مساوات_مزاحمتی_موصلیت_اور_مزاحمت}
G=\frac{1}{R}
\end{align}
کے برابر ہے۔موصلیت کی اکائی \اصطلاح{سیمنز}\فرہنگ{سیمنز}\حاشیہب{Siemens}\فرہنگ{Siemens} \عددی{\si{\siemens}} ہے جہاں
\begin{align}
\SI{1}{\siemens}=\SI{1}{\ampere\per\volt}
\end{align}
کے برابر ہے۔مساوات \حوالہ{مساوات_مزاحمتی_موصلیت_اور_مزاحمت} کے استعمال سے  اوہم کے قانون کو
\begin{align}\label{مساوات_مزاحمتی_موصلیت_تعریف}
i(t)=G v(t)
\end{align}
اور مزاحمت کی طاقت کو
\begin{align}
p(t)=G v^2(t)=\frac{i^2(t)}{G}
\end{align}
لکھا جا سکتا ہے۔
%====================
\ابتدا{مثال}
ایک عدد مزاحمت پر \عددی{\SI{20}{\volt}} لاگو کرنے سے  مزاحمت میں \عددی{\SI{4}{\ampere}} پیدا ہوتی ہے۔ اس کی موصلیت دریافت کریں۔

حل:مساوات \حوالہ{مساوات_مزاحمتی_موصلیت_تعریف} کی مدد سے
\begin{align*}
G=\frac{i}{v}=\frac{4}{20}=\SI{0.2}{\siemens}
\end{align*}
حاصل ہوتا ہے۔یہی جواب، اوہم کے قانون سے  \عددی{R=\tfrac{20}{4}=\SI{5}{\ohm}} لکھتے اور \عددی{G=\tfrac{1}{R}=\SI{0.2}{\siemens}} سے بھی حاصل ہوتا ہے۔
\انتہا{مثال}
%=================

\begin{figure}
\centering
\begin{subfigure}{0.3\textwidth}
\centering
\includegraphics{figResistanceVariableResistance}
\caption*{(الف) متغیر مزاحمت}
\end{subfigure}%
%
\begin{subfigure}{0.3\textwidth}
\centering
\includegraphics{figResistanceShortCircuit}
\caption*{(ب) قصرِ دور}
\end{subfigure}%
%
\begin{subfigure}{0.3\textwidth}
\centering
\includegraphics{figResistanceOpenCircuit}
\caption*{(پ) کھلا دور }
\end{subfigure}%
\caption{قصر دور اور کھلا دور۔}
\label{شکل_مزاحمت_قصر_اور_کھلا_دور}
\end{figure}

شکل \حوالہ{شکل_مزاحمت_قصر_اور_کھلا_دور}-الف میں برقی دور کے ساتھ \اصطلاح{متغیر مزاحمت}\فرہنگ{متغیر مزاحمت}\فرہنگ{مزاحمت!متغیر}\حاشیہب{variable resistor}\فرہنگ{resistor!variable} جڑا دکھایا گیا ہے۔مزاحمت پر ترچھا تیر کھینچ کر متغیر مزاحمت کو ظاہر کیا جاتا ہے۔اگر متغیر مزاحمت کی قیمت کم کرتے کرتے صفر کر دی جائے تو کسی بھی رو \عددی{i(t)} کی صورت میں مزاحمت پر لاگو برقی دباو، قانون اوہم کے تحت \عددی{v= i(t) \times 0 =\SI{0}{\volt}} ہو گا۔یہ صورت حال شکل-ب میں دکھائی گئی ہے اور اس صورت کو \اصطلاح{قصر دور}\فرہنگ{ْقصر دور}\حاشیہب{short circuit}\فرہنگ{short circuit} کہتے ہیں۔ دو نقطوں کو موصل تار سے جوڑ کر قصر دور کیا جاتا ہے۔اس کے برعکس اگر متغیر مزاحمت کی قیمت لامحدود کر دی جائے تب کسی بھی دباو \عددی{v(t)} پر، قانون اوہم کے تحت \عددی{i=\tfrac{v(t)}{\infty}=\SI{0}{\ampere}} ہو گی۔ایسی صورت، جسے \اصطلاح{کھلا دور}\فرہنگ{کھلا دور}\حاشیہب{open circuit}\فرہنگ{open circuit} کہتے ہیں کو شکل-پ میں دکھائی گئی ہے۔کسی بھی دو نقطوں کو کھلا دور کرنے کا مطلب یہ ہے کہ ان  نقطوں کے مابین مزاحمت لامحدود کر دی جائے۔قصر دور پر ہر صورت صفر دباو پایا جاتا ہے جبکہ کھلا دور پر ہر صورت صفر رو پائی جاتی ہے۔
%===============
\ابتدا{مثال}\شناخت{مثال_مزاحمتی_مثال_طاقت_اکلتوتا_مزاحمت_الف}
شکل \حوالہ{شکل_مزاحمتی_اکلوتا_مزاحمت_کی_طاقت}-الف میں رو اور مزاحمتی طاقت دریافت کریں۔

حل:قانون اوہم سے مزاحمت میں رو
\begin{align*}
i=\frac{12}{3}=\SI{4}{\ampere}
\end{align*}
حاصل ہوتی ہے اور یوں مزاحمتی طاقت درج ذیل ہو گا۔
\begin{align*}
p=v \times i =12 \times 4=\SI{48}{\watt}
\end{align*}
\انتہا{مثال}
%======================

\begin{figure}
\centering
\begin{subfigure}{0.3\textwidth}
\includegraphics{figResistanceExampleA}
\caption*{(الف)}
\end{subfigure}%
%
\begin{subfigure}{0.3\textwidth}
\includegraphics{figResistanceExampleB}
\caption*{(ب)}
\end{subfigure}%
%
\begin{subfigure}{0.3\textwidth}
\includegraphics{figResistanceExampleC}
\caption*{(پ)}
\end{subfigure}
\caption{مزاحمتی ادوار مثال \حوالہ{مثال_مزاحمتی_مثال_طاقت_اکلتوتا_مزاحمت_الف} تا مثال \حوالہ{مثال_مزاحمتی_مثال_طاقت_اکلتوتا_مزاحمت_پ}}
\label{شکل_مزاحمتی_اکلوتا_مزاحمت_کی_طاقت}
\end{figure}

%========================
\ابتدا{مثال}\شناخت{مثال_مزاحمتی_مثال_طاقت_اکلتوتا_مزاحمت_ب}
شکل \حوالہ{شکل_مزاحمتی_اکلوتا_مزاحمت_کی_طاقت}-ب میں رو اور مزاحمتی طاقت دریافت کریں۔

حل:مزاحمت کا بالائی سرا مثبت ہے لہٰذا اس میں رو کی سمت اوپر سے نیچے ہو گی جو دکھلائے گئی سمت کے الٹ ہے۔اس طرح دی گئی سمت میں رو کی قیمت منفی ہو گی یعنی
\begin{align*}
i=-\frac{10}{5}=\SI{-2}{\ampere}
\end{align*}
جبکہ مزاحمت طاقت درج ذیل ہو گا۔
\begin{align*}
p=i^2 R=\SI{20}{\watt}
\end{align*}
\انتہا{مثال}
%========================

\ابتدا{مثال}\شناخت{مثال_مزاحمتی_مثال_طاقت_اکلتوتا_مزاحمت_پ}
شکل \حوالہ{شکل_مزاحمتی_اکلوتا_مزاحمت_کی_طاقت}-پ میں رو اور مزاحمتی دریافت کریں۔

حل:دور میں طاقت کی پیداوار اور ضیاع برابر لیتے ہوئے طاقت کی مساوات \عددی{p=vi} سے منبع کی رو حاصل کرتے ہیں۔
\begin{align*}
i=\frac{p}{v}=\frac{2.5}{5}=\SI{0.5}{\ampere}
\end{align*}
اوہم کے قانون سے مزاحمت کی قیمت درج ذیل حاصل ہوتی ہے۔
\begin{align*}
R=\frac{v}{i}=\frac{5}{0.5}=\SI{10}{\ohm}
\end{align*} 
\انتہا{مثال}
%========================

\begin{figure}
\centering
\begin{subfigure}{0.5\textwidth}
\includegraphics{figResistanceExampleD}
\caption*{(الف)}
\end{subfigure}%
%
\begin{subfigure}{0.5\textwidth}
\includegraphics{figResistanceExampleE}
\caption*{(ب)}
\end{subfigure}%
\caption{مزاحمتی ادوار مثال \حوالہ{مثال_مزاحمتی_دو_منبع_الف} تا مثال \حوالہ{مثال_مزاحمتی_دو_منبع_ب}}
\label{شکل_مزاحمتی_اکلوتا_مزاحمت_کئی_منبع_کی_طاقت}
\end{figure}

%================
\ابتدا{مثال}\شناخت{مثال_مزاحمتی_دو_منبع_الف}
شکل \حوالہ{شکل_مزاحمتی_اکلوتا_مزاحمت_کئی_منبع_کی_طاقت}-الف میں مزاحمت کی رو اور طاقت دریافت کریں۔

حل:قانون اوہم میں مزاحمت کی دباو \عددی{\SI{15}{\volt}-\SI{3}{\volt}=\SI{12}{\volt}} لیتے ہوئے رو حاصل کرتے ہیں۔
\begin{align*}
i=\frac{12}{10}=\SI{1.2}{\ampere}
\end{align*}
اسی طرح مزاحمت کی دباو \عددی{\SI{12}{\volt}} لیتے ہوئے اس کی طاقت درج ذیل حاصل ہوتی ہے۔یہی جواب \عددی{p=i^2 R} سے بھی حاصل ہو گا۔
\begin{align*}
p=v i =12 \times 1.2=\SI{14.4}{\watt}
\end{align*}
\انتہا{مثال}
%========================
\ابتدا{مثال}\شناخت{مثال_مزاحمتی_دو_منبع_ب}
شکل \حوالہ{شکل_مزاحمتی_اکلوتا_مزاحمت_کئی_منبع_کی_طاقت}-ب میں مزاحمت میں رو اور طاقت دریافت کریں۔دائیں منبع کی طاقت بھی دریافت کریں۔

حل:بائیں منبع کی طاقت اور دباو دیے گئے جس سے منبع کی مثبت سر سے خارج ہوتی رو کی قیمت \عددی{\SI{12}{\ampere}} حاصل ہوتی ہے۔مزاحمت کی دباو \عددی{\SI{8}{\volt}} ہے  لہٰذا اس کی مزاحمت
\begin{align*}
R=\frac{8}{12}=\frac{2}{3} \, \si{\ohm}
\end{align*}
ہو گی۔اس طرح مزاحمت کی طاقت
\begin{align*}
p=v i=8 \times 12=\SI{96}{\watt}
\end{align*}
ہو گا۔دائیں منبع کو طاقت فراہم کی جا رہی ہے جس کی قیمت درج ذیل ہے۔
\begin{align*}
p=v i =2 \times 12=\SI{24}{\watt}
\end{align*}
آپ دیکھ سکتے ہیں کہ طاقت کی پیدا وار اور ضیاع برابر ہیں۔
\انتہا{مثال}
\FloatBarrier
%========================

\begin{figure}
\centering
\begin{subfigure}{0.33\textwidth}
\includegraphics{figResistanceResistorsPowerAndCurrentQuizA}
\caption*{(الف)}
\end{subfigure}%
%
\begin{subfigure}{0.33\textwidth}
\includegraphics{figResistanceResistorsPowerAndCurrentQuizB}
\caption*{(ب)}
\end{subfigure}%
\begin{subfigure}{0.33\textwidth}
\includegraphics{figResistanceResistorsPowerAndCurrentQuizC}
\caption*{(پ)}
\end{subfigure}%
\caption{مزاحمتی ادوار مشق \حوالہ{مثال_مزاحمتی_منبع_رو_طاقت_مشق_الف} تا مشق \حوالہ{مثال_مزاحمتی_منبع_رو_طاقت_مشق_پ}}
\label{شکل_مزاحمتی_اکلوتا_مزاحمت_طاقت_رو_مشق}
\end{figure}
%===================
\ابتدا{مشق}\شناخت{مثال_مزاحمتی_منبع_رو_طاقت_مشق_الف}
شکل \حوالہ{شکل_مزاحمتی_اکلوتا_مزاحمت_طاقت_رو_مشق}-الف میں مزاحمت کی رو اور طاقت حاصل کریں۔منبع کی طاقت بھی حاصل کریں۔

جوابات:\عددی{i=\SI{7}{\ampere}}، \عددی{p=\SI{127}{\watt}}، \عددی{p=\SI{-127}{\watt}}
\انتہا{مشق}
%======================
\ابتدا{مشق}
شکل \حوالہ{شکل_مزاحمتی_اکلوتا_مزاحمت_طاقت_رو_مشق}-ب میں مزاحمت کا دباو اور طاقت حاصل کریں۔منبع کی طاقت بھی دریافت کریں۔

جوابات:\عددی{v=\SI{24}{\volt}}، \عددی{p=\SI{48}{\watt}}، \عددی{p=\SI{-48}{\watt}}
\انتہا{مشق}
%======================
\ابتدا{مشق}\شناخت{مثال_مزاحمتی_منبع_رو_طاقت_مشق_پ}
شکل \حوالہ{شکل_مزاحمتی_اکلوتا_مزاحمت_طاقت_رو_مشق}-پ میں مزاحمت کی رو اور دباو حاصل کریں۔منبع کی طاقت دریافت کریں۔

جوابات:\عددی{i=\SI{2}{\ampere}}، \عددی{v=\SI{18}{\volt}}، \عددی{p=\SI{-36}{\watt}}
\انتہا{مشق}
%======================
%=======================================================================
\FloatBarrier
\حصہ{قوانین کرچاف}
اوہم کے قانون سے ایک مزاحمت اور ایک منبع پر مبنی دور آسانی سے حل ہوتا ہے البتہ زیادہ  پرزوں پر مبنی دور حل کرتے ہوئے اس کا استعمال قدر مشکل ہوتا ہے۔زیادہ پرزہ جات کے ادوار \اصطلاح{قوانین کرچاف}\فرہنگ{قانون کرچاف}\حاشیہب{Kirchoff's laws}\فرہنگ{Kirchoff's laws}\حاشیہد{جرمنی کے گستاف روبرٹ کرچاف نے ان قوانین کو $\overset{1845}{\text{؁}}$ پیش کیا۔} کی مدد سے نہایت آسانی کے ساتھ حل ہوتے ہیں۔برقی دور میں برقی پرزوں کو موصل تاروں سے آپس میں جوڑا جاتا ہے۔موصل تار کی مزاحمت کو صفر اوہم تصور کیا جاتا ہے لہٰذا ان میں طاقت کا ضیاع صفر ہو گا۔یوں طاقت کی  پیداوار اور ضیاع صرف برقی پرزوں میں ممکن ہے۔

\begin{figure}
\centering
\begin{subfigure}{0.5\textwidth}
\centering
\includegraphics{figResistanceNodesAndLoopsA}
\caption*{(الف)}
\end{subfigure}%
%
\begin{subfigure}{0.5\textwidth}
\centering
\includegraphics{figResistanceNodesAndLoopsB}
\caption*{(ب)}%
\end{subfigure}
\caption{جوڑ اور دائرے۔}
\label{شکل_مزاحمتی_جوڑ_دائرہ}
\end{figure}

اس سے پہلے کہ ہم کرچاف کے قوانین پر غور کریں، ہم کچھ اصطلاحات مثلاً \اصطلاح{جوڑ}\فرہنگ{جوڑ}\حاشیہب{node}\فرہنگ{node}، \اصطلاح{دائرہ}\فرہنگ{دائرہ}\حاشیہب{loop}\فرہنگ{loop} اور \اصطلاح{شاخ}\فرہنگ{شاخ}\حاشیہب{branch}\فرہنگ{branch} جاننے کی کوشش کرتے ہیں۔شکل \حوالہ{شکل_مزاحمتی_جوڑ_دائرہ}-الف میں مزاحمت \عددی{R_2}، \عددی{R_3} اور منبع \عددی{V_1} نقطہ \عددی{j_0} پر جڑے ہیں۔اس نقطے کو \اصطلاح{جوڑ} \عددی{j_0} کہا جائے گا۔اسی شکل میں جوڑ \عددی{j_1}، \عددی{j_2} اور \عددی{j_3} بھی دکھائے گئے ہیں۔شکل \حوالہ{شکل_مزاحمتی_جوڑ_دائرہ}-ب میں اسی شکل کو قدر مختلف طریقے سے  دکھایا گیا ہے۔یہاں بھی ان \اصطلاح{جوڑوں} کی نشاندہی کی گئی ہے۔کسی بھی دو یا دو سے زیادہ پرزوں کو جوڑنے والے موصل تار کو \اصطلاح{جوڑ} تصور کیا جاتا ہے۔یوں شکل-الف میں جوڑ \عددی{j_0} نقطہ مانند ہے جبکہ شکل-ب میں نچلی پوری تار جوڑ \عددی{j_0} ہے۔جوڑ کو ظاہر کرنے والی تار کی لمبائی کچھ بھی ہو سکتی ہے۔

کسی بھی دور میں متعدد راستے ممکن ہیں۔ شکل \حوالہ{شکل_مزاحمتی_جوڑ_دائرہ} میں جوڑ \عددی{j_1} سے  مزاحمت \عددی{R_4} کے راستے جوڑ \عددی{j_3} تک پہنچا جا سکتا ہے  جہاں سے  منبع \عددی{i_1(t)} کے راستے جوڑ \عددی{j_1} اور پھر مزاحمت \عددی{R_1} کے راستے واپس جوڑ \عددی{j_1} تک پہنچا جا سکتا ہے۔ایسا بند راستہ جو ابتدائی جوڑ پر ہی اختتام پذیر ہو \اصطلاح{بند راستہ} کہلاتا ہے۔ایسا بند راستہ جس پر کسی بھی جوڑ سے صرف ایک مرتبہ گزرا جائے \اصطلاح{دائرہ}\فرہنگ{دائرہ}\حاشیہب{loop}\فرہنگ{loop} کہلاتا ہے۔ اس طرح \عددی{R_1}، \عددی{i_1(t)} اور \عددی{R_4} \اصطلاح{دائرہ} ہے۔اسی طرح \عددی{R_1}، \عددی{R_2}، \عددی{R_3} اور \عددی{R_4} بھی دائرہ ہے۔دائرے کی ایک اور مثال \عددی{v_1(t)}، \عددی{R_4}، \عددی{i_1(t)} اور \عددی{R_2} ہے۔اس کے برعکس \عددی{R_4}،\عددی{i_1(t)}،\عددی{R_2}،\عددی{R_3}،\عددی{i_1(t)} اور \عددی{R_1} دائرہ نہیں ہے چونکہ اس میں جوڑ \عددی{j_2} اور جوڑ \عددی{j_3} سے دو مرتبہ گزرا گیا۔ 

برقی دور میں ہر برقی پرزے کو \اصطلاح{شاخ}\فرہنگ{شاخ}\حاشیہب{branch}\فرہنگ{branch} کہتے ہیں۔ شکل \حوالہ{شکل_مزاحمتی_جوڑ_دائرہ}  میں کل چھ \عددی{(6)} \اصطلاح{شاخ} ہیں۔جوڑ \عددی{j_3} پر تین شاخ یعنی \عددی{R_4}، \عددی{R_3} اور \عددی{i_1(t)} جڑتے ہیں۔جوڑ \عددی{j_0} پر تین شاخ \عددی{v_1(t)}، \عددی{R_2} اور \عددی{R_3} جڑتے ہیں۔آئیں اب \اصطلاح{قوانین کرچاف} کی بات کریں۔ 
%===============
\ابتدا{قانون}
کرچاف کا قانون برائے برقی رو کہتا ہے کہ کسی بھی جوڑ پر داخلی برقی رو کا مجموعہ خارجی برقی رو کے مجموعے کے عین برابر ہوتا ہے۔ 
\انتہا{قانون}
%====================

کرچاف کے قانون برائے برقی رو کو  \اصطلاح{کرچاف قانونِ رو} کہا جائے گا۔اس قانون کو کسی بھی جوڑ کے لئے یوں
\begin{align}\label{مساوات_مزاحمتی_کرچاف_قانون_رو_الف}
\sum i_{\text{داخلی}}=\sum i_{\text{خارجی}}  \quad \quad \quad \text{\RL{کرچاف قانونِ رو}}
\end{align}
لکھا جاتا ہے۔شکل  \حوالہ{شکل_مزاحمتی_جوڑ_دائرہ}-ب میں جوڑ \عددی{j_0} پر درج بالا مساوات سے
\begin{align}\label{مساوات_مزاحمتی_قانون_رو_کرچاف_الف}
i_3(t)+i_5(t)=i_6(t) \quad \quad \text{\RL{جوڑ $j_0$}} 
\end{align}
حاصل ہوتا ہے۔اسی طرح بقایا جوڑوں پر کرچاف قانونِ رو  سے درج ذیل حاصل ہوتے ہیں جہاں مساوی علامت \عددی{(=)} کے بائیں جانب داخلی رو کا مجموعہ اور دائیں جانب خارجی رو کا مجموعہ ہے۔
\begin{align}
i_6(t)=i_2(t)+i_4(t) \quad \quad \text{\RL{جوڑ $j_1$}} \\
i_1(t)+i_4(t)=i_5(t) \quad \quad \text{\RL{جوڑ $j_2$}}  \\
i_2(t)=i_1(t)+i_3(t) \quad \quad \text{\RL{جوڑ $j_3$}}\label{مساوات_مزاحمتی_قانون_رو_کرچاف_ب}
\end{align}

اگر جوڑ پر تمام رو کی سمت خارجی تصور کی جائے تب \اصطلاح{قانون کرچاف برائے رو}\فرہنگ{کرچاف!قانون برائے رو}\حاشیہب{Kirchoff's Current Law, KCL}\فرہنگ{Kirchoff! current law}\فرہنگ{KCL} کو درج ذیل لکھا جا سکتا ہے جہاں \عددی{i_s(t)} شاخ \عددی{s} میں جوڑ سے خارج رو ہے اور جوڑ کے ساتھ جڑے شاخوں کی تعداد \عددی{N}  ہے۔
\begin{align}\label{مساوات_مزاحمتی_کرچاف_قانون_رو_ب}
\sum_{s=1}^{N} i_s(t)=0 \quad \quad \quad \text{\RL{کرچاف قانونِ رو}}
\end{align}
اگر جوڑ پر تمام رو کی سمت داخلی تصور کی جائے تب \اصطلاح{قانون کرچاف برائے رو}  کو درج بالا لکھا جا سکتا ہے جہاں \عددی{i_s(t)} شاخ \عددی{s} میں جوڑ پر داخل رو ہے۔

مساوات \حوالہ{مساوات_مزاحمتی_کرچاف_قانون_رو_ب} کو استعمال کرتے ہوئے  شکل  \حوالہ{شکل_مزاحمتی_جوڑ_دائرہ}-ب کے لئے درج ذیل لکھا جائے گا جہاں خارجی رو مثبت اور داخلی رو منفی لکھے گئے ہیں۔
\begin{align}
i_6(t)-i_3(t)-i_5(t)&=0 \quad \quad \text{\RL{جوڑ $j_0$}}  \label{مساوات_مزاحمتی_قانون_رو_کرچاف_پ} \\
i_2(t)+i_4(t)-i_6(t)&=0 \label{مساوات_مزاحمتی_غیر_تابع_رو_الف} \\
i_5(t)-i_1(t)-i_4(t)&=0 \label{مساوات_مزاحمتی_غیر_تابع_رو_ب} \\
i_1(t)+i_3(t)-i_2(t)&=0 \label{مساوات_مزاحمتی_غیر_تابع_رو_پ} 
\end{align}

مساوات \حوالہ{مساوات_مزاحمتی_قانون_رو_کرچاف_الف} تا مساوات \حوالہ{مساوات_مزاحمتی_قانون_رو_کرچاف_ب} کو مساوات \حوالہ{مساوات_مزاحمتی_کرچاف_قانون_رو_الف} سے حاصل کیا گیا جبکہ مساوات \حوالہ{مساوات_مزاحمتی_قانون_رو_کرچاف_پ} تا مساوات \حوالہ{مساوات_مزاحمتی_غیر_تابع_رو_پ} کو مساوات \حوالہ{مساوات_مزاحمتی_کرچاف_قانون_رو_ب} سے حاصل کیا گیا۔مساوات \حوالہ{مساوات_مزاحمتی_قانون_رو_کرچاف_الف} میں داخلی رو یعنی \عددی{i_3(t)} اور \عددی{i_5(t)} کو مساوی نشان \عددی{(=)} کی دوسری جانب منتقل کرنے سے مساوات \حوالہ{مساوات_مزاحمتی_قانون_رو_کرچاف_پ} حاصل ہوتا ہے۔آپ دیکھ سکتے ہیں کہ مساوات \حوالہ{مساوات_مزاحمتی_کرچاف_قانون_رو_الف} اور مساوات \حوالہ{مساوات_مزاحمتی_کرچاف_قانون_رو_ب} عین برابر ہیں۔

مساوات \حوالہ{مساوات_مزاحمتی_غیر_تابع_رو_الف}، مساوات \حوالہ{مساوات_مزاحمتی_غیر_تابع_رو_ب} اور مساوات \حوالہ{مساوات_مزاحمتی_غیر_تابع_رو_پ} کو جمع کرنے کے بعد منفی ایک \عددی{(-1)} سے ضرب دینے سے مساوات \حوالہ{مساوات_مزاحمتی_قانون_رو_کرچاف_پ} حاصل ہوتا ہے۔یوں مندرجہ بالا چار \اصطلاح{ہمزاد مساوات}\فرہنگ{ہمزاد مساوات}\حاشیہب{simultaneous equations}\فرہنگ{simultaneous equations} میں صرف تین عدد مساوات \اصطلاح{غیر تابع}\حاشیہب{independent equations} مساوات ہیں۔ان میں کسی بھی تین مساوات کے استعمال سے چوتھی مساوات حاصل کی جا سکتی ہے۔ آپ جانتے ہیں کہ دو آزاد متغیرات حاصل کرنے کی خاطر دو عدد غیر تابع مساوات درکار ہوتے ہیں۔یوں آزاد متغیرات \عددی{x} اور \عددیء{y} مندرجہ ذیل ہمزاد مساوات  میں سے کسی بھی دو مساوات کو بیک وقت حل کرنے سے حاصل کرنا ممکن ہے۔ان میں کسی بھی دو عدد مساوات کو غیر تابع تصور کرتے ہوئے تیسری مساوات حاصل کی جا سکتی ہے لہٰذا تیسری تابع مساوات ہے جو کوئی نئی معلومات فراہم نہیں کرتی۔تابع مساوات غیر ضروری مساوات ہوتی ہے جسے لکھنے کی ضرورت نہیں ہے۔
\begin{align*}
x+y&=3\\
x-y&=1\\
x-3y&=-1
\end{align*}
جس برقی دور میں کُل \عددی{J} عدد جوڑ پائے جاتے ہوں، اس میں \عددی{J-1} غیر تابع مساوات حاصل ہوتے ہیں لہٰذا کسی بھی ایک جوڑ کے بغیر بقایا تمام پر جوڑ پر مساوات کئے جاتے ہیں۔


کرچاف قانونِ رو کے استعمال میں اصل رو کی سمت کو نہیں دیکھا جاتا بلکہ صرف متغیرات \عددی{i_1(t)}، \عددی{i_2(2)}، \عددی{i_3(t)}، \عددی{\cdots} کی سمت کو دیکھتے ہوئے مساوات لکھی جاتی ہے۔یوں شکل \حوالہ{شکل_مزاحمتی_جوڑ_دائرہ}-ب میں جوڑ \عددی{j_2} پر \عددی{i_1(t)} کو داخلی تصور کیا جائے گا اگرچہ \عددی{i_1(t)=\SI{-3}{\ampere}} کی صورت میں رو حقیقت میں دکھائی گئی سمت کے الٹ ہو گی۔

\begin{figure}
\centering
\begin{subfigure}{0.4\textwidth}
\centering
\includegraphics{figResistanceKCLgoats}
\caption*{(الف)}
\end{subfigure}%
%
\begin{subfigure}{0.4\textwidth}
\centering
\includegraphics{figResistanceKCLgoatsSurface}
\caption*{(ب)}
\end{subfigure}%
\caption{کرچاف قانونِ رو کو بکریوں پر بھی لاگو کیا جا سکتا ہے۔}
\label{شکل_مزاحمتی_بکریاں_اور_کرچاف_قانون_رو}
\end{figure}

کرچاف قانونِ رو عمومی مساوات ہے جسے ہم روزمرہ  زندگی میں برقی رو کی بجائے مختلف چیزوں پر لاگو کرتے ہیں۔شکل \حوالہ{شکل_مزاحمتی_بکریاں_اور_کرچاف_قانون_رو}-الف میں ایک گڈریا پورے دن  بکریاں چرانے کے بعد انہیں شام کو  پہاڑی سے نیچے ایک پگڈنڈی پر اتار رہا ہے۔گڈریا اپنی بکریوں کو خیر خیریت سے دکھائی گئے راستے سے نیچے اتار پاتا ہے۔نقطہ \عددی{j} سے نیچے دو پگڈنڈیاں ہیں۔اگر بالائی پگڈنڈی پر \عددی{b_1} بکریاں اترتے گننی جائیں تو آپ یقین کر سکتے ہیں کہ نچلی دو پگڈنڈیوں پر کل اتنی ہی بکریاں اترے گی یعنی \عددی{b_1=b_2+b_3} ہو گا۔ تار میں کسی بھی مقام سے فی سیکنڈ گزرتی برقی بار کو برقی رو کہتے ہیں۔یوں برقی رو کی بات کرتے ہوئے ہم حقیقت میں برقی بار کی بات کرتے ہیں۔تار میں برقی بار کا وجود الیکٹران پر ہے جس کی تعداد نا تو کم ہوتی ہے اور نا ہی بڑھتی ہے۔اسی لئے بالکل پگڈنڈی پر چلتی بکریوں کی طرح تار میں چلتے الیکٹران کی تعداد بھی برقرار رہتی ہے اور کسی جوڑ پر آمدی الیکٹران کی تعداد اس جوڑ سے خارج ہوتے الیکٹران کے برابر ہو گی۔طبیعیات کے اصولوں کے تحت کسی بھی جوڑ پر برقی بار کا انبار نہیں جمع ہوتا۔ \حاشیہد{میں امید کرتا ہوں کہ میری شاگردہ فرحانہ مشتاق کی طرح آپ کو بھی گڈریا کی مثال سے کرچاف  قانونِ رو کی سمجھ آ گئی ہو گی۔}

کرچاف قانونِ رو  کسی بھی بند سطح کے لئے درست ہے۔شکل \حوالہ{شکل_مزاحمتی_بکریاں_اور_کرچاف_قانون_رو}-ب میں ہلکی سیاہی میں بند سطح میں داخل بکریوں کی تعداد سطح سے خارج بکریوں کے برابر ہو گی۔اس شکل میں بند سطح کو جوڑ \عددی{j} تصور کیا جا سکتا ہے۔
%===================

\FloatBarrier
\ابتدا{مثال}
شکل \حوالہ{شکل_مزاحمتی_کرچاف_قانون_رو_مثال}-الف میں نا معلوم رو دریافت کریں۔ 

\begin{figure}
\centering
\begin{subfigure}{0.5\textwidth}
\centering
\includegraphics{figResistanceKCLexampleBranchesOnly}
\caption*{(الف)}
\end{subfigure}%
%
\begin{subfigure}{0.5\textwidth}
\centering
\includegraphics{figResistanceKCLexampleBranchesOnlySolution}
\caption*{(ب)}
\end{subfigure}%
\caption{کرچاف قانون رو کی مثال۔}
\label{شکل_مزاحمتی_کرچاف_قانون_رو_مثال}
\end{figure}

حل:جوڑ \عددی{j_2} پر داخلی رو \عددی{\SI{2}{\milli\ampere}+\SI{5}{\milli\ampere}} ہے جو خارجی رو \عددی{i_4} کے برابر ہو گی یعنی
\begin{align*}
i_4=\SI{5}{\milli\ampere}+\SI{2}{\milli\ampere}=\SI{7}{\milli\ampere}
\end{align*}
جوڑ \عددی{j_3} پر داخلی رو کا مجموعہ \عددی{i_4+i_3} ہے جو خارجی \عددی{\SI{6}{\milli\ampere}} کے برابر ہو گا۔یوں درج بالا حاصل کردہ \عددی{i_4} کی قیمت پُر کرتے ہوئے
\begin{align*}
\SI{7}{\milli\ampere}+i_3=\SI{6}{\milli\ampere}
\end{align*}
سے درج ذیل حاصل ہوتا ہے
\begin{align*}
i_3=\SI{-1}{\milli\ampere}
\end{align*}
جو منفی قیمت ہے۔منفی \عددی{i_3} کا مطلب ہے کہ حقیقت میں رو دکھائی گئی سمت کے الٹ ہے۔شکل \حوالہ{شکل_مزاحمتی_کرچاف_قانون_رو_مثال}-ب میں حقیقی سمت دکھائی گئی ہے۔یوں حقیقت میں جوڑ \عددی{j_3} سے جوڑ \عددی{j_4} کی جانب \عددی{\SI{1}{\milli\ampere}} رو پائی جاتی ہے۔جوڑ \عددی{j_0} پر داخلی رو \عددی{\SI{6}{\milli\ampere}} ہے جبکہ خارجی رو کا مجموعہ \عددی{i_2+\SI{9}{\milli\ampere}} ہے لہٰذا
\begin{align*}
\SI{9}{\milli\ampere}+i_2=\SI{6}{\milli\ampere}
\end{align*}
ہو گا جس سے
\begin{align*}
i_2=\SI{-3}{\milli\ampere}
\end{align*}
حاصل ہوتا ہے۔یوں حقیقت میں جوڑ \عددی{j_4} سے جوڑ \عددی{j_0} کی جانب \عددی{\SI{3}{\milli\ampere}} رو پائی جائے گی۔جوڑ \عددی{j_1} پر داخلی رو  \عددی{\SI{9}{\milli\ampere}} ہے جبکہ خارجی رو کا مجموعہ \عددی{i_1+\SI{5}{\milli\ampere}} ہے۔یوں
\begin{align*}
\SI{9}{\milli\ampere}=i_1+\SI{5}{\milli\ampere}
\end{align*}
لکھا کر
\begin{align*}
i_1=\SI{4}{\milli\ampere}
\end{align*}
حاصل ہوتا ہے۔شکل-الف میں جوڑ \عددی{j_4} پر 
\begin{align*}
i_1+i_2=i_3+\SI{2}{\milli\ampere}
\end{align*}
لکھا جا سکتا ہے۔ہم \عددی{i_3=\SI{-1}{\milli\ampere}} اور \عددی{i_2=\SI{-3}{\milli\ampere}} پہلے حاصل کر چکے ہیں۔یہ قیمتیں پُر کرتے ہوئے
\begin{align*}
i_1&=i_3+\SI{2}{\milli\ampere}-i_2 \\
&=\SI{-1}{\milli\ampere}+\SI{2}{\milli\ampere}-(\SI{-3}{\milli\ampere}) \\
&=\SI{4}{\milli\ampere}
\end{align*}
ہی حاصل ہوتا ہے۔آپ دیکھ سکتے ہیں کہ کرچاف قانونِ رو لکھتے ہوئے \عددی{i_1}، \عددی{i_2} ، \عددی{i_3}، \عددی{\نقطے} کے دکھائے گئے سمتوں سے ہی انہیں داخلی یا خارجی رو گنا جاتا ہے۔
\انتہا{مثال}
\FloatBarrier
%=============================
\ابتدا{مثال}
شکل \حوالہ{شکل_مزاحمتی_کرچاف_قانون_رو_دوسری_مثال} میں تمام جوڑ پر کرچاف قانونِ رو کی مساوات لکھیں۔

\begin{figure}
\centering
\includegraphics{figResistanceKCLwithControlledSources}
\caption{کرچاف قانون رو کی دوسری مثال۔}
\label{شکل_مزاحمتی_کرچاف_قانون_رو_دوسری_مثال}
\end{figure}

حل:جوڑ \عددی{j_0} تا جوڑ \عددی{j_4} بالترتیب مساوات لکھتے ہیں۔خارجی رو کو مثبت تصور کیا گیا ہے۔
\begin{align*}
i_1+i_2-i_3&=0\\
i_4+i_5-i_1&=0\\
i_6-i_2-i_4&=0\\
i_3-i_5-i_6&=0\\
i_5&=i_5
\end{align*}
\انتہا{مثال}
\FloatBarrier
%===============================

\ابتدا{مشق}
شکل \حوالہ{مشق_مزاحمتی_اکلوتی_مزاحمت} میں \عددی{I} دریافت کریں۔
\begin{figure}
\centering
\begin{subfigure}{0.5\textwidth}
\centering
\includegraphics{figResistanceKCLexampleA}
\caption*{(الف)}
\end{subfigure}%
%
\begin{subfigure}{0.5\textwidth}
\centering
\includegraphics{figResistanceKCLexampleB}
\caption*{(ب)}
\end{subfigure}%
\caption{کرچاف قانونِ رو کا پہلا مشق۔}
\label{مشق_مزاحمتی_اکلوتی_مزاحمت}
\end{figure}

جواب: (الف): \عددی{I=\SI{-5}{\milli\ampere}}، (ب):\عددی{I=\SI{8}{\milli\ampere}}
\انتہا{مشق}
\FloatBarrier
%===============================


\ابتدا{مشق}\شناخت{مشق_مزاحمتی_دوسری_مشق}
شکل \حوالہ{مشق_مزاحمتی_کرچاف_قانون_رو_دوسری_مشق} میں \عددی{I_S} اور \عددی{I} حاصل کریں۔
\begin{figure}[!h]
\centering
\includegraphics{figResistanceKCLexampleC}
\caption{مشق \حوالہ{مشق_مزاحمتی_دوسری_مشق} کی شکل۔}
\label{مشق_مزاحمتی_کرچاف_قانون_رو_دوسری_مشق}
\end{figure}

جوابات:\عددی{I_S=\SI{12}{\milli\ampere}}، \عددی{I=\SI{5}{\milli\ampere}}؛  برقی رو \عددی{I_S} حاصل کرنے کی خاطر نقطہ \عددی{j_1} کو جوڑ تصور کریں۔
\انتہا{مشق}
\FloatBarrier
%===============================
\ابتدا{مثال}
شکل \حوالہ{شکل_مزاحمتی_کرچاف_قانون_رو_مثال}-ب میں کسی بھی جگہ بند سطح کھینچ کر دیکھا جا سکتا ہے کہ کرچاف قانونِ رو بند سطح پر لاگو ہوتا ہے۔شکل \حوالہ{شکل_مزاحمتی_بند_سطح_قانون_رو}-الف میں ایسا ہی کیا گیا ہے۔بالائی اور نچلی سطح کے داخلی اور خارجی رو دریافت کریں۔

\begin{figure}
\centering
\begin{subfigure}{0.5\textwidth}
\centering
\includegraphics{figResistanceKCLsurfacesA}
\caption*{(الف)}
\end{subfigure}%
%
\begin{subfigure}{0.5\textwidth}
\centering
\includegraphics{figResistanceKCLsurfacesB}
\caption*{(ب)}
\end{subfigure}%
\caption{کرچاف قانونِ رو ہر بند سطح پر لاگو ہوتا ہے۔}
\label{شکل_مزاحمتی_بند_سطح_قانون_رو}
\end{figure}

حل:بالائی سطح کو جوڑ تصور کیا جا سکتا ہے۔شکل میں اس جوڑ کو \عددی{j_b} کہا گیا ہے۔بالائی سطح پر مجموعی داخلی رو
 \عددی{\SI{5}{\milli\ampere}+\SI{2}{\milli\ampere}} ہے۔ اس سے \عددی{\SI{7}{\milli\ampere}} رو خارج ہوتی ہے۔آپ دیکھ سکتے ہیں کہ داخلی اور خارجی رو برابر ہیں۔

نچلی سطح پر داخلی رو \عددی{\SI{3}{\milli\ampere}+\SI{6}{\milli\ampere}} ہے اور خارجی رو \عددی{\SI{9}{\milli\ampere}} ہے۔اس سطح پر بھی داخلی اور خارجی رو برابر ہیں۔نچلی سطح کو جوڑ \عددی{j_n} کہا گیا ہے۔ 
\انتہا{مثال}
\FloatBarrier
%============================

آپ شکل \حوالہ{شکل_مزاحمتی_کرچاف_قانون_رو_مثال}-ب پر کسی بھی جگہ پر بند سطح کھینچ کر دیکھ سکتے ہیں کہ اس سطح پر داخلی رو عین سطح سے خارجی رو کے برابر ہو گی۔

%=====================
\ابتدا{مشق}
شکل \حوالہ{شکل_مزاحمتی_بند_سطح_قانون_رو}-ب میں بند سطح کی داخلی اور خارجی رو حاصل کریں۔

جوابات:داخلی رو \عددی{\SI{9}{\milli\ampere}} ہے اور خارجی رو بھی \عددی{\SI{9}{\milli\ampere}} ہے۔
\انتہا{مشق}
\FloatBarrier
%============================
\ابتدا{مشق}\شناخت{مشق_مزاحمتی_مشق_الف}
شکل \حوالہ{شکل_مزاحمتی_مشق_الف} میں نا معلوم رو دریافت کریں۔    

\begin{figure}
\centering
\begin{subfigure}{0.5\textwidth}
\centering
\includegraphics{figResistanceKCLquizA}
\caption*{(الف)}
\end{subfigure}%
%
\begin{subfigure}{0.5\textwidth}
\centering
\includegraphics{figResistanceKCLquizB}
\caption*{(ب)}
\end{subfigure}%
\caption{مشق \حوالہ{مشق_مزاحمتی_مشق_الف} میں استعمال ہونے والا دور۔}
\label{شکل_مزاحمتی_مشق_الف}
\end{figure}
جواب:\عددی{I_1=\SI{5}{\milli\ampere}}، \عددی{I_2=\SI{11}{\milli\ampere}} اور \عددی{I_3=\SI{6}{\milli\ampere}}
\انتہا{مشق}
\FloatBarrier
%=========================
\ابتدا{مشق}\شناخت{مشق_مزاحمتی_مشق_ب}
شکل \حوالہ{شکل_مزاحمتی_مشق_ب}-الف میں \عددی{i_1} اور شکل-ب میں \عددی{i_a} دریافت کریں۔
\begin{figure}[!h]
\centering
\begin{subfigure}{0.5\textwidth}
\centering
\includegraphics{figResistanceKCLquizC}
\caption*{(الف)}
\end{subfigure}%
%
\begin{subfigure}{0.5\textwidth}
\centering
\includegraphics{figResistanceKCLquizD}
\caption*{(ب)}
\end{subfigure}%
\caption{مشق \حوالہ{مشق_مزاحمتی_مشق_ب} میں استعمال ہونے والا دور۔}
\label{شکل_مزاحمتی_مشق_ب}
\end{figure}

جوابات:\عددی{i_1=\SI{3}{\milli\ampere}}، \عددی{i_a=\tfrac{2}{3} \, \si{\milli\ampere}}
\انتہا{مشق}
\FloatBarrier
%=============================

کرچاف کا دوسرا قانون، \اصطلاح{کرچاف قانون برائے برقی دباو}  ہے۔اس قانون کو عموماً \اصطلاح{کرچاف قانون دباو}\فرہنگ{کرچاف!قانون دباو}\حاشیہب{Kirchoff's voltage law, KVL}\فرہنگ{Kirchoff!voltage law} کہا جاتا ہے۔

\ابتدا{قانون}
کرچاف قانونِ دباو کہتا ہے کہ کسی بھی بند راہ پر بڑھتے برقی دباو کا مجموعہ، گھٹتے برقی دباو کے مجموعے کے عین برابر ہو گا۔
\انتہا{قانون}

\begin{figure}
\centering
\begin{subfigure}{0.5\textwidth}
\centering
\includegraphics{figResistanceKVLvoltageRisevoltageFallA}
\caption*{(الف)}
\end{subfigure}%
%
\begin{subfigure}{0.5\textwidth}
\centering
\includegraphics{figResistanceKVLvoltageRisevoltageFallB}
\caption*{(ب)}
\end{subfigure}%
\caption{کرچاف قانونِ دباو۔}
\label{شکل_مزاحمتی_قانون_دباو}
\end{figure}

شکل \حوالہ{شکل_مزاحمتی_قانون_دباو}-الف میں جوڑ \عددی{j0} سے برقی دور میں گھڑی کے سمت گھومتے ہوئے بڑھتے دباو کا مجموعہ 
\begin{align*}
\text{\RL{بڑھتا دباو}} = 5+V_2+3
\end{align*}
حاصل ہوتا ہے جبکہ گھٹتے دباو کا مجموعہ
\begin{align*}
\text{\RL{گھٹتا دباو}}=V_1+ 3 +V_3
\end{align*}
حاصل ہوتا ہے۔کرچاف قانون دباو کے تحت یہ قیمتیں برابر ہیں یعنی
\begin{align*}
5+V_2+3=V_1+ 3 +V_3
\end{align*}
ہو گا۔اس مساوات کو یوں
\begin{align}\label{مساوات_مزاحمتی_مجموعہ_الف}
5+V_2+3-V_1-3-V_3=0
\end{align}
 بھی لکھا جا سکتا ہے۔یوں کرچاف قانون دباو کو 
\begin{align}
\sum_{b=1}^B V_b = \sum_{g=1}^{G} V_g  \quad \quad \text{\RL{کرچاف قانونِ دباو}}
\end{align}
لکھا جا سکتا ہے جہاں بند دائرے میں بڑھتے دباو کی تعداد \عددی{B} اور گھٹتے دباو کی تعداد \عددی{G} ہے۔

شکل \حوالہ{شکل_مزاحمتی_قانون_دباو}-الف میں بڑھتے دباو کو مثبت اور گھٹتے دباو کو منفی لکھتے ہوئے مجموعہ حاصل کرنے سے عین  مساوات \حوالہ{مساوات_مزاحمتی_مجموعہ_الف} حاصل ہوتا ہے لہٰذا کرچاف قانون دباو کو درج ذیل مساوات کی صورت میں بھی لکھا جا سکتا ہے۔
\begin{align}
\sum_{s=1}^{S} V_s =0  \quad \quad \text{\RL{کرچاف قانونِ دباو}}
\end{align}
اس مساوات میں اگر بڑھتے دباو کو مثبت لکھا جائے تب گھٹتے دباو کو منفی لکھا جائے گا اور اگر گھٹتے دباو کو مثبت لکھا جائے تب بڑھتے دباو کو منفی لکھا جائے گا۔

شکل \حوالہ{شکل_مزاحمتی_بکریاں_اور_کرچاف_قانون_رو} میں کرچاف قانونِ رو کو پہاڑی سے اترتی بکریوں کی مدد سے سمجھایا گیا۔آئیں کرچاف قانون دباو کو شکل \حوالہ{شکل_مزاحمتی_قانون_دباو_بلندی} کی مدد سے سمجھیں۔
\begin{figure}
\centering
\begin{subfigure}{0.5\textwidth}
\centering
\includegraphics{figResistanceKVLasHeight}
\caption*{(الف)}
\end{subfigure}%
%
\begin{subfigure}{0.5\textwidth}
\centering
\includegraphics{figResistanceKVLasHeightCircuit}
\caption*{(ب)}
\end{subfigure}%
\caption{کرچاف قانون دباو اور بلندی۔}
\label{شکل_مزاحمتی_قانون_دباو_بلندی}
\end{figure}

شکل \حوالہ{شکل_مزاحمتی_قانون_دباو_بلندی}-الف میں ایک عمارت کا بیرونی خاکہ دکھایا گیا ہے۔عمارت کے بائیں طرف سیڑھی کو استعمال کرتے ہوئے پہلی منزل \عددی{b} تک پہنچنا ممکن ہے۔آپ دیکھ سکتے ہیں کہ \عددی{a} سے پانچ سیڑھی بلندی پر \عددی{b} واقع ہے۔یوں \عددی{a} سے  \عددی{b} تک پہنچنے پر  آپ پانچ سیڑھی بلندی اختیار کریں گے۔ اس حقیقت کو ریاضیاتی طور پر \عددی{B_{ba}=5} لکھا جاتا ہے۔پہلی منزل کی چھت \عددی{b} تا \عددی{c} ہے یوں \عددی{b} سے \عددی{c} تک چلنے میں آپ کی بلندی جوں کی توں رہے گی۔اسی طرح \عددی{d} تک پہنچنے کی خاطر مزید دس سیڑھیاں چڑھنی ہو گی یعنی \عددی{B_{dc}=10}۔یوں \عددی{a} سے \عددی{d} کی اونچائی پندرہ سیڑھی ہے۔ان حقائق کو ریاضیاتی طور پر \عددی{B_{da}=B_{ba}+B_{dc}} لکھا جائے گا۔ اسی طرح \عددی{a} سے \عددی{h} تک \عددی{B_{ha}=B_{ba}+B_{dc}+B_{fe}+B_{hg}} ہو گا۔اب \عددی{i} سے \عددی{j} پہنچنے کے لئے پندرہ سیڑھی اترنا ہو گا یعنی \عددی{B_{ja}=B_{ba}+B_{dc}+B_{fe}+B_{hg}-B_{ij}} جس میں قیمتیں پر کرتے ہوئے \عددی{B_{ja}=5+10+4+6-15=10} حاصل ہوتا ہے۔اب \عددی{j} تک  پہنچنے کے لئے ضروری نہیں کہ عمارت کے بائیں جانب سے ہی ہم سیڑھیاں چرھنے شروع ہو جائیں۔ہم عمارت کے دائیں جانب سیڑھی استعمال کرتے ہوئے \عددی{a} سے \عددی{k} چڑھ سکتے ہیں جہاں سے \عددی{B_{ka}=10} حاصل ہوتا ہے۔آپ دیکھ سکتے ہیں کہ \عددی{a} سے \عددی{j} کی اونچائی کا انحصار اس پر بالکل نہیں کہ ہم کس راستے پر چلتے ہوئے اس بلندی کو ناپیں۔اگر عمارت کے بائیں جانب نقطہ \عددی{a} سے شروع ہو کر تمام سیڑھیاں استعمال کرتے ہوئے واپس  نقطہ \عددی{a} پہنچا جائے تو ہم کل پچیس سیڑھیاں بلندی تک پہنچنے کے بعد اتنا ہی واپس اتر چکے ہوں گے۔اس حقیقت کو \عددی{B_{ba}+B_{dc}+B_{fe}+B_{hg}-B_{ij}-B_{kl}=0} لکھا جا سکتا ہے جس کے تحت کسی بھی بند راہ پر چلنے سے جتنا اوپر چلا جائے اتنا ہی نیچے چلنا ہو گا۔یہی کچھ شکل \حوالہ{شکل_مزاحمتی_قانون_دباو_بکریاں} سے بھی دیکھا جا سکتا ہے جہاں فرحانہ پورا دن بکریاں چرانے کے بعد واپس ابتدائی نقطہ \عددی{a} پہنچتی ہے۔اگر پورے راستے پر ہر قدم اونچائی ناپی جائے تو جواب صفر ہی حاصل ہو گا۔ 

\begin{figure}
\centering
\includegraphics{figResistanceKVLgoats}
\caption{کرچاف قانونِ دباو اور پہاڑی پر چرتی بکریاں۔}
\label{شکل_مزاحمتی_قانون_دباو_بکریاں}
\end{figure}
 
شکل \حوالہ{شکل_مزاحمتی_قانون_دباو_بلندی}-الف کا مساوی برقی دور شکل \حوالہ{شکل_مزاحمتی_قانون_دباو_بلندی}-ب میں دکھایا گیا ہے۔شکل \حوالہ{شکل_مزاحمتی_قانون_دباو_بلندی}-الف میں \عددی{b} تا \عددی{c} بلندی برقرار رہتی ہے۔شکل \حوالہ{شکل_مزاحمتی_قانون_دباو_بلندی}-ب میں \عددی{b} تا \عددی{c} برقی دباو برقرار رہتا ہے۔اسی طرح شکل-الف میں  \عددی{d} تا \عددی{e} بلندی برقرار رہتی ہے۔شکل \حوالہ{شکل_مزاحمتی_قانون_دباو_بلندی}-ب میں \عددی{d} تا \عددی{e} برقی دباو برقرار رہتا ہے۔شکل-الف میں برقرار بلندی کو افقی دکھایا جاتا ہے جبکہ بلندی میں تبدیلی کو عمودی دکھایا جاتا ہے۔شکل-ب میں برقرار برقی دباو کو تار ظاہر کرتی ہے اور ایسی تار کو \اصطلاح{جوڑ}\فرہنگ{جوڑ}\حاشیہب{node}\فرہنگ{node} کہا جاتا ہے۔شکل-ب میں \عددی{V_{ba}=\SI{5}{\volt}} لکھا جا سکتا ہے۔اسی طرح \عددی{V_{dc}=\SI{10}{\volt}} اور \عددی{V_{da}=V_{ba}+V_{dc}} لکھا جائے گا۔اسی طرح \عددی{V_{ja}=V_{ba}+V_{dc}+V_{fe}+V_{hg}-V_{ij}} سے \عددی{V_{ja}=\SI{10}{\volt}} حاصل ہوتا ہے۔شکل-ب میں \عددی{a} سے شروع ہو کر گھڑی کی سمت میں پورا چکر کاٹتے ہوئے \عددی{V_{ba}+V_{dc}+V_{fe}+V_{hg}-V_{ij}-V_{kl}=0}  لکھا جا سکتا ہے جہاں بڑھتے دباو کو مثبت لکھا گیا ہے۔اسی طرح \عددی{j} سے شروع ہو کر گھڑی کے الٹ چلتے ہوئے \عددی{V_{ij}-V_{hg}-V_{fe}-V_{dc}-V_{ba}+V_{kl}=0} لکھا جا سکتا ہے۔اگر ہم گھٹتے دباو کو مثبت لکھیں تب  \عددی{j} سے شروع ہو کر گھڑی کے الٹ چلتے ہوئے \عددی{-V_{ij}+V_{hg}+V_{fe}+V_{dc}+V_{ba}-V_{kl}=0} لکھا جائے گا۔عام زندگی میں برقرار بلندی افقی سطح کو ظاہر کرتی ہے لہٰذا شکل \حوالہ{شکل_مزاحمتی_قانون_دباو_بلندی}-الف میں افقی لکیر برقرار بلندی کو ظاہر کرتی ہے۔برقی دور میں برقرار دباو کو افقی لکیر سے ظاہر کرنے کی کوئی روایت نہیں۔یوں شکل \حوالہ{شکل_مزاحمتی_قانون_دباو_بلندی}-ب میں افقی لکیر \عددی{b-c} اور عمودی لکیر \عددی{d-e} برقرار دباو کو ظاہر کرتے ہیں۔برقی دور میں موصل تار پر  دباو تبدیل نہیں ہوتی لہٰذا تار ہی برقرار دباو کو ظاہر کرتی ہے۔

کرچاف قانون دباو کے استعمال بند دائرے پر ہوتا ہے۔ایسا بند دائرہ فرضی بھی ہو سکتا ہے۔آئیں ایسی ایک مثال دیکھیں۔شہروں میں پانی کے تالاب  پر عموماً غوطہ لگانے کی خاطر اونچائی پر تختہ نسب ہوتا ہے جہاں سے غوطہ خور قلابازیاں کھاتا ہوا پانی تک پہنچتا ہے۔ شکل \حوالہ{شکل_مزاحمتی_فرضی_بند_دائرہ}-الف میں ایسا ہی \اصطلاح{تختہ غوطہ}\فرہنگ{تختہ غوطہ}\حاشیہب{diving board}\فرہنگ{diving board} دکھایا گیا ہے جس تک بائیں جانب نسب سیڑھی کے ذریعہ پہنچا جا سکتا ہے۔اس سیڑھی کو استعمال کرتے ہوئے غوطہ خور \عددی{a} سے \عددی{b} تک چڑھتا ہے۔یہاں سے وہ دوڑ لگاتا ہوا \عددی{c} پہنچ کر ہوا میں قلابازیاں کھاتا ہوا نیچے تالاب میں ڈبکی لگاتا ہے۔ شکل میں تیر کی لکیر غوطہ خور کے گِرنے کو دکھاتی ہے۔اب \عددی{a} سے \عددی{b} اور یہاں سے \عددی{c} تک حقیقی راہ پائی جاتی ہے جس پر غوطہ خور چلتا ہے لیکن \عددی{c} سے \عددی{d} تک کوئی سیڑھی نہیں ہے۔یہ بس خلاء میں فرضی راہ ہے جس پر غوطہ خور نیچے اترتا ہے جس کے بعد وہ واپس \عددی{a} تک لوٹتے ہوئے بند دائرے پر چال قدمی پوری کرتا ہے۔آپ دیکھ سکتے ہیں کہ بارہ سیڑھیاں چڑنے کے بعد غوطہ خور بارہ\حاشیہد{جی مجھے معلوم ہے کہ غوطہ خور اوپر چھلانگ لگا کر بارہ سیڑھی سے زیادہ بلندی سے گِرتا ہے۔مجھے امید ہے کہ آپ تمام گفتگو کی اصل مقصد سمجھ گئے ہوں گے۔} سیڑھی ہی نیچے گِرتا ہے۔

\begin{figure}
\centering
\begin{subfigure}{0.5\textwidth}
\centering
\includegraphics{figResistanceKVLexampleSwimmingPool}
\caption*{(الف)}
%\label{}
\end{subfigure}%
%
\begin{subfigure}{0.5\textwidth}
\centering
\includegraphics{figResistanceKVLexampleA}
\caption*{(ب)}
%\label{}
\end{subfigure}%
\caption{کرچاف قانون دباو کے استعمال میں بند دائرہ فرضی ہو سکتا ہے۔}
\label{شکل_مزاحمتی_فرضی_بند_دائرہ}
\end{figure}

آئیں اب یہی کچھ برقی دور میں بھی دیکھیں۔ایسا شکل \حوالہ{شکل_مزاحمتی_فرضی_بند_دائرہ}-ب کی مدد سے دیکھتے ہیں۔گھٹتے دباو کو مثبت لکھتے ہوئے،  \عددی{a} سے  گھڑی کی سمت چل کر ایک چکر کے بعد \عددی{-15+V_{R1}-4+V_{R2}-8+V_{R3}=0} لکھا جا سکتا ہے جس سے
\begin{align*}
V_{R1}+V_{R2}+V_{R3}=15+4+8
\end{align*}
حاصل ہوتا ہے۔ایسا حقیقی راہ پر کیا گیا۔آئیں اب \عددی{f} سے \عددی{a} اور یہاں سے \عددی{b} کے بعد فرضی راہ پر واپس \عددی{f} پہنچیں۔فرضی راہ کو نوک دار لکیر سے دکھایا گیا ہے جہاں تیر کا نشان مثبت سرے کو ظاہر کرتی ہے۔یوں
\begin{align*}
V_{R3}-15+V_{bf}=0
\end{align*}
لکھا جا سکتا ہے جہاں گھٹتے دباو کو مثبت لکھا گیا ہے۔اس سے 
\begin{align*}
V_{bf}=15-V_{R3}
\end{align*}
حاصل ہوتا ہے۔یوں اگر \عددی{V_{R3}=\SI{7}{\volt}} ہو تب \عددی{V_{bf}=\SI{8}{\volt}} ہو گا۔ یہاں بتلاتا چلوں کہ اس کتاب میں گھٹتے دباو کو ہی مثبت لکھا جائے گا۔ایسا لکھنے میں آپ کو شروع میں کچھ دقت ہو سکتی ہے۔اسی طرح دیگر فرضی بند دائروں پر مندرجہ ذیل لکھا جا سکتا ہے
\begin{align*}
V_{R3}-15+V_{R1}-4+V_{df}&=0\\
8-V_{R2}+V_{df}&=0\\
-V_{bf}+V_{R1}-4+V_{df}&=0
\end{align*} 
جہاں پہلی اور دوسری مساوات میں گھڑی کی سمت جبکہ دوسری مساوات میں گھڑی کی الٹ سمت چلا گیا ہے۔یوں اگر \عددی{V_{R1}=\SI{9}{\volt}}، \عددی{V_{R2}=\SI{11}{\volt}} اور \عددی{V_{R3}=\SI{7}{\volt}} ہوں تب \عددی{V_{df}=\SI{3}{\volt}} اور \عددی{V_{bf}=\SI{8}{\volt}} ہوں گے۔

\begin{figure}
\centering
\includegraphics{figResistanceKVLdependentEquations}
\caption{تابع اور غیر تابع مساوات۔}
\label{شکل_مزاحمتی_تابع_مساوات}
\end{figure}

شکل \حوالہ{شکل_مزاحمتی_تابع_مساوات} میں کرچاف قانون دباو استعمال کرتے ہوئے کل تین عدد مساوات لکھنا ممکن ہے۔یہ مساوات بائیں بند دائرہ \عددی{abcfa}، دائیں بند دائرہ \عددی{afcdea} اور بیرونی بند دائرہ \عددی{abcdea} پر لکھے جائیں گے جنہیں یہاں پیش کرتے ہیں۔
\begin{align}
-V_1-V_{R1}+V_2+V_{R4}&=0  \label{مساوات_مزاحمتی_تابع_الف}\\
-V_{R4}-V_2-V_{R2}-V_3+V_{R3}&=0  \label{مساوات_مزاحمتی_تابع_ب}\\
-V_1-V_{R1}-V_{R2}-V_3+V_{R3}&=0  \label{مساوات_مزاحمتی_تابع_پ}
\end{align}
مساوات \حوالہ{مساوات_مزاحمتی_تابع_الف} اور مساوات \حوالہ{مساوات_مزاحمتی_تابع_ب} کو آپس میں جمع کرنے سے مساوات \حوالہ{مساوات_مزاحمتی_تابع_پ} حاصل ہوتا ہے۔اسی طرح مساوات \حوالہ{مساوات_مزاحمتی_تابع_ب} سے مساوات \حوالہ{مساوات_مزاحمتی_تابع_پ} منفی کرنے سے مساوات \حوالہ{مساوات_مزاحمتی_تابع_الف} حاصل ہوتا ہے۔یوں ان میں سے کسی بھی دو مساوات سے تیسری مساوات حاصل کی جا سکتی ہے۔ایسی صورت میں دو عدد مساوات کو \اصطلاح{غیر تابع مساوات} کہتے ہیں جبکہ ان سے حاصل تیسری مساوات \اصطلاح{تابع مساوات}\فرہنگ{تابع مساوات}\حاشیہب{dependent equation}\فرہنگ{dependent equation} کہلاتی ہے۔آپ جانتے ہیں کہ دو آزاد متغیرات حاصل کرنے کی خاطر دو عدد غیر تابع مساوات درکار ہوتے ہیں۔یوں آزاد متغیرات \عددی{x} اور \عددیء{y} مندرجہ ذیل ہمزاد مساوات  میں سے کسی بھی دو مساوات کو بیک وقت حل کرنے سے حاصل کرنا ممکن ہے۔ان میں کسی بھی دو عدد مساوات کو غیر تابع تصور کرتے ہوئے تیسری مساوات حاصل کی جا سکتی ہے لہٰذا تیسری تابع مساوات ہے جو کوئی نئی معلومات فراہم نہیں کرتی۔تابع مساوات غیر ضروری مساوات ہوتی ہے جسے لکھنے کی ضرورت نہیں ہے۔
\begin{align*}
x+y&=3\\
x-y&=-1\\
3x+y&=5
\end{align*}
شکل \حوالہ{شکل_مزاحمتی_تابع_مساوات} صرف دو عدد غیر تابع مساوات مہیا کرتا ہے لہٰذا اگرچہ ہم اس دور کے لئے تین مساوات لکھ سکتے ہیں لیکن ایسا کرنے کی کوئی ضرورت نہیں۔کسی بھی دور میں مساوات لکھنے سے پہلے بند دائرے چننے جاتے ہیں۔بند دائرے یوں چنیں کہ دور میں نسب تمام اجزاء کسی نہ کسی دائرے کا حصہ بنے۔یوں کم سے کم تعداد کے بند دائرے چننے سے کم سے کم مساوات حاصل ہوں گے۔ کم تعداد کے مساوات حل کرنا نسبتاً زیادہ آسان ہوتا ہے۔
%=====================

\حصہ{سلسلہ وار جڑے پرزوں میں رو}
کرچاف کے قوانین جاننے کے بعد آئیں انہیں چند سادہ ادوار پر لاگو کرتے ہوئے کچھ کارآمد نتائج حاصل کریں۔شکل \حوالہ{شکل_مزاحمتی_دباو_تقسیم}-الف میں منبع دباو \عددی{v(t)} کے ساتھ دو عدد مزاحمت سلسلہ وار جڑے ہیں۔منبع اور \عددیء{R_1} آپس میں جوڑ \عددی{j_1} پر ملتے ہیں۔منبع کی رو \عددی{i_1} اور مزاحمت میں داخل ہوتی رو کو \عددی{i_2} تصور کرتے ہوئے جوڑ \عددی{j_1} پر کرچاف قانون رو لاگو کرتے   ہوئے \عددی{i_1=i_2} لکھا جا سکتا ہے۔یوں منبع اور مزاحمت \عددی{R_1} میں بالکل برابر رو پائی جاتی ہے۔یہی ترکیب مزاحمت \عددی{R_1} اور مزاحمت \عددی{R_2} کے جوڑ \عددی{j_2} پر لاگو کرتے ہوئے \عددی{i_2=i_3} لکھا جا سکتا ہے۔یوں اگر \عددی{i_1=\SI{3}{\milli\ampere}} ہوتی تب \عددی{i_2} اور \عددی{i_3} بھی \عددی{\SI{3}{\milli\ampere}} ہوتے اور یہ برقی رو دور میں گھڑی کی سمت گھومتی۔اس حقیقت کو یوں بہتر بیان کیا جا سکتا ہے کہ سلسلہ وار جڑے پرزوں میں یکساں برقی رو پائی جاتی ہے۔
\begin{figure}
\centering
\begin{subfigure}{0.5\textwidth}
\centering
\includegraphics{figResistanceSeriesConnectedSameCurrent}
\caption*{(الف)}
%\label{}
\end{subfigure}%
%
\begin{subfigure}{0.5\textwidth}
\centering
\includegraphics{figResistanceVoltageDivider}
\caption*{(ب)}
%\label{}
\end{subfigure}%
\caption{سلسلہ وار جڑے مزاحمت میں دباو کی تقسیم۔}
\label{شکل_مزاحمتی_دباو_تقسیم}
\end{figure}

\حصہ{تقسیمِ دباو}
گزشتہ حصے میں ہم نے دیکھا کہ سلسلہ وار دور میں ہر مقام پر یکساں رو پائی جاتی ہے۔اسی سلسلہ وار دور کو شکل \حوالہ{شکل_مزاحمتی_دباو_تقسیم}-ب میں دوبارہ پیش کیا گیا ہے جہاں دور کی رو کو \عددی{i(t)} لکھا گیا ہے۔کسی بھی مزاحمت میں گزرتی رو اور مزاحمت کے سروں کے مابین دباو کا تعلق قانون اوہم دیتا ہے۔یوں مزاحمت \عددی{R_1} پر \عددی{v_{R1}=i(t) R_1}  اور مزاحمت \عددی{R_2} پر \عددی{v_{R2}=i(t) R_2} دباو پایا جائے گا۔شکل-ب کے لئے کرچاف قانون دباو سے
\begin{align*}
v(t)&=v_{R1}+v_{R2}\\
&=i(t) \left(R_1+R_2 \right)
\end{align*}
لکھا جا سکتا ہے۔مزاحمت \عددی{R_1} کے دباو کو منبع کے دباو سے تقسیم کرتے ہوئے
\begin{align*}
\frac{v_{R1}}{v(t)}=\frac{i(t) R_1}{i(t) \left[R_1+R_2 \right]}
\end{align*}
یا
\begin{align}\label{مساوات_مزاحمتی_تقسیم_دباو_الف}
\frac{v_{R1}}{v(t)}=\frac{R_1}{R_1+R_2}
\end{align}
حاصل ہوتا ہے۔ اسی طرح مزاحمت \عددی{R_2} کے دباو کو منبع کے دباو سے تقسیم کرنے سے
\begin{align}\label{مساوات_مزاحمتی_تقسیم_دباو_ب}
\frac{v_{R2}}{v(t)}=\frac{R_2}{R_1+R_2}
\end{align}
حاصل ہوتا ہے۔مساوات \حوالہ{مساوات_مزاحمتی_تقسیم_دباو_الف} اور مساوات مساوات \حوالہ{مساوات_مزاحمتی_تقسیم_دباو_ب} تقسیم دباو کے مساوات ہیں۔آئیں ان کی افادیت مثال کی مدد سے سمجھیں۔

%============================
\ابتدا{مثال}
شکل \حوالہ{شکل_مزاحمتی_دباو_تقسیم} میں \عددی{v(t)=\SI{15}{\volt}} ہے جبکہ مزاحمت \عددی{R_1=\SI{1}{\kilo\ohm}} اور \عددی{R_2=\SI{2}{\kilo\ohm}} ہیں۔دونوں مزاحمت کے دباو حاصل کریں۔

مساوات \حوالہ{مساوات_مزاحمتی_تقسیم_دباو_الف} سے
\begin{align*}
v_{R1}=\frac{15 \times 1000}{1000+2000}=\SI{5}{\volt} 
\end{align*}
اور مساوات \حوالہ{مساوات_مزاحمتی_تقسیم_دباو_ب} سے
\begin{align*}
v_{R2}=\frac{15 \times 2000}{1000+2000}=\SI{10}{\volt} 
\end{align*}
حاصل ہوتا ہے۔آپ دیکھ سکتے ہیں کہ سلسلہ وار مزاحمت جوڑنے سے داخلی دباو کو مختلف قیمتوں میں تقسیم کیا جا سکتا ہے۔دو سے زیادہ مزاحمت سلسلہ وار جوڑتے ہوئے داخلی دباو کو زیادہ حصوں میں تقسیم کیا جا سکتا ہے۔
\انتہا{مثال}
%====================

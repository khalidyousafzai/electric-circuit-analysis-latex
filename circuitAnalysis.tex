\documentclass[leqno,b5paper]{book}
%===================
%testing tikz inside book
\usepackage{circuitikz}
\usepackage{pgfplotstable}
\usepackage{pgfplots}
\usepackage{tikz-3dplot}
\pgfplotsset{compat=newest,}
\usepgfplotslibrary{units}
%\pgfplotsset{compat=1.9}
\usepgfplotslibrary{polar}
\usepackage{ifdraft}
\usetikzlibrary{3d,shadings,fadings,intersections,calc,decorations.markings,decorations.pathreplacing,external,shapes.misc}


%\tikzexternalize[mode=list and make] %disable to generate figures
%\tikzexternaldisable  %enables figures after this command. put this is the tex file

\pgfmathsetmacro{\x}{2}     %smallest resistor sizes
\pgfmathsetmacro{\y}{2}
\pgfmathsetmacro{\xx}{2.5}   %somewhat larger resistor leads. gives more space
\pgfmathsetmacro{\yy}{2.5}
\pgfmathsetmacro{\xxx}{3}   %still larger resistor leads. gives even more space
\pgfmathsetmacro{\yyy}{3}
\pgfmathsetmacro{\dx}{0.2}     %moving labels beyond resistor outline
\pgfmathsetmacro{\dy}{0.2}
\pgfmathsetmacro{\pin}{0.3}

\pgfmathsetmacro{\boxW}{0.5}   %width of box circuit
\pgfmathsetmacro{\boxH}{2.5}   %height of box circuit

%=============================
%complex numbers, squared voltages
\newcommand*{\bZ}{{\ensuremath{{\boldsymbol{Z}}}}}           %complex impedance
\newcommand*{\bY}{{\ensuremath{{\boldsymbol{Y}}}}}           %complex admittance
\newcommand*{\bZCC}{{\ensuremath{{\boldsymbol{Z}}^{*}}}}                                            %complex conjugate impedance
\newcommand*{\bYCC}{{\ensuremath{{\boldsymbol{Y}}^{*}}}}                                            %complex conjugate

\newcommand*{\bVrms}{{\ensuremath{\hat{V}_{\textup{rms}}}}}           %phasor voltage
\newcommand*{\bIrms}{{\ensuremath{\hat{I}_{\textup{rms}}}}}           %phasor current
\newcommand*{\Vrms}{{\ensuremath{V_{\textup{rms}}}}}       %rms voltage
\newcommand*{\Irms}{{\ensuremath{I_{\textup{rms}}}}}           %rms current
\newcommand*{\Arms}{{\ensuremath{A_{\textup{rms}}}}}           %rms amps
\newcommand*{\VrmsS}{{\ensuremath{V^2_{\textup{rms}}}}}       %rms squared
\newcommand*{\IrmsS}{{\ensuremath{I^2_{\textup{rms}}}}}           %rms squared
\newcommand*{\bVrmsCC}{{\ensuremath{\hat{V}^{*}_{\textup{rms}}}}}                  %conjugate phasor voltage
\newcommand*{\bIrmsCC}{{\ensuremath{\hat{I}^{*}_{\textup{rms}}}}}           %conjugate phasor current


\newcommand*{\kx}[1]{{\ensuremath{{\boldsymbol{#1}}}}}                  %complex quantity
\newcommand*{\bS}{{\ensuremath{{\boldsymbol{S}}}}}                         %complex power
\newcommand*{\bH}{{\ensuremath{{\boldsymbol{H}}}}}                       %network functions
\newcommand*{\bA}{{\ensuremath{{\boldsymbol{A}}}}}                        %voltage gain

\newcommand*{\pf}{{\ensuremath{{\textup{pf}}}}}
\newcommand*{\rms}{{\ensuremath{\textup{rms}}}}           %rms
\newcommand*{\BW}{{\ensuremath{{\textup{BW}}}}}   %bandwidth

\newcommand*{\Laplace}{\mathcal{L}}   %Laplace transform
\newcommand*{\Fourier}{\mathcal{F}}   %Fourier transform

\newcommand*{\kB}[1]{{\ensuremath{{\textup{#1}}}}}  %Laplace symbol general use. 
									%following were used too often so gave them specific symbols
\newcommand*{\bF}{{\ensuremath{{\textup{F}}}}}    %Fourier transform of 
\newcommand*{\bP}{{\ensuremath{{\textup{P}}}}}   %Laplace fraction
\newcommand*{\bQ}{{\ensuremath{{\textup{Q}}}}}  %Laplace fraction
\newcommand*{\bV}{{\ensuremath{{\textup{V}}}}}  %Laplace Voltage
\newcommand*{\bI}{{\ensuremath{{\textup{I}}}}}  %Laplace Current

%for vertical spacing in tables use\Tstrut and \Bstrut  
\newcommand\Tstrut{\rule{0pt}{2.6ex}}       % Top strut
\newcommand\Bstrut{\rule[-1.2ex]{0pt}{0pt}} % Bottom strut

\definecolor{lgray}{cmyk}{0,0,0,0.2}
\definecolor{dgray}{cmyk}{0,0,0,0.7}

%==========================
%my styles
\pgfplotsset{
kStyleCircuitsA/.style={axis lines*=middle,scaled x ticks=false,scaled y ticks=false,
every axis x label/.style={
    at={(ticklabel* cs:1.15)},
    anchor=east,},
	every axis y label/.style={
    at={(ticklabel* cs:1.05)},
    anchor=east,}
},
}
%=====================
\tikzset{component/.style={draw,thick,circle,fill=white,minimum size =0.75cm,inner sep=0pt}}

\tikzset{
    mark position/.style args={#1(#2)}{
        postaction={
            decorate,
            decoration={
                markings,
                mark=at position #1 with \coordinate (#2);
            }
        }
    }
}
%========================
%\usepackage[hidelinks]{hyperref}  used in machines book but not here
\usepackage{./tex/khalidUrduBooksEMT}                     %my sty file
\usepackage{amsmath}
\usepackage{amsbsy} %for bold Poynting
\usepackage{ mathrsfs}   %for Poynting symbol
\usepackage{IEEEtrantools}


\sisetup{math-micro=\textup{µ},text-micro=µ,math-ohm  =\upOmega}   %with mathpazo this is needed else must not be here. now micro is smaller
\DeclareSIUnit \var {var}    %used in electric circuits volt-ampere-reactive


\input{./tex/myUrduCommandsEMT.tex}                  %turning latex into urdu
\input{./tex/myUrduGreek.tex}
\input{./tex/myEMTvectors.tex}
\input{./tex/myDeclareFunctions.tex}     %\sech, \csch, \arcsh, \arcs   hyperbolic and arc-secant etc



%\input{./tex/myElectronicsVariablesBetter.tex}         %these are all tested. to use at the very end when book is finished

\graphicspath{{./fig/figFrontPage/}{./fig/figBasicConcepts/}{./fig/figResistive/}{./fig/figNodalLoop/}{./fig/figOpamp/}{./fig/figTheorem/}{./fig/figCapacitorInductor/}{./fig/figTransients/}{./fig/figFreqResponse/}{./fig/figLaplaceTransform/}{./fig/figLaplaceTransformApplication/}{./fig/figFourierAnalysis/}}		%paths to figures


\includeonly{./tex/cktSymbols,./tex/cktPreface,./tex/prefaceFirstBook,./tex/cktBasics,./tex/cktResistive,./tex/cktNodalLoop,./tex/cktOpamp,./tex/cktTheorems,./tex/cktCapacitorInductor,./tex/cktTransients,./tex/cktSteadyStateAC,./tex/cktSteadyStatePower,./tex/cktMagneticallyCoupledNetworks,./tex/cktPolyphaseCircuits,./tex/cktFrequencyResponse,./tex/cktLaplaceTransform,./tex/cktApplicationLaplace,./tex/cktFourierAnalysis,./tex/cktTwoPortNetworks}
%\includeonly{./tex/cktSymbols,./tex/cktPreface,./tex/prefaceFirstBook,./tex/cktTheorems}%
%\includeonly{./tex/cktSymbols,./tex/cktPreface,./tex/prefaceFirstBook,./tex/cktTransients}%


\author{
خالد خان یوسفزئی\\
\\
{\small {جامعہ کامسیٹ، اسلام آباد}}\\
\texttt{khalidyousafzai@comsats.edu.pk}
}

%=========



\title{برقی ادوار}
\date{}                           %if absent gives date in arabic which is a rubbish

%\linenumbers
%\makeindex

%==========
\begin{document}
\begin{urdufont}


\renewcommand*{\contentsname}{عنوان}    %this command has to be placed right here

\frontmatter                          %just added instead of \pagenumbering{roman}
%%\pagenumbering{roman}

\maketitle

\tableofcontents
\pagestyle{empty}
\newpage
\include{./tex/cktPreface}
\newpage
\include{./tex/prefaceFirstBook}
%\newpage
%\include{./tex/cktSymbols}


\mainmatter                      %added this
\renewcommand*{\chaptername}{باب}
%%\pagenumbering{arabic}   %instead of this

\pagestyle{headings}


\باب{بنیاد}
اس کتاب میں \اصطلاح{بین الاقوامی نظام اکائی}\فرہنگ{بین الاقوامی نظام اکائی}\فرہنگ{نظام اکائی!بین الاقوامی}\حاشیہب{SI system}\فرہنگ{SI system} استعمال کی گئی ہے جس کے چند بنیادی اکایاں کلوگرام (\عددی{\si{\kilo\gram}})، میٹر (\عددی{\si{\meter}})، سیکنڈ (\عددی{\si{\second}}) اور کیلون (\عددی{\si{\kelvin}}) ہیں۔ان اکایوں کے ساتھ عموماً شکل \حوالہ{شکل_بنیادی_سابقہ} میں دکھائے گئے ضربیے استعمال کئے جاتے ہیں جن سے آپ بخوبی واقف ہیں۔
\begin{figure}
\centering
\includegraphics{figBasicSIprefix}
\caption{بین الاقوامی نظام اکائی کے ضربیے۔}
\label{شکل_بنیادی_سابقہ}
\end{figure}

\حصہ{برقی بار، برقی رو اور برقی دباو}
اس کتاب میں \اصطلاح{برقی بار}\فرہنگ{برقی بار}\حاشیہب{electric charge}\فرہنگ{charge}  اور
 \اصطلاح{برقی رو}\فرہنگ{برقی رو}\حاشیہب{electric current}\فرہنگ{current}  \,  کلیدی کردار ادا کریں گے۔ برقی بار کی اصطلاح کو چھوٹا کر کے صرف \اصطلاح{برق} یا صرف \اصطلاح{بار} کی اصطلاح استعمال کی جائے گی جبکہ برقی رو کی اصطلاح کو چھوٹا کر کے \اصطلاح{رو} کی اصطلاح استعمال کی جائے گی۔برقی بار کی حرکت کو برقی رو کہتے ہیں۔چونکہ بار کی حرکت سے توانائی ایک مقام سے دوسرے مقام منتقل ہوتی ہے لہٰذا ہماری دلچسپی کا مرکز برقی رو ہو گی۔

موصل تار  کی مدد سے برقی پرزہ جات کو مختلف انداز میں آپس میں جوڑنے سے \اصطلاح{برقی دور}\فرہنگ{برقی دور}\حاشیہب{electric circuit}\فرہنگ{circuit} حاصل ہوتا ہے۔جیسے پائپ سے پانی کو ایک مقام سے دوسرے مقام تک منتقل کیا جاتا ہے، بالکل اسی طرح برقی دور میں ایک نقطے سے دوسرے نقطے تک بار موصل تار کے ذریعہ پہنچایا جاتا ہے۔یوں اگر پانی کو بار تصور کیا جائے تو حرکت کرتے پانی کو برقی رو تصور کیا جائے گا جبکہ موصل تار کو پائپ تصور کیا جائے گا۔برقی ادوار سمجھنے میں یہ مشابہت مدد گار ثابت ہوتی ہے۔  

کسی بھی نقطے پر برقی رو سے مراد اس نقطے سے فی سیکنڈ گزرتا بار  ہے۔رو اور بار کے تعلق کو \اصطلاح{تفرقی}\فرہنگ{تفرقی صورت}\حاشیہب{differential form}\فرہنگ{differential form} صورت میں یوں
\begin{align}\label{مساوات_بنیادی_رو_تعریف}
i=\frac{\dif q}{\dif t}
\end{align}
اور \اصطلاح{تکملہ صورت}\فرہنگ{تکملہ صورت}\حاشیہب{integral form}\فرہنگ{integral form} میں یوں
\begin{align}
q=\int_{-\infty}^{t} i \dif t
\end{align}
لکھا جا سکتا ہے جہاں برقی بار کو \عددی{q} سے ظاہر کیا گیا ہے اور برقی رو کو \عددی{i} سے ظاہر کیا گیا ہے۔بدلتے متغیرات کو انگریزی کے چھوٹے حروف تہجی  مثلاً \عددی{i} یا \عددی{q} سے ظاہر کیا جاتا ہے جبکہ غیر متغیر مقدار کو انگریزی کے بڑے حروف تہجی سے ظاہر کیا جاتا ہے۔یوں غیر متغیر رو کو \عددی{I} اور غیر متغیر بار کو \عددی{Q} سے ظاہر کیا جائے گا۔

بار کی اکائی کو \اصطلاح{کولمب}\فرہنگ{کولمب}\حاشیہب{Coulomb}\فرہنگ{Coulomb} کہتے ہیں جسے \عددی{\si{\coulomb}} کی علامت سے ظاہر کیا جاتا ہے جبکہ رو کی اکائی کو \اصطلاح{ایمپیئر}\فرہنگ{ایمپیئر}\حاشیہب{Ampere}\فرہنگ{Ampere} کہتے ہیں۔ایمپیئر کی علامت \عددی{\si{\ampere}} ہے۔اگر تار سے ایک سیکنڈ دورانیے میں ایک کولمب کا بار گزر رہا ہو تب تار میں ایک ایمپیئر کی برقی رو پائی جائے گی۔

روایتی طور پر تصور کیا جاتا تھا کہ تار میں مثبت بار کی حرکت سے برقی رو پیدا ہوتی ہے۔اب ہم جانتے ہیں کہ حقیقت میں موصل تار میں مثبت ایٹم ساکن ہوتے ہیں اور آزاد منفی الیکٹران کی  حرکت سے  رو پیدا ہوتی ہے۔اس حقیقت کے باوجود، تصور کیا جاتا ہے کہ مثبت بار کی حرکت برقی رو کو جنم دیتی ہے۔شکل \حوالہ{شکل_بنیادی_مثبت_منفی_بار_رو}-الف میں فی سیکنڈ \عددی{\SI{3}{\coulomb}} کا بار بائیں سے دائیں جانب منتقل ہو رہا ہے جو بائیں سے دائیں جانب \عددی{\SI{3}{\ampere}} رو کو جنم دیتی ہے۔ شکل \حوالہ{شکل_بنیادی_مثبت_منفی_بار_رو}-ب میں فی سیکنڈ  \عددی{\SI{-2}{\coulomb}} کا بار دائیں سے بائیں جانب منتقل ہو رہا ہے جو بائیں سے دائیں جانب \عددی{\SI{2}{\ampere}} کی رو پیدا کرتی ہے۔بار کا قطب اور سمت بہاو جانتے ہوئے رو کی مقدار اور سمت کا تعین ممکن ہوتا ہے۔
\begin{figure}
\centering
\begin{subfigure}{0.5\textwidth}
\centering
\begin{tikzpicture}
\draw(0,0) rectangle ++(-0.5,\y+0.5);
\draw(\x,0) rectangle ++(0.5,\y+0.5);
\draw(-0.25,\y/2+0.25)node[rotate=90]{\RL{بایاں دور}};
\draw(\x+0.25,\y/2+0.25)node[rotate=90]{\RL{دایاں دور}};
\draw(0,0.25)--++(\x,0);
\draw(0,0.25+\y) to [short,i_={$\SI{3}{\ampere}$}]++(\x,0);
\draw[-latex](\x/4,\y+0.25+0.3)--++(\x/2,0)node[pos=0.5,above]{$\SI{3}{\coulomb\per\second}$};
\end{tikzpicture}
\caption*{(الف)}
\end{subfigure}%
\begin{subfigure}{0.5\textwidth}
\centering
\begin{tikzpicture}
\draw(0,0) rectangle ++(-0.5,\y+0.5);
\draw(\x,0) rectangle ++(0.5,\y+0.5);
\draw(-0.25,\y/2+0.25)node[rotate=90]{\RL{بایاں دور}};
\draw(\x+0.25,\y/2+0.25)node[rotate=90]{\RL{دایاں دور}};
\draw(0,0.25)--++(\x,0);
\draw(0,0.25+\y) to [short,i_={$\SI{2}{\ampere}$}]++(\x,0);
\draw[latex-](\x/4,\y+0.25+0.3)--++(\x/2,0)node[pos=0.5,above]{$\SI{-2}{\coulomb\per\second}$};
\end{tikzpicture}
\caption*{(ب)}
\end{subfigure}%
\caption{مثبت بار اور منفی بار کی حرکت سے پیدا رو۔}
\label{شکل_بنیادی_مثبت_منفی_بار_رو}
\end{figure}

غیر متغیر برقی رو کو \اصطلاح{یک سمتی رو}\فرہنگ{یک سمتی رو}\فرہنگ{رو!یک سمتی}\حاشیہب{direct current, DC}\فرہنگ{direct current}\فرہنگ{DC} کہتے ہیں۔یک سمتی رو کی مقدار وقت کے ساتھ تبدیل نہیں ہوتی۔وقت کے ساتھ تبدیل ہوتی برقی رو کو \اصطلاح{بدلتی رو}\فرہنگ{بدلتی رو}\فرہنگ{رو!بدلتی}\حاشیہب{alternating current, AC}\فرہنگ{alternating current}\فرہنگ{AC} کہتے ہیں۔ ان دونوں کو شکل میں دکھایا گیا ہے۔موبائل کی بیٹری یک سمتی رو پیدا کرتی ہے جبکہ گھریلو پنکھا بدلتی رو سے چلتا ہے۔
\begin{figure}
\centering
\includegraphics{figBasicCurrentAndDirection}
\caption{برقی رو کو بیان کرنے کے درست طریقے۔}
\label{شکل_بنیادی_رو_درست_بیان}
\end{figure}

شکل \حوالہ{شکل_بنیادی_رو_درست_بیان}-الف میں دور ت اور دور ٹ کو دو تاروں سے آپس میں جوڑا گیا ہے۔بالائی تار میں دور ت سے دور ٹ کی جانب تین ایمپیئر کی رو پائی جاتی ہے۔اس تار پر تیر کا نشان رو کی سمت کو ظاہر کرتا ہے جبکہ تار کے نیچے \عددی{\SI{3}{\ampere}} لکھ کر رو کی مقدار بیان کی گئی ہے۔اب تصور کریں کہ تار پر تیر کا نشان نہیں دیا گیا ہے۔ایسی صورت میں برقی رو \عددی{I} کو یا تو دور ت سے دور ٹ کی جانب تصور کیا جا سکتا ہے اور یا دور ٹ سے دور ت کی جانب۔پہلی صورت کو شکل-الف میں دکھایا گیا ہے جہاں تار سے ہٹ کر دور ت سے دور ٹ کی جانب تیر سے رو \عددی{I} کو دکھایا گیا ہے۔چونکہ اصل رو اسی سمت میں ہے لہٰذا \عددی{I=\SI{3}{\ampere}} لکھا جائے گا۔دوسری صورت کو شکل-ب میں دکھایا گیا ہے جہاں دور ٹ سے دور ت کی جانب تیر کھینچا گیا ہے۔یوں شکل-ب میں برقی رو کی سمت دور ٹ سے دور ت کی جانب لی گئی ہے۔چونکہ اصل رو کی سمت تصور کردہ سمت کے الٹ ہے لہٰذا یہاں \عددی{I=\SI{-3}{\ampere}} لکھا جائے گا۔شکل-الف اور شکل-ب میں دکھائے گئے دونوں طریقے درست ہیں۔
\begin{figure}
\centering
\includegraphics{figBasicVoltageCurrentRelation}
\caption{مزاحمت کی رو اور دباو لکھنے کے چار ممکنہ طریقے۔}
\label{شکل_بنیادی_غیر_عامل_ترکیب}
\end{figure}
%
\begin{figure}
\centering
\includegraphics{figBasicOhmLawPassiveConvention}
\caption{انفعالی سمت کے ترکیب کی پہچان۔}
\label{شکل_بنیادی_غیر_عامل_ترکیب_پہچان}
\end{figure}

شکل \حوالہ{شکل_بنیادی_غیر_عامل_ترکیب}-الف میں \عددی{\SI{5}{\ohm}} کی مزاحمت میں \عددی{\SI{4}{\ampere}} کی رو پائی جاتی ہے۔اس مزاحمت کے دونوں سرے مزید پرزہ جات سے جڑے ہیں جنہیں شکل میں نہیں دکھایا گیا ہے۔شکل-ب تا شکل-ٹ میں مزاحمت پر دباو اور مزاحمت میں رو کو مختلف طریقوں سے لکھا گیا ہے۔کسی بھی دو متغیرات کو کل چار انداز میں لکھا جا سکتا ہے۔یہی دو عدد متغیرات یعنی دباو اور رو کے لئے بھی درست ہے لہٰذا انہیں لکھنے کے کل چار طریقے ہیں۔شکل \حوالہ{شکل_بنیادی_غیر_عامل_ترکیب_پہچان} میں برقی دباو اور برقی رو کے مقدار لکھے بغیر یہی چار طریقے دوبارہ دکھائے گئے ہیں۔ان میں شکل-ب اور شکل-ٹ کے طرز کو \اصطلاح{انفعالی سمت کی ترکیب}\فرہنگ{انفعالی سمت کی ترکیب}\حاشیہب{passive sign convention}\فرہنگ{passive sign convention} کہتے ہیں۔انفعالی سمت کی ترکیب میں دباو \عددی{V} اور رو \عددی{I} کی سمتیں یوں چننی جاتی ہیں کہ برقی پرزے میں رو مثبت سرے سے داخل ہوتی ہے۔یوں شکل-ب میں مزاحمت کے بالائی سرے کو دباو کا مثبت سرا چنا گیا ہے لہٰذا انفعالی سمت کی ترکیب میں اسی سرے پر رو مزاحمت میں ہو گی۔اسی طرح شکل-ٹ میں مزاحمت کا نچلا سرا دباو کا مثبت سر ہے لہٰذا انفعالی سمت کی ترکیب میں اسی سر پر مزاحمت میں رو داخل ہو گی۔یاد رہے کہ انفعالی سمت کی ترکیب میں اصل برقی رو اور برقی دباو کی درست سمتوں کا کوئی کردار نہیں۔\اصطلاح{قانونِ اوہم}\فرہنگ{قانون اوہم}\فرہنگ{اوہم!قانون}\حاشیہب{Ohm's law}\فرہنگ{Ohm!law} اور طاقت کے حساب میں انفعالی سمت کی ترکیب استعمال کیا جاتا ہے۔

\ابتدا{قانون}
\اصطلاح{انفعالی سمت کی ترکیب} میں برقی پرزے پر دباو کی سمت چننے کے بعد رو کی سمت یوں چننی جاتی ہے کہ چنے گئے دباو کے مثبت سر سے پرزے میں  رو داخل ہو۔
\انتہا{قانون}

عام زندگی میں اونچائی کو زمین سے ناپا جاتا ہے جہاں زمین کی اونچائی صفر کے برابر لی جاتی ہے۔یوں اونچائی کے ناپ میں زمین کو نقطہ \اصطلاح{حوالہ}\فرہنگ{حوالہ}\حاشیہب{reference}\فرہنگ{reference} لیا جاتا ہے۔شکل \حوالہ{شکل_بنیادی_دباو_اور_اونچائی}-الف میں سات منزلہ عمارت دکھائی گئی ہے۔اگر زمین نقطہ ت پر ہو تب نقطہ ن مثبت تین پڑھا جا سکتا ہے۔اس کے برعکس اگر زمین نقطہ ٹ پر ہو تب نقطہ ن زمین یعنی صفر پر ہے جبکہ زمین نقطہ ث پر ہونے کی صورت میں نقطہ ن منفی چار پر ہو گا۔آپ دیکھ سکتے ہیں کہ نقطہ ن کی حتمی اونچائی کوئی معنی نہیں رکھتی۔اونچائی صرف اس صورت میں معنی خیز ہوتی ہے جب نقطہ حوالہ بھی بیان کیا جائے۔
\begin{figure}
\centering
\includegraphics{figBasicVoltageHeight}
\caption{برقی دباو میں نقطہ حوالہ کی اہمیت۔}
\label{شکل_بنیادی_دباو_اور_اونچائی}
\end{figure}
برقی دباو بھی بالکل اونچائی کی طرح ناپی جاتی ہے۔یوں شکل \حوالہ{شکل_بنیادی_دباو_اور_اونچائی}-ب میں نقطہ ت کے حوالے سے نقطہ ٹ مثبت دو وولٹ \عددی{\SI{2}{\volt}} پر ہے جبکہ نقطہ ث کے حوالے سے نقطہ ٹ منفی پانچ وولٹ \عددی{\SI{-5}{\volt}} پر ہے۔اسی طرح نقطہ ٹ کے حوالے سے نقطہ ت \عددی{\SI{-2}{\volt}} پر اور نقطہ ث \عددی{\SI{5}{\volt}} پر ہیں۔نقطہ ت کے حوالے سے نقطہ ث \عددی{\SI{7}{\volt}} پر ہے جبکہ نقطہ ث کے حوالے سے نقطہ ت \عددی{\SI{-7}{\volt}} پر ہے۔یاد رہے کہ نقطہ حوالہ کا برقی دباو صفر تصور کیا جاتا ہے۔

برقی دباو کی قیمت بھی بیان کرتے ہوئے ضروری ہے کہ نقطہ حوالہ بیان کیا جائے۔برقی دور میں دباو کی نشاندہی کرتے ہوئے نقطہ حوالہ کو منفی کی علامت \عددی{(-)} سے ظاہر کیا جاتا ہے جبکہ مطلوبہ نقطے کو مثبت علامت \عددی{(+)} سے ظاہر کیا جاتا ہے۔شکل \حوالہ{شکل_بنیادی_دباو_کا_اظہار}-الف میں یوں نچلی تار نقطہ حوالہ ہے۔یوں اگر \عددی{V_1=\SI{4}{\volt}} ہو تب نچلی تار کی نسبت سے بالائی تار مثبت چار وولٹ پر ہو گا۔اسی طرح \عددی{V_1=\SI{-7}{\volt}} کی صورت میں نچلی تار کی نسبت سے بالائی تار منفی سات وولٹ پر ہو گا جس کا مطلب ہے کہ بالائی تار کو حوالہ لیتے ہوئے نچلی تار کا برقی دباو مثبت سات وولٹ ہو گی۔شکل  \حوالہ{شکل_بنیادی_دباو_کا_اظہار}-ب میں نچلی تار کو \عددی{a} نام دیا گیا ہے جبکہ بالائی تار کو \عددی{b} کہا گیا ہے۔اس صورت میں نچلی تار کے حوالے سے بالائی تار کے دباو کو \عددی{V_{ba}} لکھا جاتا ہے۔یوں اگر \عددی{V_{ba}} کی قیمت منفی ہو تب بالائی تار کے حوالے سے نچلی تار پر مثبت دباو ہو گا۔برقی دور میں عموماً کسی ایک نقطے کو \اصطلاح{برقی زمین}\فرہنگ{زمین!برقی}\حاشیہب{electrical ground}\فرہنگ{ground!electrical} چننا جاتا ہے۔یوں مختلف مقامات کے دباو بیان کرتے ہوئے ہر مرتبہ برقی زمین کی نشاندہی کرنا ضروری نہیں ہوتا۔شکل \حوالہ{شکل_بنیادی_دباو_کا_اظہار}-پ میں برقی زمین کی علامت استعمال کی گئی ہے۔برقی زمین کا برقی دباو صفر کے برابر لی جاتی ہے۔جب کسی نقطے کے دباو کو برقی زمین کی نسبت سے ناپا جائے تب نقطہ حوالے کا ذکر کرنا ضروری نہیں ہوتا۔یوں اس شکل میں بالائی تار کا برقی دباو \عددی{V_b=\SI{10}{\volt}} لکھی جا سکتی ہے  جہاں زیر نوشت میں نقطہ حوالہ کا ذکر نہیں کیا گیا۔شکل-پ میں اب بھی \عددی{V_{ba}=\SI{10}{\volt}} یا \عددی{V_{ab}=\SI{-10}{\volt}} لکھا جا سکتا ہے۔
\begin{figure}
\centering
\includegraphics{figBasicVoltagePositiveNegativeSides}
\caption{برقی دباو کا اظہار۔}
\label{شکل_بنیادی_دباو_کا_اظہار}
\end{figure}
%
%=========================
\حصہ{قانونِ اوہم}
\اصطلاح{قانونِ اوہم}\فرہنگ{قانون!اوہم}\فرہنگ{اوہم!قانون}\حاشیہب{Ohm's law}\فرہنگ{Ohm's law} سے آپ بخوبی واقف ہیں
\begin{align}
V=I R
\end{align}
جو مزاحمت کی برقی رو اور مزاحمت کا برقی دباو کا تعلق بیان کرتا ہے۔اس قانون\حاشیہد{یہ قانون جرمنی کے جارج سائمن اوہم نے پیش کیا۔} کے استعمال میں دباو \عددی{V} اور رو \عددی{I} کو انفعالی سمت کی ترکیب سے چننا جاتا ہے۔شکل \حوالہ{شکل_بنیادی_قانون_اوہم_اور_غیر_عامل_ترکیب} میں ایک عدد مزاحمت اور دو عدد منبع دباو کا دور دکھایا گیا ہے۔برقی زمین کے حوالے سے مزاحمت کے بائیں سرے پر \عددی{\SI{5}{\volt}} اور دائیں سرے پر \عددی{\SI{9}{\volt}} دباو پایا جاتا ہے۔قانون اوہم میں مزاحمت کے دو سروں کے مابین برقی دباو استعمال کیا جاتا ہے۔یوں مزاحمت کے ایک سرے کو \اصطلاح{حوالہ} لیتے ہوئے مزاحمت کے دوسرے سرے پر برقی دباو لی جاتی ہے۔شکل-الف میں مزاحمت کا بایاں سرا بطور حوالہ چننا گیا ہے جبکہ مزاحمت کے دائیں سرے پر برقی دباو استعمال کی جائے گی۔یہ حقیقت مزاحمت کے قریب \عددی{V_R} کے بائیں جانب \عددی{(-)} کی علامت اور دائیں جانب \عددی{(+)}  کی علامت سے ظاہر کی جاتی ہے۔یوں انفعالی سمت کی ترکیب کے تحت برقی رو کی سمت دائیں سے بائیں جانب چننی جائے گی۔شکل-الف میں یوں
\begin{align*}
V_R=9-5=\SI{4}{\volt}
\end{align*}  
ہو گا جسے اوہم کے قانون میں استعمال کرتے ہوئے
\begin{align*}
I_R=\frac{V_R}{R}=\frac{4}{8}=\SI{0.5}{\ampere}
\end{align*}
حاصل ہوتا ہے۔حاصل برقی رو کی قیمت مثبت مقدار ہے جس کا مطلب ہے  کہ رو کی سمت وہی ہے جو شکل-الف میں چننی گئی ہے۔


\begin{figure}
\centering
\includegraphics{figBasicOhmLawExamples}
\caption{قانونِ اوہم اور انفعالی سمت کی ترکیب۔}
\label{شکل_بنیادی_قانون_اوہم_اور_غیر_عامل_ترکیب}
\end{figure}

شکل \حوالہ{شکل_بنیادی_قانون_اوہم_اور_غیر_عامل_ترکیب}-ب میں مزاحمت کا دایاں سرا بطور نقطہ حوالہ چننا گیا ہے۔یوں \عددی{V_R} کے دائیں جانب \عددی{(-)} کی علامت لگائی گئی ہے۔انفعالی سمت کی ترکیب کے تحت رو کی سمت بائیں سے دائیں کو چننی گئی ہے۔یہاں
\begin{align*}
V_R=5-9=\SI{-4}{\volt}
\end{align*}
کے برابر ہے جسے اوہم کے قانون میں استعمال کرتے ہوئے
\begin{align*}
I_R=\frac{-4}{8}=\SI{-0.5}{\ampere}
\end{align*}
حاصل ہوتا ہے۔شکل-ب میں \عددی{V_R} کی قیمت منفی حاصل ہوئی جس کا مطلب ہے کہ حقیقت میں مزاحمت پر برقی دباو چننی گئی سمت کے الٹ ہے۔اسی طرح رو \عددی{I_R} کی قیمت بھی منفی حاصل ہوئی ہے جس کا مطلب ہے کہ حقیقت میں رو چننی گئی سمت کے الٹ ہے یعنی برقی رو حقیقت میں دائیں سے بائیں جانب کو ہے۔

شکل \حوالہ{شکل_بنیادی_قانون_اوہم_صحیح_استعمال} میں قانون اوہم کا صحیح استعمال دکھایا گیا ہے۔



\begin{figure}
\centering
\includegraphics{figBasicOhmLawGeneral}
\caption{قانونِ اوہم کا صحیح استعمال۔ }
\label{شکل_بنیادی_قانون_اوہم_صحیح_استعمال}
\end{figure}

%===================

\حصہ{توانائی اور طاقت}
\اصطلاح{ثقلی میدان}\فرہنگ{ثقلی میدان}\فرہنگ{میدان!ثقلی}\حاشیہب{gravitational field}\فرہنگ{gravitational field} میں میکانی بار \عددی{m} پر قوت \عددی{F=m g} عمل کرتا ہے جہاں \عددی{g=\SI{9.8}{\meter\per\second\squared}} کے برابر ہے۔یوں ثقلی میدان کے مخالف \عددی{m} کو \عددی{h} بلندی تک پہنچانے کی خاطر \عددی{w=Fh=mgh} توانائی درکار ہے۔بالکل اسی طرح \اصطلاح{برقی میدان}\فرہنگ{برقی میدان}\فرہنگ{میدان!ثقلی}\حاشیہب{electric field}\فرہنگ{electric field} \عددی{E} میں برقی بار \عددی{q} پر \عددی{F=qE} قوت عمل کرتی ہے اور برقی میدان کے مخالف \عددی{h} فاصلے تک بار کو منتقل کرنے کی خاطر 
\begin{align}\label{مساوات_بنیادی_توانائی_دباو}
w=q E h
\end{align}
توانائی درکار ہے۔برقی میدان میں ابتدائی نقطے سے اختتامی نقطے تک اکائی برقی بار منتقل کرنے کے لئے درکار توانائی کو ابتدائی نقطے کے حوالے سے اختتامی نقطے کا برقی دباو کہا جاتا ہے۔
%========================
\ابتدا{مثال}
برقی میدان \عددی{E=\SI{600}{\volt\per\meter}} میں \عددی{\SI{0.2}{\coulomb}} بار قوت کے مخالف \عددی{\SI{12}{\milli\meter}} فاصلہ دُور منتقل کیا جاتا ہے۔درکار توانائی حاصل کریں۔ابتدائی نقطہ \عددی{i} اور اختتامی نقطہ \عددی{k} کے مابین برقی دباو حاصل کریں۔

حل:درکار توانائی
\begin{align*}
w=0.2 \times 600 \times 0.012=\SI{1.44}{\joule}
\end{align*}
کے برابر ہے جبکہ برقی دباو
\begin{align*}
V_{ki}=\frac{1.44}{0.2}=\SI{7.2}{\volt}
\end{align*}
کے برابر ہے۔
\انتہا{مثال}
%=================

مساوات \حوالہ{مساوات_بنیادی_توانائی_دباو} کی تفرقی صورت 
\begin{align*}
\dif w= E h \dif q
\end{align*}
لکھی جا سکتی ہے جو چھوٹے برقی بار \عددی{\dif q} کو منتقل کرنے کے لئے درکار توانائی \عددی{\dif w} دیتی ہے۔یوں اکائی بار کو منتقل کرنے کی خاطر \عددی{\tfrac{\dif w}{\dif q}} توانائی درکار ہو گی جسے برقی دباو \عددی{v} کہتے ہیں یعنی
\begin{align}\label{مساوات_بنیادی_دباو_تعریف}
v=\frac{\dif w}{\dif q}
\end{align}
لکھی جا سکتی ہے۔

مساوات \حوالہ{مساوات_بنیادی_دباو_تعریف} کو مساوات \حوالہ{مساوات_بنیادی_رو_تعریف} سے ضرب دینے سے
\begin{align}\label{مساوات_بنیادی_طاقت_مساوی_دباو_ضرب_رو}
v \times i = \frac{\dif w}{\dif q} \times \frac{\dif q}{\dif t}=\frac{\dif w}{\dif t}=p
\end{align}
حاصل ہوتا ہے جو \اصطلاح{طاقت}\فرہنگ{طاقت}\حاشیہب{power}\فرہنگ{power} \عددی{p} کو ظاہر کرتا ہے۔فی سیکنڈ درکار توانائی کو طاقت کہتے ہیں۔طاقت کی اکائی \اصطلاح{واٹ}\فرہنگ{واٹ}\حاشیہب{watt}\فرہنگ{watt} \عددی{\si{\watt}} ہے۔مندرجہ بالا مساوات کی تکملہ صورت درج ذیل ہے۔
\begin{align}
w=\int_{t_1}^{t_2} p \dif t=\int_{t_1}^{t_2} v i\dif t
\end{align}

آئیں ان معلومات کو مد نظر رکھتے ہوئے شکل \حوالہ{شکل_بنیادی_اوہم_قانون} پر غور کریں جہاں \عددی{\SI{10}{\volt}} کی \اصطلاح{منبع برقی دباو}\فرہنگ{منبع!برقی دباو}\حاشیہب{voltage source}\فرہنگ{source!voltage}\فرہنگ{voltage source} کے ساتھ \عددی{\SI{5}{\ohm}} کی \اصطلاح{برقی مزاحمت}\فرہنگ{برقی مزاحمت}\حاشیہب{electrical resistance}\فرہنگ{resistance} جوڑی گئی ہے۔اس دور میں برقی رو کو منبع پیدا کرتی ہے لہٰذا منبع کو \اصطلاح{فعال پرزہ}\فرہنگ{فعال پرزہ}\فرہنگ{پرزہ!عامل}\حاشیہب{active component}\فرہنگ{active component} جبکہ مزاحمت کو \اصطلاح{انفعال پرزہ}\فرہنگ{انفعال پرزہ}\فرہنگ{پرزہ!انفعال}\حاشیہب{passive component}\فرہنگ{passive component} کہا جاتا ہے۔\اصطلاح{انفعالی سمت کی ترکیب} کا نام اسی حقیقت سے نکلا ہے کہ اس ترکیب کے استعمال سے انفعالی پرزہ جات پر مثبت طاقت حاصل ہوتی ہے۔

\اصطلاح{قانون اوہم}\فرہنگ{قانون اوہم}\فرہنگ{اوہم!قانون}\حاشیہب{Ohm's law}\فرہنگ{Ohm!law} کے تحت شکل \حوالہ{شکل_بنیادی_اوہم_قانون}  کے دور میں \اصطلاح{سمتِ گھڑی}\فرہنگ{سمت گھڑی}\حاشیہب{clockwise}\فرہنگ{clockwise} \عددی{\SI{2}{\ampere}} کی برقی رو پائی جائے گی جسے دور میں بالائی تار پر تیر کے نشان سے دکھایا گیا ہے۔دور میں \عددی{\SI{2}{\ampere}} برقی رو سے مراد یہ ہے کہ دور میں کسی بھی نقطے پر اگر دیکھا جائے تو اس نقطے سے فی سیکنڈ \عددی{\SI{2}{\coulomb}} بار گزرے گا۔ اس دور میں نچلی تار کے حوالے سے بالائی تار پر مثبت دس وولٹ کا دباو ہے۔یوں مزاحمت کے بالائی یعنی مثبت سرے سے  مزاحمت کے نچلے یعنی منفی سرے کی جانب فی سیکنڈ دو کولمب بار منتقل ہوتا ہے۔یہ بالکل ایسا ہی ہے جیسے ثقلی میدان میں بلند مقام سے میکانی بار گِر رہا ہو۔دو کولمب کا بار دس وولٹ نیچے گرتے ہوئے \عددی{\SI{20}{\joule}} کی \اصطلاح{مخفی توانائی}\فرہنگ{مخفی توانائی}\فرہنگ{توانائی!مخفی}\حاشیہب{potential energy}\فرہنگ{potential energy} کھوئے\حاشیہد{مخفی توانائی کی اصطلاح خفیہ توانائی سے حاصل کی گئی ہے۔} گا جو \اصطلاح{حرارتی توانائی}\فرہنگ{حرارتی توانائی}\فرہنگ{توانائی!حرارتی}\حاشیہب{thermal energy}\فرہنگ{thermal energy} میں تبدیل ہو کر مزاحمت کو گرم کرے گی۔ہم کہتے ہیں کہ مزاحمت میں فی سیکنڈ توانائی کا \اصطلاح{ضیاع}\فرہنگ{ضیاع}\حاشیہب{loss}\فرہنگ{loss} \عددی{\SI{20}{\joule}} ہے یا کہ مزاحمت میں \اصطلاح{طاقتی ضیاع}\فرہنگ{طاقتی ضیاع}\حاشیہب{power loss}\فرہنگ{power loss} \عددی{\SI{20}{\watt}} ہے۔مزاحمت میں طاقت کے ضیاع کو \اصطلاح{حرارتی ضیاع}\فرہنگ{حرارتی ضیاع}\فرہنگ{ضیاع!حرارتی}\حاشیہب{thermal loss}\فرہنگ{thermal loss} اور \اصطلاح{مزاحمتی ضیاع}\فرہنگ{مزاحمتی ضیاع}\حاشیہب{resistive loss}\فرہنگ{resistive loss} بھی کہتے ہیں۔
\begin{figure}
\centering
\includegraphics[width=\textwidth]{figBasicOhmsLaw}
\caption{طاقت کی پیداوار اور طاقت کا ضیاع۔}
\label{شکل_بنیادی_اوہم_قانون}
\end{figure}

انفعالی سمت کی ترکیب استعمال کرتے ہوئے ہم شکل \حوالہ{شکل_بنیادی_اوہم_قانون}-الف میں منبع کے دباو کو \عددی{V_M} اور مزاحمت کے دباو کو \عددی{V_الفR} چننے کے بعد ان دباو کے مثبت سر سے منفی سر کی جانب رو کی سمت چنتے ہیں۔یوں حاصل منبع کی برقی رو \عددی{I_M} اور مزاحمت کی برقی رو \عددی{I_R} کو شکل-الف میں دکھایا گیا ہے۔شکل- کو دیکھتے ہوئے درج ذیل لکھا جا سکتا ہے۔
\begin{align*}
V_M&=\SI{10}{\volt}\\
V_R&=\SI{10}{\volt}\\
I_M&=\SI{-2}{\ampere}\\
I_R&=\SI{2}{\ampere}
\end{align*} 
ان قیمتوں کو مساوات \حوالہ{مساوات_بنیادی_طاقت_مساوی_دباو_ضرب_رو} میں پر کرتے ہوئے منبع اور مزاحمت کی طاقت حاصل کرتے ہیں۔
\begin{align*}
P_M&=10 \times (-2)=\SI{-20}{\watt} \quad \quad \text{\RL{طاقت کی منفی قیمت، طاقت کی پیداوار کو ظاہر کرتی ہے}}\\
P_R&=10 \times 2 =\SI{20}{\watt}\quad \quad\quad\quad \text{\RL{طاقت کی مثبت قیمت، طاقت کی ضیاع کو ظاہر کرتی ہے}}
\end{align*}
یہاں غیر متغیر طاقت کو بڑھے حروف تہجی میں \عددی{P_M} اور \عددی{P_R} لکھا گیا۔مزاحمت کی طاقت مثبت مقدار حاصل ہوئی ہے جبکہ منبع کی طاقت منفی مقدار ہے۔یوں مساوات \حوالہ{مساوات_بنیادی_طاقت_مساوی_دباو_ضرب_رو} سے حاصل مثبت مقدار  طاقت کے ضیاع کو ظاہر کرتی ہے جبکہ منفی مقدار طاقت کی پیدا وار کو ظاہر کرتی ہے۔

شکل \حوالہ{شکل_بنیادی_اوہم_قانون}-ب میں برقی دباو کے سمت الٹ چننے گئے جس کی وجہ سے رو کی سمتیں بھی الٹ کر دی گئی ہیں۔یوں
\begin{align*}
V_M&=\SI{-10}{\volt}\\
V_R&=\SI{-10}{\volt}\\
I_M&=\SI{2}{\ampere}\\
I_R&=\SI{-2}{\ampere}
\end{align*} 
لکھے جائیں گے جن سے  دوبارہ
\begin{align*}
P_M&=(-10) \times 2=\SI{-20}{\watt} \quad \quad\quad \text{\RL{طاقت کی منفی قیمت، طاقت کی پیداوار کو ظاہر کرتی ہے}}\\
P_R&=(-10) \times (-2) =\SI{20}{\watt}\quad \quad \text{\RL{طاقت کی مثبت قیمت، طاقت کی ضیاع کو ظاہر کرتی ہے}}
\end{align*}
حاصل ہوتے ہیں۔
%======================

\ابتدا{مثال}
شکل \حوالہ{شکل_بنیادی_فعال_انفعال_مثال} میں دو ادوار دکھائے گئے ہیں۔دریافت کریں کہ آیا بیرونی پرزہ بقایا دور کو طاقت فراہم کرتا ہے یا کہ اس سے طاقت حاصل کرتا ہے۔طاقت کی قیمت بھی دریافت کریں۔

\begin{figure}
\centering
\includegraphics{figBasicPassiveSignExample}
\caption{فعال اور انفعال پرزے کی مثال۔}
\label{شکل_بنیادی_فعال_انفعال_مثال}
\end{figure}

حل:شکل-الف میں برقی رو کی قیمت منفی لکھی گئی ہے جس کا مطلب ہے کہ حقیقت میں رو تیر کے نشان کے الٹ سمت میں ہے۔رو کی سمت الٹ تصور کرتے ہوئے ہم دیکھتے ہیں کہ بقایا دور کے مثبت سرے پر رو اندر داخل ہوتی ہے۔یوں بقایا دور انفعال ہے۔بیرونی پرزے کے مثبت سرے سے حقیقی رو خارج ہوتی ہے لہٰذا یہ فعال پرزہ ہے۔یوں بیرونی پرزہ طاقت فراہم کرتا ہے جبکہ بقایا دور میں طاقت خرچ ہوتا ہے۔یہی نتائج انفعال سمت کے ترکیب سے یوں حاصل ہوتی ہے۔بیرونی پرزے کے برقی دباو کو دیکھتے ہوئے رو کی دکھائی گئی سمت ہی استعمال کی جائے گی۔یوں بیرونی پرزے کی طاقت \عددی{P=5\times (-6)=\SI{-30}{\watt}} ہے جو طاقت کی پیداوار ہے۔بقایا دور میں رو کی انفعال سمت دکھائے گئے سمت کے الٹ ہے لہٰذا طاقت \عددی{P=5\times 6=\SI{30}{\watt}} حاصل ہوتا ہے جو طاقت کی ضیاع کو ظاہر کرتا ہے۔آپ نے دیکھا کہ بیرونی پرزہ \عددی{\SI{30}{\watt}} طاقت پیدا کرتا ہے جبکہ بقایا دور اتنی ہی طاقت استعمال کرتا ہے۔ آپ دیکھ سکتے ہیں \اصطلاح{قانونِ  بقا}\فرہنگ{قانون! بقا}\حاشیہب{law of conservation of energy}\فرہنگ{law!conservation of energy} کارآمد ہے۔کسی بھی دور میں توانائی کی پیداوار اور خرچ برابر ہوتے ہیں۔

شکل-ب میں رو نچلی تار میں دائیں سے بائیں طرف رواں ہے۔یوں بیرونی پرزے کے مثبت سرے سے رو خارج ہوتی ہے جبکہ بقایا دور کے مثبت سرے میں رو داخل ہوتی ہے۔یوں بیرونی پرزہ فعال اور بقایا دور انفعال ہے۔بیرونی پرزے کی طاقت \عددی{P=7\times (-3)=\SI{-21}{\watt}} ہے جو طاقت کی پیداوار ہے جبکہ بقایا دور کی طاقت \عددی{P=7\times 3=\SI{21}{\watt}} ہے جو طاقت کی ضیاع کو ظاہر کرتی ہے۔
\انتہا{مثال}
%===========================

\ابتدا{مشق}
شکل \حوالہ{شکل_بنیادی_فعال_انفعال_مشق} میں بیرونی پرزے کی طاقت حاصل کریں۔

\begin{figure}
\centering
\includegraphics{figBasicPassiveSignQuiz}
\caption{فعال اور انفعال پرزے کی مشق۔}
\label{شکل_بنیادی_فعال_انفعال_مشق}
\end{figure}

جوابات: (الف) \عددی{\SI{8}{\watt}}؛ (ب) \عددی{\SI{27}{\watt}}
\انتہا{مشق}
%==========================
\ابتدا{مثال}
شکل \حوالہ{شکل_بنیادی_طاقت_مثال}-الف میں برقی رو کی مقدار اور سمت حاصل کریں جبکہ شکل-ب میں برقی دباو اور اس کا مثبت سرا دریافت کریں۔
\begin{figure}
\centering
\includegraphics{figBasicPowerExample}
\caption{طاقت اور ایک متغیرہ دیا گیا ہے۔دوسرا دریافت کرنا ہے۔}
\label{شکل_بنیادی_طاقت_مثال}
\end{figure}

حل: شکل-الف میں بیرونی پرزے کی طاقت منفی ہے۔یوں بیرونی پرزہ طاقت پیدا کرتا ہے لہٰذا اس کے مثبت سرے سے رو خارج ہو گی یعنی دور میں گھڑی کے الٹ سمت میں رو پائی جائے گی۔رو کی قیمت \عددی{\SI{4}{\ampere}} ہو گی۔

شکل-ب میں بیرونی پرزے کی طاقت مثبت ہے لہٰذا اس میں طاقت کا ضیاع ہو گا اور برقی رو مثبت سرے سے پرزے میں داخل ہو گی۔دور میں گھڑی کی سمت میں منفی رو دکھائی گئی ہے لہٰذا حقیقت میں رو گھڑی کی الٹ سمت ہے۔حقیقی رو کو گھڑی کے الٹ سمت تصور کرتے ہوئے  بیرونی پرزے کا نچلا سرا مثبت ہو گا اور برقی دباو کی قیمت \عددی{\SI{2}{\volt}} ہو گی۔
\انتہا{مثال}
%==========================

\ابتدا{مشق}
شکل \حوالہ{شکل_بنیادی_طاقت_مشق} میں نا معلوم متغیرہ دریافت کریں۔ 
\begin{figure}
\centering
\includegraphics{figBasicPowerQuiz}
\caption{طاقت اور ایک متغیرہ دیا گیا ہے۔دوسرا دریافت کریں۔}
\label{شکل_بنیادی_طاقت_مشق}
\end{figure}

جوابات: (الف) گھڑی کے الٹ \عددی{\SI{3}{\ampere}}؛ (ب) بالائی تار مثبت ہے جبکہ دباو \عددی{\SI{3}{\volt}} ہے۔  
\انتہا{مشق}
%====================

آخر میں دوبارہ اس حقیقت کی نشاندہی کرتے ہیں کہ کسی بھی برقی دور میں پیداوار طاقت اور طاقت کا ضیاع برابر ہوں گے۔

\حصہ{برقی پرزے}
برقی پرزوں کو دو اقسام میں تقسیم کیا جا سکتا ہے۔وہ پرزے جو طاقت پیدا کرتے ہیں \اصطلاح{فعال پرزے}\فرہنگ{فعال پرزہ}\حاشیہب{active components}\فرہنگ{active component} کہلاتے ہیں جبکہ طاقت ضائع کرنے والے پرزوں کو \اصطلاح{انفعال پرزے}\فرہنگ{انفعال پرزہ}\حاشیہب{passive components}\فرہنگ{passive component} کہتے ہیں۔ جنریٹر اور بیٹری فعال پرزوں کی مثال ہے جبکہ مزاحمت، امالہ گیر\حاشیہب{inductor} اور برق گیر\حاشیہب{capacitor} انفعال پرزے ہیں۔

فعال پرزوں پر اس باب میں غور کیا جائے گا جبکہ انفعال پرزوں پر اگلے باب میں تفصیلاً غور کیا جائے گا۔ 

\جزوحصہ{غیر تابع منبع}
\اصطلاح{غیر تابع منبع دباو}\فرہنگ{غیر تابع منبع دباو}\فرہنگ{منبع دباو!غیر تابع}\حاشیہب{independent voltage source}\فرہنگ{independent voltage source} سے مراد ایسی منبع ہے جو، منبع میں سے گزرتی رو کے قطع نظر، اپنے دو سروں کے درمیان مخصوص برقی دباو برقرار رکھتا ہے۔ غیر تابع منبع دباو کی علامت کو شکل \حوالہ{شکل_بنیادی_آزاد_منبع_دباو} میں دکھایا گیا ہے جہاں نقطہ \عددی{A} کے حوالے سے نقطہ \عددی{B} پر \عددی{v(t)} برقی دباو برقرار رہتا ہے۔شکل میں غیر تابع منبع دباو کا دباو بالمقابل رو  \عددی{v-i} خط بھی دکھایا گیا ہے۔اس خط کے مطابق برقی دباو کی قیمت پر برقی رو کا کوئی اثر نہیں پایا جاتا۔

\begin{figure}
\centering
\includegraphics{figBasicIndependentVoltageSource}
\caption{غیر تابع منبع دباو اور اس کا \عددی{v-i} خط۔}
\label{شکل_بنیادی_آزاد_منبع_دباو}
\end{figure}

شکل \حوالہ{شکل_بنیادی_آزاد_منبع_رو} میں \اصطلاح{غیر تابع منبع رو}\فرہنگ{آزاد!منبع رو}\فرہنگ{منبع رو!غیر تابع}\حاشیہب{independent current source}\فرہنگ{independent!current source} کی علامت اور رو بالمقابل دباو \عددی{v-i} خط دکھایا گیا ہے۔غیر تابع منبع رو سے مراد ایسی منبع ہے جو، منبع پر دباو کے قطع نظر،  مخصوص برقی رو برقرار رکھتا ہے۔غیر تابع منبع رو کے دباو بالمقابل رو خط کے تحت منبع پر برقی دباو کے تبدیلی کا منبع کی رو پر کوئی اثر نہیں پایا جاتا۔منبع رو میں مثبت رو کی سمت کو تیر کے نشان سے دکھایا جاتا ہے۔
\begin{figure}
\centering
\includegraphics{figBasicIndependentCurrentSource}
\caption{غیر تابع منبع رو اور اس کا \عددی{v-i} خط۔}
\label{شکل_بنیادی_آزاد_منبع_رو}
\end{figure}

عام استعمال میں منبع بقایا دور کو طاقت فراہم کرتی ہے۔شکل \حوالہ{شکل_بنیادی_طاقت_مشق}-ب میں اگر بیرونی پرزہ منبع ہو تب آپ دیکھ سکتے ہیں کہ منبع کو بھی طاقت فراہم کی جا سکتی ہے۔

منبع محدود صلاحیت کا حامل ہے۔اگرچہ ہم توقع کرتے ہیں کہ منبع دباو کسی بھی قیمت کی برقی رو فراہم کرتے ہوئے پیدا کردہ  برقی دباو برقرار رکھے گا، حقیقت میں کوئی بھی منبع کسی محدود رو کی حد تک ایسا کر پاتا ہے۔

%==============================
\ابتدا{مثال}
شکل \حوالہ{شکل_بنیادی_طاقت_حساب}-الف میں تینوں پرزوں کی طاقت دریافت کریں۔ (اشارہ: سلسلہ وار جڑے پرزوں میں یکساں رو پائی جاتی ہے۔)
\begin{figure}
\centering
\begin{subfigure}{0.5\textwidth}
\centering
\includegraphics{figBasicInternalResistanceExample}
\caption{}
\end{subfigure}%
%
\begin{subfigure}{0.5\textwidth}
\centering
\includegraphics{figBasicInternalResistanceQuiz}
\caption{}
\end{subfigure}%
\caption{طاقت کا حساب۔}
\label{شکل_بنیادی_طاقت_حساب}
\end{figure}

حل:منبع کے مثبت سر سے رو خارج ہو رہی ہے لہٰذا یہ پرزہ طاقت فراہم کر رہا ہے جبکہ بقایا دو پرزوں کے مثبت سر سے رو پرزے میں داخل ہوتی ہے لہٰذا ان دونوں پرزوں میں طاقت ضائع ہوتا ہے۔منبع کی طاقت \عددی{{12\times(-4)=\SI{-48}{\watt}}} ہے جبکہ پرزہ-1 کی طاقت \عددی{5\times 4 =\SI{20}{\watt}} اور پرزہ-2 کی طاقت \عددی{7\times 4=\SI{28}{\watt}} ہے۔آپ دیکھ سکتے ہیں کہ طاقت کی ضیاع \عددی{\SI{20}{\watt}+\SI{28}{\watt}=\SI{48}{\watt}} عین طاقت کی پیداوار کے برابر ہے۔
\انتہا{مثال}
%==========================
\ابتدا{مشق}
شکل \حوالہ{شکل_بنیادی_طاقت_حساب}-ب میں تینوں پرزوں کی طاقت حاصل کریں۔

جوابات:منبع رو کی طاقت \عددی{\SI{-16}{\watt}} ہے۔پرزہ-1 کی طاقت \عددی{\SI{20}{\watt}} ہے۔پرزہ-2 بھی منبع ہے اور اس کی طاقت \عددی{\SI{-4}{\watt}} ہے۔
\انتہا{مشق}
%==========================

\جزوحصہ{تابع منبع}
غیر تابع منبع دباو کی پیدا کردہ دباو کا انحصار منبع سے گزرتی رو پر بالکل نہیں ہوتا۔ اسی طرح غیر تابع منبع رو کی پیدا کردہ رو کا انحصار منبع پر دباو پر بالکل نہیں ہوتا۔اس کے برعکس \اصطلاح{تابع منبع دباو}\فرہنگ{تابع!منبع دباو}\فرہنگ{منبع دباو!تابع}\فرہنگ{دباو!تابع منبع}\حاشیہب{dependent voltage source}\فرہنگ{dependent voltage source} کی پیدا کردہ دباو،  دور میں کسی مخصوص مقام کی رو یا دباو پر منحصر ہوتا ہے۔اسی طرح \اصطلاح{تابع منبع رو}\فرہنگ{تابع!منبع رو}\فرہنگ{منبع رو!تابع}\فرہنگ{رو!تابع منبع}\حاشیہب{dependent current source}\فرہنگ{dependent current source} کی پیدا کردہ رو،  دور میں کسی مخصوص مقام کی رو یا دباو پر منحصر ہوتا ہے۔تابع منبع برقیات کی میدان میں کلیدی کردار ادا کرتے ہیں جہاں برقیاتی پرزہ جات مثلاً \اصطلاح{دو جوڑ ٹرانزسٹر}\فرہنگ{ٹرانزسٹر!دو جوڑ}\حاشیہب{bipolar transistor, BJT}\فرہنگ{transistor, BJT} یا \اصطلاح{میدانی ٹرانزسٹر}\فرہنگ{ٹرانزسٹر!میدانی}\حاشیہب{MOSFET}\فرہنگ{MOSFET} کے  \اصطلاح{ریاضی نمونے}\فرہنگ{ریاضی نمومے}\فرہنگ{نمونہ!ریاضی}\حاشیہب{mathematical model}\فرہنگ{model} تابع منبع سے بنائے جاتے ہیں۔متعدد ٹرانزسٹر پر مبنی برقیاتی ادوار کا حسابی حل انہیں ریاضی نمونوں کی مدد سے حاصل کیا جاتا ہے۔

\begin{figure}
\centering
\begin{subfigure}{0.5\textwidth}
\centering
\includegraphics[scale=0.9]{figBasicDependentVoltageSource}
\caption*{(الف) تابع منبع دباو}
\end{subfigure}%
\begin{subfigure}{0.5\textwidth}
\centering
\includegraphics[scale=0.9]{figBasicDependentCurrentSource}
\caption*{(ب) تابع منبع رو}
\end{subfigure}
%
\begin{subfigure}{0.5\textwidth}
\centering
\includegraphics[scale=0.9]{figBasicDependentResistanceSource}
\caption*{(پ) تابع منبع مزاحمت-نما}
\end{subfigure}%
\begin{subfigure}{0.5\textwidth}
\centering
\includegraphics[scale=0.9]{figBasicDependentConductanceSource}
\caption*{(ت) تابع منبع موصلیت-نما}
\end{subfigure}%
\caption{تابع منبع کے چار اقسام۔}
\label{شکل_بنیادی_تابع_منبع_اقسام}
\end{figure}

غیر تابع منبع کو گول دائرے سے ظاہر کیا جاتا ہے جبکہ تابع منبع کو ہیرا شکل سے ظاہر کیا جاتا ہے۔شکل \حوالہ{شکل_بنیادی_تابع_منبع_اقسام} میں چار اقسام کے تابع منبع دکھائے گئے ہیں۔شکل-الف میں \اصطلاح{تابع منبع دباو}\فرہنگ{تابع!منبع دباو}\حاشیہب{dependent voltage source}\فرہنگ{dependent!voltage source} کی پیدا کردہ دباو کا انحصار بائیں جانب  کے دباو \عددی{v_S} پر ہے۔ یوں \عددی{v_S} \اصطلاح{ضابط دباو}\فرہنگ{ضابط!دباو}\فرہنگ{دباو!ضابط}\حاشیہب{control voltage}\فرہنگ{control voltage} کہلاتا ہے۔یہ منبع \عددی{\mu v_S} دباو پیدا کرتا ہے۔ شکل-ب میں \اصطلاح{تابع منبع رو}\فرہنگ{تابع!منبع رو}\حاشیہب{depended current source}\فرہنگ{depended!current source} کو \عددی{i_S} قابو کرتا ہے۔ان دو اقسام کے منبع کے مستقل \عددی{\mu} اور \عددی{\beta} \اصطلاح{بے بُعد}\فرہنگ{بے بعد}\حاشیہب{dimensionless}\فرہنگ{dimensionless} مقدار ہیں۔شکل-پ میں \عددی{i_s} رو پیدا کردہ دباو کو قابو کرتی ہے۔اس منبع کے مستقل \عددی{r} کا \اصطلاح{بُعد}\فرہنگ{بُعد}\حاشیہب{dimension}\فرہنگ{dimension} \عددی{\si{\volt\per\ampere}} ہے جو عین مزاحمت کی بُعد ہے۔اسی لئے اس منبع کو \اصطلاح{تابع منبع مزاحمت-نما}\فرہنگ{تابع منبع مزاحمت-نما}\حاشیہب{dependent transresistance source}\فرہنگ{dependent!transresistance source} کہا جاتا ہے۔شکل-ت میں \اصطلاح{تابع منبع موصلیت-نما}\فرہنگ{تابع منبع موصلیت-نما}\حاشیہب{dependent transconductance source}\فرہنگ{dependent!transconductance source} کی پیدا کردہ رو کا انحصار \عددی{v_S} پر ہے۔اس منبع کے مستقل \عددی{g} کا بُعد \عددی{\si{\ampere\per\volt}} ہے جو موصلیت کی بھی بُعد ہے۔
%=======================
\ابتدا{مثال}
شکل \حوالہ{شکل_تابع_منبع_دباو_اور_رو_استعمال}-الف میں خارجی دباو اور شکل-ب میں خارجی رو دریافت کریں۔

\begin{figure}
\centering
\begin{subfigure}{0.5\textwidth}
\centering
\includegraphics[scale=0.85]{figBasicDependentVoltageSourceExample}
\caption*{(الف) تابع منبع دباو کی مثال}
\end{subfigure}%
%
\begin{subfigure}{0.5\textwidth}
\centering
\includegraphics[scale=0.9]{figBasicDependentCurrentSourceExample}
\caption*{(ب) تابع منبع رو کی مثال}
\end{subfigure}%
\caption{تابع منبع دباو اور تابع منبع رو کے استعمال کی مثال۔}
\label{شکل_تابع_منبع_دباو_اور_رو_استعمال}
\end{figure} 

حل:شکل-الف میں ضابط دباو \عددی{\SI{0.2}{\volt}} اور منبع کا مستقل \عددی{7} ہے۔یوں پیدا کردہ دباو \عددی{0.2\times 7=\SI{1.4}{\volt}} ہو گا۔شکل-ب میں ضابط رو \عددی{\SI{3}{\milli\ampere}} اور منبع کا مستقل \عددی{12} ہے۔یوں پیدا کردہ رو \عددی{0.003\times 12=\SI{36}{\milli\ampere}} ہو گی۔
\انتہا{مثال}
%====================


اس مثال میں تابع منبع دباو داخلی دباو کو \عددی{7} گنا بڑھاتا ہے گویا منبع بطور \اصطلاح{ایمپلیفائر دباو}\فرہنگ{ایمپلیفائر!دباو}\حاشیہب{voltage amplifier}\فرہنگ{amplifier!voltage} کردار ادا کرتا ہے اور اس ایمپلیفائر کی \اصطلاح{افزائش دباو}\فرہنگ{افزائش!دباو}\حاشیہب{voltage gain}\فرہنگ{voltage gain} \عددی{7} ہے۔اسی طرح شکل-ب میں تابع منبع رو نے داخلی رو کو \عددی{12} گنا بڑھا کر خارج کیا، گویا یہ منبع بطور \اصطلاح{ایمپلیفائر رو}\فرہنگ{ایمپلیفائر!رو}\حاشیہب{current amplifier}\فرہنگ{amplifier!current} کردار ادا کرتا ہے اور اس ایمپلیفائر کی \اصطلاح{افزائش رو}\فرہنگ{افزائش!رو}\حاشیہب{current gain}\فرہنگ{current gain} کی قیمت \عددی{12} ہے۔

شکل \حوالہ{شکل_بنیادی_تابع_منبع_اقسام}-پ بالکل اسی طرح داخلی ضابط رو کی نسبت سے برقی دباو خارج کرتے ہوئے بطور \اصطلاح{ایمپلیفائر مزاحمت-نما}\فرہنگ{ایمپلیفائر!مزاحمت-نما}\حاشیہب{transresistance amplifier}\فرہنگ{amplifier!transresistance} کردار ادا کرتا ہے جہاں منبع کا مستقل \اصطلاح{افزائش مزاحمت-نما}\فرہنگ{افزائش!مزاحمت-نما}\حاشیہب{transresistance gain}\فرہنگ{transresistance gian}  کہلاتا ہے۔شکل \حوالہ{شکل_بنیادی_تابع_منبع_اقسام}-ت بطور \اصطلاح{ایمپلیفائر موصلیت-نما}\فرہنگ{ایمپلیفائر!موصلیت-نما}\حاشیہب{transconductance amplifier}\فرہنگ{amplifier!transconductance} کام کرتا ہے اور اس کے مستقل کو \اصطلاح{افزائش موصلیت-نما}\فرہنگ{افزائش!موصلیت-نما}\حاشیہب{transconductance gain}\فرہنگ{transconductance gain} کہتے ہیں۔

%=================

\ابتدا{مشق}
شکل \حوالہ{شکل_تابع_منبع_دباو_اور_رو_مشق} میں برقی بوجھ کی طاقت دریافت کریں۔
\begin{figure}
\centering
\begin{subfigure}{0.5\textwidth}
\centering
\includegraphics[scale=0.8]{figBasicDependentVoltageSourceExamplePower}
\caption*{(الف) تابع منبع دباو کی مشق}
\end{subfigure}%
%
\begin{subfigure}{0.5\textwidth}
\centering
\includegraphics[scale=0.8]{figBasicDependentCurrentSourceExamplePower}
\caption*{(ب) تابع منبع رو کی مشق}
\end{subfigure}%
\caption{تابع منبع دباو اور تابع منبع رو کے استعمال کی مشق۔}
\label{شکل_تابع_منبع_دباو_اور_رو_مشق}
\end{figure} 

جوابات: (الف): \عددی{\SI{69.3}{\watt}}، (ب) \عددی{\SI{120}{\watt}} 

\انتہا{مشق}
%====================
\ابتدا{مثال}\شناخت{مثال_بنیادی_طاقت_کا_حساب}
شکل \حوالہ{شکل_بنیادی_طاقت_حساب_مثال} میں تمام پرزہ جات کی طاقت دریافت کریں۔

\begin{figure}
\centering
\includegraphics{figBasicExampleTwoSourceThreeResistorPower}
\caption{مثال \حوالہ{مثال_بنیادی_طاقت_کا_حساب} کا دور۔}
\label{شکل_بنیادی_طاقت_حساب_مثال}
\end{figure}

حل: بوجھ-الف میں برقی رو صفر ہے اور اس کے دونوں سروں کے مابین دباو بھی صفر ہے لہٰذا اس کی طاقت \عددی{0\times 0 =\SI{0}{\watt}} ہے۔ بوجھ-ب کی  طاقت \عددی{5\times 1.25=\SI{6.25}{\watt}} ہے۔ بوجھ-پ کی طاقت \عددی{5\times 1=\SI{5}{\watt}} اور بوجھ-ت کی طاقت \عددی{5\times 1.25=\SI{6.25}{\watt}} ہے۔بائیں منبع کی طاقت \عددی{5\times1=\SI{5}{\watt}} جبکہ دائیں منبع کی طاقت \عددی{10\times(-2.25)=\SI{-22.5}{\watt}} ہے۔

کل طاقت کا ضیاع  \عددی{0+6.25+5+6.25+5=\SI{22.5}{\watt}} ہے۔دایاں منبع تمام طاقت پیدا کرتا ہے جبکہ بائیں منبع  کو ازخود طاقت درکار ہے۔ 
\انتہا{مثال}
%======================
\ابتدا{مشق}
شکل \حوالہ{شکل_بنیادی_طاقت_حصول_مشق} کے تمام پرزوں میں طاقت حاصل کریں۔کیا طاقت کی پیدا وار اور اس کا ضیاع برابر ہیں۔

\begin{figure}
\centering
\includegraphics{figBasicPowerExampleMultiSupplies}
\caption{طاقت کے حصول کی مشق۔}
\label{شکل_بنیادی_طاقت_حصول_مشق}
\end{figure}

جوابات:بالترتیب الف تا ٹ: \عددی{\SI{1.5125}{\watt}}، \عددی{\SI{4.5375}{\watt}}، \عددی{\SI{4.05}{\watt}}، \عددی{\SI{3.6}{\watt}}، \عددی{\SI{1.6}{\watt}}؛ منبع دباو کی طاقت \عددی{\SI{-0.3}{\watt}} اور منبع رو کی طاقت \عددی{\SI{-15}{\watt}} ہے۔دور میں کل طاقت کی پیداوار \عددی{\SI{15.3}{\watt}} ہے۔اتنی ہی طاقت پیدا بھی ہوتی ہے لہٰذا دونوں برابر ہیں۔
\انتہا{مشق}
%=======================
\ابتدا{مثال}\شناخت{مثال_بنیادی_بار_اور_رو}
شکل \حوالہ{شکل_بنیادی_بار_بالمقابل_وقت}-الف میں ڈبہ دور دکھایا گیا ہے جس میں برقی بار بھری جا رہی ہے۔برقی بار بالمقابل وقت کا خط شکل-ب میں دیا گیا ہے۔اس خط سے برقی رو بالمقابل وقت کا خط حاصل کریں۔

\begin{figure}
\centering
\begin{subfigure}{1\textwidth}
\centering
\includegraphics{figBasicChargeSuppliedToBox}
\caption{ڈبہ دور}
\end{subfigure}
%
\begin{subfigure}{1\textwidth}
\centering
\includegraphics{figBasicChargeVersusCurrent}
\caption{بار بالمقابل وقت کا خط۔}
\end{subfigure}
\caption{مثال \حوالہ{مثال_بنیادی_بار_اور_رو} کا شکل۔}
\label{شکل_بنیادی_بار_بالمقابل_وقت}
\end{figure}

حل:وقت \عددی{t=0} تا \عددی{t=\SI{0.5}{\micro\second}} تک برقی بار بلا تبدیل ہوئے \عددی{\SI{0.5}{\milli\coulomb}} رہتا ہے لہٰذا  \عددی{\Delta q=0} ہے اور یوں اس دورانیے میں
\begin{align*}
i&=\frac{\Delta q}{\Delta t}=\frac{\SI{0}{\coulomb}}{\SI{0.5}{\micro\second}}=\SI{0}{\ampere} \quad \quad (0<t<\SI{0.5}{\micro\second})
\end{align*} 
ہو گا۔وقت \عددی{t=\SI{0.5}{\micro\second}} تا \عددی{t=\SI{2}{\micro\second}} کے دوران برقی بار \عددی{\SI{0.5}{\milli\coulomb}} سے تبدیل ہو کر \عددی{\SI{2}{\milli\coulomb}} ہو جاتا ہے لہٰذا اس دورانیے کے لئے
\begin{align*}
i&=\frac{\SI{2}{\milli\coulomb}-\SI{0.5}{\milli \coulomb}}{\SI{2}{\micro\second}-\SI{0.5}{\micro\second}}=\SI{1000}{\ampere}\quad \quad (\SI{0.5}{\micro\second}<t<\SI{2}{\micro\second})
\end{align*} 
ہو گا۔اسی طرح بقایا دورانیوں میں
\begin{align*}
i&=\frac{\SI{2.5}{\milli\coulomb}-\SI{2}{\milli \coulomb}}{\SI{3.5}{\micro\second}-\SI{2}{\micro\second}}=\SI{333.33}{\ampere}\quad \quad (\SI{2}{\micro\second}<t<\SI{3.5}{\micro\second})\\
i&=\frac{\SI{2.5}{\milli\coulomb}-\SI{2.5}{\milli \coulomb}}{\SI{4}{\micro\second}-\SI{3.5}{\micro\second}}=\SI{0}{\ampere}\quad \quad (\SI{3.5}{\micro\second}<t<\SI{4}{\micro\second})\\
i&=\frac{\SI{-1}{\milli\coulomb}-\SI{2.5}{\milli \coulomb}}{\SI{5}{\micro\second}-\SI{4}{\micro\second}}=\SI{-3500}{\ampere}\quad \quad (\SI{4}{\micro\second}<t<\SI{5}{\micro\second})\\
i&\frac{\SI{-1}{\milli\coulomb}-(\SI{-1}{\milli\coulomb})}{\SI{6}{\micro\second}-\SI{5}{\micro\second}}=\SI{0}{\ampere}\quad \quad\quad \quad (\SI{5}{\micro\second}<t<\SI{6}{\micro\second})\\
i&=\frac{\SI{-0.5}{\milli\coulomb}-(\SI{-1}{\milli \coulomb})}{\SI{7}{\micro\second}-\SI{6}{\micro\second}}=\SI{500}{\ampere}\quad \quad (\SI{6}{\micro\second}<t<\SI{7}{\micro\second})\\
i&=\SI{0}{\ampere}\quad \quad\quad \quad\quad \quad\quad \quad\quad \quad (\SI{7}{\micro\second}<t)
\end{align*} 
اور اس کے بعد \عددی{i=\SI{0}{\ampere}} ہے۔ان نتائج کو شکل \حوالہ{شکل_بنیادی_برقی_رو_مثال} میں دکھایا گیا ہے۔آپ دیکھ سکتے ہیں کہ بار نہ بدلنے کی صورت میں رو صفر ہوتی ہے۔بڑھتے بار کی صورت میں مثبت رو اور گھٹتے بار کی صورت میں منفی رو پائی جاتی ہے۔

\begin{figure}
\centering
\includegraphics{figBasicChargeVersusCurrentAnswer}
\caption{برقی رو مثال \حوالہ{مثال_بنیادی_بار_اور_رو}}
\label{شکل_بنیادی_برقی_رو_مثال}
\end{figure}
\انتہا{مثال}
%======================
\ابتدا{مثال}
مندرجہ بالا مثال میں طاقت بالمقابل وقت حاصل کریں۔

حل:طاقت \عددی{p=v i} ہوتا ہے۔شکل \حوالہ{شکل_بنیادی_بار_بالمقابل_وقت}-الف سے دباو کی قیمت \عددی{\SI{15}{\volt}} ملتی ہے جبکہ شکل \حوالہ{شکل_بنیادی_برقی_رو_مثال} سے رو کی قیمت مختلف دورانیے کے لئے حاصل کی جا سکتی ہے۔یوں مختلف دورانیے کے طاقت درج ذیل حاصل ہوتے ہیں۔
\begin{align*}
p&=15 \times 0=\SI{0}{\watt} \quad \quad \quad (0<t<\SI{0.5}{\micro\second}) \\
p&=15 \times 1000=\SI{15}{\kilo\watt} \quad \quad \quad (\SI{0.5}{\micro\second}<t<\SI{2}{\micro\second}) \\
p&=15 \times 333.33=\SI{5}{\kilo\watt} \quad \quad \quad (\SI{2}{\micro\second}<t<\SI{3.5}{\micro\second}) \\
p&=15 \times 0=\SI{0}{\watt} \quad \quad \quad \quad\quad (\SI{3.5}{\micro\second}<t<\SI{4}{\micro\second}) \\
p&=15 \times (-3500)=\SI{-52.5}{\kilo\watt} \quad \quad \quad (\SI{4}{\micro\second}<t<\SI{5}{\micro\second}) \\
p&=15 \times 0=\SI{0}{\watt} \quad \quad \quad \quad\quad (\SI{5}{\micro\second}<t<\SI{6}{\micro\second}) \\
p&=15 \times 500=\SI{7.5}{\kilo\watt} \quad \quad \quad (\SI{6}{\micro\second}<t<\SI{7}{\micro\second}) \\
p&=15 \times 0=\SI{0}{\watt} \quad \quad \quad (\SI{7}{\micro\second}<t) 
\end{align*}

\begin{figure}
\centering
\includegraphics[width=\textwidth]{figBasicPowerSuppliedToBox}
\caption{طاقت بالمقابل وقت}
\label{شکل_بنیادی_طاقت_بالمقابل_وقت}
\end{figure}

ان جوابات کو شکل \حوالہ{شکل_بنیادی_طاقت_بالمقابل_وقت} میں دکھایا گیا ہے۔
\انتہا{مثال}
%===================
\ابتدا{مثال}
آج کل \اصطلاح{کمپیوٹر}\فرہنگ{کمپیوٹر}\حاشیہب{computer}\فرہنگ{computer} کا زمانہ ہے اور  یو-ایس-بی\فرہنگ{یو-ایس-بی}\حاشیہب{USB Universal Serial Port}\فرہنگ{USB, universal serial bus} یعنی \اصطلاح{عمومی سلسلہ وار پھاٹک}\فرہنگ{پھاٹک!عمومی سلسلہ وار} کا استعمال عام ہے۔کسی بھی کمپیوٹر یا \اصطلاح{عددی دور}\فرہنگ{عددی دور}\حاشیہب{digital circuit}\فرہنگ{digital circuit} کو \اصطلاح{عددی مواد}\فرہنگ{عددی مواد}\فرہنگ{مواد!عددی}\حاشیہب{digital data}\فرہنگ{digital data}\فرہنگ{data!digital} جن برقی تاروں کے ذریعہ فراہم کیا جاتا ہے وہ کمپیوٹر یا عددی دور کے \اصطلاح{داخلی پھاٹک}\فرہنگ{داخلی پھاٹک}\فرہنگ{پھاٹک!داخلی}\حاشیہب{input port}\فرہنگ{input port} کہلاتے ہیں اور جن تاروں کے ذریعہ کمپیوٹر یا عددی دور سے عددی مواد حاصل کیا جاتا ہے، کمپیوٹر یا عددی دور کے  \اصطلاح{خارجی پھاٹک}\فرہنگ{خارجی پھاٹک}\فرہنگ{پھاٹک!خارجی}\حاشیہب{output port}\فرہنگ{output port} کہلاتے ہیں۔عمومی سلسلہ وار پھاٹک (یو-ایس-بی) پر کمپیوٹر عددی مواد حاصل بھی کر سکتا ہے اور خارج بھی  کر سکتا ہے۔یوں یہ \اصطلاح{داخلی-خارجی پھاٹک}\فرہنگ{داخلی-خارجی پھاٹک}\حاشیہب{input-output port}\فرہنگ{port!input-output} ہے۔اس پھاٹک کی مدد سے کمپیوٹر کے ساتھ بیرونی آلات مثلاً موبائل فون، عددی کیمرہ وغیرہ جوڑے جا سکتے ہیں۔یہ پھاٹک بیرونی آلات کو برقی طاقت فراہم کرنے کی صلاحیت بھی رکھتا ہے۔یہ پھاٹک چار عدد برقی تاروں پر مشتمل ہے جن میں دو تار عددی مواد کے ترسیل اور دو تار برقی طاقت کی فراہمی کے لئے استعمال ہوتے ہیں۔یہ پھاٹک عام حالت میں \عددی{\SI{100}{\milli\ampere}} برقی رو فراہم کر سکتا ہے جبکہ سافٹ وئیر کے ذریعہ پھاٹک سے برقی رو کی فراہمی \عددی{\SI{500}{\milli\ampere}} تک بڑھائی جا سکتی ہے۔

یو-ایس-بی پھاٹک استعمال کرتے ہوئے موبائل کی \اصطلاح{بے بار}\فرہنگ{بے بار}\حاشیہب{discharged}\فرہنگ{discharged} بیٹری میں بار بھرا جاتا ہے۔بیٹری کی استعداد \عددی{\SI{1700}{\milli\ampere\hour}} ہے۔الف) بیٹری کی استعداد کولمب \عددی{\si{\coulomb}} میں حاصل کریں۔ ب) اگر پھاٹک \عددی{\SI{100}{\milli\ampere}} رو فراہم کر رہا ہو تب بیٹری کو مکمل بھرنے میں کتنی دیر لگے گی۔ 

حل:الف) مکمل بھری بیٹری میں کل بار ہی بیٹری کی استعداد ہوتی ہے۔بیٹری کی استعداد کو کولمب \عددی{\si{\coulomb}} کی بجائے \عددی{\si{\ampere\hour}} میں بیان کیا جاتا ہے۔دی گئی بیٹری کی استعداد
\begin{align*}
Q=I\times t=1700 \times 10^{-3} \times 3600=\SI{6120}{\coulomb}
\end{align*}
ہے جہاں ایک گھنٹہ  \عددی{3600} سیکنڈ کے برابر ہے۔ 

ب) یوں \عددی{\SI{100}{\milli\ampere}} کی رو سے بیٹری بھرنے میں
\begin{align*}
t=\frac{6120}{100\times 10^{-3}}=\SI{61200}{\second} =\SI{17}{\hour}
\end{align*}
سترہ گھنٹے درکار ہوں گے۔
\انتہا{مثال}
%=====================
%==========================
\حصہء{سوالات}
\ابتدا{سوال}
ایک تار میں \عددی{\SI{1.5}{\ampere}} رو پائی جاتی ہے۔اس تار پر کسی نقطے سے \عددی{\SI{45}{\second}} میں کتنا بار گزرتا ہے۔

جواب:\عددی{\SI{67.5}{\coulomb}}
\انتہا{سوال}
%======================
\ابتدا{سوال}
ایک تار سے \عددی{\SI{22}{\second}} میں کل \عددی{10^{21}} الیکٹران گزرتے ہیں۔تار میں اوسط رو دریافت کریں۔
 
جواب:\عددی{\SI{7.27}{\ampere}}
\انتہا{سوال}
%======================
\ابتدا{سوال}
\عددی{\SI{20}{\ampere}} بیٹری چارجر کتنی دیر میں \عددی{\SI{4000}{\coulomb}} بار فراہم کرے گا۔

جواب:\عددی{\SI{200}{\second}} 
\انتہا{سوال}
%======================
\ابتدا{سوال}
آسمانی بجلی \عددی{\SI{60}{\micro\second}} کے لئے \عددی{\SI{20}{\kilo\ampere}} فراہم کرتی ہے۔آسمانی بجلی میں کل بار دریافت کریں۔

جواب:\عددی{\SI{1.2}{\coulomb}}
\انتہا{سوال}
%======================
\ابتدا{سوال}
ایک تار میں \عددی{\SI{32}{\second}} میں \عددی{\SI{88}{\coulomb}} بار گزرتا ہے۔ تار میں رو دریافت کریں۔

جواب:\عددی{\SI{2.75}{\ampere}}
\انتہا{سوال}
%====================
\ابتدا{سوال}\شناخت{سوال_مزاحمتی_ضاع_الٹ_دباو}
شکل \حوالہ{شکل_سوال_مزاحمتی_ضاع_الٹ_دباو} میں نقطہ-الف سے نقطہ-ب تک بیرونی پرزے میں دس کولومب کا بار گزرتا ہے جبکہ پرزے میں توانائی کا ضیاع پچاس جاول ہے۔دباو \عددی{V_1} حاصل کریں۔
\begin{figure}
\centering
\begin{tikzpicture}[american voltages]
\draw(0,0) rectangle ++(-0.5,\yy+0.5);
\draw(0,0.25) to [short,-o]++(\xx/2,0)node[below]{\text{الف}}  to [short]++(\xx/2,0)to [european resistor]++(0,\yy) to [short,-o]++(-\xx/2,0)node[above]{\text{ب}} to [short]++(-\xx/2,0);
\draw (\xx/2,\yy+0.25) to [open,v={$V_1$}]++(0,-\yy);
\end{tikzpicture}
\caption{سوال \حوالہ{سوال_مزاحمتی_ضاع_الٹ_دباو} کا دور۔}
\label{شکل_سوال_مزاحمتی_ضاع_الٹ_دباو}
\end{figure}

جواب:\عددی{V_1=\SI{-5}{\volt}}
\انتہا{سوال}
%=============
\ابتدا{سوال}\شناخت{سوال_بنیادی_ترسیم_رو_بار}
ایک پرزے کا رو بالمقابل وقت ترسیم شکل \حوالہ{شکل_سوال_بنیادی_ترسیم_رو_بار} میں دیا گیا ہے۔ان بیس سیکنڈ دورانیے میں کتنا بار پرزے میں داخل ہو گا۔
\begin{figure}
\centering
\begin{tikzpicture}
\centering
\begin{axis}[small,axis lines*=middle,xlabel={$t\,(\si{\second})$},ylabel={$i(t)\, (\si{\milli\ampere})$},ylabel style={rotate=-90},ytick={2},yticklabels={$20$},xtick={2,4,6},xticklabels={$2$,$4$,$6$}]
\addplot[] plot coordinates{(0,2) (4,2) (6,0)};
\end{axis}
\end{tikzpicture}
\caption{سوال \حوالہ{سوال_بنیادی_ترسیم_رو_بار} کا ترسیم۔}
\label{شکل_سوال_بنیادی_ترسیم_رو_بار}
\end{figure}

جواب:\عددی{\SI{100}{\milli\coulomb}}
\انتہا{سوال}
%==============
\ابتدا{سوال}
ایک پرزے کے مثبت سر سے \عددی{q(t)=22e^{-5t}\,\si{\milli\ampere}} بار داخل ہوتا ہے جبکہ پرزے پر دباو \عددی{v(t)=15e^{-2t}\,\si{\volt}} ہے۔دورانیے \عددی{0 \le t \le \SI{3}{\second}} میں پرزے کو کتنی توانائی منتقل ہوتی ہے۔

جواب:\عددی{\SI{47.14}{\milli\joule}}
\انتہا{سوال}
%=================
\ابتدا{سوال}\شناخت{سوال_بنیادی_طاقت_وقت}
ایک پرزے کو منتقل توانائی بالمقابل وقت ترسیم شکل \حوالہ{شکل_سوال_بنیادی_طاقت_وقت} میں دیا گیا ہے جبکہ پرزے پر \عددی{\SI{2}{\volt}} دباو پایا جاتا ہے۔ان تیس سیکنڈ دورانیے میں پرزے کو کتنا بار فراہم کیا گیا ہے۔
\begin{figure}
\centering
\begin{tikzpicture}
\centering
\begin{axis}[small,axis lines*=middle,xlabel={$t\,(\si{\second})$},ylabel={$w(t)\, (\si{\milli\joule})$},ylabel style={rotate=-90},ytick={20,30},yticklabels={$20$,$30$},xtick={8,12,15,20,30},xticklabels={$8$,$12$,$15$,$20$,$30$}]
\addplot[] plot coordinates{(0,0) (8,20) (12,20) (15,30) (20,30) (30,0)};
\end{axis}
\end{tikzpicture}
\caption{سوال \حوالہ{سوال_بنیادی_طاقت_وقت} کا ترسیم۔}
\label{شکل_سوال_بنیادی_طاقت_وقت}
\end{figure}

جواب:\عددی{\SI{107}{\milli\coulomb}}
\انتہا{سوال}
%==============
\ابتدا{سوال}\شناخت{سوال_بنیادی_بار_وقت}
ایک پرزے کے مثبت سر پر داخلی بار بالمقابل وقت ترسیم شکل \حوالہ{شکل_سوال_بنیادی_بار_وقت}-الف میں دیا گیا ہے جبکہ پرزے پر \عددی{\SI{6}{\volt}} دباو پایا جاتا ہے۔ان بیس سیکنڈ دورانیے میں پرزے کو کتنی توانائی فراہم کی جاتی ہے۔دور کو شکل-ب میں دکھایا گیا ہے۔
\begin{figure}
\centering
\begin{subfigure}{0.7\textwidth}
\centering
\begin{tikzpicture}
\centering
\begin{axis}[small,xlabel={$t\,(\si{\second})$},ylabel={$q(t)\, (\si{\milli\coulomb})$},ylabel style={rotate=-90},ytick={-10,0,4,6},yticklabels={$-10$,$0$,$4$,$6$},xtick={0,5,7,9,12,18,20},xticklabels={$0$,$5$,$7$,$9$,$12$,$18$,$20$}]
\addplot[] plot coordinates{(0,4) (5,4) (7,0) (9,0) (12,-10) (18,6) (20,6)};
\end{axis}
\end{tikzpicture}
\caption*{(الف)}
\end{subfigure}%
\begin{subfigure}{0.3\textwidth}
\centering
\begin{tikzpicture}
\draw(0,0) rectangle ++(\xx/2,\yy+0.5);
\draw(0,0.25) to [short]++(-\xx/2,0) to [american voltage source,l={$\SI{6}{\volt}$}]++(0,\yy) to [short,i={$i(t)$}]++(\xx/2,0);
\draw(\xx/4,\yy/2+0.25)node{ڈبہ};
\end{tikzpicture}
\caption*{(ب)}
\end{subfigure}
\caption{سوال \حوالہ{سوال_بنیادی_بار_وقت} کا ترسیم۔}
\label{شکل_سوال_بنیادی_بار_وقت}
\end{figure}

جواب:\عددی{\SI{159}{\milli\joule}}
\انتہا{سوال}
%=======================
\ابتدا{سوال}\شناخت{سوال_بنیادی_توانائی_وقت}
ایک ڈبے کو فراہم طاقت بالمقابل وقت ترسیم شکل \حوالہ{شکل_سوال_بنیادی_توانائی_وقت}-الف میں دیا گیا ہے۔ان تیس سیکنڈ دورانیے میں ڈبے کو کتنی توانائی فراہم کی گئی؟۔دور کو شکل-ب میں دکھایا گیا ہے۔
\begin{figure}
\centering
\begin{subfigure}{0.7\textwidth}
\centering
\begin{tikzpicture}
\centering
\begin{axis}[small,xlabel={$t\,(\si{\second})$},ylabel={$p(t)\, (\si{\watt})$},ylabel style={rotate=-90},ytick={-6,0,10},
yticklabels={$-6$,$0$,$10$},xtick={0,5,12,20,25,30},xticklabels={$0$,$5$,$12$,$20$,$25$,$30$}]
\addplot[] plot coordinates{(-1,0) (0,0) (5,10) (12,10) (20,0) (25,0) (25,-6) (30,0)(32,0)};
\end{axis}
\end{tikzpicture}
\caption*{(الف)}
\end{subfigure}%
\begin{subfigure}{0.3\textwidth}
\centering
\begin{tikzpicture}
\draw(0,0) rectangle ++(\xx/2,\yy+0.5);
\draw(0,0.25) to [short]++(-\xx/2,0) to [american voltage source,l={$\SI{8}{\volt}$}]++(0,\yy) to [short,i={$i(t)$}]++(\xx/2,0);
\draw(\xx/4,\yy/2+0.25)node{ڈبہ};
\end{tikzpicture}
\caption*{(ب)}
\end{subfigure}
\caption{سوال \حوالہ{سوال_بنیادی_توانائی_وقت} کا ترسیم۔}
\label{شکل_سوال_بنیادی_توانائی_وقت}
\end{figure}

جواب:\عددی{\SI{120}{\joule}}
\انتہا{سوال}
%===============
\ابتدا{سوال}\شناخت{سوال_بنیادی_طاقت_مہیا_حاصل_الف}
شکل \حوالہ{شکل_بنیادی_طاقت_مہیا_حاصل_الف} میں پرزہ الف کو مہیا یا اس سے حاصل طاقت درج ذیل صورتوں میں دریافت کریں۔
\begin{itemize}
\item
$V_1=\SI{6}{\volt}$ اور $I=\SI{2}{\ampere}$ ہیں۔
\item
$V_1=\SI{-3}{\volt}$ اور $I=\SI{7}{\ampere}$ ہیں۔
\item
$V_1=\SI{5}{\volt}$ اور $I=\SI{-4}{\ampere}$ ہیں۔
\item
$V_1=\SI{-4}{\volt}$ اور $I=\SI{-2}{\ampere}$ ہیں۔
\end{itemize}
%
\begin{figure}
\centering
\begin{tikzpicture}[american voltages]
\draw(0,-0.25) rectangle ++(-0.5,\yy+0.5);
\draw(0,\yy) to [short,-o]++(\xx/2,0) to [short,o-,i={$I$}]++(\xx/2,0) to [european resistor,l={الف}]++(0,-\yy) to [short,-o]++(-\xx/2,0) to [short,o-]++(-\xx/2,0);
\draw(\xx/2,\yy) to [open,v={$V_1$}]++(0,-\yy);
\end{tikzpicture}
\caption{سوال \حوالہ{سوال_بنیادی_طاقت_مہیا_حاصل_الف} کا دور۔}
\label{شکل_بنیادی_طاقت_مہیا_حاصل_الف}
\end{figure}

جوابات:\عددی{\SI{12}{\watt}}، \عددی{\SI{-21}{\watt}}، \عددی{\SI{-20}{\watt}}، \عددی{\SI{8}{\watt}} 
\انتہا{سوال}
%===============
\ابتدا{سوال}\شناخت{سوال_بنیادی_طاقت_مہیا_حاصل_ب}
شکل \حوالہ{شکل_سوال_بنیادی_طاقت_مہیا_حاصل_ب} میں پرزہ الف اور ب کو مہیا یا حاصل طاقت  دریافت کریں۔
%
\begin{figure}
\centering
\begin{subfigure}{0.5\textwidth}
\centering
\begin{tikzpicture}[american voltages]
\draw(0,-0.25) rectangle ++(-0.5,\yy+0.5);
\draw(0,\yy) to [short,-o]++(\xx/2,0) to [short,o-,i={$\SI{2}{\ampere}$}]++(\xx/2,0) to [european resistor,v_>={$\SI{6}{\volt}$},l={الف}]++(0,-\yy) to [short,-o]++(-\xx/2,0) to [short,o-]++(-\xx/2,0);
\end{tikzpicture}
\caption*{(الف)}
\end{subfigure}%
\begin{subfigure}{0.5\textwidth}
\centering
\begin{tikzpicture}[american voltages]
\draw(0,-0.25) rectangle ++(-0.5,\yy+0.5);
\draw(0,\yy) to [short,-o]++(\xx/2,0) to [short,o-,i<={$\SI{3}{\ampere}$}]++(\xx/2,0) to [european resistor,v_>={$\SI{2}{\volt}$},l={ب}]++(0,-\yy) to [short,-o]++(-\xx/2,0) to [short,o-]++(-\xx/2,0);
\end{tikzpicture}
\caption*{(ب)}
\end{subfigure}%
\caption{سوال \حوالہ{سوال_بنیادی_طاقت_مہیا_حاصل_ب} کا دور۔}
\label{شکل_سوال_بنیادی_طاقت_مہیا_حاصل_ب}
\end{figure}

جوابات:پرزہ الف سے \عددی{\SI{12}{\watt}} طاقت حاصل کی جا رہی ہے۔ پرزہ ب کو \عددی{\SI{6}{\watt}} فراہم کی جا رہی ہے۔
\انتہا{سوال}
%===============
\ابتدا{سوال}\شناخت{سوال_بنیادی_طاقت_مہیا_حاصل_پ}
شکل \حوالہ{شکل_سوال_بنیادی_طاقت_مہیا_حاصل_پ} میں پرزہ الف کو \عددی{\SI{20}{\watt}} فراہم کی جا رہی ہے جبکہ پرزہ ب سے \عددی{\SI{12}{\watt}} حاصل کیا جا رہا ہے۔دباو \عددی{V_x} اور \عددی{V_y} دریافت کریں۔
%
\begin{figure}
\centering
\begin{subfigure}{0.5\textwidth}
\centering
\begin{tikzpicture}[american voltages]
\draw(0,-0.25) rectangle ++(-0.5,\yy+0.5);
\draw(0,\yy) to [short,-o]++(\xx/2,0) to [short,o-,i={$\SI{5}{\ampere}$}]++(\xx/2,0) to [european resistor,v_>={$V_x$},l={الف}]++(0,-\yy) to [short,-o]++(-\xx/2,0) to [short,o-]++(-\xx/2,0);
\end{tikzpicture}
\caption*{(الف)}
\end{subfigure}%
\begin{subfigure}{0.5\textwidth}
\centering
\begin{tikzpicture}[american voltages]
\draw(0,-0.25) rectangle ++(-0.5,\yy+0.5);
\draw(0,\yy) to [short,-o]++(\xx/2,0) to [short,o-,i<={$\SI{3}{\ampere}$}]++(\xx/2,0) to [european resistor,v_>={$V_y$},l={ب}]++(0,-\yy) to [short,-o]++(-\xx/2,0) to [short,o-]++(-\xx/2,0);
\end{tikzpicture}
\caption*{(ب)}
\end{subfigure}%
\caption{سوال \حوالہ{سوال_بنیادی_طاقت_مہیا_حاصل_پ} کا دور۔}
\label{شکل_سوال_بنیادی_طاقت_مہیا_حاصل_پ}
\end{figure}
س
جوابات:\عددی{V_x=\SI{-4}{\volt}}، \عددی{V_y=\SI{-4}{\volt}}
\انتہا{سوال}
%===============
\ابتدا{سوال}\شناخت{سوال_بنیادی_طاقت_مہیا_حاصل_ت}
شکل \حوالہ{شکل_سوال_بنیادی_طاقت_مہیا_حاصل_ت} میں پرزہ الف کو \عددی{\SI{48}{\watt}} فراہم کی جا رہی ہے جبکہ پرزہ ب سے \عددی{\SI{36}{\watt}} حاصل کی جا رہی ہے۔رو \عددی{I_x} اور \عددی{I_y} دریافت کریں۔
%
\begin{figure}
\centering
\begin{subfigure}{0.5\textwidth}
\centering
\begin{tikzpicture}[american voltages]
\draw(0,-0.25) rectangle ++(-0.5,\yy+0.5);
\draw(0,\yy) to [short,-o]++(\xx/2,0) to [short,o-]++(\xx/2,0) to [european resistor,v_>={$\SI{12}{\volt}$},l={الف}]++(0,-\yy) to [short,-o,i={$I_x$}]++(-\xx/2,0) to [short,o-]++(-\xx/2,0);
\end{tikzpicture}
\caption*{(الف)}
\end{subfigure}%
\begin{subfigure}{0.5\textwidth}
\centering
\begin{tikzpicture}[american voltages]
\draw(0,-0.25) rectangle ++(-0.5,\yy+0.5);
\draw(0,\yy) to [short,-o]++(\xx/2,0) to [short,o-]++(\xx/2,0) to [european resistor,v_>={$\SI{6}{\volt}$},l={ب}]++(0,-\yy) to [short,-o,i<={$I_y$}]++(-\xx/2,0) to [short,o-]++(-\xx/2,0);
\end{tikzpicture}
\caption*{(ب)}
\end{subfigure}%
\caption{سوال \حوالہ{سوال_بنیادی_طاقت_مہیا_حاصل_ت} کا دور۔}
\label{شکل_سوال_بنیادی_طاقت_مہیا_حاصل_ت}
\end{figure}

جوابات:\عددی{I_x=\SI{-4}{\ampere}}، \عددی{I_y=\SI{-6}{\ampere}}
\انتہا{سوال}
%===============
\ابتدا{سوال}\شناخت{سوال_بنیادی_طاقت_مہیا_حاصل_ٹ}
شکل \حوالہ{شکل_سوال_بنیادی_طاقت_مہیا_حاصل_ٹ}-الف میں ایک پرزے کو \عددی{\SI{36}{\watt}} فراہم کئے جاتے ہیں جبکہ پرزے پر دباو \عددی{\SI{12}{\volt}} ہے۔پرزہ الف کی طاقت دریافت کریں۔کیا اس پرزے کو طاقت فراہم کی جا رہی ہے؟ شکل-ب میں پرزہ ب کے لئے بھی حل کریں۔
%
\begin{figure}
\centering
\begin{subfigure}{0.5\textwidth}
\centering
\begin{tikzpicture}[american voltages]
\draw(0,-0.25) rectangle ++(-0.5,\yy+0.5);
\draw(0,\yy) to [short,-o]++(\xx/2,0) to [european resistor,o-,v^<={$\substack{\SI{36}{\watt} \\ \SI{12}{\volt}}$}]++(\xx,0) to
 [european resistor,v^<={$\SI{6}{\volt}$},l_={الف}]++(0,-\yy) to [short,-o]++(-\xx,0) to [short,o-]++(-\xx/2,0);
\end{tikzpicture}
\caption*{(الف)}
\end{subfigure}%
\begin{subfigure}{0.5\textwidth}
\centering
\begin{tikzpicture}[american voltages]
\draw(0,-0.25) rectangle ++(-0.5,\yy+0.5);
\draw(0,\yy) to [short,-o]++(\xx/2,0) to [european resistor,o-,v^<={$\substack{\SI{24}{\watt} \\ \SI{6}{\volt}}$}]++(\xx,0) to
 [european resistor,v^>={$\SI{5}{\volt}$},l_={ب}]++(0,-\yy) to [short,-o]++(-\xx,0) to [short,o-]++(-\xx/2,0);
\end{tikzpicture}
\caption*{(ب)}
\end{subfigure}%
\caption{سوال \حوالہ{سوال_بنیادی_طاقت_مہیا_حاصل_ٹ} کا دور۔}
\label{شکل_سوال_بنیادی_طاقت_مہیا_حاصل_ٹ}
\end{figure}

جوابات:پرزہ الف کو \عددی{\SI{18}{\watt}} فراہم کی جاتی ہے جبکہ پرزہ ب سے \عددی{\SI{20}{\watt}} حاصل کیا جاتا ہے۔
\انتہا{سوال}
%===============
\ابتدا{سوال}\شناخت{سوال_بنیادی_طاقت_مہیا_حاصل_ث}
شکل \حوالہ{شکل_سوال_بنیادی_طاقت_مہیا_حاصل_ث}-الف میں  ایک پرزے کو \عددی{\SI{10}{\watt}} فراہم کئے جاتے ہیں جبکہ پرزے پر دباو \عددی{\SI{2}{\volt}} ہے۔پرزہ الف کی طاقت دریافت کریں۔کیا اس پرزے کو طاقت فراہم کی جا رہی ہے؟ شکل-ب کو بھی حل کریں۔
%
\begin{figure}
\centering
\begin{subfigure}{0.5\textwidth}
\centering
\begin{tikzpicture}[american voltages]
\draw(0,0) to [american current source,l={$\SI{8}{\ampere}$}]++(0,\yy) to [short]++(1.5*\xx,0) to [european resistor,l={$\substack{\SI{10}{\watt} \\ \SI{2}{\volt}}$}]++(0,-\yy) to [short]++(-1.5*\xx,0);
\draw(3/4*\xx,0) to [european resistor,*-*,l={الف}]++(0,\yy);
\end{tikzpicture}
\caption*{(الف)}
\end{subfigure}%
\begin{subfigure}{0.5\textwidth}
\centering
\begin{tikzpicture}[american voltages]
\draw(0,0) to [american current source,l={$\SI{6}{\ampere}$}]++(0,\yy) to [short]++(1.5*\xx,0) to [european resistor,i={$\SI{2}{\ampere}$},l={$\SI{12}{\watt}$}]++(0,-\yy) to [short]++(-1.5*\xx,0);
\draw(3/4*\xx,0) to [european resistor,*-*,l={ب}]++(0,\yy);
\end{tikzpicture}
\caption*{(ب)}
\end{subfigure}%
\caption{سوال \حوالہ{سوال_بنیادی_طاقت_مہیا_حاصل_ث} کا دور۔}
\label{شکل_سوال_بنیادی_طاقت_مہیا_حاصل_ث}
\end{figure}

جوابات:پرزہ الف کو \عددی{\SI{2}{\watt}} فراہم کئے جاتے ہیں۔ پرزہ ب کو \عددی{\SI{24}{\watt}} فراہم کئے جاتے ہیں۔
\انتہا{سوال}
%===============
\ابتدا{سوال}\شناخت{سوال_بنیادی_طاقت_مہیا_حاصل_ج}
شکل \حوالہ{شکل_بنیادی_سوال_طاقت_مہیا_حاصل_ج} میں پرزہ الف اور ب کی طاقر دریافت کریں۔ 
%
\begin{figure}
\centering
\begin{subfigure}{0.5\textwidth}
\centering
\begin{tikzpicture}[american voltages]
\draw (0,0) to [american current source,i={$\SI{4}{\ampere}$},v_>={$\SI{9}{\volt}$}]++(0,\yy) to [european resistor,l={الف}] ++(\xx,0);
\draw(0,0) to [short]++(\xx,0) to [american voltage source,l_={$\SI{12}{\volt}$}]++(0,\yy);
\end{tikzpicture}
\caption*{(الف)}
\end{subfigure}%
\begin{subfigure}{0.5\textwidth}
\centering
\begin{tikzpicture}[american voltages]
\draw (0,0) to [american current source,i={$\SI{4}{\ampere}$},v_<={$\SI{9}{\volt}$}]++(0,\yy) to [european resistor,l={ب}] ++(\xx,0);
\draw(0,0) to [short]++(\xx,0) to [american voltage source,l_={$\SI{12}{\volt}$}]++(0,\yy);
\end{tikzpicture}
\caption*{(ب)}
\end{subfigure}%
\caption{سوال \حوالہ{سوال_بنیادی_طاقت_مہیا_حاصل_ج} کا دور۔}
\label{شکل_بنیادی_سوال_طاقت_مہیا_حاصل_ج}
\end{figure}

جوابات:پرزہ الف \عددی{\SI{12}{\volt}} فراہم کرتا ہے۔ پرزہ ب سے \عددی{\SI{84}{\watt}} حاصل کیا جاتا  ہے۔
\انتہا{سوال}
%==============
\ابتدا{سوال}\شناخت{سوال_بنیادی_طاقت_مہیا_حاصل_چ}
شکل \حوالہ{شکل_بنیادی_سوال_طاقت_مہیا_حاصل_چ}-الف میں پرزہ الف \عددی{\SI{6}{\watt}} فراہم کرتا ہے۔اس میں رو کی مقدار اور سمت دریافت کریں۔شکل-ب میں پرزہ ب کو \عددی{\SI{12}{\watt}} طاقت فرہم کی جاتی ہے۔پرزہ ب میں رو دریافت کریں۔
%
\begin{figure}
\centering
\begin{subfigure}{0.5\textwidth}
\centering
\begin{tikzpicture}[american voltages]
\draw (0,0) to [american voltage source,l={$\SI{7}{\volt}$}]++(0,\yy) to [european resistor,l={الف}] ++(\xx,0);
\draw(0,0) to [short]++(\xx,0) to [american voltage source,l_={$\SI{10}{\volt}$}]++(0,\yy);
\end{tikzpicture}
\caption*{(الف)}
\end{subfigure}%
\begin{subfigure}{0.5\textwidth}
\centering
\begin{tikzpicture}[american voltages]
\draw (0,0) to [american voltage source,l={$\SI{7}{\volt}$}]++(0,\yy) to [european resistor,l={ب}] ++(\xx,0);
\draw(0,0) to [short]++(\xx,0) to [american voltage source,l_={$\SI{10}{\volt}$}]++(0,\yy);
\end{tikzpicture}
\caption*{(ب)}
\end{subfigure}%
\caption{سوال \حوالہ{سوال_بنیادی_طاقت_مہیا_حاصل_چ} کا دور۔}
\label{شکل_بنیادی_سوال_طاقت_مہیا_حاصل_چ}
\end{figure}

جوابات:پرزہ الف کے دائیں سر سے \عددی{\SI{2}{\ampere}} نکل کر \عددی{\SI{10}{\volt}} کے منبع میں داخل ہوتی ہے۔ پرزہ ب میں \عددی{\SI{4}{\ampere}} پائی جاتی ہے جو پرزے میں دائیں سے بائیں جانب رواں ہے۔ 
\انتہا{سوال}
%===============
\ابتدا{سوال}\شناخت{سوال_بنیادی_طاقت_مہیا_حاصل_ح}
شکل \حوالہ{شکل_بنیادی_سوال_طاقت_مہیا_حاصل_ح} میں پرزہ الف اور ب کی طاقت دریافت کریں۔
%
\begin{figure}
\centering
\begin{tikzpicture}[american voltages]
\draw (0,0) to [american controlled voltage source,l={$4 I_x $}]++(0,\yy) to [american voltage source,l={$\SI{20}{\volt}$}]++(\xx,0)
 to [european resistor,v^<={$\SI{18}{\volt}$},l_={الف}]++(\xx,0) to [european resistor,v^<={$\SI{22}{\volt}$},l_={ب}]++(0,-\yy) to [short,i={$\SI{5}{\ampere}$}] (0,0);
\end{tikzpicture}
\caption{سوال \حوالہ{سوال_بنیادی_طاقت_مہیا_حاصل_ح} کا دور۔}
\label{شکل_بنیادی_سوال_طاقت_مہیا_حاصل_ح}
\end{figure}

جوابات: پرزہ الف کو \عددی{\SI{90}{\watt}} مہیا کیا جاتا ہے۔ پرزہ ب کو \عددی{\SI{110}{\watt}} مہیا کیا جاتا ہے۔
\انتہا{سوال}
%===============
\ابتدا{سوال}\شناخت{سوال_بنیادی_طاقت_مہیا_حاصل_خ}
شکل \حوالہ{شکل_بنیادی_سوال_طاقت_مہیا_حاصل_خ} میں پرزہ الف، ب اور پ کی طاقت دریافت کریں۔
%
\begin{figure}
\centering
\begin{tikzpicture}[american voltages]
\draw(0,0) to [american voltage source,l={$\SI{36}{\volt}$},i={$\SI{5}{\ampere}$}]++(0,\yy) to [european resistor,v^<={$\SI{16}{\volt}$},l_={الف}]++(\xx,0) to [american voltage source,v^>={$\SI{6}{\volt}$} ,i={$\SI{2}{\ampere}$}]++(\xx,0) to [european resistor,v={$\SI{26}{\volt}$},l={پ}]++(0,-\yy) to [short] (0,0);
\draw(\xx,\yy) to [european resistor,*-*,v={$\SI{20}{\volt}$},i={$\SI{3}{\ampere}$},l={ب}] ++(0,-\yy);
\end{tikzpicture}
\caption{سوال \حوالہ{سوال_بنیادی_طاقت_مہیا_حاصل_خ} کا دور۔}
\label{شکل_بنیادی_سوال_طاقت_مہیا_حاصل_خ}
\end{figure}

جوابات: تینوں پرزوں کو طاقت فراہم کی جاتی ہے۔الف کو \عددی{\SI{80}{\watt}}، ب  کو \عددی{\SI{60}{\watt}} اور  پ کو \عددی{\SI{52}{\watt}} طاقت فراہم کی جاتی ہے۔ 
\انتہا{سوال}
%===============
\ابتدا{سوال}\شناخت{سوال_بنیادی_طاقت_مہیا_حاصل_د}
شکل \حوالہ{شکل_بنیادی_سوال_طاقت_مہیا_حاصل_د} میں رو \عددی{I_x} دریافت کریں۔
%
\begin{figure}
\centering
\begin{tikzpicture}[american voltages]
\draw(0,0) to [american voltage source,l={$\SI{36}{\volt}$},i={$I_x$}]++(0,\yy) to [european resistor,v^<={$\SI{16}{\volt}$},l_={الف}]++(\xx,0) to [american voltage source,v^>={$\SI{6}{\volt}$} ,i={$\SI{6}{\ampere}$}]++(\xx,0) to [european resistor,v={$\SI{26}{\volt}$},l={پ}]++(0,-\yy) to [short] (0,0);
\draw(\xx,\yy) to [european resistor,*-*,v={$\SI{20}{\volt}$},i={$\SI{4}{\ampere}$},l={ب}] ++(0,-\yy);
\end{tikzpicture}
\caption{سوال \حوالہ{سوال_بنیادی_طاقت_مہیا_حاصل_د} کا دور۔}
\label{شکل_بنیادی_سوال_طاقت_مہیا_حاصل_د}
\end{figure}

جوابات: \عددی{I_x=\SI{10}{\ampere}}
\انتہا{سوال}
%===============
\ابتدا{سوال}\شناخت{سوال_بنیادی_طاقت_مہیا_حاصل_ڈ}
شکل \حوالہ{شکل_بنیادی_سوال_طاقت_مہیا_حاصل_ڈ} میں پرزہ الف اور ب کا طاقت دریافت کریں۔
%
\begin{figure}
\centering
\begin{tikzpicture}[american voltages]
\draw(0,0) to [american voltage source,l={$\SI{36}{\volt}$},i={$I_x$}]++(0,\yy) to [european resistor,v^<={$\SI{16}{\volt}$},l_={الف}]++(\xx,0) to [american voltage source,v^>={$\SI{6}{\volt}$} ,i<={$\SI{6}{\ampere}$}]++(\xx,0) to [european resistor,v={$\SI{26}{\volt}$},l={پ}]++(0,-\yy) to [short] (0,0);
\draw(\xx,\yy) to [american controlled current source,*-*,l={$4I_x$}] ++(0,-\yy);
\end{tikzpicture}
\caption{سوال \حوالہ{سوال_بنیادی_طاقت_مہیا_حاصل_ڈ} کا دور۔}
\label{شکل_بنیادی_سوال_طاقت_مہیا_حاصل_ڈ}
\end{figure}

جوابات: پرزہ الف کو \عددی{\SI{32}{\watt}} فراہم کیا جاتا ہے جبکہ پرزہ ب سے \عددی{\SI{156}{\watt}} حاصل ہوتا ہے۔
\انتہا{سوال}
%===============
\ابتدا{سوال}\شناخت{سوال_بنیادی_طاقت_مہیا_حاصل_ذ}
شکل \حوالہ{شکل_بنیادی_سوال_طاقت_مہیا_حاصل_ذ}-الف میں پرزہ الف کا طاقت دریافت کریں۔شکل-ب میں پرزہ ب کی رو، دباو اور طاقت دریافت کریں۔
%
\begin{figure}
\centering
\begin{subfigure}{0.5\textwidth}
\centering
\begin{tikzpicture}[american voltages]
\draw(0,0) to [european resistor,l={$\SI{36}{\watt}$}]++(0,\y) to [european resistor,l={$\SI{12}{\watt}$}]++(\x,0) to [european resistor,l={الف}] ++(0,-\y) to [short] (0,0);
\end{tikzpicture}
\caption*{الف}
\end{subfigure}%
\begin{subfigure}{0.5\textwidth}
\centering
\begin{tikzpicture}[american voltages]
\draw(0,0) to [european resistor,l_={$\SI{24}{\watt}$},v^>={$\SI{6}{\volt}$}]++(0,\y) to [european resistor,l_={$\SI{12}{\watt}$},v^<={$\SI{2}{\volt}$}]++(\x,0) to [european resistor,l={ب}] ++(0,-\y) to [short] (0,0);
\end{tikzpicture}
\caption*{ب}
\end{subfigure}
\caption{سوال \حوالہ{سوال_بنیادی_طاقت_مہیا_حاصل_ذ} کا دور۔}
\label{شکل_بنیادی_سوال_طاقت_مہیا_حاصل_ذ}
\end{figure}

جوابات: پرزہ الف کو \عددی{\SI{24}{\watt}} فراہم کیا جاتا ہے۔پرزہ ب کی رو \عددی{\SI{4}{\ampere}}، دباو \عددی{\SI{4}{\volt}} اور اس کو فراہم کردہ طاقت \عددی{\SI{16}{\watt}} ہے۔
\انتہا{سوال}
%===============
\ابتدا{سوال}\شناخت{سوال_بنیادی_طاقت_مہیا_حاصل_ر}
شکل \حوالہ{شکل_بنیادی_سوال_طاقت_مہیا_حاصل_ر}-الف میں پرزہ الف کا طاقت دریافت کریں۔شکل-ب میں پرزہ ب کی رو، دباو اور طاقت دریافت کریں۔
%
\begin{figure}
\centering
\begin{subfigure}{0.5\textwidth}
\centering
\begin{tikzpicture}[american voltages]
\draw(0,0) to [european resistor,l_={$\SI{3}{\watt}$}]++(0,\y) to [european resistor,l={$\SI{-12}{\watt}$}]++(\x,0) to [european resistor,l={الف}] ++(0,-\y) to [short] (0,0);
\draw(\x,\y) to [european resistor,*-,l={$\SI{5}{\watt}$}]++(\x,0) to [european resistor,l_={$\SI{13}{\watt}$}]++(0,-\y) to [short,-*]++(-\x,0);
\end{tikzpicture}
\caption*{الف}
\end{subfigure}%
\begin{subfigure}{0.5\textwidth}
\centering
\begin{tikzpicture}[american voltages]
\draw(0,0) to [european resistor,l_={$\SI{6}{\watt}$},v^>={$\SI{2}{\volt}$}]++(0,\y) to [european resistor]++(\x,0) to [european resistor,v={$\SI{6}{\volt}$},l={ب}] ++(0,-\y) to [short] (0,0);
\draw(\x,\y) to [european resistor,*-,v^<={$\SI{3}{\volt}$},l_={$\SI{6}{\watt}$}]++(\x,0) to [european resistor]++(0,-\y) to [short,-*]++(-\x,0);
\end{tikzpicture}
\caption*{ب}
\end{subfigure}
\caption{سوال \حوالہ{سوال_بنیادی_طاقت_مہیا_حاصل_ر} کا دور۔}
\label{شکل_بنیادی_سوال_طاقت_مہیا_حاصل_ر}
\end{figure}

جوابات: پرزہ الف سے \عددی{\SI{9}{\watt}} حاصل کیا جاتا ہے۔پرزہ ب سے \عددی{\SI{30}{\watt}} حاصل کیا جاتا ہے۔
\انتہا{سوال}
%===============
\ابتدا{سوال}\شناخت{سوال_بنیادی_طاقت_مہیا_حاصل_ڑ}
شکل \حوالہ{شکل_بنیادی_سوال_طاقت_مہیا_حاصل_ڑ}-الف میں پرزہ الف کا طاقت دریافت کریں۔شکل-ب میں پرزہ ب کی رو، دباو اور طاقت دریافت کریں۔
%
\begin{figure}
\centering
\begin{subfigure}{0.5\textwidth}
\centering
\begin{tikzpicture}[american voltages]
\draw(0,0) to [european resistor,l_={$\SI{10}{\watt}$},v^>={$\SI{2}{\volt}$}]++(0,\y) to [european resistor,v^>={$\SI{4}{\volt}$}]++(\x,0) to [european resistor,l={الف}] ++(0,-\y) to [short] (0,0);
\draw(\x,\y) to [european resistor,*-]++(\x,0) to [european resistor,l={$\SI{24}{\watt}$},v={$\SI{4}{\volt}$}]++(0,-\y) to [short,-*]++(-\x,0);
\end{tikzpicture}
\caption*{الف}
\end{subfigure}%
\begin{subfigure}{0.5\textwidth}
\centering
\begin{tikzpicture}[american voltages]
\draw(0,0) to [european resistor,l_={$\SI{-20}{\watt}$},v^>={$\SI{4}{\volt}$}]++(0,\y) to [european resistor,v^<={$\SI{6}{\volt}$}]++(\x,0) to [european resistor,l={ب}] ++(0,-\y) to [short] (0,0);
\draw(\x,\y) to [european resistor,*-]++(\x,0) to [european resistor,l={$\SI{20}{\watt}$},v={$\SI{2}{\volt}$}]++(0,-\y) to [short,-*]++(-\x,0);
\end{tikzpicture}
\caption*{ب}
\end{subfigure}
\caption{سوال \حوالہ{سوال_بنیادی_طاقت_مہیا_حاصل_ڑ} کا دور۔}
\label{شکل_بنیادی_سوال_طاقت_مہیا_حاصل_ڑ}
\end{figure}

جوابات: پرزہ الف سے \عددی{\SI{66}{\watt}} حاصل کیا جاتا ہے۔ پرزہ ب کو \عددی{\SI{10}{\watt}} فراہم کیا جاتا ہے۔
\انتہا{سوال}
%===============



\باب{مزاحمتی ادوار}

\حصہ{قانون اوہم}
شکل \حوالہ{شکل_مزاحمتی_سیدھے_خطوط}-الف میں \اصطلاح{کارتیسی محدد}\فرہنگ{کارتیسی محدد}\حاشیہب{Cartesian coordinates}\فرہنگ{Cartesian coordinates} پر سیدھے خطوط دکھائے گئے ہیں۔بالائی خط کی مساوات \عددی{y=m_1 x+c_1} ہے جہاں خط کی \اصطلاح{ڈھلوان}\فرہنگ{ڈھلوان}\حاشیہب{slope}\فرہنگ{slope} \عددی{m_1} ہے جبکہ خط  \عددی{y} محدد کو \عددی{c_1} پر کاٹتا ہے۔نچلی خط کی ڈھلوان \عددی{m_2} ہے جبکہ یہ محدد کے مرکز \عددی{(0,0)} سے گزرتی ہے لہٰذا یہ خط \عددی{y} محدد کو \عددی{0} پر کاٹتی ہے اور یوں اس کی مساوات \عددی{y=m_2 x} ہے۔

\begin{figure}
\centering
\begin{subfigure}{0.5\textwidth}
\includegraphics{figResistiveStraightLines}
\caption{سیدھے خطوط اور ان کی ریاضی مساوات۔}
\end{subfigure}%
%
\begin{subfigure}{0.5\textwidth}
\includegraphics{figResistiveOhmLaw}
\caption{مزاحمت کے برقی دباو بالمقابل رو خط اور اوہم کا قانون۔}
\end{subfigure}%
\caption{قانون اوہم دراصل سیدھے خط کی مساوات ہے۔}
\label{شکل_مزاحمتی_سیدھے_خطوط}
\end{figure}


مزاحمت کے دو سروں کے مابین مختلف برقی دباو \عددی{v} لاگو کرتے ہوئے برقی رو \عددی{i} ناپی گئی۔برقی دباو کو عمودی محدد اور برقی رو کو افقی محدد پر رکھتے ہوئے ان کے تعلق کو شکل \حوالہ{شکل_مزاحمتی_سیدھے_خطوط}-ب میں دکھایا گیا ہے۔اس خط کو مزاحمت کی \اصطلاح{دباو بالمقابل رو خط} کہا جاتا ہے۔شکل-ب کا شکل-الف کی نچلی خط کے ساتھ موازنہ کرتے ہوئے اس خط کو 
\begin{align}
v=R i  \quad \quad \text{\RL{قانون اوہم}}
\end{align}
لکھا جا سکتا ہے جہاں خط کی ڈھلوان کو \عددی{R} لکھا اور \اصطلاح{برقی مزاحمت}\فرہنگ{مزاحمت}\حاشیہب{electrical resistance}\فرہنگ{resistance}  یا صرف \اصطلاح{مزاحمت} پکارا جاتا ہے۔اس مساوات کو \اصطلاح{قانون اوہم}\فرہنگ{اوہم!قانون}\فرہنگ{قانون!اوہم}\حاشیہب{Ohm's law}\فرہنگ{Ohm's law} کہتے ہیں۔شکل-ب میں \اصطلاح{مزاحمت} \عددیء{R} کو بطور ڈھلوان دکھایا گیا ہے۔
\begin{align}
R=\frac{v_2-v_1}{i_2-i_1}=\frac{\Delta v}{\Delta i} \quad \quad \text{\RL{مزاحمت کی تعریف}}
\end{align}

شکل \حوالہ{شکل_مزاحمتی_سیدھے_خطوط}-ب میں دباو اور رو راست تناسب کا تعلق رکھتے ہیں۔راست تناسبی تعلق کو \اصطلاح{خطی}\فرہنگ{خطی}\حاشیہب{linear}\فرہنگ{linear} تعلق کہا جاتا ہے۔اگرچہ اس کتاب میں مزاحمت کو \اصطلاح{خطی پرزہ}\فرہنگ{خطی پرزہ}\حاشیہب{linear component}\فرہنگ{linear component} ہی تصور کیا جائے گا، یہ جاننا ضروری ہے کہ کئی نہایت اہم اقسام کے پرزے  غیر خطی مزاحمت کی خاصیت رکھتے ہیں۔عام استعمال میں \عددی{\SI{220}{\volt}} پر جلنے والا بلب غیر خطی مزاحمت کی مثال ہے۔اس بلب کے \عددی{v-i} تعلق کو شکل \حوالہ{شکل_مزاحمتی_غیر_خطی_تعلق} میں دکھایا گیا ہے۔

\begin{figure}
\centering
\includegraphics{figResistiveIncadecentBulb}
\caption{غیر خطی دباو بالمقابل رو کی تعلق۔}
\label{شکل_مزاحمتی_غیر_خطی_تعلق}
\end{figure} 

وقت کے ساتھ بدلتا دباو اور بدلتی رو کی صورت میں قانون اوہم
\begin{align}\label{مساوات_مزاحمت_قانون_اوہم}
v(t) =Ri(t) 
\end{align}
لکھا جائے گا جہاں وقت \عددی{t} کے ساتھ بدلتے برقی دباو اور بدلتی برقی رو کو چھوٹے حروف میں لکھا گیا ہے۔مساوات \حوالہ{مساوات_مزاحمت_قانون_اوہم} سے مزاحمت کا بُعد \عددی{\si{\volt\per\ampere}} حاصل ہوتا ہے جسے \اصطلاح{اوہم}\فرہنگ{اوہم}\حاشیہب{Ohm}\فرہنگ{Ohm} پکارا اور \عددی{\si{\ohm}} سے ظاہر کیا جاتا ہے۔یوں اگر کسی مزاحمت پر \عددی{\SI{10}{\volt}} کا برقی دباو لاگو کرنے سے مزاحمت میں \عددی{\SI{5}{\ampere}} کی رو گزرے تب مزاحمت کی قیمت \عددی{R=\tfrac{10}{5}=\SI{2}{\ohm}} ہو گی۔

\begin{figure}
\centering
\includegraphics{figResistanceOhmLawPower}
\caption{اوہم کا قانون اور مزاحمتی ضیاع۔}
\label{شکل_مزاحمت_اوہم_قانون_مزاحمتی_ضیاع}
\end{figure}

شکل \حوالہ{شکل_مزاحمت_اوہم_قانون_مزاحمتی_ضیاع} میں برقی دور کے ساتھ مزاحمت \عددی{R} جڑی ہے۔مزاحمت کی دباو \عددی{v(t)} اور  رو \عددی{i(t)} ہیں۔  صفحہ \حوالہصفحہ{مساوات_بنیادی_طاقت_مساوی_دباو_ضرب_رو} پر مساوات \حوالہ{مساوات_بنیادی_طاقت_مساوی_دباو_ضرب_رو} کے تحت اس مزاحمت میں طاقت کا ضیاع
\begin{align*}
p(t)=v(t) i(t)
\end{align*}
ہو گا۔ اس مساوات میں برقی دباو \عددی{v(t)} میں قانون اوہم  پُر کرتے ہوئے
\begin{align*}
p(t)=R i(t) \times i(t)=R i^2(t)
\end{align*}
حاصل ہوتا ہے۔ اسی طرح طاقتی ضیاع کی مساوات  میں \عددی{i(t)} کی جگہ قانون اوہم استعمال کرتے ہوئے
\begin{align*}
p(t)=v(t) \times \frac{v(t)}{R}= \frac{v^2(t)}{R}
\end{align*}
حاصل ہوتا ہے۔مندرجہ بالا تین مساوات کو اکٹھے لکھتے ہیں۔
\begin{align}
p(t)=v(t) i(t)=R i^2(t)=\frac{v^2(t)}{R}  \quad \quad \text{\RL{مزاحمتی ضیاع}}
\end{align}
درج بالا مساوات مزاحمت کی طاقت دیتی ہے۔یہ طاقت حرارتی توانائی میں تبدیل ہوتی ہے جس سے مزاحمت کا درجہ حرارت بڑھتا ہے۔

مزاحمت کے علاوہ \اصطلاح{موصلیت}\فرہنگ{موصلیت}\حاشیہب{conductance}\فرہنگ{conductance} \عددی{G} بھی بہت مقبول ہے جہاں
\begin{align}\label{مساوات_مزاحمتی_موصلیت_اور_مزاحمت}
G=\frac{1}{R}
\end{align}
کے برابر ہے۔موصلیت کی اکائی \اصطلاح{سیمنز}\فرہنگ{سیمنز}\حاشیہب{Siemens}\فرہنگ{Siemens} \عددی{\si{\siemens}} ہے جہاں
\begin{align}
\SI{1}{\siemens}=\SI{1}{\ampere\per\volt}
\end{align}
کے برابر ہے۔مساوات \حوالہ{مساوات_مزاحمتی_موصلیت_اور_مزاحمت} کے استعمال سے  اوہم کے قانون کو
\begin{align}\label{مساوات_مزاحمتی_موصلیت_تعریف}
i(t)=G v(t)
\end{align}
اور مزاحمت کی طاقت کو
\begin{align}
p(t)=G v^2(t)=\frac{i^2(t)}{G}
\end{align}
لکھا جا سکتا ہے۔
%====================
\ابتدا{مثال}
ایک عدد مزاحمت پر \عددی{\SI{20}{\volt}} لاگو کرنے سے  مزاحمت میں \عددی{\SI{4}{\ampere}} پیدا ہوتی ہے۔ اس کی موصلیت دریافت کریں۔

حل:مساوات \حوالہ{مساوات_مزاحمتی_موصلیت_تعریف} کی مدد سے
\begin{align*}
G=\frac{i}{v}=\frac{4}{20}=\SI{0.2}{\siemens}
\end{align*}
حاصل ہوتا ہے۔یہی جواب، اوہم کے قانون سے  \عددی{R=\tfrac{20}{4}=\SI{5}{\ohm}} لکھتے اور \عددی{G=\tfrac{1}{R}=\SI{0.2}{\siemens}} سے بھی حاصل ہوتا ہے۔
\انتہا{مثال}
%=================

\begin{figure}
\centering
\begin{subfigure}{0.3\textwidth}
\centering
\includegraphics{figResistanceVariableResistance}
\caption*{(الف) متغیر مزاحمت}
\end{subfigure}%
%
\begin{subfigure}{0.3\textwidth}
\centering
\includegraphics{figResistanceShortCircuit}
\caption*{(ب) قصرِ دور}
\end{subfigure}%
%
\begin{subfigure}{0.3\textwidth}
\centering
\includegraphics{figResistanceOpenCircuit}
\caption*{(پ) کھلا دور }
\end{subfigure}%
\caption{قصر دور اور کھلا دور۔}
\label{شکل_مزاحمت_قصر_اور_کھلا_دور}
\end{figure}

شکل \حوالہ{شکل_مزاحمت_قصر_اور_کھلا_دور}-الف میں برقی دور کے ساتھ \اصطلاح{متغیر مزاحمت}\فرہنگ{متغیر مزاحمت}\فرہنگ{مزاحمت!متغیر}\حاشیہب{variable resistor}\فرہنگ{resistor!variable} جڑا دکھایا گیا ہے۔مزاحمت پر ترچھا تیر کھینچ کر متغیر مزاحمت کو ظاہر کیا جاتا ہے۔اگر متغیر مزاحمت کی قیمت کم کرتے کرتے صفر کر دی جائے تو کسی بھی رو \عددی{i(t)} کی صورت میں مزاحمت پر لاگو برقی دباو، قانون اوہم کے تحت \عددی{v= i(t) \times 0 =\SI{0}{\volt}} ہو گا۔یہ صورت حال شکل-ب میں دکھائی گئی ہے اور اس صورت کو \اصطلاح{قصر دور}\فرہنگ{ْقصر دور}\حاشیہب{short circuit}\فرہنگ{short circuit} کہتے ہیں۔ دو نقطوں کو موصل تار سے جوڑ کر قصر دور کیا جاتا ہے۔اس کے برعکس اگر متغیر مزاحمت کی قیمت لامحدود کر دی جائے تب کسی بھی دباو \عددی{v(t)} پر، قانون اوہم کے تحت \عددی{i=\tfrac{v(t)}{\infty}=\SI{0}{\ampere}} ہو گی۔ایسی صورت، جسے \اصطلاح{کھلا دور}\فرہنگ{کھلا دور}\حاشیہب{open circuit}\فرہنگ{open circuit} کہتے ہیں کو شکل-پ میں دکھائی گئی ہے۔کسی بھی دو نقطوں کو کھلا دور کرنے کا مطلب یہ ہے کہ ان  نقطوں کے مابین مزاحمت لامحدود کر دی جائے۔قصر دور پر ہر صورت صفر دباو پایا جاتا ہے جبکہ کھلا دور پر ہر صورت صفر رو پائی جاتی ہے۔
%===============
\ابتدا{مثال}\شناخت{مثال_مزاحمتی_مثال_طاقت_اکلتوتا_مزاحمت_الف}
شکل \حوالہ{شکل_مزاحمتی_اکلوتا_مزاحمت_کی_طاقت}-الف میں رو اور مزاحمتی طاقت دریافت کریں۔

حل:قانون اوہم سے مزاحمت میں رو
\begin{align*}
i=\frac{12}{3}=\SI{4}{\ampere}
\end{align*}
حاصل ہوتی ہے اور یوں مزاحمتی طاقت درج ذیل ہو گا۔
\begin{align*}
p=v \times i =12 \times 4=\SI{48}{\watt}
\end{align*}
\انتہا{مثال}
%======================

\begin{figure}
\centering
\begin{subfigure}{0.3\textwidth}
\includegraphics{figResistanceExampleA}
\caption*{(الف)}
\end{subfigure}%
%
\begin{subfigure}{0.3\textwidth}
\includegraphics{figResistanceExampleB}
\caption*{(ب)}
\end{subfigure}%
%
\begin{subfigure}{0.3\textwidth}
\includegraphics{figResistanceExampleC}
\caption*{(پ)}
\end{subfigure}
\caption{مزاحمتی ادوار مثال \حوالہ{مثال_مزاحمتی_مثال_طاقت_اکلتوتا_مزاحمت_الف} تا مثال \حوالہ{مثال_مزاحمتی_مثال_طاقت_اکلتوتا_مزاحمت_پ}}
\label{شکل_مزاحمتی_اکلوتا_مزاحمت_کی_طاقت}
\end{figure}

%========================
\ابتدا{مثال}\شناخت{مثال_مزاحمتی_مثال_طاقت_اکلتوتا_مزاحمت_ب}
شکل \حوالہ{شکل_مزاحمتی_اکلوتا_مزاحمت_کی_طاقت}-ب میں رو اور مزاحمتی طاقت دریافت کریں۔

حل:مزاحمت کا بالائی سرا مثبت ہے لہٰذا اس میں رو کی سمت اوپر سے نیچے ہو گی جو دکھلائے گئی سمت کے الٹ ہے۔اس طرح دی گئی سمت میں رو کی قیمت منفی ہو گی یعنی
\begin{align*}
i=-\frac{10}{5}=\SI{-2}{\ampere}
\end{align*}
جبکہ مزاحمت طاقت درج ذیل ہو گا۔
\begin{align*}
p=i^2 R=\SI{20}{\watt}
\end{align*}
\انتہا{مثال}
%========================

\ابتدا{مثال}\شناخت{مثال_مزاحمتی_مثال_طاقت_اکلتوتا_مزاحمت_پ}
شکل \حوالہ{شکل_مزاحمتی_اکلوتا_مزاحمت_کی_طاقت}-پ میں رو اور مزاحمتی دریافت کریں۔

حل:دور میں طاقت کی پیداوار اور ضیاع برابر لیتے ہوئے طاقت کی مساوات \عددی{p=vi} سے منبع کی رو حاصل کرتے ہیں۔
\begin{align*}
i=\frac{p}{v}=\frac{2.5}{5}=\SI{0.5}{\ampere}
\end{align*}
اوہم کے قانون سے مزاحمت کی قیمت درج ذیل حاصل ہوتی ہے۔
\begin{align*}
R=\frac{v}{i}=\frac{5}{0.5}=\SI{10}{\ohm}
\end{align*} 
\انتہا{مثال}
%========================

\begin{figure}
\centering
\begin{subfigure}{0.5\textwidth}
\includegraphics{figResistanceExampleD}
\caption*{(الف)}
\end{subfigure}%
%
\begin{subfigure}{0.5\textwidth}
\includegraphics{figResistanceExampleE}
\caption*{(ب)}
\end{subfigure}%
\caption{مزاحمتی ادوار مثال \حوالہ{مثال_مزاحمتی_دو_منبع_الف} تا مثال \حوالہ{مثال_مزاحمتی_دو_منبع_ب}}
\label{شکل_مزاحمتی_اکلوتا_مزاحمت_کئی_منبع_کی_طاقت}
\end{figure}

%================
\ابتدا{مثال}\شناخت{مثال_مزاحمتی_دو_منبع_الف}
شکل \حوالہ{شکل_مزاحمتی_اکلوتا_مزاحمت_کئی_منبع_کی_طاقت}-الف میں مزاحمت کی رو اور طاقت دریافت کریں۔

حل:قانون اوہم میں مزاحمت کی دباو \عددی{\SI{15}{\volt}-\SI{3}{\volt}=\SI{12}{\volt}} لیتے ہوئے رو حاصل کرتے ہیں۔
\begin{align*}
i=\frac{12}{10}=\SI{1.2}{\ampere}
\end{align*}
اسی طرح مزاحمت کی دباو \عددی{\SI{12}{\volt}} لیتے ہوئے اس کی طاقت درج ذیل حاصل ہوتی ہے۔یہی جواب \عددی{p=i^2 R} سے بھی حاصل ہو گا۔
\begin{align*}
p=v i =12 \times 1.2=\SI{14.4}{\watt}
\end{align*}
\انتہا{مثال}
%========================
\ابتدا{مثال}\شناخت{مثال_مزاحمتی_دو_منبع_ب}
شکل \حوالہ{شکل_مزاحمتی_اکلوتا_مزاحمت_کئی_منبع_کی_طاقت}-ب میں مزاحمت میں رو اور طاقت دریافت کریں۔دائیں منبع کی طاقت بھی دریافت کریں۔

حل:بائیں منبع کی طاقت اور دباو دیے گئے جس سے منبع کی مثبت سر سے خارج ہوتی رو کی قیمت \عددی{\SI{12}{\ampere}} حاصل ہوتی ہے۔مزاحمت کی دباو \عددی{\SI{8}{\volt}} ہے  لہٰذا اس کی مزاحمت
\begin{align*}
R=\frac{8}{12}=\frac{2}{3} \, \si{\ohm}
\end{align*}
ہو گی۔اس طرح مزاحمت کی طاقت
\begin{align*}
p=v i=8 \times 12=\SI{96}{\watt}
\end{align*}
ہو گا۔دائیں منبع کو طاقت فراہم کی جا رہی ہے جس کی قیمت درج ذیل ہے۔
\begin{align*}
p=v i =2 \times 12=\SI{24}{\watt}
\end{align*}
آپ دیکھ سکتے ہیں کہ طاقت کی پیدا وار اور ضیاع برابر ہیں۔
\انتہا{مثال}
%========================
\begin{figure}
\centering
\begin{subfigure}{0.33\textwidth}
\includegraphics{figResistanceResistorsPowerAndCurrentQuizA}
\caption*{(الف)}
\end{subfigure}%
%
\begin{subfigure}{0.33\textwidth}
\includegraphics{figResistanceResistorsPowerAndCurrentQuizB}
\caption*{(ب)}
\end{subfigure}%
\begin{subfigure}{0.33\textwidth}
\includegraphics{figResistanceResistorsPowerAndCurrentQuizC}
\caption*{(پ)}
\end{subfigure}%
\caption{مزاحمتی ادوار مشق \حوالہ{مثال_مزاحمتی_منبع_رو_طاقت_مشق_الف} تا مشق \حوالہ{مثال_مزاحمتی_منبع_رو_طاقت_مشق_پ}}
\label{شکل_مزاحمتی_اکلوتا_مزاحمت_طاقت_رو_مشق}
\end{figure}
%===================
\ابتدا{مشق}\شناخت{مثال_مزاحمتی_منبع_رو_طاقت_مشق_الف}
شکل \حوالہ{شکل_مزاحمتی_اکلوتا_مزاحمت_طاقت_رو_مشق}-الف میں مزاحمت کی رو اور طاقت حاصل کریں۔منبع کی طاقت بھی حاصل کریں۔

جوابات:\عددی{i=\SI{7}{\ampere}}، \عددی{p=\SI{127}{\watt}}، \عددی{p=\SI{-127}{\watt}}
\انتہا{مشق}
%======================
\ابتدا{مشق}
شکل \حوالہ{شکل_مزاحمتی_اکلوتا_مزاحمت_طاقت_رو_مشق}-ب میں مزاحمت کا دباو اور طاقت حاصل کریں۔منبع کی طاقت بھی دریافت کریں۔

جوابات:\عددی{v=\SI{24}{\volt}}، \عددی{p=\SI{48}{\watt}}، \عددی{p=\SI{-48}{\watt}}
\انتہا{مشق}
%======================
\ابتدا{مشق}\شناخت{مثال_مزاحمتی_منبع_رو_طاقت_مشق_پ}
شکل \حوالہ{شکل_مزاحمتی_اکلوتا_مزاحمت_طاقت_رو_مشق}-پ میں مزاحمت کی رو اور دباو حاصل کریں۔منبع کی طاقت دریافت کریں۔

جوابات:\عددی{i=\SI{2}{\ampere}}، \عددی{v=\SI{18}{\volt}}، \عددی{p=\SI{-36}{\watt}}
\انتہا{مشق}
%======================
%=======================================================================

\حصہ{قوانین کرچاف}
اوہم کے قانون سے ایک مزاحمت اور ایک منبع پر مبنی دور آسانی سے حل ہوتا ہے البتہ زیادہ  پرزوں پر مبنی دور حل کرتے ہوئے اس کا استعمال قدر مشکل ہوتا ہے۔زیادہ پرزہ جات کے ادوار \اصطلاح{قوانین کرچاف}\فرہنگ{قانون کرچاف}\حاشیہب{Kirchoff's laws}\فرہنگ{Kirchoff's laws}\حاشیہد{جرمنی کے گستاف روبرٹ کرچاف نے ان قوانین کو $\overset{1845}{\text{؁}}$ پیش کیا۔} کی مدد سے نہایت آسانی کے ساتھ حل ہوتے ہیں۔برقی دور میں برقی پرزوں کو موصل تاروں سے آپس میں جوڑا جاتا ہے۔موصل تار کی مزاحمت کو صفر اوہم تصور کیا جاتا ہے لہٰذا ان میں طاقت کا ضیاع صفر ہو گا۔یوں طاقت کی  پیداوار اور ضیاع صرف برقی پرزوں میں ممکن ہے۔

\begin{figure}
\centering
\begin{subfigure}{0.5\textwidth}
\centering
\includegraphics{figResistanceNodesAndLoopsA}
\caption*{(الف)}
\end{subfigure}%
%
\begin{subfigure}{0.5\textwidth}
\centering
\includegraphics{figResistanceNodesAndLoopsB}
\caption*{(ب)}%
\end{subfigure}
\caption{جوڑ اور دائرے۔}
\label{شکل_مزاحمتی_جوڑ_دائرہ}
\end{figure}

اس سے پہلے کہ ہم کرچاف کے قوانین پر غور کریں، ہم کچھ اصطلاحات مثلاً \اصطلاح{جوڑ}\فرہنگ{جوڑ}\حاشیہب{node}\فرہنگ{node}، \اصطلاح{دائرہ}\فرہنگ{دائرہ}\حاشیہب{loop}\فرہنگ{loop} اور \اصطلاح{شاخ}\فرہنگ{شاخ}\حاشیہب{branch}\فرہنگ{branch} جاننے کی کوشش کرتے ہیں۔شکل \حوالہ{شکل_مزاحمتی_جوڑ_دائرہ}-الف میں مزاحمت \عددی{R_2}، \عددی{R_3} اور منبع \عددی{V_1} نقطہ \عددی{n_0} پر جڑے ہیں۔اس نقطے کو \اصطلاح{جوڑ} \عددی{n_0} کہا جائے گا۔اسی شکل میں جوڑ \عددی{n_1}، \عددی{n_2} اور \عددی{n_3} بھی دکھائے گئے ہیں۔شکل \حوالہ{شکل_مزاحمتی_جوڑ_دائرہ}-ب میں اسی شکل کو قدر مختلف طریقے سے  دکھایا گیا ہے۔یہاں بھی ان \اصطلاح{جوڑوں} کی نشاندہی کی گئی ہے۔کسی بھی دو یا دو سے زیادہ پرزوں کو جوڑنے والے موصل تار کو \اصطلاح{جوڑ} تصور کیا جاتا ہے۔یوں شکل-الف میں جوڑ \عددی{n_0} نقطہ مانند ہے جبکہ شکل-ب میں نچلی پوری تار جوڑ \عددی{n_0} ہے۔جوڑ کو ظاہر کرنے والی تار کی لمبائی کچھ بھی ہو سکتی ہے۔

کسی بھی دور میں متعدد راستے ممکن ہیں۔ شکل \حوالہ{شکل_مزاحمتی_جوڑ_دائرہ} میں جوڑ \عددی{n_1} سے  مزاحمت \عددی{R_4} کے راستے جوڑ \عددی{n_3} تک پہنچا جا سکتا ہے  جہاں سے  منبع \عددی{I_1} کے راستے جوڑ \عددی{n_1} اور پھر مزاحمت \عددی{R_1} کے راستے واپس جوڑ \عددی{n_1} تک پہنچا جا سکتا ہے۔ایسا بند راستہ جو ابتدائی جوڑ پر ہی اختتام پذیر ہو \اصطلاح{بند راستہ} کہلاتا ہے۔ایسا بند راستہ جس پر کسی بھی جوڑ سے صرف ایک مرتبہ گزرا جائے \اصطلاح{دائرہ}\فرہنگ{دائرہ}\حاشیہب{loop}\فرہنگ{loop} کہلاتا ہے۔ اس طرح \عددی{R_1}، \عددی{I_1} اور \عددی{R_4} \اصطلاح{دائرہ} ہے۔اسی طرح \عددی{R_1}، \عددی{R_2}، \عددی{R_3} اور \عددی{R_4} بھی دائرہ ہے۔دائرے کی ایک اور مثال \عددی{V_1}، \عددی{R_4}، \عددی{I_1} اور \عددی{R_2} ہے۔اس کے برعکس \عددی{R_4}،\عددی{I_1}،\عددی{R_2}،\عددی{R_3}،\عددی{I_1} اور \عددی{R_1} دائرہ نہیں ہے چونکہ اس میں جوڑ \عددی{n_2} اور جوڑ \عددی{n_3} سے دو مرتبہ گزرا گیا۔ 

برقی دور میں ہر برقی پرزے کو \اصطلاح{شاخ}\فرہنگ{شاخ}\حاشیہب{branch}\فرہنگ{branch} کہتے ہیں۔ شکل \حوالہ{شکل_مزاحمتی_جوڑ_دائرہ}  میں کل چھ \عددی{(6)} \اصطلاح{شاخ} ہیں۔جوڑ \عددی{n_3} پر تین شاخ یعنی \عددی{R_4}، \عددی{R_3} اور \عددی{I_1} جڑتے ہیں۔آئیں اب \اصطلاح{قوانین کرچاف} کی بات کریں۔ 
%===============
\ابتدا{قانون}
کرچاف کا قانون برائے برقی رو کہتا ہے کہ کسی بھی جوڑ پر داخلی برقی رو کا مجموعہ خارجی برقی رو کے مجموعے کے عین برابر ہوتا ہے۔ 
\انتہا{قانون}
%====================

اگر جوڑ پر تمام رو کی سمت خارجی تصور کی جائے تب \اصطلاح{قانون کرچاف برائے رو}\فرہنگ{کرچاف!قانون برائے رو}\حاشیہب{Kirchoff's current law, KCL}\فرہنگ{Kirchoff! current law}\فرہنگ{KCL} کو درج ذیل لکھا جا سکتا ہے جہاں \عددی{i_j(t)} شاخ \عددی{j} میں جوڑ سے داخل ہوتی رو ہے۔
\begin{align}
\sum_{j=1}^{N} i_j(t)=0 \quad \quad \quad \text{\RL{کرچاف قانونِ رو}}
\end{align}
کرچاف کے قانون برائے برقی رو کو عموماً \اصطلاح{کرچاف قانونِ رو} کہا جائے گا۔

\باب{جوڑ اور دائری تجزیہ}
گزشتہ باب میں سادہ ترین ادوار کو کرچاف قوانین سے حل کرنا دکھایا گیا۔اس باب میں متعدد جوڑ اور متعدد دائروں والے ادوار کو کرچاف قوانین سے حل کرنا دکھایا جائے گا۔کرچاف قانون رو سے ہر جوڑ پر داخلی اور خارجی رو کے مجموعوں کو برابر پر کرتے ہوئے دور کے تمام جوڑوں پر دباو حاصل کیا جاتا ہے۔اس کے برعکس کرچاف قانون دباو کی مدد سے دور کے ہر دائرے میں دباو کے گھٹاو کے مجموعے کو دائرے میں دباو کے  بڑھاو کے مجموعے کے برابر پر کرتے ہوئے تمام دائروں کی رو حاصل کی جاتی ہے۔عموماً  دور یا تو کرچاف قانون دباو اور یا کرچاف قانون رو سے زیادہ آسانی سے حل ہوتا ہے۔آسان طریقہ چننا اس باب میں سکھایا جائے گا۔

\حصہ{تجزیہ جوڑ}

\باب{حسابی ایمپلیفائر}
شکل \حوالہ{شکل_حسابی_علامت_الف} میں \اصطلاح{حسابی ایمپلیفائر}\فرہنگ{حسابی ایمپلیفائر}\فرہنگ{ایمپلیفائر!حسابی}\حاشیہب{operational amplifier, opamp}\فرہنگ{opamp} کی علامت دکھائی گئی ہے۔حسابی ایمپلیفائر کے دو عدد داخلی سرے (پنیے) ہیں جنہیں \اصطلاح{مثبت داخلی سرا}\فرہنگ{مثبت داخلی پنیا}\فرہنگ{پنیا!مثبت داخلی}
\حاشیہب{non-inverting pin}\فرہنگ{non-inverting!pin} اور \اصطلاح{منفی داخلی سرا}\فرہنگ{منفی داخلی سرا}\فرہنگ{پنیا!منفی داخلی}\حاشیہب{inverting pin}\فرہنگ{inverting!pin} کہا جاتا ہے جبکہ اس کا ایک عدد \اصطلاح{خارجی سرا} (پنیا)  ہے۔اس کے علاوہ دو عدد \اصطلاح{طاقتی پنیے}\فرہنگ{پنیا!طاقتی}\فرہنگ{طاقتی پنیے}\حاشیہب{power pins}\فرہنگ{pins!power} حسابی ایمپلیفائر کو برقی طاقت فراہم کرنے کے لئے استعمال کئے جاتے ہیں جن میں ایک پر مثبت طاقتی دباو اور دوسرے پر منفی طاقتی دباو فراہم کی جاتی ہے۔حسابی ایمپلیفائر کے ادوار کرخوف کے قوانین سے با آسانی حل ہوتے ہیں۔ 

\begin{figure}
\centering
\begin{circuitikz}
\draw(0,0) node[op amp,yscale=-1](u1){};
\draw(u1.-)node[left]{\RL{منفی داخلی سرا}};
\draw(u1.+)node[left]{\RL{مثبت داخلی سرا}};
\draw(u1.out)node[right]{\RL{خارجی سرا}};
\draw(u1.up)--++(0,-\pin)node[below]{\RL{منفی طاقتی دباو}};
\draw(u1.down)--++(0,\pin)node[above]{\RL{مثبت طاقتی دباو}};
\end{circuitikz}%
\caption{حسابی ایمپلیفائر کی علامت۔}
\label{شکل_حسابی_علامت_الف}
\end{figure}
%========

شکل \حوالہ{شکل_حسابی_علامت_فراہم_طاقت}-الف میں حسابی ایمپلیفائر کو دو عدد منبع دباو سے طاقت فراہم کی گئی ہے جبکہ شکل-ب میں ایک عدد منبع دباو سے حسابی ایمپلیفائر کو طاقت کی فراہمی کی گئی ہے۔مثبت طاقتی دباو کو \عددی{V_{CC}} اور منفی طاقتی دباو کو \عددی{V_{EE}} لکھا جاتا ہے۔شکل-الف میں \عددی{V_{CC}=\SI{12}{\volt}} اور \عددی{V_{EE}=\SI{-10}{\volt}} ہیں۔عموماً ادوار میں مثبت اور منفی طاقتی دباو کے مطلق قیمتیں برابر \عددی{\abs{V_{CC}}=\abs{V_{EE}}} ہوتی ہیں۔حسابی ایمپلیفائر کے داخلی سروں پر \اصطلاح{برقی اشارات}\فرہنگ{اشارہ}\حاشیہب{electrical signals}\فرہنگ{signals} فراہم کئے جاتے ہیں۔
\begin{figure}
\centering
\begin{subfigure}{0.5\textwidth}
\centering
\begin{circuitikz}
\draw(0,0) node[op amp,yscale=-1](u1){};
\draw(u1.down)--++(0,\y)node[above right]{$V_{CC}=\SI{12}{\volt}$}--++(\xx,0)coordinate(kupper);
\draw(u1.up)--++(0,-\y)node[above right]{$V_{EE}=\SI{-10}{\volt}$}--++(\xx,0)coordinate(klower);
\draw(klower) to [american voltage source,l_={$\SI{10}{\volt}$}] ($(kupper)!0.5!(klower)$)coordinate(kmiddle) to [american voltage source,l_={$\SI{12}{\volt}$}] (kupper);
\draw(kmiddle) to [short,*-]++(\x/2,0)node[ground]{};
\end{circuitikz}%
\caption{دو عدد منبع دباو سے طاقت کی فراہمی۔}
\end{subfigure}%
\begin{subfigure}{0.5\textwidth}
\centering
\begin{circuitikz}
\draw(0,0) node[op amp,yscale=-1](u1){};;
\draw(u1.down)--++(0,\y-\dy)node[above right]{$V_{CC}=\SI{15}{\volt}$}--++(\xx,0)coordinate(kupper);
\draw(u1.up)--++(0,-\y+\dy)node[above right]{$V_{EE}=\SI{0}{\volt}$}node[ground]{}++(\xx,0)coordinate(klower);
\draw(klower)node[ground]{} to [american voltage source,l_={$\SI{15}{\volt}$}] (kupper);
\end{circuitikz}%
\caption{ایک عدد منبع دباو سے طاقت کی فراہمی۔}
\end{subfigure}%
\caption{حسابی ایمپلیفائر کو طاقت کی فراہمی کے طریقے۔}
\label{شکل_حسابی_علامت_فراہم_طاقت}
\end{figure}
%===========

حسابی ایمپلیفائر داخلی سروں پر فراہم کردہ اشارات \عددی{v_k} اور \عددی{v_n} میں فرق \عددی{v_d}
\begin{align}
v_d=v_k-v_n
\end{align}
 کو \عددی{A_d} گنّا بڑھا کر خارجی پنیا پر خارج کرتا ہے۔
\begin{align}
v_0=A_d v_d=A_d(v_k-v_n)
\end{align}
\عددی{v_d} کو \اصطلاح{داخلی تفرقی اشارہ}\فرہنگ{داخلی تفرقی اشارہ}\فرہنگ{اشارہ!داخلی تفرقی}\حاشیہب{difference signal}\فرہنگ{signal!difference} کہتے ہیں۔داخلی تفرقی  اشارہ بڑھانے کی صلاحیت کو \اصطلاح{افزائش}\فرہنگ{افزائش}\حاشیہب{gain}\فرہنگ{gain} کہتے اور \عددی{A_d} سے ظاہر کرتے ہیں۔حسابی ایمپلیفائر کے ادوار کے اشکال میں عموماً طاقتی پنیے نہیں دکھائے جاتے تا کہ اشکال  صاف ستھرے نظر آئیں ۔شکل \حوالہ{شکل_حسابی_فرق_کی_افزائش} میں ایسا ہی کرتے ہوئے حسابی ایمپلیفائر کے طاقتی پنیے نہیں دکھائے گئے ہیں۔
\begin{figure}
\centering
\begin{tikzpicture}
\draw (0,0) node[op amp,yscale=-1](u1){};
\draw(u1.out)++(-1,0)node{$A_d$};
\draw(u1.out) to [short,-o]++(\x/4,0)coordinate(kkout);
\draw(u1.-)--++(-\x/3,0)++(0,-\y)coordinate(klowerR)coordinate(klow)node[ground]{} to [american voltage source,l_={$v_n$}]++(0,\y);
\draw(klowerR)++(-\x/2,0)node[ground]{} to [american voltage source,l={$v_k$}]++(0,\y) |- (u1.+);
\draw($(u1.+)!0.5!(u1.-)$) node{$v_d$}coordinate(diffV);
\draw(diffV)node[shift={(0,0.3)}]{$+$}node[shift={(0,-0.3)}]{$-$};
\draw[](klow)++(3+\x/4,0.5)coordinate(koutL)node[ground]{};
\draw($(koutL)!0.5!(kkout)$)node[shift={(0.6,0)}]{$\begin{aligned}  &+ \\   &v_0=A_d v_d \\  &-\end{aligned}$};
\end{tikzpicture}
\caption{حسابی ایمپلیفائر داخلی اشارات کے فرق کو بڑھاتا ہے۔}
\label{شکل_حسابی_فرق_کی_افزائش}
\end{figure}
شکل \حوالہ{شکل_حسابی_نمونہ} میں حسابی ایمپلیفائر کے \اصطلاح{ریاضی نمونے}\فرہنگ{نمونہ!حسابی ایمپلیفائر}\فرہنگ{حسابی ایمپلیفائر!نمونہ}\حاشیہب{model}\فرہنگ{model} کا دور دکھایا گیا ہے جس سے حسابی ایمپلیفائر کی کارکردگی سمجھی جا سکتی ہے۔اس نمونے سے ظاہر ہے کہ حسابی ایمپلیفائر کے داخلی سروں پر داخلی رو \عددی{i_d}  اور داخلی تفرقی دباو \عددی{v_d} راست تناسب کا تعلق رکھتے ہیں۔یہ حقیقت داخلی پنیوں کے مابین مزاحمت \عددی{R_i=\tfrac{v_d}{i_i}} ظاہر کرتی ہے۔اسی طرح خارجی جانب بھی مزاحمتی اثر پایا جاتا ہے جسے \عددی{R_o}  سے ظاہر کیا گیا ہے۔
\begin{figure}
\centering
\begin{tikzpicture}
\draw(0,0)to [short,o-]++(\x/2,0)to [resistor,l_={$R_i$}]++(0,\y) to [short,-o]++(-\x-\x/2,0);
\draw(\x/2-\dx,\y/2)[left]node{$\begin{aligned} &+ \\ &v_d \\ &- \end{aligned}$};
\draw(2*\x,-\y/2) node[ground]{} to [american controlled voltage source,l_={$A_d v_d$}]++(0,\y+\y/2) to [resistor,l={$R_o$},-o]++(\x,0)node[right]{$v_0$};
\draw(0,-\y/2)node{$\begin{aligned} &+ \\ & v_n \\ &- \end{aligned}$};
\draw(0,-\y)node[ground]{};
\draw(-\x,0)node{$\begin{aligned} &+ \\ \\ \\  & v_k \\ \\ \\&- \end{aligned}$};
\draw(-\x,-\y)node[ground]{};
\end{tikzpicture}
\caption{حسابی ایمپلیفائر کا ریاضی نمونہ۔}
\label{شکل_حسابی_نمونہ}
\end{figure}    
آئیں حسابی ایمپلیفائر کا دور، اس کے ریاضی نمونے کی مدد سے حل کریں۔شکل \حوالہ{شکل_حسابی_ایمپلیفائر_دور_الف} میں حسابی ایمپلیفائر کے داخلی جانب منفی داخلی پنیے پر  اشارہ \عددی{v_s} اور مزاحمت \عددی{R_S} سلسلہ وار جوڑے گئے ہیں جبکہ مثبت پنیا کو زمین کے ساتھ جوڑا گیا ہے۔خارجی جانب حسابی ایمپلیفائر پر مزاحمتی بوجھ \عددی{R_B} ڈالا گیا ہے۔داخلی جانب تقسیم دباو سے
\begin{align*}
v_d=\left(\frac{R_i}{R_i+R_S}\right)v_s
\end{align*}
لکھا جائے گا۔خارجی جانب تقسیم دباو سے درج ذیل لکھا جاتا ہے۔
\begin{align*}
v_0=\left(\frac{R_B}{R_B+R_o}\right) A_d v_d
\end{align*}
 مندرجہ بالا دو مساوات کو ملاتے ہوئے
\begin{align}\label{مساوات_حسابی_افزائش_الف}
\frac{v_0}{v_s}= A_d \left(\frac{R_B}{R_B+R_o}\right)\left(\frac{R_i}{R_i+R_S}\right)=A_v
\end{align}
حاصل ہوتا ہے جہاں \عددی{A_v} بوجھ بردار حسابی ایمپلیفائر کی \اصطلاح{افزائشِ دباو}\فرہنگ{افزائش!دباو}\حاشیہب{voltage gain}\فرہنگ{voltage gain}\فرہنگ{gain!voltage} کہلاتی ہے۔ 
\begin{figure}
\centering
\begin{tikzpicture}
\draw(0,0)node[ground]{}to [short]++(\x/2,0)to [resistor,l_={$R_i$}]++(0,\y) to [resistor,l_={$R_S$}]++(-\x-\x/2,0)coordinate(vtop);
\draw(vtop)++(0,-\y) node[ground]{} to [american voltage source,l={$v_s$}]++(0,\y);
\draw(\x/2-\dx,\y/2)[left]node{$\begin{aligned} &+ \\ &v_d \\ &- \end{aligned}$};
\draw(1.5*\x,0) node[ground]{} to [american controlled voltage source,l_={$A_d v_d$}]++(0,\y) to [resistor,l={$R_o$}]++(\x,0)coordinate(kout) to [resistor,l={$R_B$}]++(0,-\y)node[ground]{};
\draw[](kout) to [short,*-o]++(\x/4,0)node[right]{$v_0$};
\end{tikzpicture}
\caption{حسابی ایمپلیفائر کا دور۔}
\label{شکل_حسابی_ایمپلیفائر_دور_الف}
\end{figure}

مساوات \حوالہ{مساوات_حسابی_افزائش_الف} میں دونوں قوسین کی قیمت اکائی سے کم ہے لہٰذا \عددی{A_v} کی قیمت \عددی{A_d} سے کم ہو گی۔زیادہ سے زیادہ \عددی{A_v} حاصل کرنے کی خاطر دونوں قوسین کی قیمت اکائی کے قریب ترین ہونا ضروری ہے۔ایسا تب ممکن ہو گا جب
\begin{gather}
\begin{aligned}
R_i \gg R_S\\
R_o \ll R_B
\end{aligned}
\end{gather}
ہوں۔

جدول \حوالہ{جدول_حسابی_نمونہ_متغیرات} میں حسابی ایمپلیفائر کے ریاضی نمونے کے متغیرات کی قیمتوں کے عمومی حدود دیے گئے ہیں۔آپ دیکھ سکتے ہیں کہ ایسے حسابی ایمپلیفائر دستیاب ہیں جن کی افزائش  \عددی{\SI{50000}{\volt\per\volt}} ہے اور ایسے ایمپلیفائر بھی دستیاب ہیں جن کی افزائش \عددی{\SI{1000000}{\volt\per\volt}} ہے۔
\begin{table} 
\caption{حسابی ایمپلیفائر کے نمونے کے متغیرات کی عمومی قیمتیں۔}
\centering
\begin{tabular}{ccc}
 $A_d (\si{\volt\per\volt})$ & $R_i (\si{\ohm})$ & $R_0 (\si{\ohm}) $\\
\hline
$\num{50000} - \num{1000000} $& $\num{e5} - \num{e12}$ & $2-200 \rule{0pt}{2.5ex} $ 
\end{tabular}
\label{جدول_حسابی_نمونہ_متغیرات}
\end{table}

%==================
\ابتدا{مثال}\شناخت{مثال_حسابی_ایمپلیفائر_الف}
شکل \حوالہ{شکل_حسابی_ایمپلیفائر_دور_الف} میں \عددی{A_d=\SI{100000}{\volt\per\volt}}، \عددی{R_i=\SI{e12}{\ohm}}، \عددی{R_o=\SI{100}{\ohm}}، \عددی{R_S=\SI{50}{\kilo\ohm}} اور \عددی{R_B=\SI{10}{\kilo\ohm}} ہیں۔ایمپلیفائر کی افزائش دباو \عددی{A_v} حاصل کریں۔

حل:مساوات \حوالہ{مساوات_حسابی_افزائش_الف} میں دی گئی قیمتیں پُر کرتے ہیں۔
\begin{align*}
A_v=\num{100000} \left(\frac{\num{10000}}{\num{10000}+100}\right)\left(\frac{\num{e12}}{\num{e12}+\num{50000}}\right)=\SI{99010}{\volt\per\volt}
\end{align*}
\انتہا{مثال}
%====================

حسابی ایمپلیفائر کا خارجی اشارہ  کسی بھی صورت مثبت طاقتی دباو \عددی{V_{CC}} سے زیادہ نہیں اور منفی طاقتی دباو \عددی{V_{EE}} سے کم نہیں ہو سکتا۔کئی اقسام کے حسابی ایمپلیفائر کا خارجی اشارہ طاقتی دباو سے چند ملی وولٹ کے فاصلے تک پہنچ پاتا ہے۔عموماً حسابی ایمپلیفائر ایسا کرنے کی صلاحیت نہیں رکھتے اور ان کا خارجی اشارہ مثبت طاقتی دباو سے \عددی{\SI{1}{\volt}} تا \عددی{\SI{3}{\volt}} کم اور منفی  طاقتی دباو سے \عددی{\SI{1}{\volt}} تا \عددی{\SI{3}{\volt}} زیادہ  ہی رہتا ہے۔
\begin{align}\label{مساوات_حسابی_خارجی_حدود}
V_{CC}-\Delta_+ > v_0 > V_{EE}+\Delta_-
\end{align}
آئیں اس حقیقت کے اثرات ایک مثال کی مدد سے دیکھیں۔
%===============
\ابتدا{مثال}
مثال \حوالہ{مثال_حسابی_ایمپلیفائر_الف} میں \عددی{v_s=\SI{50}{\micro\volt}}، \عددی{v_s=\SI{200}{\micro\volt}}، \عددی{v_s=\SI{2}{\volt}} اور \عددی{v_s=\SI{-150}{\micro\volt}} کی صورت میں \عددی{v_0} حاصل کریں۔حسابی ایمپلیفائر کے \عددی{\Delta_+ = \SI{1.5}{\volt}} اور \عددی{\Delta_-=\SI{1.2}{\volt}} تصور کریں  جبکہ طاقتی دباو \عددی{\SI{12}{\volt}} اور \عددی{\SI{-12}{\volt}} ہیں۔

حل:مساوات \حوالہ{مساوات_حسابی_خارجی_حدود} کے تحت خارجی اشارے کے حدود درج ذیل ہیں۔
\begin{gather}
\begin{aligned}\label{مساوات_حسابی_حدود_ب}
12-1.5 &> v_0 > -12+1.2\\
\SI{10.5}{\volt} & > v_0 > \SI{-10.8}{\volt}
\end{aligned}
\end{gather}
گزشتہ مثال میں ہم \عددی{A_v} کی قیمت حاصل کر چکے ہیں۔چونکہ \عددی{A_v=\tfrac{v_0}{v_s}} ہوتا ہے لہٰذا \عددی{v_s=\SI{50}{\micro\volt}} کی صورت میں
\begin{align*}
v_0=A_v v_s=99010 \times 50 \times 10^{-6}=\SI{4.95}{\volt} \quad \quad (v_s=\SI{50}{\micro\volt})
\end{align*}
ہو گا۔اسی طرح \عددی{v_s=\SI{200}{\micro\volt}} کی صورت میں جواب
\begin{align*}
v_0=99010\times 200\times 10^{-6}=\SI{19.8}{\volt}\quad \quad{\text{\RL{(اس جواب کو رد کیا جاتا ہے)}}}
\end{align*}
متوقع ہے۔مساوات \حوالہ{مساوات_حسابی_حدود_ب} کے تحت \عددی{v_0} کی قیمت \عددی{\SI{10.5}{\volt}} سے زیادہ نہیں ہو سکتی۔ایسی صورت میں حسابی ایمپلیفائر کوشش کرتا ہے کہ اس کا خارجی اشارہ \عددی{\SI{19.8}{\volt}} تک پہنچے لیکن ایسا ممکن نہیں ہے لہٰذا \عددی{v_0} بڑھتے بڑھتے \عددی{\SI{10.5}{\volt}} پر جا رکتا ہے۔یوں درست جواب درج ذیل ہے۔
\begin{align*}
v_0=\SI{10.5}{\volt} \quad \quad (v_s=\SI{200}{\micro\volt})
\end{align*}
داخلی اشارہ \عددی{\SI{2}{\volt}} ہونے کی صورت میں \عددی{v_0=\SI{198}{\kilo\volt}} متوقع ہے جو حسابی ایمپلیفائر کے لئے حاصل کرنا نا ممکن  ہے لہٰذا اب بھی
\begin{align*}
v_0=\SI{10.5}{\volt} \quad \quad (v_s=\SI{2}{\volt})
\end{align*}
ہو گا۔آخری داخلی اشارے کے لئے \عددی{v_0=99010\times (-150 \times 10^{-6})=\SI{-14.9}{\volt}} متوقع لیکن نا قابل حصول جواب ہے اور یوں 
\begin{align*}
v_0=\SI{-10.8}{\volt} \quad \quad (v_s=\SI{-150}{\micro\volt})
\end{align*}
ہو گا۔
\انتہا{مثال}
%===============
\ابتدا{مثال}\شناخت{مثال_حسابی_خطی_حدود_الف}
گزشتہ مثال میں مختلف داخلی اشارات مہیا کرتے ہوئے حسابی ایمپلیفائر کا خارجی اشارہ حاصل کیا گیا۔آپ سے گزارش ہے کہ داخلی اشارے کے وہ حدود حاصل کریں جن کے اندر رہتے ہوئے \عددی{v_0} اور \عددی{v_s} کا تعلق خطی ہو گا۔

حل: ہم دیکھتے ہیں کہ جب تک خارجی اشارہ مساوات \حوالہ{مساوات_حسابی_خارجی_حدود} میں دیے حدود کے اندر رہتا ہے اس وقت تک \عددی{v_0} اور \عددی{v_s} \اصطلاح{خطی تعلق}\فرہنگ{خطی تعلق}\حاشیہب{linear relationship}\فرہنگ{linear} \عددی{\tfrac{v_0}{v_s}=A_v} رکھتے ہیں۔مندرجہ بالا مثال میں بالائی حد
\begin{align*}
v_{s,\text{بلندتر}}= \frac{v_0}{A_d}=\frac{10.5}{99010}=\SI{106}{\micro\volt}
\end{align*}
پر اور نچلی حد
\begin{align*}
v_{s,\text{کمتر}}= \frac{v_0}{A_d}=\frac{-10.8}{99010}=\SI{-109}{\micro\volt}
\end{align*}
حاصل ہوتے ہیں۔یوں حسابی ایمپلیفائر اس وقت تک داخلی اشارے کو خطی طور پر بڑھاتا ہے جب تک داخلی اشارہ درج ذیل حدود میں رہے۔
\begin{align*}
\SI{-109}{\micro\volt} < v_s < \SI{106}{\micro\volt} 
\end{align*} 
ان حدود میں رہتے ہوئے \عددی{v_d} کے حدود شکل \حوالہ{شکل_حسابی_ایمپلیفائر_دور_الف} سے بذریعہ تقسیم دباو یوں حاصل ہوتے ہیں۔
\begin{align*}
v_{d,\text{بلندتر}}&=\frac{R_i v_s}{R_i+R_S}=\frac{10^{12} \times \SI{106}{\micro\volt}}{10^{12}+\num{5e4}} \approx \SI{106}{\micro\volt}\\
v_{d,\text{کمتر}}&=\frac{10^{12} \times (\SI{-109}{\micro\volt})}{10^{12}+\num{5e4}} \approx \SI{-109}{\micro\volt}
\end{align*}
یوں جب تک 
\begin{align}\label{مساوات_حسابی_خطی_حدود_داخلی_اشارہ}
\SI{-109}{\micro\volt} < v_d < \SI{106}{\micro\volt}
\end{align} 
رہے، حسابی ایمپلیفائر خطی رہتا ہے۔ 

\انتہا{مثال}
%======================
\ابتدا{مثال}\شناخت{مثال_حسابی_منفی_الف}
شکل \حوالہ{شکل_حسابی_لبریز} میں حسابی ایمپلیفائر کو یوں پلٹایا گیا ہے کہ اس کا مثبت سرا نیچے اور منفی سرا اوپر ہے۔اس کی افزائش دباو \عددی{A_v=\tfrac{v_0}{v_s}} حاصل کریں۔

\begin{figure}
\centering
\begin{subfigure}{1\textwidth}
\centering
\begin{tikzpicture}
\draw(0,0)node[op amp](u1){};
\draw(u1.-) to [resistor,l_={$R_1$}]++(-\x,0)++(0,-\y) node[ground]{} to [american voltage source,l={$v_s$}]++(0,\y);
\draw(u1.-) to [short,*-]++(0,\y/2) to [resistor,l={$R_2$}] ++(\x,0)-| (u1.out);
\draw(u1.out) to [short,*-o]++(\x/4,0)node[right]{$v_0$};
\draw(u1.+) to [short]++(0,-\y/2)node[ground]{};
\end{tikzpicture}
\caption{منفی ایمپلیفائر کا دور۔}
\end{subfigure}
\begin{subfigure}{1\textwidth}
\centering
\begin{tikzpicture}
\draw(0,0)node[ left]{$v_k$}node[ground]{}to [short]++(\x/2,0)to [resistor,l_={$R_i$}]++(0,\y) to [short,i<_={$i_d$}]++(-\x/2,0)coordinate(vn)node[above left]{$v_n$} to [resistor,l_={$R_1$}]++(-\x,0)coordinate(vtop);
\draw(vtop)++(0,-\y) node[ground]{} to [american voltage source,l={$v_s$}]++(0,\y);
\draw(\x/2-\dx,\y/2)[left]node{$\begin{aligned} &- \\ &v_d \\ &+ \end{aligned}$};
\draw(2*\x,-\y/2) node[ground]{} to [american controlled voltage source,l_={$A_d v_d$}]++(0,\y+\y/2) to [resistor,l={$R_o$}]++(\x,0)coordinate(kout);
\draw[](kout) to [short,*-o]++(\x/4,0)node[right]{$v_0$};
\draw(vn) to [short,*-]++(0,\y/2) to [resistor,l={$R_2$}]++(3*\x,0) -| (kout);
\end{tikzpicture}
\caption{منفی دور کا مساوی برقی دور۔}
\end{subfigure}
\caption{منفی ایمپلیفائر اور اس کا مساوی دور۔}
\label{شکل_حسابی_لبریز}
\end{figure}

حل:شکل \حوالہ{شکل_حسابی_لبریز}-الف میں حسابی ایمپلیفائر کی جگہ اس کا نمونہ نسب کرنے سے شکل-ب حاصل ہوتا ہے جسے کرخوف کے قوانین سے حل کیا جا سکتا ہے۔شکل-ب  ایمپلیفائر کا مساوی دور ہے۔منفی داخلی پنیے  پر کرخوف مساوات رو لکھتے ہیں
\begin{align*}
\frac{v_n-v_s}{R_1}+\frac{v_n}{R_i}+\frac{v_n-v_0}{R_2}&=0
\end{align*}
جسے
\begin{align*}
v_n\left(\frac{1}{R_1}+\frac{1}{R_i}+\frac{1}{R_2}\right)=\frac{v_s}{R_1}+\frac{v_o}{R_2}
\end{align*}
لکھتے ہوئے \عددی{v_n} حاصل کرتے ہیں۔
\begin{align}\label{مساوات_حسابی_منفی_پنیا_الف}
v_n=\frac{\frac{v_s}{R_1}+\frac{v_o}{R_2}}{\frac{1}{R_1}+\frac{1}{R_i}+\frac{1}{R_2}}
\end{align}
خارجی جوڑ پر کرخوف مساوات رو لکھتے ہیں
\begin{align*}
\frac{v_0-v_n}{R_2}+\frac{v_0-A_d v_d}{R_o}=0
\end{align*}
جس میں \عددی{v_d=-v_n} پُر کرتے اور ترتیب دیتے ہوئے
\begin{align*}
v_0\left(\frac{1}{R_2}+\frac{1}{R_o}\right)=v_n\left(\frac{1}{R_2}-\frac{A_d}{R_o}\right)
\end{align*}
لکھا جا سکتا ہے۔مساوات \حوالہ{مساوات_حسابی_منفی_پنیا_الف} کی مدد سے اس کو
\begin{align*}
v_0\left(\frac{1}{R_2}+\frac{1}{R_o}\right)&=\frac{\left(\frac{v_s}{R_1}+\frac{v_o}{R_2}\right)\left(\frac{1}{R_2}-\frac{A_d}{R_o}\right)}{\frac{1}{R_1}+\frac{1}{R_i}+\frac{1}{R_2}}
\end{align*}
یا
\begin{align*}
v_0\left(\frac{1}{R_2}+\frac{1}{R_o}\right)\left(\frac{1}{R_1}+\frac{1}{R_i}+\frac{1}{R_2}\right)&=\left(\frac{v_s}{R_1}+\frac{v_o}{R_2}\right)\left(\frac{1}{R_2}-\frac{A_d}{R_o}\right)
\end{align*}
یعنی
\begin{align*}
v_0\left(\frac{1}{R_2}+\frac{1}{R_o}\right)\left(\frac{1}{R_1}+\frac{1}{R_i}+\frac{1}{R_2}\right)-\frac{v_0}{R_o}\left(\frac{1}{R_2}-\frac{A_d}{R_o}\right)&=\frac{v_s}{R_1}\left(\frac{1}{R_2}-\frac{A_d}{R_o}\right)
\end{align*}
لکھا جا سکتا ہے جس کو حل کرتے ہوئے درج ذیل افزائش دباو \عددی{A_v} ملتی ہے۔
\begin{align*}
\frac{v_0}{v_s}=A_v=\frac{\frac{1}{R_1} \left(\frac{1}{R_2}-\frac{A_d}{R_o}\right)}{\left(\frac{1}{R_2}+\frac{1}{R_o}\right)\left(\frac{1}{R_1}+\frac{1}{R_i}+\frac{1}{R_2}\right)-\frac{1}{R_2}\left(\frac{1}{R_2}-\frac{A_d}{R_o}\right)}
\end{align*}
اس کو درج ذیل صورت میں لکھ سکتے ہیں۔
\begin{align}\label{مساوات_حسابی_منفی_الف}
\frac{v_0}{v_s}=A_v=\frac{-\frac{R_2}{R_1}}{1-\left[\frac{\left(\frac{1}{R_2}+\frac{1}{R_o}\right)\left(\frac{1}{R_1}+\frac{1}{R_i}+\frac{1}{R_2}\right)}{\left(\frac{1}{R_2}\right)\left(\frac{1}{R_2}-\frac{A_d}{R_o}\right)}\right]}
\end{align}
\انتہا{مثال}
%===============

مثال \حوالہ{مثال_حسابی_منفی_الف} میں  عمومی قیمتیں یعنی 
\begin{align*}
R_1=\SI{1}{\kilo\ohm}, \quad R_2=\SI{10}{\kilo\ohm}, \quad R_i=\SI{e8}{\ohm}, \quad R_o=\SI{100}{\ohm}, \quad A_d=\SI{e5}{\volt\per\volt}
\end{align*}
پُر کرتے ہیں۔
\begin{align*}
A_v&=\frac{-10}{1-\left[\frac{\left(0.0101\right)\left(0.001101\right)}{\left(0.0001\right)\left(0.0001-\frac{100000000}{100}\right)}\right]} \\
&=\SI{-9.999998888}{\volt\per\volt}
\end{align*}
آپ دیکھ سکتے ہیں کہ \عددی{\tfrac{A_d}{R_o}} جزو کے علاوہ تمام قوسین کی قیمتیں انتہائی چھوٹی ہیں۔آپ یہ بھی دیکھ سکتے ہیں کہ \عددی{A_d} کی قیمت زیادہ ہونے کی وجہ سے چکور قوسین کی قیمت تقریباً صفر کے برابر حاصل ہوتی ہے لہٰذا چکور قوسین کی قیمت کو رد کیا جا سکتا ہے اور یوں مساوات \حوالہ{مساوات_حسابی_منفی_الف} کو درج ذیل لکھا جا سکتا ہے۔
\begin{align}\label{مساوات_حسابی_غیر_کامل_حل}
A_v=\frac{v_0}{v_s}=-\frac{R_2}{R_1}
\end{align}
اس مساوات سے افزائش دباو 
\begin{align*}
A_v=-\frac{10000}{1000}=\SI{-10}{\volt\per\volt}
\end{align*}
حاصل ہوتی ہے۔بالائی دو جوابات تقریباً برابر ہیں جبکہ نچلا جواب انتہائی آسانی سے حاصل ہوا۔آئیں حسابی ایمپلیفائر حل کرنے کا انتہائی آسان طریقہ سیکھیں۔اس طریقے میں کامل حسابی ایمپلیفائر استعمال کیا جاتا ہے لہٰذا پہلے کامل حسابی ایمپلیفائر پر غور کرتے ہیں۔

\حصہ{کامل حسابی ایمپلیفائر}
ہم نے دیکھا کہ حسابی ایمپلیفائر کے داخلی مزاحمت \عددی{R_i} کی قیمت بڑی مقدار ہے۔اسی طرح \عددی{A_d} کی قیمت بھی بڑی مقدار ہے جبکہ \عددی{R_0} کی قیمت بیرونی لاگو مزاحمتوں کی نسبت سے بہت کم ہے۔\اصطلاح{کامل حسابی ایمپلیفائر}\فرہنگ{کامل حسابی ایمپلیفائر}\فرہنگ{حسابی ایمپلیفائر!کامل}\حاشیہب{ideal opamp}\فرہنگ{opamp!ideal} میں \عددی{R_i} اور \عددی{A_d} کو لامحدود جبکہ \عددی{R_0} کو صفر  تصور کیا جاتا ہے۔
\begin{align}
R_i& \to \infty \label{مساوات_حسابی_کامل_شرط_الف}\\
A_d& \to \infty  \label{مساوات_حسابی_کامل_شرط_ب}\\
R_o& \to 0  \label{مساوات_حسابی_کامل_شرط_پ}
\end{align}
مثال \حوالہ{مثال_حسابی_خطی_حدود_الف} میں ہم نے \عددی{v_d} کے وہ حدود حاصل کئے جن میں رہتے ہوئے \عددی{v_0} اور \عددی{v_s} کا تعلق خطی ہوتا ہے۔حسابی ایمپلیفائر کو خطی خطے میں ہی چلایا جاتا ہے۔مساوات \حوالہ{مساوات_حسابی_خطی_حدود_داخلی_اشارہ} میں یہ حدود دیے گئے ہیں جہاں سے واضح ہے کہ کسی بھی حقیقی دور میں  \عددی{v_d} کی مطلق قیمت تقریباً سو ملی وولٹ رہتی ہے جو نہایت کم مقدار ہے۔کامل حسابی ایمپلیفائر میں \عددی{v_d} کو صفر تصور کیا جاتا ہے۔
\begin{align}\label{مساوات_حسابی_کامل_شرط_ت}
v_d \to 0  
\end{align}
چونکہ \عددی{v_d=v_k-v_n} کے برابر ہے لہٰذا مندرجہ بالا مساوات کو درج ذیل صورت میں بھی لکھا جا سکتا ہے۔
\begin{align}\label{مساوات_حسابی_کامل_شرط_ٹ}
v_k=v_n
\end{align}
اگر \عددی{v_d=\SI{100}{\micro\volt}} اور {\عددی{R_i=\SI{e12}{\ohm}}} لیا جائے تو شکل \حوالہ{شکل_حسابی_لبریز}-ب میں \عددی{i_d=\tfrac{\SI{100}{\micro\volt}}{\num{e12} \, \si{\ohm}} \approx 0} حاصل ہوتا ہے۔یوں کامل حسابی ایمپلیفائر کے دونوں داخلی پنیوں پر رو کی قیمت صفر تصور کی جاتی ہے۔
\begin{align}\label{مساوات_حسابی_کامل_شرط_ث}
i_d=0
\end{align}

%==========================
\حصہ{منفی ایمپلیفائر}
گزشتہ مثال میں شکل \حوالہ{شکل_حسابی_لبریز} کو حل کیا گیا جسے یہاں بطور شکل \حوالہ{شکل_حسابی_کامل_حل_الف} دوبارہ پیش کیا گیا ہے۔کامل حسابی ایمپلیفائر تصور کرتے ہوئے اسے حل کرتے ہیں۔
\begin{figure}
\centering
\begin{tikzpicture}
\draw(0,0)node[op amp](u1){};
\draw(u1.-) to [short,i<_={$i_d$}]++(-\x/4,0)coordinate(kL)node[above left]{$v_n$} to [resistor,l_={$R_1$}]++(-\x,0)++(0,-\y) node[ground]{} to [american voltage source,l={$v_s$}]++(0,\y);
\draw(kL) to [short,*-]++(0,\y/2) to [resistor,l={$R_2$}] ++(\x+\x/4,0)-| (u1.out);
\draw(u1.out) to [short,*-o]++(\x/4,0)node[right]{$v_0$};
\draw(u1.+) to [short] ++(-\x/4,0)node[left]{$v_k$} to [short]++(0,-\y/2)node[ground]{};
\end{tikzpicture}
\caption{منفی ایمپلیفائر۔}
\label{شکل_حسابی_کامل_حل_الف}
\end{figure}
شکل میں داخلی دباو \عددی{v_k} اور \عددی{v_n} کی نشاندہی کی گئی ہے۔ساتھ ہی ساتھ حسابی ایمپلیفائر کی داخلی رو \عددی{i_d} بھی ظاہر کی گئی ہے۔کامل حسابی ایمپلیفائر کے ادوار حل کرتے ہوئے جوڑ \عددی{v_k} اور \عددی{v_n} پر کرخوف مساوات لکھ کر ان سے \عددی{v_k} اور \عددی{v_n} حاصل کریں۔مساوات \حوالہ{مساوات_حسابی_کامل_شرط_ٹ} کے تحت یہ قیمتیں برابر ہونی چاہیں لہٰذا انہیں برابر پُر کرتے ہوئے \عددی{v_0} کے لئے حل کریں۔آئیں ایسا ہی کرتے ہیں۔

چونکہ جوڑ \عددی{v_k} زمین کے ساتھ جڑا ہے لہٰذا اس کے لئے ہم لکھ سکتے ہیں۔
\begin{align*}
v_k=0
\end{align*}
جوڑ \عددی{v_n} پر مساوات \حوالہ{مساوات_حسابی_کامل_شرط_ث} کے تحت \عددی{i_d=0} لیتے ہوئے کرخوف قانون رو لکھتے ہیں۔
\begin{align*}
\frac{v_n-v_s}{R_1}+\frac{v_n-v_0}{R_2}=0
\end{align*}
چونکہ \عددی{v_k=0} ہے لہٰذا مساوات \حوالہ{مساوات_حسابی_کامل_شرط_ٹ} کے تحت \عددی{v_n=0} ہو گا۔یہ قیمت درج بالا مساوات میں پُر کرتے ہیں۔
\begin{align*}
\frac{0-v_s}{R_1}+\frac{0-v_0}{R_2}=0
\end{align*}
اس کو حل کرتے ہوئے درج ذیل حاصل ہوتا ہے۔
\begin{align}
\frac{v_0}{v_s}=-\frac{R_2}{R_1}
\end{align}
مساوات \حوالہ{مساوات_حسابی_غیر_کامل_حل} سے موازنہ کریں۔آپ دیکھ سکتے ہیں کہ کامل حسابی ایمپلیفائر تصور کرتے ہوئے جواب نہایت آسانی سے حاصل ہوتا ہے۔

شکل \حوالہ{شکل_حسابی_لبریز} کا دور  داخلی اشارہ \عددی{v_s} کو بڑھانے کے ساتھ ساتھ منفی سے ضرب بھی دیتا ہے لہٰذا اس دور کو \اصطلاح{منفی ایمپلیفائر}\فرہنگ{منفی ایمپلیفائر}\حاشیہب{inverting amplifier}\فرہنگ{amplifier!inverting} کہتے ہیں۔

عموماً \عددی{R_2>R_1} ہوتا ہے اور یوں خارجی اشارے کا حیطہ داخلی اشارے کے حیطے سے زیادہ ہوتا ہے۔افزائش سے مراد اشارے کا حیطہ بڑھانا ہی ہے البتہ ایسی کوئی وجہ نہیں کہ \عددی{R_1>R_2} نہ رکھا جا سکے۔ایسا کرنے سے خارجی اشارے کا حیطہ داخلی اشارے کے حیطے سے کم ہو گا۔دونوں صورتوں میں \عددی{-\tfrac{R_2}{R_1}} کو افزائش ہی کہا جاتا ہے۔
%==============================

مندرجہ بالا مثال میں افزائش \عددی{A_v} کی مقدار حسابی ایمپلیفائر کے ساتھ بیرونی جڑے مزاحمت \عددی{R_1} اور \عددی{R_2} پر منحصر ہے۔حسابی ایمپلیفائر کے  متغیرات \عددی{A_d}، \عددی{R_i} اور \عددی{R_o} کا افزائش پر کوئی اثر نہیں۔اس کا مطلب ہے کہ شکل \حوالہ{شکل_حسابی_کامل_حل_الف} میں حسابی ایمپلیفائر تبدیل کرنے سے افزائش تبدیل نہیں ہوتی۔حسابی ایمپلیفائر کے متغیرات درجہ حرارت، وقت اور دیگر طبعی اثرات کے ساتھ تبدیل ہوتے ہیں جبکہ مزاحمت کی قیمت میں تبدیلی انتہائی کم ہوتی ہے جسے رد کیا جا سکتا ہے۔چونکہ منفی ایمپلیفائر کی افزائش ان متغیرات پر منحصر نہیں لہٰذا اس کی افزائش اٹل تصور کی جا سکتی ہے۔اس کتاب میں یہاں سے آگے حسابی ایمپلیفائر کو کامل تصور کرتے ہوئے تمام ادوار حل کئے جائیں گے۔
%========================

\ابتدا{مثال}
منفی ایمپلیفائر کی افزائش \عددی{A_v=\SI{-15}{\volt\per\volt}} درکار ہے۔مزاحمتوں کی قیمتیں دریافت کریں۔اگر \عددی{v_s=\SI{-0.2}{\volt}} ہو تب \عددی{v_0} کیا ہو گا۔

حل:منفی ایمپلیفائر کے افزائش کا قلیہ \عددی{A_v=-\tfrac{R_2}{R_1}} ہے  جس سے \عددی{R_2=15 R_1} لکھا جا سکتا ہے۔ادوار تخلیق کرتے ہوئے عموماً ایسی صورت کا سامنا کرنا پڑتا ہے جہاں کلیات سے تمام متغیرات حاصل کرنا ممکن نہیں ہوتا۔موجودہ مثال  بھی ایسی ہے۔ایسی صورت میں کسی ایک متغیرہ یا ایک سے زیادہ  متغیرات کے قیمتیں چنی جاتی ہیں جس کے بعد بقایا متغیرات کو کلیات سے حاصل کیا جاتا ہے۔عموماً متغیرات چنتے وقت دیگر ضروریات کو مد نظر رکھا جاتا ہے۔

حسابی ایمپلیفائر کے ادوار میں مزاحمتوں کی قیمت \عددی{\SI{1}{\kilo\ohm}} تا \عددیء{\SI{100}{\kilo\ohm}} رکھتے ہوئے ٹھیک ادوار بنتے ہیں لہٰذا ہم
\begin{align*}
R_1=\SI{1}{\kilo\ohm}
\end{align*}
چن سکتے ہیں جس سے \عددی{R_2=\SI{15}{\kilo\ohm}} حاصل ہوتا ہے۔

دیے گیے اشارے کی صورت میں خارجی اشارہ
\begin{align*}
v_0=A_v v_s = -15 \times (-0.2)=\SI{3}{\volt}
\end{align*}
ہو گا۔
\انتہا{مثال}
%========================
\حصہ{مثبت ایمپلیفائر}
\اصطلاح{مثبت ایمپلیفائر}\فرہنگ{مثبت ایمپلیفائر}\حاشیہب{non-inverting amplifier}\فرہنگ{amplifier!non-inverting} کو شکل \حوالہ{شکل_حسابی_مثبت_ایمپلیفائر} میں دکھایا گیا ہے۔اس کی افزائش \عددی{\tfrac{v_0}{v_s}} حاصل کرتے ہیں۔

\begin{figure}
\centering
\begin{tikzpicture}
\draw(0,0) node[op amp](u1){};
\draw(u1.+) to [short]++(-\x/4,0)++(0,-\y)node[ground]{} to [american voltage source,l={$v_s$}]++(0,\y)node[above right]{$v_k$};
\draw(u1.-) to [short]++(-\x/4,0)coordinate(kL) to [resistor,l_={$R_1$}]++(-\x,0) node[ground]{};
\draw(kL)node[above right]{$v_n$} to [short,*-]++(0,\y/2) to [resistor,l={$R_2$}]++(\x+\x/4,0) -| (u1.out) to [short,*-o]++(\x/4,0)node[right]{$v_0$};
\end{tikzpicture}
\caption{مثبت ایمپلیفائر۔}
\label{شکل_حسابی_مثبت_ایمپلیفائر}
\end{figure}
مثبت داخلی پنیا کی مساوات لکھتے ہیں۔
\begin{align}\label{مساوات_حسابی_مثبت_ایمپلیفائر_الف}
v_k=v_s
\end{align}
منفی داخلی پنیا پر \عددی{i_d=0} لیتے ہوئے  کرخوف مساوات رو لکھ
\begin{align*}
\frac{v_n}{R_1}+\frac{v_n-v_0}{R_2}=0
\end{align*}
کر \عددی{v_n} کے لئے حل کرتے ہیں۔
\begin{align}\label{مساوات_حسابی_مثبت_ایمپلیفائر_ب}
v_n=\frac{\frac{v_0}{R_2}}{\frac{1}{R_1}+\frac{1}{R_2}}
\end{align}
مساوات \حوالہ{مساوات_حسابی_مثبت_ایمپلیفائر_الف} اور مساوات \حوالہ{مساوات_حسابی_مثبت_ایمپلیفائر_ب} میں حاصل کردہ \عددی{v_k} اور \عددی{v_n} کی قیمتیں برابر پُر کرتے ہیں۔
\begin{align*}
v_s=\frac{\frac{v_0}{R_2}}{\frac{1}{R_1}+\frac{1}{R_2}}
\end{align*}
اس کو \عددی{\tfrac{v_0}{v_s}} کے لئے حل کرتے ہوئے درج ذیل حاصل ہوتا ہے۔
\begin{align}\label{مساوات_حسابی_مثبت_ایمپلیفائر_افزائش}
A_v=\frac{v_0}{v_s}=1+\frac{R_2}{R_1}
\end{align}
%===========================

\ابتدا{مثال}
مثبت ایمپلیفائر میں \عددیء{R_1=\SI{2}{\kilo\ohm}} اور \عددیء{R_2=\SI{8}{\kilo\ohm}} ہیں جبکہ \عددی{v_s=0.5\sin 100 t} ہے۔ خارجی اشارہ حاصل کریں۔

حل:افزائش 
\begin{align*}
A_v=1+\frac{8000}{2000}=\SI{5}{\volt\per\volt}
\end{align*}
حاصل ہوتا ہے جبکہ خارجی اشارہ درج ذیل ہو گا۔
\begin{align*}
v_0=A_v v_s =5 \times 0.5 \sin 100 t =2.5 \sin 100 t  \quad (\si{\volt})
\end{align*}
\انتہا{مثال}
%===========================
\حصہ{مستحکم کار}
شکل \حوالہ{شکل_حسابی_مثبت_ایمپلیفائر} میں \عددی{R_1=\infty} اور \عددی{R_2=0} پُر کرنے سے  شکل \حوالہ{شکل_حسابی_مستحکم_کار} حاصل ہوتا ہے اور مساوات \حوالہ{مساوات_حسابی_مثبت_ایمپلیفائر_افزائش} سے
\begin{align*}
A_v=\frac{v_0}{v_s}=1+\frac{0}{\infty}=1
\end{align*}
یعنی
\begin{align}\label{مساوات_حسابی_مستحکم_کار_الف}
v_0=v_s
\end{align}
حاصل ہوتا ہے۔شکل \حوالہ{شکل_حسابی_مستحکم_کار} \اصطلاح{مستحکم کار}\فرہنگ{مستحکم کار}\حاشیہب{buffer}\فرہنگ{buffer} کہلاتا ہے۔مساوات \حوالہ{مساوات_حسابی_مستحکم_کار_الف} حاصل کرنے کی دوسری منطق یہ ہے کہ چونکہ \عددی{v_k=v_s} ہے لہٰذا \عددی{v_n} بھی \عددی{v_s} کے برابر ہو گا۔اب \عددی{v_n} اور \عددی{v_0} ایک ہی جوڑ کے دو نام ہیں لہٰذا
\begin{align}
v_0=v_s
\end{align}
ہو گا۔
\begin{figure}
\centering
\begin{tikzpicture}
\draw(0,0)node[op amp](u1){};
\draw(u1.+) to [short]++(-\x/4,0)node[above right]{$v_k$} to [short,-o]++(-\x/4,0)node[left]{$v_s$};
\draw(u1.-) to [short]++(-\x/4,0)node[above right]{$v_n$} to [short]++(0,\y/2) -| (u1.out) to [short,*-o] ++(\x/4,0) node[right]{$v_0$};
\end{tikzpicture}
\caption{مستحکم کار۔}
\label{شکل_حسابی_مستحکم_کار}
\end{figure}
%==========================================
\حصہ{منفی کار}
شکل \حوالہ{شکل_حسابی_منفی_کار} میں \عددی{R_1} دو جگہ نسب ہے۔اس کا مطلب ہے کہ دونوں جگہ پر \عددی{R_1} قیمت کے مزاحمت نسب ہیں۔اسی طرح دو جگہوں پر \عددی{R_2} نسب ہے جس کا مطلب ہے کہ ان جگہوں پر \عددی{R_2} قیمت کے مزاحمت نسب ہیں۔
\begin{figure}
\centering
\begin{tikzpicture}
\draw(0,0) node[op amp](u1){};
\draw(u1.-) to [short]++(-\x/4,0)coordinate(kTL)node[above right]{$v_n$} to  [resistor,-o,l_={$R_1$}]++(-\x,0) node[left]{$v_1$};
\draw(kTL) to [short,*-]++(0,\y/2) to [resistor,l={$R_2$}]++(\x+\x/4,0) -| (u1.out) to [short,*-o]++(\x/4,0)node[right]{$v_0$};
\draw(u1.+) to [short]++(-\x/4,0)coordinate(kBL)node[above right]{$v_k$} to [resistor,-o,l={$R_1$}]++(-\x,0)node[left]{$v_2$};
\draw(kBL) to [resistor,*-,l={$R_2$}]++(0,-\y)node[ground]{};
\end{tikzpicture}
\caption{منفی کار۔}
\label{شکل_حسابی_منفی_کار}
\end{figure}
مثبت اور منفی داخلی پنیوں کے کرخوف مساوات رو لکھتے ہیں۔
\begin{align*}
\frac{v_n-v_1}{R_1}+\frac{v_n-v_0}{R_2}&=0\\
\frac{v_k-v_2}{R_1}+\frac{v_k}{R_2}&=0
\end{align*}
ان سے \عددی{v_n} اور \عددی{v_k} حاصل کرتے ہیں۔
\begin{align*}
v_n&=\frac{\frac{v_1}{R_1}+\frac{v_0}{R_2}}{\frac{1}{R_1}+\frac{1}{R_2}}\\
v_k&=\frac{\frac{v_2}{R_1}}{\frac{1}{R_1}+\frac{1}{R_2}}
\end{align*}
\عددی{v_n} اور \عددی{v_k} کو برابر پُر کرتے ہیں۔
\begin{align*}
\frac{\frac{v_1}{R_1}+\frac{v_0}{R_2}}{\frac{1}{R_1}+\frac{1}{R_2}}=\frac{\frac{v_2}{R_1}}{\frac{1}{R_1}+\frac{1}{R_2}}
\end{align*}
مساوی نشان کے دونوں اطراف کسر کے نچلے حصے برابر ہونے کی وجہ سے کٹ جاتے ہیں۔بقایا مساوات کو \عددی{v_0} کے لئے حل کرتے ہوئے درج ذیل حاصل ہوتا ہے۔ 
\begin{align}
v_0=\frac{R_2}{R_1} \left(v_2-v_1\right)
\end{align} 
اس مساوات میں \عددی{R_1=R_2} کی صورت میں خارجی اشارہ داخلی اشارات کے فرق کے برابر ہے۔اسی لئے اس دور کو \اصطلاح{منفی کار}\فرہنگ{منفی کار}\حاشیہب{subtractor}\فرہنگ{subtractor} کہتے ہیں۔بیرونی مزاحمت برابر نہ ہونے کی صورت میں داخلی اشارات کے فرق کو \عددی{\tfrac{R_2}{R_1}} گنا بڑھایا بھی جاتا ہے۔

%=======================
\ابتدا{مشق}\شناخت{مشق_حسابی_منفی_کار}
شکل \حوالہ{شکل_حسابی_مثال_منفی_کار} میں \عددی{v_0} کی مساوات دریافت کریں۔خارجی اشارہ \عددی{v_1=\SI{-0.15}{\volt}} اور
 \عددی{v_2=0.7\cos 50t} کی صورت میں کیا ہو گی؟

\begin{figure}
\centering
\begin{tikzpicture}
\draw(0,0) node[op amp](u1){};
\draw(u1.-) to [short]++(-\x/4,0)coordinate(kTL)node[above right]{$v_n$} to  [resistor,-o,l_={$\SI{1}{\kilo\ohm}$}]++(-\x,0) node[left]{$v_1$};
\draw(kTL) to [short,*-]++(0,\y/2) to [resistor,l={$\SI{5}{\kilo\ohm}$}]++(\x+\x/4,0) -| (u1.out) to [short,*-o]++(\x/4,0)node[right]{$v_0$};
\draw(u1.+) to [short]++(-\x/4,0)coordinate(kBL)node[above right]{$v_k$} to [resistor,-o,l={$\SI{2}{\kilo\ohm}$}]++(-\x,0)node[left]{$v_2$};
\draw(kBL) to [resistor,*-,l={$\SI{12}{\kilo\ohm}$}]++(0,-\y)node[ground]{};
\end{tikzpicture}
\caption{مشق \حوالہ{مشق_حسابی_منفی_کار} کا دور۔}
\label{شکل_حسابی_مثال_منفی_کار}
\end{figure}

جوابات:\عددی{v_0=-5v_1+\frac{36v_2}{7}}، \عددی{v_0=\frac{3}{4}+3.6\cos 50t}
\انتہا{مشق}
%=============================
\حصہ{جمع کار}
\اصطلاح{جمع کار}\فرہنگ{جمع کار}\حاشیہب{adder}\فرہنگ{adder} کو شکل \حوالہ{شکل_حسابی_جمع_کار} میں دکھایا گیا ہے۔
\begin{figure}
\centering
\begin{tikzpicture}
\draw(0,0)node[op amp](u1){};
\draw(u1.+) to [short]++(-\x/4,0)node[above right]{$v_k$}node[ground]{};
\draw(u1.-) to [short]++(-\x/4,0)node[above right]{$v_n$}coordinate(kL) to [resistor,-o,l_={$R_3$}]++(-\x,0)node[left]{$v_3$};
\draw(kL) to [short,*-] ++(0,\y/2) coordinate(kM) to [resistor,-o,l_={$R_2$}]++(-\x,0)node[left]{$v_2$};
\draw(kM) to [short,*-] ++(0,\y/2) coordinate(kH) to [resistor,-o,l_={$R_1$}]++(-\x,0)node[left]{$v_1$};
\draw(kM) to [resistor,l={$R_0$}]++(\x+\x/4,0) -|(u1.out) to [short,*-o]++(\x/4,0)node[right]{$v_0$};
\end{tikzpicture}
\caption{جمع کار۔}
\label{شکل_حسابی_جمع_کار}
\end{figure}
داخلی پنیوں پر مساوات لکھتے ہیں۔
\begin{align*}
v_k&=0\\
\frac{v_n-v_1}{R_1}+\frac{v_n-v_2}{R_2}+\frac{v_n-v_3}{R_3}+\frac{v_n-v_0}{R_0}&=0
\end{align*}
چونکہ \عددی{v_k=0} ہے لہٰذا \عددی{v_n=0} ہو گا۔یہ قیمت مندرجہ بالا مساوات میں پر کرتے ہیں۔
\begin{align*}
\frac{0-v_1}{R_1}+\frac{0-v_2}{R_2}+\frac{0-v_3}{R_3}+\frac{0-v_0}{R_0}&=0
\end{align*}
اسے \عددی{v_0} کے لئے حل کرتے ہیں۔
\begin{align}
v_0=-R_0\left(\frac{v_1}{R_1}+\frac{v_2}{R_2}+\frac{v_3}{R_3}\right)
\end{align}
اگر تمام بیرانی مزاحمتوں کی قیمتیں برابر ہوں یعنی اگر \عددی{R_1=R_2=R_3=R_0} ہو تب مندرجہ بالا مساوات درج ذیل صورت اختیار کرتی ہے۔
\begin{align}
v_0=-\left(v_1+v_2+v_3\right)
\end{align}
اس مساوات کے تحت خارجی اشارہ تمام داخلی اشارات کے مجموعے کے منفی برابر ہے۔اسی لئے اس دور کو \اصطلاح{جمع کار} کہتے ہیں۔بیرونی مزاحمتیں برابر نہ ہونے کی صورت میں داخلی اشارات کے \اصطلاح{قدر}\فرہنگ{قدر}\حاشیہب{weightage}\فرہنگ{weightage} مختلف تصور کرتے ہوئے ان کا مجموعہ لیا جاتا ہے۔یوں پہلے اشارے کی قدر \عددی{\tfrac{R_0}{R_1}} لی گئی ہے جبکہ دوسرے اشارے کی قدر \عددی{\tfrac{R_0}{R_2}} لی گئی ہے۔شکل \حوالہ{شکل_حسابی_جمع_کار} میں مزید داخلی اشارات شامل کئے جا سکتے ہیں۔


%=====================
\ابتدا{مثال}\شناخت{مثال_حسابی_الف}
شکل \حوالہ{شکل_حسابی_دور_مثال_الف} میں \عددی{v_0} دریافت کریں۔

\begin{figure}
\centering
\begin{tikzpicture}
\draw(0,0) node[op amp](u1){};
\draw(u1.+) to [short]++(-\x/4,0)node[above right]{$v_k$} to [short,-o]++(-\x,0)node[left]{$\SI{5}{\volt}$};
\draw(u1.-) to [short]++(-\x/4,0)coordinate(kL)node[above right]{$v_n$} to [resistor,-o,l_={$\SI{2}{\kilo\ohm}$}]++(-\x,0) node[left]{$\SI{2}{\volt}$};
\draw(kL) to [short,*-]++(0,\y/2) to [resistor,l={$\SI{8}{\kilo\ohm}$}]++(\x+\x/4,0) -| (u1.out) to [short,*-o]++(\x/4,0)node[right]{$v_0$};
\end{tikzpicture}
\caption{مثال \حوالہ{مثال_حسابی_الف} کا دور۔}
\label{شکل_حسابی_دور_مثال_الف}
\end{figure}

حل:جوڑ \عددی{v_n} پر کرخوف مساوات لکھتے ہیں۔
\begin{align*}
\frac{v_n-2}{2000}+\frac{v_n-v_0}{8000}&=0
\end{align*}
جس سے
\begin{align*}
v_n=\frac{8+v_0}{5}
\end{align*}
حاصل ہوتا ہے۔جوڑ \عددی{v_k} کے لئے
\begin{align*}
v_k=5
\end{align*}
لکھا جا سکتا ہے۔دونوں جوڑ کی قیمتیں برابر پُر کرتے ہیں۔
\begin{align*}
\frac{8+v_0}{5}=5
\end{align*}
اس سے درج ذیل حاصل ہوتا ہے۔
\begin{align*}
v_0=\SI{17}{\volt}
\end{align*}
اگر مثبت طاقتی دباو اس قیمت سے زیادہ ہو تب یہی جواب درست ہو گا۔
\انتہا{مثال}
%======================

\حصہ{متوازن اور غیر متوازن صورت}
حسابی ایمپلیفائر \اصطلاح{مخلوط دور}\فرہنگ{مخلوط دور}\حاشیہب{integrated circuit, IC}\فرہنگ{integrated circuit, IC} ہے جس میں متعدد مزاحمت اور \اصطلاح{ٹرانزسٹر}\فرہنگ{ٹرانزسٹر}\حاشیہب{transistor}\فرہنگ{transistor} پائے جاتے ہیں۔ٹرانزسٹر کے بارے میں آپ \اصطلاح{برقیات}\فرہنگ{برقیات}\حاشیہب{electronics}\فرہنگ{electronics} کی کتاب میں پڑھیں گے۔

برقی اشارہ موصل تار میں تقریباً روشنی کی رفتار سے سفر کرتا ہے۔یوں ٹرانزسٹر کا داخلی اشارہ تبدیل ہونے کا اثر ٹرانزسٹر کے خارجی اشارے پر کچھ دیر بعد ہوتا ہے، اگرچہ یہ دورانیہ انتہائی کم ہوتا ہے۔حسابی ایمپلیفائر میں متعدد ٹرانزسٹر پائے جاتے ہیں لہٰذا حسابی ایمپلیفائر کے داخلی اشارے کے تبدیل ہونے کا اثر خارجی اشارے پر کچھ دیر بعد رونما ہو گا۔ اسی طرح خارجی اشارہ کسی ایک قیمت سے دوسری قیمت کے دباو تک پہنچتے ہوئے کچھ وقت لیتا ہے۔شکل \حوالہ{شکل_حسابی_اشارے_کا_سفر} میں مثبت ایمپلیفائر کے داخلی اشارے کو یک دم\حاشیہد{آپ یہاں سوال کر سکتے ہیں کہ اگر خارجی اشارہ یکدم تبدیل نہیں ہو سکتا تب داخلی اشارہ کس طرح یک دم تبدیل ہو سکتا ہے۔فی الحال بس فرض کریں کہ ایسا ہے۔} تبدیل ہوتا دکھایا گیا ہے۔مثبت ایمپلیفائر کی قلیہ افزائش
\begin{align}\label{مساوات_حسابی_مثبت_ایمپلیفائر_افزائش_دباو_ب}
A_v=1+\frac{R_2}{R_1}
\end{align}
 سے \عددی{A_v=\SI{2}{\volt\per\volt}} حاصل ہوتا ہے۔ شکل میں خارجی اشارہ بھی دکھایا گیا ہے جہاں خارجی اشارہ تبدیل ہونے کے دورانیے کو بڑھا چھڑھا کر پیش کیا گیا ہے۔حقیقت میں یہ دورانیہ چند مائیکرو سیکنڈ کا ہوتا ہے۔

\begin{figure}
\centering
\begin{subfigure}{1\textwidth}
\centering
\begin{tikzpicture}
\pgfmathsetmacro{\del}{0.2}
\pgfmathsetmacro{\sl}{0.5}
\draw[gray](0,0)--++(0,0.75)node[left]{$v_s$};
\draw[gray](0,0)--++(6,0)node[right]{$t$};
\draw(0,0)--++(1,0) --++(0,0.5) --++(3,0)coordinate(kin)--++(0,-0.5)--++(2,0);
\draw[gray](0,-2)--++(6,0)node[right]{$t$};
\draw[gray](0,-2)--++(0,1.3)node[left]{$v_0$};
\draw(0,-2)--++(1+\del,0)--++(\sl,1)--(4+\del,-2+1)coordinate(kout)--++(\sl,-1)--(6,-2);
\draw[stealth-](kin)++(0.1,0.1) to [out=45,in=180]++(0.5,0.3)node[right]{\RL{داخلی اشارہ}};
\draw[stealth-](kout)++(0.1,0.1) to [out=45,in=180]++(0.5,0.3)node[right]{\RL{خارجی اشارہ}};
%
\draw[gray,dashed,stealth-](1,0)--++(0,-2)++(0,-0.1) to [out=-90,in=0]++(-0.5,-0.3)node[left]{$t_1$};
\draw[gray,dashed,stealth-](1+\del,-2) --++(0,-0.5)node[below]{$t_2$};
\draw[gray,dashed](1+\del+\sl,-1)--++(0,-1)coordinate(kt);
\draw[gray,dashed,stealth-](kt) to [out=-90,in=0]++(0.5,-0.3)node[right]{$t_3$};
\end{tikzpicture}%
\end{subfigure}

\begin{subfigure}{1\textwidth}
\centering
\begin{tikzpicture}
\draw(0,0) node[op amp](u1){};
\draw(u1.+) to [short]++(-\x/4,0)++(0,-\y)node[ground]{} to [american voltage source,l={$v_s$}]++(0,\y)node[above right]{$v_k$};
\draw(u1.-) to [short]++(-\x/4,0)coordinate(kL) to [resistor,l_={$\SI{1}{\kilo\ohm}$}]++(-\x,0) node[ground]{};
\draw(kL)node[above right]{$v_n$} to [short,*-]++(0,\y/2)coordinate(kU) to [resistor,l={$\SI{1}{\kilo\ohm}$}]++(\x+\x/4,0) -| (u1.out) to [short,*-o]++(\x/4,0)node[right]{$v_0$};
\draw(kL)++(-1,-0.5)node{$R_1$};
\draw(kU)++(1.4,-0.5)node{$R_2$};
\end{tikzpicture}
\end{subfigure}
\caption{متوازن دور کی مثال۔}
\label{شکل_حسابی_اشارے_کا_سفر}
\end{figure}

آئیں شکل \حوالہ{شکل_حسابی_اشارے_کا_سفر} پر تفصیلاً غور کریں۔منفی جوڑ پر کرخوف مساوات رو
\begin{align*}
\frac{v_n}{R_1}+\frac{v_n-v_0}{R_2}=0
\end{align*}
یعنی
\begin{align}\label{مساوات_حسابی_مثبت_ایمپلیفائر_منفی_جوڑ}
v_n=\frac{R_1 v_0}{R_1+R_2}
\end{align}
ہے۔یہی مساوات شکل کو دیکھ کر تقسیم دباو کے قلیے سے بھی لکھی جا سکتی ہے۔

وقت \عددی{t=0} پر داخلی اشارہ \عددی{\SI{0}{\volt}} ہے اور یوں مساوات \حوالہ{مساوات_حسابی_مثبت_ایمپلیفائر_افزائش_دباو_ب} کے تحت \عددی{v_0=\SI{0}{\volt}} ہو گا۔مساوات \حوالہ{مساوات_حسابی_مثبت_ایمپلیفائر_منفی_جوڑ} میں \عددی{v_0=\SI{0}{\volt}} پُر کرتے ہوئے آپ دیکھ سکتے ہیں کہ \عددی{v_n=\SI{0}{\volt}} ہے لہٰذا\عددی{v_k} اور \عددی{v_n} برابر ہیں۔

لمحہ \عددی{t_1} پر داخلی اشارہ تبدیل ہو کر \عددی{\SI{1}{\volt}} ہو جاتا ہے۔ابتدائی طور پر داخلی اشارے کا اثر خارجی اشارے پر نہیں ہو گا لہٰذا \عددی{t_1} سے \عددی{t_2} تک  \عددی{v_0=\SI{0}{\volt}} ہی رہے گا۔مساوات \حوالہ{مساوات_حسابی_مثبت_ایمپلیفائر_منفی_جوڑ} میں \عددی{v_0=\SI{0}{\volt}} پُر کرنے سے \عددی{v_n=\SI{0}{\volt}} حاصل ہوتا ہے جبکہ \عددی{v_k=v_s=\SI{1}{\volt}} ہے۔یوں \عددی{t_1} تا \عددی{t_2} دورانیے میں \عددی{v_k} اور \عددی{v_n} برابر نہیں ہیں۔اس طرح \عددی{v_d=v_k-v_n=\SI{1}{\volt}} ہو گا۔تفرقی داخلی اشارہ \عددی{v_d} مثبت ہونے کی وجہ سے حسابی ایمپلیفائر خارجی اشارے کو مثبت طاقتی دباو کی جانب بڑھانا شروع کرتا ہے۔یہ رد  عمل خارجی اشارے پر  لمحہ \عددی{t_2} پر نمودار ہونا شروع ہوتا ہے۔لمحہ \عددی{t_3} پر خارجی اشارہ \عددی{\SI{2}{\volt}} پر پہنچتا ہے۔یوں \عددی{t_3} پر  مساوات \حوالہ{مساوات_حسابی_مثبت_ایمپلیفائر_منفی_جوڑ} کے تحت
\begin{align*}
v_n=\frac{1000 \times 2}{1000+1000}=\SI{1}{\volt}
\end{align*}
ہو گا۔یوں ایک مرتبہ پھر \عددی{v_k=v_n} یعنی \عددی{v_d=\SI{0}{\volt}} ہو گا۔داخلی تفرقی اشارہ صفر ہوتے ہی حسابی ایمپلیفائر خارجی اشارہ تبدیل کرنا روک دیتا ہے۔یوں \عددی{v_0=\SI{2}{\volt}} پر برقرار رہتا ہے۔

آئیں دیکھیں کہ اگر کسی وجہ سے \عددی{v_0} کی قیمت درکار قیمت (\عددی{\SI{2}{\volt}}) سے مختلف ہو تب حسابی ایمپلیفائر کا رد عمل کیا ہو گا۔فرض کریں  کہ کسی طرح \عددی{v_0=\SI{2.2}{\volt}} ہو جائے۔ایسی صورت میں مساوات \حوالہ{مساوات_حسابی_مثبت_ایمپلیفائر_منفی_جوڑ} کے تحت
\begin{align*}
v_n=\frac{1000 \times 2.2}{1000+1000}=\SI{1.1}{\volt}
\end{align*}
ہو گا جبکہ \عددی{v_k=\SI{1}{\volt}} ہے لہٰذا \عددی{v_d=\SI{-0.1}{\volt}} ہو گا جس کی وجہ سے حسابی ایمپلیفائر خارجی اشارے کو منفی طاقتی دباو کی جانب لے جانا شروع کرے گا یعنی \عددی{v_0} کی قیمت \عددی{\SI{2.2}{\volt}} سے گھٹنے شروع ہو جائے گی۔ہم دیکھتے ہیں کہ \عددی{v_k=\SI{1}{\volt}}  کی صورت میں حسابی ایمپلیفائر کسی بھی صورت \عددی{v_0} کی قیمت \عددی{\SI{2}{\volt}} سے زیادہ برداشت نہیں کرتا۔اسی طرح  \عددی{v_0} کی قیمت \عددی{\SI{2}{\volt}} سے کم ہونے کی صورت حال دیکھتے ہیں۔فرض کریں کہ \عددی{v_0=\SI{1.8}{\volt}} ہو جائے تب مساوات  \حوالہ{مساوات_حسابی_مثبت_ایمپلیفائر_منفی_جوڑ} کے تحت
\begin{align*}
v_n=\frac{1000 \times 1.8}{1000+1000}=\SI{0.9}{\volt}
\end{align*}
 اور \عددی{v_d=1-0.9=\SI{+0.1}{\volt}} ہو گا اور یوں حسابی ایمپلیفائر کا خارجی اشارہ  \عددی{v_0} مثبت طاقتی دباو کی جانب بڑھے گا۔آپ دیکھ سکتے ہیں کہ حسابی ایمپلیفائر کا خارجی اشارہ عین \عددی{\SI{2}{\volt}} پر آ رکتا ہے۔

مندرجہ بالا تبصرے سے آپ دیکھ سکتے ہیں کہ مثبت ایمپلیفائر کا خارجی اشارہ مساوات \حوالہ{مساوات_حسابی_مثبت_ایمپلیفائر_افزائش_دباو_ب} اور \عددی{v_s} کی قیمت سے تعین ہوتا ہے۔آپ نے دیکھا کہ خارجی اشارہ درکار قیمت پر ہی ٹہرتا ہے۔اس خاصیت کو \اصطلاح{متوازن}\فرہنگ{متوازن}\حاشیہب{stable}\فرہنگ{stable} صورت کہتے ہیں۔ 

\begin{figure}
\centering
\begin{tikzpicture}
\draw(0,0)node[op amp](u1){};
\draw(u1.+) to [short]++(-\x/4,0)coordinate(kL) to [resistor,l={$R_1$}]++(-\x,0) node[ground]{};
\draw(kL) to [short,*-]++(0,-\y/2) to [resistor,l_={$R_2$}]++(\x+\x/4,0)-|(u1.out) to [short,*-o]++(\x/4,0)node[right]{$v_0$};
\draw(u1.-) to [short,-o]++(-\x/4,0)node[left]{$v_s$};
\end{tikzpicture}
\caption{غیر متوازن دور کی مثال۔}
\label{شکل_حسابی_غیر_متوازن}
\end{figure}
آئیں اب شکل \حوالہ{شکل_حسابی_غیر_متوازن} میں دکھائے دور پر غور کرتے ہیں۔فرض کریں کہ \عددی{R_1=R_2=\SI{1}{\kilo\ohm}} اور \عددی{v_s=\SI{1}{\volt}} ہیں۔اگر \عددی{v_0=\SI{2}{\volt}} ہو تب منفی داخلی جوڑ پر کرخوف مساوات رو
\begin{align}
v_k=\frac{R_1 v_n}{R_1+R_2}
\end{align} 
سے \عددی{v_n=\SI{1}{\volt}} اور یوں \عددی{v_d=\SI{0}{\volt}} حاصل ہوتا ہے۔ایسا معلوم ہوتا ہے کہ یہی صحیح جواب ہے۔آئیں \عددی{v_0} کی قیمت میں تبدیلی کے اثرات دیکھیں۔فرض کریں کہ \عددی{v_0=\SI{2.2}{\volt}} ہو جاتا ہے۔ایسی صورت میں درج بالا مساوات کے تحت
\begin{align*}
v_k=\frac{1000\times 2.2}{1000+1000}=\SI{1.1}{\volt}
\end{align*}
اور \عددی{v_d=v_k-v_n=1.1-1=\SI{0.1}{\volt}} ہو گا۔یوں خارجی اشارہ بڑھنے شروع ہو گا لیکن خارجی اشارہ جتنا بڑھتا ہے \عددی{v_k} اور \عددی{v_d} کی قیمتیں اتنی ہی  زیادہ ہوتی چلی جاتی ہیں۔آخر کار \عددی{v_0=V_{CC}-\Delta_+} تک پہنچ کر رک جائے گا اور یہیں رکا رہے گا۔اس کے برعکس \عددی{v_0} کی قیمت دو وولٹ سے کم ہونے کی صورت میں \عددی{v_d} منفی ہو گا لہٰذا خارجی اشارہ منفی جانب چل پڑیگا اور آخر کار \عددی{V_{EE}+\Delta_-} پر جا رکے گا۔

آپ دیکھ سکتے ہیں کہ قوانین کرخوف سے شکل  \حوالہ{شکل_حسابی_غیر_متوازن} کا حاصل جواب (یعنی \عددی{v_0=\SI{2}{\volt}}) \اصطلاح{غیر متوازن}\فرہنگ{غیر متوازن}\حاشیہب{unstable}\فرہنگ{unstable} صورت کو ظاہر کرتی  ہے جو برقرار نہیں رہ  سکتی۔ یوں حسابی ایمپلیفائر کے ادوار حل کرتے ہوئے دور کا متوازن یا غیر متوازن ہونے پر غور ضروری ہے۔اس کتاب میں ہم صرف متوازن ادوار پر غور کریں گے جو قوانین کرخوف سے قابل حل ہوں گے۔
%======================

\حصہ{موازنہ کار}
حسابی ایمپلیفائر کی ایک مخصوص صورت کو \اصطلاح{موازنہ کار}\فرہنگ{موازنہ کار}\حاشیہب{comparator}\فرہنگ{comparator} کہتے ہیں۔\حاشیہط{لکھنا باقی ہے}
%=========================

\حصہ{آلاتی ایمپلیفائر}
\اصطلاح{آلاتی ایمپلیفائر} کو شکل \حوالہ{شکل_حسابی_آلاتی_ایمپلیفائر} میں دکھایا گیا ہے۔باریک اور حساس اشارات کی افزائش کے لئے اسے استعمال کیا جاتا ہے۔
\begin{figure}
\centering
\begin{tikzpicture}
\draw(2*\xx,\yy) node[op amp](u3){$u3$};
\draw(0,0) node[op amp](u2){$u2$};
\draw(0,2*\yy)node[op amp,yscale=-1](u1){};
\draw(u1.out)++(-\x/2,0.1)node{$u1$};
\draw(u2.+)node[above left]{$v_{k2}$} to [short,-o]++(-\x/2,0)node[left]{$v_2$};
\draw(u1.+)node[above left]{$v_{k1}$} to [short,-o]++(-\x/2,0)node[left]{$v_1$};
\draw(u2.-)node[left]{$v_{n2}$} to [short]++(0,\y/2)to [resistor,i>^={$i_1$},l={$R_1$}]++(0,\y) -- (u1.-);
\draw(u2.-)++(0,\y/2) to [resistor,i_<={$i_1$},l={$R_2$},*-]++(\x,0) -|(u2.out);
\draw(u1.-)node[left]{$v_{n1}$}++(0,-\y/2) to [resistor,i^={$i_1$},l_={$R_2$},*-]++(\x,0) -|(u1.out);
\draw(u2.out)node[right]{$v_{02}$}++(0,\y/2) to [resistor,l={$R_3$},*-]++(\x,0) -| (u3.+);
\draw(u1.out)node[right]{$v_{01}$}++(0,-\y/2) to [resistor,l={$R_3$},*-]++(\x,0) -| (u3.-);
\draw(u3.+)++(0,-\y/2) to [resistor,l={$R_4$},*-]++(0,-\y)node[ground]{};
\draw(u3.-)++(0,\y/2) to [resistor,*-,l={$R_4$}]++(\x,0) -|(u3.out) to [short,*-o]++(\x/4,0)node[right]{$v_0$};
\draw [decorate,decoration={brace,amplitude=10pt}](u1.out)++(0,0.5) -- ++(3*\x,0) node [shift={(0,0.75)},midway] {\RL{منفی کار}};
\end{tikzpicture}
\caption{آلاتی ایمپلیفائر۔}
\label{شکل_حسابی_آلاتی_ایمپلیفائر}
\end{figure}
اس کی کارکردگی پر غور کرتے ہیں۔

چونکہ \عددی{v_{k2}=v_2} ہے لہٰذا \عددی{v_{n2}=v_2} ہو گا۔اسی طرح \عددی{v_{k1}=v_1} کی بنا \عددی{v_{n1}=v_1} ہو گا۔اس طرح مزاحمت \عددی{R_1} پر دباو 
\begin{align*}
v_{n2}-v_{n1}=v_2-v_1
\end{align*}
ہو گا جس سے اس کی رو قانون اوہم سے
\begin{align}\label{مساوات_حسابی_آلاتی_رو}
i_1=\frac{v_2-v_1}{R_1}
\end{align}
لکھی جا سکتی ہے۔حسابی ایمپلیفائر کی داخلی رو صفر لیتے ہوئے صاف ظاہر ہے کہ \عددی{i_1} نچلی اور بالائی مزاحمت \عددی{R_2} سے گزرے گی۔یوں بالائی اور نچلی \عددی{R_2} کے دو سروں کے مابین دباو
\begin{align*}
v_{n1}-v_{01}&=v_{1}-v_{01}=i_1 R_2\\
v_{02}-v_{n2}&=v_{02}-v_2=i_1 R_2
\end{align*}
ہو گا جس سے درج ذیل حاصل ہوتے ہیں۔
\begin{align}\label{مساوات_حسابی_آلاتی_دباو}
v_{01}&=v_1-i_1 R_2\\
v_{02}&=v_2+i_1 R_2
\end{align}
شکل \حوالہ{شکل_حسابی_آلاتی_ایمپلیفائر} میں \عددی{R_3}، \عددی{R_4} اور حسابی ایمپلیفائر \عددی{u3} منفی کار کا دور ہے جس کی مساوات درج ذیل ہے۔
\begin{align*}
v_0=\frac{R_4}{R_3} (v_{02}-v_{01})
\end{align*}
اس میں مساوات \حوالہ{مساوات_حسابی_آلاتی_رو} اور مساوات \حوالہ{مساوات_حسابی_آلاتی_دباو} استعمال کرتے ہوئے آلاتی ایمپلیفائر کی مساوات ملتی ہے۔
\begin{align}
v_0=\frac{R_4}{R_3}\left(1+\frac{2R_2}{R_1}\right)(v_2-v_1) \quad \quad \text{\RL{آلاتی ایمپلیفائر کی مساوات}}
\end{align}

%==============================
%===================================
\حصہء{سوالات}
%======================
\ابتدا{سوال}\شناخت{سوال_حسابی_داخلی_خارجی_اشارہ_الف}
ایک حسابی ایمپلیفائر کی افزائش \عددی{6} ہے۔اس کا داخلی اشارہ شکل \حوالہ{شکل_سوال_حسابی_داخلی_خارجی_اشارہ_الف}-الف میں دیا گیا ہے۔خارجی اشارے کا خط کھینچیں۔
\begin{figure}
\centering
\begin{subfigure}{0.5\textwidth}
\centering
\begin{tikzpicture}
\begin{axis}[small,xlabel={$t\,(\si{\second})$},ylabel={$v_s\,(\si{\milli\volt})$},xtick={0,1,3,4,5,6,7.5},xticklabels={$0$,$1$,$3$,$4$,$5$,$6$,$7.5$},ytick={-1,0,1,2},yticklabels={$-1$,$0$,$1$,$2$},ylabel style={rotate=-90},ylabel style={at={(axis description cs:0,1.05)}}]
\addplot[] plot coordinates {(-0.5,0)(0,0) (1,1) (3,1) (3,0) (4,0) (4,-1) (5,-1) (6,2) (7.5,0) (8,0)};
\end{axis}
\end{tikzpicture}
\caption*{(الف)}
\end{subfigure}%
\begin{subfigure}{0.5\textwidth}
\centering
\begin{tikzpicture}
\begin{axis}[small,xlabel={$t\,(\si{\second})$},ylabel={$v_0\,(\si{\milli\volt})$},xtick={0,1,3,4,5,6,7.5},xticklabels={$0$,$1$,$3$,$4$,$5$,$6$,$7.5$},ytick={-1,0,1,2},yticklabels={$-6$,$0$,$6$,$12$},ylabel style={rotate=-90},ylabel style={at={(axis description cs:0,1.05)}}]
\addplot[] plot coordinates {(-0.5,0)(0,0) (1,1) (3,1) (3,0) (4,0) (4,-1) (5,-1) (6,2) (7.5,0) (8,0)};
\end{axis}
\end{tikzpicture}
\caption*{(ب)}
\end{subfigure}%
\caption{سوال \حوالہ{سوال_حسابی_داخلی_خارجی_اشارہ_الف} کے اشکال۔}
\label{شکل_سوال_حسابی_داخلی_خارجی_اشارہ_الف}
\end{figure}

جواب:شکل-ب
\انتہا{سوال}
%======================
\ابتدا{سوال}\شناخت{سوال_حسابی_داخلی_خارجی_اشارہ_ب}
ایک حسابی ایمپلیفائر کی افزائش \عددی{-10} ہے۔اس کا داخلی اشارہ شکل \حوالہ{شکل_سوال_حسابی_داخلی_خارجی_اشارہ_ب}-الف میں دیا گیا ہے۔خارجی اشارے کا خط کھینچیں۔
\begin{figure}
\centering
\begin{subfigure}{0.5\textwidth}
\centering
\begin{tikzpicture}
\begin{axis}[small,xlabel={$t\,(\si{\second})$},ylabel={$v_s\,(\si{\milli\volt})$},xtick={0,2,3,4,5,6,7,8},xticklabels={$0$,$2$,$3$,$4$,$5$,$6$,$7$,$8$},ytick={-2,-1.5,0,1,2,3},yticklabels={$-2$,$-1.5$,$0$,$1$,$2$,$3$},ylabel style={rotate=-90},ylabel style={at={(axis description cs:0,1.05)}}]
\addplot[] plot coordinates {(-0.5,0)(0,0)(1,2)(3,3)(4,2)(5,2)(6,-1.5)(7,-1.5)(7,-2) (8,0)(8.5,0)};
\end{axis}
\end{tikzpicture}
\caption*{(الف)}
\end{subfigure}%
\begin{subfigure}{0.5\textwidth}
\centering
\begin{tikzpicture}
\begin{axis}[small,xlabel={$t\,(\si{\second})$},ylabel={$v_0\,(\si{\milli\volt})$},xtick={0,2,3,4,5,6,7,8},xticklabels={$0$,$2$,$3$,$4$,$5$,$6$,$7$,$8$},ytick={2,1.5,0,-1,-2,-3},yticklabels={$20$,$15$,$0$,$-10$,$-20$,$-30$},ylabel style={rotate=-90},ylabel style={at={(axis description cs:0,1.05)}}]
\addplot[] plot coordinates {(-0.5,0)(0,0)(1,-2)(3,-3)(4,-2)(5,-2)(6,1.5)(7,1.5)(7,2) (8,0)(8.5,0)};
\end{axis}
\end{tikzpicture}
\caption*{(ب)}
\end{subfigure}%
\caption{سوال \حوالہ{سوال_حسابی_داخلی_خارجی_اشارہ_ب} کے اشکال۔}
\label{شکل_سوال_حسابی_داخلی_خارجی_اشارہ_ب}
\end{figure}

جواب:شکل-ب
\انتہا{سوال}
%====================
\ابتدا{سوال}\شناخت{سوال_حسابی_داخلی_خارجی_اشارہ_پ}
ایک حسابی ایمپلیفائر کی افزائش \عددی{-20} ہے اور اس کو \عددی{\SI{\mp15}{\volt}} طاقت فراہم کی جا رہی ہے۔اس کا خارجی اشارہ شکل \حوالہ{شکل_سوال_حسابی_داخلی_خارجی_اشارہ_پ}-الف میں دیا گیا ہے۔داخلی اشارے کا خط کھینچیں۔
\begin{figure}
\centering
\begin{subfigure}{0.5\textwidth}
\centering
\begin{tikzpicture}
\begin{axis}[small,xlabel={$t\,(\si{\second})$},ylabel={$v_0\,(\si{\volt})$},xtick={0,4,6,7,9},xticklabels={$0$,$4$,$6$,$7$,$9$},ytick={-2,0,3},yticklabels={$-2$,$0$,$3$},ylabel style={rotate=-90},ylabel style={at={(axis description cs:0,1.05)}}]
\addplot[] plot coordinates {(0,0)(4,3)(6,-2)(7,-2)(10,1)};
\end{axis}
\end{tikzpicture}
\caption*{(الف)}
\end{subfigure}%
\begin{subfigure}{0.5\textwidth}
\centering
\begin{tikzpicture}
\begin{axis}[small,xlabel={$t\,(\si{\second})$},ylabel={$v_s\,(\si{\volt})$},xtick={0,1,3,4,5,6,7,9},xticklabels={$0$,$1$,$3$,$4$,$5$,$6$,$7$,$9$},ytick={-0.15,0,0.1},yticklabels={$-0.15$,$0$,$0.1$},ylabel style={rotate=-90},ylabel style={at={(axis description cs:0,1.05)}}]
\addplot[] plot coordinates {(0,0)(4,-0.15)(6,0.1)(7,0.1)(10,-0.05)};
\end{axis}
\end{tikzpicture}
\caption*{(ب)}
\end{subfigure}%
\caption{سوال \حوالہ{سوال_حسابی_داخلی_خارجی_اشارہ_پ} کے اشکال۔}
\label{شکل_سوال_حسابی_داخلی_خارجی_اشارہ_پ}
\end{figure}

جواب:شکل-ب
\انتہا{سوال}
%======================
\ابتدا{سوال}
کامل حسابی ایمپلیفائر کی داخلی مزاحمت لامتناہی اور خارجی مزاحمت صفر ہوتے ہیں۔ان کا داخلی رو اور داخلی دباو پر کیا اثر پایا جاتا ہے۔

حل:داخلی رو صفر ہوتی ہے۔ داخلی دباو صفر ہوتا ہے۔
\انتہا{سوال}
%=========================
\ابتدا{سوال}\شناخت{سوال_حسابی_ایمپلیفائر_الف}
شکل \حوالہ{شکل_سوال_حسابی_ایمپلیفائر_الف} کا افزائش دباو دریافت کریں اور \عددی{v_s=0.23\sin 10t\,\si{\volt}} کی صورت میں \عددی{v_0} حاصل کریں۔
\begin{figure}
\centering
\begin{tikzpicture}
\draw(0,0) node[op amp](u1){};
\draw(u1.+) to [short]++(-\x/8,0) node[ground]{};
\draw(u1.-) to [short]++(-\x/8,0) to [resistor,l_={$\SI{2.2}{\kilo\ohm}$}]++(-\x,0) ++(0,-\y) node[ground]{} to [american voltage source,l={$v_s$}]++(0,\y);
\draw(u1.-)++(-\x/8,0) to [short,*-] ++(0,\y/2) to [resistor,l={$\SI{14.7}{\kilo\ohm}$}]++(\x+\x/4,0)-|(u1.out);
\draw(u1.out) to [short,*-o]++(\x/2,0)node[right]{$v_0$};
\end{tikzpicture}
\caption{سوال \حوالہ{سوال_حسابی_ایمپلیفائر_الف} کا دور۔}
\label{شکل_سوال_حسابی_ایمپلیفائر_الف}
\end{figure}

جوابات:\عددی{A_v=\SI{-6.68}{\volt\per\volt}}، \عددی{v_0=-1.54\sin 10t\,\si{\volt}}
\انتہا{سوال}
%===========================
\ابتدا{سوال}\شناخت{سوال_حسابی_ایمپلیفائر_ب}
شکل \حوالہ{شکل_سوال_حسابی_ایمپلیفائر_ب} کا افزائش دباو دریافت کریں اور \عددی{v_s=0.16\cos 314t\,\si{\volt}} کی صورت میں \عددی{v_0} حاصل کریں۔
\begin{figure}
\centering
\begin{tikzpicture}
\draw(0,0) node[op amp](u1){};
\draw(u1.+) to [short]++(-\x/8,0)++(0,-\y)  node[ground]{} to [american voltage source,l={$v_s$}]++(0,\y);
\draw(u1.-) to [short]++(-\x/8,0) to [resistor,l_={$\SI{1.8}{\kilo\ohm}$}]++(-\x,0) node[ground]{};
\draw(u1.-)++(-\x/8,0) to [short,*-] ++(0,\y/2) to [resistor,l={$\SI{22}{\kilo\ohm}$}]++(\x+\x/4,0)-|(u1.out);
\draw(u1.out) to [short,*-o]++(\x/2,0)node[right]{$v_0$};
\end{tikzpicture}
\caption{سوال \حوالہ{سوال_حسابی_ایمپلیفائر_ب} کا دور۔}
\label{شکل_سوال_حسابی_ایمپلیفائر_ب}
\end{figure}

جوابات:\عددی{A_v=\SI{13.22}{\volt\per\volt}}، \عددی{v_0=2.12\cos 314t\,\si{\volt}}
\انتہا{سوال}
%===========================
\ابتدا{سوال}\شناخت{سوال_حسابی_ایمپلیفائر_پ}
شکل \حوالہ{شکل_سوال_حسابی_ایمپلیفائر_پ} کا افزائش دباو دریافت کریں۔
\begin{figure}
\centering
\begin{tikzpicture}
\draw(0,0) node[op amp](u1){};
\draw(u1.+) to [short]++(-\x/8,0)++(0,-\y)  node[ground]{} to [resistor,l_={$\SI{10}{\kilo\ohm}$}]++(0,\y) to [resistor,*-,l={$\SI{40}{\kilo\ohm}$}]++(-\x,0)++(0,-\y) node[ground]{} to [american voltage source,l={$v_s$}]++(0,\y);
\draw(u1.-) to [short]++(-\x/8,0) to [resistor,l_={$\SI{2}{\kilo\ohm}$}]++(-\x,0) node[ground]{};
\draw(u1.-)++(-\x/8,0) to [short,*-] ++(0,\y/2) to [resistor,l={$\SI{48}{\kilo\ohm}$}]++(\x+\x/4,0)-|(u1.out);
\draw(u1.out) to [short,*-o]++(\x/2,0)node[right]{$v_0$};
\end{tikzpicture}
\caption{سوال \حوالہ{سوال_حسابی_ایمپلیفائر_پ} کا دور۔}
\label{شکل_سوال_حسابی_ایمپلیفائر_پ}
\end{figure}

جوابات:\عددی{A_v=\SI{5}{\volt\per\volt}}
\انتہا{سوال}
%===========================
\ابتدا{سوال}\شناخت{سوال_حسابی_ایمپلیفائر_ت}
شکل \حوالہ{شکل_سوال_حسابی_ایمپلیفائر_ت} کا افزائش دباو دریافت کریں۔کامل حسابی ایمپلیفائر استعمال کیا گیا ہے۔
\begin{figure}
\centering
\begin{tikzpicture}
\draw(0,0) node[op amp,yscale=-1](u1){};
\draw(u1.+) to [short]++(-\x/8,0) to [resistor,-o,l_={$\SI{2}{\kilo\ohm}$}]++(-\x,0)node[left]{$v_s$};
\draw(u1.out) to [resistor,-*,l={$R_2$}]++(0,-\y)coordinate(kM) to [resistor,l={$R_1$}]++(0,-\y) node[ground]{};
\draw(u1.-) |-(kM);
\draw(u1.out) to [short,*-o]++(\x/2,0)node[right]{$v_0$};
\end{tikzpicture}
\caption{سوال \حوالہ{سوال_حسابی_ایمپلیفائر_ت} کا دور۔}
\label{شکل_سوال_حسابی_ایمپلیفائر_ت}
\end{figure}

جوابات:داخلی جانب نسب \عددی{\SI{2}{\kilo\ohm}} کا افزائش پر کوئی اثر نہیں پایا جاتا لہٰذا \عددی{A_v=1+\tfrac{R_2}{R_1}} ہے۔
\انتہا{سوال}
%===========================
\ابتدا{سوال}\شناخت{سوال_حسابی_ایمپلیفائر_ٹ}
شکل \حوالہ{شکل_سوال_حسابی_ایمپلیفائر_ٹ} کا افزائش دباو دریافت کریں۔داخلی دباو \عددی{v_s=\SI{0.4}{\volt}} کی صورت میں \عددی{I_0} دریافت کریں۔
\begin{figure}
\centering
\begin{tikzpicture}
\draw(0,0) node[op amp](u1){};
\draw(u1.+) to [short]++(-\x/8,0) node[ground]{};
\draw(u1.-) to [short]++(-\x/2,0) to [resistor,l_={$\SI{12}{\kilo\ohm}$}]++(-\x,0) ++(0,-\y) node[ground]{} to [american voltage source,l={$v_s$}]++(0,\y);
\draw(u1.-)++(-\x/2,0) to [short,*-] ++(0,\y/2) to [resistor,l={$\SI{82}{\kilo\ohm}$}]++(\x+\x/4,0)-|(u1.out);
\draw(u1.-)++(-\x/2,0) to [resistor,l_={$\SI{10}{\kilo\ohm}$}]++(0,-\y)node[ground]{};
\draw(u1.out) to [short,*-o]++(\x/2,0)node[right]{$v_0$};
\draw(u1.out) to [resistor,l={$\SI{5}{\kilo\ohm}$},i={$I_0$}]++(0,-\y)node[ground]{};
\end{tikzpicture}
\caption{سوال \حوالہ{سوال_حسابی_ایمپلیفائر_ٹ} کا دور۔}
\label{شکل_سوال_حسابی_ایمپلیفائر_ٹ}
\end{figure}

جوابات:\عددی{A_v=-\tfrac{41}{6}\,\si{\volt\per\volt}}، \عددی{I_0=-\tfrac{41}{75}\,\si{\milli\ampere}}
\انتہا{سوال}
%===========================
\ابتدا{سوال}\شناخت{سوال_حسابی_ایمپلیفائر_ث}
شکل \حوالہ{شکل_سوال_حسابی_ایمپلیفائر_ث} کا افزائش دباو دریافت کریں۔
\begin{figure}
\centering
\begin{tikzpicture}
\draw(0,0) node[op amp](u1){};
\draw(u1.+) to [short]++(-\x/8,0) node[ground]{};
\draw(u1.-) to [short]++(-\x/2,0) to [resistor,l_={$\SI{12}{\kilo\ohm}$}]++(-\x,0) to [resistor,l_={$\SI{8}{\kilo\ohm}$}]++(-\x,0) ++(0,-\y) node[ground]{} to [american voltage source,l={$v_s$}]++(0,\y);
\draw(u1.-)++(-\x/2,0) to [short,*-] ++(0,\y/2)coordinate(kUL) to [resistor,l={$\SI{82}{\kilo\ohm}$}]++(\x+\x/4,0)-|(u1.out);
%
\path[name path={kvert}](u1.out)--++(0,\y);
\path[name path={khort}](kUL)++(\x+\x/4,0)--++(\x,0);
%
\draw[name intersections={of=kvert and khort}](kUL) to [short,*-]++(0,\y/2) to [resistor,l={$\SI{36}{\kilo\ohm}$}]++(\x+\x/4,0)-|(intersection-1)node[circ]{};
\draw(u1.-)++(-\x/2,0) to [resistor,l_={$\SI{10}{\kilo\ohm}$}]++(0,-\y)node[ground]{};
\draw(u1.out) to [short,*-o]++(\x/2,0)node[right]{$v_0$};
\draw(u1.out) to [resistor,l={$\SI{5}{\kilo\ohm}$}]++(0,-\y)node[ground]{};
\end{tikzpicture}
\caption{سوال \حوالہ{سوال_حسابی_ایمپلیفائر_ث} کا دور۔}
\label{شکل_سوال_حسابی_ایمپلیفائر_ث}
\end{figure}

جوابات:\عددی{A_v=-\tfrac{369}{295}\,\si{\volt\per\volt}}
\انتہا{سوال}
%===========================
\ابتدا{سوال}\شناخت{سوال_حسابی_ایمپلیفائر_ج}
شکل \حوالہ{شکل_سوال_حسابی_ایمپلیفائر_ج} میں کامل حسابی ایمپلیفائر استعمال کیا گیا ہے۔اس میں \عددی{I_1} اور \عددی{I_2} حاصل کریں۔
\begin{figure}
\centering
\begin{tikzpicture}
\draw(0,0) node[op amp,yscale=-1](u1){};
\draw(u1.+) to [short,-o,i^<={$I_1$}]++(-\x/2,0)node[left]{$\SI{-2}{\volt}$};
\draw(u1.out) to [short,*-o]++(\x/2,0)node[right]{$v_0$};
\draw(u1.-) to [short,-*]++(0,-\y/2)coordinate(kL) to [resistor,l_={$\SI{2}{\kilo\ohm}$}]++(0,-\y)node[ground]{};
\draw(kL) to [resistor,l_={$\SI{8}{\kilo\ohm}$},i<={$I_2$}]++(\x,0)-|(u1.out);
\end{tikzpicture}
\caption{سوال \حوالہ{سوال_حسابی_ایمپلیفائر_ج} کا دور۔}
\label{شکل_سوال_حسابی_ایمپلیفائر_ج}
\end{figure}

جوابات:\عددی{I_1=0}، \عددی{I_2=\SI{-1}{\milli\ampere}}
\انتہا{سوال}
%===========================
\ابتدا{سوال}\شناخت{سوال_حسابی_ایمپلیفائر_دیگر_الف}
شکل \حوالہ{شکل_سوال_حسابی_ایمپلیفائر_دیگر_الف} کا حسابی ایمپلیفائر \عددی{\SI{60}{\milli\ampere}} سے زیادہ رو مہیا نہیں کر سکتا۔ایمپلیفائر کو \عددی{\SI{\mp 18}{\volt}} سے طاقت فراہم کی گئی ہے۔اس کی افزائش \عددی{R_1} اور \عددی{R_2} سے اس طرح تعین کی جاتی ہے کہ \عددی{R_1+R_2=\SI{20}{\kilo\ohm}} رہے۔زیادہ سے زیادہ ممکنہ افزائش کیا ممکن ہے۔
\begin{figure}
\centering
\begin{tikzpicture}
\draw(0,0) node[op amp](u1){};
\draw(u1.+) to [short]++(-\x/8,0) node[ground]{};
\draw(u1.-) to [short]++(-\x/8,0) to [resistor,l_={$R_1$}]++(-\x,0) ++(0,-\y) node[ground]{} to [american voltage source,l={$\SI{0.24}{\volt}$}]++(0,\y);
\draw(u1.-)++(-\x/8,0) to [short,*-] ++(0,\y/2) to [resistor,l={$R_2$}]++(\x+\x/4,0)-|(u1.out);
\draw(u1.out) to [short,*-o]++(\x/2,0)node[right]{$v_0$};
\draw(u1.out) to [resistor,l={$\SI{200}{\ohm}$}]++(0,-\y) node[ground]{};
\end{tikzpicture}
\caption{سوال \حوالہ{سوال_حسابی_ایمپلیفائر_دیگر_الف} کا دور۔}
\label{شکل_سوال_حسابی_ایمپلیفائر_دیگر_الف}
\end{figure}

جوابات:\عددی{A_v=\SI{-50}{\volt\per\volt}}
\انتہا{سوال}
%========================
\ابتدا{سوال}\شناخت{سوال_حسابی_ایمپلیفائر_دیگر_ب}
شکل \حوالہ{شکل_سوال_حسابی_ایمپلیفائر_دیگر_ب} رو سے دباو حاصل کرتا ہے۔خارجی دباو \عددی{v_0} حاصل کریں۔بیرونی بوجھ مزاحمت کی رو \عددی{I_0} بھی حاصل کریں۔
\begin{figure}
\centering
\begin{tikzpicture}
\draw(0,0) node[op amp](u1){};
\draw(u1.+) to [short]++(-\x/8,0) node[ground]{};
\draw(u1.-) to [short]++(-\x/8,0) to [short]++(-\x,0) ++(0,-\y) node[ground]{} to [american current source,l={$\SI{2}{\milli\ampere}$}]++(0,\y);
\draw(u1.-)++(-\x/8,0) to [short,*-] ++(0,\y/2) to [resistor,l={$\SI{4}{\kilo\ohm}$}]++(\x+\x/4,0)-|(u1.out);
\draw(u1.out) to [short,*-o]++(\x/2,0)node[right]{$v_0$};
\draw(u1.out) to [resistor,l={$\SI{80}{\ohm}$},i={$I_0$}]++(0,-\y) node[ground]{};
\end{tikzpicture}
\caption{سوال \حوالہ{سوال_حسابی_ایمپلیفائر_دیگر_ب} کا دور۔}
\label{شکل_سوال_حسابی_ایمپلیفائر_دیگر_ب}
\end{figure}

جوابات:\عددی{v_0=\SI{-8}{\volt}}، \عددی{I_0=\SI{-100}{\milli\ampere}}
\انتہا{سوال}
%========================
\ابتدا{سوال}\شناخت{سوال_حسابی_ایمپلیفائر_دیگر_پ}
شکل \حوالہ{شکل_سوال_حسابی_ایمپلیفائر_دیگر_پ} میں موصل نما افزائش \عددی{\tfrac{I_0}{V_s}} دریافت کریں۔
\begin{figure}
\centering
\begin{tikzpicture}
\draw(0,0) node[op amp](u1){};
\draw(u1.+) to [short]++(-\x/8,0) ++(0,-\y) node[ground]{} to [american voltage source,l={$V_s$}]++(0,\y);
\draw(u1.-) to [short]++(-\x/8,0) to [resistor,l_={$\SI{2}{\kilo\ohm}$}]++(-\x,0) node[ground]{};
\draw(u1.-)++(-\x/8,0) to [short,*-] ++(0,\y/2) to [resistor,l={$\SI{6}{\kilo\ohm}$},i<={$I_0$}]++(\x+\x/4,0)-|(u1.out);
\end{tikzpicture}
\caption{سوال \حوالہ{سوال_حسابی_ایمپلیفائر_دیگر_پ} کا دور۔}
\label{شکل_سوال_حسابی_ایمپلیفائر_دیگر_پ}
\end{figure}

جوابات:\عددی{\tfrac{I_0}{V_s}=\tfrac{1}{2}\,\si{\milli\ampere\per\volt}}
\انتہا{سوال}
%========================
\ابتدا{سوال}\شناخت{سوال_حسابی_ایمپلیفائر_دیگر_ت}
شکل \حوالہ{شکل_سوال_حسابی_ایمپلیفائر_دیگر_ت} میں خارجی دباو \عددی{v_0} دریافت کریں۔
\begin{figure}
\centering
\begin{tikzpicture}
\draw(0,0) node[op amp](u1){};
\draw(u1.+) to [short]++(-\x/8,0) ++(0,-\y) node[ground]{} to [american voltage source,l={$\SI{2.2}{\volt}$}]++(0,\y);
\draw(u1.-) to [short]++(-\x/8,0) to [resistor,l_={$\SI{800}{\ohm}$}]++(-\x,0) ++(0,-\y) node[ground]{} to [american voltage source,l={$\SI{2}{\volt}$}]++(0,\y);
\draw(u1.-)++(-\x/8,0) to [short,*-] ++(0,\y/2) to [resistor,l={$\SI{12}{\kilo\ohm}$}]++(\x+\x/4,0)-|(u1.out);
\draw(u1.out) to [short,*-o]++(\x/4,0)node[right]{$v_0$};
\end{tikzpicture}
\caption{سوال \حوالہ{سوال_حسابی_ایمپلیفائر_دیگر_ت} کا دور۔}
\label{شکل_سوال_حسابی_ایمپلیفائر_دیگر_ت}
\end{figure}

جوابات:\عددی{v_0=\SI{5.2}{\volt}}
\انتہا{سوال}
%========================
\ابتدا{سوال}\شناخت{سوال_حسابی_ایمپلیفائر_دیگر_ٹ}
شکل \حوالہ{شکل_سوال_حسابی_ایمپلیفائر_دیگر_ٹ} میں خارجی دباو \عددی{v_0} اور داخلی دباو \عددی{v_1}، \عددی{v_2} کا تعلق دریافت کریں۔
\begin{figure}
\centering
\begin{tikzpicture}
\draw(0,0) node[op amp](u1){};
\draw(u1.+) to [short]++(-\x/8,0) ++(0,-\y) node[ground]{} to [american voltage source,l={$v_2$}]++(0,\y);
\draw(u1.-) to [short]++(-\x/8,0) to [resistor,l_={$R_1$}]++(-\x,0) ++(0,-\y) node[ground]{} to [american voltage source,l={$v_1$}]++(0,\y);
\draw(u1.-)++(-\x/8,0) to [short,*-] ++(0,\y/2) to [resistor,l={$R_2$}]++(\x+\x/4,0)-|(u1.out);
\draw(u1.out) to [short,*-o]++(\x/4,0)node[right]{$v_0$};
\end{tikzpicture}
\caption{سوال \حوالہ{سوال_حسابی_ایمپلیفائر_دیگر_ٹ} کا دور۔}
\label{شکل_سوال_حسابی_ایمپلیفائر_دیگر_ٹ}
\end{figure}

جوابات:\عددی{v_0=(1+\tfrac{R_2}{R_1})v_2-\tfrac{R_2}{R_1}v_1}
\انتہا{سوال}
%========================
\ابتدا{سوال}\شناخت{سوال_حسابی_ایمپلیفائر_دیگر_ث}
شکل \حوالہ{شکل_سوال_حسابی_ایمپلیفائر_دیگر_ث} میں خارجی دباو \عددی{v_0} اور داخلی دباو \عددی{v_1}، \عددی{v_2} کا تعلق دریافت کریں۔
\begin{figure}
\centering
\begin{tikzpicture}
\draw(0,0) node[op amp](u1){};
\draw(u1.+) to [short]++(-\x/8,0)coordinate(kUL) to [resistor,l={$R_4$}]++(0,-\y)node[ground]{};
\draw(kUL) to [resistor,*-o,l={$R_3$}]++(-\x,0)node[left]{$v_2$};
\draw(u1.-) to [short]++(-\x/8,0) to [resistor,-o,l_={$R_1$}]++(-\x,0)node[left]{$v_1$};
\draw(u1.-)++(-\x/8,0) to [short,*-] ++(0,\y/2) to [resistor,l={$R_2$}]++(\x+\x/4,0)-|(u1.out);
\draw(u1.out) to [short,*-o]++(\x/4,0)node[right]{$v_0$};
\end{tikzpicture}
\caption{سوال \حوالہ{سوال_حسابی_ایمپلیفائر_دیگر_ث} کا دور۔}
\label{شکل_سوال_حسابی_ایمپلیفائر_دیگر_ث}
\end{figure}

جوابات:\عددی{v_0=(\tfrac{R_1+R_2}{R_3+R_4})(\tfrac{R_4}{R_1})v_2-\tfrac{R2}{R1}v_1}
\انتہا{سوال}
%========================
\ابتدا{سوال}\شناخت{سوال_حسابی_مشکل_الف}
شکل \حوالہ{شکل_سوال_حسابی_مشکل_الف} میں \عددی{v_0} حاصل کریں۔ 
\begin{figure}
\centering
\begin{tikzpicture}
\draw(0,0)node[op amp](u1){};
\draw(u1.+) to [short]++(-\x/8,0)node[ground]{};
\draw(u1.-)to [short,-*]++(-\x/8,0)coordinate(kR) to [short,-*]++(-\x/4,0)coordinate(kL);
\draw(kR) to [short]++(0,\y/2) to [resistor,l={$R_0$}]++(\x+\x/4,0)-|(u1.out);
\draw(u1.out) to [short,*-o]++(\x/4,0)node[right]{$v_0$};
\draw(kL)  to [short] ++(0,\y/2) to [resistor,-o,l_={$R_1$}]++(-\x,0)node[left]{$v_1$};
\draw(kL) to [resistor,-o,l_={$R_2$}]++(-\x,0)node[left]{$v_2$};
\draw(kL) to [short] ++(0,-\y/2) to [resistor,-o,l_={$R_3$}]++(-\x,0)node[left]{$v_3$};
\end{tikzpicture}
\caption{سوال \حوالہ{سوال_حسابی_مشکل_الف} کا دور۔}
\label{شکل_سوال_حسابی_مشکل_الف}
\end{figure}

جواب:\عددی{v_0=-R_0(\tfrac{v_1}{R_1}+\tfrac{v_2}{R_2}+\tfrac{v_3}{R_3})}
\انتہا{سوال}
%====================================
\ابتدا{سوال}\شناخت{سوال_حسابی_مشکل_ب}
شکل \حوالہ{شکل_سوال_حسابی_مشکل_ب} میں \عددی{v_0} حاصل کریں۔ 
\begin{figure}
\centering
\begin{tikzpicture}
\draw(0,0)node[op amp,yscale=-1](u1){};
\draw(u1.+)to [short,-*]++(-\x/2,0)coordinate(kL);
\draw(u1.out) to [short,*-o]++(\x/4,0)node[right]{$v_0$};
\draw(kL)  to [short] ++(0,\y) to [resistor,-o,l_={$R_1$}]++(-\x,0)node[left]{$v_1$};
\draw(kL) to [short] ++(0,\y/2) to [resistor,*-o,l_={$R_2$}]++(-\x,0)node[left]{$v_2$};
\draw(kL)  to [resistor,-o,l_={$R_3$}]++(-\x,0)node[left]{$v_3$};
\draw(u1.-) to [short]++(-\x/8,0) to [short]++(0,-\y/2) to [resistor,l={$R_5$}]++(\x+\x/4,0)-|(u1.out);
\draw(u1.-)++(-\x/8,0) to [resistor,*-,l={$R_4$}]++(-\x,0)node[ground]{};
\end{tikzpicture}
\caption{سوال \حوالہ{سوال_حسابی_مشکل_ب} کا دور۔}
\label{شکل_سوال_حسابی_مشکل_ب}
\end{figure}

جواب:
$v_0=\frac{1+\frac{R_5}{R_4}}{\frac{1}{R_1}+\frac{1}{R_2}+\frac{1}{R_3}} \left[\frac{v_1}{R_1}+\frac{v_2}{R_2}+\frac{v_3}{R_3}\right]$
\انتہا{سوال}
%====================
\ابتدا{سوال}\شناخت{سوال_حسابی_مشکل_پ}
شکل \حوالہ{شکل_سوال_حسابی_مشکل_پ} میں \عددی{v_{01}} اور \عددی{v_1} کا تعلق دریافت کریں۔اب \عددی{v_{02}} کا \عددی{v_2} اور \عددی{v_{01}} کے ساتھ تعلق دریافت کریں۔ان نتائج کو استعمال کرتے ہوئے \عددی{v_{02}} کا \عددی{v_1} اور \عددی{v_2} کے ساتھ تعلق لکھیں۔
\begin{figure}
\centering
\begin{tikzpicture}
\draw(0,0)node[op amp](u1){};
\draw(2*\x+\x/2,-\y/3) node[op amp](u2){};
\draw(u1.+) to [short]++(-\x/8,0)node[ground]{};
\draw(u1.-) to [resistor,-o,l_={$R_1$}]++(-\x,0)node[left]{$v_1$};
\draw(u1.-) to [short,*-]++(0,\y/2) to [resistor,l={$R_2$}]++(\x,0)-|(u1.out);
\draw(u1.out)node[above right]{$v_{01}$};
\draw(u2.-) to [resistor,l_={$R_3$}]++(-\x,0)-|(u1.out);
\draw(u2.+) to [short,-o]++(-\x/4,0)node[left]{$v_2$};
\draw(u2.-) to [short,*-]++(0,\y/2) to [resistor,l={$R_4$}]++(\x,0)-|(u2.out);
\draw(u2.out) to [short,*-o]++(\x/4,0)node[right]{$v_{02}$};
\draw(u1.out)node[circ]{};
\end{tikzpicture}
\caption{سوال \حوالہ{سوال_حسابی_مشکل_پ} کا دور۔}
\label{شکل_سوال_حسابی_مشکل_پ}
\end{figure}

جوابات:
\begin{align*}
v_{01}&=-\frac{R_2}{R_1}v_1\\
v_{02}&=\left(1+\frac{R_4}{R_3}\right)v_2-\frac{R_4}{R_3}v_{01}\\
v_{02}&=\left(1+\frac{R_4}{R_3}\right)v_2+\frac{R_4 R_2}{R_3 R_1}v_1
\end{align*}
\انتہا{سوال}
%======================

\باب{ادوار حل کرنے کے دیگر ترکیب}
گزشتہ بابوں میں ہم نے ادوار میں مختلف مقامات پر دباو اور رو حاصل کرنے کے چند ترکیب دیکھے۔ایسا کرتے ہوئے ہم نے چند حقائق کا استعمال کیا جنہیں یہاں دوبارہ پیش کرتے ہیں۔

\حصہ{مساوی دور}
آپ جانتے ہیں کہ سلسلہ وار مزاحمتوں  کی جگہ ان کا مساوی مزاحمت نسب کرتے ہوئے ان کی رو حاصل کی جا سکتی ہے۔اسی طرح متوازی مزاحمتوں کی جگہ ان کا مساوی مزاحمت نسب کرتے ہوئے ان  پر دباو حاصل کیا جا سکتا ہے۔یہ عمل شکل \حوالہ{شکل_مسئلہ_مساوی_ادوار} میں دکھائے گئے ہیں۔اسی طرح سلسہ وار منبع دباو کا مساوی منبع اور متوازی منبع رو کی مساوی منبع شکل-ج اور شکل-د میں دکھائے گئے ہیں۔یاد رہے کہ دو یا دو سے زیادہ منبع رو کو صرف اور صرف اس صورت سلسلہ وار جوڑا جا سکتا ہے جب تمام کی رو برابر ہو اور تمام رو کی ایک ہی سمت ہو۔ اسی طرح دو یا دو سے زیادہ منبع دباو کو صرف اور صرف اس صورت متوازی جوڑا جا سکتا ہے جب تمام منبع کی دباو برابر اور سمت ایک ہو۔

\begin{figure}
\centering
\begin{subfigure}{0.5\textwidth}
\centering
\begin{tikzpicture}
\draw(0,0) to [resistor,l={$R_1$}]++(0,\y) to [resistor,l={$R_2$}]++(0,\y) to [short,-o]++(\x/4,0);
\draw(0,0) to [short,-o]++(\x/4,0);
\draw[thick,-stealth] (0.3,\y)--++(0.3,0);
\draw(\x,\y/2) to [short,o-]++(-\x/4,0) to [resistor,l_={$R_1+R_2$}]++(0,\y) to [short,-o]++(\x/4,0);
\end{tikzpicture}
\caption{سلسلہ وار مزاحمتوں کا مساوی مزاحمت}
\end{subfigure}%
\begin{subfigure}{0.5\textwidth}
\centering
\begin{tikzpicture}
\draw(0,0) to [resistor,l={$R_1$}]++(0,\y) ;
\draw(\x,0) to [resistor,*-*,l={$R_2$}]++(0,\y);
\draw(0,0) to [short,-o]++(\x+\x/4,0);
\draw(0,\y) to [short,-o]++(\x+\x/4,0);
\draw[thick,-stealth] (\x+0.3,\y/2)--++(0.3,0);
\draw(2*\x,0) to [short,o-]++(-\x/4,0) to [resistor,l_={$\frac{R_1 R_2}{R_1+R_2}$}]++(0,\y) to [short,-o]++(\x/4,0);
\end{tikzpicture}
\caption{متوازی مزاحمتوں کا مساوی مزاحمت۔}
\end{subfigure}
%
\begin{subfigure}{0.5\textwidth}
\centering
\begin{tikzpicture}
\draw(0,0) to [american voltage source,l={$V_1$}]++(0,\y) to [american voltage source,l={$V_2$}]++(0,\y) to [short,-o]++(\x/4,0);
\draw(0,0) to [short,-o]++(\x/4,0);
\draw[thick,-stealth] (0.3,\y)--++(0.3,0);
\draw(\x,\y/2) to [short,o-]++(-\x/4,0) to [american voltage source,l_={$V_1+V_2$}]++(0,\y) to [short,-o]++(\x/4,0);
\end{tikzpicture}
\caption{سلسلہ وار منبع دباو کا مساوی منبع۔}
\end{subfigure}%
\begin{subfigure}{0.5\textwidth}
\centering
\begin{tikzpicture}
\draw(0,0) to [american current source,l={$I_1$}]++(0,\y) ;
\draw(\x,0) to [american current source,*-*,l={$I_2$}]++(0,\y);
\draw(0,0) to [short,-o]++(\x+\x/4,0);
\draw(0,\y) to [short,-o]++(\x+\x/4,0);
\draw[thick,-stealth] (\x+\x/4,\y/2)--++(0.3,0);
\draw(2*\x,0) to [short,o-]++(-\x/4,0) to [american current source,l_={$I_1+I_2$}]++(0,\y) to [short,-o]++(\x/4,0);
\end{tikzpicture}
\caption{متوازی منبع رو کا مساوی منبع۔}
\end{subfigure}
\caption{مساوی ادوار کی مثال۔}
\label{شکل_مسئلہ_مساوی_ادوار}
\end{figure}

\حصہ{خطیّت}
برقی ادوار میں دباو اور رو درکار متغیرات ہیں۔ اس کتاب میں صرف ایسے ادوار پر غور کیا جائے گا جن میں دباو اور رو کا تعلق \اصطلاح{خطی}\فرہنگ{خطی}\حاشیہب{linear}\فرہنگ{linear} ہے۔انہیں خطی ادوار کہا جاتا ہے۔خطی ادوار میں ایک متغیرہ کو \عددی{n} گنا کرنے سے دوسری متغیرہ بھی \عددی{n} گنا ہو گی۔آئیں خطیت کی خاصیت سے دور حل کرنا دیکھیں۔

%==========================
\ابتدا{مثال}\شناخت{مثال_مسئلہ_خطیت_سے_دور_حل}
شکل \حوالہ{شکل_مسئلہ_خطیت_دور_حل} میں \عددی{\SI{60}{\ohm}} پر دباو معلوم کریں۔
\begin{figure}
\centering
\begin{tikzpicture}
\draw(0,0) to [american current source,l={$\SI{5}{\milli\ampere}$}]++(0,\yy)node[above]{$V_2$} to [resistor,i_={$I_3$},l={$\SI{100}{\ohm}$}]++(\xx,0)node[above]{$V_1$} to [resistor,i_={$I_1$},l={$\SI{40}{\ohm}$}]++(\xx,0)node[above]{$V_0$} to [resistor,i={$I_0$},l_={$\SI{60}{\ohm}$}]++(0,-\yy) to [short]++(-2*\xx,0);
\draw(\xx,0) to [resistor,i<_={$I_2$},*-*,l={$\SI{100}{\ohm}$}]++(0,\yy);
\draw(2*\xx+\dx,\yy/2)node[right]{$\begin{aligned}&+\\&V_R\\&-  \end{aligned}$};
\end{tikzpicture}
\caption{مثال \حوالہ{مثال_مسئلہ_خطیت_سے_دور_حل} کا دور۔}
\label{شکل_مسئلہ_خطیت_دور_حل}
\end{figure}

حل:ہم اس دور کو با آسانی قوانین کرخوف سے حل کر سکتے ہیں۔آئیں اس دور کو خطیت کی خاصیت  کی مدد سے حل کریں۔اس ترکیب میں ہم درکار دباو کو \عددی{\SI{1}{\volt}} تصور کرتے ہوئے منبع رو کی قیمت دریافت کریں گے۔اس کے بعد خطیت کو استعمال کرتے ہوئے منبع رو کی اصل قیمت کے مطابقت سے درکار دباو حاصل کی جائے گی۔

یوں \عددی{V_R=\SI{1}{\volt}} تصور کرتے ہوئے 
\begin{align*}
V_0&=\SI{1}{\volt}\\
I_0&=\frac{V_0}{60}=\frac{1}{60} \, \si{\ampere}\\
I_1&=I_0=\frac{1}{60} \,\si{\ampere}
\end{align*}
حاصل ہوتے ہیں۔قانون اوہم استعمال کرتے ہوئے
\begin{align*}
V_1-V_0=40 \times \frac{1}{60}=\frac{2}{3} \, \si{\volt}
\end{align*}
یعنی
\begin{align*}
V_1=1+\frac{2}{3}=\frac{5}{3}\,\si{\volt}
\end{align*}
حاصل ہوتا ہے۔قانون اوہم کا دوبارہ استعمال کرنے سے
\begin{align*}
I_2=\frac{\frac{5}{3}}{100}=\frac{1}{60} \, \si{\ampere}
\end{align*}
ملتا ہے لہٰذا
\begin{align*}
I_3=I_1+I_2=\frac{1}{60} +\frac{1}{60}=\frac{1}{30}\,\si{\ampere}
\end{align*}
ہو گا۔یوں \عددی{V_R=\SI{1}{\volt}} تصور کرتے ہوئے منبع کی رو \عددی{\tfrac{1}{30} \, \si{\ampere}} متوقع ہے۔

اب ہم کہہ سکتے ہیں کہ اگر منبع کی رو \عددی{\tfrac{1}{30} \, \si{\ampere}} ہو تب \عددی{V_R=\SI{1}{\volt}} ہو گا لہٰذا خطیت کے اصول کو استعمال کرتے ہوئے ہم کہہ سکتے ہیں کہ منبع کی رو \عددی{\SI{5}{\milli\ampere}} ہونے کی صورت میں \عددی{V_R} کی قیمت
\begin{align*}
\frac{0.005\times 1}{\frac{1}{30}}=\SI{0.15}{\volt}
\end{align*}
ہو گی۔
\انتہا{مثال}
%=========================
\ابتدا{مشق}\شناخت{مشق_مسئلہ_خطیت_الف}
شکل \حوالہ{شکل_مسئلہ_خطیت_الف} میں \عددی{I_0=\SI{10}{\milli\ampere}} تصور کرتے ہوئے \عددی{I_M} حاصل کریں۔اب \عددی{I_M=\SI{20}{\milli\ampere}} کی صورت میں خطیت کے استعمال سے \عددی{I_0} معلوم کریں۔
\begin{figure}
\centering
\begin{tikzpicture}
\draw(0,0) to [american current source,l={$I_M$}]++(0,\y) to [resistor,l_={$\SI{2}{\kilo\ohm}$}]++(-\x,0) to [resistor,l_={$\SI{6}{\kilo\ohm}$}]++(0,-\y) to [short]++(\x,0);
\draw(0,0) to [short,*-]++(2*\x,0) to [resistor,i<_={$I_0$},l_={$\SI{8}{\kilo\ohm}$}]++(0,\y) to [resistor,l_={$\SI{6}{\kilo\ohm}$}]++(-\x,0) to [resistor,-*,l_={$\SI{4}{\kilo\ohm}$}]++(-\x,0);
\draw(\x,0) to [resistor,*-*,l={$\SI{12}{\kilo\ohm}$}]++(\x,\y);
\draw(\x,0) to [resistor,-*,l={$\SI{10}{\kilo\ohm}$}]++(0,\y);
\end{tikzpicture}
\caption{مشق \حوالہ{مشق_مسئلہ_خطیت_الف} کا دور۔}
\label{شکل_مسئلہ_خطیت_الف}
\end{figure}
\انتہا{مشق}
%=====================
\ابتدا{مشق}\شناخت{مشق_مسئلہ_خطیت_ب}
شکل \حوالہ{شکل_مسئلہ_خطیت_ب} میں \عددی{V_R=\SI{2}{\volt}}  تصور کرتے ہوئے منبع دباو کی قیمت دریافت کریں۔خطیت کے استعمال سے منبع دباو کی اصل قیمت پر \عددی{V_R} دریافت کریں۔ 
\begin{figure}
\centering
\begin{tikzpicture}
\draw(0,0) to [american voltage source,l={$\SI{50}{\volt}$}]++(0,\y) to [resistor,l={$\SI{2}{\kilo\ohm}$}]++(\x,0) to [resistor,l={$\SI{4}{\kilo\ohm}$}]++(\x,0) to [resistor,l={$\SI{8}{\kilo\ohm}$}]++(\x,0) to [resistor,l_={$\SI{2}{\kilo\ohm}$}]++(0,-\y) to [resistor,l={$\SI{4}{\kilo\ohm}$}]++(-\x,0) to [resistor,l={$\SI{2}{\kilo\ohm}$}]++(-\x,0) to [short]++(-\x,0);
\draw(\x,0) to [resistor,*-*,l={$\SI{20}{\kilo\ohm}$}]++(0,\y);
\draw(2*\x,0) to [resistor,*-*,l={$\SI{10}{\kilo\ohm}$}]++(0,\y);
\draw(3*\x+\dx,\y/2)node[right]{$\begin{aligned} &+ \\ &V_R \\ &- \end{aligned}$};
\end{tikzpicture}
\caption{مشق \حوالہ{مشق_مسئلہ_خطیت_ب} کا دور۔}
\label{شکل_مسئلہ_خطیت_ب}
\end{figure}
\انتہا{مشق}
%========================

\حصہ{منبع کے انفرادی اثرات کا مجموعہ تمام منبع کا مجموعی اثر ہوتا ہے۔}
اس خاصیت کو سمجھنے کی خاطر شکل \حوالہ{شکل_مسئلہ_منبع_انفرادی_اثر}-الف پر غور کرتے ہیں۔
\begin{figure}
\centering
\begin{subfigure}{1\textwidth}
\centering
\begin{tikzpicture}
\draw(0,0) to [american voltage source,l={$\SI{4}{\volt}$}]++(0,\yy) to [resistor,l={$\SI{2}{\kilo\ohm}$}]++(\xx,0) to [resistor]++(0,-\yy) to [resistor,l={$\SI{6}{\kilo\ohm}$}]++(-\xx,0);
\draw(\xx+\dx,1/4*\yy)node[right]{$\SI{4}{\kilo\ohm}$};
\draw(\xx,0) to [short,*-]++(\xx,0) to [american voltage source,l_={$\SI{6}{\volt}$}]++(0,\yy) to [resistor,-*,l_={$\SI{8}{\kilo\ohm}$}]++(-\xx,0);
%loop currents
\draw[stealth-]([shift={(-150:\xx/5.5)}]\xx/2,\yy/2) arc (-150:150:\xx/5.5);
\draw(\xx/2,\yy/2)node{$i_1$};
\draw[stealth-]([shift={(-150:\xx/5.5)}]\xx+\xx/2,\yy/2) arc (-150:150:\xx/5.5);
\draw(\xx+\xx/2,\yy/2)node{$i_2$};
\end{tikzpicture}
\caption*{(الف) دو عدد انفرادی منبع کا مجموعی اثر۔}
\end{subfigure}
\begin{subfigure}{0.5\textwidth}
\centering
\begin{tikzpicture}
\draw(0,0) to [american voltage source,l={$\SI{4}{\volt}$}]++(0,\y) to [resistor,l={$\SI{2}{\kilo\ohm}$}]++(\x,0) to [resistor]++(0,-\y) to [resistor,l={$\SI{6}{\kilo\ohm}$}]++(-\x,0);
\draw(\x+\dx,1/4*\y-\dy)node[right]{$\SI{4}{\kilo\ohm}$};
\draw(\x,0) to [short,*-]++(\x,0) to [short]++(0,\y) to [resistor,-*,l_={$\SI{8}{\kilo\ohm}$}]++(-\x,0);
%loop currents
\draw[stealth-]([shift={(-150:\x/5.5)}]\x/2,\y/2) arc (-150:150:\x/5.5);
\draw(\x/2,\y/2)node{$i'_1$};
\draw[stealth-]([shift={(-150:\x/5.5)}]\x+\x/2,\y/2) arc (-150:150:\x/5.5);
\draw(\x+\x/2,\y/2)node{$i'_2$};
\end{tikzpicture}
\caption*{(ب) بائیں منبع کا اثر دیکھتے وقت دائیں منبع کے اثر کو ختم کیا گیا ہے۔}
\end{subfigure}%
\begin{subfigure}{0.5\textwidth}
\centering
\begin{tikzpicture}
\draw(0,0) to [short]++(0,\y) to [resistor,l={$\SI{2}{\kilo\ohm}$}]++(\x,0) to [resistor]++(0,-\y) to [resistor,l={$\SI{6}{\kilo\ohm}$}]++(-\x,0);
\draw(\x+\dx,1/4*\y-\dy)node[right]{$\SI{4}{\kilo\ohm}$};
\draw(\x,0) to [short,*-]++(\x,0) to [american voltage source,l_={$\SI{6}{\volt}$}]++(0,\y) to [resistor,-*,l_={$\SI{8}{\kilo\ohm}$}]++(-\x,0);
%loop currents
\draw[stealth-]([shift={(-150:\xx/5.5)}]\x/2,\y/2) arc (-150:150:\x/5.5);
\draw(\x/2,\y/2)node{$i''_1$};
\draw[stealth-]([shift={(-150:\x/5.5)}]\x+\x/2,\y/2) arc (-150:150:\x/5.5);
\draw(\x+\x/2,\y/2)node{$i''_2$};
\end{tikzpicture}
\caption*{(پ) دائیں منبع کا اثر دیکھتے وقت بائیں منبع کے اثر کو ختم کیا گیا ہے۔}
\end{subfigure}%
\caption{مجموعی اثر انفرادی اثرات کا مجموعہ ہے۔}
\label{شکل_مسئلہ_منبع_انفرادی_اثر}
\end{figure}
دونوں منبع کا مجموعی اثر دیکھنے کی خاطر دونوں منبع کی موجودگی میں اس دور کو حل کرتے ہیں۔دو خانوں کی مساوات لکھتے ہیں۔
\begin{align*}
-4+2000i_1+4000(i_1-i_2)+6000i_1&=0\\
4000(i_2-i_1)+8000i_2+6&=0
\end{align*}
ان کا حل درج ذیل ہے۔
\begin{align*}
i_1&=\frac{3}{16}\, \si{\milli\ampere}\\
i_2&=-\frac{7}{16}\, \si{\milli\ampere}
\end{align*}
آئیں انفرادی منبع سے پیدا رو دریافت کریں۔ایسا کرنے کی خاطر باری باری ایک منبع کے علاوہ بقایا تمام منبع کے اثر کو ختم کرتے ہوئے دور کو حل کیا جاتا ہے۔منبع دباو کا اثر ختم کرنے کی خاطر اس کو کسر دور کیا جاتا ہے جبکہ منبع رو کے اثر کو ختم کرنے کی خاطر اس کو کھلے دور کیا جاتا ہے۔یوں  \عددی{\SI{4}{\volt}} منبع کی رو حاصل کرتے وقت \عددی{\SI{6}{\volt}} کی منبع کو کسر دور کرتے ہیں۔ایسا کرنے سے شکل \حوالہ{شکل_مسئلہ_منبع_انفرادی_اثر}-ب حاصل ہوتا ہے جس کے مساوات
\begin{align*}
-4+2000i'_1+4000(i'_1-i'_2)+6000i'_1&=0\\
4000(i'_2-i'_1)+8000i'_2&=0
\end{align*}
اور حل درج ذیل ہے۔
\begin{align*}
i'_1&=\frac{3}{8}\, \si{\milli\ampere}\\
i'_2&=\frac{1}{8}\, \si{\milli\ampere}
\end{align*}
اسی طرح \عددی{\SI{6}{\volt}} منبع کا اثر دیکھنے کی خاطر \عددی{\SI{4}{\volt}} منبع کو کسر دور کیا جاتا ہے۔ایسا شکل \حوالہ{شکل_مسئلہ_منبع_انفرادی_اثر}-پ میں دکھایا گیا ہے جس کے مساوات
\begin{align*}
2000i''_1+4000(i''_1-i''_2)+6000i''_1&=0\\
4000(i''_2-i''_1)+8000i''_2+6&=0
\end{align*}
اور حل درج ذیل ہے۔
\begin{align*}
i''_1&=-\frac{3}{16}\, \si{\milli\ampere}\\
i''_2&=-\frac{9}{16}\, \si{\milli\ampere}
\end{align*}
آپ دیکھ سکتے ہیں کہ انفرادی منبع کے اثرات کا مجموعہ تمام منبع کے مجموعی اثر کے برابر ہے یعنی
\begin{align*}
i_1&=i'_1+i''_1\\
i_2&=i'_2+i''_2
\end{align*}

اس حقیقت کو درج ذیل طریقے سے بیان کیا جا سکتا ہے۔

\ابتدا{قانون}
کسی بھی خطی دور، جس میں متعدد غیر تابع منبع دباو اور غیر تابع منبع رو پائے جاتے ہوں، میں  کسی بھی مقام پر دباو یا رو، تمام منبع کے انفرادی اثرات کے مجموعے  کے برابر ہو گا۔
\انتہا{قانون}

اس حقیقت کا عمومی ثبوت پیش کرتے ہیں۔صفحہ \حوالہصفحہ{مساوات_جوڑ_عمومی_مساوات_متعدد_منبع} پر مساوات \حوالہ{مساوات_جوڑ_عمومی_مساوات_متعدد_منبع} متعدد منبع دباو استعمال کرنے والے دور کی عمومی مساوات ہے جسے یہاں دوبارہ پیش کرتے ہیں۔
\begin{align}\label{مساوات_جوڑ_عمومی_مساوات_متعدد_منبع_دوبارہ}
\begin{bmatrix}
R_{11} & -R_{12}& -R_{13}& \cdots -R_{1m}\\
-R_{21} & R_{22}& -R_{23}& \cdots -R_{2m}\\
-R_{31} & -R_{32}& R_{33}& \cdots -R_{3m}\\
\vdots\\
-R_{m1}&-R_{m2}&-R_{m3}&\cdots R_{mm}
\end{bmatrix}
\begin{bmatrix}
i_1\\
i_2\\
i_3\\
\vdots\\
i_m
\end{bmatrix}
=
\begin{bmatrix}
v_{1}\\
v_{2}\\
v_{3}\\
\vdots\\
v_{m}
\end{bmatrix}
\end{align}
اس مساوات میں مزاحمتی قالب کا دارومدار صرف اور صرف مزاحمتوں پر ہے۔دور میں موجود منبع دباو کا اس قالب پر کوئی اثر نہیں ہے۔اس قالبی مساوات \عددی{\bf{R}  \bf{I} = \bf{V}} کا حل \عددی{\bf{I} = \bf{R^{-1}}  \bf{V}} ہے۔ چونکہ مزاحمتی قالب \عددی{\bf{R}} کے اجزاء صرف اور صرف دور کے مزاحمتوں پر مبنی ہے لہٰذا اس کے ریاضی معکوس \عددیء{\bf{R^{-1}}} کے اجزاء بھی صرف مزاحمتوں پر مبنی ہوں گے۔ریاضی معکوس کے قالب کو درج ذیل عمومی شکل میں لکھا جا سکتا ہے۔
\begin{align*}
\bf{R^{-1}}=
\begin{bmatrix}
g_{11} & -g_{12}& -g_{13}& \cdots -g_{1m}\\
-g_{21} & g_{22}& -g_{23}& \cdots -g_{2m}\\
-g_{31} & -g_{32}& g_{33}& \cdots -g_{3m}\\
\vdots\\
-g_{m1}&-g_{m2}&-g_{m3}&\cdots g_{mm}
\end{bmatrix}
\end{align*}
یوں حل درج ذیل ہو گا
\begin{align*}
\begin{bmatrix}
i_1\\
i_2\\
i_3\\
\vdots\\
i_m
\end{bmatrix}
=
\begin{bmatrix}
g_{11} & -g_{12}& -g_{13}& \cdots -g_{1m}\\
-g_{21} & g_{22}& -g_{23}& \cdots -g_{2m}\\
-g_{31} & -g_{32}& g_{33}& \cdots -g_{3m}\\
\vdots\\
-g_{m1}&-g_{m2}&-g_{m3}&\cdots g_{mm}
\end{bmatrix}
\begin{bmatrix}
v_{1}\\
v_{2}\\
v_{3}\\
\vdots\\
v_{m}
\end{bmatrix}
\end{align*}
جس سے \عددی{i_1} درج ذیل لکھا جا سکتا ہے۔ 
\begin{align}\label{مساوات_مسئلہ_عمومی_رو_حل}
i_1=g_{11} v_1-g_{12}v_2-g_{13}v_3 -\cdots -\g_{1m}v_m
\end{align}
اگر \عددی{v_1} کے علاوہ تمام منبع دباو کو کسر دور کیا جائے تب ان کی قیمت \عددی{\SI{0}{\volt}} پُر کرتے ہوئے مساوات \حوالہ{مساوات_مسئلہ_عمومی_رو_حل} سے 
\begin{align*}
i'_1=g_{11} v_1
\end{align*}
حاصل ہوتا ہے۔ یہ صرف اور صرف \عددی{v_1} کا پیدا کردہ رو ہے۔اسی طرح \عددی{v_2} کے علاوہ تمام منبع کو کسر دور کرنے سے \عددی{i''_1=-g_{12}v_2}  پیدا ہوتا ہے۔اسی طرح بقایا منبع دباو کے انفرادی رو بھی حاصل کئے جا سکتے ہیں۔آپ دیکھ سکتے ہیں کہ تمام انفرادی منبع سے پیدا رو کا مجموعہ مساوات \حوالہ{مساوات_مسئلہ_عمومی_رو_حل} ہی دیتی ہے۔

مساوات \حوالہ{مساوات_جوڑ_عمومی_مساوات_متعدد_منبع_دوبارہ} ان ادوار کو ظاہر کرتی ہے جن میں صرف منبع دباو پائے جاتے ہوں۔آپ اسی ترکیب کو استعمال کرتے ہوئے منبع رو کے اثرات کو بھی شامل کر سکتے ہیں۔

یہ اصول ان ادوار کے لئے بھی درست ہے جن میں تابع منبع بھی پائے جاتے ہوں البتہ تابع منبع دباو کو کسر دور اور تابع منبع رو کو کھلے دور نہیں کیا جاتا۔ آئیں چند مثال دیکھیں۔

%================
\ابتدا{مثال}\شناخت{مثال_مسئلہ_متعدد_منبع_انفرادی_اثر_الف}
شکل \حوالہ{شکل_مسئلہ_مثال_منبع_انفرادی_اثر_الف} میں منبع دباو اور منبع رو کے انفرادی اثرات حاصل کرتے ہوئے کل \عددی{V_0} حاصل کریں۔
\begin{figure}
\centering
\begin{subfigure}{1\textwidth}
\centering
\begin{tikzpicture}
\draw(0,0) to [american voltage source,l={$\SI{10}{\volt}$}]++(0,\y) to [resistor,l={$\SI{1}{\kilo\ohm}$}]++(\x,0) to [american current source,l_={$\SI{5}{\milli\ampere}$}]++(0,-\y) to [short]++(-\x,0);
\draw(\x,0) to [short,*-]++(\x,0) to [resistor,l={$\SI{4}{\kilo\ohm}$}]++(0,\y) to [short,-*]++(-\x,0);
\draw(2*\x+\dx,\y/2)node[right]{$\begin{aligned} &+\\& V_0 \\ &- \end{aligned}$};
\end{tikzpicture}
\caption*{(الف)}
\end{subfigure}
\begin{subfigure}{0.5\textwidth}
\centering
\begin{tikzpicture}
\draw(0,0) to [american voltage source,l={$\SI{10}{\volt}$}]++(0,\y) to [resistor,l={$\SI{1}{\kilo\ohm}$}]++(\x,0) ++(0,-\y) to [short]++(-\x,0);
\draw(\x,0) to [short]++(\x/2,0) to [resistor,l={$\SI{4}{\kilo\ohm}$}]++(0,\y) to [short]++(-\x/2,0);
\draw(1.5*\x+\dx,\y/2)node[right]{$\begin{aligned} &+\\& V_0 \\ &- \end{aligned}$};
\end{tikzpicture}
\caption*{(ب)}
\end{subfigure}%
\begin{subfigure}{0.5\textwidth}
\centering
\begin{tikzpicture}
\draw(0,0) to [short]++(0,\y) to [resistor,l={$\SI{1}{\kilo\ohm}$}]++(\x,0) to [american current source,l_={$\SI{5}{\milli\ampere}$}]++(0,-\y) to [short]++(-\x,0);
\draw(\x,0) to [short,*-]++(\x,0) to [resistor,l={$\SI{4}{\kilo\ohm}$}]++(0,\y) to [short,-*]++(-\x,0);
\draw(2*\x+\dx,\y/2)node[right]{$\begin{aligned} &+\\& V_0 \\ &- \end{aligned}$};
\end{tikzpicture}
\caption*{(پ)}
\end{subfigure}
\caption{مثال \حوالہ{مثال_مسئلہ_متعدد_منبع_انفرادی_اثر_الف} کا دور۔}
\label{شکل_مسئلہ_مثال_منبع_انفرادی_اثر_الف}
\end{figure}
\انتہا{مثال}
%====================
\ابتدا{مثال}\شناخت{مثال_مسئلہ_منبع_دباو_منبع_رو_مجموعی_دباو}
شکل \حوالہ{شکل_مسئلہ_منبع_دباو_منبع_رو_مجموعی} میں منبع دباو اور منبع رو کو باری باری لیتے ہوئے \عددی{\SI{12}{\kilo\ohm}} پر دباو حاصل کرتے ہوئے دونوں منبع کی موجودگی میں کُل دباو حاصل کریں۔
\begin{figure}
\centering
\begin{tikzpicture}
\draw(0,0) to [american current source,l={$\SI{2}{\milli\ampere}$}]++(0,\y) to [american voltage source,l={$\SI{4}{\volt}$}]++(0,\y) to [short]++(\x,0) to [resistor,l={$\SI{10}{\kilo\ohm}$}]++(0,-\y) to [resistor,l={$\SI{8}{\kilo\ohm}$}]++(0,-\y) to [resistor,l={$\SI{4}{\kilo\ohm}$}]++(-\x,0);
\draw(0,\y) to [resistor,*-*,l={$\SI{1}{\kilo\ohm}$}]++(\x,0);
\draw(\x,0) to [short,*-] ++(\x,0) to [resistor,l={$\SI{12}{\kilo\ohm}$}]++(0,2*\y) to [short,-*]++(-\x,0);
\draw(2*\x+\dx,\y)node[right]{$\begin{aligned} &+ \\ &V_0 \\ &- \end{aligned}$};
\end{tikzpicture}
\caption{مثال \حوالہ{مثال_مسئلہ_منبع_دباو_منبع_رو_مجموعی_دباو} کا دور۔}
\label{شکل_مسئلہ_منبع_دباو_منبع_رو_مجموعی}
\end{figure}

\begin{figure}
\begin{subfigure}{0.5\textwidth}
\centering
\begin{tikzpicture}
\draw(0,0)++(0,\y) to [american voltage source]++(0,\y) to [short]++(\x,0) to [resistor,l={$\SI{10}{\kilo\ohm}$}]++(0,-\y) to [resistor,l={$\SI{8}{\kilo\ohm}$}]++(0,-\y) to [resistor,l={$\SI{4}{\kilo\ohm}$}]++(-\x,0);
\draw(-\dx,\y+3/4*\y)node[left]{$\SI{4}{\volt}$};
\draw(0,\y) to [resistor,-*,l={$\SI{1}{\kilo\ohm}$}]++(\x,0);
\draw(\x,0) to [short,*-] ++(\x,0) to [resistor,l={$\SI{12}{\kilo\ohm}$}]++(0,2*\y) to [short,-*]++(-\x,0);
\draw(2*\x+\dx,\y)node[right]{$\begin{aligned} &+ \\ &V'_0 \\ &- \end{aligned}$};
\end{tikzpicture}
\caption*{(الف)}
\end{subfigure}%
\begin{subfigure}{0.5\textwidth}
\centering
\begin{tikzpicture}
\draw(0,0) to [american voltage source,l={$\SI{4}{\volt}$}]++(0,2*\y) to [short]++(\x,0) to [resistor,l={$\SI{10}{\kilo\ohm}$}]++(0,-2*\y) to [resistor,l={$\SI{1}{\kilo\ohm}$}]++(-\x,0);
\draw(\x,0) to [short,*-]++(\x,0) to [resistor,l={$\SI{8}{\kilo\ohm}$}]++(0,\y) to [resistor,l={$\SI{12}{\kilo\ohm}$}]++(0,\y) to [short,-*]++(-\x,0);
\draw(2*\x+\dx,\y+\y/2)node[right]{$\begin{aligned} &+ \\ &V'_0 \\ &- \end{aligned}$};
\draw(\x-\dx,\y)node[left]{$\begin{aligned} &+ \\  \\ \\ &V'_1 \\ \\ \\ &- \end{aligned}$};
\end{tikzpicture}
\caption*{(ب)}
\end{subfigure}
\begin{subfigure}{0.5\textwidth}
\centering
\begin{tikzpicture}
\draw(0,0) to [american voltage source,l={$\SI{4}{\volt}$}]++(0,2*\y) to [short]++(\x,0) to [resistor,l={$\SI{10}{\kilo\ohm}$}]++(0,-2*\y) to [resistor,l={$\SI{1}{\kilo\ohm}$}]++(-\x,0);
\draw(\x,0) to [short,*-]++(\x,0) to [resistor,l_={$\SI{20}{\kilo\ohm}$}]++(0,2*\y) to [short,-*]++(-\x,0);
\draw(\x-\dx,\y)node[left]{$\begin{aligned} &+ \\  \\ \\ &V'_1 \\ \\ \\ &- \end{aligned}$};
\end{tikzpicture}
\caption*{(پ)}
\end{subfigure}%
\begin{subfigure}{0.5\textwidth}
\centering
\begin{tikzpicture}
\draw(0,0) to [american voltage source,l={$\SI{4}{\volt}$}]++(0,2*\y) to [short]++(\x,0) to [resistor,l={$\frac{20}{3}\,\si{\kilo\ohm}$}]++(0,-2*\y) to [resistor,l={$\SI{1}{\kilo\ohm}$}]++(-\x,0);
\draw(\x-\dx,\y)node[left]{$\begin{aligned} &+ \\  \\ \\ &V'_1 \\ \\ \\ &- \end{aligned}$};
\end{tikzpicture}
\caption*{(ت)}
\end{subfigure}%
\caption{منبع دباو کا حصہ معلوم کرتے ہیں۔ }
\label{شکل_مسئلہ_مثال_منبع_دباو_حصہ}
\end{figure}

حل:شکل \حوالہ{شکل_مسئلہ_مثال_منبع_دباو_حصہ}-الف میں منبع رو کو کھلے دور کیا گیا ہے تا کہ منبع دباو سے پیدا دباو کا حصہ دریافت کریں۔شکل \حوالہ{شکل_مسئلہ_مثال_منبع_دباو_حصہ}-ب میں شکل کو قدر مختلف صورت دی گئی ہے۔چونکہ \عددی{\SI{4}{\kilo\ohm}} کا ایک سرا کہیں نہیں جڑا لہٰذا اس کا بقایا دور پر کوئی اثر نہیں ہو گا اور اسی لئے اس کو شکل-ب میں نہیں دکھایا گیا ہے۔

شکل-ب میں \عددی{\SI{12}{\kilo\ohm}} اور \عددی{\SI{8}{\kilo\ohm}} سلسلہ وار جڑے ہیں لہٰذا ان کا مساوی مزاحمت \عددی{\SI{20}{\kilo\ohm}} ہو گا۔شکل-پ میں ایسا دکھایا گیا ہے۔شکل-پ میں \عددی{\SI{20}{\kilo\ohm}} اور \عددی{\SI{10}{\kilo\ohm}} متوازی جڑے ہیں لہٰذا ان کا مساوی مزاحمت
 \عددی{\tfrac{\SI{20}{\kilo\ohm} \times \SI{10}{\kilo\ohm}}{\SI{20}{\kilo\ohm} +\SI{10}{\kilo\ohm} }=\tfrac{20}{3}\,\si{\kilo\ohm}} ہو گا جسے شکل-ت میں دکھایا گیا ہے جہاں سے تقسیم دباو کے کلیے سے
\begin{align*}
V'_1=4\left(\frac{\frac{20}{3} \, \si{\kilo\ohm}}{\SI{1}{\kilo\ohm}+\frac{20}{3} \, \si{\kilo\ohm}}\right) =\frac{80}{23}\,\si{\volt}
\end{align*}
لکھا جا سکتا ہے۔شکل-ب کو دیکھتے ہوئے تقسیم دباو کے کلیے سے درج ذیل حاصل ہوتا ہے۔
\begin{align*}
V'_0=\frac{80}{23}\left(\frac{\SI{12}{\kilo\ohm}}{\SI{12}{\kilo\ohm}+\SI{8}{\kilo\ohm}}\right)=\frac{48}{23}\, \si{\volt}
\end{align*}
آئیں اب منبع دباو کو کسر دور کرتے ہوئے حل کریں ۔شکل \حوالہ{شکل_مسئلہ_منبع_دباو_کسر_دور_کیا_گیا_ہے}-الف  میں منبع دباو کو کسر دور کیا گیا ہے۔آپ دیکھ سکتے ہیں کہ \عددی{\SI{1}{\kilo\ohm}} اور \عددی{\SI{10}{\kilo\ohm}} متوازی جڑے ہیں لہٰذا ان کی جگہ \عددی{\tfrac{\SI{1}{\kilo\ohm} \times \SI{10}{\kilo\ohm}}{\SI{1}{\kilo\ohm}+\SI{10}{\kilo\ohm}}=\tfrac{10}{11}\,\si{\kilo\ohm}} نسب کیا جا سکتا ہے۔ایسا ہی شکل-ب میں کیا گیا ہے جہاں \عددی{\tfrac{10}{11}\,\si{\kilo\ohm}} اور \عددی{\SI{8}{\kilo\ohm}} سلسلہ وار جڑے ہیں لہٰذا ان کی جگہ شکل-پ میں \عددی{\tfrac{98}{11}\,\si{\kilo\ohm}} نسب کیا گیا ہے۔شکل-ت میں متوازی جڑے \عددی{\tfrac{98}{11}\,\si{\kilo\ohm}} اور \عددی{\SI{12}{\kilo\ohm}} کی جگہ \عددی{\tfrac{588}{115}\,\si{\kilo\ohm}} نسب کیا گیا ہے۔اس شکل سے درج ذیل لکھا جا سکتا ہے۔
\begin{align*}
V''_0=\frac{588}{115} \, \si{\kilo\ohm} \times \SI{2}{\milli\ampere}=\frac{1176}{115}\, \si{\volt}
\end{align*}
یوں دونوں منبع کی موجودگی میں جواب درج ذیل ہو گا۔
\begin{align*}
V_0=V'_0+V''_0=12\frac{36}{115}\,\si{\volt}
\end{align*}

\begin{figure}
\centering
\begin{subfigure}{0.5\textwidth}
\centering
\begin{tikzpicture}
\draw(0,0) to [american current source,l={$\SI{2}{\milli\ampere}$}]++(0,\y) to [short]++(0,\y) to [short]++(\x,0) to [resistor,l={$\SI{10}{\kilo\ohm}$}]++(0,-\y) to [resistor,l={$\SI{8}{\kilo\ohm}$}]++(0,-\y) to [resistor,l={$\SI{4}{\kilo\ohm}$}]++(-\x,0);
\draw(0,\y) to [resistor,*-*,l={$\SI{1}{\kilo\ohm}$}]++(\x,0);
\draw(\x,0) to [short,*-] ++(\x,0) to [resistor,l={$\SI{12}{\kilo\ohm}$}]++(0,2*\y) to [short,-*]++(-\x,0);
\draw(2*\x+\dx,\y)node[right]{$\begin{aligned} &+ \\ &V''_0 \\ &- \end{aligned}$};
\end{tikzpicture}
\caption*{(الف)}
\end{subfigure}%
\begin{subfigure}{0.5\textwidth}
\centering
\begin{tikzpicture}
\draw(0,0) to [american current source,l={$\SI{2}{\milli\ampere}$}]++(0,2*\y) to [short]++(\x,0) to [resistor,l={$\frac{10}{11}\,\si{\kilo\ohm}$}]++(0,-\y) to [resistor,l={$\SI{8}{\kilo\ohm}$}]++(0,-\y) to [resistor,l={$\SI{4}{\kilo\ohm}$}]++(-\x,0);
\draw(\x,0) to [short,*-] ++(\x,0) to [resistor,l={$\SI{12}{\kilo\ohm}$}]++(0,2*\y) to [short,-*]++(-\x,0);
\draw(2*\x+\dx,\y)node[right]{$\begin{aligned} &+ \\ &V''_0 \\ &- \end{aligned}$};
\end{tikzpicture}
\caption*{(ب)}
\end{subfigure}
\begin{subfigure}{0.5\textwidth}
\centering
\begin{tikzpicture}
\draw(0,0) to [american current source,l={$\SI{2}{\milli\ampere}$}]++(0,2*\y) to [short]++(\x,0) to [resistor,l_={$\frac{98}{11}\,\si{\kilo\ohm}$}]++(0,-2*\y)  to [resistor,l={$\SI{4}{\kilo\ohm}$}]++(-\x,0);
\draw(\x,0) to [short,*-] ++(\x,0) to [resistor,l={$\SI{12}{\kilo\ohm}$}]++(0,2*\y) to [short,-*]++(-\x,0);
\draw(2*\x+\dx,\y)node[right]{$\begin{aligned} &+ \\ &V''_0 \\ &- \end{aligned}$};
\end{tikzpicture}
\caption*{(پ)}
\end{subfigure}%
\begin{subfigure}{0.5\textwidth}
\centering
\begin{tikzpicture}
\draw(0,0) to [american current source,l={$\SI{2}{\milli\ampere}$}]++(0,2*\y) to [short]++(\x+\x/2,0) to [resistor,l_={$\frac{588}{115}\,\si{\kilo\ohm}$}]++(0,-2*\y)  to [resistor,l={$\SI{4}{\kilo\ohm}$}]++(-\x-\x/2,0);
\draw(\x+\x/2+\dx,\y)node[right]{$\begin{aligned} &+ \\ &V''_0 \\ &- \end{aligned}$};
\end{tikzpicture}
\caption*{(ت)}
\end{subfigure}%
\caption{منبع دباو کو کسر دور کیا گیا ہے۔}
\label{شکل_مسئلہ_منبع_دباو_کسر_دور_کیا_گیا_ہے}
\end{figure}



\انتہا{مثال}
%===================

\باب{برق گیر اور امالہ گیر}\شناخت{باب_برق_گیر_امالہ_گیر}

\حصہ{برق گیر}
متوازی چادر \اصطلاح{برق گیر}\فرہنگ{برق گیر}\حاشیہب{capacitor}\فرہنگ{capacitor} جسے شکل \حوالہ{شکل_امالہ_متوازی_چادر}-الف میں دکھایا گیا ہے کے بارے میں آپ نے چھوٹی جماعتوں میں پڑھا ہو گا۔خالی خلاء میں دو عدد یکساں، سیدھے متوازی موصل چادر جن کے مابین فاصلہ \عددی{d} ہو اور  ایک چادر کا رقبہ \عددی{S} ہو کی \اصطلاح{برقی گنجائش}\فرہنگ{برقی گنجائش}\فرہنگ{گنجائش!برقی}\حاشیہب{capacitance}\فرہنگ{capacitance} \عددی{C} درج ذیل مساوات دیتی ہے
\begin{align}\label{مساوات_امالہ_تعریف_مساوات}
C=\frac{\epsilon_0 S}{d}
\end{align}
جہاں \عددی{\epsilon_0} خالی خلاء کا \اصطلاح{برقی مستقل}\فرہنگ{برقی مستقل}\حاشیہب{permitivity, electric constant}\فرہنگ{permitivity}\فرہنگ{electric constant} ہے جس کی قیمت \عددی{\SI{8.85e-12}{\farad\per\meter}} ہے۔برقی گنجائش کو کولمب فی وولٹ \عددی{\si{\coulomb\per\volt}} یا فیراڈ \عددی{\si{\farad}} میں ناپا جاتا ہے۔فیراڈ\حاشیہد{فیراڈ کی اکائی انگلستان کے مشہور ماہر طبیعیات مائکل فیراڈے کے نام سے منسوب ہے۔} کی اکائی انتہائی بڑی مقدار ہے  لہٰذا برقی گنجائش کو عموماً مائیکرو فیراڈ \عددی{\si{\micro\farad}} اور نینو فیراڈ \عددی{\nano\farad} میں ناپا جاتا ہے۔
%=================
\ابتدا{مثال}
متوازی چادر برق گیر میں چادروں کے مابین فاصلہ \عددی{\SI{0.1}{\milli\meter}} ہے جبکہ اس کی برقی گنجائش \عددی{\SI{0.1}{\micro\farad}} ہے۔ایک چادر کا رقبہ دریافت کریں۔

حل:مساوات \حوالہ{مساوات_امالہ_تعریف_مساوات} استعمال کرتے ہوئے
\begin{align*}
S=\frac{C d}{\epsilon_0}=\frac{0.1\times 10^{-6} \times 0.1\times 10^{-3}}{8.854\times 10^{-12}}=\SI{1.129}{\meter\squared}
\end{align*}
حاصل ہوتا ہے۔
\انتہا{مثال}
%===================

\begin{figure}
\centering
\begin{subfigure}{0.5\textwidth}
\centering
\begin{tikzpicture}
\draw(\x/2+\x/8,0)--++(0,-\y/2) to [short,-o]++(-\x,0);
\draw[fill=white,opacity=1](0,-\dy)--++(\x,0)--++(\x/4,\y/2)--++(-\x,0)--++(-\x/4,-\y/2);
\draw[fill=white,opacity=1](0,0)--++(\x,0)--++(\x/4,\y/2)--++(-\x,0)--++(-\x/4,-\y/2);
\draw(\x/2+\x/8,\y/4)--++(0,\y/2) to [short,-o]++(-\x,0);
\draw(\x/4,\y/8)node{$S$};
\draw(\x+\x/4,\y/2)++(\dx,0)--++(2*\dx,0)coordinate(kup);
\draw(\x+\x/4,\y/2-\dy)++(\dx,0)--++(2*\dx,0)coordinate(klow);
\draw[stealth-] (kup)++(-\dx,0)--++(0,\dy)node[above]{$d$};
\draw[stealth-](klow)++(-\dx,0)--++(0,-\dy);
\end{tikzpicture}
\caption*{(الف)}
\end{subfigure}%
\begin{subfigure}{0.5\textwidth}
\centering
\begin{tikzpicture}
\draw(\dx,\dy) to [short,-o]++(\x/2,0) to [short]++(\x,0) to [capacitor,l={$C$}]++(0,\y) to [short,i<_={${i(t)=\frac{\dif q(t)}{\dif t}}$},-o]++(-\x,0) to [short]++(-\x/2,0);  
\draw[fill=white,opacity=1] plot [smooth cycle] coordinates {(0,0) (\x/4,0) (\x/4,\y+2*\dy) (0,\y+2*\dy)};
\draw(\x/2+\dx,\y/2+\dy) node{$\begin{aligned} &+ \\ &v(t) \\ &- \end{aligned}$};
\draw(\x/2+\x+\x/4+\dx,\y/2+\dy) node[right]{$\begin{aligned} &+ \\ & q(t) \\ &- \end{aligned}$};
\end{tikzpicture}
\caption*{(ب)}
\end{subfigure}%
\caption{متوازی چادر برق گیر۔}
\label{شکل_امالہ_متوازی_چادر}
\end{figure}

شکل \حوالہ{شکل_امالہ_متوازی_چادر}-ب میں برقی گیر کو \عددی{v(t)} منبع دباو کے ساتھ جوڑا گیا ہے  جس کی وجہ سے برق گیر کے ایک چادر پر مثبت برقی بار \عددی{+q(t)} اور دوسرے چادر پر منفی برقی بار \عددی{-q(t)} جمع ہوتا ہے جبکہ دونوں چادروں کے مابین دباو \عددی{v(t)} پایا جاتا ہے۔برق گیر کے چادروں پر بار اور ان کے مابین دباو خطی تعلق
\begin{align}\label{مساوات_امالہ_بار_دباو_تعلق}
q(t)=C v(t)
\end{align}
رکھتے ہیں جہاں خطی تعلق کے مستقل کو \عددی{C} سے ظاہر اور  \اصطلاح{برقی گنجائش}\فرہنگ{برقی گنجائش}\فرہنگ{گنجائش!برقی}\حاشیہب{capacitance}\فرہنگ{capacitance} کہتے ہیں۔ برقی گنجائش کے نام کو چھوٹا کرتے ہوئے عموماً \اصطلاح{گنجائش} کہا جاتا ہے۔وقت کے ساتھ بدلتی بار کو برقی رو کہا جاتا ہے۔یوں برق گیر کے چادروں پر بار کی تبدیلی رو کو جنم دیتی ہے جسے
\begin{align}
i=\frac{\dif q}{\dif t}
\end{align}
لکھا جا سکتا ہے جسے شکل \حوالہ{شکل_امالہ_متوازی_چادر}-ب میں دکھایا گیا ہے۔برق گیر کے مثبت برقی سر پر مثبت رو داخل ہوتی ہے۔یوں مزاحمت کی طرح  برق گیر پر بھی دباو اور رو انفعالی رائج سمت کے تحت ہیں۔ مساوات \حوالہ{مساوات_امالہ_بار_دباو_تعلق} کو استعمال کرتے ہوئے 
\begin{align}
i=\frac{\dif (C v)}{\dif t}
\end{align}
لکھا جا سکتا ہے۔مستقل برقی گنجائش کی صورت میں اسے
\begin{align}\label{مساوات_امالہ_بار_دباو_تعلق_ب}
i=C\frac{\dif v}{\dif t}
\end{align}
لکھا جا سکتا ہے۔مساوات \حوالہ{مساوات_امالہ_بار_دباو_تعلق_ب} کو
\begin{align*}
\dif v = \frac{1}{C} i \dif t
\end{align*}
لکھ کر تکمل لینے سے
\begin{align}\label{مساوات_امالہ_بار_دباو_تعلق_پ}
v(t)=\frac{1}{C} \int_{-\infty}^{t} i \dif t
\end{align}
حاصل ہوتا ہے جہاں \عددی{t = -\infty} پر برق گیر کا دباو \عددی{v(-\infty) = 0} لیا گیا ہے۔مندرجہ بالا مساوات میں \عددی{v(t)} لکھ کر وقت کو \اصطلاح{آزاد متغیر}\فرہنگ{آزاد متغیر}\حاشیہب{independent variable}\فرہنگ{independent variable} اور دباو کو \اصطلاح{تابع متغیر}\فرہنگ{تابع متغیر}\حاشیہب{dependent variable}\فرہنگ{dependent variable} کے طور پر لکھا گیا ہے۔اس مساوات کو دو ٹکڑوں میں درج ذیل لکھا جا سکتا ہے
\begin{gather}
\begin{aligned}\label{مساوات_امالہ_ابتدائی_دباو_الف}
v(t)&=\frac{1}{C} \int_{-\infty}^{t_0} i \dif t+\frac{1}{C} \int_{t_0}^{t} i \dif t\\
&=v(t_0)+\frac{1}{C} \int_{t_0}^{t} i \dif t
\end{aligned}
\end{gather}
جہاں وقت \عددی{t=-\infty} تا \عددی{t=t_0} کے دوران برق گیر پر جمع ہونے والے بار کی وجہ سے  برق گیر پر وقت \عددی{t=t_0} پر دباو \عددی{v(t_0)} پایا جاتا ہے۔ 

برق گیر میں ذخیرہ توانائی \عددی{w_C(t)} کو طاقت کے تکمل سے حاصل کیا جا سکتا ہے۔برق گیر کو منتقل طاقت \عددی{p(t)} کو
\begin{align}\label{مساوات_امالہ_طاقت_برق_گیر}
p(t)=v(t)i(t)=v(t) C \frac{\dif v(t)}{\dif t}
\end{align}
لکھا جا سکتا ہے۔چونکہ \عددی{p=\tfrac{\dif w}{\dif t}} کے برابر ہے  لہٰذا برق گیر میں ذخیرہ توانائی کو
\begin{align*}
w_C(t)&=\int_{-\infty}^{t} C v(t) \frac{\dif v(t)}{\dif t} \dif t\\
&=C\int_{v(-\infty)}^{v(t)} v(t) \dif v(t)\\
&=\left. C \frac{v^2(t)}{2} \right|_{v(-\infty)}^{v(t)}
\end{align*}
یعنی
\begin{align}\label{مسوات_امالہ_توانائی_برق_گیر_الف}
w_C(t)=\frac{C v^2(t)}{2}
\end{align}
لکھا جا سکتا ہے جہاں \عددی{v(-\infty) =0} لیا گیا ہے۔مساوات \حوالہ{مساوات_امالہ_بار_دباو_تعلق} کی مدد سے اس مساوات کو درج ذیل لکھا جا سکتا ہے۔
\begin{align}\label{مسوات_امالہ_توانائی_برق_گیر_ب}
w_C(t)=\frac{q^2(t)}{2C}
\end{align}
مساوات \حوالہ{مسوات_امالہ_توانائی_برق_گیر_الف} اور مساوات \حوالہ{مسوات_امالہ_توانائی_برق_گیر_ب} برقی گیر میں ذخیرہ \اصطلاح{مخفی توانائی}\فرہنگ{مخفی توانائی}\فرہنگ{توانائی!مخفی}\حاشیہب{potential energy}\فرہنگ{potential energy} دیتے ہیں۔یہ وہی توانائی ہے جو برق گیر میں بار بھرتے ہوئے خرچ کی جاتی ہے۔

مساوات \حوالہ{مساوات_امالہ_بار_دباو_تعلق_ب} کے تحت برقی گیر پر دباو کے تبدیلی کی شرح اور رو کا راست تناسب تعلق ہے۔چونکہ یک سمتی دباو تبدیل نہیں ہوتی لہٰذا برق گیر پر یک سمتی دباو کی صورت میں اس میں کوئی رو نہیں گزرے گی۔یوں یک سمتی دباو کی نقطہ نظر سے برق گیر کھلا دور ہے لہٰذا ادوار کے یک سمتی حل کے دوران تمام برق گیروں کو کھلے دور تصور کیا جاتا ہے۔

مساوات \حوالہ{مساوات_امالہ_طاقت_برق_گیر} کے تحت برق گیر کو منتقل طاقت، دباو کی شرح تبدیلی  کے راست تناسب ہے۔یوں برق گیر کا دباو فوراً \عددی{(\dif t \to 0)} تبدیل کرنے کے لئے لامحدود طاقت درکار ہو گی۔کائنات میں لامحدود طاقت کا منبع نہیں پایا جاتا لہٰذا برق گیر کا دباو فوراً کسی صورت تبدیل نہیں کیا جا سکتا۔اسی حقیقت کی وضاحت مساوات \حوالہ{مساوات_امالہ_بار_دباو_تعلق_ب} کے استعمال سے  مثال \حوالہ{مثال_امالہ_برق_گیر_درکار_رو} میں کی گئی ہے۔ مساوات \حوالہ{مساوات_امالہ_بار_دباو_تعلق_ب} کے تحت برق گیر کا دباو فوراً تبدیل کرنے کے لئے لا محدود رو درکار ہو گی۔چونکہ لا محدود رو کائنات میں کہیں نہیں پائی جاتی لہٰذا ایسا ممکن نہیں ہے۔ یہ ایک اہم نتیجہ ہے جس کے تحت دور میں سوئچ کو چالو سے غیر چالو (یا غیر چالو سے چالو) کرنے کے فوراً بعد دور میں موجود  برق گیر کے دباو کی قیمت وہی ہو گی جو سوئچ چالو (یا غیر چالو) کرنے سے پہلے تھی۔اس حقیقت کی مساواتی شکل درج ذیل ہے۔
\begin{align}\label{مساوات_امالہ_برق_گیر_دباو_بلا_جوڑ_ہے}
v_C(t_+)=v_C(t_-)
\end{align}
مساوات \حوالہ{مساوات_امالہ_برق_گیر_دباو_بلا_جوڑ_ہے} کے تحت برق گیر کا دباو کسی بھی لمحے \عددی{t} کے فوراً بعد \عددی{t_+} اور اس لمحے کے فوراً  پہلے \عددی{t_-} برابر ہوں گے۔ یوں برق گیر کا دباو \اصطلاح{بلا جوڑ تفاعل}\فرہنگ{بلا جوڑ!تفاعل}\فرہنگ{تفاعل!بلا جوڑ}\حاشیہب{continuous function}\فرہنگ{function!continuous}\فرہنگ{continuous!function} ہے جس میں \اصطلاح{سیڑھی نما}\فرہنگ{سیڑھی نما}\حاشیہب{step}\فرہنگ{step} یکدم تبدیلی ممکن نہیں ہے۔

مساوات \حوالہ{مساوات_امالہ_بار_دباو_تعلق} برق گیر کی عمومی مساوات ہے۔کسی بھی دو موصل جن کے درمیان دباو \عددی{v} اور جن میں مثبت موصل پر \عددی{+q} اور منفی موصل پر \عددی{-q} بار پایا جاتا ہو کی گنجائش مساوات \حوالہ{مساوات_امالہ_بار_دباو_تعلق} دیتی ہے۔یوں دور کے مختلف موصل حصوں مثلاً مزاحمت، باقی تار، برق گیر وغیرہ کے مابین \اصطلاح{غیر مطلوب}\فرہنگ{غیر مطلوب}\حاشیہب{stray}\فرہنگ{stray} برقی گنجائش پائی جائے گی۔بعض ادوار میں غیر مطلوب برقی گنجائش کو کم سے کم رکھنا ضروری ہوتا ہے جبکہ یک سمتی ادوار میں ان کے کردار کو رد کیا جاتا ہے
%==============
\ابتدا{مثال}\شناخت{مثال_امالہ_برق_گیر_درکار_رو}
برق گیر کی دباو \عددی{\SI{20}{\volt}} سے \عددی{\SI{20.1}{\volt}} کرنے کی خاطر منبع رو استعمال کیا جاتا ہے۔برق گیر کی گنجائش \عددی{\SI{1}{\micro\farad}} ہے۔تبدیلی کا دورانیہ ایک سیکنڈ، ایک نینو سیکنڈ، ایک فیمٹو سیکنڈ اور صفر سیکنڈ تصور کرتے ہوئے درکار رو کی قیمت حاصل کریں۔دباو کے تبدیلی کے دوران رو کی قیمت مستقل تصور کریں۔

حل:دورانیہ ایک سیکنڈ تصور کرتے ہوئے مساوات   \حوالہ{مساوات_امالہ_بار_دباو_تعلق_ب} کے تحت
\begin{align*}
i=10^{-6} \times \left(\frac{20.1-20}{1}\right)=\SI{0.1}{\micro\ampere}
\end{align*}
درکار ہو گی۔اسی طرح بالترتیب بقایا دورانیوں کے لئے درج ذیل رو حاصل ہوتی ہیں۔
\begin{align*}
i&=10^{-6} \times \left(\frac{20.1-20}{10^{-9}}\right)=\SI{100}{\ampere}\\
i&=10^{-6} \times \left(\frac{20.1-20}{10^{-15}}\right)=\SI{e8}{\ampere}\\
i&=10^{-6} \times \left(\frac{20.1-20}{0}\right)=\infty\, \si{\ampere}\quad \quad \text{\RL{دباو میں فوراً تبدیلی کے لئے لامحدود رو درکار ہے}}
\end{align*}

\انتہا{مثال}
%=======================

\ابتدا{مثال}
دو  قریبی موصل تاروں پر \عددی{\SI{300}{\nano\coulomb}} بار ذخیرہ کرنے سے ان کے مابین \عددی{\SI{15}{\volt}} دباو پیدا ہوتا ہے۔ان جوڑی موصل کی برقی گنجائش دریافت کریں۔

حل:مساوات \حوالہ{مساوات_امالہ_بار_دباو_تعلق} کے تحت
\begin{align*}
C=\frac{q}{v}=\frac{300 \times 10^{-9}}{15}=\SI{20}{\nano\farad}
\end{align*}
ہو گا۔ 
\انتہا{مثال}

\FloatBarrier
%=======================
\ابتدا{مثال}\شناخت{مثال_امالہ_برق_گیر_دباو_رو_تعلق_الف}
شکل \حوالہ{شکل_مثال_امالہ_برق_گیر_دباو_رو_تعلق_الف} میں \عددی{v_1=\SI{17}{\volt}} اور \عددی{v_2=\SI{3}{\volt}} کی صورت میں برق گیر پر دباو اور بار دریافت کریں۔
\begin{figure}
\centering
\begin{tikzpicture}
\draw(0,0) to [american voltage source,l={$v_1$}]++(0,\y) to [capacitor,l={$\SI{2}{\micro\farad}$}]++(\x,0);
\draw(0,0) to [short]++(\x,0) to [american voltage source,l_={$v_2$}]++(0,\y);
\end{tikzpicture}
\caption{مثال \حوالہ{مثال_امالہ_برق_گیر_دباو_رو_تعلق_الف} اور مثال \حوالہ{مثال_امالہ_برق_گیر_دباو_رو_تعلق_ب} کا دور۔}
\label{شکل_مثال_امالہ_برق_گیر_دباو_رو_تعلق_الف}
\end{figure}

حل:برق گیر پر دباو سے مراد اس کے دو برقی سروں کے مابین دباو ہے۔برق گیر کے دائیں سر کو برقی زمین تصور کرتے ہوئے  برق گیر کا دباو درج ذیل لکھا جا سکتا ہے۔
\begin{align*}
v_C=\SI{17}{\volt}-\SI{3}{\volt}=\SI{14}{\volt}
\end{align*}
یوں مساوات \حوالہ{مساوات_امالہ_بار_دباو_تعلق} کے تحت
\begin{align*}
q=\left(\SI{2}{\micro\farad}\right)\left(\SI{14}{\volt}\right)=\SI{28}{\micro\coulomb}
\end{align*}
ہو گا۔اس طرح برق گیر کے بائیں طرف پر \عددی{\SI{+28}{\micro\coulomb}} جبکہ اس کے دائیں طرف پر \عددی{\SI{-28}{\coulomb}} بار ہو گا۔
\انتہا{مثال}
%======================
\ابتدا{مثال}\شناخت{مثال_امالہ_برق_گیر_دباو_رو_تعلق_ب}
شکل \حوالہ{شکل_مثال_امالہ_برق_گیر_دباو_رو_تعلق_الف} میں \عددی{v_1=\SI{20}{\volt}} اور \عددی{v_2=0.1\sin 100 t \, \si{\volt}} ہے۔برقی رو دریافت کریں۔

حل:برق گیر کے بائیں سر کو زمین تصور کرتے ہیں۔یوں برق گیر پر دباو \عددی{v_C} درج ذیل ہو گا
\begin{align*}
v_C=0.1\sin 100t -20
\end{align*}
جبکہ اس میں رو کی مثبت سمت دائیں سے بائیں جانب ہو گی۔رو کی قیمت درج ذیل ہو گی۔
\begin{align*}
i_C&=C\frac{\dif v_C}{\dif t}\\
&=\left(\SI{2}{\micro\farad}\right)\left(0.1\times 100 \cos 100 t\right)\\
&=20 \cos 100 t \, \si{\micro\ampere}
\end{align*}
آپ دیکھ سکتے ہیں کہ رو کی قیمت، وقت کے ساتھ بدلتے دباو پر منحصر ہے۔بیس وولٹ کا یک سمتی دباو برق گیر میں رو نہیں پیدا کرتا۔  
\انتہا{مثال} 
%=============================
\ابتدا{مثال}\شناخت{مثال_برق_گیر_رو_درکار_ہے}
شکل \حوالہ{شکل_مثال_برق_گیر_رو_درکار_ہے} میں \عددی{\SI{2}{\micro\farad}} برق گیر پر دباو دکھایا گیا ہے۔برق گیر کی رو دریافت کریں۔

\begin{figure}
\centering
\begin{subfigure}{1\textwidth}
\centering
\begin{tikzpicture}
\draw[gray](0,-2)--(0,3)node[left]{$v(t)$};
\draw[gray](0,0)--++(6.5,0)node[right]{$t(\si{\milli\second})$};
\draw(-0.5,0)--(0,0)--++(1,2)--++(1.5,0)--++(2,-3)--++(2,0);
\draw(1,0)--++(0,-0.1)node[below]{$10$};
\draw(2.5,0)--++(0,-0.1)node[below]{$25$};
\draw(4.5,0)--++(0,-0.1)node[below]{$45$};
\draw(0,2)--++(-0.1,0)node[left]{$\SI{50}{\volt}$};
\draw(0,-1)--++(-0.1,0)node[left]{$\SI{-25}{\volt}$};
\end{tikzpicture}
\caption*{(الف)}
\end{subfigure}
\begin{subfigure}{1\textwidth}
\centering
\begin{tikzpicture}
\draw[gray](0,-1)--(0,1.5)node[left]{$i(t)$};
\draw[gray](0,0)--++(6.5,0)node[right]{$t(\si{\milli\second})$};
\draw(-0.5,0)--(0,0)--++(0,1)--++(1,0)--++(0,-1)--++(1.5,0)--++(0,-0.7)--++(2,0)--++(0,0.7)--++(2,0);
\draw(1,0)node[below]{$10$};
\draw(2.5,0)node[above]{$25$};
\draw(4.5,0)node[above]{$45$};
\draw(0,1)--++(-0.1,0)node[left]{$\SI{10}{\milli\ampere}$};
\draw(0,-0.7)--++(-0.1,0)node[left]{$\SI{-7}{\milli\ampere}$};
\end{tikzpicture}
\caption*{(ب)}
\end{subfigure}
\caption{مثال \حوالہ{مثال_برق_گیر_رو_درکار_ہے} کے خط۔}
\label{شکل_مثال_برق_گیر_رو_درکار_ہے}
\end{figure}

حل:دورانیہ \عددی{\SI{0}{\second}} تا \عددی{\SI{10}{\milli\second}} میں دباو مسلسل مستقل شرح
\begin{align*}
\frac{\Delta v}{\Delta t}=\frac{\SI{50}{\volt}-\SI{0}{\volt}}{\SI{10}{\milli\second}-\SI{0}{\second}}=\SI{5000}{\volt\per\second}
\end{align*}
 سے بڑھتا ہے لہٰذا اس دوران دباو بالمقابل وقت کی مساوات 
\begin{align*}
v(t)=5000 t \quad \quad (0 \le t \le \SI{10}{\milli\second})
\end{align*}
لکھی جا سکتی ہے۔وقت \عددی{\SI{10}{\milli\second}} تا \عددی{\SI{25}{\milli\second}} دباو بغیر تبیل ہوئے مستقل \عددی{\SI{50}{\volt}} پر برقرار رہتا ہے لہٰذا اس دوران دباو کی مساوات درج ذیل ہے۔
\begin{align*}
v(t)=50 \quad \quad (\SI{10}{\milli\second} \le t \le \SI{25}{\milli\second})
\end{align*}
اس کے بعد \عددی{\SI{25}{\milli\second}} تا \عددی{\SI{45}{\milli\second}} کے دوران دباو مستقل شرح
\begin{align*}
\frac{\Delta v}{\Delta t}=\frac{\SI{-25}{\volt}-\SI{50}{\volt}}{\SI{45}{\milli\second}-\SI{25}{\milli\second}}=\SI{-3500}{\volt \per \second}
\end{align*}
سے گھٹتا ہے لہٰذا اس دوران دباو کی مساوات
\begin{align*}
v(t)=-3500 t +75 \quad \quad (\SI{25}{\milli\second} \le t \le \SI{45}{\milli\second})
\end{align*}
ہو گی۔اس کے بعد دباو برقرار \عددی{\SI{-25}{\volt}} پر رہتا ہے لہٰذا اس کی مساوات درج ذیل ہو گی۔
\begin{align*}
v(t)=-25 \quad \quad (\SI{45}{\milli\second} \le t)
\end{align*}
مساوات \حوالہ{مساوات_امالہ_بار_دباو_تعلق_ب} استعمال کرتے ہوئے ان دورانیوں میں رو حاصل کرتے ہیں۔
\begin{align*}
i&=2\times 10^{-6} \times 5000=\SI{10}{\milli\ampere} \quad \quad \quad \quad  (0 \le t \le \SI{10}{\milli\second})\\
i&=2\times 10^{-6} \times 0=\SI{0}{\milli\ampere} \quad \quad \quad \quad \quad   (\SI{10}{\milli\second} \le t \le \SI{25}{\milli\second})\\
i&=2\times 10^{-6} \times (-3500)=\SI{-7}{\milli\ampere} \quad \quad   \quad  (\SI{25}{\milli\second} \le t \le \SI{45}{\milli\second})\\
i&=2\times 10^{-6} \times 0=\SI{0}{\milli\ampere} \quad \quad \quad \quad   \quad(\SI{45}{\milli\second} \le t)
\end{align*}
رو بالمقابل وقت کو شکل-ب میں دکھایا گیا ہے۔
\انتہا{مثال}
%=====================
\ابتدا{مثال}
گزشتہ مثال میں لمحہ \عددی{t=\SI{10}{\milli\second}}، \عددی{t=\SI{20}{\milli\second}} اور \عددی{t=\SI{50}{\milli\second}} پر برق گیر میں ذخیرہ مخفی تونائی دریافت کریں۔

حل:مساوات \حوالہ{مسوات_امالہ_توانائی_برق_گیر_الف} کے تحت جوابات درج ذیل ہیں۔
\begin{align*}
w_C(\SI{10}{\milli\second})&=\frac{2\times 10^{-6}\times 50^2}{2}=\SI{2.5}{\milli\joule}\\
w_C(\SI{20}{\milli\second})&=\frac{2\times 10^{-6}\times 50^2}{2}=\SI{2.5}{\milli\joule}\\
w_C(\SI{50}{\milli\second})&=\frac{2\times 10^{-6}\times (-25)^2}{2}=\SI{0.625}{\milli\joule}
\end{align*}
\انتہا{مثال}
%====================
\ابتدا{مشق}
برق گیر پر ذخیرہ بار کی قیمت \عددی{\SI{5}{\nano\coulomb}} ہے جبکہ اس پر دباو \عددی{\SI{100}{\volt}} ہیں۔برقی گنجائش دریافت کریں۔

جواب:\عددی{\SI{50}{\pico\farad}}
\انتہا{مشق}
%====================
\ابتدا{مثال}
ابتدائی طور پر بے بار \عددی{\SI{22}{\micro\farad}} کے برق گیر کی رو کو شکل \حوالہ{شکل_امالہ_مثال_تبدیل_ہوتی_رو_الف} میں دکھایا گیا ہے۔برق گیر کے دباو، طاقت اور ذخیرہ توانائی کے مساوات حاصل کرتے ہوئے  خط کھینچیں۔

\begin{figure}
\centering
\begin{tikzpicture}
\draw[gray](0,-0.75)--(0,2)node[left]{$i(t)$};
\draw[gray](0,0)--++(6,0)node[right]{$t(\si{\milli\second})$};
\draw(0,0)--++(2,0.9)--++(0,0.35)--++(1,0)--++(0,-1.75)--++(2,0)--++(0,0.5)--++(1,0);
\draw(0,0.9)--++(-0.1,0)node[left]{$\SI{1.8}{\milli\ampere}$};
\draw(0,1.25)--++(-0.1,0)node[left]{$\SI{2.5}{\milli\ampere}$};
\draw(0,-0.5)--++(-0.1,0)node[left]{$\SI{-1}{\milli\ampere}$};
\draw(2,0)--++(0,-0.1)node[below]{$20$};
\draw(3,0)node[above right]{$30$};
\draw(5,0)node[above]{$50$};
\end{tikzpicture}
\caption{(الف)}
\label{شکل_امالہ_مثال_تبدیل_ہوتی_رو_الف}
\end{figure} 

حل:دورانیہ \عددی{t=\SI{0}{\second}} تا \عددی{t=\SI{20}{\milli\second}} میں شرح رو
\begin{align*}
\frac{\dif i}{\dif t}=\frac{\Delta i}{\Delta t}=\frac{\SI{18}{\milli\ampere}-\SI{0}{\milli\ampere}}{\SI{20}{\milli\second}-\SI{0}{\milli\second}}=\SI{0.9}{\ampere\per\second}
\end{align*}
ہے جسے
\begin{align*}
\dif i =0.9 \dif t
\end{align*}
لکھ کر تکمل لیتے ہوئے رو کی مساوات
\begin{align*}
i=\int_{0}^{t} 0.9 \dif t=\left. 0.9  t\right|_{0}^{t}=0.9 t
\end{align*}
حاصل ہوتی ہے۔برق گیر پر ذخیرہ بار دریافت کرنے کی خاطر رو کی مساوات کو 
\begin{align*}
i=\frac{\dif q}{\dif t}=0.9 t
\end{align*}
لکھتے ہوئے تکمل لیتے ہیں۔
\begin{align*}
q=\int_{0}^{t}0.9 t \dif t=\left. 0.45 t^2 \right|_{0}^{t}=0.45 t^2
\end{align*}
مساوات \حوالہ{مساوات_امالہ_بار_دباو_تعلق} سے 
\begin{align*}
v(t)=\frac{q}{C}=\frac{0.45 t^2}{22\times 10^{-6}}=20455 t^2
\end{align*}
لکھا جائے گا اور یوں طاقت کی مساوات
\begin{align*}
p=vi =20455 t^2 \times 0.9 t=18410 t^3
\end{align*}
اور ذخیرہ توانائی کی مساوات
\begin{align*}
w_C=\int_0^t p \dif t=4603 t^4
\end{align*}
ہو گی۔ان مساوات سے لمحہ \عددی{t=\SI{20}{\milli\second}} پر 
\begin{gather}
\begin{aligned}\label{مساوات_امالہ_ابتدائی_قیمتیں_الف}
q(0.02)&=0.45 t^2=0.45\times 0.02^2=\SI{180}{\micro\coulomb}\\
v(0.02)&=20455 t^2=20455\times 0.02^2=\SI{8.182}{\volt}\\
w_C(0.02)&=4603 t^4=4603\times 0.02^4=\SI{737}{\micro\joule}
\end{aligned}
\end{gather}
ہوں گے۔

اسی طرح \عددی{\SI{20}{\milli\second}} تا \عددی{\SI{30}{\milli\second}} دورانیے کے لئے مساوات \حوالہ{مساوات_امالہ_ابتدائی_قیمتیں_الف} میں حاصل کی گئی مقداریں ابتدائی مقداریں تصور کی جائیں گی۔اس دورانیے میں
\begin{align*}
i=\SI{2.5}{\milli\ampere}
\end{align*}
ہے لہٰذا مساوات \حوالہ{مساوات_امالہ_ابتدائی_دباو_الف} کے تحت
\begin{align*}
v&=v(0.02)+\frac{1}{C}\int_{0.02}^t i \dif t\\
&=8.182 +\frac{1}{22\times 10^{-6}}\int_{0.02}^t 2.5\times 10^{-3} \dif t\\
&=33.182+113.636t
\end{align*}
اور
\begin{align*}
p&=iv=0.0025 (33.182+113.636t)=0.083+0.284t\\
w_C&=\frac{C v^2}{2}=\frac{22\times 10^{-6}}{2} (33.182+113.636t)^2
\end{align*}
ہوں گے جن سے اس دورانیے کے آخری لمحے پر
\begin{gather}
\begin{aligned}\label{مساوات_امالہ_ابتدائی_قیمتیں_ب}
v(0.03)&=33.182+113.636\times 0.03=\SI{36.591}{\volt}\\
w_C(0.03)&=\frac{Cv^2}{2}=\frac{22\times 10^{-6} \times 36.591^2}{2}=\SI{14.73}{\milli\joule}
\end{aligned}
\end{gather}
حاصل ہوتے ہیں۔

شکل \حوالہ{شکل_امالہ_مثال_تبدیل_ہوتی_رو_الف} میں \عددی{t=\SI{30}{\milli\second}} تا \عددی{t=\SI{50}{\milli\second}} کے متغیرات حاصل کرتے ہوئے مساوات \حوالہ{مساوات_امالہ_ابتدائی_قیمتیں_ب} کی قیمتیں ابتدائی قیمتیں تصور کی جائیں گی۔پہلے دباو کی مساوات حاصل کرتے ہیں۔
\begin{align*}
v&=v(0.03)+\frac{1}{C}\int_{0.03}^{t}-10^{-3} \dif t\\
&=\left. 36.591-\frac{10^{-3}}{22\times 10^{-6}}t\right|_{0.03}^{t}\\
&=37.955-45.455t
\end{align*}
طاقت کی مساوات درج ذیل ہے
\begin{align*}
p=&iv  \\
&=-0.001(37.955-45.455t)\\
&=-0.038+0.0455t
\end{align*}
جبکہ ذخیرہ توانائی
\begin{align*}
w_C&=\frac{Cv^2}{2}\\
&=\frac{22\times 10^{-6} (37.955-45.455t)^2}{2}
\end{align*}
ہے۔لمحہ \عددی{\SI{50}{\milli\second}} کے بعد رو صفر کے برابر ہے لہٰذا نہ تو برق گیر کا دباو تبدیل ہو گا اور نہ ہی اس میں ذخیرہ توانائی  کی قیمت تبدیل ہو گی۔
\انتہا{مثال}
%=================
\ابتدا{مشق}
شکل \حوالہ{شکل_امالہ_مشق_دباو_سے_رو-الف} میں \عددی{\SI{68}{\micro\farad}} کے برق گیر کا دباو دیا گیا ہے۔رو کی شکل کھینچیں۔

\begin{figure}
\centering
\begin{tikzpicture}
\draw[gray](0,0)--++(0,1.5)node[left]{$v(t)$};
\draw[gray](0,0)--++(6,0)node[right]{$t(\si{\milli\second})$};
\draw(0,0)--(3,1)--(5,0)--(6,0);
\draw(0,1)--++(-0.1,0)node[left]{$\SI{50}{\volt}$};
\draw(3,0)--++(0,-0.1)node[below]{$20$};
\draw(5,0)--++(0,-0.1)node[below]{$30$};
\end{tikzpicture}
\caption{دباو کا خط۔}
\label{شکل_امالہ_مشق_دباو_سے_رو-الف}
\end{figure}

جواب:پہلی \عددی{\SI{20}{\milli\second}} کے لئے \عددی{\SI{0.17}{\ampere}} اور اگلے \عددی{\SI{10}{\milli\second}} کے لئے \عددی{\SI{-0.34}{\ampere}} جبکہ بقایا وقت رو صفر ہے۔
\انتہا{مشق}
%=================
\ابتدا{مشق}
گزشتہ مثال میں لمحہ \عددی{t=\SI{20}{\milli\second}} پر برقی گیر میں ذخیرہ توانائی دریافت کریں۔

جواب:\عددی{\SI{85}{\milli\joule}}
\انتہا{مشق}
%============================
\ابتدا{مشق}
شکل \حوالہ{شکل_امالہ_مشق_دباو_سے_رو-ب} میں \عددی{\SI{2.2}{\micro\farad}} کے برق گیر کا دباو دیا گیا ہے۔رو کی شکل کھینچیں۔لمحہ \عددی{t=\SI{4}{\milli\second}} پر ذخیرہ توانائی دریافت کریں۔لمحہ \عددی{t=\SI{1.5}{\milli\second}} اور \عددی{\SI{5.5}{\milli\second}} پر رو دریافت کریں۔

\begin{figure}
\centering
\begin{tikzpicture}
\draw[gray](0,-1.5)--(0,1.5)node[left]{$v(t)$};
\draw[gray](0,0)--++(7,0)node[right]{$t(\si{\milli\second})$};
\draw(0,0)--(1,1)--(2,1)--(3,-0.5)--(4,-1)--(5,-1)--(6,0)--(7,0);
\foreach \x/\xx in {1/1,2/2,3/3,4/4,5/5,6/6}{\draw (\x,0)--++(0,-0.1)node[below]{$\xx$};}
\foreach \y/\yy in {-1/-10,-0.5/-5,1/10}{\draw(0,\y)--++(-0.1,0)node[left]{$\yy\, \si{\volt}$};}
\end{tikzpicture}
\caption{دباو کا خط۔}
\label{شکل_امالہ_مشق_دباو_سے_رو-ب}
\end{figure}

جواب:\عددی{\SI{110}{\micro\joule}}، \عددی{\SI{0}{\ampere}}، \عددی{\SI{-22}{\milli\ampere}}
\انتہا{مشق}
%=================

\ابتدا{مشق}
شکل \حوالہ{شکل_امالہ_مشق_دباو_سے_رو-پ} میں \عددی{\SI{100}{\micro\farad}} کے برق گیر کی رو دی گئی ہے۔دباو کا خط کھینچیں۔لمحہ \عددی{t=\SI{3}{\milli\second}} پر ذخیرہ توانائی دریافت کریں۔

\begin{figure}
\centering
\begin{tikzpicture}
\draw[gray](0,-2)--(0,1)node[left]{$i(t)$};
\draw[gray](0,0)--++(6,0)node[right]{$t(\si{\milli\second})$};
\draw(0,0)--(2,-1)--(3,-1)--(3,-1.5)--(4,-1.5)--(4,0.5)--(5,0.5)--(5,0)--(6,0);
\foreach \x/\xx in {1/1,2/2,3/3,4/4,5/5}{\draw (\x,0)--++(0,-0.1)node[below]{$\xx$};}
\foreach \y/\yy in {-1.5/-15,-1/-10,0.5/5}{\draw(0,\y)--++(-0.1,0)node[left]{$\yy\, \si{\ampere}$};}
\end{tikzpicture}
\caption{رو کا خط۔}
\label{شکل_امالہ_مشق_دباو_سے_رو-پ}
\end{figure}

جواب:\عددی{\SI{2}{\joule}}
\انتہا{مشق}
%=================

\حصہ{امالہ گیر}
\اصطلاح{امالہ گیر}\فرہنگ{امالہ گیر}\حاشیہب{inductor}\فرہنگ{inductor} عموماً موصل تار کے \اصطلاح{لچھے}\فرہنگ{لچھا}\حاشیہب{coil}\فرہنگ{coil} کی صورت کا ہوتا ہے۔ایسا لچھا کسی \اصطلاح{مقناطیسی مرکز}\فرہنگ{مرکز!مقناطیسی}\فرہنگ{مقناطیسی مرکز}\حاشیہب{magnetic core}\فرہنگ{core!magnetic} یا \اصطلاح{غیر مقناطیسی مرکز}\فرہنگ{غیر مقناطیسی مرکز}\فرہنگ{مرکز!غیر مقناطیسی}\حاشیہب{non-magnetic core}\فرہنگ{core!non-magnetic} پر لپیٹا ہو سکتا ہے۔ مقناطیسی مرکز کے لچھے \اصطلاح{ٹرانسفارمر}\فرہنگ{ٹرانسفارمر}\حاشیہب{transformer}\فرہنگ{transformer} اور \اصطلاح{فلٹر}\فرہنگ{فلٹر}\حاشیہب{filter}\فرہنگ{filter} میں استعمال کئے جاتے ہیں جبکہ غیر مقناطیسی مرکز کے لچھے مواصلاتی نظام میں اہم کردار ادا کرتے ہیں۔

تاریخی طور پر پہلے یہ معلوم ہوا کہ رو گزارتی تار کے گرد مقناطیسی میدان پیدا ہوتا ہے۔ایسی مقناطیسی میدان اور میدان پیدا کرنے والی رو کے مابین راست تناسبی تعلق پایا جاتا ہے۔اس کے بعد معلوم ہوا کہ بدلتا مقناطیسی میدان برقی دباو پیدا کرتا ہے جہاں دباو اور مقناطیسی میدان پیدا کرنے والی رو کی شرح کے مابین راست تناسبی تعلق پایا جاتا ہے۔اسی تعلق کو درج ذیل مساوات پیش کرتی ہے
\begin{align}\label{مساوات_امالہ_بنیادی_مساوات_امالہ_الف}
v=L \frac{\dif i}{\dif t}
\end{align}
جہاں تناسبی مستقل کو \عددی{L} لکھا اور \اصطلاح{امالہ}\فرہنگ{امالہ}\حاشیہب{inductance}\فرہنگ{inductance} پکارا جاتا ہے۔امالہ کی اکائی\حاشیہد{امالہ کی اکائی امریکی تخلیق کار یوسف ہینری کے نام سے منسوب ہے۔} کو \اصطلاح{ہینری}\فرہنگ{ہینری}\حاشیہب{Henry}\فرہنگ{Henry} پکارا اور \عددی{\si{\henry}} سے ظاہر کیا جاتا ہے۔آپ دیکھ سکتے ہیں کہ ایک وولٹ سیکنڈ فی ایمپیئر \عددی{\si{\volt\second\per\ampere}} کو ہینری کہا گیا ہے۔ 

اس مساوات کی تکمل صورت سے رو حاصل ہوتی ہے
\begin{align}
i=\int_{-\infty}^{t}\frac{1}{L} v \dif t
\end{align}
جہاں ازل \عددی{-\infty} سے لمحہ \عددی{t} تک تکمل لیا گیا ہے۔مستقل قیمت کی امالہ کی صورت میں \عددی{L} کو تکمل کے باہر نکالا جا سکتا ہے۔
\begin{align}
i=\frac{1}{L}\int_{-\infty}^{t} v \dif t
\end{align}
اس تکمل کو دو ٹکڑوں میں لکھا جا سکتا ہے 
\begin{gather}
\begin{aligned}
i&=\frac{1}{L}\int_{-\infty}^{t_0} v \dif t+\frac{1}{L} \int_{t_0}^{t} v \dif t\\
&=i(t_0)+\frac{1}{L} \int_{t_0}^{t} v \dif t
\end{aligned}
\end{gather}
جہاں پہلا ٹکڑا ازل سے \عددی{t_0} تک اور دوسرا ٹکڑا \عددی{t_0} سے \عددی{t} حاصل کیا گیا ہے۔مندرجہ بالا مساوات میں لمحہ \عددی{t_0} پر امالہ گیر کی رو کو \عددی{i(t_0)} کہا گیا ہے۔

امالہ کو فراہم طاقت سے امالہ کو منتقل توانائی \عددی{w_L} دریافت کی جا سکتی ہے۔
\begin{align}
p=vi
\end{align}
سے
\begin{align}\label{مساوات_امالہ_کو_مہیا_طاقت_الف}
p=\frac{\dif w_L}{\dif t}=\left[L \frac{\dif i}{\dif t}\right] i
\end{align}
لکھتے ہوئے اور تکمل لینے سے
\begin{align*}
w_L&=\int_{-\infty}^{t} \left[L\frac{\dif i}{\dif t}\right]i\dif t\\
&=L\int_{0}^{i} i \dif i
\end{align*}

\begin{align}
w_L=\frac{Li^2}{2}
\end{align}
حاصل ہوتا ہے جہاں وقت کی ابتدا \عددی{t=-\infty} پر \عددی{i=0} تصور کی گئی ہے۔

تصور کریں کہ ایک دور میں یک سمتی رو پائی جاتی ہو۔اب یک سمتی رو وقت کے ساتھ تبدیل نہیں ہوتی لہٰذا مساوات \حوالہ{مساوات_امالہ_بنیادی_مساوات_امالہ_الف} کے تحت اس دور میں موجود امالہ پر دباو صفر کے برابر ہو گا۔ہم کہہ سکتے ہیں کہ یک سمتی رو کی نقطہ نظر سے امالہ بطور قصر دور کردار ادا کرتی ہے۔یوں کسی بھی دور کا یک سمتی تجزیہ کرتے ہوئے دور میں موجود تمام امالہ کو قصر دور تصور کیا جاتا ہے۔

امالہ میں فوراً رو تبدیل کرنے کے لئے مساوات \حوالہ{مساوات_امالہ_کو_مہیا_طاقت_الف} کے تحت  لامحدود طاقت درکار ہو گی۔کائنات میں لامحدود طاقت کا منبع کہیں نہیں پایا جاتا لہٰذا امالہ کی رو کو فوراً تبدیل کرنا ناممکن ہے۔اس حقیقت کی مساواتی صورت درج ذیل ہے۔
\begin{align}\label{مساوات_امالہ_امالہ_گیر_رو_بلا_جوڑ_ہے}
i_L(t_+)=i_L(t_-)
\end{align}
مساوات \حوالہ{مساوات_امالہ_امالہ_گیر_رو_بلا_جوڑ_ہے} کے تحت امالہ گیر کی رو کسی بھی لمحے \عددی{t} کے فوراً بعد \عددی{t_+} اور اس لمحے کے فوراً  پہلے \عددی{t_-} برابر ہوں گے۔یوں امالہ گیر کی رو \اصطلاح{بلا جوڑ تفاعل}\فرہنگ{بلا جوڑ!تفاعل}\فرہنگ{تفاعل!بلا جوڑ}\حاشیہب{continuous function}\فرہنگ{function!continuous}\فرہنگ{continuous!function} ہے جس میں \اصطلاح{سیڑھی نما}\فرہنگ{سیڑھی نما}\فرہنگ{step} یکدم تبدیلی ممکن نہیں ہے۔یہ ایک اہم نتیجہ ہے جس کے تحت دور میں سوئچ کو چالو سے غیر چالو (یا غیر چالو سے چالو) کرنے کے فوراً بعد امالہ میں رو کی قیمت وہی ہو گی جو سوئچ چالو (یا غیر چالو) کرنے سے پہلے تھی۔

%=============================
\ابتدا{مثال}\شناخت{مثال_امالہ_یکسمتی_دور_الف}
شکل \حوالہ{شکل_امالہ_یکسمتی_دور_الف} میں ذخیرہ توانائی دریافت کریں۔

\begin{figure}
\centering
\begin{subfigure}{1\textwidth}
\centering
\begin{tikzpicture}
\draw(0,0) to [american voltage source,l={$\SI{10}{\volt}$}]++(0,\y) to [inductor,l={${L_1=\SI{2}{\milli\henry}}$}]++(\x,0) to [resistor,l={$\SI{6}{\ohm}$}]++(\x,0) to [inductor,l={${L_2=\SI{1}{\milli\henry}}$}]++(\x,0) to [resistor,l={$\SI{5}{\ohm}$}]++(\x,0) to [short]++(\x,0) to [resistor,l={$\SI{2}{\ohm}$}]++(0,-\y) to [short] (0,0);
\draw(\x,0) to [capacitor,*-*,l={$\SI{3}{\micro\farad}$}]++(0,\y);
\draw(3*\x,0) to [american current source,*-*,l={$\SI{4}{\ampere}$}]++(0,\y);
\draw(4*\x,0) to [capacitor,*-*,l={$\SI{10}{\micro\farad}$}]++(0,\y);
\draw(\x-\dx,1/4*\y)node[left]{$C_1$};
\draw(4*\x-\dx,1/4*\y)node[left]{$C_2$};
\end{tikzpicture}
\caption*{(الف)}
\end{subfigure}
\begin{subfigure}{1\textwidth}
\centering
\begin{tikzpicture}
\draw(0,0) to [american voltage source,l={$\SI{10}{\volt}$}]++(0,\y) to [short]++(\x,0) to [resistor,l={$\SI{6}{\ohm}$}]++(\x,0) to [short,i={$I_1$}]++(\x,0) to [resistor,l={$\SI{5}{\ohm}$}]++(\x,0) to [short,i={$I_2$}]++(\x,0) to [resistor,l={$\SI{2}{\ohm}$}]++(0,-\y) to [short] (0,0);
\draw(3*\x,0)node[ground]{} to [american current source,*-*,l={$\SI{4}{\ampere}$}]++(0,\y)node[above]{$J$};
\draw(\x,0) to [short,*-o]++(0,\y/8);
\draw(\x,\y) to [short,*-o]++(0,-\y/8);
\draw(4*\x,0) to [short,*-o]++(0,\y/8);
\draw(4*\x,\y) to [short,*-o]++(0,-\y/8);
\draw(\x+\dx/2,\y/2)node{$\begin{aligned}&+ \\& V_{C1} \\ &- \end{aligned}$};
\draw(4*\x+\dx/2,\y/2)node{$\begin{aligned}&+ \\& V_{C2} \\ &- \end{aligned}$};
\end{tikzpicture}
\caption*{(ب)}
\end{subfigure}
\caption{مثال \حوالہ{مثال_امالہ_یکسمتی_دور_الف} کا دور۔}
\label{شکل_امالہ_یکسمتی_دور_الف}
\end{figure}

حل:اس دور میں صرف یک سمتی منبع پائے جاتے ہیں۔ہم اس حقیقت پر بحث کر چکے ہیں کہ یک سمتی ادوار میں امالہ کو قصر دور اور برق گیر کو کھلا دور تصور کیا جاتا ہے۔ایسا ہی کرتے ہوئے  شکل-ب حاصل ہوتا ہے جسے آپ اپنی پسندیدہ  ترکیب سے حل کر سکتے ہیں۔نچلی جوڑ کو زمین لیتے ہوئے جوڑ \عددی{J} پر کرخوف مساوات رو
\begin{align*}
I_1+4=I_2
\end{align*} 
جبکہ بیرونی دائرے پر کرخوف مساوات دباو
\begin{align*}
10=6I_1+(5+2)I_2
\end{align*}
 لکھتے ہیں۔انہیں حل کرتے ہوئے درج ذیل حاصل ہوتا ہے۔
\begin{align*}
I_1&=-\frac{18}{13}\, \si{\ampere}\\
I_2&=\frac{34}{13}\,\si{\ampere}
\end{align*}
برق گیر \عددی{C_1} پر دباو شکل کو دیکھ کر لکھی جا سکتی ہے  جبکہ \عددی{C_2}  پر دباو کو اوہم کے قانون کی مدد سے لکھا جا سکتا ہے۔
\begin{align*}
V_{C1}&=\SI{10}{\volt}\\
V_{C2}&=2 \times \frac{34}{13}=\frac{68}{13}\,\volt
\end{align*}
ان حقائق کو استعمال کرتے ہوئے برق گیر اور امالہ میں ذخیرہ توانائی دریافت کر سکتے ہیں۔
\begin{align*}
w_{C1}&=\frac{3\times 10^{-6} \times 10^2}{2}=\SI{0.15}{\milli\joule}\\
w_{C2}&=\frac{10 \times 10^{-6} \left(\frac{68}{13}\right)^2}{2}=\SI{0.14}{\milli\joule}\\
w_{L1}&=\frac{0.002\times \left(\frac{18}{13}\right)^2 }{2}=\SI{1.92}{\milli\joule}\\
w_{L2}&=\frac{0.001\times \left(\frac{18}{13}\right)^2}{2}=\SI{0.96}{\milli\joule}
\end{align*}
\انتہا{مثال}
%================================
\ابتدا{مثال}\شناخت{مثال_امالہ_یکسمتی_دور_ب}
امالہ کی رو کے خط کو شکل \حوالہ{شکل_امالہ_یکسمتی_دور_ب}-الف میں دکھایا گیا ہے۔اس کے دباو کا خط کھینچیں۔امالہ کی قیمت \عددی{\SI{30}{\milli\henry}} ہے۔

\begin{figure}
\centering
\begin{subfigure}{0.5\textwidth}
\begin{tikzpicture}
\draw[gray](0,0)--++(4,0)node[right]{$t(\si{\milli\second})$};
\draw[gray](0,-2.5)--(0,1.5)node[left]{$i(t)$};
\draw(-0.5,0)--(0,0)--(2,1)--(3,0)--(4,0);
\draw(0,1)--++(-0.1,0)node[left]{$\SI{60}{\milli\ampere}$};
\draw(2,0)--++(0,-0.1)node[below]{$20$};
\draw(3,0)--++(0,-0.1)node[below]{$30$};
\end{tikzpicture}
\caption*{(الف)}
\end{subfigure}%
\begin{subfigure}{0.5\textwidth}
\begin{tikzpicture}
\draw[gray](0,0)--++(4,0)node[right]{$t(\si{\milli\second})$};
\draw[gray](0,-2.5)--(0,1.5)node[left]{$i(t)$};
\draw(-0.5,0)--(0,0)--(0,1)--(2,1)--(2,-2)--(3,-2)--(3,0)--(4,0);
\draw(0,1)node[left]{$\SI{90}{\milli\volt}$};
\draw(0,-2)node[left]{$\SI{180}{\milli\volt}$};
\draw(2,0)--++(0,-0.1)node[below left]{$20$};
\draw(3,0)--++(0,-0.1)node[below right]{$30$};
\end{tikzpicture}
\caption*{(ب)}
\end{subfigure}%
\caption{مثال \حوالہ{مثال_امالہ_یکسمتی_دور_ب} کا دور۔}
\label{شکل_امالہ_یکسمتی_دور_ب}
\end{figure}

حل:امالہ گیر کی رو سے امالہ گیر کا دباو مساوات \حوالہ{مساوات_امالہ_بنیادی_مساوات_امالہ_الف} کی مدد سے حاصل کیا جاتا ہے۔وقت \عددی{t=-\infty} تا \عددی{t=0} رو صفر کے برابر ہے لہٰذا
\begin{align*}
v=30\times 10^{-3} \left(\frac{0}{-\infty-0}\right)=\SI{0}{\volt}
\end{align*}
ہو گا۔اگلا دورانیہ \عددی{t=0} تا \عددی{t=\SI{20}{\milli\second}} ہے جس میں رو کی قیمت  یکساں شرح سے مسلسل بڑھتے ہوئے \عددی{i=0} سے \عددی{i=\SI{60}{\milli\ampere}} ہو جاتی ہے لہٰذا اس دوران
\begin{align*}
v=30\times 10^{-3} \left(\frac{0.06-0}{0.02-0}\right)=\SI{90}{\milli\volt}
\end{align*}
ہو گا۔دورانیہ \عددی{\SI{20}{\milli\second}} تا \عددی{\SI{30}{\milli\second}} میں دباو درج ذیل ہو گا۔
\begin{align*}
v=30\times 10^{-3} \left(\frac{0-0.06}{0.03-0.02}\right)=\SI{-180}{\milli\volt}
\end{align*}
\عددی{\SI{30}{\milli\second}} کے بعد رو صفر رہتی ہے لہٰذا
\begin{align*}
v=30\times 10^{-3} \left(\frac{0}{\infty-0.03}\right)=\SI{0}{\volt}
\end{align*}
ہو گا۔ان نتائج کو شکل \حوالہ{شکل_امالہ_یکسمتی_دور_ب}-ب میں دکھایا گیا ہے۔
\انتہا{مثال}
%================================
\ابتدا{مثال}
امالہ گیر کی رو \عددی{i(t)=5\cos 377 t} جبکہ اس کی امالہ \عددی{\SI{100}{\milli\henry}} ہے۔امالہ گیر کا دباو اور اس میں ذخیرہ توانائی کی مساوات حاصل کریں۔

حل: مساوات \حوالہ{مساوات_امالہ_بنیادی_مساوات_امالہ_الف}  سے دباو درج ذیل لکھا جاتا ہے۔
\begin{align*}
v&=L \frac{\dif i}{\dif t}\\
&=0.1 \times (-5 \times 377 \sin 377 t)\\
&=-188.5 \sin 377 t \quad \si{\volt}
\end{align*}
ذخیرہ توانائی کو درج ذیل لکھا جا سکتا ہے۔
\begin{align*}
w_L(t)&=\frac{L i^2}{2}\\
&=\frac{0.1 \times \left(5\cos 377 t\right)^2}{2}\\
&=1.25 \cos^2 377t \, \si{\joule}
\end{align*}
\انتہا{مثال}
%==================================
\ابتدا{مشق}\شناخت{مشق_امالہ_رو_دباو_خط_الف}
رو کا خط شکل \حوالہ{شکل_امالہ_مشق_رو_دباو_خط_الف} میں دکھایا گیا ہے۔دباو کا خط کھینچیں۔امالہ کی قیمت \عددی{\SI{2}{\henry}} ہے۔

\begin{figure}
\centering
\begin{subfigure}{0.5\textwidth}
\centering
\begin{tikzpicture}
\draw[gray](0,-1.35)--(0,1.5)node[left]{$i(t)$};
\draw[gray](0,0)--++(4.25,0)node[right]{$t(\si{\milli\second})$};
\draw(-0.25,0)--(0,0)--(2,1)--(3,1)--(4,0)--(4.25,0);
\draw(0,1)node[left]{$\SI{12}{\ampere}$};
\draw(2,0)node[below]{$2$};
\draw(3,0)node[below]{$3$};
\draw(4,0)node[below]{$4$};
\end{tikzpicture}
\caption*{(الف)}
\end{subfigure}%
\begin{subfigure}{0.5\textwidth}
\centering
\begin{tikzpicture}
\draw[gray](0,-1.35)--(0,1.5)node[left]{$v(t)$};
\draw[gray](0,0)--++(4.25,0)node[right]{$t(\si{\milli\second})$};
\draw(-0.25,0) --(0,0)--(0,0.6)--(2,0.6)--(2,0)--(3,0)--(3,-1.2)--(4,-1.2)--(4,0)--(4.25,0);
\draw(0,0.6)node[left]{$\SI{12}{\volt}$};
\draw(0,-1.2)node[left]{$\SI{-24}{\volt}$};
\draw(2,0)node[below]{$2$};
\draw(3,0)node[above]{$3$};
\draw(4,0)node[above]{$4$};
\end{tikzpicture}
\caption*{(ب)}
\end{subfigure}%
\caption{مشق \حوالہ{مشق_امالہ_رو_دباو_خط_الف} کا دور۔}
\label{شکل_امالہ_مشق_رو_دباو_خط_الف}
\end{figure}

جواب:شکل \حوالہ{شکل_امالہ_مشق_رو_دباو_خط_الف}-ب میں دباو کا خط دکھایا گیا ہے۔
\انتہا{مشق}
%=================================
\ابتدا{مشق}
گزشتہ مشق میں لمحہ \عددی{t=\SI{3.5}{\milli\second}} پر امالہ گیر میں ذخیرہ توانائی دریافت کریں۔

جواب:\عددی{\SI{36}{\joule}}
\انتہا{مشق}
%===========================
\ابتدا{مشق}\شناخت{مشق_امالہ_رو_دباو_خط_ب}
پانچ ہینری امالہ گیر کا دباو شکل \حوالہ{شکل_امالہ_مشق_رو_دباو_خط_ب}-الف میں دکھایا گیا ہے۔رو کا خط کھینچیں۔

\begin{figure}
\centering
\begin{subfigure}{0.5\textwidth}
\centering
\begin{tikzpicture}
\draw[gray](0,0)--(4.25,0)node[right]{$t(\si{\milli\second})$};
\draw[gray](0,-1)--(0,2.5)node[left]{$v(t)$};
\draw(-0.25,0)--(0,0)--(0,2)--(2,2)--(2,1)--(3,1)--(3,-0.5)--(4,-0.5)--(4,0)--(4.25,0);
\draw(0,-0.5)node[left]{$\SI{-50}{\volt}$};
\draw(0,1)node[left]{$\SI{100}{\volt}$};
\draw(0,2)node[left]{$\SI{200}{\volt}$};
\draw(1,0)--++(0,-0.1)node[below]{$1$};
\draw(2,0)--++(0,-0.1)node[below]{$2$};
\draw(3,0)++(0,-0.1)node[below left]{$3$};
\draw(4,0)node[above]{$4$};
\end{tikzpicture}
\caption*{(الف)}
\end{subfigure}%
\begin{subfigure}{0.5\textwidth}
\centering
\begin{tikzpicture}
\draw[gray](0,0)--(4.25,0)node[right]{$t(\si{\milli\second})$};
\draw[gray](0,-1)--(0,2.5)node[left]{$i(t)$};
\draw(-0.25,0)--(0,0)--(2,1.6)--(3,2)--(4,1.8)--(4.25,1.8);
\draw(0,1.6)--++(-0.1,0)node[below left]{$\SI{80}{\ampere}$};
\draw(0,1.8)--++(-0.1,0)node[left]{$\SI{90}{\ampere}$};
\draw(0,2)--++(-0.1,0)node[above left]{$\SI{100}{\ampere}$};
\draw(1,0)--++(0,-0.1)node[below]{$1$};
\draw(2,0)--++(0,-0.1)node[below]{$2$};
\draw(3,0)--++(0,-0.1)node[below]{$3$};
\draw(4,0)--++(0,-0.1)node[below]{$4$};
\end{tikzpicture}
\caption*{(ب)}
\end{subfigure}%
\caption{مشق \حوالہ{مشق_امالہ_رو_دباو_خط_ب} کا دور۔}
\label{شکل_امالہ_مشق_رو_دباو_خط_ب}
\end{figure}

جواب:رو کا خط شکل \حوالہ{شکل_امالہ_مشق_رو_دباو_خط_ب}-ب میں دکھایا گیا ہے۔
\انتہا{مشق}
%=============================
\ابتدا{مشق}\شناخت{مشق_امالہ_رو_دباو_خط_پ}
امالہ گیر کے دباو کا خط شکل \حوالہ{شکل_امالہ_مشق_رو_دباو_خط_پ} میں دکھایا گیا ہے۔لمحہ \عددی{t=0} پر \عددی{i(0)=\SI{0.1}{\ampere}} کی صورت میں رو کا خط حاصل کریں۔امالہ \عددی{\SI{0.1}{\henry}} کے برابر ہے۔ لمحہ \عددی{t=\SI{3}{\milli\second}} پر امالہ گیر میں ذخیرہ توانائی دریافت کریں۔

\begin{figure}
\centering
\begin{subfigure}{0.5\textwidth}
\centering
\begin{tikzpicture}
\draw[gray](0,0)--(4.25,0)node[right]{$t(\si{\milli\second})$};
\draw[gray](0,-1.5)--(0,1.5)node[left]{$v(t)$};
\draw(0,0)--(1,1)--(1,0)--(2,0)--(2,-0.5)--(3,-0.5)--(3,-1)--(4,-1)--(4,0)--(4.25,0);
\draw(0,1)--++(-0.1,0)node[left]{$\SI{10}{\volt}$};
\draw(0,-1)--++(-0.1,0)node[left]{$\SI{-10}{\volt}$};
\draw(0,-0.5)--++(-0.1,0)node[left]{$\SI{-5}{\volt}$};
\draw(1,0)--++(0,-0.1)node[below]{$1$};
\draw(2,0)--++(0,-0.1)node[below left]{$2$};
\draw(3,0)--++(0,-0.1)node[below]{$3$};
\draw(4,0)--++(0,-0.1)node[below left]{$4$};
\end{tikzpicture}
\caption*{(الف)}
\end{subfigure}%
\begin{subfigure}{0.5\textwidth}
\centering
\begin{tikzpicture}
\draw[gray](0,0)--(4.25,0)node[right]{$t(\si{\milli\second})$};
\draw[gray](0,0)--(0,2)node[left]{$i(t)$};
\draw(0,1)
\foreach \kx in {0,0.01,...,1}{--(\kx,1+0.5*\kx*\kx)}--(1,1.5)--(2,1.5)--(3,1)--(4,0);
\draw(0,1)--++(-0.1,0)node[left]{$\SI{0.1}{\ampere}$};
\draw(0,1.5)--++(-0.1,0)node[left]{$\SI{0.15}{\ampere}$};
%\draw(0,-0.5)--++(-0.1,0)node[left]{$\SI{-5}{\volt}$};
\draw(1,0)--++(0,-0.1)node[below]{$1$};
\draw(2,0)--++(0,-0.1)node[below left]{$2$};
\draw(3,0)--++(0,-0.1)node[below]{$3$};
\draw(4,0)--++(0,-0.1)node[below]{$4$};
\end{tikzpicture}
\caption*{(ب)}
\end{subfigure}%
\caption{مشق \حوالہ{مشق_امالہ_رو_دباو_خط_پ} کا دور۔}
\label{شکل_امالہ_مشق_رو_دباو_خط_پ}
\end{figure}

جواب:رو کا خط شکل \حوالہ{شکل_امالہ_مشق_رو_دباو_خط_پ} میں دکھایا گیا ہے۔لمحہ \عددی{t=\SI{3}{\milli\second}} پر \عددی{w_L(\SI{3}{\milli\second})=\SI{0.5}{\milli\joule}} ہے۔
\انتہا{مشق}
%=============================
\ابتدا{مشق}\شناخت{مشق_امالہ_رو_دباو_خط_ت}
شکل \حوالہ{شکل_امالہ_مشق_رو_دباو_خط_ت} میں \عددی{\SI{1}{\milli\henry}}، \عددی{\SI{4}{\milli\henry}}، \عددی{\SI{3}{\micro\farad}} اور \عددیء{\SI{4}{\micro\farad}} میں ذخیرہ توانائی دریافت کریں۔

\begin{figure}
\centering
\begin{tikzpicture}
\draw(0,0) to [american voltage source,l={$\SI{10}{\volt}$}]++(0,\y) to [resistor,l={$\SI{2}{\ohm}$}]++(0,\y) to [short]++(\x,0) to [inductor,l={$\SI{1}{\milli\henry}$}]++(\x,0) to [resistor,l={$\SI{6}{\ohm}$}]++(\x,0) to [inductor,l={$\SI{4}{\milli\henry}$}]++(\x,0) to [resistor,l={$\SI{10}{\ohm}$}]++(0,-2*\y) to [short] (0,0);
\draw(\x,2*\y) to [short,*-]++(0,\y) to [american current source,l={$\SI{2}{\ampere}$}]++(3*\x,0) to [short,-*]++(0,-\y);
\draw(\x,0) to [capacitor,*-*,l={$\SI{3}{\micro\farad}$}]++(0,2*\y);
\draw(2*\x,0) to [resistor,*-,l={$\SI{8}{\ohm}$}]++(0,\y) to [capacitor,-*,l={$\SI{4}{\micro\farad}$}]++(0,\y);
\end{tikzpicture}
\caption{مشق \حوالہ{مشق_امالہ_رو_دباو_خط_ت} کا دور۔}
\label{شکل_امالہ_مشق_رو_دباو_خط_ت}
\end{figure}

جوابات:\عددی{\SI{302}{\micro\joule}}، \عددی{\SI{0.907}{\micro\joule}}، \عددی{\SI{85.6}{\micro\joule}}، \عددی{\SI{114}{\micro\joule}}
\انتہا{مشق}
%=============================

\حصہ{برق گیر اور امالہ گیر کے خصوصیات}
برقی گنجائش، برقی گنجائش  کی قیمت میں خلل اور دباو، برق گیر کے  اہم خصوصیات ہیں۔ معیاری برق گیر چند \عددی{\si{\pico\farad}} سے تقریباً \عددی{\SI{50}{\milli\farad}} تک کی قیمتوں میں عام دستیاب ہے۔ان سے کم اور زیادہ قیمتیں بھی دستیاب ہیں۔ یہ برق گیر عموماً \عددی{\SI{6.3}{\volt}} تا \عددی{\SI{500}{\volt}} تک کے مختلف دباو کے لئے دستیاب ہیں۔زیادہ دباو کے برق گیر بھی دستیاب ہیں۔برق گیر کو اس کی معین دباو سے زیادہ دباو پر ہرگز استعمال نہ کریں چونکہ ایسا کرنے سے  برق گیر تباہ ہو سکتا ہے۔برقی گنجائش میں خلل کی عمومی قیمتیں \عددی{\SI{\mp 5}{\percent}}، \عددی{\SI{\mp 10}{\percent}} اور \عددی{\SI{\mp 20}{\percent}} ہیں۔ جدول \حوالہ{جدول_امالہ_معیاری_برقی_گنجائش} میں معیاری دستیاب برقی گیر کی گنجائش دی گئی ہے۔
%==============================

\begin{table}\caption{معیاری برق گیر کے گنجائش کی قیمتیں۔}
\centering
\begin{tabular}{lllllllllll}
$\si{\pico\farad}$ & $\si{\pico\farad}$ & $\si{\pico\farad}$ & $\si{\pico\farad}$ & $\si{\micro\farad}$ & $\si{\micro\farad}$ &  $\si{\micro\farad}$ &  $\si{\micro\farad}$ &  $\si{\micro\farad}$ &  $\si{\micro\farad}$ &  $\si{\micro\farad}$ \\
\hline
$1$ & $10$ &$100$ & $1000$ &$0.010$ & $0.10$ &$1.0$ & $10$ &$100$ & $1000$ & $\num{10000}$\\
 & $12$ &$120$ & $1200$ &$0.012$ & $0.12$ &$1.2$ & $12$ &$120$ & $1200$ & $\num{12000}$\\
 $1.5$& $15$ &$150$ & $1500$ &$0.015$ & $0.15$ &$1.5$ & $15$ &$150$ & $1500$ & $\num{15000}$\\
 & $18$ &$180$ & $1800$ &$0.018$ & $0.18$ &$1.8$ & $18$ &$180$ & $1800$ & $\num{18000}$\\
$2$ & $20$ &$200$ & $2000$ &$0.020$ & $0.20$ &$2.0$ & $20$ &$200$ & $2000$ & $\num{20000}$\\
 & $22$ &$220$ & $2200$ &$0.022$ & $0.22$ &$2.2$ & $22$ &$220$ & $2200$ & $\num{22000}$\\
 & $27$ &$270$ & $2700$ &$0.027$ & $0.27$ &$2.7$ & $27$ &$270$ & $2700$ & $\num{27000}$\\
$3$ & $33$ &$330$ & $3300$ &$0.330$ & $0.33$ &$3.3$ & $33$ &$330$ & $3300$ & $\num{33000}$\\
$4$ & $39$ &$390$ & $3900$ &$0.390$ & $0.39$ &$3.9$ & $39$ &$390$ & $3900$ & $\num{39000}$\\
$5$ & $47$ &$470$ & $4700$ &$0.470$ & $0.47$ &$3.3$ & $47$ &$470$ & $4700$ & $\num{47000}$\\
$6$ & $51$ &$510$ & $5100$ &$0.510$ & $0.51$ &$3.3$ & $51$ &$510$ & $5100$ & $\num{51000}$\\
$7$ & $56$ &$560$ & $5600$ &$0.560$ & $0.56$ &$3.3$ & $56$ &$560$ & $5600$ & $\num{56000}$\\
$8$ & $68$ &$680$ & $6800$ &$0.680$ & $0.68$ &$3.3$ & $68$ &$680$ & $6800$ & $\num{68000}$\\
$9$ & $82$ &$820$ & $8200$ &$0.820$ & $0.82$ &$3.3$ & $82$ &$820$ & $8200$ & $\num{82000}$
\end{tabular}
\label{جدول_امالہ_معیاری_برقی_گنجائش}
\end{table}
%==============================

امالہ گیر کو موصل تار سے بنایا جاتا ہے لہٰذا نہ چاہتے ہوئے بھی اس کی مزاحمت ہو گی۔ امالہ گیر کے اہم خصوصیات اس کی امالہ اور مزاحمت ہیں۔امالہ گیر \عددی{\SI{1}{\nano\henry}} تا \عددی{\SI{100}{\milli\henry}} کی قیمتوں میں عام دستیاب ہے۔اس سے کم یا زیادہ قیمتیں بھی دستیاب ہیں۔امالہ کی قیمتیں \عددی{\SI{\mp 5}{\percent}} اور \عددی{\SI{\mp 10}{\percent}} کے خلل میں دستیاب ہیں۔جدول \حوالہ{جدول_امالہ_امالہ_عمومی_دستیاب_قیمتیں} میں امالہ کی عمومی دستیاب قیمتیں دی گئی ہیں۔

\begin{table}\caption{امالہ کی عمومی دستیاب قیمتیں۔}
\centering
\begin{tabular}{lllllllll}
$\si{\nano\henry}$ & $\si{\nano\henry}$ &$\si{\nano\henry}$ &$\si{\micro\henry}$ &$\si{\micro\henry}$ &$\si{\micro\henry}$ &$\si{\milli\henry}$ &$\si{\milli\henry}$ &$\si{\milli\henry}$ \\
\hline
$1$ & $10$ &$100$ & $1.0$ & $10$ &$100$ & $1.0$ & $10$ & $100$ \\
$1.2$ & $12$ &$120$ & $1.2$ & $12$ &$120$ & $1.2$ & $12$ &  \\
$1.5$ & $15$ &$150$ & $1.5$ & $15$ &$150$ & $1.5$ & $15$ &  \\
$1.8$ & $18$ &$180$ & $1.8$ & $18$ &$180$ & $1.8$ & $18$ &  \\
$2$ & $20$ &$200$ & $2.0$ & $20$ &$200$ & $2.0$ & $20$ &  \\
$2.2$ & $22$ &$220$ & $2.2$ & $22$ &$220$ & $2.2$ & $22$ &  \\
$2.7$ & $27$ &$270$ & $2.7$ & $27$ &$270$ & $2.7$ & $27$ &  \\
$3$ & $33$ &$330$ & $3.3$ & $33$ &$330$ & $3.3$ & $33$ &  \\
$4$ & $39$ &$390$ & $3.9$ & $39$ &$390$ & $3.9$ & $39$ &  \\
$5$ & $47$ &$470$ & $4.7$ & $47$ &$470$ & $4.7$ & $47$ &  \\
$6$ & $51$ &$510$ & $5.1$ & $51$ &$510$ & $5.1$ & $51$ &  \\
$7$ & $56$ &$560$ & $5.6$ & $56$ &$560$ & $5.6$ & $56$ &  \\
$8$ & $68$ &$680$ & $6.8$ & $68$ &$680$ & $6.8$ & $68$ &  \\
$9$ & $82$ &$820$ & $8.2$ & $82$ &$820$ & $8.2$ & $82$ &  
\end{tabular}
\label{جدول_امالہ_امالہ_عمومی_دستیاب_قیمتیں}
\end{table}

%===================
\ابتدا{مثال}\شناخت{مثال_امالہ_گنجائش_اور_قیمتیں_الف}
شکل \حوالہ{شکل_امالہ_گنجائش_اور_قیمتیں_الف}-الف میں \عددی{\SI{100}{\nano\farad}} برق گیر کا دباو دکھایا گیا ہے۔برقی گنجائش میں خلل \عددی{\SI{\mp 10}{\percent}} ممکن ہے۔کم سے کم اور زیادہ سے زیادہ گنجائش کی صورت میں رو کے خط حاصل کریں۔اس برقی گنجائش کو عموماً \عددی{\SI{100}{\nano\farad}\SI{\mp10}{\percent}} لکھا جاتا ہے۔

\begin{figure}
\centering
\begin{subfigure}{0.5\textwidth}
\centering
\begin{tikzpicture}
\draw[gray](0,-1)--(0,1.5)node[left]{$v(t)$};
\draw[gray](0,0)--(4.25,0)node[right]{$t(\si{\micro\second})$};
\draw(0,0)--(1,1)--(2,1)--(3,-0.5)--(4,0)--(4.25,0);
\draw(0,1)--++(-0.1,0)node[left]{$\SI{50}{\volt}$};
\draw(0,-0.5)--++(-0.1,0)node[left]{$\SI{-25}{\volt}$};
\draw(1,0)--++(0,-0.1)node[below]{$1$};
\draw(2,0)--++(0,-0.1)node[below]{$2$};
\draw(3,0)--++(0,-0.1);
\draw(4,0)--++(0,-0.1)node[below]{$4$};
\end{tikzpicture}
\caption*{(الف) برق گیر کا دباو۔}
\end{subfigure}%
\begin{subfigure}{0.5\textwidth}
\centering
\begin{tikzpicture}
\draw[gray](0,-1.75)--(0,1.5)node[left]{$i(t)$};
\draw[gray](0,0)--(4.25,0)node[right]{$t(\si{\micro\second})$};
\draw(0,0)--(0,1)--(1,1)--(1,0)--(2,0)--(2,-1.5)--(3,-1.5)--(3,0.5)--(4,0.5)--(4,0)--(4.25,0);
\draw(0,1)--++(-0.1,0)node[left]{$\SI{5.5}{\ampere}$};
\draw(0,0.5)--++(-0.1,0)node[left]{$\SI{2.75}{\ampere}$};
\draw(0,-1.5)--++(-0.1,0)node[left]{$\SI{-8.25}{\ampere}$};
\draw(1,0)node[below]{$1$};
\draw(2,0)node[below left]{$2$};
\draw(3,0)node[below right]{$3$};
\draw(4,0)node[below]{$4$};
\end{tikzpicture}
\caption*{(ب) \عددی{\SI{110}{\nano\farad}} کی رو۔}
\end{subfigure}
\begin{subfigure}{0.5\textwidth}
\centering
\begin{tikzpicture}
\draw[gray](0,-1.75)--(0,1.5)node[left]{$i(t)$};
\draw[gray](0,0)--(4.25,0)node[right]{$t(\si{\micro\second})$};
\draw(0,0)--(0,0.818)--(1,0.818)--(1,0)--(2,0)--(2,-1.23)--(3,-1.23)--(3,0.41)--(4,0.41)--(4,0)--(4.25,0);
\draw(0,0.818)--++(-0.1,0)node[left]{$\SI{4.5}{\ampere}$};
\draw(0,0.41)--++(-0.1,0)node[left]{$\SI{2.25}{\ampere}$};
\draw(0,-1.23)--++(-0.1,0)node[left]{$\SI{-6.75}{\ampere}$};
\draw(1,0)node[below]{$1$};
\draw(2,0)node[below left]{$2$};
\draw(3,0)node[below right]{$3$};
\draw(4,0)node[below]{$4$};
\end{tikzpicture}
\caption*{(پ) \عددی{\SI{90}{\nano\farad}} کی رو۔}
\end{subfigure}%
\caption{مثال \حوالہ{مثال_امالہ_گنجائش_اور_قیمتیں_الف} کا دور۔}
\label{شکل_امالہ_گنجائش_اور_قیمتیں_الف}
\end{figure}

حل:برق گیر کی زیادہ سے زیادہ قیمت دی گئی قیمت سے \عددی{\SI{10}{\percent}} زیادہ ہو سکتی ہے۔یوں اس کی زیادہ سے زیادہ گنجائش \عددی{\SI{110}{\nano\farad}} ممکن ہے۔اس قیمت کے گنجائش کی رو کو شکل \حوالہ{شکل_امالہ_گنجائش_اور_قیمتیں_الف}-ب میں دکھایا گیا ہے جہاں پہلے ایک مائیکرو سیکنڈ میں دباو کی تبدیلی کی شرح 
\begin{align*}
\frac{\dif v}{\dif t}=\frac{50-0}{\SI{1}{\micro\second}-\SI{0}{\micro\second}}=\SI{50}{\mega \volt \per\second}
\end{align*} 
ہونے کی بنا اس دورانیے کی رو
\begin{align*}
i=C \frac{\dif v}{\dif t}=110 \times 10^{-9} \times 50 \times 10^{6}=\SI{5.5}{\ampere}
\end{align*}
ہے۔اگلے ایک مائیکرو سیکنڈ میں دباو تبدیل نہیں ہوتا لہٰذا \عددی{\frac{\dif v}{\dif t}=0} ہے اور یوں رو بھی صفر کے برابر ہے۔دورانیہ \عددی{t=\SI{2}{\micro\second}} تا  \عددی{t=\SI{3}{\micro\second}} دباو کی شرح تبدیلی
\begin{align*}
\frac{\dif v}{\dif t}=\frac{-25-50}{\SI{3}{\micro\second}-\SI{2}{\micro\second}-\SI{0}{\micro\second}}=\SI{-75}{\mega \volt \per\second}
\end{align*}
ہے لہٰذا رو
\begin{align*}
i=C \frac{\dif v}{\dif t}=110 \times 10^{-9} \times \left(-75 \times 10^{6}\right)=\SI{-8.25}{\ampere}
\end{align*}
ہو گی۔دورانیہ \عددی{t=\SI{3}{\micro\second}} تا  \عددی{t=\SI{4}{\micro\second}} دباو کی شرح تبدیلی
\begin{align*}
\frac{\dif v}{\dif t}=\frac{0-(-25)}{\SI{4}{\micro\second}-\SI{3}{\micro\second}-\SI{0}{\micro\second}}=\SI{25}{\mega \volt \per\second}
\end{align*}
ہے لہٰذا رو
\begin{align*}
i=C \frac{\dif v}{\dif t}=110 \times 10^{-9} \times 25\times 10^{6}=\SI{2.75}{\ampere}
\end{align*}
ہو گی۔

خلل کی قیمت سے برق گیر کی کم سے کم ممکنہ گنجائش \عددی{\SI{90}{\nano\farad}} حاصل ہوتی ہے۔دباو کی تبدیلی کی شرح استعمال کرتے ہوئے رو درج ذیل حاصل ہوتی ہے۔
\begin{equation*}
i=
\begin{cases}
90 \times 10^{-9} \times 50 \times 10^{6}=\SI{4.5}{\ampere} & \SI{0}{\micro\second} \le t \le \SI{1}{\micro\second}\\
90 \times 10^{-9} \times 0=\SI{0}{\ampere} & \SI{1}{\micro\second} \le t \le \SI{2}{\micro\second}\\
90 \times 10^{-9} \times (-75) \times 10^{6}=\SI{-6.75}{\ampere} & \SI{2}{\micro\second}  \le t \le \SI{3}{\micro\second}\\
90 \times 10^{-9} \times 25 \times 10^{6}=\SI{2.25}{\ampere} & \SI{3}{\micro\second}  \le t \le \SI{4}{\micro\second}
\end{cases}
\end{equation*}
\انتہا{مثال}
%======================

\حصہ{سلسلہ وار جڑے برق گیر}
شکل \حوالہ{شکل_امالہ_متعدد_سلسلہ_وار_برق_گیر_مساوی_حصول} میں متعدد برق گیر سلسلہ وار جڑے دکھائے گئے ہیں۔تمام سلسلہ وار جڑے پرزوں میں رو کی قیمت یکساں ہوتی ہے۔کرخوف قانون دباو سے اس دور کے لئے درج ذیل لکھا جا سکتا ہے۔
\begin{align*}
v(t)=v_1(t)+v_2(t)+v_3(t)+\cdots +v_N(t)
\end{align*}
انفرادی برق گیر کے لئے
\begin{align*}
v_1(t)&=v_1(t_0)+\frac{1}{C_1}\int_{t_0}^{t} i(t) \dif t\\
v_2(t)&=v_2(t_0)+\frac{1}{C_2}\int_{t_0}^{t} i(t) \dif t\\
v_3(t)&=v_3(t_0)+\frac{1}{C_3}\int_{t_0}^{t} i(t) \dif t\\
& \vdots\\
v_N(t)&=v_N(t_0)+\frac{1}{C_N}\int_{t_0}^{t} i(t) \dif t
\end{align*}
لکھا جا سکتا ہے۔ مندرجہ بالا دو مساوات کو ملاتے ہوئے
\begin{multline*}
v(t)=v_1(t_0)+\frac{1}{C_1}\int_{t_0}^{t} i(t) \dif t +v_2(t_0)+\frac{1}{C_2}\int_{t_0}^{t} i(t) \dif t +\cdots\\
+v_N(t_0)+\frac{1}{C_N}\int_{t_0}^{t} i(t) \dif t
\end{multline*}
یعنی
\begin{multline*}
v(t)=v_1(t_0)+v_2(t_0)+\cdots+v_N(t_0)+\left(\frac{1}{C_1}+\frac{1}{C_2} +\cdots + \frac{1}{C_N}\right)\int_{t_0}^{t} i(t) \dif t
\end{multline*}
لکھا جا سکتا ہے۔اس مساوات میں
\begin{align}\label{مساوات_امالہ_سلسلہ_وار_برق_گیر_کا_مساوی}
\frac{1}{C_s}=\sum_{i=1}^{N} \frac{1}{C_i}=\frac{1}{C_1}+\frac{1}{C_2}+\frac{1}{C_3}+\cdots+\frac{1}{C_N}
\end{align}
اور
\begin{align}\label{مساوات_امالہ_سلسلہ_وار_برق_گیر_کا_مساوی_ابتدائی_دباو}
v(t_0)=v_1(t_0)+v_2(t_0)+v_3(t_0)+\cdots+v_N(t_0)
\end{align}
لکھتے ہوئے
\begin{align}
v(t)=v(t_0)+\frac{1}{C_s} \int_{t_0}^{t} i(t)\dif t
\end{align}
حاصل ہوتا ہے جو ایک عدد برقی گیر کی مساوات ہے جسے شکل-ب میں دکھایا گیا ہے۔مساوات \حوالہ{مساوات_امالہ_سلسلہ_وار_برق_گیر_کا_مساوی} متعدد سلسلہ وار جڑے برق گیروں کی مساوی برق گنجائش \عددی{C_s} دیتی ہے جبکہ مساوات \حوالہ{مساوات_امالہ_سلسلہ_وار_برق_گیر_کا_مساوی_ابتدائی_دباو} ان کا مساوی ابتدائی دباو دیتی ہے۔آپ دیکھ سکتے ہیں کہ سلسلہ وار جڑے برق گیروں کی مساوات متوازی جڑے مزاحمتوں کی مساوات کی طرح ہے۔

\begin{figure}
\centering
\begin{subfigure}{1\textwidth}
\centering
\begin{tikzpicture}
\draw(0,0) to [short,i={$i(t)$},o-]++(\x/2,0) to [capacitor,l={$C_1$}]++(\x,0) to [capacitor,l={$C_2$}]++(\x,0) to [capacitor,l={$C_3$}]++(\x,0)coordinate(kup);
\draw[dashed](kup)--++(\x/4,0)--++(0,-\y)--++(-2*\x-\x/2,0)coordinate(klow);
\draw(klow) to [capacitor,l_={$C_N$}]++(-\x,0) to [short,-o]++(-\x/4,0);
\draw(0,-\y/2)node{$\begin{aligned} &+ \\ &v(t)\\ &- \end{aligned}$};
\draw(\x/2+\x/2,-2*\dy)node[below]{$+ \, v_1(t) \, -$};
\draw(\x/2+\x/2+\x,-2*\dy)node[below]{$+ \, v_2(t) \, -$};
\draw(\x/2+\x/2+2*\x,-2*\dy)node[below]{$+ \, v_3(t) \, -$};
\draw(\x/4+\x/2,-\y-2*\dy)node[below]{$- \, v_N(t) \, +$};
\end{tikzpicture}
\caption*{(الف) متعدد سلسلہ وار جڑے برق گیر۔}
\end{subfigure}
\begin{subfigure}{1\textwidth}
\centering
\begin{tikzpicture}
\draw(0,0) to [short,i={$i(t)$},o-]++(\x,0) to [capacitor,l={$C_s$}]++(0,-\y) to [short,-o]++(-\x,0);
\draw(0,-\y/2)node{$\begin{aligned} &+ \\ &v(t)\\ &- \end{aligned}$};
\end{tikzpicture}
\caption*{(ب) متعدد سلسلہ وار جڑے برقی گیروں کا مساوی برق گیر۔}
\end{subfigure}
\caption{متعدد سلسلہ وار جڑے برق گیر کی مساوی برق گنجائش کا حصول۔}
\label{شکل_امالہ_متعدد_سلسلہ_وار_برق_گیر_مساوی_حصول}
\end{figure}

%================
\ابتدا{مثال}\شناخت{مثال_امالہ_سلسلہ_وار_برق_گیر_الف}
شکل \حوالہ{شکل_امالہ_مثال_سلسلہ_وار_برق_گیر_الف}-الف میں مساوی سلسلہ وار گنجائش اور ان کے انفرادی ابتدائی دباو دکھائے گئے ہیں۔ ان کا مساوی گنجائش اور مساوی ابتدائی دباو حاصل کریں۔

\begin{figure}
\centering
\begin{subfigure}{0.5\textwidth}
\centering
\begin{tikzpicture}
\draw(0,0) to  [capacitor,o-,l_={$+\, \SI{5}{\volt} \, -$}]++(\x+\x,0) to [capacitor,l_={$\SI{10}{\micro\farad}$}]++(0,-\y) to [capacitor,-o,l={$+ \, \SI{3}{\volt} \, -$}]++(-\x-\x,0);
\draw(\x+\x+2*\dx,-\y/2)node[right]{$\begin{aligned} &+\\ &\SI{6}{\volt} \\ &- \end{aligned}$};
\draw(0,-\y/2)node{$\begin{aligned}&+ \\ &v(t) \\ &-  \end{aligned}$};
\draw(\x,2*\dy)node[above]{$\SI{8}{\micro\farad}$};
\draw(3/4*\x-\dx,-\y)node[above]{$\SI{6}{\micro\farad}$};
\end{tikzpicture}
\caption*{(الف) سلسلہ وار جڑے برق گیر بمع ابتدائی دباو۔}
\end{subfigure}%
\begin{subfigure}{0.5\textwidth}
\centering
\begin{tikzpicture}
\draw(0,0) to  [short,o-] ++(\x+\x/2,0)  to [capacitor]++(0,-\y) to [short,-o]++(-\x-\x/2,0);
\draw(0,-\y/2)node{$\begin{aligned} &+ \\ &v(t) \\ &- \end{aligned}$};
\draw(1.5*\x-2*\dx,-\y/2)node[left]{$\frac{120}{47}\,\si{\micro\farad}$};
\draw(1.5*\x+2*\dx,-\y/2)node[right]{$\begin{aligned} &+ \\ &\SI{8}{\volt} \\ &- \end{aligned}$};
\end{tikzpicture}
\caption*{(ب) مساوی برقی گیر بمع ابتدائی دباو۔}
\end{subfigure}
\caption{مثال \حوالہ{مثال_امالہ_سلسلہ_وار_برق_گیر_الف} کا دور۔}
\label{شکل_امالہ_مثال_سلسلہ_وار_برق_گیر_الف}
\end{figure}

حل:مساوات \حوالہ{مساوات_امالہ_سلسلہ_وار_برق_گیر_کا_مساوی} سے
\begin{align*}
\frac{1}{C_s}=\frac{1}{\SI{8}{\micro\farad}}+\frac{1}{\SI{10}{\micro\farad}}+\frac{1}{\SI{6}{\micro\farad}}=\frac{47}{120}\,\si{\micro\farad}
\end{align*}
لکھتے ہوئے
\begin{align*}
C_s=\frac{120}{47} \, \si{\micro\farad}
\end{align*}
حاصل ہوتا ہے۔مساوات \حوالہ{مساوات_امالہ_سلسلہ_وار_برق_گیر_کا_مساوی_ابتدائی_دباو} سے ابتدائی دباو درج ذیل حاصل ہوتی ہے۔
\begin{align*}
v(t_0)=5+6-3=\SI{8}{\volt}
\end{align*}
شکل \حوالہ{شکل_امالہ_مثال_سلسلہ_وار_برق_گیر_الف}-ب میں مساوی برقی گنجائش اور ابتدائی دباو دکھائے گئے ہیں۔
\انتہا{مثال}
%============================
\ابتدا{مثال}
ابتدائی طور پر بے بار، دو عدد برق گیر کو سلسلہ وار جوڑنے کے بعد ان میں \عددی{\SI{50}{\volt}} منبع سے برقی بار بھرا جاتا ہے۔ان میں ایک برق گیر \عددی{\SI{20}{\micro\farad}} گنجائش کا ہے جبکہ دوسرے برق گیر کی گنجائش کے بارے میں ہمیں معلوم نہیں ہے۔نا معلوم برق گیر پر \عددی{\SI{10}{\volt}} جبکہ \عددی{\SI{20}{\micro\farad}} برق گیر پر \عددی{\SI{40}{\volt}} دباو پایا جاتا ہے۔نا معلوم  گنجائش دریافت کریں۔

حل:\عددی{\SI{20}{\micro\farad}} پر بار درج ذیل ہے۔
\begin{align*}
q=C v=\left(\SI{20}{\micro\farad}\right)(\SI{40}{\volt})=\SI{800}{\micro\coulomb}
\end{align*} 
سلسلہ وار جڑے پرزوں میں یکساں رو پائی جاتی ہے لہٰذا دونوں برق گیر پر یکساں بار پایا جاتا ہے۔یوں نا معلوم گنجائش درج ذیل حاصل ہوتی ہے۔
\begin{align*}
C=\frac{q}{v}=\frac{\SI{800}{\micro\farad}}{\SI{10}{\volt}}=\SI{80}{\micro\farad}
\end{align*}
\انتہا{مثال}
%==========================

\حصہ{متوازی جڑے برق گیر}
متوازی جڑے برق گیروں کی مساوی گنجائش شکل \حوالہ{شکل_امالہ_متوازی_برق_گیر_مساوی_حصول}-الف سے کرخوف قانون رو کی مدد سے حاصل کرتے ہیں۔
\begin{align*}
i(t)&=i_1(t)+i_2(t)+i_3(t)+\cdots +i_N(t)\\
&=C_1 \frac{\dif v(t)}{\dif t}+C_2 \frac{\dif v(t)}{\dif t}+C_3 \frac{\dif v(t)}{\dif t}+\cdots+C_N \frac{\dif v(t)}{\dif t}\\
&=\left(C_1+C_2+C_3+\cdots+C_N\right) \frac{\dif v(t)}{\dif t}
\end{align*}
اس مساوات میں 
\begin{align}\label{مساوات_امالہ_متوازی_برق_گیر_کا_مساوی}
C_m=\sum_{i=1}^{N} C_i=C_1+C_2+C_3+\cdots +C_N
\end{align}
لکھتے ہوئے
\begin{align}
i(t)=C_m \frac{\dif v(t)}{\dif t}
\end{align}
حاصل ہوتا ہے جو ایک عدد برق گیر کی مساوات ہے۔مساوات \حوالہ{مساوات_امالہ_متوازی_برق_گیر_کا_مساوی} متعدد متوازی جڑے برق گیروں کی مساوی گنجائش دیتی ہے جو سلسلہ وار جڑے مزاحمتوں کی مساوت کی طرح ہے۔شکل \حوالہ{شکل_امالہ_متوازی_برق_گیر_مساوی_حصول}-ب میں مساوی برق گیر دکھایا گیا ہے۔

\begin{figure}
\centering
\begin{subfigure}{1\textwidth}
\centering
\begin{tikzpicture}
\draw(0,0)  to [short,o-]++(3*\x+\x/2,0);
\draw(0,\y)to [short,i={$i(t)$},o-]++(\x/2,0)  to [short]++(3*\x,0);
\draw[dashed](3*\x+\x/2,0)--++(\x/2,0);
\draw[dashed](3*\x+\x/2,\y)--++(\x/2,0);
\draw(\x,\y) to [capacitor,*-*,i>_={$i_1(t)$},l={$C_1$}]++(0,-\y);
\draw(2*\x,\y) to [capacitor,*-*,i>_={$i_2(t)$},l={$C_2$}]++(0,-\y);
\draw(3*\x,\y) to [capacitor,*-*,i>_={$i_3(t)$},l={$C_3$}]++(0,-\y);
\draw(4*\x,\y) to [short] ++(\x/4,0) to [capacitor,i>_={$i_N(t)$},l={$C_N$}]++(0,-\y) to [short]++(-\x/4,0);
\draw(0,\y/2)node{$\begin{aligned}&+ \\ &v(t) \\ &- \end{aligned}$};
\end{tikzpicture}
\caption*{(الف)}
\end{subfigure}
\begin{subfigure}{1\textwidth}
\centering
\begin{tikzpicture}
\draw(0,0) to [short,i={$i(t)$},o-]++(\x,0) to [capacitor,l={$C_m$}]++(0,-\y) to [short,-o] ++(-\x,0);
\draw(0,-\y/2)node{$\begin{aligned}&+ \\ &v(t) \\ &- \end{aligned}$};
\end{tikzpicture}
\caption*{(ب)}
\end{subfigure}
\caption{متوازی جڑے برق گیروں کی مساوی گنجائش۔}
\label{شکل_امالہ_متوازی_برق_گیر_مساوی_حصول}
\end{figure}

%===============
\ابتدا{مثال}\شناخت{مثال_امالہ-متوازی_برق_گیر_مساوی_الف}
شکل \حوالہ{شکل_امالہ_متوازی_برق_گیر_مساوی_الف}-الف میں چار عدد برق گیر متوازی جوڑے گئے ہیں۔ان کی مساوی گنجائش دریافت کریں۔

\begin{figure}
\centering
\begin{subfigure}{1\textwidth}
\centering
\begin{tikzpicture}
\draw(0,0)  to [short,o-]++(4*\x,0);
\draw(0,\y)to [short,i={$i(t)$},o-]++(\x,0)  to [short]++(3*\x,0);
\draw(\x,\y) to [capacitor,*-*,l={$\SI{10}{\micro\farad}$}]++(0,-\y);
\draw(2*\x,\y) to [capacitor,*-*,l={$\SI{2}{\micro\farad}$}]++(0,-\y);
\draw(3*\x,\y) to [capacitor,*-*,l={$\SI{7}{\micro\farad}$}]++(0,-\y);
\draw(4*\x,\y)  to [capacitor,l={$\SI{3}{\micro\farad}$}]++(0,-\y);
\draw(0,\y/2)node{$\begin{aligned}&+ \\ &v(t) \\ &- \end{aligned}$};
\end{tikzpicture}
\caption*{(الف)}
\end{subfigure}
\begin{subfigure}{1\textwidth}
\centering
\begin{tikzpicture}
\draw(0,0) to [short,i={$i(t)$},o-]++(\x,0) to [capacitor,l={$\SI{22}{\micro\farad}$}]++(0,-\y) to [short,-o] ++(-\x,0);
\draw(0,-\y/2)node{$\begin{aligned}&+ \\ &v(t) \\ &- \end{aligned}$};
\end{tikzpicture}
\caption*{(ب)}
\end{subfigure}
\caption{مثال \حوالہ{مثال_امالہ-متوازی_برق_گیر_مساوی_الف} کا دور۔}
\label{شکل_امالہ_متوازی_برق_گیر_مساوی_الف}
\end{figure}

حل: مساوات \حوالہ{مساوات_امالہ_متوازی_برق_گیر_کا_مساوی} سے متوازی جڑے برق گیروں کی مساوی برقی گنجائش حاصل کرتے ہیں۔
\begin{align*}
C_m=\SI{10}{\micro\farad}+\SI{2}{\micro\farad}+\SI{7}{\micro\farad}+\SI{3}{\micro\farad}=\SI{22}{\micro\farad}
\end{align*}
شکل \حوالہ{شکل_امالہ_متوازی_برق_گیر_مساوی_الف}-ب میں مساوی گنجائش دکھائی گئی ہے۔
\انتہا{مثال}
%===================
\ابتدا{مشق}\شناخت{مشق_امالہ_سلسلہ_وار_برق_گیر_الف}
ابتدائی طور پر بے بار، دو عدد برق گیر سلسلہ وار جوڑے جاتے ہیں۔لمحہ \عددی{t} پر صورت حال شکل \حوالہ{شکل_امالہ_مشق_سلسلہ_وار_برق_گیر_الف} میں دکھائی گئی ہے۔نا معلوم گنجائش دریافت کریں۔

\begin{figure}
\centering
\begin{tikzpicture}
\draw(0,0) to [short,o-]++(\x+\x/2,0) to [capacitor,l_={$\SI{14}{\micro\farad}$}]++(0,\y) to [capacitor,-o, l_={$C_1$}]++(-\x-\x/2,0);
\draw(0,\y/2)node{$\begin{aligned} &+ \\& \SI{15}{\volt} \\ &- \end{aligned}$};
\draw(\x+\x/2-2*\dx,\y/2)node[left]{$\begin{aligned}  &+ \\ &\SI{5}{\volt} \\ &-\end{aligned}$};
\end{tikzpicture}
\caption{مشق \حوالہ{مشق_امالہ_سلسلہ_وار_برق_گیر_الف} کا دور۔}
\label{شکل_امالہ_مشق_سلسلہ_وار_برق_گیر_الف}
\end{figure} 

جواب:\عددی{\SI{7}{\micro\farad}}
\انتہا{مشق}
%==================
\ابتدا{مشق}\شناخت{مشق_امالہ_سلسلہ_وار_برق_گیر_ب}
شکل \حوالہ{شکل_امالہ_مشق_سلسلہ_وار_برق_گیر_ب} میں مساوی گنجائش دریافت کریں۔

\begin{figure}
\centering
\begin{tikzpicture}
\draw(0,0) to [capacitor,o-,l={$\SI{3}{\micro\farad}$}]++(\x,0) to [short]++(3*\x,0) to [capacitor,l={$\SI{4.5}{\micro\farad}$}]++(0,2*\y) to [short]++(-3*\x,0) to [capacitor,-o,l={$\SI{2}{\micro\farad}$}]++(-\x,0);
\draw(\x,0) to [capacitor,*-,l_={$\SI{2}{\micro\farad}$}]++(0,\y);
\draw(2*\x,0) to [capacitor,*-*,l_={$\SI{4}{\micro\farad}$}]++(0,\y);
\draw(3*\x,0) to [capacitor,*-,l_={$\SI{3}{\micro\farad}$}]++(0,\y);
\draw(\x,\y) to [short]++(2*\x,0);
\draw(2*\x,\y) to [capacitor,-*,l={$\SI{9}{\micro\farad}$}]++(0,\y);
\draw[stealth-](0,\y)--++(-\x/2,0)--++(0,-\y/4)node[below]{$C_{\text{مساوی}}$};
\end{tikzpicture}
\caption{مشق \حوالہ{مشق_امالہ_سلسلہ_وار_برق_گیر_ب} کا دور۔}
\label{شکل_امالہ_مشق_سلسلہ_وار_برق_گیر_ب}
\end{figure} 

جواب:\عددی{\frac{18}{17} \, \si{\micro\farad}}
\انتہا{مشق}
%==================
\ابتدا{مشق}\شناخت{مشق_امالہ_سلسلہ_وار_برق_گیر_پ}
شکل \حوالہ{شکل_امالہ_مشق_سلسلہ_وار_برق_گیر_پ} میں کل گنجائش حاصل کریں۔

\begin{figure}
\centering
\begin{tikzpicture}
\draw(0,0) to [capacitor,o-,l_={$\SI{10}{\micro\farad}$}]++(\xx,0) to [capacitor,l_={$\SI{4}{\micro\farad}$}]++(\xx,0) to [capacitor,l_={$\SI{3}{\micro\farad}$}]++(\xx,0);
\draw(0,\yy) to [capacitor,o-,l={$\SI{5}{\micro\farad}$}]++(\xx,0) to [capacitor,l={$\SI{12}{\micro\farad}$}]++(\xx,0) to [capacitor,l={$\SI{12}{\micro\farad}$}]++(\xx,0);
\draw(\xx,0) to [capacitor,*-*,l={$\SI{7}{\micro\farad}$}]++(0,\yy);
\draw(2*\xx,0) to [capacitor,*-*]++(0,\yy);
\draw(2*\xx,\yy/3)node[left]{$\SI{2}{\micro\farad}$};
\draw(3*\xx,0) to [capacitor,*-,l_={$\SI{4}{\micro\farad}$}]++(0,\yy);
\draw(\xx,0) to [capacitor,l={$\SI{2}{\micro\farad}$}]++(\xx,\yy) to [capacitor,l={$\SI{3}{\micro\farad}$}]++(\xx,-\yy);
\draw[stealth-](0,\yy/2)--++(-\xx/4,0)--++(0,-\yy/8)node[below]{$C_{\text{کل}}$};
\end{tikzpicture}
\caption{مشق \حوالہ{مشق_امالہ_سلسلہ_وار_برق_گیر_پ} کا دور۔}
\label{شکل_امالہ_مشق_سلسلہ_وار_برق_گیر_پ}
\end{figure} 

جواب:\عددی{\frac{5}{2} \, \si{\micro\farad}}
\انتہا{مشق}
%==================

\حصہ{سلسلہ وار امالہ گیر}
متعدد سلسلہ وار جڑے امالہ گیر کو شکل \حوالہ{شکل_امالہ_متعدد_سلسلہ_وار_امالہ_گیر_مساوی_حصول}-الف میں دکھایا گیا ہے۔کرخوف قانون دباو سے
\begin{align*}
v(t)&=v_1(t)+v_2(t)+v_3(t)+\cdots+v_N(t)\\
&=L_1 \frac{\dif i(t)}{\dif t}+L_2 \frac{\dif i(t)}{\dif t}+L_3 \frac{\dif i(t)}{\dif t}+\cdots+L_N \frac{\dif i(t)}{\dif t}\\
&=\left(L_1+L_2+L_3+\cdots+L_N\right)\frac{\dif i(t)}{\dif t}
\end{align*}
لکھ کر اس میں
\begin{align}\label{مساوات_امالہ_سلسلہ_وار_امالہ_گیر_الف}
L_s=\sum_{i=1}^{N} L_i=L_1+L_2+L_3+\cdots+L_N
\end{align}
پُر کرنے سے
\begin{align*}
v(t)=L_s \frac{\dif i(t)}{\dif t}
\end{align*}
حاصل ہوتا ہے جو ایک عدد امالہ گیر کی مساوات ہے جسے شکل \حوالہ{شکل_امالہ_متعدد_سلسلہ_وار_امالہ_گیر_مساوی_حصول}-ب میں دکھایا گیا ہے۔
مساوات \حوالہ{مساوات_امالہ_سلسلہ_وار_امالہ_گیر_الف} سلسلہ وار امالہ کی مساوی امالہ دیتی ہے۔یہ سلسلہ وار مزاحمتوں کی مساوات کی طرح مساوات ہے۔


\begin{figure}
\centering
\begin{subfigure}{1\textwidth}
\centering
\begin{tikzpicture}
\draw(0,0) to [short,i={$i(t)$},o-]++(\x/2,0) to [inductor,l={$L_1$}]++(\x,0) to [inductor,l={$L_2$}]++(\x,0) to [inductor,l={$L_3$}]++(\x,0)coordinate(kup);
\draw[dashed](kup)--++(\x/4,0)--++(0,-\y)--++(-2*\x-\x/2,0)coordinate(klow);
\draw(klow) to [inductor,l_={$L_N$}]++(-\x,0) to [short,-o]++(-\x/4,0);
\draw(0,-\y/2)node{$\begin{aligned} &+ \\ &v(t)\\ &- \end{aligned}$};
\draw(\x/2+\x/2,-\dy)node[below]{$+ \, v_1(t) \, -$};
\draw(\x/2+\x/2+\x,-\dy)node[below]{$+ \, v_2(t) \, -$};
\draw(\x/2+\x/2+2*\x,-\dy)node[below]{$+ \, v_3(t) \, -$};
\draw(\x/4+\x/2,-\y-\dy)node[below]{$- \, v_N(t) \, +$};
\end{tikzpicture}
\caption*{(الف) متعدد سلسلہ وار جڑے امالہ گیر۔}
\end{subfigure}
\begin{subfigure}{1\textwidth}
\centering
\begin{tikzpicture}
\draw(0,0) to [short,i={$i(t)$},o-]++(\x,0) to [inductor,l={$L_s$}]++(0,-\y) to [short,-o]++(-\x,0);
\draw(0,-\y/2)node{$\begin{aligned} &+ \\ &v(t)\\ &- \end{aligned}$};
\end{tikzpicture}
\caption*{(ب) متعدد سلسلہ وار جڑے امالہ گیروں کی مساوی امالہ۔}
\end{subfigure}
\caption{متعدد سلسلہ وار جڑے امالہ گیر کی مساوی امالہ کا حصول۔}
\label{شکل_امالہ_متعدد_سلسلہ_وار_امالہ_گیر_مساوی_حصول}
\end{figure}

%=====================
\ابتدا{مثال}\شناخت{مثال_امالہ_سلسلہ_وار_امالہ_الف}
شکل \حوالہ{شکل_امالہ_سلسلہ_وار_امالہ_الف} میں مساوی امالہ دریافت کریں۔

\begin{figure}
\centering
\begin{tikzpicture}
\draw(0,0) to [inductor,o-,l={$\SI{12}{\milli\henry}$}]++(\x,0) to [inductor,l={$\SI{3}{\milli\henry}$}]++(\x,0) to [inductor,l={$\SI{7}{\milli\henry}$}]++(\x,0) to [inductor,l={$\SI{5}{\milli\henry}$}]++(0,-\y) to [short,-o]++(-3*\x,0);
\draw[stealth-](0,-\y/2)--++(-\x/4,0)--++(0,-\y/8)node[below]{$L_{\text{مساوی}}$};
\end{tikzpicture}
\caption{مثال \حوالہ{مثال_امالہ_سلسلہ_وار_امالہ_الف} کا دور۔}
\label{شکل_امالہ_سلسلہ_وار_امالہ_الف}
\end{figure}

جواب:\عددی{\SI{27}{\milli\henry}}
\انتہا{مثال}
%====================

\حصہ{متوازی امالہ گیر}
متوازی جڑے امالہ گیروں کی مساوی امالہ شکل \حوالہ{شکل_امالہ_متوازی_امالہ_گیر_مساوی_حصول}-الف کی مدد سے حاصل کرتے ہیں جسے دیکھتے ہوئے کرخوف مساوات رو
\begin{align}\label{مساوات_امالہ_متوازی_امالہ_کرخوف_مساوات_الف}
i(t)&=i_1(t)+i-2(t)+i_3(t)+\cdots+i_N(t)
\end{align}
لکھی جا سکتی ہے۔انفرادی امالہ گیر کے لئے درج ذیل مساوات لکھے جا سکتے ہیں
\begin{align*}
i_1(t)&=i_1(t_0)+\frac{1}{L_1}\int_{t_0}^{t} v(t) \dif t\\
i_2(t)&=i_2(t_0)+\frac{1}{L_2}\int_{t_0}^{t} v(t) \dif t\\
i_3(t)&=i_3(t_0)+\frac{1}{L_3}\int_{t_0}^{t} v(t) \dif t\\
&\vdots\\
i_N(t)&=i_N(t_0)+\frac{1}{L_N}\int_{t_0}^{t} v(t) \dif t
\end{align*}
جنہیں مساوات \حوالہ{مساوات_امالہ_متوازی_امالہ_کرخوف_مساوات_الف} میں پُر کرتے ہوئے
\begin{multline*}
i(t)=i_1(t_0)+\frac{1}{L_1}\int_{t_0}^{t} v(t) \dif t+i_2(t_0)+\frac{1}{L_2}\int_{t_0}^{t} v(t) \dif t+\cdots \\
+i_N(t_0)+\frac{1}{L_N}\int_{t_0}^{t} v(t) \dif t
\end{multline*}
حاصل ہوتا ہے۔اس مساوات کو ترتیب دیتے ہوئے
\begin{align*}
i(t)&=i_1(t_0)+i_2(t_0)+\cdots+i_N(t_0)+\left(\frac{1}{L_1}+\frac{1}{L_2}+\cdots+\frac{1}{L_N}\right)\int_{t_0}^{t} v(t)\dif t
\end{align*}
لکھا جا سکتا ہے جس میں 
\begin{align}\label{مساوات_امالہ_متوازی_مساوی_مساوات_قیمت}
\frac{1}{L_m}=\sum_{i=1}^{N}\frac{1}{L_i}=\frac{1}{L_1}+\frac{1}{L_2}+\frac{1}{L_3}+\cdots+\frac{1}{L_N}
\end{align}
اور
\begin{align}\label{مساوات_امالہ_متوازی_امالہ_ابتدائی_رو}
i(t_0)=i_1(t_0)+i_2(t_0)+i_3(t_0)+\cdots+i_N(t_0)
\end{align}
پُر کرنے سے
\begin{align*}
i(t)=i(t_0)+\frac{1}{L_m}\int_{t_0}^{t} v(t) \dif t
\end{align*}
حاصل ہوتا ہے جو ایک عدد امالہ گیر کی مساوات ہے جسے شکل \حوالہ{شکل_امالہ_متوازی_امالہ_گیر_مساوی_حصول}-ب میں دکھایا گیا ہے۔مساوات \حوالہ{مساوات_امالہ_متوازی_مساوی_مساوات_قیمت} متوازی جڑے امالہ گیر کی مساوی امالہ \عددی{L_m} دیتی ہے جبکہ مساوات \حوالہ{مساوات_امالہ_متوازی_امالہ_ابتدائی_رو} مساوی امالہ میں ابتدائی رو \عددی{i(t_0)} دیتی ہے۔
\begin{figure}
\centering
\begin{subfigure}{1\textwidth}
\centering
\begin{tikzpicture}
\draw(0,0)  to [short,o-]++(3*\x+\x/2,0);
\draw(0,\y)to [short,i={$i(t)$},o-]++(\x/2,0)  to [short]++(3*\x,0);
\draw[dashed](3*\x+\x/2,0)--++(\x/2,0);
\draw[dashed](3*\x+\x/2,\y)--++(\x/2,0);
\draw(\x,\y) to [inductor,*-*,i>_={$i_1(t)$},l={$L_1$}]++(0,-\y);
\draw(2*\x,\y) to [inductor,*-*,i>_={$i_2(t)$},l={$L_2$}]++(0,-\y);
\draw(3*\x,\y) to [inductor,*-*,i>_={$i_3(t)$},l={$L_3$}]++(0,-\y);
\draw(4*\x,\y) to [short] ++(\x/4,0) to [inductor,i>_={$i_N(t)$},l={$L_N$}]++(0,-\y) to [short]++(-\x/4,0);
\draw(0,\y/2)node{$\begin{aligned}&+ \\ &v(t) \\ &- \end{aligned}$};
\end{tikzpicture}
\caption*{(الف)}
\end{subfigure}
\begin{subfigure}{1\textwidth}
\centering
\begin{tikzpicture}
\draw(0,0) to [short,i={$i(t)$},o-]++(\x,0) to [inductor,l={$L_m$}]++(0,-\y) to [short,-o] ++(-\x,0);
\draw(0,-\y/2)node{$\begin{aligned}&+ \\ &v(t) \\ &- \end{aligned}$};
\end{tikzpicture}
\caption*{(ب)}
\end{subfigure}
\caption{متوازی جڑے امالہ گیروں کی مساوی امالہ۔}
\label{شکل_امالہ_متوازی_امالہ_گیر_مساوی_حصول}
\end{figure}
%======================================
\ابتدا{مثال}\شناخت{مثال_امالہ_متوازی_امالہ_مثال_الف}
شکل \حوالہ{شکل_امالہ_متوازی_امالہ_مثال_الف}-الف میں متوازی امالہ گیر اور ان میں ابتدائی رو دی گئی ہیں۔مساوی امالہ اور اس کی ابتدائی رو دریافت کریں۔
\begin{figure}
\centering
\begin{subfigure}{1\textwidth}
\centering
\begin{tikzpicture}
\draw(0,0)  to [short,o-]++(3*\x,0);
\draw(0,\y)to [short,i={$i(t)$},o-]++(\x/2,0)  to [short]++(2*\x+\x/2,0);
\draw(\x,\y) to [inductor,*-*,i>_={$\SI{2}{\milli\ampere}$},l={$\SI{100}{\micro\henry}$}]++(0,-\y);
\draw(2*\x,\y) to [inductor,*-*,i<_={$\SI{8}{\milli\ampere}$},l={$\SI{300}{\micro\henry}$}]++(0,-\y);
\draw(3*\x,\y) to [inductor,i>_={$\SI{4}{\milli\ampere}$},l={$\SI{450}{\micro\henry}$}]++(0,-\y);
\draw(0,\y/2)node{$\begin{aligned}&+ \\ &v(t) \\ &- \end{aligned}$};
\end{tikzpicture}
\caption*{(الف)}
\end{subfigure}
\begin{subfigure}{1\textwidth}
\centering
\begin{tikzpicture}
\draw(0,0) to [short,i={$i(t)$},o-]++(\x,0) to [inductor,i<^={$\SI{2}{\milli\ampere}$}]++(0,-\y) to [short,-o] ++(-\x,0);
\draw(\x+\dx,-\y/2)node[right]{$\frac{450}{7} \, \si{\micro\henry}$};
\draw(0,-\y/2)node{$\begin{aligned}&+ \\ &v(t) \\ &- \end{aligned}$};
\end{tikzpicture}
\caption*{(ب)}
\end{subfigure}
\caption{مثال \حوالہ{مثال_امالہ_متوازی_امالہ_مثال_الف} کا دور۔}
\label{شکل_امالہ_متوازی_امالہ_مثال_الف}
\end{figure}

حل: مساوات \حوالہ{مساوات_امالہ_متوازی_مساوی_مساوات_قیمت} سے 
\begin{align*}
\frac{1}{L_m}=\frac{1}{\SI{100}{\micro\henry}}+\frac{1}{\SI{300}{\micro\henry}}+\frac{1}{\SI{450}{\micro\henry}}
\end{align*}
لکھ کر
\begin{align*}
L_m=\frac{450}{7}\,\si{\micro\henry}
\end{align*}
حاصل ہوتی ہے۔مساوات \حوالہ{مساوات_امالہ_متوازی_امالہ_ابتدائی_رو} سے ابتدائی رو درج ذیل حاصل ہوتی ہے۔
\begin{align*}
i(t_0)=\SI{2}{\milli\ampere}-\SI{8}{\milli\ampere}+\SI{4}{\milli\ampere}=\SI{-2}{\milli\ampere}
\end{align*}
شکل \حوالہ{شکل_امالہ_متوازی_امالہ_مثال_الف}-ب میں مساوی امالہ بمع ابتدائی رو دکھائی گئی ہے۔منفی رو \عددی{i(t)} کے الٹ ہے۔
\انتہا{مثال}
%=============================
\ابتدا{مشق}\شناخت{مثال_امالہ_متوازی_امالہ_مثال_ب}
شکل \حوالہ{شکل_امالہ_متوازی_امالہ_مثال_ب} میں تمام انفرادی امالہ \عددی{\SI{12}{\milli\henry}} ہیں۔ان کی مساوی امالہ دریافت کریں۔

\begin{figure}
\centering
\begin{tikzpicture}
\draw(0,0) to [inductor,o-]++(\x,0) to [short]++(\x,0) to [inductor]++(0,\y) to [short]++(0,\y) to [short,-o]++(-2*\x,0);
\draw(\x,0) to [inductor,*-*]++(0,\y);
\draw(2*\x,\y) to [inductor,*-]++(-\x,0) to [short]++(-\x/2,0) to [inductor,-*]++(0,\y);
\draw[stealth-](0,\y)--++(-\x/4,0)--++(0,-\y/8)node[below]{$L_{\text{مساوی}}$};
\end{tikzpicture}
\caption{مشق \حوالہ{مثال_امالہ_متوازی_امالہ_مثال_ب} کا دور۔}
\label{شکل_امالہ_متوازی_امالہ_مثال_ب}
\end{figure}

جواب:\عددی{\frac{96}{5} \, \si{\milli\henry}}
\انتہا{مشق}
%==============================

\ابتدا{مشق}\شناخت{مثال_امالہ_متوازی_امالہ_مثال_پ}
شکل \حوالہ{شکل_امالہ_متوازی_امالہ_مثال_پ} میں کل امالہ دریافت کریں۔

\begin{figure}
\centering
\begin{tikzpicture}
\draw(0,0) to [short]++(\x,0) to [inductor,l_={$\SI{5}{\milli\henry}$}]++(\x,0) to [inductor,l_={$\SI{1}{\milli\henry}$}]++(\x,0) to [inductor,l_={$\SI{2}{\milli\henry}$}]++(\x,0);
\draw(0,\y) to [short]++(\x,0) to [inductor,l_={$\SI{3}{\milli\henry}$}]++(\x,0) to [inductor,l_={$\SI{4}{\milli\henry}$}]++(\x,0) to [inductor,l_={$\SI{5}{\milli\henry}$}]++(\x,0);
\draw(0,0) to [inductor,l={$\SI{6}{\milli\henry}$}]++(0,\y);
\draw(\x,0) to [inductor,*-*,l={$\SI{3}{\milli\henry}$}]++(0,\y);
\draw(3*\x,0) to [inductor,*-*,l_={$\SI{10}{\milli\henry}$}]++(0,\y);
\draw(4*\x,0) to [inductor,l_={$\SI{3}{\milli\henry}$}]++(0,\y);
\draw(2*\x,0) to [short,*-o]++(0,\y/4);
\draw(2*\x,\y) to [short,*-o]++(0,-\y/4);
\draw(2*\x,\y/2)node{$L_{\text{کل}}$};
\end{tikzpicture}
\caption{مشق \حوالہ{مثال_امالہ_متوازی_امالہ_مثال_پ} کا دور۔}
\label{شکل_امالہ_متوازی_امالہ_مثال_پ}
\end{figure}

جواب:\عددی{\SI{5}{\milli\henry}}
\انتہا{مشق}
%==============================

\حصہ{حسابی ایمپلیفائر کے \عددی{RC} ادوار}
شکل \حوالہ{شکل_امالہ_تکمل_کار} میں \اصطلاح{تکمل کار}\فرہنگ{تکمل کار}\حاشیہب{integrator}\فرہنگ{integrator} دکھایا گیا ہے۔جوڑ \عددی{v_k} زمین کے ساتھ جڑا ہے لہٰذا
\begin{align*}
v_k=0
\end{align*}
ہو گا۔جوڑ \عددی{v_n} پر کرخوف مساوات رو
\begin{align*}
\frac{v_n-v_i}{R}+C \frac{\dif }{\dif t}(v_n-v_0)=0
\end{align*}
لکھتے ہوئے \عددی{v_n=v_k=0} پُر کرنے سے
\begin{align*}
\frac{0-v_i}{R}+C \frac{\dif }{\dif t}(0-v_0)=0
\end{align*}
یعنی
\begin{align*}
-\frac{v_i}{R}-C \frac{\dif v_0}{\dif t}=0
\end{align*}
حاصل ہوتا  ہے۔اس کو
\begin{align*}
\dif v_0=-\frac{v_i}{RC} \dif t
\end{align*}
لکھ کر تکمل لیتے ہوئے
\begin{align}
v_0=-\frac{1}{RC} \int_{-\infty}^{t} v_i \dif t
\end{align}
یا
\begin{align}\label{مساوات_امالہ_تکمل_کار_الف}
v_0=v(t_0)-\frac{1}{RC} \int_{t_0}^{t} v_i \dif t
\end{align}
حاصل ہوتا ہے۔اس مساوات کے تحت \عددی{v_0} اشارہ  \عددی{v_i} کے تکمل کے \عددی{\tfrac{1}{RC}} گنا ہے۔اسی لئے اس دور کو تکمل کار کہتے ہیں۔
\begin{figure}
\centering
\begin{tikzpicture}
\draw(0,0)node[op amp](u1){};
\draw(u1.+) to [short]++(-\x/4,0)node[above right]{$v_k$}node[ground]{};
\draw(u1.-) to [short]++(-\x/4,0)coordinate(kup)node[above right]{$v_n$} to [resistor,l_={$R$}]++(-\x,0)++(0,-\y)node[ground]{} to [american voltage source,l={$v_i$}]++(0,\y);
\draw(kup) to [short,*-]++(0,\y/2) to [short]++(\x/4,0) to [capacitor,l={$C$}]++(\x,0) -| (u1.out) to [short,*-o]++(\x/4,0)node[right]{$v_0$};
\end{tikzpicture}
\caption{تکمل کار۔}
\label{شکل_امالہ_تکمل_کار}
\end{figure}
%==================
\حصہ{تفرق کار}
شکل \حوالہ{شکل_امالہ_تفرق_کار} میں
\begin{align*}
v_k=0
\end{align*}
کے برابر ہے۔جوڑ \عددی{v_n} پر کرخوف مساوات رو
\begin{align*}
C\frac{\dif }{\dif t}(v_n-v_i)+\frac{v_n-v_0}{R}=0
\end{align*}
میں \عددی{v_n=v_k=0} پُر کرنے سے
\begin{align*}
C\frac{\dif}{\dif t} (0-v_i)+\frac{0-v_0}{R}=0
\end{align*}
حاصل ہوتا ہے جسے ترتیب دیتے ہوئے
\begin{align}\label{مساوات_امالہ_تفرق_کار_الف}
v_0=-RC \frac{\dif v_i}{\dif t}
\end{align}
لکھا جا سکتا ہے۔اس مساوات کے تحت \عددی{v_0} اشارہ \عددی{v_i} کے تفرق کے \عددی{-RC} گنا ہے۔اس لئے اس دور کو \اصطلاح{تفرق کار}\فرہنگ{تفرق کار}\حاشیہب{differentiator}\فرہنگ{differentiator} کہتے ہیں۔
\begin{figure}
\centering
\begin{tikzpicture}
\draw(0,0)node[op amp](u1){};
\draw(u1.+) to [short]++(-\x/4,0)node[above right]{$v_k$}node[ground]{};
\draw(u1.-) to [short]++(-\x/4,0)coordinate(kup)node[above right]{$v_n$} to [capacitor,l_={$C$}]++(-\x,0)++(0,-\y)node[ground]{} to [american voltage source,l={$v_i$}]++(0,\y);
\draw(kup) to [short,*-]++(0,\y/2) to [short]++(\x/4,0) to [resistor,l={$R$}]++(\x,0) -| (u1.out) to [short,*-o]++(\x/4,0)node[right]{$v_0$};
\end{tikzpicture}
\caption{تفرق کار۔}
\label{شکل_امالہ_تفرق_کار}
\end{figure}
%================
\ابتدا{مثال}\شناخت{مثال_امالہ_تکمل_کار_الف}
تکمل کار میں \عددی{R=\SI{10000}{\kilo\ohm}} اور \عددی{C=\SI{0.2}{\micro\farad}} ہیں جبکہ داخلی اشارہ شکل \حوالہ{شکل_امالہ_تکمل_کار_مثال_الف} میں دیا گیا ہے۔خارجی اشارہ حاصل کریں۔

\begin{figure}
\centering
\begin{subfigure}{0.5\textwidth}
\centering
\begin{tikzpicture}
\draw[gray](0,-2)--(0,2.5)node[left]{$v_i$};
\draw[gray](0,0)--++(4.25,0)node[right]{$t(\si{\milli\second})$};
\draw(-0.25,0)--(0,0)--(0,2)--(1,2)--(1,1)--(2,1)--(2,0)--(3,0)--(3,-1)--(4,-1)--(4,0)--(4.25,0);
\draw(0,1)--++(-0.1,0)node[left]{$\SI{2}{\volt}$};
\draw(0,2)--++(-0.1,0)node[left]{$\SI{4}{\volt}$};
\draw(0,-1)--++(-0.1,0)node[left]{$\SI{-2}{\volt}$};
\draw(1,0)--++(0,-0.1)node[below]{$1$};
\draw(2,0)--++(0,-0.1)node[below]{$2$};
\draw(3,0)node[above]{$3$};
\draw(4,0)node[above]{$4$};
\end{tikzpicture}
\caption*{(الف)}
\end{subfigure}%
\begin{subfigure}{0.5\textwidth}
\centering
\begin{tikzpicture}
\draw[gray](0,-2)--(0,2.5)node[left]{$v_0$};
\draw[gray](0,0)--++(4.25,0)node[right]{$t(\si{\milli\second})$};
\draw(-0.25,0)--(0,0)--(1,-1)--(2,-1.5)--(3,-1.5)--(4,-1)--(4.25,-1);
\draw(0,-1)--++(-0.1,0)node[left]{$\SI{-2}{\volt}$};
\draw(0,-1.5)--++(-0.1,0)node[left]{$\SI{-3}{\volt}$};
\draw(1,0)--++(0,-0.1)node[below]{$1$};
\draw(2,0)--++(0,-0.1)node[below]{$2$};
\draw(3,0)--++(0,-0.1)node[below]{$3$};
\draw(4,0)--++(0,-0.1)node[below]{$4$};
\end{tikzpicture}
\caption*{(ب)}
\end{subfigure}%
\caption{مثال \حوالہ{مثال_امالہ_تکمل_کار_الف} کا داخلی اشارہ۔}
\label{شکل_امالہ_تکمل_کار_مثال_الف}
\end{figure}

حل:مساوات \حوالہ{مساوات_امالہ_تکمل_کار_الف} کے تحت
\begin{align*}
v_0(t)&=v(t_0)-\frac{1}{\num{10000}\times 0.2\times 10^{-6} }\int_{t_0}^{t}v_i \dif t\\
&=v(t_0)-500\int_{t_0}^{t}v_i \dif t
\end{align*}
کے برابر ہے۔لمحہ \عددی{t=0} سے بالکل پہلے داخلی اشارے کی ابتدائی قیمت \عددی{v_i(0_-)=\SI{0}{\volt}} ہے جبکہ \عددی{t=\SI{0}{\milli\second}} تا \عددی{t=\SI{1}{\milli\second}} تک \عددی{v_i=\SI{2}{\volt}} ہے۔ان قیمتوں کو استعمال کرتے ہوئے
\begin{align*}
v_0(t)&=0-500\int_0^{t} 4 \dif t\\
&=-2000 t
\end{align*}
لکھا جا سکتا ہے جو سیدھے خط کی مساوات ہے جس کی ڈھلوان \عددی{\SI{-2000}{\volt\per\second}} ہے۔اس دورانیے کے اختتام \عددی{t=\SI{1}{\milli\second}} پر
\begin{align*}
v_0(\SI{1}{\milli\second})=-2000 \times 10^{-3}=\SI{-2}{\volt}
\end{align*}
حاصل ہوتا ہے۔شکل \حوالہ{شکل_امالہ_تکمل_کار_مثال_الف}-ب میں اس خط کو دکھایا گیا ہے۔اگلے ایک ملی سیکنڈ کی ابتدائی قیمتیں \عددی{t_0=\SI{1}{\milli\second}} اور \عددی{v_0(\SI{1}{\milli\second})=\SI{-2}{\volt}} ہوں گی لہٰذا مساوات \حوالہ{مساوات_امالہ_تکمل_کار_الف} درج ذیل لکھا جائے گا
\begin{align*}
v_0(t)&=-2-500\int_{\SI{1}{\milli\second}}^t 2\dif t\\
&=-2-1000(t-0.001)\\
&=-1-1000t
\end{align*}
جس سے  لمحہ \عددی{t=\SI{2}{\milli\second}} پر
\begin{align*}
v_0(\SI{2}{\milli\second})=-1-1000\times 0.002=\SI{-3}{\volt}
\end{align*}
حاصل ہوتا ہے۔لمحہ \عددی{t=\SI{2}{\milli\second}} تا \عددی{t=\SI{3}{\milli\second}} داخلی اشارہ صفر کے برابر ہے لہٰذا اس کا تکمل صفر ہو گا۔یوں خارجی اشارے میں اس دوران کوئی تبدیلی نہیں آئے گی اور یہ \عددی{\SI{-3}{\volt}} پر برقرار رہے گا۔آخری ایک ملی سیکنڈ میں اسی طرح حل کرتے ہوئے شکل-ب کا آخری حصہ ملتا ہے۔ 
\انتہا{مثال}
%======================
\ابتدا{مثال}\شناخت{مثال_امالہ_تفرق_کار_الف}
تفرق کار میں \عددی{R=\SI{2}{\kilo\ohm}} اور \عددی{C=\SI{0.5}{\micro\farad}} ہیں جبکہ اس کا داخلی اشارہ شکل \حوالہ{شکل_امالہ_تفرق_کار_اشارات_الف}-الف میں دیا گیا ہے۔خارجی اشارہ حاصل کریں۔

\begin{figure}
\centering
\begin{subfigure}{0.5\textwidth}
\centering
\begin{tikzpicture}
\draw[gray](0,-1.5)--(0,2.5)node[left]{$v_i$};
\draw[gray](0,0)--++(4.25,0)node[right]{$t(\si{\milli\second})$};
\draw(-0.25,0)--(0,0)--(2,2)--(4,0)--(4.25,0);
\draw(0,2)--++(-0.1,0)node[left]{$\SI{10}{\volt}$};
\draw(1,0)--++(0,-0.1)node[below]{$1$};
\draw(2,0)--++(0,-0.1)node[below]{$2$};
\draw(3,0)--++(0,-0.1)node[below]{$3$};
\draw(4,0)--++(0,-0.1)node[below]{$4$};
\end{tikzpicture}
\caption*{(الف)}
\end{subfigure}%
\begin{subfigure}{0.5\textwidth}
\centering
\begin{tikzpicture}
\draw[gray](0,-1.5)--(0,2.5)node[left]{$v_0$};
\draw[gray](0,0)--++(4.25,0)node[right]{$t(\si{\milli\second})$};
\draw(-0.25,0)--(0,0)--(0,-1)--(2,-1)--(2,1)--(4,1)--(4,0)--(4.25,0);
\draw(0,1)--++(-0.1,0)node[left]{$\SI{5}{\volt}$};
\draw(0,-1)--++(-0.1,0)node[left]{$\SI{-5}{\volt}$};
\draw(1,0)--++(0,-0.1)node[below]{$1$};
\draw(2,0)--++(0,-0.1)node[below left]{$2$};
\draw(3,0)--++(0,-0.1)node[below]{$3$};
\draw(4,0)node[below left]{$4$};
\end{tikzpicture}
\caption*{(ب)}
\end{subfigure}%
\caption{مثال \حوالہ{مثال_امالہ_تفرق_کار_الف} کے اشارات۔}
\label{شکل_امالہ_تفرق_کار_اشارات_الف}
\end{figure}

حل:شکل \حوالہ{شکل_امالہ_تفرق_کار_اشارات_الف}-الف میں چار عدد دورانیے منتخب کیے جا سکتے ہیں جن کے دوران داخلی اشارے کے تفرق درج ذیل ہیں۔
\begin{align*}
\frac{\dif v_i}{\dif t}=
\begin{cases}
0, & t<0\\
+5000, & 0< t < \SI{2}{\milli\second}\\
-5000,  &\SI{2}{\milli\second}<t<\SI{4}{\milli\second}\\
0, &\SI{4}{\milli\second}
\end{cases}
\end{align*}
مساوات \حوالہ{مساوات_امالہ_تفرق_کار_الف} میں دی گئی قیمتیں پُر کرنے سے
\begin{align*}
v_0=-0.001 \frac{\dif v_i}{\dif t}
\end{align*}
حاصل ہوتا ہے جس میں \عددی{\tfrac{\dif v_i}{\dif t}} کی قیمتیں پُر کرنے سے درج ذیل حاصل ہوتا ہے جس کو شکل \حوالہ{شکل_امالہ_تفرق_کار_اشارات_الف}-ب میں دکھایا گیا ہے۔
\begin{align*}
v_0=
\begin{cases}
-0.001(0)=\SI{0}{\volt}, & t<0\\
-0.001( 5000)=\SI{-5}{\volt}, & 0< t < \SI{2}{\milli\second}\\
-0.001(-5000)=\SI{5}{\volt},  &\SI{2}{\milli\second}<t<\SI{4}{\milli\second}\\
-0.001(0)=\SI{0}{\volt}, &\SI{4}{\milli\second}
\end{cases}
\end{align*}
\انتہا{مثال}
%=======================

\حصہء{سوالات}
%============================
\ابتدا{سوال}
ایک سو مائیکرو فیراڈ کے برق گیر میں دس سیکنڈ کے لئے  ایک ملی ایمپیئر رو سے بار بھرنے کے بعد برق گیر کا دباو دریافت کریں۔

جواب:\عددی{\SI{100}{\volt}}  
\انتہا{سوال}
%========================
\ابتدا{سوال}
\عددی{\SI{8}{\micro\farad}} کے برق گیر پر \عددی{\SI{4}{\milli\coulomb}} بار پایا جاتا ہے۔اس پر دباو دریافت کریں۔

جواب:\عددی{\SI{500}{\volt}}
\انتہا{سوال}
%=======================
\ابتدا{سوال}
ایک برق گیر پر \عددی{\SI{12}{\volt}} دباو اور \عددی{\SI{96}{\nano\coulomb}}  بار پایا جاتا ہے۔اس کی گنجائش دریافت کریں۔

جواب:\عددی{C=\SI{8}{\nano\farad}}
\انتہا{سوال}
%=====================
\ابتدا{سوال}
ایک برق گیر پر ابتدائی دباو \عددی{\SI{-20}{\volt}} ہے جبکہ اس کی گنجائش \عددی{C=\SI{5}{\micro\farad}} ہے۔اس میں \عددی{\SI{2}{\micro\ampere}} سے \عددی{\SI{90}{\second}} کے لئے بار بھرا جاتا ہے۔برق گیر پر اختتامی دباو حاصل کریں۔

جواب:\عددی{\SI{16}{\volt}}
\انتہا{سوال}
%=====================
\ابتدا{سوال}
\عددی{\SI{12}{\micro\farad}} برق گیر میں ذخیرہ توانائی \عددی{6\cos^2 3000t \, \si{\micro\joule}} ہے۔برق گیر کی رو دریافت کریں۔ 

جواب:\عددی{i_C=-0.036\sin 3000t \,\si{\ampere}}
\انتہا{سوال}
%====================
\ابتدا{سوال}\شناخت{سوال_برق_گیر_رو_دباو_الف}
ابتدائی طور پر بے بار \عددی{\SI{0.2}{\milli\farad}} برق گیر کو شکل \حوالہ{شکل_سوال_برق_گیر_رو_دباو_الف} کی رو سے بھرا جاتا ہے۔برق گیر پر دباو کا خط کھینچیں۔
\begin{figure}
\centering
\begin{subfigure}{0.5\textwidth}
\centering
\begin{tikzpicture}
\begin{axis}[small,xlabel={$t\,(\si{\second})$},ylabel={$i,\,(\si{\milli\ampere})$},xtick={0,3,4,6},xticklabels={$0$,$3$,$4$,$6$},ytick={0,1,2},yticklabels={$0$,$10$,$20$},ylabel style={rotate=-90},ylabel style={at={(axis description cs:0,1.05)}}]
\addplot[] plot coordinates {(-0.5,0) (0,0) (0,1) (3,1) (3,0) (4,0) (4,2) (6,2) (6,0)(6.5,0) };
\end{axis}
\end{tikzpicture}
\caption*{(الف)}
\end{subfigure}%
\begin{subfigure}{0.5\textwidth}
\centering
\begin{tikzpicture}
\begin{axis}[small,xlabel={$t\,(\si{\second})$},ylabel={$v_C,\,(\si{\volt})$},xtick={0,3,4,6},xticklabels={$0$,$3$,$4$,$6$},ytick={0,150,350},yticklabels={$0$,$150$,$350$},ylabel style={rotate=-90},ylabel style={at={(axis description cs:0,1.05)}}]
\addplot[] plot coordinates {(-0.5,0) (0,0)  (3,150) (4,150)  (6,350) (6.5,350) };
\end{axis}
\end{tikzpicture}
\caption*{(ب)}
\end{subfigure}%
\caption{سوال \حوالہ{سوال_برق_گیر_رو_دباو_الف} کے اشکال۔}
\label{شکل_سوال_برق_گیر_رو_دباو_الف}
\end{figure}

جواب:شکل-ب میں دباو دکھایا گیا ہے۔
\انتہا{سوال}
%=====================
\ابتدا{سوال}\شناخت{سوال_برق_گیر_رو_دباو_ب}
  \عددی{\SI{10}{\micro\farad}} برق گیر کے دباو کو شکل \حوالہ{شکل_سوال_برق_گیر_رو_دباو_ب} میں دکھایا گیا ہے۔اس کی رو کا خط کھینچیں۔
\begin{figure}
\centering
\begin{subfigure}{0.5\textwidth}
\centering
\begin{tikzpicture}
\begin{axis}[small,xlabel={$t\,(\si{\milli\second})$},ylabel={$v_C,\,(\si{\volt})$},xtick={0,4,6,8},xticklabels={$0$,$4$,$6$,$8$},ytick={0,100},yticklabels={$0$,$100$},ylabel style={rotate=-90},ylabel style={at={(axis description cs:0,1.05)}}]
\addplot[] plot coordinates {(-0.5,0) (0,0) (4,100) (6,100) (8,0) (8.5,0)};
\end{axis}
\end{tikzpicture}
\caption*{(الف)}
\end{subfigure}%
\begin{subfigure}{0.5\textwidth}
\centering
\begin{tikzpicture}
\begin{axis}[small,xlabel={$t\,(\si{\milli\second})$},ylabel={$i_C,\,(\si{\milli\ampere})$},xtick={0,4,6,8},xticklabels={$0$,$4$,$6$,$8$},ytick={0,250,-500},yticklabels={$0$,$250$,$-500$},ylabel style={rotate=-90},ylabel style={at={(axis description cs:0,1.05)}}]
\addplot[] plot coordinates {(-0.5,0) (0,0) (0,250) (4,250) (4,0) (6,0) (6,-500) (8,-500) (8,0) (8.5,0)};
\end{axis}
\end{tikzpicture}
\caption*{(ب)}
\end{subfigure}%
\caption{سوال \حوالہ{سوال_برق_گیر_رو_دباو_ب} کے اشکال۔}
\label{شکل_سوال_برق_گیر_رو_دباو_ب}
\end{figure}

جواب:شکل-ب میں رو دکھائی گئی ہے۔
\انتہا{سوال}
%=====================
\ابتدا{سوال}\شناخت{سوال_برق_گیر_رو_دباو_پ}
  \عددی{\SI{0.4}{\farad}} برق گیر کی رو کو شکل \حوالہ{شکل_سوال_برق_گیر_رو_دباو_پ} میں دکھایا گیا ہے۔اس پر دباو کا خط کھینچیں۔
\begin{figure}
\centering
\begin{subfigure}{0.5\textwidth}
\centering
\begin{tikzpicture}
\begin{axis}[small,xlabel={$t\,(\si{\second})$},ylabel={$i_C,\,(\si{\ampere})$},xtick={0,5,7,10},xticklabels={$0$,$5$,$7$,$10$},ytick={0,12,40},yticklabels={$0$,$12$,$40$},ylabel style={rotate=-90},ylabel style={at={(axis description cs:0,1.05)}}]
\addplot[] plot coordinates {(-0.5,0) (0,0) (5,40) (5,0)(7,0) (7,12) (10,12) (10,0) (11,0)};
\end{axis}
\end{tikzpicture}
\caption*{(الف)}
\end{subfigure}%
\begin{subfigure}{0.5\textwidth}
\centering
\begin{tikzpicture}
\begin{axis}[small,xlabel={$t\,(\si{\second})$},ylabel={$v_C,\,(\si{\volt})$},xtick={0,5,7,10},xticklabels={$0$,$5$,$7$,$10$},ytick={0,250,340},yticklabels={$0$,$250$,$340$},ylabel style={rotate=-90},ylabel style={at={(axis description cs:0,1.05)}}]
\addplot[] plot coordinates {(-0.5,0)(0,0)};
\addplot[domain=0:5]{10*x^2};
\addplot[]plot coordinates {(5,250) (7,250) (10,340) (11,340)}; 
\end{axis}
\end{tikzpicture}
\caption*{(ب)}
\end{subfigure}%
\caption{سوال \حوالہ{سوال_برق_گیر_رو_دباو_پ} کے اشکال۔}
\label{شکل_سوال_برق_گیر_رو_دباو_پ}
\end{figure}

جواب:شکل-ب میں دباو دکھایا گیا ہے۔
\انتہا{سوال}
%=====================
\ابتدا{سوال}\شناخت{سوال_برق_گیر_رو_دباو_ت}
 \عددی{\SI{0.1}{\farad}} برق گیر کا دباو شکل \حوالہ{شکل_سوال_برق_گیر_رو_دباو_ت} میں دیا گیا ہے۔اس کی رو کا خط کھینچیں۔
\begin{figure}
\centering
\begin{subfigure}{0.5\textwidth}
\centering
\begin{tikzpicture}
\begin{axis}[small,xlabel={$t\,(\si{\second})$},ylabel={$v_C,\,(\si{\volt})$},ylabel style={rotate=-90},ylabel style={at={(axis description cs:0,1.05)}},xtick={2.5,5,7.5,10},xticklabels={$2.5$,$5$,$7.5$,$10$},ytick={-1,0,1},yticklabels={$-1$,$0$,$1$}]
\addplot[domain=0:10,smooth] {sin(36*x)}; 
\end{axis}
\end{tikzpicture}
\caption*{(الف)}
\end{subfigure}%
\begin{subfigure}{0.5\textwidth}
\centering
\begin{tikzpicture}
\begin{axis}[small,xlabel={$t\,(\si{\second})$},ylabel={$i_C,\,(\si{\ampere})$},ylabel style={rotate=-90},ylabel style={at={(axis description cs:0,1.05)}},xtick={2.5,5,7.5,10},xticklabels={$2.5$,$5$,$7.5$,$10$},ytick={-3.6,0,3.6},yticklabels={$-3.6$,$0$,$3.6$}]
\addplot[domain=0:10,smooth] {3.6*cos(36*x)}; 
\end{axis}
\end{tikzpicture}
\caption*{(ب)}
\end{subfigure}%
\caption{سوال \حوالہ{سوال_برق_گیر_رو_دباو_ت} کے اشکال۔}
\label{شکل_سوال_برق_گیر_رو_دباو_ت}
\end{figure}

جواب:شکل-ب میں رو دکھائی گئی ہے۔
\انتہا{سوال}
%=====================
\ابتدا{سوال}\شناخت{سوال_برق_گیر_رو_دباو_ٹ}
ابتدائی طور پر بے بار \عددی{\SI{10}{\micro\farad}} برق گیر کی رو شکل \حوالہ{شکل_سوال_برق_گیر_رو_دباو_ٹ} میں دی گئی ہے۔اس کے دباو کا خط کھینچیں۔
\begin{figure}
\centering
\begin{subfigure}{0.5\textwidth}
\centering
\begin{tikzpicture}
\begin{axis}[small,xlabel={$t\,(\si{\milli\second})$},ylabel={$i_C,\,(\si{\milli\ampere})$},ylabel style={rotate=-90},ylabel style={at={(axis description cs:0,1.05)}},xtick={10,20,30,40},xticklabels={$10$,$20$,$30$,$40$},ytick={-20,0,20},yticklabels={$-20$,$0$,$20$}]
\addplot[] plot coordinates {(-5,0) (0,0) (0,20) (10,20) (10,-20) (20,-20) (20,20) (30,20) (30,-20) (40,-20) (40,0) (45,0)}; 
\end{axis}
\end{tikzpicture}
\caption*{(الف)}
\end{subfigure}%
\begin{subfigure}{0.5\textwidth}
\centering
\begin{tikzpicture}
\begin{axis}[small,xlabel={$t\,(\si{\milli\second})$},ylabel={$v_C,\,(\si{\volt})$},ylabel style={rotate=-90},ylabel style={at={(axis description cs:0,1.05)}},xtick={10,20,30,40},xticklabels={$10$,$20$,$30$,$40$},ytick={0,20},yticklabels={$0$,$20$}]
\addplot[] plot coordinates {(-5,0) (0,0) (10,20) (20,0) (30,20) (40,0) (45,0)};
\end{axis}
\end{tikzpicture}
\caption*{(ب)}
\end{subfigure}%
\caption{سوال \حوالہ{سوال_برق_گیر_رو_دباو_ٹ} کے اشکال۔}
\label{شکل_سوال_برق_گیر_رو_دباو_ٹ}
\end{figure}

جواب:شکل-ب میں دباو دکھایا گیا ہے۔
\انتہا{سوال}
%=====================
\ابتدا{سوال}
ایک امالہ گیر میں \عددی{\SI{5}{\milli\second}} کے دورانیے میں رو \عددی{\SI{0}{\milli\ampere}} سے بڑھ کر \عددی{\SI{100}{\milli\ampere}} ہو جاتی ہے۔اس دورانیے میں امالی دباو \عددی{\SI{400}{\milli\volt}} ہوتا ہے۔امالہ گیر کی گنجائش دریافت کریں۔

جواب:\عددی{\SI{2}{\milli\henry}}
\انتہا{سوال}
%=============================
\ابتدا{سوال}
\عددی{\SI{50}{\milli\henry}} امالہ گیر کی رو \عددی{i=7\sin 314 t \, \si{\ampere}} ہے۔اس کے دباو کی مساوات حاصل کریں۔امالہ گیر میں ذخیرہ توانائی کی مساوات حاصل کریں۔

جواب:\عددی{v_L=109.9\cos 314t \,\si{\volt}}، \عددی{w=\tfrac{49}{40}\sin^2 314t \,\si{\joule}}
\انتہا{سوال}
%==========================
\ابتدا{سوال}
\عددی{\SI{0.4}{\henry}} امالہ گیر کی رو درج ذیل ہے۔لمحہ \عددی{t=\SI{-3}{\second}} اور \عددی{t=\SI{0.5}{\second}} پر امالہ کی رو اور امالہ میں ذخیرہ توانائی دریافت کریں۔
\begin{align*}
i_L=
\begin{cases}
0 & t<0\\
50(1-e^{-2t})\,\si{\milli\ampere} & t>0
\end{cases}
\end{align*}

جوابات:\عددی{\SI{0}{\ampere}}، \عددی{\SI{0}{\joule}}، \عددی{\SI{31.61}{\milli\ampere}}، \عددی{\SI{199.8}{\micro\joule}}
\انتہا{سوال}
%=========================
\ابتدا{سوال}\شناخت{سوال_امالہ_دباو_رو_خط_الف}
شکل \حوالہ{شکل_سوال_امالہ_دباو_رو_خط_الف} میں \عددی{\SI{3}{\henry}} کا دباو دیا گیا ہے۔اس کی رو کا خط کھینچیں۔ابتدائی رو صفر ہے۔
\begin{figure}
\centering
\begin{subfigure}{0.5\textwidth}
\centering
\begin{tikzpicture}
\begin{axis}[small,xlabel={$t\,(\si{\second})$},ylabel={$v,\,(\si{\volt})$},ylabel style={rotate=-90},ylabel style={at={(axis description cs:0,1.05)}},xtick={0,2,3,5},xticklabels={$0$,$2$,$3$,$5$},ytick={-10,0,10},yticklabels={$-12$,$0$,$12$}]
\addplot[] plot coordinates {(-0.5,0) (0,0) (0,12) (2,12) (2,0) (3,0) (3,-12) (5,-12) (5,0) (5.5,0) }; 
\end{axis}
\end{tikzpicture}
\caption*{(الف)}
\end{subfigure}%
\begin{subfigure}{0.5\textwidth}
\centering
\begin{tikzpicture}
\begin{axis}[small,xlabel={$t\,(\si{\second})$},ylabel={$i,\,(\si{\ampere})$},ylabel style={rotate=-90},ylabel style={at={(axis description cs:0,1.05)}},xtick={0,2,3,5},xticklabels={$0$,$2$,$3$,$5$},ytick={0,8},yticklabels={$0$,$8$}]
\addplot[] plot coordinates {(-0.5,0) (0,0)(2,8) (3,8) (5,0) (5.5,0) }; 
\end{axis}
\end{tikzpicture}
\caption*{(ب)}
\end{subfigure}%
\caption{سوال \حوالہ{سوال_امالہ_دباو_رو_خط_الف} کے اشکال۔}
\label{شکل_سوال_امالہ_دباو_رو_خط_الف}
\end{figure}

جواب:شکل-ب میں رو دی گئی ہے۔
\انتہا{سوال}
%=====================
\ابتدا{سوال}\شناخت{سوال_امالہ_دباو_رو_خط_ب}
شکل \حوالہ{شکل_سوال_امالہ_دباو_رو_خط_ب} میں \عددی{\SI{10}{\milli\henry}} کا دباو دیا گیا ہے۔اس کی رو کا خط کھینچیں۔ابتدائی رو صفر ہے۔
\begin{figure}
\centering
\begin{subfigure}{0.5\textwidth}
\centering
\begin{tikzpicture}
\begin{axis}[small,xlabel={$t\,(\si{\milli\second})$},ylabel={$v,\,(\si{\milli\volt})$},ylabel style={rotate=-90},ylabel style={at={(axis description cs:0,1.05)}},xtick={0,1,2,3,4,5,6,7,8},xticklabels={$0$,$1$,$2$,$3$,$4$,$5$,$6$,$7$,$8$},ytick={0,1},yticklabels={$0$,$50$}]
\addplot[] plot coordinates {(-0.5,0) (0,0) (0,1) (1,1) (1,0) (2,0) (2,1) (3,1) (3,0) (4,0) (4,1) (5,1) (5,0) (6,0) (6,1) (7,1) (7 ,0) (8,0) (8,1) (8.5,1) }; 
\end{axis}
\end{tikzpicture}
\caption*{(الف)}
\end{subfigure}%
\begin{subfigure}{0.5\textwidth}
\centering
\begin{tikzpicture}
\begin{axis}[small,xlabel={$t\,(\si{\milli\second})$},ylabel={$i,\,(\si{\milli\ampere})$},ylabel style={rotate=-90},ylabel style={at={(axis description cs:0,1.05)}},xtick={0,1,2,3,4,5,6,7,8},xticklabels={$0$,$1$,$2$,$3$,$4$,$5$,$6$,$7$,$8$},ytick={0,1,2,3,4},yticklabels={$0$,$5$,$10$,$15$,$20$}]
\addplot[] plot coordinates {(-0.5,0) (0,0) (1,1)(2,1) (3,2)  (4,2) (5,3) (6,3) (7,4) (7.5,4) }; 
\end{axis}
\end{tikzpicture}
\caption*{(ب)}
\end{subfigure}%
\caption{سوال \حوالہ{سوال_امالہ_دباو_رو_خط_ب} کے اشکال۔}
\label{شکل_سوال_امالہ_دباو_رو_خط_ب}
\end{figure}

جواب:شکل-ب میں رو دی گئی ہے۔
\انتہا{سوال}
%=====================
\ابتدا{سوال}\شناخت{سوال_امالہ_دباو_رو_خط_پ}
شکل \حوالہ{شکل_سوال_امالہ_دباو_رو_خط_پ} میں کل \عددی{\SI{2.5}{\joule}} توانائی ذخیرہ ہے۔امالہ \عددی{L} دریافت کریں۔
\begin{figure}
\centering
\begin{tikzpicture}
\draw(0,0) to [american current source,l={$\SI{3}{\ampere}$}]++(0,\y);
\draw(\x,0) to [resistor,*-*,l={$\SI{100}{\ohm}$}]++(0,\y);
\draw(2*\x,0) to [capacitor,*-*,l={$\SI{100}{\micro\henry}$}]++(0,\y);
\draw(3*\x,0) to [resistor,l_={$\SI{200}{\ohm}$}]++(0,\y);
\draw(0,0) to [short]++(3*\x,0);
\draw(0,\y) to [short]++(2*\x,0) to [inductor,l={$L$}]++(\x,0);
\end{tikzpicture}
\caption{سوال \حوالہ{سوال_امالہ_دباو_رو_خط_پ} کا دور۔}
\label{شکل_سوال_امالہ_دباو_رو_خط_پ}
\end{figure}

جواب:\عددی{L=\SI{1}{\henry}}
\انتہا{سوال}
%======================
\ابتدا{سوال}\شناخت{سوال_امالہ_دباو_رو_خط_ت}
شکل \حوالہ{شکل_سوال_امالہ_دباو_رو_خط_ت}-الف میں امالہ گیر اور برق گیر میں برابر توانائی ذخیرہ ہے۔برق گیر کی گنجائش دریافت کریں۔
\begin{figure}
\centering
\begin{subfigure}{0.5\textwidth}
\centering
\begin{tikzpicture}
\draw(0,0) to [american voltage source,l={$\SI{10}{\volt}$}]++(0,\y);
\draw(\x,0) to [resistor,*-*,l={$\SI{5}{\ohm}$}]++(0,\y);
\draw(2*\x,0) to [capacitor,l={$C$}]++(0,\y);
\draw(0,0) to [short]++(2*\x,0);
\draw(0,\y) to [inductor,l={$\SI{2}{\henry}$}]++(\x,0) to [short]++(\x,0);
\end{tikzpicture}
\caption*{(الف)}
\end{subfigure}%
\begin{subfigure}{0.5\textwidth}
\centering
\begin{tikzpicture}
\draw(0,0) to [short]++(2*\x,0);
\draw(0,\y) to [capacitor,l={$\SI{2}{\micro\farad}$}]++(\x,0) to [short]++(\x,0);
\draw(0,2*\y) to [short]++(2*\x,0);
\draw(0,0) to [short]++(0,\y) to [capacitor,*-,l={$\SI{14}{\micro\farad}$}]++(0,\y);
\draw(\x,0) to [capacitor,*-*,l={$\SI{6}{\micro\farad}$}]++(0,\y);
\draw(2*\x,0) to [capacitor,l={$\SI{6}{\micro\farad}$}]++(0,\y) to [capacitor,*-,l={$\SI{7}{\micro\farad}$}]++(0,\y);
\draw(\x,\y) to [short,*-o] ++(0,\y/4);
\draw(\x,2*\y) to [short,*-o] ++(0,-\y/4);
\draw[stealth-](\x,\y+\y/2)--++(-\x/4,0)--++(0,\y/8) node[above]{$C$}; 
\end{tikzpicture}
\caption*{(ب)}
\end{subfigure}%
\caption{سوال \حوالہ{سوال_امالہ_دباو_رو_خط_ت} اور سوال \حوالہ{سوال_امالہ_کل_برق_گیر_الف} کے ادوار۔}
\label{شکل_سوال_امالہ_دباو_رو_خط_ت}
\end{figure}

جواب:\عددی{C=\SI{0.08}{\farad}}
\انتہا{سوال}
%======================
\ابتدا{سوال}\شناخت{سوال_امالہ_کل_برق_گیر_الف}
شکل \حوالہ{شکل_سوال_امالہ_دباو_رو_خط_ت}-ب میں کل \عددی{C} دریافت کریں۔

جواب:\عددی{C=\SI{14}{\micro\farad}}
\انتہا{سوال}
%========================
\ابتدا{سوال}\شناخت{سوال_امالہ_کل_برق_گیر_ب}
شکل \حوالہ{شکل_سوال_امالہ_کل_برق_گیر_ب}-الف میں کل \عددی{C} دریافت کریں۔
\begin{figure}
\centering
\begin{tikzpicture}
\draw(0,0) to [short]++(0,\y) to [capacitor,l={$\SI{3}{\micro\farad}$}]++(0,\y);
\draw(\x,0) to [capacitor,l={$\SI{3}{\micro\farad}$}]++(0,\y) to [short]++(0,\y) ;
\draw(2*\x,0) to [short]++(0,\y) to [capacitor,l={$\SI{2}{\micro\farad}$}]++(0,\y);
\draw(3*\x,0) to [capacitor,l={$\SI{2}{\micro\farad}$}]++(0,\y) to [short]++(0,\y) ;
\draw(0,0) to [short]++(\x,0) to [capacitor,*-*,l={$\SI{6}{\micro\farad}$}]++(\x,0) to [short]++(\x,0);
\draw(0,2*\y) to [short]++(\x,0) to [capacitor,*-*,l={$\SI{4}{\micro\farad}$}]++(\x,0) to [short]++(\x,0);
\draw(\x,\y) to [short,*-o]++(\x/4,0);
\draw(2*\x,\y) to [short,*-o]++(-\x/4,0);
\draw[stealth-](\x+\x/2,\y)--++(0,-\y/8)node[below]{$C$};
\end{tikzpicture}
\caption{سوال \حوالہ{سوال_امالہ_کل_برق_گیر_ب} کا دور۔}
\label{شکل_سوال_امالہ_کل_برق_گیر_ب}
\end{figure}
جواب:\عددی{C=\SI{5}{\micro\farad}}
\انتہا{سوال}
%========================
\ابتدا{سوال}\شناخت{سوال_امالہ_کل_برق_گیر_پ}
شکل \حوالہ{شکل_سوال_امالہ_کل_برق_گیر_پ}-الف میں کل \عددی{C} حاصل کریں۔
\begin{figure}
\centering
\begin{subfigure}{0.5\textwidth}
\centering
\begin{tikzpicture}
\draw(0,0) to [capacitor,l={$\SI{4}{\micro\farad}$}]++(0,\y) to [capacitor,l={$\SI{8}{\micro\farad}$}]++(0,\y);
\draw(\x,0) to [short,*-]++(0,\y) to [capacitor,-*,l={$\SI{6}{\micro\farad}$}]++(0,\y);
\draw(2*\x,0) to [capacitor,l={$\SI{6}{\micro\farad}$}]++(0,\y) to [capacitor,l={$\SI{12}{\micro\farad}$}]++(0,\y);
\draw(0,0) to [short]++(2*\x,0);
\draw(0,2*\y) to [capacitor,l={$\SI{16}{\micro\farad}$}]++(\x,0) to [short]++(\x,0);
\draw(0,\y) to [short,*-o]++(\x/4,0);
\draw(\x,\y) to [short,*-o]++(-\x/4,0);
\draw[stealth-] (\x/2,\y)--++(0,-\y/4)node[below]{$C$};
\end{tikzpicture}
\caption*{(الف)}
\end{subfigure}%
\begin{subfigure}{0.5\textwidth}
\centering
\begin{tikzpicture}
\draw(0,0) to [capacitor,o-,l={$\SI{3}{\micro\farad}$}]++(\x,0) to [capacitor,l_={$C$}]++(0,-\y) to [short,-o]++(-\x,0);
\draw(\x,0) to [short,*-]++(\x/2,0) to [capacitor,l={$\SI{5}{\micro\farad}$}]++(0,-\y) to [short,-*]++(-\x/2,0);
\draw[stealth-](0,-\y/2)--++(-\x/8,0)--++(0,-\y/8)node[below]{$C_{\text{کل}}$};
\end{tikzpicture}
\caption*{(ب)}
\end{subfigure}%
\caption{سوال \حوالہ{سوال_امالہ_کل_برق_گیر_پ} اور سوال \حوالہ{سوال_امالہ_کل_برق_گیر_ت} کے ادوار۔}
\label{شکل_سوال_امالہ_کل_برق_گیر_پ}
\end{figure}

جواب:\عددی{C=\SI{8}{\micro\farad}}
\انتہا{سوال}
%==========================
\ابتدا{سوال}\شناخت{سوال_امالہ_کل_برق_گیر_ت}
شکل \حوالہ{شکل_سوال_امالہ_کل_برق_گیر_پ}-ب میں \عددی{C_{\text{کل}}=\tfrac{15}{23}\,\si{\micro\farad}} ہے۔آپ سے گزارش ہے کہ \عددی{C} کی قیمت معلوم کریں۔

جواب:\عددی{C=\SI{1}{\micro\farad}}
\انتہا{سوال}
%=========================
\ابتدا{سوال}\شناخت{سوال_امالہ_کل_برق_گیر_ٹ}
شکل \حوالہ{شکل_سوال_امالہ_کل_برق_گیر_ٹ}-الف میں سلسلہ وار برق گیر دکھائے گئے ہیں جن میں رو بار بھرتی ہے۔دباو \عددی{V_0} دریافت کریں۔
\begin{figure}
\centering
\begin{subfigure}{0.5\textwidth}
\centering
\begin{tikzpicture}[american voltages]
\draw(0,0) to [capacitor,o-,l={$\SI{30}{\micro\farad}$},v={$V_0$}]++(\x,0) to [capacitor,l_={$\SI{5}{\micro\farad}$},v^<={$\SI{20}{\volt}$}]++(0,-\y-\y/2) to [short,-o]++(-\x,0);
\end{tikzpicture}
\caption*{(الف)}
\end{subfigure}%
\begin{subfigure}{0.5\textwidth}
\centering
\begin{tikzpicture}[american voltages]
\draw(0,0) to [short,o-]++(\x/2,0) to [capacitor,l={$\SI{6}{\micro\farad}$},v={$V_1$}]++(0,-\y) to
 [capacitor,l={$\SI{12}{\micro\farad}$},v={$V_2$}]++(0,-\y) to [short,-o]++(-\x/2,0);
\draw(\x/2,0) to [short,*-]++(\x,0) to [capacitor,v^<={$\SI{21}{\volt}$}]++(0,-2*\y) to [short,-*]++(-\x,0);
\end{tikzpicture}
\caption*{(ب)}
\end{subfigure}%
\caption{سوال \حوالہ{سوال_امالہ_کل_برق_گیر_ٹ} اور سوال \حوالہ{سوال_امالہ_کل_برق_گیر_ث} کے ادوار۔}
\label{شکل_سوال_امالہ_کل_برق_گیر_ٹ}
\end{figure}

جواب:\عددی{V_0=\tfrac{10}{3}\,\si{\volt}}
\انتہا{سوال}
%==========================
\ابتدا{سوال}\شناخت{سوال_امالہ_کل_برق_گیر_ث}
شکل \حوالہ{شکل_سوال_امالہ_کل_برق_گیر_ٹ}-ب میں سلسلہ وار برق گیر دکھائے گئے ہیں جن میں رو بار بھرتی ہے۔دباو \عددی{V_1} اور \عددی{V_2} حاصل کریں۔

جوابات:\عددی{V_1=\SI{14}{\volt}}، \عددی{V_2=\SI{7}{\volt}}
\انتہا{سوال}
%====================
\ابتدا{سوال}\شناخت{سوال_امالہ+امالہ_گیر_جڑے_الف}
شکل \حوالہ{شکل_سوال_امالہ+امالہ_گیر_جڑے_الف} میں کل امالہ \عددی{L} دریافت کریں۔
\begin{figure}
\centering
\begin{tikzpicture}
\draw(0,0) to [inductor,o-,l={$\SI{0.5}{\milli\henry}$}]++(\x,0) to [short]++(2*\x,0) to [inductor,l_={$\SI{1}{\milli\henry}$}]++(0,\y) to [inductor,l_={$\SI{2}{\milli\henry}$}]++(0,\y) to [short]++(-2*\x,0) to [inductor,-o,l={$\SI{2}{\milli\henry}$}]++(-\x,0);
\draw(\x,2*\y) to [inductor,*-,l={$\SI{4}{\milli\henry}$}]++(0,-\y)to[short]++(\x,0);
\draw(2*\x,2*\y) to [inductor,*-,l={$\SI{8}{\milli\henry}$}]++(0,-\y);
\draw(\x+\x/2,0) to [inductor,*-*,l={$\frac{1}{3}\,\si{\milli\henry}$}]++(0,\y);
\draw[stealth-] (\x/4,\y) --++(-\x/4,0)--++(0,-\y/8)node[below]{$L$};
\end{tikzpicture}
\caption{سوال \حوالہ{سوال_امالہ+امالہ_گیر_جڑے_الف} کا دور۔}
\label{شکل_سوال_امالہ+امالہ_گیر_جڑے_الف}
\end{figure}

جواب:\عددی{L=\SI{4}{\milli\henry}}
\انتہا{سوال}
%=====================
\ابتدا{سوال}\شناخت{سوال_امالہ+امالہ_گیر_جڑے_ب}
شکل \حوالہ{شکل_سوال_امالہ+امالہ_گیر_جڑے_ب} میں \عددی{L} دریافت کریں۔
\begin{figure}
\centering
\begin{tikzpicture}
\draw(0,0) to [short,o-]++(0,-\y/2) to [inductor,l={$\SI{2}{\milli\henry}$}]++(0,-\y) to [inductor,l={$\SI{12}{\milli\henry}$}]++(\x,0) to [inductor,l={$\SI{3}{\milli\henry}$}]++(\x,0) to [inductor,l={$\SI{4}{\milli\henry}$}]++(0,\y) to [short,-o]++(0,\y/2);
\draw(0,-\y/2) to [inductor,*-*,l={$\SI{6}{\milli\henry}$}]++(2*\x,0);
\draw(\x,-\y-\y/2) to [inductor,*-*,l={$\SI{4}{\milli\henry}$}]++(0,-\y);
\draw(0,-\y-\y/2) to [short,*-]++(0,-\y) to [short]++(2*\x,0) to [short,-*]++(0,\y);
\draw[stealth-] (\x,-\y/8)--++(0,\y/8)--++(-\x/8,0)node[left]{$L$};
\end{tikzpicture}
\caption{سوال \حوالہ{سوال_امالہ+امالہ_گیر_جڑے_ب} کا دور۔}
\label{شکل_سوال_امالہ+امالہ_گیر_جڑے_ب}
\end{figure}

جواب:\عددی{L=\SI{3}{\milli\henry}}
\انتہا{سوال}
%=======================
\ابتدا{سوال}\شناخت{سوال_امالہ+امالہ_گیر_جڑے_پ}
شکل \حوالہ{شکل_سوال_امالہ+امالہ_گیر_جڑے_پ} میں \عددی{CD} کو کھلے سر رکھتے ہوئے \عددی{L_{AB}} دریافت کریں۔ \عددی{CD} کو قصر دور کرتے ہوئے امالہ \عددی{L_{AB}} کو دوبارہ حاصل کریں۔
\begin{figure}
\centering
\begin{tikzpicture}
\pgfmathsetmacro{\ang}{atan(\y/(2.5*\x))}
\pgfmathsetmacro{\len}{\y/sin(\ang)-\y}
\draw(0,0)node[left]{$A$} to [short,o-]++(\x,0) to [inductor,l={$\SI{10}{\micro\henry}$}]++(\x,0) to [short,-o]++(\x,0)node[right]{$C$};
\draw(0,-\y)node[left]{$B$} to [short,o-]++(\x,0) to [inductor,l_={$\SI{4}{\micro\henry}$}]++(\x,0) to [short,-o]++(\x,0)node[right]{$D$};
\draw(\x/4,0) to [inductor,*-]++(-\ang:\y) to [short,-*]++(-\ang:\len);
\draw(3*\x-\x/4,0) to [inductor,*-]++(-180+\ang:\y) to [short,-*]++(-180+\ang:\len);
\draw(\x/2,-\y/2)node{$\SI{8}{\micro\henry}$};
\draw(2*\x+\x/2,-\y/2)node{$\SI{6}{\micro\henry}$};
\end{tikzpicture}
\caption{سوال \حوالہ{سوال_امالہ+امالہ_گیر_جڑے_پ} کا دور۔}
\label{شکل_سوال_امالہ+امالہ_گیر_جڑے_پ}
\end{figure}

جواب:\عددی{L=\tfrac{48}{7}\,\si{\micro\henry}}، \عددی{L=\tfrac{308}{45}\,\si{\micro\henry}}
\انتہا{سوال}
%=======================
\ابتدا{سوال}\شناخت{سوال_امالہ_حسابی_الف}
شکل \حوالہ{شکل_سوال_امالہ_حسابی_الف}-الف کے دور کا داخلی اشارہ \عددی{v_s} شکل-ب میں دیا گیا ہے۔خارجی اشارے \عددی{v_0} کا خط کھینچیں۔
\begin{figure}
\centering
\begin{subfigure}{1\textwidth}
\centering
\begin{tikzpicture}
\draw(0,0) node [op amp](u1){};
\draw(u1.+) --++(-\x/8,0)node[ground]{};
\draw(u1.-)--++(-\x/8,0) to [resistor,*-o,l={$\SI{40}{\kilo\ohm}$}]++(-\x,0)node[left]{$v_s$};
\draw(u1.-)++(-\x/8,0) to [short]++(0,\y/2) to [capacitor,l={$\SI{5}{\micro\farad}$}]++(\x+\x/4,0)-|(u1.out);
\draw(u1.out) to [short,*-o]++(\x/4,0)node[right]{$v_0$};
\end{tikzpicture}
\caption*{(الف)}
\end{subfigure}
\begin{subfigure}{0.5\textwidth}2
\centering
\begin{tikzpicture}
\begin{axis}[small,xlabel={$t\, (\si{\second})$},ylabel={$v_s\,(\si{\volt})$},ylabel style={rotate=-90},ylabel style={at={(axis description cs:0,1.05)}},xtick={0,0.2,0.3,0.4},xticklabels={$0$,$0.2$,$0.3$,$0.4$},ytick={-3,0,2},yticklabels={$-3$,$0$,$2$}]
\addplot[]plot coordinates {(-0.1,0) (0,0) (0,2) (0.2,2) (0.2,0) (0.3,0) (0.3,-3) (0.4,-3) (0.4,0) (0.5,0)};
\end{axis}
\end{tikzpicture}
\caption*{(ب)}
\end{subfigure}%
\begin{subfigure}{0.5\textwidth}
\centering
\begin{tikzpicture}
\begin{axis}[small,xlabel={$t\, (\si{\second})$},ylabel={$v_0\,(\si{\volt})$},ylabel style={rotate=-90},ylabel style={at={(axis description cs:0,1.05)}},xtick={0,0.2,0.3,0.4},xticklabels={$0$,$0.2$,$0.3$,$0.4$},ytick={-2,-1.5,0},yticklabels={$-2$,$-1.5$,$0$}]
\addplot[]plot coordinates {(-0.1,0) (0,0) (0.2,-2)  (0.3,-2) (0.4,-1.5)(0.5,-1.5)};
\end{axis}
\end{tikzpicture}
\caption*{(پ)}
\end{subfigure}%
\caption{سوال \حوالہ{سوال_امالہ_حسابی_الف} کا دور۔}
\label{شکل_سوال_امالہ_حسابی_الف}
\end{figure}

جواب:خارجی اشارہ شکل-پ میں دکھایا گیا ہے۔
\انتہا{سوال}
%=======================
\ابتدا{سوال}\شناخت{سوال_امالہ_حسابی_ب}
شکل \حوالہ{شکل_سوال_امالہ_حسابی_ب}-الف کے دور کا داخلی اشارہ \عددی{v_s} شکل-ب میں دیا گیا ہے۔خارجی اشارے \عددی{v_0} کا خط کھینچیں۔
\begin{figure}
\centering
\begin{subfigure}{1\textwidth}
\centering
\begin{tikzpicture}
\draw(0,0) node [op amp](u1){};
\draw(u1.+) --++(-\x/8,0)node[ground]{};
\draw(u1.-)--++(-\x/8,0) to [capacitor,*-o,l_={$\SI{2}{\micro\farad}$}] ++(-\x,0)node[left]{$v_s$};
\draw(u1.-)++(-\x/8,0) to [short]++(0,\y/2) to [resistor,l={$\SI{2}{\kilo\ohm}$}]++(\x+\x/4,0)-|(u1.out);
\draw(u1.out) to [short,*-o]++(\x/4,0)node[right]{$v_0$};
\end{tikzpicture}
\caption*{(الف)}
\end{subfigure}
\begin{subfigure}{0.5\textwidth}2
\centering
\begin{tikzpicture}
\begin{axis}[small,xlabel={$t\, (\si{\milli\second})$},ylabel={$v_s\,(\si{\volt})$},ylabel style={rotate=-90},ylabel style={at={(axis description cs:0,1.05)}},xtick={0,3,5,9},xticklabels={$0$,$3$,$5$,$9$},ytick={0,2,6},yticklabels={$0$,$2$,$6$}]
\addplot[]plot coordinates {(0,2) (3,2) (5,6) (9,0) (10,0)};
\end{axis}
\end{tikzpicture}
\caption*{(ب)}
\end{subfigure}%
\begin{subfigure}{0.5\textwidth}
\centering
\begin{tikzpicture}
\begin{axis}[small,xlabel={$t\, (\si{\milli\second})$},ylabel={$v_0\,(\si{\volt})$},ylabel style={rotate=-90},ylabel style={at={(axis description cs:0,1.05)}},xtick={0,3,5,9},xticklabels={$0$,$3$,$5$,$9$},ytick={-8,0,6},yticklabels={$-8$,$0$,$6$}]
\addplot[]plot coordinates {(0,0) (3,0) (3,-8) (5,-8) (5,6) (9,6) (9,0) (10,0)};
\end{axis}
\end{tikzpicture}
\caption*{(پ)}
\end{subfigure}%
\caption{سوال \حوالہ{سوال_امالہ_حسابی_ب} کا دور۔}
\label{شکل_سوال_امالہ_حسابی_ب}
\end{figure}

جواب:خارجی اشارہ شکل-پ میں دکھایا گیا ہے۔
\انتہا{سوال}
%=======================

\باب{عارضی رد عمل}
\حصہ{تعارف}
ایسے ادوار جن میں امالہ گیر اور (یا) برق گیر پائے جاتے ہوں میں توانائی ذخیرہ کرنے کی صلاحیت ہوتی ہے۔توانائی ذخیرہ کرنے والے ادوار کا رد عمل منبع طاقت کے علاوہ ذخیرہ توانائی پر بھی منحصر ہوتا ہے۔ایسے ادوار میں کسی بھی طرح کی تبدیلی سے ذخیرہ توانائی میں تبدیلی رونما ہو سکتی ہے۔دور میں تبدیلی مثلاً  کسی سوئچ کے چالو یا غیر چالو کرنے سے پیدا ہو سکتی ہے۔ایسی صورت جہاں دور یکساں ایک ہی حالت میں رہے کو \اصطلاح{برقرار حالت}\فرہنگ{برقرار حالت}\حاشیہب{steady state}\فرہنگ{steady state} کہتے ہیں۔تبدیلی کے بعد دور متبادل برقرار حالت اختیار کرتا ہے۔ایک برقرار حالت سے دوسری برقرار حالت تک پہنچنے کے دوران، دور  \اصطلاح{عارضی حالت}\فرہنگ{عارضی حالت}\حاشیہب{transient state}\فرہنگ{transient state} میں ہوتا ہے۔

\حصہ{ایک درجی ادوار}
وہ ادوار جن میں صرف امالہ گیر توانائی ذخیرہ کرتے ہوں کی کرخوف مساوات \اصطلاح{ایک درجی تفرقی مساوات}\فرہنگ{تفرقی مساوات!ایک درجی}\فرہنگ{ایک درجی!تفرقی مساوات}\حاشیہب{first order differential equation}\فرہنگ{differential equation!first order} ہوتی ہے۔اسی طرح وہ ادوار جن میں صرف برق گیر توانائی ذخیرہ کرتے ہوں بھی ایک درجی کرخوف مساوات  دیتے ہیں۔اسی لئے انہیں \اصطلاح{یک درجی ادوار}\فرہنگ{یک درجی ادوار}\فرہنگ{دور!یک درجی}\حاشیہب{first order circuits}\فرہنگ{first order!circuits} کہتے ہیں۔اس کے برعکس ایسے ادوار جن میں امالہ گیر اور برق گیر دونوں پائے جاتے ہوں \اصطلاح{دو درجی تفرقی مساوات}\فرہنگ{دو درجی!تفرقی مساوات}\فرہنگ{تفرقی مساوات!دو درجی}\حاشیہب{second order differential equations}\فرہنگ{differential equation!second order} دیتے ہیں اور انہیں \اصطلاح{دو درجی ادوار}\فرہنگ{دو درجی!ادوار}\فرہنگ{دور!دو درجی}\حاشیہب{second order circuits}\فرہنگ{second order!circuits} کہا جاتا ہے۔

\begin{figure}
\centering
\begin{subfigure}{0.5\textwidth}
\centering
\begin{tikzpicture}
\draw(0,0) to [american voltage source,l={$v_i(t)$}]++(0,\y) to [resistor,i>^={$i(t)$},l={$R$}]++(\x,0) to [inductor,l={$L$}]++(0,-\y) to [short]++(-\x,0);
\end{tikzpicture}
\caption*{(الف)}
\end{subfigure}%
\begin{subfigure}{0.5\textwidth}
\centering
\begin{tikzpicture}
\draw(0,0) to [american current source,l={$i_i(t)$}]++(0,\y) to [short]++(2*\x,0) to [capacitor,i={$i_C(t)$},l_={$C$}]++(0,-\y) to [short]++(-2*\x,0);
\draw(\x,0)node[ground]{} to [resistor,*-*,i<_={$i_R(t)$},l={$R$}]++(0,\y)node[above]{$v(t)$};
\end{tikzpicture}
\caption*{(ب)}
\end{subfigure}%
\caption{ایک درجی ادوار کی مثالیں۔}
\label{شکل_عارضی_ایک_درجی_ادوار_الف}
\end{figure}

شکل \حوالہ{شکل_عارضی_ایک_درجی_ادوار_الف} میں ایک درجی ادوار کی مثالیں دی گئی ہیں۔آئیں ان کی کرخوف مساوات لکھ کر دیکھیں۔شکل-الف کی مساوات درج ذیل ہے۔
\begin{align}
v(t)=i(t)R+L \frac{\dif i(t)}{\dif t}
\end{align}
اسی طرح شکل-ب کی کرخوف مساوات درج ذیل ہے۔
\begin{align}
i_i(t)=\frac{v(t)}{R}+C\frac{\dif v(t)}{\dif t}
\end{align}
آپ دیکھ سکتے ہیں کہ درج بالا دونوں مساوات ایک درجی تفرقی مساوات ہیں۔

\begin{figure}
\centering
\begin{tikzpicture}
\draw(0,0) to [american voltage source,l={$v_i(t)$}]++(0,\y) to [resistor,i={$i(t)$},l={$R$}]++(\x,0) to [inductor,l={$L$}]++(\x,0) to [capacitor,l={$C$}]++(0,-\y) to [short]++(-2*\x,0);
\end{tikzpicture}
\caption{دو درجی دور۔}
\label{شکل_عارضی_دور_درجی_دور_الف}
\end{figure}

شکل \حوالہ{شکل_عارضی_دور_درجی_دور_الف} میں دو درجی دور دکھایا گیا ہے جس کی کرخوف مساوات درج ذیل ہے۔
\begin{align*}
v_i(t)=R i(t)+L\frac{\dif i(t)}{\dif t}+\frac{1}{C} \int_{-\infty}^{t}i(t) \dif t
\end{align*}
اس مساوات میں تکمل کی علامت ختم کرنے سے تفرقی مساوات حاصل ہو گی۔تکمل کی علامت ختم کرنے کی خاطر اس کا تفرق لیتے ہیں۔
\begin{align}
\frac{\dif v_i(t)}{\dif t}=R \frac{\dif i(t)}{\dif t}+L\frac{\dif^{\,2} i(t)}{\dif t^2}+\frac{i(t)}{C}
\end{align} 
آپ دیکھ سکتے ہیں کہ امالہ گیر اور برق گیر دونوں کی موجودگی سے دو درجی تفرقی مساوات حاصل ہوتی ہے۔

\جزوحصہ{رد عمل کی عمومی مساوات}
ایک درجی ادوار کے رد عمل جاننے کی خاطر ان کی تفرقی مساوات حل کی جاتی ہے جس سے دور کے مختلف مقامات پر دباو اور رو حاصل کی جاتی ہے۔ان یک درجی مساوات کی عمومی صورت درج ذیل ہوتی ہے
\begin{align}\label{مساوات_عارضی_یک_درجی_عمومی_مساوات}
\frac{\dif y(t)}{\dif t}+a y(t)= g(t)
\end{align}
جہاں \عددی{y(t)} دباو یا رو کو ظاہر کرتی ہے، \عددی{a} مستقل ہے اور \عددی{g(t)} \اصطلاح{تفاعل عملی}\فرہنگ{تفاعل عملی}\حاشیہب{forcing function}\فرہنگ{forcing function} ہے۔اس مساوات کا آزاد متغیرہ وقت \عددی{t} ہے۔تفرقی مساوات کا ایک بنیادی مسئلہ کہتا ہے کہ مساوات \حوالہ{مساوات_عارضی_یک_درجی_عمومی_مساوات} کا مکمل حل اس کے \اصطلاح{فطری رد عمل}\فرہنگ{فطری رد عمل}\فرہنگ{رد عمل!فطری}\حاشیہب{natural response, complementary solution}\فرہنگ{complementary solution}\فرہنگ{natural response}\فرہنگ{response!natural} \عددی{y_f(t)} اور \اصطلاح{جبری رد عمل}\فرہنگ{جبری رد عمل}\فرہنگ{رد عمل!جبری}\حاشیہب{forced response, particular solution}\فرہنگ{particular solution}\فرہنگ{forced response} 
\عددی{y_j(t)} کا مجموعہ ہے۔مساوات \حوالہ{مساوات_عارضی_یک_درجی_عمومی_مساوات} کے کسی بھی حل کو بطور جبری رد عمل لیا جا سکتا ہے جبکہ درج ذیل \اصطلاح{ہم جنسی مساوات}\فرہنگ{ہم جنسی مساوات}\فرہنگ{مساوات!ہم جنسی}\حاشیہب{homogenous equation}\فرہنگ{homogenous equation}
\begin{align}\label{مساوات_عارضی_یک_درجی_عمومی_مساوات_ب}
\frac{\dif y(t)}{\dif t}+a y(t)=0
\end{align}
 کے کسی بھی حل کو فطری رد عمل تصور کیا جا سکتا ہے۔مساوات \حوالہ{مساوات_عارضی_یک_درجی_عمومی_مساوات} میں \عددی{g(t)=0} پُر کرنے سے ہم جنسی مساوات  حاصل ہوتی ہے۔

آئیں \عددی{g(t)=A} کی صورت میں مساوات \حوالہ{مساوات_عارضی_یک_درجی_عمومی_مساوات} کا حل حاصل کریں جہاں \عددی{A} ایک مستقل ہے۔یوں ہمیں درج ذیل دو مساوات کے حل درکار ہیں۔
\begin{align}
\frac{\dif y_j(t)}{\dif t}+a y_j(t)&=A \label{مساوات_عارضی_یک_درجی_عمومی_مساوات_پ}\\
\frac{\dif y_f(t)}{\dif t}+a y_f(t)&=0\label{مساوات_عارضی_یک_درجی_عمومی_مساوات_ت}
\end{align}
جبری حل کو قیاس کے ذریعہ حاصل کیا جائے گا۔  جبری حل کو تفاعل عملی اور اس کے تمام ممکنہ تفرق کے مجموعے کے برابر تصور کرتے ہوئے آگے بڑھتے ہیں۔چونکہ مستقل کا تفرق \عددی{(\tfrac{\dif A}{\dif t}=0)} صفر کے برابر ہے لہٰذا جبری حل کو مستقل \عددی{K_1} تصور کرتے ہیں۔
\begin{align}
y_j(t)=K_1
\end{align}
اس قیمت کو مساوات \حوالہ{مساوات_عارضی_یک_درجی_عمومی_مساوات_پ} میں پُر کرتے ہوئے حل کرنے سے
\begin{align*}
\frac{\dif K_1}{\dif t}+a K_1&=A \\
0+a K_1&=A
\end{align*}
یعنی
\begin{align}\label{مساوات_عارضی_یک_درجی_عمومی_مساوات_ٹ}
K_1=\frac{A}{a}
\end{align}
حاصل ہوتا ہے۔مساوات \حوالہ{مساوات_عارضی_یک_درجی_عمومی_مساوات_ت} کو ترتیب دیتے ہوئے
\begin{align*}
\frac{\dif y_f(t)}{y_f(t)}=-a \dif t
\end{align*}
لکھا جا سکتا ہے  جس کا تکمل
\begin{align*}
\ln y_f(t)=-a t +c
\end{align*}
یعنی
\begin{align}\label{مساوات_عارضی_یک_درجی_عمومی_مساوات_ث}
y_f(t)=K_2e^{-at}
\end{align}
کے برابر ہے جہاں \عددی{c} تکمل کا مستقل ہے اور \عددی{K_2=e^{c}} کے برابر ہے۔مساوات \حوالہ{مساوات_عارضی_یک_درجی_عمومی_مساوات_ٹ} اور مساوات \حوالہ{مساوات_عارضی_یک_درجی_عمومی_مساوات_ث} سے مکمل حل درج ذیل حاصل ہوتا ہے۔
\begin{align}
y(t)=\frac{A}{a}+K_2 e^{-at}
\end{align}
کسی بھی لمحے پر \عددی{y(t)} جاننے سے درج بالا مساوات میں نا معلوم مستقل \عددی{K_2} دریافت کیا جا سکتا ہے۔درج بالا مساوات کو درج ذیل عمومی حل کی صورت میں لکھا جا سکتا ہے
\begin{align}\label{مساوات_عارضی_یک_درجی_عمومی_مساوات_ج}
y(t)=K_1+K_2 e^{-\frac{t}{\tau}}
\end{align}
جہاں \عددی{\tau=\tfrac{1}{a}} کے برابر ہے۔


مساوات \حوالہ{مساوات_عارضی_یک_درجی_عمومی_مساوات_ج} کے مختلف اجزاء کو نام دیے گئے ہیں۔یوں \عددی{\tau} \اصطلاح{وقتی مستقل}\فرہنگ{وقتی مستقل}\حاشیہب{time constant}\فرہنگ{time constant} کہلاتا ہے جبکہ \عددی{K_1} \اصطلاح{برقرار حالت حل}\فرہنگ{برقرار حالت حل}\فرہنگ{حل:برقرار حالت}\فرہنگ{برقرار حالت:حل}\حاشیہب{steady state solution}\فرہنگ{steady state solution} کہلاتا ہے۔مساوات \حوالہ{مساوات_عارضی_یک_درجی_عمومی_مساوات_ج} میں \عددی{t=\infty} پُر کرنے سے برقرار حالت حل حاصل ہوتا ہے۔یوں کسی بھی تبدیلی کے بہت دیر بعد دور برقرار حالت میں ہو گا یعنی ابدی صورت کو برقرار حالت کہا جاتا ہے۔

\begin{figure}
\begin{tikzpicture}
\begin{axis}[name=ka,axis lines*=middle,
	 every axis x label/.style={
    at={(ticklabel* cs:1.05)},
    anchor=east,}, 
	every axis y label/.style={
    at={(ticklabel* cs:1.05)},
    anchor=east,}
,xlabel=$t$,ylabel=$K_2 e^{-\frac{t}{\tau}}$,ytick={1,0.368},yticklabel style={/pgf/number format/precision=3},yticklabels={$K_2$,$0.368 K_2$},xtick={0.5,1,1.5,2,2.5},xticklabels={$\tau$,$2\tau$,$3\tau$,$4\tau$,$5\tau$}]
\addplot[width=4cm,domain=0:3,samples=100]{e^(-x/0.5)};
\draw[dashed](axis cs:0,1)--(axis cs:0.5,0);
\draw[dashed](axis cs:0,0.368)--(axis cs:0.5,0.368)--(axis cs:0.5,0);
\end{axis}
\node [anchor=north] at (ka.south){(الف)};
\end{tikzpicture}%
\begin{tikzpicture}
\begin{axis}[name=kb,axis lines*=middle,
 every axis x label/.style={
    at={(ticklabel* cs:1.05)},
    anchor=east,}, 
	every axis y label/.style={
    at={(ticklabel* cs:1.05)},
    anchor=east,},
 xlabel=$t$,ylabel=$e^{-\frac{t}{\tau}}$]
\addplot[width=4cm,domain=0:3,samples=100]{e^(-x/0.5)}node[pos=0.25,above right]{$\tau=0.5$};
\addplot[width=4cm,domain=0:3,samples=100]{e^(-x/2)}node[pos=0.25,above right]{$\tau=2$};
\end{axis}
\node[anchor=north] at (ka.south){(ب)};
\end{tikzpicture}
\caption{وقتی مستقل}
\label{شکل_عارضی_وقتی_مستقل_الف}
\end{figure}

شکل \حوالہ{شکل_عارضی_وقتی_مستقل_الف}-الف میں مثبت \عددی{a} کی صورت میں جبری حل دکھایا گیا ہے۔ابتدائی لمحہ \عددی{t=0} پر \عددی{y_j(0)=K_2} کے برابر ہے جبکہ ایک وقتی مستقل برابر وقت بعد اس کی قیمت \عددی{y_j(\tau)=0.368K_2} رہ گئی ہے یعنی \عددی{\tau} دورانیے میں جبری حل کی قیمت میں \عددی{\SI{63.2}{\percent}} کمی واقع ہوئی ہے۔اسی طرح دو وقتی مستقل وقفے کے بعد \عددی{y_j(2\tau)=0.135K_2} ہے جو \عددی{y_p(\tau)} کے \عددی{0.368} گنا ہے۔حقیقت میں کسی بھی لمحہ \عددی{t_1} پر \عددی{y_j} کی قیمت میں لمحہ \عددی{t_1+\tau} پر \عددی{\SI{63.2}{\percent}} کمی واقع ہو گی۔پانچ وقتی مستقل وقفے کے بعد \عددی{y_j(5\tau)=0.0067K_2} رہ جاتا ہے جو ابتدائی قیمت کے \عددی{\SI{0.67}{\percent}} ہے۔

مساوات \حوالہ{مساوات_عارضی_یک_درجی_عمومی_مساوات_ث}  \اصطلاح{قوت نمائی انحطاطی}\فرہنگ{قوت نمائی!انحطاط}\حاشیہب{exponential decaying}\فرہنگ{exponential decay} خط ہے۔قوت نمائی انحطاطی خط کی ایک خصوصیت یہ ہے کہ ابتدائی لمحے  پر اس کا مماس افقی محور کو \عددی{\tau} پر کاٹتا ہے۔اس مماس کو شکل \حوالہ{شکل_عارضی_وقتی_مستقل_الف}-الف میں \عددی{(0,K_2)} تا \عددی{(\tau,0)} نقطہ دار لکیر سے دکھایا گیا ہے۔ شکل \حوالہ{شکل_عارضی_وقتی_مستقل_الف}-ب میں مختلف \عددی{\tau} کی قیمتوں کے لئے مساوات \حوالہ{مساوات_عارضی_یک_درجی_عمومی_مساوات_ث}  کو کھینچا گیا ہے۔آپ دیکھ سکتے ہیں کہ کم وقتی مستقل کا خط جلد اختتامی قیمت تک پہنچتا ہے۔یوں وقتی مستقل کسی بھی دور کے رد عمل کے دورانیے کی ناپ ہے۔
%=======================

\ابتدا{مثال}\شناخت{مثال_عارضی_یک_درجی_دور_الف}
شکل \حوالہ{شکل_عارضی_سلسلہ_وار_مزاحمت_برق_گیر_الف} میں مزاحمت اور بے بار برق گیر سلسلہ وار جڑے ہیں۔لمحہ \عددی{t=0} پر \اصطلاح{سوئچ}\فرہنگ{سوئچ}\حاشیہد{اس طرز کے سوئچ کا پورا نام ایک قطب ایک چال سوئچ ہے۔}\حاشیہب{switch, spst, single pole single throw}\فرہنگ{switch} چالو کرتے ہوئے انہیں مستقل منبع دباو \عددی{V_I} کے ساتھ جوڑا جاتا ہے۔برق گیر کا دباو \عددی{v(t)} اور رو \عددی{i(t)} دریافت کریں۔

\begin{figure}
\centering
\begin{tikzpicture}
\draw(0,0) to [american voltage source,l={$V_I$}]++(0,\y) to [cspst,l={${t=0}$}]++(\x,0) to [resistor,l={$R$}]++(\x,0)node[above]{$v(t)$} to [capacitor,-*,l={$C$}]++(0,-\y) node[ground]{} to [short] (0,0);
\draw(\x/2+\dx,\y-\dy)node[below]{سوئچ};
\draw(\x,-0.5)node{(الف)};
\end{tikzpicture}%
\begin{tikzpicture}
\begin{axis}[name=kb,axis lines*=middle,
	 every axis x label/.style={
    at={(ticklabel* cs:1.05)},
    anchor=east,}, 
	every axis y label/.style={
    at={(ticklabel* cs:1.05)},
    anchor=east,}
,xlabel=$t$,ylabel=$v(t)$,ytick={1,0.5},yticklabel style={/pgf/number format/precision=3},yticklabels={$V_I$,$0.5 V_I$},xtick={1,2,3,4,5},xticklabels={$\tau$,$2\tau$,$3\tau$,$4\tau$,$5\tau$}]
\addplot[width=4cm,domain=0:5,samples=100]{1-e^(-x)}node[pos=0.3,below right]{$v(t)=V_I \left(1-e^{-\frac{t}{RC}}\right)$};
\end{axis}%
\node [anchor=north] at (kb.south){(ب)};
\end{tikzpicture}%
\begin{tikzpicture}
\begin{axis}[name=kc,axis lines*=middle,
	 every axis x label/.style={
    at={(ticklabel* cs:1.05)},
    anchor=east,}, 
	every axis y label/.style={
    at={(ticklabel* cs:1.05)},
    anchor=east,}
,xlabel=$t$,ylabel=$i(t)$,ytick={1,0.5},yticklabel style={/pgf/number format/precision=3},yticklabels={$\frac{V_I}{R}$,$0.5\frac{V_I}{R}$},xtick={1,2,3,4,5},xticklabels={$\tau$,$2\tau$,$3\tau$,$4\tau$,$5\tau$}]
\addplot[width=4cm,domain=0:5,samples=100]{e^(-x)}node[pos=0.3,above right]{$i(t)=\frac{V_I}{R}e^{-\frac{t}{RC}}$};
\end{axis}%
\node [anchor=north] at (kc.south){(پ)};
\end{tikzpicture}%
\caption{مثال \حوالہ{مثال_عارضی_یک_درجی_دور_الف} کا دور، دباو اور رو۔}
\label{شکل_عارضی_سلسلہ_وار_مزاحمت_برق_گیر_الف}
\end{figure} 

حل:سوئچ چالو کرنے سے پہلے برق گیر بے بار ہے لہٰذا اس پر دباو صفر کے برابر ہے۔صفحہ \حوالہصفحہ{مساوات_امالہ_برق_گیر_دباو_بلا_جوڑ_ہے} پر مساوات \حوالہ{مساوات_امالہ_برق_گیر_دباو_بلا_جوڑ_ہے} کے تحت \عددی{v_C(0_+)=v_C(0_-)} ہو گا یعنی یوں سوئچ چالو کرنے کے فوراً بعد برق گیر کا دباو صفر ہی ہو گا۔سوئچ چالو کرنے کے بعد  دباو جوڑ \عددی{v(t)} کے استعمال سے کرخوف مساوات رو لکھتے ہیں
\begin{align*}
\frac{v(t)-V_I}{R}+C\frac{\dif v(t)}{\dif t}=0
\end{align*}
جسے ترتیب دیتے ہوئے
\begin{align}\label{مساوات_عارضی_برق_گیر_عارضی_حل_الف}
\frac{\dif v(t)}{\dif t}+\frac{v(t)}{RC}=\frac{V_I}{RC}
\end{align}
لکھا  جا سکتا ہے جو عمومی مساوات \حوالہ{مساوات_عارضی_یک_درجی_عمومی_مساوات} کی طرح ہے۔چونکہ \عددی{V_I} مستقل قیمت ہے لہٰذا اس مساوات کا جبری حل
\begin{align*}
v_j(t)=K_1
\end{align*}
 تصور کیا جا سکتا ہے جسے  مساوات \حوالہ{مساوات_عارضی_برق_گیر_عارضی_حل_الف} میں پُر کرتے ہوئے حل کرنے سے
\begin{align*}
\frac{\dif  K_1}{\dif t}+\frac{K_1}{RC}&=\frac{V_I}{RC}\\
0+\frac{K_1}{RC}&=\frac{V_I}{RC}
\end{align*}
یعنی
\begin{align*}
K_1=V_I
\end{align*}
حاصل ہوتا ہے۔یوں جبری حل درج ذیل حاصل ہوتا ہے۔
\begin{align*}
v_j(t)=V_I
\end{align*}
اس نتیجے کے تحت سوئچ چالو کرنے کے بہت دیر بعد برق گیر پر دباو عین منبع دباو کے برابر ہو گا۔شکل کو دیکھتے ہوئے اسی نتیجے تک یوں پہنچا جا سکتا ہے کہ سوئچ چالو کرنے کے بعد دور میں رو کی وجہ سے برق گیر پر بار جمع ہونا شروع ہو جائے گا۔جب تک برق گیر کا دباو منبع کے دباو سے کم ہو، مزاحمت پر دباو پایا جائے گا لہٰذا اس میں رو پائی جائے گی۔یہ رو برق گیر پر جمع بار میں اضافہ کرتی رہے گی۔عین اس وقت جب برق گیر اور منبع کے دباو برابر ہو جائیں، رو کی قیمت صفر ہو جائے گی اور برق گیر کا دباو اسی قیمت پر ابد تک برقرار رہے گا۔ 

آئیں اب فطری حل دریافت کریں۔فطری حل ہم جنسی مساوات سے حاصل ہوتا ہے۔مساوات \حوالہ{مساوات_عارضی_برق_گیر_عارضی_حل_الف} کے دائیں بازو کو صفر کے برابر پُر کرنے سے ہم جنسی مساوات
\begin{align}\label{مساوات_عارضی_برق_گیر_ہم_جنسی_الف}
\frac{\dif v(t)}{\dif t}+\frac{v(t)}{RC}=0
\end{align}
 حاصل ہوتی ہے۔اس کو
\begin{align*}
\frac{\dif v(t)}{v(t)}=-\frac{\dif t}{RC}
\end{align*}
لکھتے ہوئے تکمل لینے سے
\begin{align*}
\ln v(t)=-\frac{t}{RC}+c
\end{align*}
یعنی
\begin{align*}
v_f(t)=K_2 e^{-\frac{t}{RC}}
\end{align*}
فطری حل حاصل ہوتا ہے۔ جبری اور فطری حل کا مجموعہ مکمل حل ہو گا۔
\begin{align*}
v(t)=V_I+K_2 e^{-\frac{t}{RC}}
\end{align*}
مکمل حل میں نا معلوم مستقل کو \اصطلاح{ابتدائی شرائط}\فرہنگ{ابتدائی شرائط}\حاشیہب{initial conditions}\فرہنگ{initial conditions} سے حاصل کرتے ہیں جس کے تحت \عددی{t=0_+} پر \عددی{v_C(0_+)=0} کی قیمت معلوم ہے۔ان قیمتوں کو درج بالا مساوات میں پُر کرتے ہوئے حل کرنے سے
\begin{align*}
0&=V_I+K_2 e^{-\frac{0}{RC}}\\
0&=V_I+K_2
\end{align*}
یعنی
\begin{align*}
K_2=-V_I
\end{align*}
حاصل ہوتا ہے۔ 

جبری حل اور فطری حل کا مجموعہ مکمل حل دیتا ہے
\begin{gather}
\begin{aligned}
v(t)&=v_j(t)+v_f(t)\\
&=V_I\left(1-e^{-\frac{t}{RC}}\right)\\
&=V_I\left(1-e^{-\frac{t}{\tau}}\right)
\end{aligned}
\end{gather}
درج بالا مساوات میں وقتی مستقل درج ذیل ہے۔
\begin{align}
\tau=RC
\end{align}
یوں \عددی{R} یا (اور) \عددی{C} بڑھانے سے  وقتی مستقل بڑھے گا جس سے دور برقرار صورت زیادہ دیر کے بعد اختیار کرے گا۔ 

رو \عددی{i(t)} کو درج بالا مساوات سے حاصل کرتے ہیں۔
\begin{align*}
i(t)&=C\frac{\dif v(t)}{\dif t}\\
&=C V_I \left(0+\frac{1}{RC}e^{-\frac{t}{RC}}\right)\\
&=\frac{V_I}{R}e^{-\frac{t}{RC}}
\end{align*}
یہی رو مزاحمت پر اوہم کے قانون کی مدد سے بھی حاصل کی جا سکتی ہے یعنی
\begin{align*}
i(t)&=\frac{V_I-v(t)}{R}\\
&=\frac{V_I}{R}e^{-\frac{t}{RC}}
\end{align*} 
\انتہا{مثال}
%===============
\ابتدا{مثال}\شناخت{مثال_عارضی_مزاحمت_امالہ_عارضی_الف}
شکل \حوالہ{شکل_عارضی_مزاحمت_امالہ_الف} میں لمحہ \عددی{t=0} پر سوئچ چالو کیا جاتا ہے۔رو کا خط کھینچیں۔
\begin{figure}
\centering
\begin{tikzpicture}
\draw(0,0) to [american voltage source,l={$V_I$}]++(0,\y) to [cspst,l={${t=0}$}]++(\x,0) to [resistor,i>_={$i(t)$},l={$R$}]++(\x,0) to [inductor,-*,l={$L$}]++(0,-\y)node[ground]{} to [short] (0,0);
\end{tikzpicture}%
\begin{tikzpicture}
\begin{axis}[name=kb,xlabel=$t$,ylabel=$i(t)$,ytick={1,0.5},yticklabels={$\frac{V_I}{R}$,$0.5\frac{V_I}{R}$},xtick={1,2,3,4,5},xlabel={$t$},xticklabels={$\tau$,$2\tau$,$3\tau$,$4\tau$,$5\tau$},
axis lines*=middle,
 every axis x label/.style={
    at={(ticklabel* cs:1.05)},
    anchor=east,}, 
	every axis y label/.style={
    at={(ticklabel* cs:1.05)},
    anchor=east,}
]
\addplot[domain=0:5]{1-e^(-x)};
\end{axis}%
\end{tikzpicture}%
\caption{مثال \حوالہ{مثال_عارضی_مزاحمت_امالہ_عارضی_الف} کے اشکال۔}
\label{شکل_عارضی_مزاحمت_امالہ_الف}
\end{figure}

حل:کرخوف مساوات دباو
\begin{align*}
V_I=i(t)R+L\frac{\dif i(t)}{\dif t}
\end{align*}
کو ترتیب دیتے ہوئے عمومی شکل میں لاتے ہیں
\begin{align}\label{مساوات_عارضی_امالہ_گیر_عمومی_الف}
\frac{\dif i(t)}{\dif t}+\frac{R}{L} i(t)=\frac{V_I}{L}
\end{align}
جس کا جبری حل
\begin{align*}
i_j(t)=K_1
\end{align*}
ہو گا۔جبری حل کو عمومی مساوات میں پُر کرتے ہوئے حل کرنے سے
\begin{align*}
\frac{\dif K_1}{\dif t}+\frac{R}{L} K_1&=\frac{V_I}{L}\\
0+\frac{R}{L} K_1&=\frac{V_I}{L}
\end{align*}
یعنی
\begin{align*}
K_1=\frac{V_I}{R}
\end{align*}
حاصل ہوتا ہے جس سے جبری حل درج ذیل لکھا جائے گا۔
\begin{align*}
i_j(t)=\frac{V_I}{R}
\end{align*}
یہی جواب منطق سے بھی حاصل کیا جا سکتا ہے۔چونکہ  یک سمتی رو کے لئے امالہ گیر بطور قصر دور کردار ادا کرتا ہے لہٰذا عارضی دورانیہ گزر جانے کے بعد ہم امالہ گیر کو قصر دور تصور کر سکتے ہیں۔شکل \حوالہ{شکل_عارضی_مزاحمت_امالہ_الف} میں امالہ گیر کو قصر دور کرتے ہوئے اوہم کے قانون سے \عددی{i_j(t)=\tfrac{V_I}{R}} لکھا جا سکتا ہے۔

فطری حل حاصل کرنے کی خاطر مساوات \حوالہ{مساوات_عارضی_امالہ_گیر_عمومی_الف} میں دیے گئے عمومی مساوات کا دایاں ہاتھ صفر کے برابر پُر کرتے ہوئے درج ذیل ہم جنسی مساوات حاصل کرتے ہیں۔
\begin{align*}
\frac{\dif i(t)}{\dif t}+\frac{R}{L} i(t)=0
\end{align*}
اس کو ترتیب دیتے ہوئے
\begin{align*}
\frac{\dif i(t)}{i(t)}=-\frac{R}{L} \dif t
\end{align*}
تکمل لینے سے
\begin{align*}
\ln i(t)=-\frac{R}{L}t +c
\end{align*}
یعنی
\begin{align*}
i_f(t)=K_2e^{-\frac{R}{L}t}
\end{align*}
حاصل ہوتا ہے۔

جبری اور فطری حل کا مجموعہ مکمل حل دیتا ہے
\begin{gather}
\begin{aligned}\label{مساوات_عارضی_مزاحمت_امالہ_مکمل_حل_الف}
i(t)&=i_j(t)+i_f(t)\\
&=\frac{V_I}{R}+K_2e^{-\frac{R}{L}t}\\
&=\frac{V_I}{R}+K_2e^{-\frac{t}{\tau}}
\end{aligned}
\end{gather}
جہاں وقتی مستقل درج ذیل ہے۔
\begin{align}
\tau=\frac{R}{L}
\end{align}
مکمل حل میں نا معلوم مستقل \عددی{K_2} کو ابتدائی معلومات سے حاصل کیا جا سکتا ہے۔سوئچ چالو کرنے سے پہلے دور میں رو صفر کے برابر ہے۔صفحہ \حوالہصفحہ{مساوات_امالہ_امالہ_گیر_رو_بلا_جوڑ_ہے} پر مساوات \حوالہ{مساوات_امالہ_امالہ_گیر_رو_بلا_جوڑ_ہے} کے تحت امالہ کی رو بلا جوڑ تفاعل
\begin{align*}
i_L(t_+)=i_L(t_-)
\end{align*} 
ہے لہٰذا سوئچ چالو کرنے کے فوراً بعد امالہ کی رو وہی ہو گی جو سوئچ چالو کرنے کے فوراً پہلے تھی یعنی لمحہ \عددی{t=0_+} پر \عددی{i_L(0_+)=i_L(0_-)=0} ہو گی۔ان معلومات کو مساوات \حوالہ{مساوات_عارضی_مزاحمت_امالہ_مکمل_حل_الف} میں دیے مکمل حل میں پُر کرنے سے
\begin{align*}
0&=\frac{V_I}{R}+K_2e^{-\frac{0}{\tau}}
\end{align*}
یعنی
\begin{align*}
K_2=-\frac{V_I}{R}
\end{align*}
حاصل ہوتا ہے۔یوں مکمل حل درج ذیل لکھا جائے گا۔
\begin{align}
i(t)=\frac{V_I}{R} \left(1-e^{-\frac{t}{\tau}}\right)
\end{align}
رو کے خط کو شکل \حوالہ{شکل_عارضی_مزاحمت_امالہ_الف}-ب میں دکھایا گیا ہے۔
\انتہا{مثال}
%=====================
\ابتدا{مثال}\شناخت{مثال_عارضی_دو_عدد_مزاحمت_برق_گیر_الف}
ازل سے شکل \حوالہ{شکل_عارضی_دو_عدد_مزاحمت_برق_گیر_الف} میں \اصطلاح{ایک قطب دو چال سوئچ}\فرہنگ{سوئچ!ایک قطب دو چال}\حاشیہب{single pole double throw switch, spdt}\فرہنگ{switch!spdt} اسی جگہ پر ہے۔لمحہ \عددی{t=0} پر اس کی جگہ تبدیل کرتے ہوئے \عددی{\SI{5}{\kilo\ohm}} مزاحمت کو زمین کے ساتھ جوڑا جاتا ہے۔برق گیر پر دباو دریافت کریں۔
\begin{figure}
\centering
\begin{tikzpicture}
\draw(0,0) to [american voltage source,l={$\SI{20}{\volt}$}]++(0,\y)coordinate(kP)++(\x,0) node[spdt,xscale=-1](kspdt){};
\draw[name path=kkk](kspdt.in) to [resistor,l={$\SI{5}{\kilo\ohm}$}]++(\x,0) to [short]++(\x,0) to [capacitor,l={$\SI{200}{\micro\farad}$}]++(0,-\y)coordinate(kbot)-|(0,0);
\draw(kP)|-(kspdt.out 1);
\draw(kbot)++(-\x,0)node[ground]{} to [resistor,*-*,l={$\SI{15}{\kilo\ohm}$}]++(0,\y);
\path[name path=kvert](kspdt.out 2)--++(0,-\y);
\draw[name intersections={of=kvert and kkk}] (kspdt.out 2) to [short,-*](intersection-1);
\draw(\x,-\y/4)node{(الف)};
\end{tikzpicture}%
\begin{tikzpicture}
\draw(0,0) to [american voltage source,l={$\SI{20}{\volt}$}]++(0,\y)coordinate(kP)++(\x,0) node[spdt,xscale=-1](kspdt){};
\draw[](kspdt.in) to [resistor,l={$\SI{5}{\kilo\ohm}$}]++(\x,0) to [short]++(\x,0) to [short,-o]++(0,-\y/8)++(0,-6/8*\y) to [short,o-]++(0,-\y/8)coordinate(kbot);
\draw[name path=kkk](0,0)--++(kbot);
\draw(kP)|-(kspdt.out 1);
\draw(kbot)++(-\x,0)node[ground]{} to [resistor,*-*,l={$\SI{15}{\kilo\ohm}$}]++(0,\y);
\path[name path=kvert](kspdt.out 2)--++(0,-\y);
\draw[name intersections={of=kvert and kkk}] (kspdt.out 2) to [short,-*](intersection-1);
\draw(kbot)++(0,\y/2)node{$\begin{aligned}&+ \\ &v_C \\ &-  \end{aligned}$};
\draw(\x,-\y/4)node{(ب)};
\end{tikzpicture}%
\begin{tikzpicture}
\draw(0,0) to [american voltage source,l={$\SI{20}{\volt}$}]++(0,\y)coordinate(kP)++(\x,0) node[spdt,xscale=-1,yscale=-1](kspdt){};
\draw[name path=kkk](kspdt.in) to [resistor,l={$\SI{5}{\kilo\ohm}$}]++(\x,0) to [short]++(\x,0)node[above]{$v_C(t)$} to [capacitor,l={$\SI{200}{\micro\farad}$}]++(0,-\y)coordinate(kbot)-|(0,0);
\draw(kP)|-(kspdt.out 2);
\draw(kbot)++(-\x,0)node[ground]{} to [resistor,*-*,l={$\SI{15}{\kilo\ohm}$}]++(0,\y);
\path[name path=kvert](kspdt.out 1)--++(0,-\y);
\draw[name intersections={of=kvert and kkk}] (kspdt.out 1) to [short,-*](intersection-1);
\draw(\x,-\y/4)node{(پ)};
\end{tikzpicture}%
\begin{tikzpicture}
\begin{axis}[axis lines*=middle,ytick={5,10},xlabel=$t$,ylabel=$v_C(t)$,
every axis x label/.style={
    at={(ticklabel* cs:1.05)},
    anchor=east,}, 
	every axis y label/.style={
    at={(ticklabel* cs:1.05)},
    anchor=east,}
]
\addplot[domain=0:6,samples=100]{15*e^(-4*x/3)};
\addplot[domain=-1:0,samples=10]{15}node[right]{$\SI{15}{\volt}$};
\end{axis}
\end{tikzpicture}
\caption{مثال \حوالہ{مثال_عارضی_دو_عدد_مزاحمت_برق_گیر_الف} کے اشکال۔}
\label{شکل_عارضی_دو_عدد_مزاحمت_برق_گیر_الف}
\end{figure}

حل:ازل سے دور منبع کے ساتھ جڑا رہا ہے۔یوں دور برقرار حالت میں ہو گا اور برق گیر کو کھلا دور تصور کیا جاتا ہے۔ایسا کرنے سے شکل-ب حاصل ہوتی ہے جہاں سے تقسیم دباو کے کلیے سے برق گیر کا ابتدائی دباو درج ذیل حاصل ہوتا ہے۔
\begin{align*}
v_C(0_-)=20\left(\frac{\SI{15}{\kilo\ohm}}{\SI{5}{\kilo\ohm}+\SI{15}{\kilo\ohm}}\right)=\SI{15}{\volt}
\end{align*}
برق گیر کا دباو بلا جوڑ ہے لہٰذا
\begin{align*}
v_C(0_+)=v_C(0_-)=\SI{15}{\volt} \quad \quad \text{\RL{ابتدائی حالت}}
\end{align*}
ہو گا۔لمحہ \عددیء{t=0} کے بعد کی صورت شکل-پ میں دکھائی گئی ہے۔ہمیں اس شکل میں \عددی{v(t)} درکار ہے جسے کرخوف مساوات رو کی مدد سے حاصل کرتے ہیں۔
\begin{align*}
\frac{v_C(t)}{5000}+\frac{v_C(t)}{15000}+200\times 10^{-6}\frac{\dif v_C(t)}{\dif t}=0
\end{align*}
اس ہم جنسی مساوات کو ترتیب دیتے ہوئے
\begin{align*}
\frac{\dif v_C(t)}{v_C(t)}=-\frac{4}{3}\dif t
\end{align*}
لکھا جا سکتا ہے جس کا تکمل
\begin{align*}
\ln v_C(t)=-\frac{4}{3}t+c
\end{align*}
یا
\begin{align*}
v_C(t)=Ke^{-\frac{4}{3}t}
\end{align*}
کے برابر ہے جہاں تکمل کے مستقل کو \عددی{c} یا \عددی{K} لکھا گیا ہے۔ابتدائی حالت کی معلومات اس مساوات میں پُر کرتے ہوئے 
\begin{align*}
15=Ke^{0}
\end{align*}
سے \عددی{K} کی قیمت درج ذیل
\begin{align*}
K=15
\end{align*}
حاصل ہوتی ہے۔یوں 
\begin{align*}
v_C(t)=15 e^{-\frac{4}{3}t}
\end{align*}
حاصل ہوتا ہے جس میں وقتی مستقل \عددی{\tau=\tfrac{3}{4}} کے برابر ہے۔یوں سوئچ چالو کرنے کے \عددی{\SI{0.75}{\second}} بعد برق گیر کا دباو ابتدائی قیمت کے \عددی{\SI{36.8}{\percent}} یعنی \عددی{0.368 \times 15=\SI{5.52}{\volt}} ہو گا۔ 
\انتہا{مثال}
%======================
\ابتدا{مثال}\شناخت{مثال_عارضی_دو_عدد_مزاحمت_امالہ_گیر_الف}
ازل سے شکل \حوالہ{شکل_عارضی_دو_عدد_مزاحمت_امالہ_گیر_الف} میں سوئچ غیر چالو تھا جسے \عددی{t=0} پر چالو کیا جاتا ہے۔امالہ گیر کی رو \عددی{i_L(t)} دریافت کریں۔

\begin{figure}
\centering
\begin{subfigure}{0.5\textwidth}
\centering
\begin{tikzpicture}
\draw(0,0) to [american voltage source,l={$\SI{24}{\volt}$}]++(0,\y) to [resistor,l={$\SI{1}{\kilo\ohm}$}]++(\x,0) to [cspst,l={${t=0}$}]++(\x,0) to [resistor,l={$\SI{1}{\kilo\ohm}$}]++(\x,0) to [inductor,i={$i_L(t)$},l={$\SI{1}{\milli\henry}$}]++(0,-\y) to [short] (0,0);
\draw(2*\x,0) to [american current source,*-*,l={$\SI{4}{\milli\ampere}$}]++(0,\y);
\end{tikzpicture}
\caption*{(الف)}
\end{subfigure}
\begin{subfigure}{0.5\textwidth}
\centering
\begin{tikzpicture}
\draw(0,0) to [american voltage source,l={$\SI{24}{\volt}$}]++(0,\y) to [resistor,l={$\SI{1}{\kilo\ohm}$}]++(\x,0) to [resistor,-o,l={$\SI{1}{\kilo\ohm}$}]++(\x,0);
\draw(0,0) to [short,-o]++(2*\x,0);
% to [inductor,i={$i_L(t)$},l={$\SI{1}{\milli\henry}$}]++(0,-\y) to [short] (0,0);
\draw(\x,0) to [american current source,*-*,l={$\SI{4}{\milli\ampere}$}]++(0,\y)node[above]{$v_t$};
\draw(2*\x,\y/2)node{$\begin{aligned} &+ \\ &v_{\text{تھونن}}\\ &- \end{aligned}$};
\end{tikzpicture}%
\caption*{(ب)}
\end{subfigure}%
\begin{subfigure}{0.5\textwidth}
\centering
\begin{tikzpicture}
\draw(0,0) to [short]++(0,\y) to [resistor,l={$\SI{1}{\kilo\ohm}$}]++(\x,0) to [resistor,-o,l={$\SI{1}{\kilo\ohm}$}]++(\x,0);
\draw(0,0) to [short,-o]++(2*\x,0);
% to [inductor,i={$i_L(t)$},l={$\SI{1}{\milli\henry}$}]++(0,-\y) to [short] (0,0);
%\draw(\x,0) to [american current source,*-*,l={$\SI{4}{\milli\ampere}$}]++(0,\y);
\draw[stealth-](2*\x,\y/2)--++(\x/4,0)--++(0,-\y/8)node[below]{$R_{\text{تھونن}}$};
\end{tikzpicture}%
\caption*{(پ)}
\end{subfigure}
\begin{subfigure}{0.5\textwidth}
\centering
\begin{tikzpicture}
\draw(0,0) to [american voltage source,l={$\SI{28}{\volt}$}]++(0,\y) to [resistor,l={$\SI{2}{\kilo\ohm}$}]++(\x,0) to [inductor,i={$i_L(t)$},l={$\SI{1}{\milli\henry}$}]++(0,-\y) --(0,0);
\end{tikzpicture}%
\caption*{(ت)}
\end{subfigure}%
\begin{subfigure}{0.5\textwidth}
\centering
\pgfplotsset{scaled y ticks=false, scaled x ticks=false}
\begin{tikzpicture}
\begin{axis}[axis lines*=middle,
ytick={0.014},yticklabels={$\SI{14}{\milli\ampere}$},xlabel=$t(\si{\micro\second})$,ylabel=$i_L(t)$,
xtick={-0.000001,0.000001,0.000002,0.000003,0.000004,0.000005}, xticklabels={$-1$,$1$,$2$,$3$,$4$,$5$},
every axis x label/.style={
    at={(ticklabel* cs:1.15)},
    anchor=east,}, 
	every axis y label/.style={
    at={(ticklabel* cs:1.05)},
    anchor=east,}
]
\addplot[domain=0:3*10^(-6),samples=20]{0.014-0.01*e^(-2000000*x)};
\addplot[domain=-1*10^(-6):0,samples=10]{0.004}node[right]{$\SI{4}{\milli\ampere}$};
\end{axis}
\end{tikzpicture}%
\caption*{(ٹ)}
\end{subfigure}%
\caption{مثال \حوالہ{مثال_عارضی_دو_عدد_مزاحمت_امالہ_گیر_الف} کے اشکال۔}
\label{شکل_عارضی_دو_عدد_مزاحمت_امالہ_گیر_الف}
\end{figure}

حل:غیر چالو سوئچ کی صورت میں منبع رو کی تمام رو امالہ گیر سے گزرتی ہے لہٰذا
\begin{align*}
i_L(0_-)=i_L(0_+)=\SI{4}{\milli\ampere}
\end{align*}
ہو گا۔اس دور کو مسئلہ تھونن کی مدد سے حل کرتے ہیں۔یوں امالہ کو بوجھ تصور کرتے ہوئے بقایا دور کا تھونن مساوی حاصل کرتے ہیں۔تھونن دباو حاصل کرنے کی خاطر بوجھ کو کھلے دور کیا جاتا ہے جس سے شکل \حوالہ{شکل_عارضی_دو_عدد_مزاحمت_امالہ_گیر_الف}-ب حاصل ہوتی ہے۔اس شکل میں منبع رو کی تمام رو بائیں مزاحمت اور منبع دباو سے گزرے گی لہٰذا مزاحمت پر \عددی{\SI{4}{\volt}} کا دباو ہو گا۔یوں
\begin{align*}
v_t=v_{\text{تھونن}}=\SI{24}{\volt}+\SI{4}{\volt}=\SI{28}{\volt}
\end{align*}
لکھا جا سکتا ہے۔یاد رہے کہ بالائی دائیں مزاحمت میں رو صفر کے برابر ہے لہٰذا اس پر دباو بھی صفر ہو گا اور یوں \عددی{v_t} اور \عددی{v_{\text{تھونن}}} برابر ہوں گے۔

منبع دباو کو قصر دور اور منبع رو کو کھلے دور کرتے ہوئے شکل-پ حاصل ہوتی ہے جسے دیکھتے ہوئے تھونن مزاحمت
\begin{align*}
R_{\text{تھونن}}=\SI{2}{\kilo\ohm}
\end{align*}
لکھی جا سکتی ہے۔

تھونن مساوی دور استعمال کرتے ہوئے شکل-الف کو شکل-ت کی طرز پر بنایا جا سکتا ہے۔شکل-ت کی کرخوف مساوات
\begin{align*}
28=2000 i(t)+0.001 \frac{\dif i(t)}{\dif t}
\end{align*}
کو عمومی صورت میں لکھتے ہیں۔
\begin{align*}
\frac{\dif i(t)}{\dif t}+2\times 10^6 i(t)=28000
\end{align*}
اس مساوات کا جبری حل 
\begin{align*}
i_j(t)=K_1=\SI{14}{\milli\ampere}
\end{align*}
حاصل ہوتا ہے اور اس کا فطری حل
\begin{align*}
i_f(t)=K_2 e^{-2\times 10^6 t}
\end{align*}
ہے-یوں امالہ گیر کے رو کا مکمل حل
\begin{align*}
i(t)=0.014+K_2 e^{-2\times 10^6 t}
\end{align*}
ہے۔ابتدائی معلومات کو اس مساوات میں حل کرتے ہوئے
\begin{align*}
0.004=0.014+K_2 e^{0}
\end{align*}
سے
\begin{align*}
K_2=\SI{-10}{\milli\ampere}
\end{align*}
حاصل ہوتا ہے۔یوں مکمل حل درج ذیل ہے۔
\begin{align}\label{مساوات_عارضی_امالہ_گیر_مکمل_حل_ب}
i_L(t)=0.014-0.01e^{-2\times 10^6 t}
\end{align} 
اس مساوات کا وقتی مستقل \عددی{\tau=\SI{0.5}{\micro\second}} ہے۔یوں تقریباً \عددی{5\tau=\SI{2.5}{\micro\second}} میں دور پہلی برقرار حالت سے دوسری برقرار حالت اختیار کر پاتا ہے۔مساوات \حوالہ{مساوات_عارضی_امالہ_گیر_مکمل_حل_ب} کو شکل-ٹ میں دکھایا گیا ہے۔
\انتہا{مثال}
%=======================
\ابتدا{مشق}\شناخت{مشق_عارضی_برق_گیر_الف}
شکل \حوالہ{شکل_عارضی_مشق_برق_گیر_الف} میں ازل سے چالو سوئچ کو  لمحہ \عددی{t=0} پر منقطع کیا جاتا ہے۔برق گیر پر ابتدائی دباو دریافت کرتے ہوئے \عددی{v_0(t)} دریافت کریں۔ اس دور کا وقتی مستقل کیا ہے۔
\begin{figure}
\centering
\begin{tikzpicture}
\draw(0,0) to [american voltage source,l={$\SI{24}{\volt}$}]++(0,\y) to [resistor,l={$\SI{2}{\kilo\ohm}$}]++(\x,0) to [ospst,l={${t=0}$}] ++(\x,0) to [resistor,l={$\SI{6}{\kilo\ohm}$}]++(\x,0) to [resistor,l_={$\SI{4}{\kilo\ohm}$}]++(0,-\y) to [short] (0,0);
\draw(2*\x,0) to [capacitor,*-*,l={$\SI{50}{\micro\farad}$}]++(0,\y);
\draw(3*\x+\dx,\y/2)node[right]{$\begin{aligned}&+ \\ &v_0(t) \\ &- \end{aligned}$};
\end{tikzpicture}
\caption{مشق \حوالہ{مشق_عارضی_برق_گیر_الف} کا دور۔}
\label{شکل_عارضی_مشق_برق_گیر_الف}
\end{figure}

جوابات:\عددی{v_C(0_+)=\SI{20}{\volt}}، \عددی{v_0(t)=8 e^{-\frac{t}{0.5}} \, \si{\volt}}، \عددی{\tau=\SI{0.5}{\second}}
\انتہا{مشق}
%===================

\ابتدا{مشق}\شناخت{مشق_عارضی_برق_گیر_ب}
شکل \حوالہ{شکل_عارضی_مشق_برق_گیر_ب} میں ازل سے چالو سوئچ کو  لمحہ \عددی{t=0} پر منقطع کیا جاتا ہے۔برق گیر پر ابتدائی دباو دریافت کرتے ہوئے \عددی{v_0(t)} دریافت کریں۔
\begin{figure}
\centering
\begin{tikzpicture}
\draw(0,0) to [american voltage source,l={$\SI{12}{\volt}$}]++(0,\y) to [resistor,l={$\SI{2}{\kilo\ohm}$}]++(\x,0) to [ospst,l={${t=0}$}] ++(\x,0) to [short]++(2*\x,0) to [capacitor,l_={$\SI{10}{\micro\farad}$}]++(0,-\y) to [short] (0,0);
\draw(\x,0) to [resistor,*-*,l={$\SI{4}{\kilo\ohm}$}]++(0,\y);
\draw(3*\x,0) to [resistor,*-*,l={$\SI{8}{\kilo\ohm}$}]++(0,\y);
\draw(2*\x,0) to [american current source,*-*,l={$\SI{4}{\milli\ampere}$}]++(0,\y);
\draw(4*\x+2*\dx,\y/2)node[right]{$\begin{aligned}&+ \\ &v_0(t) \\ &- \end{aligned}$};
\end{tikzpicture}
\caption{مشق \حوالہ{مشق_عارضی_برق_گیر_ب} کا دور۔}
\label{شکل_عارضی_مشق_برق_گیر_ب}
\end{figure}

جوابات:\عددی{v_0(0_+)=\frac{80}{7} \, \si{\volt}}، \عددی{v_0(t)=32-\frac{144}{7}e^{-\frac{100t}{7}} \, \si{\volt}}
\انتہا{مشق}
%===================


\ابتدا{مشق}\شناخت{مشق_عارضی_برق_گیر_پ}
شکل \حوالہ{شکل_عارضی_مشق_برق_گیر_پ} میں ازل سے چالو سوئچ کو  لمحہ \عددی{t=0} پر منقطع کیا جاتا ہے۔امالہ گیر میں ابتدائی رو دریافت کرتے ہوئے \عددی{i_L(t)} دریافت کریں۔دور کا وقتی مستقل  حاصل کریں۔
\begin{figure}
\centering
\begin{tikzpicture}
\draw(0,0) to [american voltage source,l={$\SI{10}{\volt}$}]++(0,2*\y) to [ospst,l={${t=\SI{0}{\second}}$}]++(\x,0) to [short] ++(\x,0) to [short]++(\x,0) to [resistor,l_={$\SI{10}{\ohm}$}]++(0,-2*\y) to [short] (0,0);
\draw(\x,0) to [resistor,*-,l={$\SI{8}{\ohm}$}]++(0,\y) to [inductor,i<_={$i_L$},-*,l={$\SI{4}{\henry}$}]++(0,\y);
\end{tikzpicture}
\caption{مشق \حوالہ{مشق_عارضی_برق_گیر_پ} کا دور۔}
\label{شکل_عارضی_مشق_برق_گیر_پ}
\end{figure}

جوابات:\عددی{i_L(0_+)=\SI{1.25}{\ampere}}، \عددی{i_L(t)=1.25e^{-3000t} \, \si{\ampere}}، \عددی{\tau=\tfrac{1}{3} \, \si{\milli\second}}
\انتہا{مشق}
%===================
\ابتدا{مثال}\شناخت{مثال_عارضی_برق_گیر_وقت_صفر_نہیں_الف}
شکل \حوالہ{شکل_عارضی_برق_گیر_وقت_صفر_نہیں_الف} میں ازل سے چالو سوئچ لمحہ \عددی{t=\SI{2}{\second}} پر منقطع کیا جاتا ہے۔رو \عددی{i(t)} دریافت کریں۔

\begin{figure}
\centering
\begin{subfigure}{0.5\textwidth}
\centering
\begin{tikzpicture}
\draw(0,0) to [american voltage source,l={$\SI{10}{\volt}$}]++(0,\y) to [resistor,l={$\SI{2}{\kilo\ohm}$}]++(\x,0) to [resistor,i<^={$i(t)$},l={$\SI{4}{\kilo\ohm}$}]++(\x,0) to [resistor,l={$\SI{6}{\kilo\ohm}$}]++(\x,0);
\draw(0,0) to [short]++(3*\x,0) to [american voltage source,l_={$\SI{20}{\volt}$}]++(0,\y);
\draw(\x,0) to [ospst,*-*,l={${t=\SI{2}{\second}}$}]++(0,\y);
\draw(2*\x,0) node[ground]{}to [capacitor,*-*,l={$\SI{5}{\micro\farad}$}]++(0,\y)node[above]{$v(t)$};
\end{tikzpicture}
\caption*{(الف)}
\end{subfigure}
\begin{subfigure}{0.5\textwidth}
\centering
\begin{tikzpicture}
\draw(0,0) to [american voltage source,l={$\SI{10}{\volt}$}]++(0,\y) to [resistor,l={$\SI{2}{\kilo\ohm}$}]++(\x,0) to [resistor,i<^={$i(t)$},l={$\SI{4}{\kilo\ohm}$}]++(\x,0) to [resistor,l={$\SI{6}{\kilo\ohm}$}]++(\x,0);
\draw(0,0) to [short]++(3*\x,0) to [american voltage source,l_={$\SI{20}{\volt}$}]++(0,\y);
\draw(\x,0) to [short,*-*]++(0,\y);
\draw(2*\x,0)node[ground]{} to [short,*-o]++(0,\y/8);
\draw(2*\x,\y) to [short,*-o]++(0,-\y/8);
\draw(2*\x+2*\dx,\y/2)node[]{$\begin{aligned} &+ \\ &v_C(2_-) \\ &- \end{aligned}$};
\end{tikzpicture}
\caption*{(ب)}
\end{subfigure}
\begin{subfigure}{0.5\textwidth}
\pgfplotsset{scaled y ticks=false, scaled x ticks=false}
\begin{tikzpicture}
\begin{axis}[axis lines*=middle,xlabel={$t$},ylabel={$i(t)$},ytick={0.002,-0.00033,0.000833},yticklabels={$\SI{2}{\milli\ampere}$,$\SI{-0.33}{\milli\ampere}$,${\frac{5}{6}\,\si{\milli\ampere}}$},
every axis x label/.style={
    at={(ticklabel* cs:1.15)},
    anchor=east,}, 
	every axis y label/.style={
    at={(ticklabel* cs:1.05)},
    anchor=east,}
]
\addplot[domain=1.99:2,samples=10]{0.002};
\addplot[domain=2:2.075,samples=100]{5/6000-7/6000*e^(200/3*(2-x))};
\draw(axis cs:2,0.002)--(axis cs:2,-1/3000);
\end{axis}
\end{tikzpicture}
\caption*{(پ)}
\end{subfigure}
\caption{مثال \حوالہ{مثال_عارضی_برق_گیر_وقت_صفر_نہیں_الف} کے اشکال۔}
\label{شکل_عارضی_برق_گیر_وقت_صفر_نہیں_الف}
\end{figure}

حل:سوئچ منقطع کرنے سے فوراً پہلے کی صورت حال شکل-ب میں دکھائی گئی ہے۔چونکہ ازل سے سوئچ چالو تھا لہٰذا دور برقرار حالت میں ہو گا اور یوں برق گیر کو کھلا دور تصور کیا جائے گا۔شکل-ب کو دیکھ کر
\begin{align*}
i(t<\SI{2}{\second})=\frac{20}{4000+6000}=\SI{2}{\milli\ampere}
\end{align*}
اور
\begin{align*}
v_C(2_-)=v_C(2_+)=20\left(\frac{4000}{4000+6000}\right)=\SI{8}{\volt}
\end{align*}
لکھا جا سکتا ہے۔سوئچ منقطع ہونے کے بعد کی صورت حال شکل-الف میں دی گئی ہے۔جوڑ \عددی{v(t)} پر کرخوف مساوات رو لکھتے ہوئے
\begin{align*}
\frac{v(t)-10}{2000+4000}+5\times 10^{-6}\frac{\dif v(t)}{\dif t}+\frac{v(t)-20}{6000}=0
\end{align*}
ترتیب دینے سے
\begin{align*}
\frac{\dif v(t)}{\dif t}+\frac{200}{3} v(t)=1000
\end{align*}
حاصل ہوتا ہے۔اس کے جبری اور فطری  حل درج ذیل ہیں
\begin{align*}
v_j(t)&=K_1=\SI{15}{\volt}\\
v_f(t)&=K_2e^{-\frac{200}{3}t}
\end{align*}
جن کا مجموعہ مکمل حل
\begin{align*}
v(t>2)=15+K_2e^{-\frac{200}{3}t}
\end{align*}
 دیتا ہے۔ابتدائی معلومات  \عددی{v(2_+)=\SI{8}{\volt}}لمحہ \عددی{t=\SI{2}{\second}} پر ہم جانتے ہیں جنہیں درج بالا مساوات میں پُر کرتے ہوئے
\begin{align*}
8=15+K_2e^{-\frac{200}{3}\times 2}
\end{align*}
\عددی{K_2} کی قیمت درج ذیل حاصل ہوتی ہے۔
\begin{align*}
K_2=-7e^{\frac{400}{3}}
\end{align*}
یوں مکمل حل درج ذیل ہو گا۔
\begin{align*}
v(t>2)=15-7e^{\frac{200}{3}(2-t)}
\end{align*}
اب شکل-الف کو دیکھ کر
\begin{align*}
i(t>2)&=\frac{v(t>2)-10}{6000}\\
&=\frac{5}{6}-\frac{7}{6} e^{\frac{200}{3}(2-t)} \, \si{\milli\ampere}
\end{align*}
لکھا جا سکتا ہے جو درکار مساوات ہے۔یوں سوئچ منقطع کرنے سے پہلے اور اس کے بعد کے جوابات سے درج ذیل لکھا جا سکتا ہے
\begin{align*}
i(t)=
\begin{cases}
\SI{2}{\milli\ampere} & t<\SI{2}{\second}\\
\frac{5}{6}-\frac{7}{6} e^{\frac{200}{3}(2-t)} \, \si{\milli\ampere} & t>\SI{2}{\second}
\end{cases}
\end{align*}
جسے شکل-پ میں دکھایا گیا ہے جہاں سے آپ دیکھ سکتے ہیں کہ سوئچ منقطع کرنے سے پہلے برقرار رو \عددی{\SI{2}{\milli\ampere}} تھی جبکہ سوئچ منقطع کرنے کے بعد برقرار حالت \عددی{(t \to \infty)} میں رو \عددی{\tfrac{5}{6}\,\si{\milli\ampere}} ہے۔یاد رہے کہ برق گیر کا دباو فوراً تبدیل نہیں ہو سکتا البتہ اس میں رو یک دم تبدیل ہو سکتی ہے۔

وقت \عددی{t \to \infty} پر دور برقرار حالت اختیار کر چکا ہو گا لہٰذا برق گیر کو کھلا دور کرتے ہوئے شکل \حوالہ{شکل_عارضی_برق_گیر_وقت_صفر_نہیں_الف}-الف سے برقرار حالت  رو درج ذیل لکھی جا سکتی ہے۔
\begin{align*}
i(t\to\infty)=\frac{20-10}{2000+4000+6000}=\frac{5}{6} \, \si{\milli\ampere}
\end{align*}


\انتہا{مثال}
%=======================
\ابتدا{مثال}\شناخت{مثال_عارضی_امالہ_گیر_سوئچ_چالو}
شکل \حوالہ{شکل_عارضی_امالہ_گیر_سوئچ_چالو_ب}-الف میں ازل سے منقطع سوئچ لمحہ \عددی{t=\SI{7}{\second}} پر چالو کیا جاتا ہے۔رو \عددی{i(t)} دریافت کریں۔ 
\begin{figure}
\centering
\begin{subfigure}{0.5\textwidth}
\centering
\begin{tikzpicture}
\draw(0,0) to [american current source,l={$\SI{6}{\ampere}$}]++(0,\y) to [resistor,l={$\SI{1}{\ohm}$}]++(\x,0) to [inductor,i={$i_L(t)$},l={$\SI{5}{\henry}$}]++(\x,0) to [short]++(\x,0) to [resistor,l={$\SI{2}{\ohm}$}]++(0,-\y) to [short] (0,0);
\draw(\x,0) to [resistor,i<_={$i(t)$},*-*,l={$\SI{4}{\ohm}$}]++(0,\y);
\draw(2*\x,0) to [cspst,*-*,l_={${t=\SI{7}{\second}}$}]++(0,\y);
\end{tikzpicture}%
\caption*{(الف)}
\end{subfigure}
\begin{subfigure}{0.5\textwidth}
\centering
\begin{tikzpicture}
\draw(0,0) to [american current source,l={$\SI{6}{\ampere}$}]++(0,\y) to [resistor,l={$\SI{1}{\ohm}$}]++(\x,0) to [short,i={$i_L(t)$}]++(\x,0)to [resistor,l={$\SI{2}{\ohm}$}]++(0,-\y) to [short] (0,0);
\draw(\x,0) to [resistor,i<_={$i(t)$},*-*,l={$\SI{4}{\ohm}$}]++(0,\y);
\end{tikzpicture}%
\caption*{(ب)}
\end{subfigure}%
\begin{subfigure}{0.5\textwidth}
\centering
\begin{tikzpicture}
\draw(0,0) to [american current source,l={$\SI{6}{\ampere}$}]++(0,\y) to [resistor,l={$\SI{1}{\ohm}$}]++(\x,0) to [inductor,l={$\SI{5}{\henry}$}]++(\x,0)to [short]++(0,-\y) to [short] (0,0);
\draw(\x,0) to [resistor,i<_={$i(t)$},*-*]++(0,\y);
\draw(\x,3/4*\y)node[left]{$\SI{4}{\ohm}$};
%loop currents
\draw[stealth-]([shift={(-150:\x/6)}]\x/2,\y/2) arc (-150:150:\x/6);
\draw(\x/2,\y/2)node{$i_1$};
\draw[stealth-]([shift={(-150:\x/6)}]\x+\x/2,\y/2) arc (-150:150:\x/6);
\draw(\x+\x/2,\y/2)node{$i_2$};
\end{tikzpicture}
\caption*{(پ)}
\end{subfigure}
\begin{subfigure}{0.5\textwidth}
\centering
\pgfplotsset{scaled y ticks=false, scaled x ticks=false}
\begin{tikzpicture}
\begin{axis}[axis lines*=middle,
xlabel={$t$},
ylabel={$i(t)$},
every axis x label/.style={
    at={(ticklabel* cs:1.15)},
    anchor=east,}, 
	every axis y label/.style={
    at={(ticklabel* cs:1.05)},
    anchor=east,}
]
\addplot[domain=6:7,samples=10]{2};
\addplot[domain=7:13,samples=100]{2*e^(4/5*(7-x))};
\end{axis}
\end{tikzpicture}%
\caption*{(ت)}
\end{subfigure}%
\caption{مثال \حوالہ{مثال_عارضی_امالہ_گیر_سوئچ_چالو} کا اشکال۔}
\label{شکل_عارضی_امالہ_گیر_سوئچ_چالو_ب}
\end{figure}

حل:منقطع سوئچ کی صورت میں دور برقرار حالت میں ہو گا لہٰذا امالہ گیر کو قصر دور تصور کرتے ہوئے شکل-ب حاصل کی گئی ہے۔تقسیم رو کے کلیے سے
\begin{align*}
i_L(7_-)=i_L(7_+)=6\left(\frac{4}{4+2}\right)=\SI{4}{\ampere}
\end{align*}
اور
\begin{align}\label{مساوات_عارضی_امالہ_گیر_سات_سیکنڈ_الف}
i(t)=\SI{6}{\ampere}-i_L(t)=6-4=\SI{2}{\ampere} \quad \quad (t<\SI{7}{\second})
\end{align}
لکھا جا سکتا ہے۔سوئچ چالو کرنے کے بعد کی صورت حال شکل-پ میں دکھائی گئی ہے جہاں سے درج ذیل لکھا جا سکتا ہے۔
\begin{align*}
i_1&=\SI{6}{\ampere}\\
5\frac{\dif i_2}{\dif t}+4(i_2-i_1)&=0
\end{align*}
ان مساوات کو ملاتے ہوئے
\begin{align*}
5\frac{\dif i_2}{\dif t}+4(i_2-6)&=0
\end{align*}
یعنی
\begin{align*}
\frac{\dif i_2}{\dif t}+\frac{4}{5} i_2=\frac{24}{5}
\end{align*}
حاصل ہوتا ہے جس کا مکمل حل درج ذیل ہے۔
\begin{align*}
i_2=6+K_2e^{-\frac{4}{5}t}
\end{align*}
چونکہ \عددی{i_2} درحقیقت \عددی{i_L} ہی ہے لہٰذا نا معلوم مستقل \عددی{K_2} کو ابتدائی معلومات سے حاصل کرتے ہیں۔درج بالا مساوات میں  \عددی{t=\SI{7}{\second}} پر \عددی{i_L(7_+)=\SI{4}{\ampere}} پُر کرتے ہوئے 
\begin{align*}
4=6+K_2e^{-\frac{4}{5}\times 7}
\end{align*} 
سے
\begin{align*}
K_2=-2e^{\frac{4}{5}\times 7}
\end{align*}
حاصل ہوتا ہے۔یوں سوئچ چالو کرنے کے بعد \عددی{i_2} کا مکمل حل درج ذیل لکھا جائے گا۔
\begin{align*}
i_2=6-2e^{\frac{4}{5}(7-t)}
\end{align*}
اب شکل-پ کو دیکھتے ہوئے
\begin{align*}
i(t)&=i_1-i_2\\
&=6-\left(6-2^{\frac{4}{5}(7-t)}\right)\\
&=2e^{\frac{4}{5}(7-t)}\quad \quad (t>\SI{7}{\second})
\end{align*}
لکھا جا سکتا ہے۔یوں ازل سے ابد تک \عددی{i(t)} کو مساوات \حوالہ{مساوات_عارضی_امالہ_گیر_سات_سیکنڈ_الف} اور درج بالا مساوات  پیش کرتے ہیں۔انہیں اکٹھے لکھتے  اور شکل-ت میں پیش کرتے ہیں۔
\begin{align}
i(t)=
\begin{cases}
\SI{2}{\ampere} & t<\SI{7}{\second}\\
2e^{\frac{4}{5}(7-t)} \, \si{\ampere} & t>\SI{7}{\second}
\end{cases}
\end{align}
\انتہا{مثال}
%========================
\ابتدا{مشق}\شناخت{مشق_عارضی_امالہ_دو_منبع_الف}
شکل \حوالہ{شکل_عارضی_امالہ_دو_منبع_الف} میں ابتدائی حالت \عددی{i_L(0_+)} دریافت کریں۔دائرہ \عددی{abcfa} میں \عددی{i_1} اور \عددی{abdea} میں \عددی{i_2} لیتے ہوئے  کرخوف مساوات دباو لکھیں۔ان مساوات  سے صرف \عددی{i_1} پر مبنی مساوات حاصل کریں۔یوں ازل سے ابد تک \عددی{i_L} دریافت کریں۔
\begin{figure}
\centering
\begin{tikzpicture}
\draw(0,0)node[left]{$a$} to [american voltage source,l={$\SI{12}{\volt}$}]++(0,2*\y)node[left]{$b$} to [resistor,l={$\SI{2}{\ohm}$}]++(\x,0) node[above]{$c$} to [resistor,l={$\SI{2}{\ohm}$}] ++(\x,0)to [short]++(\x,0)node[right]{$d$} to [resistor,l_={$\SI{4}{\ohm}$}]++(0,-2*\y)node[right]{$e$} to [short] (0,0);
\draw(\x,0)node[below]{$f$} to [inductor,i<_={$i_L$},*-,l={$\SI{2}{\henry}$}]++(0,\y) to [resistor,-*,l={$\SI{3}{\ohm}$}]++(0,\y);
\draw(2*\x,0) to [american voltage source,*-,l={$\SI{16}{\volt}$}]++(0,\y) to [ospst,-*,l={${t=\SI{0}{\second}}$}]++(0,\y);
%\draw(3*\x+\dx,\y)node[right]{$\begin{aligned} &+ \\ \\ &v_0(t)  \\  \\ &- \end{aligned}$};
\end{tikzpicture}
\caption{مشق \حوالہ{مشق_عارضی_امالہ_دو_منبع_الف} کا دور۔}
\label{شکل_عارضی_امالہ_دو_منبع_الف}
\end{figure}

جوابات:\عددی{i_L(0_+)=\SI{3.5}{\ampere}}، \عددی{\tfrac{\dif i_1}{\dif t}+2.25i_1=4.5}، \عددی{i_L(t>0)=2+1.5e^{-2.25t}\,\si{\ampere}}
\انتہا{مشق}
%=======================
\ابتدا{مشق}\شناخت{مشق_عارضی_برق_گیر_دو_منبع_الف}
شکل \حوالہ{شکل_عارضی_برق_گیر_دو_منبع_الف} میں \عددی{v_0(t)} حاصل کریں۔
\begin{figure}
\centering
\begin{tikzpicture}
\draw(0,0) to [american voltage source,l={$\SI{12}{\volt}$}]++(0,2*\y) to [resistor,l={$\SI{1}{\ohm}$}]++(\x,0) to [resistor,l={$\SI{1}{\ohm}$}] ++(\x,0) to [resistor,l={$\SI{2}{\ohm}$}] ++(\x,0);
\draw(0,0) to [short]++(3*\x,0) to [ospst,l_={${t=0}$}] ++(0,\y) to [american voltage source,l_={$\SI{8}{\volt}$}]++(0,\y);
\draw(\x,0) to [capacitor,*-*,l={$\SI{2}{\farad}$}]++(0,2*\y);
\draw(2*\x,0) to [resistor,*-*,l={$\SI{2}{\ohm}$}]++(0,2*\y);
\draw(2*\x+\dx,\y)node[right]{$\begin{aligned} &+ \\ \\ &v_0(t)  \\  \\ &- \end{aligned}$};
\end{tikzpicture}
\caption{مشق \حوالہ{مشق_عارضی_برق_گیر_دو_منبع_الف} کا دور۔}
\label{شکل_عارضی_برق_گیر_دو_منبع_الف}
\end{figure}

جوابات:\عددی{v_0(t)=\tfrac{24}{5}+\tfrac{1}{5}e^{-\tfrac{5}{8}t} \, \si{\volt}}
\انتہا{مشق}
%=======================
\ابتدا{مشق}\شناخت{مشق_عارضی_برق_گیر_دو_منبع_ب}
شکل \حوالہ{شکل_عارضی_برق_گیر_دو_منبع_ب} میں سوئچ منقطع کرنے کے بعد \عددی{v_0} حاصل کریں۔
\begin{figure}
\centering
\begin{tikzpicture}
\draw(0,2*\y) to [american voltage source,l={$\SI{18}{\volt}$}]++(0,-2*\y);
\draw(0,2*\y) to [resistor,l={$\SI{4}{\ohm}$}]++(\x,0) to [resistor,l={$\SI{4}{\ohm}$}] ++(\x,0);
\draw(0,0) to [short]++(2*\x,0) to [american voltage source,l={$\SI{6}{\volt}$}]++(0,\y) to [ospst,l_={${t=0}$}] ++(0,\y);
\draw(\x,0) to [inductor,*-,l={$\SI{6}{\henry}$}]++(0,\y) to [resistor,-*,l={$\SI{8}{\ohm}$}]++(0,\y);
\draw(\x+\dx,\y+\y/2)node[right]{$\begin{aligned} &+ \\ &v_0  \\  &- \end{aligned}$};
\end{tikzpicture}
\caption{مشق \حوالہ{مشق_عارضی_برق_گیر_دو_منبع_ب} کا دور۔}
\label{شکل_عارضی_برق_گیر_دو_منبع_ب}
\end{figure}

جوابات:\عددی{v_0=-12+\tfrac{9}{2}e^{-2t} \, \si{\volt}}
\انتہا{مشق}
%=======================

\حصہ{دھڑکن}
گزشتہ حصے میں سوئچ کو چالو یا منقطع کرتے ہوئے ادوار میں یکدم تبدیلی پیدا کی گئی۔فوراً تبدیلی پیدا کرنے والے دو عدد تفاعل نہایت اہم ہیں۔انہیں \اصطلاح{اکائی سیڑھی تفاعل}\فرہنگ{اکائی سیڑھی تفاعل}\حاشیہب{unit step function}\فرہنگ{unit step function} اور \اصطلاح{اکائی جھٹکا تفاعل}\فرہنگ{اکائی جھٹکا تفاعل}\حاشیہب{unit impulse function}\فرہنگ{unit impulse function} کہتے ہیں۔آئیں اکائی سیڑھی تفاعل پر غور کریں۔

\اصطلاح{اکائی سیڑھی تفاعل} \عددی{u(t)} کی الجبرائی تعریف درج ذیل ہے۔
\begin{align}
u(t)=
\begin{cases}
0 & t<0\\
1 & t>0
\end{cases}
\end{align}
یوں یہ تفاعل \اصطلاح{بے بعد}\فرہنگ{بے بعد}\حاشیہب{dimensionless}\فرہنگ{dimensionless} ہے  جو منفی\عددی{t} کی صورت میں صفر کے برابر جبکہ مثبت  \عددی{t} کی صورت میں اکائی کے برابر ہے۔شکل \حوالہ{شکل_عارضی_اکائی_سیڑھی_تفاعل_الف}-الف میں اکائی سیڑھی تفاعل کو دکھایا گیا ہے۔اکائی سیڑھی تفاعل کے متغیرہ کو \عددی{t-t_0} لکھتے ہوئے شکل \حوالہ{شکل_عارضی_اکائی_سیڑھی_تفاعل_الف}-ب حاصل ہوتا ہے جو افقی محدد پر \عددی{t_0} دائیں منتقل اکائی سیڑھی تفاعل \عددی{u(t-t_0)} ہے۔یہ تفاعل منفی \عددی{t-t_0} کی صورت میں صفر کے برابر ہے جبکہ مثبت \عددی{t-t_0} کی صورت میں یہ اکائی کے برابر ہے۔اکائی سیڑھی تفاعل کو \عددی{A} سے ضرب دینے سے \عددی{A} گنا اونچی سیڑھی حاصل ہو گی۔شکل \حوالہ{شکل_عارضی_اکائی_سیڑھی_تفاعل_الف}-پ میں مثبت \عددی{A} کی صورت میں \عددی{Au(t)} اور شکل \حوالہ{شکل_عارضی_اکائی_سیڑھی_تفاعل_الف}-ت میں \عددی{-Au(t-t_0)} دکھائے گئے ہیں۔ 
\begin{figure}
\centering
\begin{subfigure}{0.5\textwidth}
\centering
\begin{tikzpicture}
\draw[gray](0,-0.5)--(0,2)node[left]{$u(t)$};
\draw[gray](-0.5,0)--(3,0)node[right]{$t$};
\draw(-0.25,0)--(0,0)--(0,1)node[left]{$1$}--(3,1);
\end{tikzpicture}%
\caption*{(الف)}
\end{subfigure}%
\begin{subfigure}{0.5\textwidth}
\centering
\begin{tikzpicture}
\draw[gray](0,-0.5)--(0,2)node[left]{$u(t-t_0)$};
\draw[gray](-0.5,0)--(3,0)node[right]{$t$};
\draw(-0.25,0)--(1,0)node[below]{$t_0$}--(1,1)--(3,1);
\draw[gray,dashed](1,1)--(0,1)node[left,black]{$1$};
\end{tikzpicture}%
\caption*{(ب)}
\end{subfigure}
\begin{subfigure}{0.5\textwidth}
\centering
\begin{tikzpicture}
\draw[gray](0,-1.5)--(0,1.5)node[left]{$Au(t)$};
\draw[gray](-0.5,0)--(3,0)node[right]{$t$};
\draw(-0.25,0)--(0,0)--(0,1)node[left]{$A$}--(3,1);
\end{tikzpicture}%
\caption*{(پ)}
\end{subfigure}%
\begin{subfigure}{0.5\textwidth}
\centering
\begin{tikzpicture}
\draw[gray](0,-1.5)--(0,1.5)node[left]{$-Au(t-t_0)$};
\draw[gray](-0.5,0)--(3,0)node[right]{$t$};
\draw(-0.25,0)--(1,0)node[above]{$t_0$}--(1,-1)--(3,-1);
\draw[gray,dashed](1,-1)--(0,-1)node[left,black]{$-A$};
\end{tikzpicture}%
\caption*{(ت)}
\end{subfigure}%
\caption{اکائی سیڑھی تفاعل۔}
\label{شکل_عارضی_اکائی_سیڑھی_تفاعل_الف}
\end{figure}

اکائی سیڑھی تفاعل سے مستطیل تفاعل حاصل کیا جا سکتا ہے۔یہ عمل شکل \حوالہ{شکل_عارضی_اکائی_سیڑھی_تفاعل_ب} میں دکھایا گیا ہے جہاں \عددی{Au(t)} اور
 \عددی{-Au(t-t_0)} کا مجموعہ
\begin{align}
v(t)=Au(t)-Au(t-t_0)
\end{align}
 لیتے ہوئے \عددی{A} حیطے کا مستطیل تفاعل حاصل کیا گیا ہے۔ 
\begin{figure}
\centering
\begin{subfigure}{0.5\textwidth}
\centering
\begin{tikzpicture}
\draw[gray](0,-0.5)--++(0,2)node[left]{$Au(t)$};
\draw[gray](-0.5,0)--++(3.5,0)node[right]{$t$};
\draw(-0.25,0)--++(0.25,0)--++(0,1)node[left]{$A$}--++(3,0);
%
\pgfmathsetmacro{\ky}{-2}
\draw[gray](0,\ky-1.5)--++(0,2)node[left]{$-Au(t-t_0)$};
\draw[gray](-0.5,\ky)--++(3.5,0)node[right]{$t$};
\draw(-0.25,\ky)--++(1.25,0)--++(0,-1)--++(2,0);
\draw(0,\ky-1)node[left]{$-A$};
%
\pgfmathsetmacro{\ky}{-3}
\draw[gray](0,2*\ky-0.5)--++(0,2)node[left]{$v(t)$};
\draw[gray](-0.5,2*\ky)--++(3.5,0)node[right]{$t$};
\draw(-0.25,2*\ky)--++(0.25,0)--++(0,1)node[left]{$A$}--++(1,0)--++(0,-1)node[below]{$t_0$}--++(2,0);
\end{tikzpicture}%
\end{subfigure}%
\caption{اکائی سیڑھی تفاعل کے استعمال سے دیگر تفاعل کا حصول۔}
\label{شکل_عارضی_اکائی_سیڑھی_تفاعل_ب}
\end{figure}


\باب{برقرار حالت بدلتی رو}
جبری تفاعل میں یکدم تبدیلی سے دور عارضی حالت اختیار کرتا ہے۔محدود قیمت کے وقتی مستقل کی صورت میں آخر کار عارضی دورانیہ گزر جاتا ہے اور دور ایک بار پھر برقرار حالت اختیار کر لیتا ہے۔جبری تفاعل میں یکدم تبدیلی کی غیر موجودگی میں دور برقرار صورت میں رہتا ہے۔اس باب میں ایسے ہی ادوار پر غور کیا جائے گا جن کے جبری تفاعل میں یکدم تبدیلی نہیں پائی جاتی۔ایسی صورت میں جبری حل ہی مکمل حل ہو گا۔اس باب میں مکمل حل  سے مراد جبری حل ہو گا۔ 
 
\حصہ{مخلوط اعداد}
\اصطلاح{حقیقی}\فرہنگ{حقیقی!عدد}\حاشیہب{real number}\فرہنگ{real!number} عدد اور \اصطلاح{خیالی}\فرہنگ{خیالی!عدد}\حاشیہب{imaginary number}\فرہنگ{imaginary!number} عدد کے مجموعے کو \اصطلاح{مخلوط}\فرہنگ{مخلوط!عدد}\حاشیہب{complex number}\فرہنگ{complex!number} عدد کہتے ہیں۔مخلوط اعداد کو \اصطلاح{مخلوط سطح}\فرہنگ{مخلوط سطح}\حاشیہب{complex plane}\فرہنگ{complex!plane} پر دکھایا جایا ہے۔مخلوط سطح پر افقی محدد حقیقی اعداد کو ظاہر کرتا ہے جبکہ عمودی محدد خیالی اعداد کو ظاہر کرتا ہے۔

شکل \حوالہ{شکل_بدلتا_مخلوط_اعداد_لکھنا}-الف میں مخلوط عدد \عددی{3+j2} دکھایا گیا ہے۔اسی شکل میں ایک مستطیل بھی دکھایا گیا ہے۔اس عدد کے حقیقی اور خیالی اجزاء مستطیل کے اطراف ہیں۔یوں مخلوط عدد کو حقیقی اور خیالی اجزاء کے مجموعے یعنی \عددی{3+j2} کے طرز پر لکھنے کو \اصطلاح{مستطیلی طرز}\فرہنگ{مستطیلی طرز}\حاشیہب{rectangular form}\فرہنگ{rectangular form} کہتے ہیں۔ 

شکل \حوالہ{شکل_بدلتا_مخلوط_اعداد_لکھنا}-الف میں مخلوط نقطہ \عددی{(3+j2)} سے محدد کے مرکز \عددی{(0,0)} تک لکیر کھینچی گئی ہے۔اس لکیر کی لمبائی \عددی{r} کو مسئلہ فیثاغورث کی مدد سے
\begin{align*}
r=\sqrt{3^2+2^2}=\sqrt{13}
\end{align*}
لکھا جا سکتا ہے۔اسی طرح افقی محدد سے لکیر تک کا زاویہ درج ذیل ہو گا۔
\begin{align*}
\theta=\tan^{-1}\frac{2}{3}=33.69^{\circ}
\end{align*}
شکل \حوالہ{شکل_بدلتا_مخلوط_اعداد_لکھنا}-ب میں اسی مخلوط عدد کو \عددی{r\phase{\theta}} کی شکل میں دکھایا گیا ہے۔مخلوط عدد کو حیطے اور زاویے سے ظاہر کرنے کو \اصطلاح{زاویائی طرز}\فرہنگ{زاویائی طرز}\حاشیہب{angular form}\فرہنگ{angular form} کہتے ہیں۔

کسی بھی مخلوط عدد \عددی{m} کو 
\begin{align}
m=x+jy\quad \quad \text{\RL{مستطیل طرز}}
\end{align}
یا
\begin{align}
m=r\phase{\theta} \quad \quad \text{\RL{زاویائی طرز}}
\end{align}
میں لکھا جا سکتا ہے جہاں مستطیلی طرز سے زاویائی طرز درج ذیل طریقے سے حاصل کی جاتی ہے
\begin{gather}
\begin{aligned}\label{مساوات_بدلتا_مستطیل_سے_زاویائی}
r&=\sqrt{x^2+y^2}\\
\theta&=\tan^{-1}\frac{y}{x}
\end{aligned}
\end{gather}
جبکہ زاویائی طرز سے مستطیل طرز درج ذیل سے حاصل کی جاتی ہے۔
\begin{gather}
\begin{aligned}
x&=r \cos \theta\\
y&=r\sin \theta
\end{aligned}
\end{gather}
%
\begin{figure}
\centering
\begin{subfigure}{0.5\textwidth}
\centering
\begin{tikzpicture}
\pgfmathsetmacro{\ang}{atan(2/3)};
\draw(0,0)--++(3.2,0)node[right]{حقیقی};
\draw(0,0)--++(0,2.7)node[left]{خیالی};
\foreach \x in {1,2,3}{\draw(\x,0)--++(0,-0.1)node[below]{$\x$};}
\foreach \y in {1,2}{\draw(0,\y)--++(-0.1,0)node[left]{$j \y$};}
\draw[gray](3,0)--++(0,2)--++(-3,0);
\draw[gray,dashed](0,0)--(3,2)node[pos=0.6,fill=white]{r};
\draw[fill](3,2) circle (1.5pt)node[right]{$3+j2$};
\draw[gray,-stealth]([shift={(0:0.8)}]0,0) arc (0:\ang:0.8);
\draw[gray](2/3*\ang:0.9)node[right]{$\theta$};
\end{tikzpicture}
\caption*{(الف) مخلوط عدد لکھنے کی مستطیل طرز۔}
\end{subfigure}%
\begin{subfigure}{0.5\textwidth}
\centering
\begin{tikzpicture}
\pgfmathsetmacro{\ang}{atan(2/3)};
\draw(0,0)--++(4.2,0)node[right]{حقیقی};
\draw(0,0)--++(0,2.7)node[left]{خیالی};
\foreach \x in {1,2,3,4}{\draw(\x,0)--++(0,-0.1)node[below]{$\x$};}
\foreach \y in {1,2}{\draw(0,\y)--++(-0.1,0)node[left]{$j \y$};}
\draw[gray](0,0)--++(3,2)node[pos=0.6,fill=white]{$\sqrt{13}$};
\draw[fill](3,2) circle (1.5pt)node[right]{$\sqrt{13}\phase{33.69^{\circ}}$};
\draw[gray,-stealth]([shift={(0:0.8)}]0,0) arc (0:\ang:0.8);
\draw[gray](2/3*\ang:0.9)node[right]{$33.69^{\circ}$};
\end{tikzpicture}
\caption*{(ب) مخلوط عدد لکھنے کی زاویائی طرز۔}
\end{subfigure}%
\caption{مخلوط اعداد کو لکھنے کے طریقے۔}
\label{شکل_بدلتا_مخلوط_اعداد_لکھنا}
\end{figure}

مخلوط اعداد کو جمع، منفی، ضرب اور تقسیم کرنے کی چند مثالیں دیکھتے ہیں۔

%================================
\ابتدا{مثال}
مخلوط اعداد \عددی{a=2+j3} اور \عددی{b=4+j5} دیے گئے ہیں۔درج ذیل حاصل کریں۔
\begin{align*}
a+b, \quad \quad a-b, \quad \quad a b, \quad \quad \frac{a}{b}
\end{align*}

حل:مخلوط اعداد جمع (منفی) کرتے وقت حقیقی اجزاء کو علیحدہ جمع (منفی) کیا جاتا ہے اور خیالی اجزاء کو علیحدہ جمع (منفی) کیا جاتا ہے۔
\begin{align*}
a+b&=(2+j3)+(4+j5)=(2+4)+j(3+5)=6+j8\\
a-b&=(2+j3)-(4+j5)=(2-4)+j(3-5)=-2-j2
\end{align*}
مخلوط اعداد کو ضرب دیتے ہوئے \عددی{j^2=(\sqrt{-1})^2=-1} لکھا جاتا ہے۔
\begin{align*}
ab&=(2+j3)(4+j5)=8+j10+j12+j^2 15=(8-15)+j(10+12)=-7+j22
\end{align*}
مخلوط اعداد کو تقسیم کرتے ہیں۔
\begin{align*}
\frac{a}{b}&=\frac{2+j3}{4+j5}\\
&=\left(\frac{2+j3}{4+j5}\right)\left(\frac{4-j5}{4-j5}\right)\\
&=\frac{8-j10+j12-j^2 15}{4^2-(j 5)^2}\\
&=\frac{23+j2}{16+25}\\
&=\frac{23}{41}+j\frac{2}{41}\\
&=0.56098+j0.04878
\end{align*}
\انتہا{مثال}
%===============================
\ابتدا{مثال}
گزشتہ مثال میں مخلوط اعداد کو زاویائی طرز پر لکھتے ہوئے \عددی{ab} اور \عددی{\tfrac{a}{b}} حاصل کریں۔

حل:مساوات \حوالہ{مساوات_بدلتا_مستطیل_سے_زاویائی} استعمال کرتے ہوئے \عددی{a=2+j3} کا حیطہ اور زاویہ حاصل کرتے ہیں۔
\begin{align*}
r_a&=\sqrt{2^2+3^2}=\sqrt{13}\\
\theta_a&=\tan^{-1}\frac{3}{2}=56.31^{\circ}
\end{align*}
یوں
\begin{align*}
a=\sqrt{13}\phase{56.31^{\circ}}
\end{align*}
لکھا جائے گا۔اسی طرح \عددی{b=4+j5} کا حیطہ اور زاویہ حاصل کرتے ہوئے
\begin{align*}
r_b&=\sqrt{4^2+5^2}=\sqrt{41}\\
\theta_b&=\tan^{-1}\frac{5}{4}=51.34^{\circ}
\end{align*}
درج ذیل لکھا جائے گا۔
\begin{align*}
b=\sqrt{41}\phase{51.34^{\circ}}
\end{align*}
اس طرح
\begin{align*}
ab&=\left(\sqrt{13}\phase{56.31^{\circ}}\right)\left(\sqrt{41}\phase{51.34^{\circ}}\right)\\
&=\sqrt{13}\sqrt{41}\phase{56.31^{\circ}+51.34^{\circ}}\\
&=\sqrt{533}\phase{107.65^{\circ}}
\end{align*}
اور
\begin{align*}
\frac{a}{b}&=\frac{\sqrt{13}\phase{56.31^{\circ}}}{\sqrt{41}\phase{51.34^{\circ}}}\\
&=\frac{\sqrt{13}}{\sqrt{41}}\phase{56.31^{\circ}-51.34^{\circ}}\\
&=\sqrt{\frac{13}{41}}\phase{4.97^{\circ}}
\end{align*}
حاصل ہوتے ہیں۔

ان جوابات کو مستطیلی طرز میں درج ذیل لکھا جائے گا جو گزشتہ مثال کے جوابات ہیں۔
\begin{align*}
ab&=\sqrt{533} \cos 107.65^{\circ}+j \sqrt{533}\sin{107.65^{\circ}}=-7+j22\\
\frac{a}{b}&=\sqrt{\frac{13}{41}} \cos 4.97^{\circ}+j\sqrt{\frac{13}{41}}\sin 4.97^{\circ}=0.56098+j0.04878
\end{align*}
\انتہا{مثال}
%======================

ہم نے دیکھا کہ زاویائی طرز میں لکھا مخلوط عدد \عددی{a=r\phase{\theta}}  مستطیل طرز میں بھی لکھا جا سکتا ہے یعنی
\begin{align}
a=r\phase{\theta}=r \cos \theta +j r \sin \theta
\end{align}
\اصطلاح{یولر مساوات}\فرہنگ{یولر مساوات}\حاشیہب{Euler's equation}\فرہنگ{Euler's equation} درج ذیل ہے۔
\begin{align}
e^{j \theta}=\cos \theta+j \sin \theta
\end{align}
مندرجہ بالا دو مساوات کو ملاتے ہوئے درج ذیل لکھا جا سکتا ہے۔
\begin{align}\label{مساوات_بدلتا_مخلوط_عدد_طرز_لکھائی}
r\phase{\theta}=r e^{j\theta}=r\left(\cos \theta+j  \sin \theta\right)
\end{align}
%=================================
\ابتدا{مثال}
مخلوط عدد \عددی{m=5-j12} کو زاویائی طرز میں لکھیں۔

حل:مساوات \حوالہ{مساوات_بدلتا_مستطیل_سے_زاویائی} کے استعمال سے درج ذیل حاصل کرتے ہیں
\begin{align*}
r&=\sqrt{5^2+12^2}=13\\
\theta&=\tan^{-1}\frac{-12}{5}=-67.38^{\circ}
\end{align*}
لہٰذا درج ذیل لکھے جا سکتے ہیں۔
\begin{align*}
m&=13 e^{-j67.38^{\circ}}\\
m&=13\phase{-67.38^{\circ}}
\end{align*}
\انتہا{مثال}
%======================
\حصہ{سائن نما تفاعل}
\اصطلاح{سائن نما}\فرہنگ{سائن نما}\حاشیہب{sinusoidal}\فرہنگ{sinusoidal} تفاعل سے مراد \اصطلاح{سائن} تفاعل \عددی{\sin \theta} اور \اصطلاح{کوسائن} تفاعل \عددی{\cos \theta} ہیں۔شکل \حوالہ{شکل_بدلتی_رو_سائن_تفاعل}-الف میں رداس \عددی{A_0} کے گول دائرے پر ایک نقطہ یکساں رفتار کے ساتھ، گھڑی کی گردش کی الٹ سمت میں، حرکت کر رہا ہے۔یہ دائرہ \اصطلاح{کارتیسی محدد}\فرہنگ{کارتیسی محدد}\حاشیہب{Cartesian coordinates}\فرہنگ{Cartesian coordinates} کے مرکز \عددی{(0,0)} پر پایا جاتا ہے۔لمحہ \عددی{t} پر زاویہ \عددی{\phase{aox}} کی قیمت \عددی{\theta} کے برابر ہے۔نقطے سے \عددی{x} محدد پر عمودی لکیر محدد کو  \عددی{x(t)}  پر ٹکراتی ہے جبکہ \عددی{y} محدد پر عمودی لکیر \عددی{y(t)} پر ٹکراتی ہے۔شکل کو دیکھتے ہوئے درج ذیل لکھا جا سکتا ہے
\begin{align}\label{مساوات_بدلتا_سائن_نما_تفاعل_الف}
y(t)&=A_0 \sin \theta
\end{align}
جہاں \عددی{A_0} موج کی چوٹی ہے جسے موج کا \اصطلاح{حیطہ}\فرہنگ{حیطہ}\حاشیہب{amplitude}\فرہنگ{amplitude} کہتے ہیں اور \عددی{\theta} کو تفاعل کا \اصطلاح{دلیل}\فرہنگ{دلیل}\حاشیہد{ایک ماہر ریاضی اپنی خیالی دنیا میں کوسائن \عددی{\cos \theta} تفاعل کے ساتھ بحث میں مصورف ہوتا ہے۔ماہر ریاضی تفاعل کو دلیل کے طور پر صفر پیش کرتا ہے۔تفاعل اس کا فوراً جواب اکائی \عددی{(\cos 0=1)} دیتا ہے۔}\حاشیہب{argument}\فرہنگ{argument} کہتے ہیں۔اس مساوات میں \عددی{\theta} از خود وقت\عددی{t} پر منحصر ہے۔

 گردش کرتا نقطہ ایک چکر میں \عددی{360^{\circ}} درجے کا زاویہ یعنی \عددی{2\pi} ریڈیئن طے کرتا ہے۔ایک چکر کاٹنے کے لئے درکار دورانیے کو \اصطلاح{دوری عرصہ}\فرہنگ{دوری عرصہ}\حاشیہب{time period}\فرہنگ{time period} کہتے ہیں جسے \عددی{T} سے ظاہر کیا جاتا ہے۔
%=========================
\ابتدا{مشق}
شکل \حوالہ{شکل_بدلتی_رو_سائن_تفاعل}-الف میں نقطہ ایک چکر \عددی{\SI{20}{\milli\second}} میں پورا کرتا ہے۔یہ نقطہ ایک سیکنڈ میں کتنے چکر پورا کرے گا۔یہ نقطہ ایک سیکنڈ میں کتنے ریڈیئن کا زاویہ طے کرتا ہے۔

جوابات:\عددی{50} چکر، \عددی{100\pi \, \si{\radian}}
\انتہا{مشق}
%========================

اگر ایک چکر کاٹنے کے لئے \عددی{T} سیکنڈ کا وقت درکار ہو تب ایک سیکنڈ میں چکروں کی تعداد  \عددی{\tfrac{1}{T}} ہو گی جسے \اصطلاح{تعدد}\فرہنگ{تعدد}\حاشیہب{frequency}\فرہنگ{frequency} کہتے اور \عددی{f} سے ظاہر کرتے ہیں۔
\begin{align}
f=\frac{1}{T}
\end{align}
تعدد کی اکائی \اصطلاح{ہرٹز}\فرہنگ{ہرٹز}\حاشیہب{Hertz}\فرہنگ{Hertz} ہے جسے \عددی{\si{\hertz}} سے ظاہر کیا جاتا ہے۔

ایک چکر \عددی{2\pi} ریڈیئن کو کہتے ہیں لہٰذا \عددی{f} چکر سے مراد \عددی{2\pi f} ریڈیئن کا زاویہ ہے۔یوں \عددی{f} تعدد پر گردش کرتا نقطہ ایک سیکنڈ میں \عددی{2\pi f} ریڈیئن کا زاویہ طے کرے گا یعنی اس کی \اصطلاح{زاویائی رفتار}\فرہنگ{زاویائی رفتار}\حاشیہب{angular speed}\فرہنگ{angular speed} کی قیمت \عددی{2\pi f} ہو گی۔زاویائی رفتار کو \عددی{\omega} سے ظاہر کیا جاتا ہے جبکہ اس کی اکائی ریڈیئن فی سیکنڈ \عددی{\si{\radian\per\second}} ہے۔
\begin{align}
\omega=2\pi f
\end{align}
زاویائی رفتار \عددی{\omega} سے گردش کرتا ہوا نقطہ \عددی{t} سیکنڈ میں \عددی{2\pi f t} ریڈیئن کا زاویہ طے کرے گا۔یوں اگر \عددی{t=0} پر نقطہ عین \عددی{x} محدد کے مثبت حصے پر ہو تب لمحہ \عددی{t} پر
\begin{align}
\theta=\omega t=2\pi f t
\end{align}
لکھا جائے گا۔یوں مساوات  \حوالہ{مساوات_بدلتا_سائن_نما_تفاعل_الف} کو
\begin{gather}
\begin{aligned}\label{مساوات_بدلتا_سائن_نما_تفاعل_ب}
y(t)&=A_0 \sin 2\pi f t\\
&=A_0\sin \frac{2\pi}{T}t\\
&=A_0 \sin \omega t
\end{aligned}
\end{gather}
لکھا جا سکتا ہے۔

برقی میدان میں \عددی{y(t)} وقت کے ساتھ بدلتے دباو یا وقت کے ساتھ بدلتی رو کو ظاہر کر سکتی ہے۔مساوات \حوالہ{مساوات_بدلتا_سائن_نما_تفاعل_ب} میں دیے تفاعل، جسے شکل \حوالہ{شکل_بدلتی_رو_سائن_تفاعل}-ب میں دکھایا گیا ہے،  کا آزاد متغیرہ وقت \عددی{t} ہے۔آپ دیکھ سکتے ہیں کہ یہ تفاعل ہر \عددی{T} سیکنڈ کے بعد اپنے آپ کو دہراتا ہے۔ اس حقیقت کو ریاضی میں درج ذیل لکھا جاتا ہے۔
\begin{align}
y(t+T)=y( t)
\end{align}
جس سے مراد یہ ہے کہ تفاعل کی قیمت لمحہ \عددی{t} اور لمحہ \عددی{t+T} پر برابر ہیں۔

\begin{figure}
\centering
\begin{subfigure}{0.5\textwidth}
\centering
\begin{tikzpicture}
\pgfmathsetmacro{\ang}{50}
\pgfmathsetmacro{\rad}{1.8}
\pgfmathsetmacro{\xp}{\rad*cos(\ang)}
\pgfmathsetmacro{\yp}{\rad*sin(\ang)}
\draw[](0,0)--++(\rad+0.5,0)node[right]{$x$};
\draw[](0,0)--++(0,\rad+0.5)node[left]{$y$};
\draw[](0,0) circle (\rad);
\draw[fill](\ang:\rad) circle (1.5pt);
\draw(0,0)node[below left]{$o$}--++(\ang:\rad)node[pos=0.7,fill=white]{$A_0$}node[above right]{$a$};
\draw[-stealth] ([shift={(0:0.5)}]0,0) arc (0:\ang:0.5);
\draw(\ang/2:0.8)node{$\theta$};
\draw[dashed](\xp,\yp)--(\xp,0)node[below]{$x(t)$};
\draw[dashed](\xp,\yp)--(0,\yp)node[left]{$y(t)$};
\end{tikzpicture}
\caption*{الف}
\end{subfigure}
\begin{subfigure}{0.5\textwidth}
\centering
\begin{tikzpicture}
\begin{axis}[kStyleCircuitsA,small, xlabel=$t$, ylabel=$y(t)$, xtick={90,180,270,360},xticklabels={$\dfrac{T}{4}$,$\dfrac{T}{2}$,$\dfrac{3T}{2}$,$T$},ytick={-1,1},yticklabels={$-A_0$,$A_0$},
]
\addplot[domain=0:400,samples=100]{sin(x)};
\draw[gray,dashed](axis cs:0,1)--(axis cs:90,1);
\draw[gray,dashed](axis cs:0,-1)--(axis cs:270,-1);
\end{axis}%
\end{tikzpicture}%
\caption*{(ب)}
\end{subfigure}%
\begin{subfigure}{0.5\textwidth}
\centering
\begin{tikzpicture}
\begin{axis}[kStyleCircuitsA,small, xlabel=$\omega t$, ylabel=$y(\omega t)$, xtick={90,180,270,360},xticklabels={$\dfrac{\pi}{2}$,$\pi$,$\dfrac{3\pi}{2}$,$2\pi$},ytick={-1,1},yticklabels={$-A_0$,$A_0$},
]
\addplot[domain=0:400,samples=100]{sin(x)};
\draw[gray,dashed](axis cs:0,1)--(axis cs:90,1);
\draw[gray,dashed](axis cs:0,-1)--(axis cs:270,-1);
\end{axis}%
\end{tikzpicture}%
\caption*{(پ)}
\end{subfigure}%
\caption{سائن موج۔}
\label{شکل_بدلتی_رو_سائن_تفاعل}
\end{figure}%

مساوات \حوالہ{مساوات_بدلتا_سائن_نما_تفاعل_ب} کے خط کو \عددی{\omega t} کے ساتھ بھی کھینچا جا سکتا ہے۔ایسا ہی شکل \حوالہ{شکل_بدلتی_رو_سائن_تفاعل}-پ میں دکھایا گیا ہے جہاں سے واضح ہے کہ یہ تفاعل ہر \عددی{2\pi} ریڈیئن کے بعد اپنے آپ کو دہراتا ہے۔
%====================
\ابتدا{مشق}
شکل \حوالہ{شکل_بدلتی_رو_سائن_تفاعل}-الف میں گردش کرتا نقطہ \عددی{\SI{0.2}{\second}} میں \عددی{40^{\circ}} کا زاویہ طے کرتا ہے۔زاویائی رفتار، تعدد اور دوری عرصہ دریافت کریں۔

جوابات:\عددی{\omega=\tfrac{10\pi}{9}\,\si{\radian\per\second}}، \عددی{f=\SI{1.8}{\hertz}}، \عددی{T=\frac{5}{9} \, \si{\second}}
\انتہا{مشق}
%=================

شکل \حوالہ{شکل_بدلتا_لمحہ_صفر_پر_مقام_الفا_ہے} میں عمومی صورت حال دکھائی گئی ہے جہاں \عددی{\omega} زاویائی رفتار سے گردش کرتا نقطہ، لمحہ \عددی{t=0} پر  زاویہ \عددی{\alpha} پر پایا جاتا ہے۔یہ نقطہ وقت \عددی{t} کے دوران \عددی{\omega t} زاویہ طے کرتے ہوئے \عددی{\theta=\omega t+\alpha} پہنچ جائے گا لہٰذا اس کے لئے
\begin{align}\label{مساوات_بدلتا_سائن_نما_تفاعل_پ}
y(t)=A_0 \sin(\omega t +\alpha)
\end{align}
لکھا جا سکتا ہے جہاں \عددی{\alpha} کو \اصطلاح{زاویائی ہٹاو}\فرہنگ{زاویائی ہٹاو}\حاشیہب{phase angle}\فرہنگ{phase angle} کہتے ہیں۔اس مساوات کا دلیل \عددی{\omega t+\alpha} ہے۔شکل \حوالہ{شکل_بدلتا_لمحہ_صفر_پر_مقام_الفا_ہے}-ب میں مساوات \حوالہ{مساوات_بدلتا_سائن_نما_تفاعل_ب} اور مساوات \حوالہ{مساوات_بدلتا_سائن_نما_تفاعل_پ} کو دکھایا گیا ہے۔آپ دیکھ سکتے ہیں کہ ان مساوات میں \عددی{\alpha} \اصطلاح{زاویائی فرق}\فرہنگ{زاویائی فرق}\حاشیہب{phase difference}\فرہنگ{phase difference} پایا جاتا ہے۔ مساوات \حوالہ{مساوات_بدلتا_سائن_نما_تفاعل_ب} سے  مساوات \حوالہ{مساوات_بدلتا_سائن_نما_تفاعل_پ} \عددی{\alpha} ریڈیئن \اصطلاح{آگے}\فرہنگ{آگے}\حاشیہب{lead}\فرہنگ{lead} ہے۔ یہ بھی کہا جا سکتا ہے کہ مساوات \حوالہ{مساوات_بدلتا_سائن_نما_تفاعل_پ} سے مساوات \حوالہ{مساوات_بدلتا_سائن_نما_تفاعل_ب} \عددی{\alpha} ریڈیئن \اصطلاح{پیچھے}\فرہنگ{پیچھے}\حاشیہب{lag}\فرہنگ{lag} ہے۔ایک ہی تعدد کے دو تفاعل
\begin{gather}
\begin{aligned}
y_1(t)&=A_{01} \sin (\omega t +\alpha)\\
y_2(t)&=A_{02}\sin(\omega t+\beta)
\end{aligned}
\end{gather}
میں \عددی{y_1(t)} تفاعل \عددی{y_2(t)} سے  \عددی{\alpha-\beta} ریڈیئن  آگے ہے۔ہم یہ بھی کہہ سکتے ہیں کہ \عددی{y_2(t)} تفاعل \عددی{y_1(t)} سے 
 \عددی{\beta-\alpha}  ریڈیئن آگے ہے یا کہ \عددی{y_1(t)} تفاعل \عددی{y_2(t)} سے  \عددی{\beta-\alpha} ریڈیئن پیچھے ہے۔اگر \عددی{\alpha=\beta} ہو تب  تفاعل \اصطلاح{ہم زاویہ}\فرہنگ{ہم زاویہ}\حاشیہب{in phase}\فرہنگ{in phase}\فرہنگ{phase!in} کہلاتے ہیں جبکہ \عددی{\alpha\ne\beta} کی صورت میں تفاعل \اصطلاح{الگ زاویہ}\فرہنگ{الگ زاویہ}\حاشیہب{out of phase}\فرہنگ{out of phase}\فرہنگ{phase!out of} کہلاتے ہیں۔


\begin{figure}
\centering
\begin{subfigure}{0.5\textwidth}
\centering
\begin{tikzpicture}
\pgfmathsetmacro{\angA}{20}
\pgfmathsetmacro{\angB}{50}
\pgfmathsetmacro{\rad}{1.8}
\pgfmathsetmacro{\xp}{\rad*cos(\angB)}
\pgfmathsetmacro{\yp}{\rad*sin(\angB)}
\draw[](0,0)--++(\rad+1.5,0)node[right]{$x$};
\draw[](0,0)--++(0,\rad+0.5)node[left]{$y$};
\draw[](0,0) circle (\rad);
\draw[gray,fill](\angA:\rad) circle (1.5pt);
\draw[fill](\angB:\rad) circle (1.5pt);
\draw[dashed,gray](0,0)--++(\angA:\rad);
\draw(0,0)--++(\angB:\rad)node[pos=0.7,fill=white]{$A_0$};
\draw[dashed](\xp,\yp)--(0,\yp)node[left]{$y(t)$};
%angles
\draw[-stealth] ([shift={(0:0.7)}]0,0) arc (0:\angA:0.7);
\draw(\angA/2:0.9)node{$\alpha$};
\draw[-stealth] ([shift={(\angA:0.6)}]0,0) arc (\angA:\angB:0.6);
\draw(\angA/2+\angB/2:0.9)node{$\omega t$};
\draw[-stealth] ([shift={(0:\rad+0.5)}]0,0) arc (0:\angB:\rad+0.5);
\draw(\angB/2:\rad+0.8)node{$\theta$};
\end{tikzpicture}
\caption*{(الف)}
\end{subfigure}%
\begin{subfigure}{0.5\textwidth}
\centering
\begin{tikzpicture}
\begin{axis}[kStyleCircuitsA,small,xlabel=$t$, ylabel=$y(t)$,ytick={1},yticklabels={$A_0$},xtick={180,360},xticklabels={$\pi$,$2\pi$},]
\addplot[domain=0:370,samples=100,dashed]{sin(x)}node[pos=0.42,pin={[font=\small]10:${A_0\sin \omega t}$},inner sep=0pt]{};
\addplot[domain=-60:320,samples=100]{sin(x+60)}node[pos=0.35,pin={[pin distance=0.75cm,font=\small]10:${A_0\sin (\omega t+\alpha)}$},inner sep=0pt]{};
\draw(axis cs:-60,-0.1)--(axis cs:-60,-0.3);
\draw[stealth-](axis cs:0,-0.2)--(axis cs:30,-0.2);
\draw[stealth-](axis cs:-60,-0.2)--(axis cs:-80,-0.2)--(axis cs:-80,-0.4)node[below]{$\alpha$};
\end{axis}
\end{tikzpicture}
\caption*{(ب) زاویائی ہٹاو۔}
\end{subfigure}%
\caption{لمحہ \عددی{t=0} پر زاویہ \عددی{\alpha} ہے۔}
\label{شکل_بدلتا_لمحہ_صفر_پر_مقام_الفا_ہے}
\end{figure}

زاویائی ہٹاو کو عموماً درجوں میں بیان کیا جاتا ہے لہٰذا \عددی{\alpha=\tfrac{\pi}{4}} کی صورت میں درج ذیل لکھا جا سکتا  ہے۔
\begin{align}\label{مساوات_بدلتا_زاویہ_ہٹاو_روایتی_طریقہ}
y(t)=A_0 \sin \left(\omega t +\frac{\pi}{4}\right)=A_0 \sin\left(\omega t+45^{\circ}\right)
\end{align}
با ضابطہ طور پر چونکہ \عددی{\omega t} کی قیمت ریڈیئن میں ہے لہٰذا \عددی{\alpha} کی قیمت بھی ریڈیئن میں ہونا لازم ہے لہٰذا تفاعل لکھنے کا صحیح طریقہ \عددی{y(t)=A_0\sin\left(\omega t+\tfrac{\pi}{4}\right)} ہی ہے لیکن زاویائی ہٹاو کو درجوں میں لکھنے کی روایت نہایت مقبول ہے لہٰذا اس کتاب میں بھی اس روایت کو برقرار رکھا جائے گا۔مساوات \حوالہ{مساوات_بدلتا_زاویہ_ہٹاو_روایتی_طریقہ} میں \عددی{45^{\circ}} لکھتے ہوئے زیر بالا میں درجے کی علامت \عددی{(^\circ)} استعمال کی گئی ہے جبکہ \عددی{\tfrac{\pi}{4}} پر کوئی علامت نہیں لگائی گئی۔اسی علامت سے ریڈیئن یا درجوں کی پہچان کی جاتی ہے۔
%================
\ابتدا{مثال}
مساوات \عددی{y_1(t)=15\sin(100t+60^{\circ})} اور \عددی{y_2(t)=22\sin(200t +0.2\pi)} کی قیمت \عددی{t=\SI{25}{\milli\second}} پر دریافت کریں۔

حل:پہلی تفاعل میں \عددی{50^{\circ}} کا زاویائی ہٹاو \عددی{\tfrac{60^{\circ}}{180^{\circ}}\times \pi=\tfrac{\pi}{3}} ریڈیئن کے برابر ہے۔یوں
 لمحہ \عددی{t=\SI{25}{\milli\second}} پر
\begin{align*}
y_1(0.025)&=15\sin\left(100\times 25\times 10^{-3}+\frac{\pi}{3}\right)=-5.918619766
\end{align*}
اور
\begin{align*}
y_2(0.025)&=22\sin(200\times 0.025+0.2\pi)=-13.39917888
\end{align*}
حاصل ہوتے ہیں۔
\انتہا{مثال}
%================== 

اگرچہ اب تک کی بحث میں ہم نے سائن تفاعل استعمال کیا، ہم اس کی جگہ کوسائن تفاعل بھی استعمال کر سکتے تھے۔ان دو تفاعل کی صورت بالکل یکساں ہے پس دونوں میں \عددی{90^{\circ}} کا زاویائی فرق پایا جاتا ہے۔
\begin{align}
\sin \left(\omega t+\frac{\pi}{2}\right)&=\cos \omega t\\
\cos \left(\omega t-\frac{\pi}{2}\right)&=\sin \omega t
\end{align}
سائن نما تفاعل کے دلیل  کے ساتھ \عددی{2\pi} ریڈیئن یا \عددی{360^{\circ}} کا مضرب جمع کرنے سے  تفاعل کی قیمت تبدیل نہیں ہوتی۔
\begin{align}
\cos(\omega t +\alpha+2\pi n)&=\cos(\omega t +\alpha) \quad \quad  n=0,\pm 1, \pm 2, \cdots \label{مساوات_بدلتا_طول_بعد_وہی_موج}\\
\sin(\omega t +\alpha+2\pi n)&=\sin(\omega t +\alpha) \quad \quad  n=0,\pm 1, \pm 2, \cdots \label{مساوات_بدلتا_طول_بعد_وہی_موج_ب}
\end{align}

دو سائن نما تفاعل میں زاویائی فرق  تین شرائط پورا کرنے کے بعد دریافت کیا جا سکتا ہے۔پہلی شرط یہ ہے کہ دونوں تفاعل کی تعدد برابر ہو۔دوسری شرط یہ ہے کہ دونوں کو  سائن تفاعل اور یا پھر دونوں کو  کوسائن تفاعل کی صورت میں لکھا جائے۔تیسری اور آخری شرط یہ ہے کہ دوسری شرط میں لکھے گئے تفاعل کے حیطے مثبت ہوں۔درج ذیل مماثل ان شرائط کو پورا کرنے میں مدد دیتے ہیں۔
\begin{align}
-\sin(\omega t+\alpha)&=\sin(\omega t+\alpha \pm 180^{\circ})\label{مساوات_بدلتا_آدھی_طول_منفی_موج_الف}\\
-\cos(\omega t+\alpha)&=\cos(\omega t+\alpha \pm 180^{\circ})\label{مساوات_بدلتا_آدھی_طول_منفی_موج_ب}
\end{align}
ان کے علاوہ درج ذیل مماثل بھی نہایت اہم ثابت ہوتے ہیں۔
\begin{align}
\sin(\alpha\pm\beta)&=\sin \alpha \cos \beta \pm \cos \alpha \sin \beta \label{مساوات_بدلتا_آدھی_طول_منفی_موج_پ}\\
\cos(\alpha \pm \beta)&=\cos \alpha \cos \beta \mp \sin \alpha \sin \beta \label{مساوات_بدلتا_آدھی_طول_منفی_موج_ت}
\end{align}
ایک آخری تفاعل جس کا ذکر ضروری ہے درج ذیل ہے۔
\begin{align} \label{مساوات_بدلتا_آدھی_طول_منفی_موج_ٹ}
\cos^2 \alpha+\sin^2 \alpha=1
\end{align}
%===============
\ابتدا{مثال}\شناخت{مثال_بدلتا_کوسائن_تفاعل_بالمقابل_ہٹاو}
درج ذیل تفاعل کے خط کھینچیں۔
\begin{itemize}
\item
$v(t)=1 \cos (\omega t +60^{\circ})$
\item
$v(t)=1 \cos (\omega t +240^{\circ})$
\item
$v(t)=1 \cos (\omega t -300^{\circ})$
\end{itemize}

حل: شکل \حوالہ{شکل_بدلتا_کوسائن_تفاعل_بالمقابل_ہٹاو}-الف میں \عددی{v(\omega t)=1\cos \omega t} کا خط دکھایا گیا ہے۔اس کو افقی محدد پر \عددی{60^{\circ}} درجے  بائیں منتقل کرنے سے \عددی{v(\omega t)=1\cos(\omega t +60^{\circ})} کا خط حاصل ہوتا ہے جسے شکل-ب میں دکھایا گیا ہے۔ ہم درج ذیل لکھ سکتے ہیں
\begin{align*}
v(\omega t)=1\cos (\omega t +240^{\circ})=1\cos (\omega t +60^{\circ}+180^{\circ})=-1\cos (\omega t +60^{\circ})
\end{align*}
جہاں مساوات \حوالہ{مساوات_بدلتا_آدھی_طول_منفی_موج_ب} کا استعمال کیا گیا ہے۔درج بالا مساوات کو شکل-پ میں دکھایا گیا ہے۔آپ دیکھ سکتے ہیں کہ یہ شکل-ب کا منفی ہے۔اسی طرح مساوات \حوالہ{مساوات_بدلتا_طول_بعد_وہی_موج} کی مدد سے
\begin{align*}
v(\omega t)=1\cos(\omega t-300^{\circ})=1\cos(\omega t-300^{\circ}+360^{\circ})=1\cos(\omega t+60^{\circ})
\end{align*}
لکھتے ہوئے شکل-ت حاصل ہوتی ہے جو عین شکل-ب ہی ہے۔
\begin{figure}
\centering
\begin{subfigure}{0.5\textwidth}
\centering
\begin{tikzpicture}
\begin{axis}[kStyleCircuitsA,small,xlabel=$\omega t$, ylabel=$v(\omega t)$,xtick={90,180,270,360}, xticklabels={$90^{\circ}$,$180^{\circ}$,$270^{\circ}$,$360^{\circ}$},ytick={-1,1},yticklabels={$-1$,$1$},]
\addplot[domain=0:360,samples=100]{1*cos(x)}node[pos=0.1,pin={[font=\small]10:${\cos \omega t}$},inner sep=0pt]{};
\end{axis}
\end{tikzpicture}
\caption*{(الف)}
\end{subfigure}%
\begin{subfigure}{0.5\textwidth}
\centering
\begin{tikzpicture}
\begin{axis}[kStyleCircuitsA,small,xlabel=$\omega t$, ylabel=$v(\omega t)$,xtick={-60,30,120,210,300}, xticklabels={$-60^{\circ}$,$30^{\circ}$,$120^{\circ}$,$210^{\circ}$,$300^{\circ}$},ytick={-1,1},yticklabels={$-1$,$1$},]
\addplot[domain=-60:300,samples=100]{1*cos(x+60)}node[pos=0.85,pin={[font=\small]170:${\cos (\omega t+60^{\circ})}$},inner sep=0pt]{};
\end{axis}
\end{tikzpicture}
\caption*{(ب)}
\end{subfigure}
\begin{subfigure}{0.5\textwidth}
\centering
\begin{tikzpicture}
\begin{axis}[kStyleCircuitsA,small,xlabel=$\omega t$, ylabel=$v(\omega t)$,xtick={-60,30,120,210,300},
 xticklabels={$-60^{\circ}$,$30^{\circ}$,$120^{\circ}$,$210^{\circ}$,$300^{\circ}$},ytick={-1,1},yticklabels={$-1$,$1$},]
\addplot[domain=-60:300,samples=100]{1*cos(x+240)}node[pos=0.85,pin={[font=\small]-170:${\cos( \omega t+240^{\circ})}$},inner sep=0pt]{};
\end{axis}
\end{tikzpicture}
\caption*{(پ)}
\end{subfigure}%
\begin{subfigure}{0.5\textwidth}
\centering
\begin{tikzpicture}
\begin{axis}[kStyleCircuitsA,small,xlabel=$\omega t$, ylabel=$v(\omega t)$,xtick={-60,30,120,210,300}, 
xticklabels={$-60^{\circ}$,$30^{\circ}$,$120^{\circ}$,$210^{\circ}$,$300^{\circ}$},ytick={-1,1},yticklabels={$-1$,$1$},]
\addplot[domain=-60:300,samples=100]{1*cos(x-300)}node[pos=0.85,pin={[font=\small]170:${\cos( \omega t-300^{\circ})}$},inner sep=0pt]{};
\end{axis}
\end{tikzpicture}
\caption*{(ت)}
\end{subfigure}%
\caption{مثال \حوالہ{مثال_بدلتا_کوسائن_تفاعل_بالمقابل_ہٹاو} کے خط۔}
\label{شکل_بدلتا_کوسائن_تفاعل_بالمقابل_ہٹاو}
\end{figure}
\انتہا{مثال}
%==========================
\ابتدا{مثال}
درج ذیل امواج کی تعدد ہرٹز میں حاصل کریں۔ امواج کے مابین زاویائی فرق دریافت کریں۔یہ بھی بتلائیں کہ کونسی موج آگے ہے۔
\begin{align*}
v_1(\omega t)&=100\sin(400t -30^{\circ})\\
v_2(\omega t)&=-250\cos(400t+0.2\pi)
\end{align*}

حل:ان امواج میں \عددی{\omega=\SI{400}{\radian\per\second}} ہے لہٰذا
\begin{align*}
f=\frac{\omega}{2\pi}=\frac{400}{2\pi}=\SI{63.66}{\hertz}
\end{align*}
ہو گا۔زاویائی فرق دریافت کرنے کی خاطر دونوں امواج کو مثبت حیطے کے کوسائن موج کی صورت میں لکھتے ہیں۔ساتھ ہی ساتھ ان کے زاویائی ہٹاو کو درجوں میں لکھتے ہیں۔یوں
\begin{align*}
v_1(\omega t)&=100\sin(400t -30^{\circ})\\
&=100\cos(400t-30^{\circ}-90^{\circ})\\
&=100\cos(400t-120^{\circ})\\
&=100\cos(400t+240^{\circ})
\end{align*}
لکھا جا سکتا ہے جہاں آخری قدم پر مساوات \حوالہ{مساوات_بدلتا_طول_بعد_وہی_موج} کا استعمال کیا گیا۔اسی طرح
\begin{align*}
v_2(\omega t)&=-250\cos(400t+0.2\pi)\\
&=250\cos(400t+0.2\pi+\pi)\\
&=250\cos(400t+216^{\circ})
\end{align*}
بھی لکھا جا سکتا ہے جہاں آخری قدم پر \عددی{1.2\pi} ریڈیئن کو \عددی{216^{\circ}} درجے لکھا گیا ہے۔ان امواج کے مابین
\begin{align*}
240^{\circ}-216^{\circ}=24^{\circ}
\end{align*}
کا زاویائی فرق پایا جاتا ہے اور موج \عددی{v_1(\omega t)} آگے ہے۔
\انتہا{مثال}
%========================
\ابتدا{مشق}
ایک دور میں درج ذیل تین رو پائے جاتے ہیں۔
\begin{align*}
i_1(1)&=30\cos(100\pi t+30^{\circ})\\
i_2(2)&=55\sin(100\pi t +40^{\circ})\\
i_3(t)&=20\sin(100 \pi t+60^{\circ})
\end{align*}
\عددی{i_2} سے \عددی{i_1} کتنی آگے ہے اور \عددی{i_3} سے \عددی{i_1} کتنی پیچھے ہے۔

جوابات:\عددی{80^{\circ}}، \عددی{-60^{\circ}} یا \عددی{300^{\circ}} 
\انتہا{مشق}
%=========================
\حصہ{سائن نما اور مخلوط جبری تفاعل}
گزشتہ باب میں دور پر مستقل جبری تفاعل مسلط  کرتے ہوئے، دور کا جبری ردعمل بھی مستقل قیمت کا حاصل ہوا۔تفرقی مساوات کا جبری ردعمل، مسلط جبری تفاعل اور اس کے تمام بلند درجی تفرق کا مجموعہ ہوتا ہے۔یوں  دور پر جبری دباو \عددی{v(t)=\sin \omega t} مسلط کرنے سے رو کا جبری ردعمل \عددی{i(t)=c_1\sin\omega t+c_2\cos\omega t} متوقع ہو گا۔پس جبری ردعمل کے مستقل \عددی{c_1} اور \عددی{c_2} معلوم کرنا باقی ہے۔
%===============
\ابتدا{مثال}\شناخت{مثال_بدلتا_مزاحمت_امالہ_جبری_حل_الف}
شکل \حوالہ{شکل_بدلتا_مزاحمت_امالہ_جبری_حل_الف} میں رو \عددی{i_J(t)} حاصل کریں۔

\begin{figure}
\centering
\begin{tikzpicture}
\draw(0,0) to [american voltage source,l={${v(t)=V_0\cos \omega t}$}]++(0,\y) to [resistor,i={$i(t)$},l={$R$}]++(\x,0) to [inductor,l={$L$}]++(0,-\y) to [short] (0,0);
\end{tikzpicture}
\caption{مثال \حوالہ{مثال_بدلتا_مزاحمت_امالہ_جبری_حل_الف} کا دور۔}
\label{شکل_بدلتا_مزاحمت_امالہ_جبری_حل_الف}
\end{figure}

حل: دور کی تفرقی مساوات لکھتے ہیں۔
\begin{align}\label{مساوات_بدلتا_مزاحمت_امالہ_جبری_حل_الف}
R i(t)+L \frac{\dif i(t)}{\dif t}=V_0 \cos \omega t
\end{align}
دور پر مسلط جبری تفاعل اور اس تفاعل کے تمام بلند درجی تفرق کا مجموعہ جبری حل کے برابر ہو گا۔
\begin{align*}
i_J(t)&=c_1 \cos \omega t+c_2\sin \omega t
\end{align*} 
اس جبری حل کو مساوات \حوالہ{مساوات_بدلتا_مزاحمت_امالہ_جبری_حل_الف} میں پُر کرتے ہوئے \عددی{c_1} اور \عددی{c_2} مستقل دریافت کرتے ہیں۔ 
\begin{align*}
R(c_1 \cos \omega t+c_2\sin \omega t)+L (-c_1 \omega \sin\omega t+c_2 \omega \cos \omega t)=V_0 \cos \omega t
\end{align*}
درج بالا مساوات میں دونوں اطراف \عددی{\cos \omega t} کے  عددی سر برابر ہوں گے۔اسی طرح دونوں اطراف \عددی{\sin \omega t} کے عددی سر برابر ہوں گے۔
\begin{align*}
c_1 R+c_2 \omega L&=V_0\\
-c_1 \omega L+c_2 R&=0
\end{align*}
ان ہمزاد مساوات کو \عددی{c_1} اور \عددی{c_2} کے لئے حل کرتے ہوئے درج ذیل ملتا ہے
\begin{align*}
c_1&=\frac{R V_0}{R^2+\omega^2 L^2}\\
c_2&=\frac{\omega L V_0}{R^2+\omega^2 L^2}
\end{align*}
لہٰذا جبری حل
\begin{align}\label{مساوات_بدلتا_جبری_حل_الف}
i_J(t)=\frac{R V_0}{R^2+\omega^2 L^2} \cos \omega t+\frac{\omega L V_0}{R^2+\omega^2 L^2} \sin \omega t
\end{align}
ہو گا۔
\انتہا{مثال}
%=================
\ابتدا{مثال}\شناخت{مثال_بدلتا_مزاحمت_امالہ_جبری_حل_ب}
درج بالا مثال میں \عددی{R=\SI{100}{\ohm}}، \عددی{L=\SI{5}{\milli\henry}}، \عددی{V_0=\SI{310}{\volt}} اور \عددی{\omega=\SI{10}{\kilo\radian\per\second}} کی صورت میں جبری حل کو مساوات \حوالہ{مساوات_بدلتا_آدھی_طول_منفی_موج_ت} کی مدد سے \عددی{i(t)=I_0\cos(\omega t-\phi)} کے طرز پر لکھیں۔

حل: مساوات \حوالہ{مساوات_بدلتا_جبری_حل_الف} میں دی گئی قیمتیں پُر کرنے سے
\begin{align*}
i_J(t)&=\frac{100\times 310}{100^2+(\num{10000}\times 0.005)^2} \cos \omega t+\frac{\num{10000}\times 0.005\times 310}{100^2+(\num{10000}\times 0.005)^2} \sin \omega t
\end{align*}
یعنی درج ذیل حاصل ہوتا ہے۔
\begin{align}\label{مساوات_بدلتا_جبری_حل_ب}
i_J(t)=2.48\cos \omega t+1.24\sin \omega t
\end{align}
مساوات \حوالہ{مساوات_بدلتا_آدھی_طول_منفی_موج_ت} سے  جبری حل کی درکار صورت کو درج ذیل لکھا جا سکتا ہے۔ 
\begin{align}\label{مساوات_بدلتا_جبری_حل_پ}
i(t)=I_0 \cos(\omega t-\phi)=I_0 \cos \phi \cos \omega t+I_0 \sin \phi \sin \omega t
\end{align}
مساوات \حوالہ{مساوات_بدلتا_جبری_حل_ب} میں \عددی{\cos \omega t} اور \عددی{\sin \omega t} کے عددی سر کو مساوات \حوالہ{مساوات_بدلتا_جبری_حل_پ} کے عددی سر کے برابر پُر کرتے ہیں۔
\begin{align}
I_0 \cos \phi &=2.48 \label{مساوات_بدلتا_جبری_حل_ت}\\
I_0 \sin \phi&=1.24 \label{مساوات_بدلتا_جبری_حل_ٹ}
\end{align}
ان ہمزاد مساوات کے مربع جمع کرتے ہوئے 
\begin{align*}
I_0^2 \cos^2 \phi+I_0^2 \sin^2 \phi=2.48^2+1.24^2
\end{align*}
ملتا ہے جس میں مساوات \حوالہ{مساوات_بدلتا_آدھی_طول_منفی_موج_ٹ} کے استعمال سے  \عددی{\cos^2 \phi+\sin^2 \phi=1} پُر کرتے ہوئے
\begin{align*}
I_0=\sqrt{2.48^2+1.24^2}=2.7727
\end{align*}
ملتا ہے۔اسی طرح مساوات \حوالہ{مساوات_بدلتا_جبری_حل_ٹ} کو مساوات \حوالہ{مساوات_بدلتا_جبری_حل_ت} سے تقسیم کرنے سے
\begin{align*}
\frac{\sin \phi}{\cos \phi}=\frac{1.24}{2.48}=\tan \phi
\end{align*}
یعنی
\begin{align*}
\phi=\tan ^{-1}\frac{1.24}{2.48}=\phase{26.6^{\circ}}
\end{align*}
ملتا ہے۔یوں جبری حل درج ذیل لکھا جائے گا
\begin{align}
i_J(t)=2.77\cos(\omega t-26.6^{\circ})=2.77\cos(\num{10000} t-26.6^{\circ})
\end{align}
جہاں سے ظاہر ہے کہ دباو سے رو \عددی{26.6^{\circ}} درجے پیچھے ہے۔ مخلوط جبری حل درج ذیل لکھا جائے گا جس کا حقیقی جزو درج بالا مساوات ہے۔
\begin{align}\label{مساوات_بدلتا_مخلوط_رو_صورت}
i_M(t)=2.77 e^{j(\num{10000} t-26.6^{\circ})}
\end{align}
\انتہا{مثال}
%===================
\ابتدا{مثال}\شناخت{مثال_بدلتا_مزاحمت_امالہ_جبری_حل_پ}
مثال \حوالہ{مثال_بدلتا_مزاحمت_امالہ_جبری_حل_ب} کے طرز پر مثال \حوالہ{مثال_بدلتا_مزاحمت_امالہ_جبری_حل_الف} میں حاصل کئے گئے جبری حل کو \عددی{i_J(t)=I_0\cos(\omega t -\phi)} کی صورت میں لکھیں۔

حل:مساوات \حوالہ{مساوات_بدلتا_جبری_حل_الف} میں \عددی{\cos \omega t} اور \عددی{\sin \omega t} کے عددی سر کو مساوات \حوالہ{مساوات_بدلتا_جبری_حل_پ} میں \عددی{\cos \omega t} اور \عددی{\sin \omega t} کے عددی سر کے برابر پُر کرتے ہوئے درج ذیل ملتا ہے۔
\begin{align*}
I_0 \cos \phi&=\frac{R V_0}{R^2+\omega^2 L^2}\\
I_0 \sin \phi&=\frac{\omega L V_0}{R^2+\omega^2 L^2}
\end{align*}
ان ہمزاد مساوات میں دوسری مساوات کو پہلی سے تقسیم کرتے ہوئے
\begin{align*}
\frac{\sin \phi}{\cos \phi}=\tan \phi=\frac{\omega L}{R}
\end{align*}
یعنی
\begin{align}
\phi=\tan^{-1}{\frac{\omega L}{R}}
\end{align}
ملتا ہے جبکہ دونوں ہمزاد مساوات کے مربع کا مجموعہ لیتے ہوئے
\begin{align*}
I_0^2 \cos^2 \phi+I_0^2 \sin^2 \phi=I_0^2&=\left(\frac{R V_0}{R^2+\omega^2 L^2}\right)^2+\left(\frac{\omega L V_0}{R^2+\omega^2 L^2}\right)^2 \\
&=\frac{(R^2+\omega^2 L^2)V_0^2}{(R^2+\omega^2 L^2)^2}\\
&=\frac{V_0^2}{R^2+\omega^2 L^2}
\end{align*}
یعنی
\begin{align}
I_0=\frac{V_0}{\sqrt{R^2+\omega^2 L^2}}
\end{align}
ملتا ہے۔یوں جبری حل درج ذیل لکھا جائے گا۔
\begin{align}\label{مساوات_بدلتا_جبری_حل_امالہ_مزاحمت}
i_J(t)=\frac{V_0}{\sqrt{R^2+\omega^2 L^2}} \cos \left(\omega t -\tan^{-1}{\frac{\omega L}{R}}\right)
\end{align}
\انتہا{مثال}
%========================

مساوات \حوالہ{مساوات_بدلتا_جبری_حل_امالہ_مزاحمت} سے ظاہر ہے کہ \عددی{L=0} کی صورت میں \عددی{\phi=0} ہو گا لہٰذا دباو اور رو ہم زاویہ ہوں گے جبکہ \عددی{R=0} کی صورت میں \عددی{\phi=90^{\circ}} ہو گا لہٰذا دباو سے رو \عددی{90^{\circ}} درجے پیچھے ہو گی۔مزاحمت اور امالہ کے دیگر قیمتوں کی صورت میں دباو سے رو \عددی{0^{\circ}} تا \عددی{90^{\circ}} کے مابین کسی مخصوص  درجے پر پیچھے رہے گی۔اسی لئے مزاحمت اور امالہ کے ادوار کو پیچھے رہنے والے ادوار کہا جاتا ہے۔

سلسلہ وار جڑے مزاحمت اور امالہ کے دور کا حل آپ نے دیکھا۔یقیناً اس دور کا حل سلسلہ وار جڑے دو عدد مزاحمتی دور کے حل سے کئی گنا مشکل تھا۔آپ خود تصور کر سکتے ہیں کہ زیادہ تعداد کے پرزوں کا دور حل کرنا کتنا مشکل ہو گا۔اسی مشکل کو مد نظر رکھتے ہوئے ہم \اصطلاح{مخلوط تفاعل}\فرہنگ{مخلوط تفاعل}\حاشیہب{complex function}\فرہنگ{complex function} کو پیش کرتے ہیں جس سے ادوار کا حل انتہائی آسان ثابت ہوتا ہے۔

مخلوط تفاعل اور سائن نما تفاعل کا تعلق \اصطلاح{یولر مساوات}\فرہنگ{یولر مساوات}\حاشیہب{Euler's equation}\فرہنگ{Euler's equation}
\begin{align}
e^{j\omega t}=\cos \omega t +j \sin \omega t \quad \quad \text{\RL{یولر مساوات}}
\end{align}
دیتی ہے جہاں \عددی{j=\sqrt{-1}} خیالی عدد ہے۔یولر مساوات میں \عددی{\cos \omega t} \اصطلاح{حقیقی}\فرہنگ{حقیقی}\حاشیہب{real}\فرہنگ{real} مقدار اور \عددی{\sin \omega t} \اصطلاح{خیالی}\فرہنگ{خیالی}\حاشیہب{imaginary}\فرہنگ{imaginary} مقدار ہیں۔

حقیقی دنیا میں مخلوط جبری تفاعل نہیں پایا جاتا۔اس کے باوجود، دور پر سائن نما جبری تفاعل کی جگہ مخلوط جبری تفاعل مسلط کرتے ہوئے  مخلوط حل حاصل کیا جا سکتا ہے۔مخلوط جبری تفاعل کو حقیقی جبری تفاعل اور خیالی جبری تفاعل کا مجموعہ تصور کیا جا سکتا ہے۔خطی ادوار میں مسئلہ نفاذ کے تحت تمام جبری تفاعل کی علیحدہ علیحدہ اثرات کا مجموعہ لیا جا سکتا ہے۔یوں جبری تفاعل کے حقیقی جزو سے حل کا حقیقی جزو جبکہ جبری تفاعل کے خیالی جزو سے حل کا خیالی جزو حاصل ہو گا۔یوں مخلوط حل کے خیالی جزو کو رد کرتے ہوئے حقیقی جزو کو سائن نما تفاعل کا ردعمل تسلیم کیا جاتا ہے۔اس ترکیب کو مثال کی مدد سے زیادہ آسانی سے سمجھا جا سکتا ہے۔
%====================
\ابتدا{مثال}\شناخت{مثال_بدلتا_مخلوط_تفاعل_الف}
شکل \حوالہ{شکل_بدلتا_مزاحمت_امالہ_جبری_حل_الف} میں حقیقی جبری تفاعل \عددی{V_0\cos \omega t} کی جگہ مخلوط جبری تفاعل نسب کرتے ہوئے حقیقی \عددی{i(t)} کے لئے حل کریں۔

حل:حقیقی جبری تفاعل \عددی{v(t)=V_0\cos \omega t} کی جگہ دور میں مخلوط جبری تفاعل \عددی{v(t)=V_0 e^{j\omega t}} نسب کرتے ہوئے کرخوف مساوات لکھتے ہیں۔
\begin{align*}
R i(t) +L\frac{\dif i(t)}{\dif t}=V_0 e^{j \omega t}
\end{align*}
جبری تفاعل \عددی{e^{j\omega t}} کا تفرق \عددی{j\omega e^{j\omega t}} بھی جبری تفاعل ہی ہے لہٰذا درج بالا مساوات کا مخلوط حل \عددی{i_M(t)=I_0e^{j\omega t}} فرض  کرتے ہیں جہاں \عددی{I_0} نا معلوم مخلوط مستقل ہے۔اس حل کو درج بالا مساوات میں پُر کرتے ہوئے
\begin{align*}
R I_0 e^{j\omega t}+L \frac{\dif}{\dif t}\left(I_0 e^{j\omega t} \right)=V_0 e^{j \omega t}
\end{align*}
درکار تفرق کے بعد
\begin{align}\label{مساوات_بدلتا_مخلوط_الف}
R I_0  e^{j \omega t}+j \omega L I_0e^{j\omega t}&=V_0 e^{j \omega t}
\end{align}
ملتا ہے جس کے دونوں اطراف کو \عددی{e^{j\omega t}} سے تقسیم کرتے ہوئے درج ذیل ملتا ہے۔
\begin{align}\label{مساوات_بدلتا_مخلوط_ب}
R I_0+j \omega L I_0&=V_0 
\end{align}
اس سے \عددی{I_0} حاصل کرتے ہیں۔
\begin{align}\label{مساوات_بدلتا_مخلوط_مستقل_الف}
I_0=\frac{V_0}{R+j\omega L}
\end{align}
یوں مخلوط رو درج ذیل لکھی جا سکتی ہے۔
\begin{gather}
\begin{aligned}\label{مساوات_بدلتا_مخلوط_پ}
i_M(t)&=I_0 e^{j\omega t}\\
&=\frac{V_0 e^{j\omega t}}{R+j\omega L}
\end{aligned}
\end{gather}
ہمیں اس کا حقیقی جزو درکار ہے۔یولر مساوات کی مدد سے درج بالا مساوات کو درج ذیل لکھا جا سکتا ہے۔
\begin{align*}
i_M(t)&=\frac{V_0 (\cos \omega t+j \sin \omega t)}{R+j\omega L}
\end{align*}
دائیں ہاتھ کسر کے بالائی اور نچلے حصے کو \عددی{R-j\omega t} سے ضرب دیتے ہیں
 \begin{align*}
i_M(t)&=\frac{V_0 (\cos \omega t+j \sin \omega t)(R-j\omega L)}{(R+j\omega L)(R-j\omega L)}\\
&=\frac{V_0(R \cos \omega t+\omega L \sin \omega t)+jV_0(R\sin \omega t-\omega L \cos \omega t)}{R^2+\omega^2 L^2}
\end{align*}
جہاں دوسرا قدم ترتیب دیتے ہوئے لکھا گیا ہے۔اس کا حقیقی جزو درکار حل ہے
\begin{align}\label{مساوات_بدلتا_مخلوط_ت}
i(t)=\frac{V_0(R \cos \omega t+\omega L \sin \omega t)}{R^2+\omega^2 L^2}
\end{align}
جو عین مساوات \حوالہ{مساوات_بدلتا_جبری_حل_الف} ہی ہے۔

ہم مساوات \حوالہ{مساوات_بدلتا_مخلوط_مستقل_الف} کے مخلوط مستقل \عددی{I_0} کو زاویائی شکل میں لکھ کر بھی آگے بڑھ سکتے ہیں۔مخلوط مستقل کو درج ذیل لکھا جا سکتا ہے
\begin{align*}
I_0&=\frac{V_0}{R+j\omega L}\\
&=\frac{V_0}{\sqrt{R^2+\omega^2 L^2} \phase{\tan^{-1}\frac{\omega L}{R}}}\\
&=\frac{V_0}{\sqrt{R^2+\omega^2 L^2}\,  e^{j\tan^{-1}\frac{\omega L}{R}}}\\
&=\frac{V_0}{\sqrt{R^2+\omega^2 L^2}} e^{-j\tan^{-1}\frac{\omega L}{R}}
\end{align*}
جہاں دوسری قدم پر کسر کے نچلی حصے کو مساوات \حوالہ{مساوات_بدلتا_مستطیل_سے_زاویائی} کی مدد سے زاویائی صورت میں لکھا گیا ہے اور تیسری قدم پر یولر مساوات کا استعمال کیا گیا ہے۔زاویہ \عددی{\theta=\tan^{-1} \tfrac{\omega L}{R}} کو شکل \حوالہ{شکل_بدلتا_مخلوط_تفاعل_الف} میں دکھایا گیا ہے۔ یوں مخلوط رو درج ذیل لکھی جائے گی۔
\begin{align*}
i_M&=I_0 e^{j\omega t}\\
&=\frac{V_0}{\sqrt{R^2+\omega^2 L^2}} e^{j\left(\omega t-\tan^{-1}\frac{\omega L}{R}\right)} 
\end{align*}
اس مساوات میں \عددی{\tan^{-1}\tfrac{\omega L}{R}=\theta} لکھتے ہوئے حقیقی جزو لے کر حقیقی رو حاصل کرتے ہیں۔
\begin{align*}
i(t)&=\left. \frac{V_0}{\sqrt{R^2+\omega^2 L^2}} e^{j\left(\omega t-\theta\right)} \right|_{\text{حقیقی}} \\
&=\frac{V_0}{\sqrt{R^2+\omega^2 L^2}} \cos (\omega t-\theta)\\
&=\frac{V_0}{\sqrt{R^2+\omega^2 L^2}} \left(\cos \omega t \cos \theta+\sin \omega t \sin \theta \right)
\end{align*}
شکل \حوالہ{شکل_بدلتا_مخلوط_تفاعل_الف} سے  \عددی{\cos \theta=\tfrac{R}{\sqrt{R^2+\omega^2L^2}}} اور
 \عددی{\sin \theta=\tfrac{\omega L}{\sqrt{R^2+\omega^2L^2}}} پُر کرتے ہوئے
\begin{align*}
i(t)&=\frac{V_0}{\sqrt{R^2+\omega^2 L^2}} \left(\cos \omega t \frac{R}{\sqrt{R^2+\omega^2 L^2}}+\sin \omega t \frac{\omega L}{\sqrt{R^2+\omega^2 L^2}}\right)\\
&=\frac{V_0 (R \cos \omega t +\omega L \sin \omega t)}{R^2+\omega^2 L^2}
\end{align*}
%
\begin{figure}
\centering
\begin{tikzpicture}
\pgfmathsetmacro{\l}{3};
\pgfmathsetmacro{\ang}{20};
\pgfmathsetmacro{\x}{\l*cos(\ang)};
\pgfmathsetmacro{\y}{\l*sin(\ang)};
\draw(0,0)--++(4,0)node[right]{حقیقی};
\draw(0,0)--++(0,1.5)node[left]{خیالی};
\draw(0,0)--++(\ang:\l)coordinate(kp)node[pos=0.5,above,sloped]{$\sqrt{R^2+\omega^2 L^2}$};
\draw(\x,0)--(kp)node[pos=0.5,right]{$\omega L$};
\draw(\x/2,0)node[below]{$R$};
\draw[-stealth]([shift={(0:0.8)}]0,0) arc (0:\ang:0.8);
\draw(\ang/2:1)node{$\theta$};
\end{tikzpicture}
\caption{مثال \حوالہ{مثال_بدلتا_مخلوط_تفاعل_الف} کا شکل۔}
\label{شکل_بدلتا_مخلوط_تفاعل_الف}
\end{figure}
\انتہا{مثال}
%================

\حصہ{دوری سمتیہ}
درج بالا حصے میں ہم نے دیکھا کہ حقیقی جبری تفاعل کی جگہ مخلوط جبری تفاعل نسب کرتے ہوئے مخلوط حل حاصل کیا جا سکتا ہے جس کا حقیقی جزو حقیقی جبری رد عمل ہو گا۔ اس ترکیب کو مثال \حوالہ{مثال_بدلتا_مخلوط_تفاعل_الف} میں استعمال کیا گیا جہاں مساوات \حوالہ{مساوات_بدلتا_مخلوط_الف} کو \عددی{e^{j\omega t}} سے تقسیم کرتے ہوئے مساوات \حوالہ{مساوات_بدلتا_مخلوط_ب} حاصل کی گئی۔ مساوات \حوالہ{مساوات_بدلتا_مخلوط_ب} سے \عددی{I_0} حاصل کی گئی جسے  \عددی{e^{j\omega t}} سے ضرب دیتے ہوئے مخلوط حل حاصل کیا گیا۔مخلوط حل کا حقیقی جزو یعنی مساوات \حوالہ{مساوات_بدلتا_مخلوط_ت} درکار جواب ہے۔مثال \حوالہ{مثال_بدلتا_مزاحمت_امالہ_جبری_حل_ب} میں  مخصوص قیمتیں استعمال کرتے  ہوئے مخلوط رو کو مساوات \حوالہ{مساوات_بدلتا_مخلوط_رو_صورت}  میں پیش کیا گیا۔آپ دیکھ سکتے ہیں کہ جزو \عددی{e^{j \omega t}}  جوں کا توں مخلوط جبری تفاعل اور مخلوط جبری حل میں پایا جاتا ہے۔

حقیقت میں کسی بھی خطی دور پر مخلوط جبری تفاعل مثلاً
\begin{align}
v_M=V_0 e^{j \omega t}
\end{align}
مسلط کرنے سے دور میں تمام رو کی صورت \عددی{i_M(t)=I_0e^{j(\omega t+\phi)}} اور دباو کی صورت \عددی{v_M(t)=V_0e^{j(\omega t+\phi)}} ہو گی جہاں تمام رو اور دباو  کی تعدد \عددی{\omega} جبکہ ان کے انفرادی  حیطے مختلف ہوں گے۔ ان کے انفرادی زاویہ ہٹاو بھی مختلف ہوں گے۔یہاں حیطہ حقیقی مقدار ہے۔

یوں تعدد جانتے ہوئے کسی بھی مخلوط تفاعل مثلاً مخلوط رو کو اس کے حیطے \عددی{I_0} اور زاویائی ہٹاو \عددی{\phi} سے مکمل طور پر ظاہر کیا جا سکتا ہے۔مخلوط تفاعل مثلاً 
\begin{align}
i_M(t)=I_0 e^{j(\omega t+\phi)}
\end{align}
سے حقیقی تفاعل درج ذیل
\begin{align}\label{مساوات_بدلتا_حقیقی_رو_الف}
i(t)&=\left. I_0 e^{j(\omega t+\phi)} \right|_{\text{حقیقی}}
\end{align}
لکھا جا سکتا ہے جہاں \عددی{I_0} حقیقی مقدار ہے اور زیر نوشت میں لفظ "حقیقی" لکھنے کا مطلب ہے کہ اس تفاعل کا حقیقی جزو لیا جائے یعنی
\begin{align}\label{مساوات_بدلتا_حقیقی_رو_ب}
i(t)&=I_0 \cos (\omega t+\phi)
\end{align}
مساوات \حوالہ{مساوات_بدلتا_حقیقی_رو_الف} حقیقی رو دیتی ہے۔اس طرز کے تمام مساوات میں \عددی{e^{j\omega t}} پایا جاتا ہے اور مساوات کا حقیقی جزو ہی حقیقی مقدار ہوتا ہے۔یوں ایسے مساوات میں لفظ "حقیقی" اور \عددی{e^{j\omega t}} کو ذہن میں رکھتے ہوئے انہیں لکھنے سے گریز کیا جاتا ہے۔مساوات \حوالہ{مساوات_بدلتا_حقیقی_رو_الف} میں ایسا ہی کرتے ہوئے درج ذیل لکھا جائے گا
\begin{align}\label{مساوات_بدلتا_دوری_سمتیہ_الف}
\hat{I}=I_0e^{j\phi}
\end{align}
جہاں رو کو ٹوپی والے بڑے حرف سے ظاہر کیا گیا ہے۔دباو کی صورت میں تفاعل کو \عددی{\hat{V}} لکھا جاتا۔ٹوپی والے بڑے حرف سے ظاہر کردہ تفاعل کو \عددی{e^{j\omega t}} سے ضرب دے کر حقیقی جزو لینے سے حقیقی تفاعل حاصل کیا جاتا ہے۔

مساوات \حوالہ{مساوات_بدلتا_دوری_سمتیہ_الف} کا صفحہ \حوالہصفحہ{مساوات_بدلتا_مخلوط_عدد_طرز_لکھائی} پر مساوات \حوالہ{مساوات_بدلتا_مخلوط_عدد_طرز_لکھائی} سے موازنہ کریں۔ایسا معلوم ہوتا ہے جیسے  \عددی{\hat{I}} مخلوط عدد کو ظاہر کرتا ہے۔اگرچہ مساوات \حوالہ{مساوات_بدلتا_دوری_سمتیہ_الف} درحقیقت میں مساوات \حوالہ{مساوات_بدلتا_حقیقی_رو_الف} کو چھوٹا لکھنے کا طریقہ ہے لہٰذا \عددی{\hat{I}} مخلوط عدد کو ظاہر نہیں کرتا لیکن دیکھا یہ گیا ہے کہ \عددی{\hat{I}} کو مخلوط عدد تصور کر لینے سے ہمارے لئے آسانی پیدا ہوتی ہے۔آئیں \عددی{\hat{V}} کو مخلوط عدد فرض کرتے ہوئے اس کو مخلوط سطح پر ظاہر کریں۔
%=========================== 
\ابتدا{مثال}\شناخت{مثال_بدلتا_دوری_سمتیہ_مثال_الف}
مخلوط دباو \عددی{v_M(t)=50e^{j(100\pi t-35^{\circ})}}سے \عددی{\hat{V}} حاصل کرتے ہوئے \عددی{\hat{V}} کو مخلوط سطح پر دکھائیں۔

حل:مخلوط دباو سے حقیقی دباو لکھتے ہیں۔
\begin{align*}
v(t)=\left. 50e^{j(100\pi t-35^{\circ})} \right|_{\text{حقیقی}}
\end{align*}
اس مساوات کی تعدد \عددی{(\omega=100\pi)} کو ذہن نشین کرتے ہوئے لفظ "حقیقی" اور \عددی{e^{j100\pi t}} لکھنے سے گریز کرتے ہوئے درج ذیل لکھا جائے گا
\begin{align*}
\hat{V}&=50e^{-j35^{\circ}}\\
&=50\phase{-35^{\circ}}
\end{align*}
جسے شکل \حوالہ{شکل_بدلتا_دوری_سمتیہ_مثال_الف} میں مخلوط سطح پر دکھایا گیا ہے۔حقیقی محدد سے گھڑی کی گردش کی جانب مثبت زاویہ ناپا جاتا ہے لہٰذا منفی زاویے کو گھڑی کی گردش کے الٹ جانب دکھایا گیا ہے۔مخلوط اعداد اور \عددی{\hat{V}} میں فرق رکھنے کی خاطر \عددی{\hat{V}} کو مخلوط سطح پر تیر کی نشان سے ظاہر کیا جاتا ہے۔
\begin{figure}
\centering
\begin{tikzpicture}
\draw(0,0)--++(3,0)node[right]{حقیقی};
\draw(0,-1.5)--(0,0.5)node[left]{خیالی};
\draw[-latex] (0,0)--++(-35:2.5)node[right]{$\hat{V}$}node[pos=0.6,below left]{$50$};
\draw[-stealth]([shift={(0:0.5)}]0,0) arc (0:-35:0.5);
\draw(-35*2/3:0.6)node[right]{$35^{\circ}$};
\end{tikzpicture}
\caption{مثال \حوالہ{مثال_بدلتا_دوری_سمتیہ_مثال_الف} کی دوری سمتیہ۔}
\label{شکل_بدلتا_دوری_سمتیہ_مثال_الف}
\end{figure}
\انتہا{مثال}
%===========================

مثال \حوالہ{مثال_بدلتا_دوری_سمتیہ_مثال_الف} میں \عددی{\hat{V}} کو مخلوط سطح پر تیر کے نشان سے ظاہر کیا گیا ہے جسے دیکھ کر یوں معلوم ہوتا ہے جیسے \عددی{\hat{V}} ایک سمتیہ ہے۔اسی حقیقت کی بنا پر \عددی{\hat{V}} یا \عددی{\hat{I}} کو \اصطلاح{دوری سمتیہ}\فرہنگ{دوری سمتیہ}\حاشیہب{phasor}\فرہنگ{phasor} کہتے ہیں اور شکل \حوالہ{شکل_بدلتا_دوری_سمتیہ_مثال_الف} کو \اصطلاح{دوری سمتیہ شکل}\فرہنگ{دوری سمتیہ شکل}\حاشیہب{phasor diagram}\فرہنگ{phasor diagram} کہتے ہیں۔

مخلوط عدد لکھنے کے تمام طرز پر دوری سمتیہ کو لکھا جاتا ہے لہٰذا درج ذیل لکھنا ممکن ہے۔
\begin{gather}
\begin{aligned}\label{مساوات_بدلتا_تعددی_طرز}
\hat{I}&=I_0e^{j\phi}\\
 &=I_0\phase{\phi}\\
&=I_x+jI_y
\end{aligned}
\end{gather}

دوری سمتیہ کا حیطہ حقیقی اور مثبت مقدار ہوتا ہے۔یوں درج بالا مساوات میں \عددی{I_0} حقیقی مثبت مقدار ہے۔

مساوات \حوالہ{مساوات_بدلتا_حقیقی_رو_ب} کو تفاعل کی \اصطلاح{وقتی دائرہ کار}\فرہنگ{وقتی دائرہ کار}\حاشیہب{time domain}\فرہنگ{time domain} صورت کہتے ہیں جبکہ مساوات \حوالہ{مساوات_بدلتا_تعددی_طرز} کو تفاعل کی \اصطلاح{تعددی دائرہ کار}\فرہنگ{تعددی دائرہ کار}\حاشیہب{frequency domain}\فرہنگ{frequency form} صورت کہتے ہیں۔
%====================
\ابتدا{مثال}
درج ذیل تفاعل کے دوری سمتیہ دریافت کریں۔
\begin{align*}
v_1(t)=20 \cos (100t +30^{\circ}), \quad v_2(t)=-40 \sin(310t -40^{\circ}), \quad i(t)=22\cos(\omega t+0.2\pi)
\end{align*}

حل:دباو \عددی{v_1(t)} کو مخلوط تفاعل کا حقیقی جزو لکھ کر
\begin{align*}
v_1(t)=\left. 20 e^{j(100 t+30^{\circ})}\right|_{\text{حقیقی}}
\end{align*}
تعدد کو ذہن نشین کرتے ہوئے،  \عددی{e^{j 100t}} نہ لکھتے  ہوئے اور زیر نوشت میں لفظ "حقیقی" نہ لکھتے ہوئے  دوری سمتیہ حاصل ہوتا ہے۔
\begin{align*}
\hat{V}_1=20 e^{j30^{\circ}}=20\phase{30^{\circ}}
\end{align*}
اسی طرح \عددی{v_2(t)} کو \عددی{\cos} کی صورت میں یوں لکھتے ہیں کہ حیطہ مثبت لکھا جائے۔
\begin{align*}
v_2=-40 \sin(310t-40^{\circ})=40 \cos (310 t-40^{\circ}+90^{\circ})=40 \cos (310t+50^{\circ})
\end{align*} 
اس کو مخلوط تفاعل کا حقیقی جزو لکھتے ہیں۔
\begin{align*}
v_2=\left.40 e^{j(310t+50^{\circ})}\right|_{\text{حقیقی}}
\end{align*} 
اس مساوات کے زیر نوشت میں لفظ "حقیقی" نہ لکھتے ہوئے اور ساتھ ہی ساتھ \عددی{e^{j 310 t}} نہ لکھتے ہوئے دوری سمتیہ حاصل ہوتی ہے یعنی
\begin{align*}
\hat{V}_2=40 e^{ j50^{\circ}}
\end{align*} 
جس کو درج ذیل بھی لکھا جا سکتا ہے۔
\begin{align*}
\hat{V}_2=40 \phase{50^{\circ}}
\end{align*} 
رو کو بھی مخلوط تفاعل کا حقیقی جزو لکھ کر
\begin{align*}
i(t)=\left. 22 e^{j(\omega t+0.2\pi)} \right|_{\text{حقیقی}}
\end{align*}
دوری سمتیہ حاصل کرتے ہیں۔
\begin{align*}
\hat{I}=22 e^{j 0.2\pi}=22\phase{0.2 \pi}
\end{align*}
\انتہا{مثال}
%=======================
\ابتدا{مشق}
درج ذیل کو تعددی دائرہ کار میں لکھیں جہاں \عددی{\omega=\SI{400}{\radian \per\second}} ہے۔
\begin{align*}
\hat{I}=35\phase{44^{\circ}}, \quad \hat{V}=12 e^{j\tfrac{\pi}{4}}, \quad \hat{I}=33\phase{-77^{\circ}}
\end{align*}

جوابات:\عددی{i(t)=35\cos(400t+44^{\circ})}، \عددی{v(t)=12\cos(400t+\frac{\pi}{4})}،\\
 \عددی{i(t)=33\cos(400t-77^{\circ})}
\انتہا{مشق}
%=====================

شکل \حوالہ{شکل_بدلتا_دوری_سمتیات} میں \عددی{\hat{I}=25\phase{20^{\circ}}} اور \عددی{\hat{V}=30 e^{j55^{\circ}}} کھینچے گئے ہیں جہاں سے دوری سمتیات کا زاویائی تعلق بھی ظاہر ہوتا ہے۔شکل \حوالہ{شکل_بدلتا_دوری_سمتیات} میں دباو سے رو \عددی{33^{\circ}} درجے پیچھے ہے۔
\begin{figure}
\centering
\begin{tikzpicture}
\pgfmathsetmacro{\iMag}{2.5}
\pgfmathsetmacro{\iAng}{20}
\pgfmathsetmacro{\vMag}{3}
\pgfmathsetmacro{\vAng}{55}
\draw(0,0)--++(4,0)node[right]{حقیقی};
\draw(0,0)--++(0,2.7)node[left]{خیالی};
\draw[-latex](0,0)--++(\iAng:\iMag)node[right]{$\hat{I}$};
\draw[-latex](0,0)--++(\vAng:\vMag)node[right]{$\hat{V}$};
\draw[-stealth]([shift={(0:0.8)}]0,0) arc (0:\iAng:0.8);
\draw(2/3*\iAng:0.9)node[right]{$20^{\circ}$};
\draw[-stealth]([shift={(0:1.6)}]0,0) arc (0:\vAng:1.6);
\draw(3/4*\vAng:1.7)node[right]{$55^{\circ}$};
\end{tikzpicture}
\caption{دوری سمتیات کے اشکال۔}
\label{شکل_بدلتا_دوری_سمتیات}
\end{figure}

کسی بھی حقیقی تفاعل مثلاً حقیقی دباو کو \عددی{v(t)=V_0\cos(\omega t+\phi)} صورت میں لکھتے ہوئے جہاں \عددی{V_0} مثبت حقیقی مقدار ہو، \عددی{V_0} اور \عددی{\phi} استعمال کرتے ہوئے دوری سمتیہ فوراً
\begin{align}
\hat{V}=V_0\phase{\phi}
\end{align}
لکھا جا سکتا ہے۔
%=================
\ابتدا{مثال}
درج ذیل کے دوری سمتیات فوراً لکھیں۔
\begin{align*}
i_1(t)&=20\cos(132t-27^{\circ}) \\
 v_1(t)&=-100\cos(20t-60^{\circ})\\
 i_2(t)&=-90\sin(450t-100^{\circ})
\end{align*}

حل:رو \عددی{i_1} میں \عددی{I_0=20} اور \عددی{\phi=-27^{\circ}} ہے لہٰذا درج ذیل لکھا جائے گا۔
\begin{align*}
\hat{I}_1=20\phase{-27^{\circ}}
\end{align*}
دباو کا حیطہ منفی ہے لہٰذا مثبت حیطہ حاصل کرنے کی خاطر دباو کو درج ذیل لکھتے ہیں
\begin{align*}
v_1(t)=100\cos(20t-60^{\circ}+180^{\circ})=100\cos(20t+120^{\circ})
\end{align*}
جس سے دوری سمتیہ درج ذیل لکھا جا سکتا ہے۔
\begin{align*}
\hat{V}_1=100\phase{120^{\circ}}
\end{align*}
رو \عددی{i_2(t)} کو \عددی{i(t)=I_0\cos(\omega t+\phi)} کی صورت میں لکھتے ہیں۔
\begin{align*}
i_2(t)=90\cos(450t-100^{\circ}+90^{\circ})=90\cos(450t-10^{\circ})
\end{align*}
یوں دوری سمتیہ درج ذیل ہو گا۔
\begin{align*}
\hat{I}_2=90\phase{-10^{\circ}}
\end{align*}
\انتہا{مثال}
%========================

\حصہ{مزاحمت، امالہ گیر اور برق گیر کے انفرادی دوری سمتی تعلق}
شکل \حوالہ{شکل_بدلتا_مزاحمت_تعددی_اور_وقتی_تفاعل} پر نظر رکھتے ہوئے پڑھیں۔مزاحمت \عددی{R} پر مخلوط دباو \عددی{v(t)=V_0e^{j(\omega t+\phi_v)}} مسلط کرنے سے  مزاحمت میں مخلوط رو \عددی{i(t)=I_0e^{j(\omega t+\phi_i)}} گزرے گی۔اوہم کے قانون کے تحت
\begin{align*}
V_0 e^{j(\omega t +\phi_v)}=R I_0 e^{j(\omega t+\phi_i)}
\end{align*}
یعنی
\begin{align*}
V_0 e^{j\phi_v}=R I_0 e^{j\phi_i}
\end{align*}
ہو گا۔اس کو دوری سمتیہ کی صورت میں
\begin{align}\label{مساوات_بدلتا_مزاحمت_دوری_تعلق}
\hat{V}=R \hat{I}
\end{align}
لکھا جا سکتا ہے جہاں
\begin{align*}
\hat{V}&=V_0e^{j\phi_v}\\
\hat{I}&=I_0 e^{j \phi_i}
\end{align*}
یعنی
\begin{gather}
\begin{aligned}\label{مساوات_بدلتا_مزاحمت_دوری_تعلق_الف}
\hat{V}&=V_0 \phase{\phi_v}\\
\hat{I}&=I_0 \phase{\phi_i}
\end{aligned}
\end{gather}
کے برابر ہیں۔اس طرح مساوات \حوالہ{مساوات_بدلتا_مزاحمت_دوری_تعلق} کو درج ذیل لکھا جا سکتا ہے۔
\begin{align*}
V_0 \phase{\phi_v} = R I_0 \phase{\phi_i}
\end{align*}

یاد رہے کہ دوری سمتیات میں \عددی{V_0} اور \عددی{I_0} حقیقی اور مثبت مقدار ہیں۔درج بالا مساوات میں بائیں ہاتھ اور دائیں ہاتھ کے مخلوط اعداد صرف اور صرف اس صورت برابر ہوں گے جب ان کے حیطے برابر ہوں اور ان کے زاویے برابر ہوں یعنی
\begin{gather}
\begin{aligned}\label{مساوات_بدلتا_مزاحمت_دوری_تعلق_ب}
V_0&=I_0 R\\
\phi_v&=\phi_i
\end{aligned}
\end{gather} 
اس طرح مزاحمت کی رو اور دباو ہم زاویہ ہیں۔مساوات \حوالہ{مساوات_بدلتا_مزاحمت_دوری_تعلق_ب} کی مدد سے مساوات \حوالہ{مساوات_بدلتا_مزاحمت_دوری_تعلق_الف} درج ذیل صورت اختیار کرتے ہیں۔
\begin{gather}
\begin{aligned}\label{مساوات_بدلتا_مزاحمت_دوری_تعلق_پ}
\hat{V}&=V_0\phase{\phi_v}\\
\hat{I}&=\frac{V_0}{R} \phase{\phi_v}
\end{aligned}
\end{gather}

شکل \حوالہ{شکل_بدلتا_مزاحمت_تعددی_اور_وقتی_تفاعل}-پ میں مزاحمت کے \عددی{\hat{I}} اور \عددی{\hat{V}} دوری سمتیات دکھائے گئے ہیں جو تعددی تفاعل ہیں جبکہ شکل \حوالہ{شکل_بدلتا_مزاحمت_تعددی_اور_وقتی_تفاعل}-ت میں مزاحمت کے \عددی{i(t)} اور \عددی{v(t)} دکھائے گئے ہیں جو وقتی تفاعل ہیں۔
\begin{figure}
\centering
\begin{subfigure}{0.5\textwidth}
\centering
\begin{tikzpicture}
\coordinate (a) at (0,0);
\coordinate (b) at (-0.025,0.5);
\coordinate (c) at (-0.04,1);
\coordinate (d) at (-0.12,1.5);
\coordinate (e) at (-0.2,2);
\coordinate (f) at (-0.15,2.5);
\coordinate (g) at (0.5,3);

\coordinate (h) at (0.7,2.5);
\coordinate (i) at (0.6,2);
\coordinate (j) at (0.75,1.5);
\coordinate (k) at (0.7,1);
\coordinate (l) at (0.7,0.5);
\coordinate (m) at (0.6,0);
%box circuit
\draw [] plot [smooth cycle] coordinates {(a) (b) (c) (d) (e) (f) (g) (h) (i) (j) (k) (l) (m)};
%controlled circuit
\draw (h) to [short,-o]++(1,0)coordinate(HH);
\draw (l) to [short,-o]++(1,0)coordinate(LL);
\draw ($(HH)!0.5!(LL)$)++(-0.3,0) node[shift={(\x/4,0)}]{$\begin{aligned} &+ \\ \hat{V} &=\hat{I} R\\&-   \end{aligned}$};
\draw(HH) to [short,i={$\hat{I}$},o-]++(\x,0) to [resistor,l={$R$}]++(0,-\y) to [short,-o] (LL);
\end{tikzpicture}
\caption*{(الف)}
\end{subfigure}%
\begin{subfigure}{0.5\textwidth}
\centering
\begin{tikzpicture}
\coordinate (a) at (0,0);
\coordinate (b) at (-0.025,0.5);
\coordinate (c) at (-0.04,1);
\coordinate (d) at (-0.12,1.5);
\coordinate (e) at (-0.2,2);
\coordinate (f) at (-0.15,2.5);
\coordinate (g) at (0.5,3);

\coordinate (h) at (0.7,2.5);
\coordinate (i) at (0.6,2);
\coordinate (j) at (0.75,1.5);
\coordinate (k) at (0.7,1);
\coordinate (l) at (0.7,0.5);
\coordinate (m) at (0.6,0);
%box circuit
\draw [] plot [smooth cycle] coordinates {(a) (b) (c) (d) (e) (f) (g) (h) (i) (j) (k) (l) (m)};
%controlled circuit
\draw (h) to [short,-o]++(1,0)coordinate(HH);
\draw (l) to [short,-o]++(1,0)coordinate(LL);
\draw ($(HH)!0.5!(LL)$)++(-0.3,0) node[shift={(\x/4,0)}]{$\begin{aligned} &+ \\ v(t) &=i(t)R\\&-   \end{aligned}$};
\draw(HH) to [short,i={$i(t)$},o-]++(\x,0) to [resistor,l={$R$}]++(0,-\y) to [short,-o] (LL);
\end{tikzpicture}
\caption*{(ب)}
\end{subfigure}
\begin{subfigure}{0.5\textwidth}
\centering
\begin{tikzpicture}
\pgfmathsetmacro{\l}{3}
\pgfmathsetmacro{\ang}{30}
%
\draw(0,0)--++(3,0)node[right]{حقیقی};
\draw(0,0)--++(0,2)node[left]{خیالی};
\draw[-latex](0,0)--++(\ang:\l)node[above left]{$\hat{V}$};
\draw[-latex](0,0)--++(\ang:0.6*\l)node[above left]{$\hat{I}$};
\draw[-stealth]([shift={(0:0.5)}]0,0) arc (0:\ang:0.5);
\draw(2/3*\ang:0.6)node[right]{$\phi_v=\phi_i$};
\end{tikzpicture}
\caption*{(پ)}
\end{subfigure}%
\begin{subfigure}{0.5\textwidth}
\centering
\begin{tikzpicture}
\begin{axis}[kStyleCircuitsA,small,xlabel=$\omega t$, ylabel=${v,i}$,ticks=none]
\addplot[domain=-120:250,samples=100]{cos(x+30)}node[pos=0.15,above left]{$v$};
\addplot[domain=-120:250,samples=100]{0.6*cos(x+30)}node[pos=0.15,below right]{$i$};
\draw[dashed](axis cs:-30,1)--(axis cs:-30,-0.3);
\draw[stealth-](axis cs:0,-0.2)--(axis cs:30,-0.2);
\draw[stealth-](axis cs:-30,-0.2)--(axis cs:-60,-0.2)--(axis cs:-60,-0.4)node[below]{$\phi_v$};
\end{axis}
\end{tikzpicture}
\caption*{(ت)}
\end{subfigure}%
\caption{مزاحمت کے دباو اور رو کے تعددی اور وقتی تفاعل۔}
\label{شکل_بدلتا_مزاحمت_تعددی_اور_وقتی_تفاعل}
\end{figure}
%==============
\ابتدا{مثال}
شکل \حوالہ{شکل_بدلتا_مزاحمت_تعددی_اور_وقتی_تفاعل}-ب میں \عددی{\SI{10}{\ohm}} کے مزاحمت پر \عددی{v(t)=22\cos(30t-66^{\circ})} دباو مسلط کی گئی ہے۔ مزاحمت کے رو کو وقتی دائرہ کار میں لکھیں۔رو کی تعددی دائرہ کار صورت شکل \حوالہ{شکل_بدلتا_مزاحمت_تعددی_اور_وقتی_تفاعل}-الف سے دریافت کریں۔

حل:اوہم کے قانون کی مدد سے رو کی وقتی دائرہ کار صورت معلوم کرتے ہیں۔
\begin{align*}
i(t)=\frac{v(t)}{R}=\frac{22\cos(30t-66^{\circ})}{10}=2.2\cos(30t-66^{\circ}) \, \si{\ampere}
\end{align*}
آئیں اب رو کی تعددی دائرہ کار صورت حاصل کرتے ہیں۔دوری دباو
\begin{align*}
\hat{V}=22\phase{-66^{\circ}}\, \si{\volt}
\end{align*}
ہے لہٰذا دوری رو درج ذیل ہو گی۔
\begin{align*}
\hat{I}=\frac{\hat{V}}{R}=\frac{22\phase{-66^{\circ}}}{10}=2.2\phase{-66^{\circ}} \, \si{\ampere}
\end{align*}
\انتہا{مثال}
%=============================
\ابتدا{مشق}
بارہ اوہم کے مزاحمت میں دوری رو \عددی{\hat{I}=37\phase{43^{\circ}} \, \si{\ampere}} ہے جبکہ تعدد \عددی{\omega=\SI{172}{\radian\per\second}} ہے۔دباو کی وقتی دائرہ کار صورت لکھیں۔

جواب:\عددی{v(t)=444\cos(172 t +43^{\circ}) \, \si{\volt}}
\انتہا{مشق}
%============================
شکل \حوالہ{شکل_بدلتا_امالہ_تعددی_اور_وقتی_تفاعل} پر نظر رکھتے ہوئے پڑھیں۔امالہ گیر کے دباو اور رو کا تعلق درج ذیل ہے۔
\begin{align}\label{مساوات_بدلتا_امالہ_وقتی_تعلق}
v=L\frac{\dif i(t)}{\dif t}
\end{align}
امالہ گیر پر مخلوط دباو \عددی{v(t)=V_0e^{j(\omega t+\phi_v)}} مسلط کرنے سے اس میں مخلوط رو \عددی{i(t)=I_0e^{j(\omega t+\phi_i)}} پیدا ہو گی۔ان قیمتوں کو درج بالا مساوات میں پُر کرنے سے
\begin{align*}
V_0 e^{j(\omega t+\phi_v)}&=L \frac{\dif}{\dif t} \left[I_0 e^{j(\omega t +\phi_i)}\right]\\
&=j \omega L I_0 e^{j(\omega t+\phi_i)}
\end{align*}
یعنی
\begin{align*}
V_0 e^{j\phi_v}=j\omega L I_0 e^{j\phi_i}
\end{align*}
ملتا ہے جو دوری مساوات ہے۔یہ دوری مساوات درج ذیل لکھی جائے گی۔
\begin{align}\label{مساوات_بدلتا_امالہ_تعددی_تعلق_الف}
\hat{V}=j\omega L \hat{I}
\end{align} 
آپ نے دیکھا کہ مساوات \حوالہ{مساوات_بدلتا_امالہ_وقتی_تعلق} جو تفرقی اور وقتی مساوات ہے سے مساوات \حوالہ{مساوات_بدلتا_امالہ_تعددی_تعلق_الف} حاصل ہوتا ہے جو تعددی اور الجبرائی مساوات ہے۔دوری سمتیات کی مدد سے تفرقی مساوات سے الجبرائی مساوات حاصل ہوتے ہیں۔آپ جانتے ہیں کہ الجبرائی مساوات حل کرنا نہایت آسان ہوتا ہے جبکہ تفرقی مساوات کو حل کرنا دشوار ہوتا ہے۔یہی وجہ ہے کہ دوری سمتیات اتنے مقبول ہیں۔

آپ جانتے ہیں کہ
\begin{align}\label{مساوات_بدلتا_عمود}
\phase{90^{\circ}}=e^{j90^{\circ}}=\cos 90^{\circ}+j \sin 90^{\circ}=j
\end{align}
لکھا جا سکتا ہے لہٰذا مساوات \حوالہ{مساوات_بدلتا_امالہ_تعددی_تعلق_الف} کو 
\begin{align*}
\hat{V}=\omega L \hat{I} e^{j90^{\circ}}
\end{align*} 
یعنی
\begin{align}\label{مساوات_بدلتا_امالہ_تعددی_تعلق_ب}
V_0 e^{j\phi_v}=\omega L I_0 e^{j (\phi_i+90^{\circ})}
\end{align} 
لکھا جا سکتا ہے۔مساوات \حوالہ{مساوات_بدلتا_امالہ_تعددی_تعلق_ب} میں دونوں ہاتھ کے مخلوط اعداد صرف اور صرف اس وقت برابر ہوں گے جب ان کے حیطے برابر ہوں اور ان کے زاویے برابر ہوں لہٰذا اس مساوات کے تحت
\begin{gather}
\begin{aligned}\label{مساوات_بدلتا_امالہ_رو_پیچھے_ہے}
V_0&=\omega L I_0\\
\phi_v&=\phi_i+90^{\circ}
\end{aligned}
\end{gather}
ہوں گے۔یوں دباو کا زاویہ، رو کے زاویے سے \عددی{90^{\circ}} درجے زیادہ ہے لہٰذا رو سے دباو \عددی{90^{\circ}} درجے آگے ہے یا دباو سے رو \عددی{90^{\circ}} پیچھے ہے۔شکل \حوالہ{شکل_بدلتا_امالہ_تعددی_اور_وقتی_تفاعل}-پ میں دوری سمتیات دکھائے گئے ہیں جہاں دباو سے رو \عددی{90^{\circ}} درجے پیچھے دکھایا گیا ہے۔

مساوات \حوالہ{مساوات_بدلتا_امالہ_تعددی_تعلق_ب} سے وقتی مساوات درج ذیل لکھی جائے گی جہاں مساوات \حوالہ{مساوات_بدلتا_امالہ_رو_پیچھے_ہے} کے تحت \عددی{\phi_v=\phi_i+90^{\circ}} ہو گا۔
\begin{align}
V_0 \cos (\omega t +\phi_v)=\omega L I_0 \cos (\omega t +\phi_i+90^{\circ})
\end{align}
درج بالا مساوات میں دیے دباو اور رو کو شکل \حوالہ{شکل_بدلتا_امالہ_تعددی_اور_وقتی_تفاعل}-ت میں دکھایا گیا ہے۔
%
\begin{figure}
\centering
\begin{subfigure}{0.5\textwidth}
\centering
\begin{tikzpicture}
\coordinate (a) at (0,0);
\coordinate (b) at (-0.025,0.5);
\coordinate (c) at (-0.04,1);
\coordinate (d) at (-0.12,1.5);
\coordinate (e) at (-0.2,2);
\coordinate (f) at (-0.15,2.5);
\coordinate (g) at (0.5,3);

\coordinate (h) at (0.7,2.5);
\coordinate (i) at (0.6,2);
\coordinate (j) at (0.75,1.5);
\coordinate (k) at (0.7,1);
\coordinate (l) at (0.7,0.5);
\coordinate (m) at (0.6,0);
%box circuit
\draw [] plot [smooth cycle] coordinates {(a) (b) (c) (d) (e) (f) (g) (h) (i) (j) (k) (l) (m)};
%controlled circuit
\draw (h) to [short,-o]++(1,0)coordinate(HH);
\draw (l) to [short,-o]++(1,0)coordinate(LL);
\draw ($(HH)!0.5!(LL)$)++(-0.3,0) node[shift={(\x/4,0)}]{$\begin{aligned} &+ \\ \hat{V} &=j \omega L \hat{I} \\&-   \end{aligned}$};
\draw(HH) to [short,i={$\hat{I}$},o-]++(\x,0) to [inductor,l={$L$}]++(0,-\y) to [short,-o] (LL);
\end{tikzpicture}
\caption*{(الف)}
\end{subfigure}%
\begin{subfigure}{0.5\textwidth}
\centering
\begin{tikzpicture}
\coordinate (a) at (0,0);
\coordinate (b) at (-0.025,0.5);
\coordinate (c) at (-0.04,1);
\coordinate (d) at (-0.12,1.5);
\coordinate (e) at (-0.2,2);
\coordinate (f) at (-0.15,2.5);
\coordinate (g) at (0.5,3);

\coordinate (h) at (0.7,2.5);
\coordinate (i) at (0.6,2);
\coordinate (j) at (0.75,1.5);
\coordinate (k) at (0.7,1);
\coordinate (l) at (0.7,0.5);
\coordinate (m) at (0.6,0);
%box circuit
\draw [] plot [smooth cycle] coordinates {(a) (b) (c) (d) (e) (f) (g) (h) (i) (j) (k) (l) (m)};
%controlled circuit
\draw (h) to [short,-o]++(1,0)coordinate(HH);
\draw (l) to [short,-o]++(1,0)coordinate(LL);
\draw ($(HH)!0.5!(LL)$)++(-0.3,0) node[shift={(\x/4,0)}]{$\begin{aligned} &+ \\ v(t) &=L \frac{\dif i(t)}{\dif t}\\&-   \end{aligned}$};
\draw(HH) to [short,i={$i(t)$},o-]++(\x,0) to [inductor,l={$L$}]++(0,-\y) to [short,-o] (LL);
\end{tikzpicture}
\caption*{(ب)}
\end{subfigure}
\begin{subfigure}{0.5\textwidth}
\centering
\begin{tikzpicture}
\pgfmathsetmacro{\l}{3}
\pgfmathsetmacro{\ang}{30}
\pgfmathsetmacro{\angA}{\ang-90}
%
\draw(0,0)--++(3,0)node[right]{حقیقی};
\draw(0,0)--++(0,2)node[left]{خیالی};
\draw[-latex](0,0)--++(\ang:\l)node[above left]{$\hat{V}$};
\draw[-latex](0,0)--++(\ang-90:0.7*\l)node[below right]{$\hat{I}$};
\draw[-stealth]([shift={(0:0.6)}]0,0) arc (0:\ang:0.6);
\draw(2/3*\ang:0.7)node[right]{$\phi_v$};
\draw[-stealth]([shift={(0:0.5)}]0,0) arc (0:\angA:0.5);
\draw(2/3*\angA:0.6)node[right]{$\phi_i$};
\draw[stealth-stealth]([shift={(\angA:1.5)}]0,0) arc (\angA:\ang:1.5);
\draw(0.5*\angA:1.6)node[right]{$90^{\circ}$};
\end{tikzpicture}
\caption*{(پ)}
\end{subfigure}%
\begin{subfigure}{0.5\textwidth}
\centering
\begin{tikzpicture}
\begin{axis}[kStyleCircuitsA,small,xlabel=$\omega t$, ylabel=${v,i}$,ticks=none]
\addplot[domain=-120:250,samples=100]{cos(x+30)}node[pos=0.15,above left]{$v$};
\addplot[domain=-120:250,samples=100]{0.7*cos(x+30-90)}node[pos=0.5,above  right]{$i$};
\draw[dashed](axis cs:-30,1)--(axis cs:-30,-0.3);
\draw[dashed](axis cs:60,0.7)--(axis cs:60,-0.3);
\draw[stealth-](axis cs:60,-0.2)--(axis cs:90,-0.2);
\draw[stealth-](axis cs:-30,-0.2)--(axis cs:-90,-0.2)--(axis cs:-90,-0.3)node[left]{$90^{\circ}$};
\end{axis}
\end{tikzpicture}
\caption*{(ت)}
\end{subfigure}%
\caption{امالہ کے دباو اور رو کے تعددی اور وقتی تفاعل۔}
\label{شکل_بدلتا_امالہ_تعددی_اور_وقتی_تفاعل}
\end{figure}
%============
\ابتدا{مثال}
شکل \حوالہ{شکل_بدلتا_امالہ_تعددی_اور_وقتی_تفاعل} میں \عددی{\SI{4}{\milli\henry}} امالہ گیر پر \عددی{v(t)=12\cos(1000t+22^{\circ})} دباو مسلط کی جاتی ہے۔امالہ گیر کی رو دریافت کریں۔

حل:دوری سمتیہ دباو درج ذیل ہے۔
\begin{align*}
\hat{V}=12\phase{22^{\circ}}
\end{align*}
مساوات \حوالہ{مساوات_بدلتا_امالہ_تعددی_تعلق_الف} کی مدد سے دوری سمتیہ رو حاصل کرتے ہیں
\begin{align*}
\hat{I}&=\frac{\hat{V}}{j \omega L}\\
&=\frac{12\phase{22^{\circ}}}{j1000\times 0.004}\\
&=\frac{12\phase{22^{\circ}}}{4\phase{90^{\circ}}}\\
&=3\phase{-68^{\circ}} \, \si{\ampere}
\end{align*}
جہاں مساوات \حوالہ{مساوات_بدلتا_عمود} کا استعمال کرتے ہوئے \عددی{j=\phase{90^{\circ}}} لکھا گیا  ہے۔یوں رو کی وقتی دائرہ کار صورت درج ذیل ہو گی۔
\begin{align*}
i(t)=3\cos(1000t-68^{\circ}) \, \si{\ampere}
\end{align*}
\انتہا{مثال}
%================
\ابتدا{مشق}
امالہ کی قیمت \عددی{\SI{10}{\milli\henry}} جبکہ اس میں رو \عددی{\hat{I}=8\phase{44^{\circ}}} کی تعدد \عددی{\SI{500}{\radian \per\second}} ہے۔دباو کی وقتی دائرہ کار صورت دریافت کریں۔

جواب:\عددی{v(t)=40\cos(500t+134^{\circ})\, \si{\volt}}
\انتہا{مشق}
%===============
شکل \حوالہ{شکل_بدلتا_برق_گیر_تعددی_اور_وقتی_تفاعل} پر نظر رکھتے ہوئے پڑھیں جہاں برق گیر پر دباو \عددی{v(t)=V_0\cos(\omega t +\phi_v)} مسلط کی گئی ہے۔برق گیر کی تفرقی مساوات
\begin{align}\label{مساوات_بدلتا_برق_گیر_تفرقی_مساوات}
i(t)=C \frac{\dif v(t)}{\dif t}
\end{align}
میں مخلوط دباو اور مخلوط رو پُر کرتے ہوئے
\begin{align*}
I_0 e^{j(\omega t +\phi_i)}&=C \frac{\dif}{\dif t} \left[V_0 e^{j(\omega t +\phi_v)} \right]\\
&=j \omega C V_0 e^{j(\omega t +\phi_v)}
\end{align*}
یعنی
\begin{align*}
I_0 e^{j\phi_i}=j \omega C e^{j \phi_v}
\end{align*}
حاصل ہوتا ہے جس کو دوری سمتیہ کی صورت میں لکھتے ہیں۔
\begin{align}\label{مساوات_بدلتا_برق_گیر_دوری_مساوات_الف}
\hat{I}=j \omega C \hat{V}
\end{align}
مساوات \حوالہ{مساوات_بدلتا_برق_گیر_تفرقی_مساوات} برق گیر کی تفرقی مساوات ہے جبکہ مساوات \حوالہ{مساوات_بدلتا_برق_گیر_دوری_مساوات_الف} برق گیر کی الجبرائی مساوات ہے۔
 
مساوات \حوالہ{مساوات_بدلتا_برق_گیر_دوری_مساوات_الف} میں \عددی{j=e^{j90^{\circ}}} لکھنے سے درج ذیل ملتا ہے۔
\begin{align}\label{مساوات_بدلتا_برق_گیر_دوری_مساوات_ب}
I_0 e^{j\phi_i}= \omega C e^{j (\phi_v+90^{\circ})}
\end{align}
اس مساوات کے دونوں اطراف صرف اور صرف اس صورت برابر ہو سکتے ہیں جب دونوں اطراف کے حیطے برابر ہوں اور ان کے زاویے برابر ہوں۔
\begin{gather}
\begin{aligned}
I_0&=\omega C V_0\\
\phi_i&=\phi_v+90^{\circ}
\end{aligned}
\end{gather}
درج بالا مساوات کے تحت دباو سے رو \عددی{90^{\circ}} درجے آگے ہے۔

مساوات \حوالہ{مساوات_بدلتا_برق_گیر_دوری_مساوات_ب} سے وقتی دائرہ کار صورت لکھتے ہیں جہاں درج بالا مساوات کے تحت \عددی{\phi_i=\phi_v+90^{\circ}} ہو گا۔
\begin{align}
I_0 \cos (\omega t +\phi_i)=\omega C V_0 \cos(\omega t +\phi_v+90^{\circ})
\end{align}
شکل \حوالہ{شکل_بدلتا_برق_گیر_تعددی_اور_وقتی_تفاعل}-پ میں دوری سمتیات دکھائے گئے ہیں جبکہ شکل-ت میں دباو اور رو کی وقتی دائرہ کار صورت دکھائی گئی ہے۔

% 
\begin{figure}
\centering
\begin{subfigure}{0.5\textwidth}
\centering
\begin{tikzpicture}
\coordinate (a) at (0,0);
\coordinate (b) at (-0.025,0.5);
\coordinate (c) at (-0.04,1);
\coordinate (d) at (-0.12,1.5);
\coordinate (e) at (-0.2,2);
\coordinate (f) at (-0.15,2.5);
\coordinate (g) at (0.5,3);

\coordinate (h) at (0.7,2.5);
\coordinate (i) at (0.6,2);
\coordinate (j) at (0.75,1.5);
\coordinate (k) at (0.7,1);
\coordinate (l) at (0.7,0.5);
\coordinate (m) at (0.6,0);
%box circuit
\draw [] plot [smooth cycle] coordinates {(a) (b) (c) (d) (e) (f) (g) (h) (i) (j) (k) (l) (m)};
%controlled circuit
\draw (h) to [short,-o]++(1,0)coordinate(HH);
\draw (l) to [short,-o]++(1,0)coordinate(LL);
\draw ($(HH)!0.5!(LL)$)++(-0.3,0) node[shift={(\x/4,0)}]{$\begin{aligned} &+ \\& \hat{V} \\&-   \end{aligned}$};
\draw(HH) to [short,i={${\hat{I}=j \omega C \hat{V}}$},o-]++(\x,0) to [capacitor,l={$C$}]++(0,-\y) to [short,-o] (LL);
\end{tikzpicture}
\caption*{(الف)}
\end{subfigure}%
\begin{subfigure}{0.5\textwidth}
\centering
\begin{tikzpicture}
\coordinate (a) at (0,0);
\coordinate (b) at (-0.025,0.5);
\coordinate (c) at (-0.04,1);
\coordinate (d) at (-0.12,1.5);
\coordinate (e) at (-0.2,2);
\coordinate (f) at (-0.15,2.5);
\coordinate (g) at (0.5,3);

\coordinate (h) at (0.7,2.5);
\coordinate (i) at (0.6,2);
\coordinate (j) at (0.75,1.5);
\coordinate (k) at (0.7,1);
\coordinate (l) at (0.7,0.5);
\coordinate (m) at (0.6,0);
%box circuit
\draw [] plot [smooth cycle] coordinates {(a) (b) (c) (d) (e) (f) (g) (h) (i) (j) (k) (l) (m)};
%controlled circuit
\draw (h) to [short,-o]++(1,0)coordinate(HH);
\draw (l) to [short,-o]++(1,0)coordinate(LL);
\draw ($(HH)!0.5!(LL)$)++(-0.3,0) node[shift={(\x/4,0)}]{$\begin{aligned} &+ \\& v(t)\\&-   \end{aligned}$};
\draw(HH) to [short,i={${i(t)=C\frac{\dif v(t)}{\dif t}}$},o-]++(\x,0) to [capacitor,l={$C$}]++(0,-\y) to [short,-o] (LL);
\end{tikzpicture}
\caption*{(ب)}
\end{subfigure}
\begin{subfigure}{0.5\textwidth}
\centering
\begin{tikzpicture}
\pgfmathsetmacro{\l}{3}
\pgfmathsetmacro{\ang}{30}
\pgfmathsetmacro{\angA}{\ang+90}
%
\draw(0,0)--++(3,0)node[right]{حقیقی};
\draw(0,0)--++(0,2)node[left]{خیالی};
\draw[-latex](0,0)--++(\ang:\l)node[above left]{$\hat{V}$};
\draw[-latex](0,0)--++(\angA:0.7*\l)node[above left]{$\hat{I}$};
\draw[-stealth]([shift={(0:0.7)}]0,0) arc (0:\ang:0.7);
\draw(2/3*\ang:0.8)node[right]{$\phi_v$};
\draw[-stealth]([shift={(0:0.5)}]0,0) arc (0:\angA:0.5);
\draw(2/3*\angA:0.6)node[right]{$\phi_i$};
\draw[stealth-stealth]([shift={(\angA:1.5)}]0,0) arc (\angA:\ang:1.5);
\draw(0.5*\angA:1.6)node[right]{$90^{\circ}$};
\end{tikzpicture}
\caption*{(پ)}
\end{subfigure}%
\begin{subfigure}{0.5\textwidth}
\centering
\begin{tikzpicture}
\begin{axis}[kStyleCircuitsA,small,xlabel=$\omega t$, ylabel=${v,i}$,ticks=none]
\addplot[domain=-30:340,samples=100]{cos(x-60)}node[pos=0.25,above right]{$v$};
\addplot[domain=-120:340,samples=100]{0.7*cos(x-60+90)}node[pos=0.2,above left]{$i$};
\draw[dashed](axis cs:-30,0)--(axis cs:-30,-0.3);
\draw[dashed](axis cs:-120,0)--(axis cs:-120,-0.3);
\draw[stealth-](axis cs:-30,-0.2)--(axis cs:-10,-0.2);
\draw[stealth-](axis cs:-120,-0.2)--(axis cs:-150,-0.2);
\draw(axis cs:-75,-0.2)node[]{$90^{\circ}$};
\end{axis}
\end{tikzpicture}
\caption*{(ت)}
\end{subfigure}%
\caption{برق گیر کے دباو اور رو کے تعددی اور وقتی تفاعل۔}
\label{شکل_بدلتا_برق_گیر_تعددی_اور_وقتی_تفاعل}
\end{figure}
%=========================
\ابتدا{مثال}
شکل \حوالہ{شکل_بدلتا_برق_گیر_تعددی_اور_وقتی_تفاعل} میں \عددی{\SI{100}{\micro\farad}} برق گیر پر \عددی{v(t)=7\cos(5000t-60^{\circ}) \, \si{\volt}} کا دباو مسلط کیا گیا ہے۔رو حاصل کریں۔

حل:مسلط دباو کی دوری سمتیہ لکھتے ہیں۔
\begin{align*}
\hat{V}=7\phase{-60^{\circ}}
\end{align*}
یوں رو درج ذیل ہو گی
\begin{align*}
\hat{I}&=j \omega C \hat{V}\\
&=j 5000 \times 100 \times 10^{-6} 7 \phase{-60^{\circ}}\\
&=3.5 \phase{-60^{\circ}+90^{\circ}}\\
&=3.5\phase{30^{\circ}} \, \si{\ampere}
\end{align*}
جس کی وقتی دائرہ کار صورت درج ذیل ہے۔
\begin{align*}
i(t)=3.5\cos(5000 t+30^{\circ}) \, \si{\ampere}
\end{align*}
\انتہا{مثال}
%===========================
\ابتدا{مشق}
شکل \حوالہ{شکل_بدلتا_برق_گیر_تعددی_اور_وقتی_تفاعل} میں \عددی{\SI{330}{\micro\farad}} برق گیر کی رو \عددی{\hat{I}=11\phase{-12^{\circ}} \, \si{\ampere}} ہے۔ رو کی تعدد \عددی{\SI{6000}{\hertz}} ہے۔دباو کی وقتی دائرہ کار صورت حاصل کریں۔

جواب:\عددی{v(t)=0.884 \cos(12000\pi t-102^{\circ}) \, \si{\volt}}
\انتہا{مشق}
%=============================

\حصہ{برقی رکاوٹ اور برقی فراوانی}
قانون اوہم کے تحت برقی مزاحمت کو \عددی{R=\tfrac{V}{I}} لکھا جا سکتا ہے۔بالکل اسی طرح، شکل \حوالہ{شکل_بدلتا_برقی_رکاوٹ_تعریف} میں دوری سمتیہ دباو اور دوری سمتیہ رو کی شرح کو \اصطلاح{برقی رکاوٹ}\فرہنگ{رکاوٹ}\فرہنگ{برقی!رکاوٹ}\حاشیہب{impedance}\فرہنگ{impedance} کہتے اور  \عددی{\bZ} سے ظاہر کرتے ہیں۔
\begin{align}\label{مساوات_بدلتا_رکاوٹ_تعریف_الف}
\bZ=\frac{\hat{V}}{\hat{I}}
\end{align}
برقی رکاوٹ کو عموماً \اصطلاح{رکاوٹ} کہا جاتا ہے۔چونکہ \عددی{\hat{V}} اور \عددی{\hat{I}} مخلوط اعداد ہیں لہٰذا \عددی{{\bf{Z}}} بھی مخلوط عدد ہو گا۔
\begin{align}
\bZ=\frac{V_0\phase{\phi_v}}{I_0\phase{\phi_i}}=\frac{V_0}{I_0}\phase{\phi_v-\phi_i}=Z_0\phase{\phi_z}
\end{align}
چونکہ دباو اور رو کی شرح کو اوہم \عددی{\si{\ohm}} میں ناپتے ہیں لہٰذا رکاوٹ کی اکائی بھی اوہم ہے۔یوں بدلتی رو دور کی رکاوٹ یک سمتی رو دور کی مزاحمت کی مانند ہے۔رکاوٹ کو مستطیل طرز میں بھی لکھا جا سکتا ہے
\begin{align}
\bZ(\omega)=R(\omega)+j X(\omega)
\end{align} 
جہاں \عددی{R} حقیقی جزو  یعنی \اصطلاح{مزاحمت}\فرہنگ{مزاحمت}\حاشیہب{resistive}\فرہنگ{resistive} ہے جبکہ \عددی{X} خیالی جزو یعنی \اصطلاح{متعاملیت}\فرہنگ{متعاملیت}\حاشیہب{reactance}\فرہنگ{reactance} ہے۔ رکاوٹ مخلوط عدد ہے نا کہ دوری سمتیہ چونکہ دوری سمتیہ سائن نما تفاعل کو ظاہر کرتی ہے جبکہ رکاوٹ سائن نما تفاعل نہیں ہے۔
\tikzexternaldisable
\begin{figure}
\centering
\begin{tikzpicture}
\draw(0,0)--++(0,2.5)--++(2,0)--++(0,-2.5)--cycle;
\draw(1,1.25)node{\RL{بدلتی رو دور}};
\draw(0,0.25) to [short]++(-\x,0) to [american voltage source,l={${V_0 \phase{\phi_v}}$}]++(0,\y) to [short,i={$\vspace{25pt} {I_0\phase{\phi_i}}$}]++(\x,0);
\draw[stealth-] (-0.5,1.25)--++(-\x/4,0)--++(0,-\y/8)node[below]{$Z \phase{\phi_z}$};
\end{tikzpicture}
\caption{برقی رکاوٹ کی تعریف۔}
\label{شکل_بدلتا_برقی_رکاوٹ_تعریف}
\end{figure}

کسی بھی مخلوط عدد کی طرح، رکاوٹ کو بھی مستطیل طرز اور زاویائی طرز میں لکھا جا سکتا ہے
\begin{align}
\bZ=Z\phase{\phi_z}=R+jX
\end{align}
جہاں ایک طرز سے دوسری طرز میں تبادلہ درج ذیل مساوات سے کیا جاتا ہے۔
\begin{gather}
\begin{aligned}
R&=Z \cos \phi_z\\
X&=Z \sin \phi_z \quad \quad \text{\RL{زاویائی سے مستطیل طرز}}
\end{aligned}
\end{gather}  
%
\begin{gather}
\begin{aligned}
Z&=\sqrt{R^2+X^2}\\
\phi_z&=\tan^{-1}\frac{X}{R} \quad \quad \text{\RL{مستطیل سے زاویائی طرز}}
\end{aligned}
\end{gather}

مساوات \حوالہ{مساوات_بدلتا_رکاوٹ_تعریف_الف} رکاوٹ کی تعریف ہے۔اسے استعمال کرتے ہوئے مساوات \حوالہ{مساوات_بدلتا_مزاحمت_دوری_تعلق}، مساوات \حوالہ{مساوات_بدلتا_امالہ_تعددی_تعلق_الف} اور مساوات \حوالہ{مساوات_بدلتا_برق_گیر_دوری_مساوات_الف} سے بالترتیب مزاحمت، امالہ گیر اور برق گیر کی رکاوٹ لکھتے ہیں۔
\begin{gather}
\begin{aligned}\label{مساوات_بدلتا_پرزوں_کی_رکاوٹ_الف}
Z_R&=R\\
Z_L&=j \omega L=j X_L\\
Z_C&=\frac{1}{j \omega C}=-\frac{j}{\omega C}=-j X_C
\end{aligned}
\end{gather}
درج بالا میں برق گیر کی رکاوٹ لکھتے ہوئے \عددی{\tfrac{1}{j}=\tfrac{1\times j}{j \times j}=\frac{j}{-1}=-j} کا استعمال کیا گیا ہے۔یوں امالی متعاملیت اور برق گیری متعاملیت درج ذیل ہیں۔
\begin{gather}
\begin{aligned}
X_L&=\omega L\\
X_C&=\frac{1}{\omega C}
\end{aligned}
\end{gather} 

%========================
\ابتدا{مشق}
مزاحمت \عددی{R=\SI{30}{\ohm}}، امالہ \عددی{L=\SI{20}{\milli\henry}} اور برق گیر \عددی{C=\SI{2000}{\micro\farad}} کی رکاوٹ \عددی{\SI{100}{\radian\per\second}}، \عددی{\SI{1000}{\radian\per\second}} اور \عددی{\SI{9000}{\hertz}} تعدد پر دریافت کریں۔

جوابات:پہلی تعدد پر \عددی{Z_R=\SI{30}{\ohm}}، \عددی{Z_L=j2 \, \si{\ohm}}، \عددی{Z_C=-j5 \, \si{\ohm}} ہیں۔\\
دوسری تعدد پر \عددی{Z_R=\SI{30}{\ohm}}، \عددی{Z_L=j20 \, \si{\ohm}}، \عددی{Z_C=-j0.5 \, \si{\ohm}} ہیں۔\\
تیسری تعدد پر \عددی{Z_R=\SI{30}{\ohm}}، \عددی{Z_L=j1131 \, \si{\ohm}}، \عددی{Z_C=-j0.00884 \, \si{\ohm}} ہیں۔\\
\انتہا{مشق}
%==================

قوانین کرخوف وقتی دائرہ کار کے علاوہ تعددی دائرہ کار میں بھی لاگو ہوتے ہیں۔صفحہ \حوالہصفحہ{حصہ_مزاحمت_سلسلہ_وار_مساوی} پر حصہ \حوالہ{حصہ_مزاحمت_سلسلہ_وار_مساوی} میں سلسلہ وار جڑے مزاحمتوں کا مساوی مزاحمت مساوات \حوالہ{مساوات_مزاحمتی_متعدد_سلسلہ_مساوی_مزاحمت} میں حاصل کیا گیا۔اسی طرح صفحہ \حوالہصفحہ{حصہ_مزاحمت_متعدد_متوازی_کا_مساوی} پر حصہ \حوالہ{حصہ_مزاحمت_متعدد_متوازی_کا_مساوی} میں متعدد مساوی مزاحمتوں کا مساوی مزاحمت مساوات \حوالہ{مساوات_مزاحمتی_متوازی_مساوی} میں پیش کیا گیا۔بالکل اسی طرح متعدد سلسلہ وار جڑے رکاوٹ اور متعدد متوازی رکاوٹ کے مساوی رکاوٹ حاصل کی جا سکتی ہے۔مشق میں آپ سے ایسا ہی کرنے کو کہا گیا ہے۔

مساوات \حوالہ{مساوات_بدلتا_سلسلہ_وار_مساوی_رکاوٹ_الف} متعدد سلسلہ وار رکاوٹ کی مساوی رکاوٹ دیتی ہے جبکہ مساوات \حوالہ{مساوات_بدلتا_متوازی_مساوی_رکاوٹ_الف} متعدد متوازی رکاوٹوں کی مساوی رکاوٹ دیتی ہے۔
\begin{align}\label{مساوات_بدلتا_سلسلہ_وار_مساوی_رکاوٹ_الف}
\bZ_s=\bZ_1+\bZ_2+\bZ_3+\cdots+\bZ_n \quad \text{\RL{سلسلہ وار رکاوٹوں کا مساوی رکاوٹ}}
\end{align}
%
\begin{align}\label{مساوات_بدلتا_متوازی_مساوی_رکاوٹ_الف}
\frac{1}{\bZ_m}=\frac{1}{\bZ_1}+\frac{1}{\bZ_2}+\frac{1}{\bZ_3}+\cdots+\frac{1}{\bZ_n} \quad \text{\RL{متوازی رکاوٹوں کا مساوی رکاوٹ}}
\end{align}
آپ دیکھ سکتے ہیں کہ یہ مساوات ہوبہو مزاحمتوں کی مساوات کی طرح ہیں۔
%=============
\ابتدا{مشق}
صفحہ \حوالہصفحہ{شکل_مزاحمتی_متعدد_مزاحمت_تقسیم_دباو} پر شکل \حوالہ{شکل_مزاحمتی_متعدد_مزاحمت_تقسیم_دباو} میں سلسلہ وار مزاحمت جڑے دکھائے گئے ہیں۔مزاحمتوں کی جگہ رکاوٹ نسب کرتے ہوئے، مخلوط دباو اور مخلوط رو کے استعمال سے مساوی رکاوٹ کی مساوات حاصل کریں۔ اسی طرح متعدد رکاوٹوں کو متوازی جوڑتے ہوئے ان کا مساوی رکاوٹ حاصل کریں۔

جوابات:مساوات \حوالہ{مساوات_بدلتا_سلسلہ_وار_مساوی_رکاوٹ_الف} اور مساوات \حوالہ{مساوات_بدلتا_متوازی_مساوی_رکاوٹ_الف}
\انتہا{مشق}
%===============
\ابتدا{مثال}
متعدد برق گیر سلسلہ وار جڑے ہیں۔ان کی انفرادی رکاوٹیں استعمال کرتے ہوئے مساوی رکاوٹ حاصل کریں۔مساوی رکاوٹ سے مساوی برقی گیر دریافت کریں۔

حل:برق گیر \عددی{C_1} تا \عددی{C_n} کی \عددی{\omega} تعدد پر رکاوٹیں \عددی{\tfrac{1}{j\omega C_1}}، \عددی{\tfrac{1}{j\omega C_2}} ،\عددی{\cdots}،\عددی{\tfrac{1}{j\omega C_n}} ہوں گی۔ان کے مساوی برق گیر کو \عددی{C_s} کہتے ہوئے مساوی رکاوٹ \عددی{\tfrac{1}{j\omega C_s}} لکھا جائے گا۔یوں مساوات \حوالہ{مساوات_بدلتا_سلسلہ_وار_مساوی_رکاوٹ_الف} کے تحت درج ذیل لکھا جا سکتا ہے۔
\begin{align*}
\frac{1}{j\omega C_s}=\frac{1}{j\omega C_1}+\frac{1}{j\omega C_2}+\frac{1}{j\omega C_3}+\cdots+\frac{1}{j\omega C_n}
\end{align*}
اس مساوات کے دونوں اطراف کو \عددی{j\omega} سے ضرب دیتے ہوئے درج ذیل ملتا ہے جو عین مساوات \حوالہ{مساوات_امالہ_سلسلہ_وار_برق_گیر_کا_مساوی} ہی ہے۔
\begin{align*}
\frac{1}{C_s}=\frac{1}{C_1}+\frac{1}{C_2}+\frac{1}{C_3}+\cdots+\frac{1}{C_n}
\end{align*}
\انتہا{مثال}
%====================================
\ابتدا{مشق}
متعدد برق گیر متوازی جڑے ہیں۔ان کی رکاوٹیں استعمال کرتے ہوئے مساوات \حوالہ{مساوات_بدلتا_متوازی_مساوی_رکاوٹ_الف} کی مدد سے ان کا مساوی رکاوٹ حاصل کریں۔مساوی رکاوٹ سے مساوی برق گیر کی مساوات حاصل کریں۔متعدد متوازی برق گیر کا مساوی برق گیر مساوات \حوالہ{مساوات_امالہ_متوازی_برق_گیر_کا_مساوی} دیتی ہے۔
\انتہا{مشق}
%==================================
\ابتدا{مشق}
متعدد امالہ گیر متوازی جڑے ہیں۔ان کی رکاوٹیں استعمال کرتے ہوئے مساوات \حوالہ{مساوات_بدلتا_متوازی_مساوی_رکاوٹ_الف} کی مدد سے ان کا مساوی رکاوٹ حاصل کریں۔مساوی رکاوٹ سے مساوی امالہ گیر کی مساوات حاصل کریں۔

جواب:صفحہ \حوالہصفحہ{مساوات_امالہ_متوازی_مساوی_مساوات_قیمت} پر مساوات \حوالہ{مساوات_امالہ_متوازی_مساوی_مساوات_قیمت}
\انتہا{مشق}
%==================================
\ابتدا{مشق}
متعدد امالہ گیر سلسہ جڑے ہیں۔ان کی رکاوٹیں استعمال کرتے ہوئے مساوات \حوالہ{مساوات_بدلتا_سلسلہ_وار_مساوی_رکاوٹ_الف} کی مدد سے ان کا مساوی رکاوٹ حاصل کریں۔مساوی رکاوٹ سے مساوی امالہ گیر کی مساوات حاصل کریں۔

جواب:صفحہ \حوالہصفحہ{مساوات_امالہ_سلسلہ_وار_امالہ_گیر_الف} پر مساوات \حوالہ{مساوات_امالہ_سلسلہ_وار_امالہ_گیر_الف}
\انتہا{مشق}
%==================================
\ابتدا{مثال}\شناخت{مثال_بدلتا_سلسلہ_وار_مزاحمت_امالہ_برق_گیر_الف}
شکل \حوالہ{شکل_بدلتا_سلسلہ_وار_مزاحمت_امالہ_برق_گیر_الف} میں منبع دباو کو درپیش مساوی رکاوٹ \عددی{\SI{50}{\hertz}} اور \عددی{\SI{2000}{\radian\per\second}} تعدد پر دریافت کریں۔دباو \عددی{v(t)=30\cos(\omega t+45^{\circ}) \, \si{\volt}} کی صورت میں دونوں تعدد پر وقتی دائرہ کار میں رو دریافت کریں۔
\begin{figure}
\centering
\begin{tikzpicture}
\draw(0,0) to [american voltage source,l={$v(t)$}]++(0,\y) to [resistor,i={$i(t)$},l={$\SI{10}{\ohm}$}]++(\x,0) to [inductor,l={$\SI{2}{\milli\henry}$}]++(\x,0) to [capacitor,l={$\SI{500}{\micro\farad}$}]++(0,-\y) to [short] (0,0);
\end{tikzpicture}
\caption{مثال \حوالہ{مثال_بدلتا_سلسلہ_وار_مزاحمت_امالہ_برق_گیر_الف} کا دور۔}
\label{شکل_بدلتا_سلسلہ_وار_مزاحمت_امالہ_برق_گیر_الف}
\end{figure}

حل:مساوات \حوالہ{مساوات_بدلتا_پرزوں_کی_رکاوٹ_الف} سے انفرادی پرزوں کی رکاوٹ \عددی{\SI{50}{\hertz}} تعدد پر حاصل کرتے ہیں۔
\begin{align*}
Z_R&=\SI{10}{\ohm}\\
Z_L&=j 2 \pi \times 50 \times 2\times 10^{-3}=j0.6283 \, \si{\ohm}\\
Z_C&=\frac{1}{j 2\pi \times 50 \times 500 \times 10^{-6}}=-j6.3662\, \si{\ohm}
\end{align*}
چونکہ تمام پرزے سلسلہ وار جڑے ہیں لہٰذا ان کا مساوی رکاوٹ درج ذیل ہو گا۔
\begin{align*}
\bZ_s=10+j0.6283-j6.3662=10-j5.7379 \, \si{\ohm}
\end{align*}
دباو کو دوری سمتیہ صورت میں لکھتے ہوئے تعددی دائرہ کار میں رو حاصل کرتے ہیں۔
\begin{align*}
\hat{I}=\frac{\hat{V}}{\bZ_s}=\frac{30\phase{45^{\circ}}}{10-j5.7379}=\frac{30\phase{45^{\circ}}}{11.5292\phase{-29.85^{\circ}}}=2.6\phase{74.85^{\circ}} \, \si{\ampere}
\end{align*}
اس سے وقتی دائرہ کار میں رو لکھتے ہیں۔
\begin{align*}
i(t)=2.6\cos(100\pi t +74.85^{\circ})\, \si{\ampere}
\end{align*}
اب \عددی{\SI{2000}{\radian\per\second}} پر قیمتیں دریافت کرتے ہیں۔انفرادی رکاوٹ درج ذیل ہیں
\begin{align*}
Z_R&=\SI{10}{\ohm}\\
Z_L&=j 2000\times 2\times 10^{-3}=j4 \, \si{\ohm}\\
Z_C&=\frac{1}{j 2000\times 500 \times 10^{-6}}=-j1\, \si{\ohm}
\end{align*}
جن سے مساوی رکاوٹ درج ذیل حاصل ہوتا ہے۔
\begin{align*}
\bZ_s=10+j4-j1=10+j3=10.44\phase{16.7^{\circ}} \, \si{\ohm}
\end{align*}
یوں دوری رو درج ذیل ہو گی
\begin{align*}
\hat{I}=\frac{30\phase{45^{\circ}}}{10.44\phase{16.7^{\circ}}}=2.87\phase{28.3^{\circ}}
\end{align*}
جس سے وقتی دائرہ کار میں رو لکھتے ہیں۔
\begin{align*}
i(t)=2.87\cos(2000t+28.3^{\circ})\, \si{\ampere}
\end{align*}
آپ نے دیکھا کہ \عددی{\SI{50}{\hertz}} پر کُل رکاوٹ برق گیر کی خاصیت رکھتا ہے یعنی \عددی{\bZ=R-jX} لکھا جاتا ہے جبکہ \عددی{\SI{2000}{\radian\per\second}} پر \عددی{\bZ=R+jX} لکھا جاتا ہے جو امالی خاصیت کو ظاہر کرتا ہے۔
\انتہا{مثال}
%=================================
\ابتدا{مشق}\شناخت{مشق_بدلتا_متوازی_الف}
شکل \حوالہ{شکل_بدلتا_متوازی_الف} میں وقتی دائرہ کار میں رو حاصل کریں۔تعدد \عددی{\SI{1000}{\radian\per\second}} ہے۔

\begin{figure}
\centering
\begin{tikzpicture}
\draw(0,0) to [american voltage source,l={${55\phase{-66^{\circ}}}$}]++(0,\y) to [short,i={$i(t)$}]++(\x,0) to [short]++(2*\x,0) to [capacitor,l_={$\SI{50}{\micro\farad}$}]++(0,-\y) to [short]++(-3*\x,0);
\draw(\x,0) to [resistor,*-*,l={$\SI{80}{\ohm}$}]++(0,\y);
\draw(2*\x,0) to [inductor,*-*,l={$\SI{5}{\milli\henry}$}]++(0,\y);
\end{tikzpicture}
\caption{مشق \حوالہ{مشق_بدلتا_متوازی_الف} کا دور۔}
\label{شکل_بدلتا_متوازی_الف}
\end{figure} 

جواب:\عددی{8.28\cos(1000t-151.23^{\circ}) \, \si{\ampere}}
\انتہا{مشق}
%=======================

بدلتی رو ادوار میں برقی رکاوٹ \عددی{\bZ} کے علاوہ \اصطلاح{برقی فراوانی}\فرہنگ{فراوانی}\حاشیہب{admittance}\فرہنگ{admittance} \عددی{\bY} بھی نہایت اہم ثابت ہوتی ہے۔رکاوٹ کے بالعکس متناسب کو فراوانی کہتے ہیں۔
\begin{align}
\bY=\frac{1}{\bZ}
\end{align}
مخلوط رکاوٹ کی صورت میں فراوانی بھی مخلوط ہو گی۔فراوانی کو سیمنز \عددی{\si{\siemens}} میں ناپا جاتا ہے۔فراوانی کو مستطیل طرز درج ذیل لکھا جاتا ہے
\begin{align}
\bY=G+jB
\end{align}
جہاں \عددی{G} کو \اصطلاح{ایصالیت}\فرہنگ{ایصالیت}\حاشیہب{conductance}\فرہنگ{conductance} اور \عددی{B} کو \اصطلاح{تاثریت}\فرہنگ{تاثریت}\حاشیہب{susceptance}\فرہنگ{susceptance} کہتے ہیں۔

رکاوٹ سے فراوانی کے اجزاء درج ذیل مساوات سے شروع کرتے ہوئے
\begin{align}
G+jB=\frac{1}{R+jX}
\end{align}
حاصل کی جا سکتی ہے یعنی
\begin{gather}
\begin{aligned}
G&=\frac{R}{R^2+X^2}\\
B&=\frac{-X}{R^2+X^2}
\end{aligned}
\end{gather}
اسی طرح فراوانی کے اجزاء سے رکاوٹ کے اجزاء درج ذیل حاصل ہوتے ہیں۔
\begin{gather}
\begin{aligned}
R&=\frac{G}{G^2+B^2}\\
X&=\frac{-B}{G^2+B^2}
\end{aligned}
\end{gather}
آپ دیکھ سکتے ہیں کہ مخلوط رکاوٹ کی صورت میں \عددی{G} اور \عددی{R} آپس میں بالعکس متناسب نہیں ہیں۔اسی طرح \عددی{B} اور \عددی{X} بھی آپس میں بالعکس متناسب نہیں ہیں۔اگر رکاوٹ میں \عددی{X=0} ہو تب \عددی{G=\tfrac{1}{R}} ہو گا۔

انفرادی پرزوں کی فراوانی درج ذیل ہے۔
\begin{gather}
\begin{aligned}
Y_R&=\frac{1}{R}=G\\
Y_L&\frac{1}{j \omega L}=\frac{1}{\omega L} \phase{-90^{\circ}}\\
Y_C&=j \omega C=\omega C \phase{90^{\circ}}
\end{aligned}
\end{gather}
جہاں انفرادی مزاحمت کی صورت میں \عددی{\tfrac{1}{R}=G} لکھا گیا ہے۔

قوانین کرخوف فراوانی پر بھی لاگو ہوتے ہیں لہٰذا باب دوم کی طرز پر سلسلہ وار اور متوازی جڑے فراوانی کی مساوی فراوانی بالترتیب درج ذیل مساوات سے حاصل کی جا سکتی ہے۔
\begin{align}
\frac{1}{\bZ_s}&=\frac{1}{\bY_1}+\frac{1}{\bY_2}+\frac{1}{\bY_3}+\cdots+\frac{1}{\bY_n} \quad \text{\RL{سلسلہ وار جڑے}}\\
\bY_m &=\bY_1+\bY_2+\bY_3+\cdots+\bY_n\quad \text{\RL{متوازی جڑے}}
\end{align}
%================
\ابتدا{مثال}\شناخت{مثال_بدلتا_متوازی_ب}
شکل \حوالہ{شکل_بدلتا_متوازی_ب} میں منبع کے متوازی جڑے دور کی فراوانی \عددی{\SI{500}{\radian\per\second}} پر دریافت کرتے ہوئے رو \عددی{\hat{I}} حاصل کریں۔ 

\begin{figure}
\centering
\begin{tikzpicture}
\draw(0,0) to [american voltage source,l={${120\phase{56^{\circ}}}$}]++(0,2*\y) to [short,i={${\hat{I}}$}]++(\x,0) to [short]++(2*\x,0) to [capacitor,l_={$\SI{100}{\micro\farad}$}]++(0,-2*\y) to [short] (0,0);
\draw (\x,0)  to [resistor,*-*,l={$\SI{10}{\ohm}$}]++(0,2*\y);
\draw(2*\x,0) to [resistor,*-,l={$\SI{5}{\ohm}$}]++(0,\y) to [inductor,-*,l={$\SI{10}{\milli\henry}$}]++(0,\y);
\end{tikzpicture}
\caption{مثال \حوالہ{مثال_بدلتا_متوازی_ب} کا دور۔}
\label{شکل_بدلتا_متوازی_ب}
\end{figure}

حل:دور میں تین متوازی حصوں کے انفرادی رکاوٹ لکھتے  ہیں۔
\begin{align*}
Z_1&=\SI{10}{\ohm}\\
Z_2&=5+j 500\times 10\times 10^{-3}=5+j5 \, \si{\ohm}\\
Z_3&=\frac{1}{j 500 \times 100 \times 10^{-6}}=-j20 \, \si{\ohm}
\end{align*}
یوں تینوں حصوں کے فراوانی درج ذیل ہو گی۔
\begin{align*}
Y_1&=\frac{1}{10}=\SI{0.1}{\siemens}\\
Y_2&=\frac{1}{5+j5}=0.1-j0.1 \, \si{\siemens}\\
Y_3&=\frac{1}{-j20}=j0.05 \, \si{\siemens}
\end{align*}
یوں تینوں حصوں کو متوازی جوڑنے سے درج ذیل مساوی فراوانی حاصل ہو گی
\begin{align*}
\bY_m=(0.1)+(0.1-j0.1)+(j0.05)=0.2-j0.05 \, \si{\siemens}
\end{align*}
جسے استعمال کرتے ہوئے رو حاصل کرتے ہیں۔
\begin{align*}
\hat{I}&=\bY \hat{V}\\
&=(0.2-j0.05) (120 \phase{56^{\circ}})\\
&=24.74\phase{41.96^{\circ}} \, \si{\ampere}
\end{align*} 
\انتہا{مثال}
%=================
\ابتدا{مشق}\شناخت{مشق_بدلتا_متوازی_پ}
شکل \حوالہ{شکل_بدلتا_متوازی_پ} میں منبع کے متوازی دور کی فراوانی دریافت کرتے ہوئے \عددی{\hat{I}} حاصل کریں۔
\begin{figure}
\centering
\begin{tikzpicture}
\draw(0,0) to [american voltage source,l={${22\phase{10^{\circ}}}$}]++(0,\y) to [short,i={${\hat{I}}$}]++(\x,0) to [short]++(3*\x,0) to [inductor,l_={${j4 \, \si{\ohm}}$}]++(0,-\y) to [short] (0,0);
\draw (\x,0)  to [resistor,*-*,l={$\SI{4}{\ohm}$}]++(0,\y);
\draw(2*\x,0) to [inductor,*-*,l={${j2 \, \si{\ohm}}$}]++(0,\y);
\draw(3*\x,0) to [capacitor,*-*,l={${-j4 \, \si{\ohm}}$}]++(0,\y);
\end{tikzpicture}
\caption{مشق \حوالہ{مشق_بدلتا_متوازی_پ} کا دور۔}
\label{شکل_بدلتا_متوازی_پ}
\end{figure}

جواب:\عددی{12.298\phase{-53.4^{\circ}}\,\si{\ampere}}
\انتہا{مشق}
%============

آئیں مختلف انداز میں جڑے متعدد پرزوں کی مساوی رکاوٹ حاصل کرنا ایک مثال کی مدد سے سیکھیں۔مساوی رکاوٹ حاصل کرنے کا عمل بالکل ویسا ہی ہے جیسا مزاحمتی دور کا مساوی مزاحمت حاصل کرنے کا عمل۔مزاحمتی دور میں حقیقی اعداد استعمال ہوتے ہیں جبکہ رکاوٹی دور میں مخلوط اعداد استعمال ہوتے ہیں۔
%=============
\ابتدا{مثال}\شناخت{مثال_بدلتا_متعدد_رکاوٹ_مساوی_الف}
شکل \حوالہ{شکل_بدلتا_متعدد_رکاوٹ_مساوی_الف}-الف میں متعدد پرزے مختلف طرز پر جڑے دکھائے گئے ہیں۔دور کے دو سروں پر مساوی رکاوٹ \عددی{\bZ} دریافت کریں۔

\begin{figure}
\centering
\begin{subfigure}{1\textwidth}
\centering
\begin{circuitikz}
\draw(0,0) to [short,o-]++(\x/2,0)coordinate(kA) to [inductor,l={$j6\,\si{\ohm}$}]++(\x,0) to [resistor,l={$\SI{8}{\ohm}$}]++(\x,0)coordinate(kB);
\draw(kA) to [short,*-]++(0,\y) to [capacitor,l={$-j4 \,\si{\ohm}$}]++(\x,0) to [resistor,l={$\SI{2}{\ohm}$}]++(\x,0) to [short,-*] (kB);
\draw(kB) to [inductor,l={$j12\,\si{\ohm}$}]++(0,-\y) to [capacitor,l={$-j10\,\si{\ohm}$}]++(0,-\y) to [resistor,-*,l={$\SI{6}{\ohm}$}]++(0,-\y)coordinate(kD);
\draw(kB) to [resistor,l={$\SI{4}{\ohm}$}]++(\x,0) to [inductor,l={$j2 \, \si{\ohm}$}]++(\x,0) to [capacitor,l={$-j6\,\si{\ohm}$}]++(\x,0) to [short,-*]++(0,-\y)coordinate(kC);
\draw(0,-3*\y) to [short,o-] ++(5*\x+\x/2,0) to [short,-*]++(0,\y) to [short]++(\x/2,0) to [capacitor,l={$-j2 \,\si{\ohm}$}]++(0,\y) to [short]++(-\x,0) to [inductor,l_={$j10\,\si{\ohm}$}]++(0,-\y) to [short]++(\x/2,0);
\draw[stealth-](\x/4,-\y-\y/2) --++(-\x/4,0) --++(0,-\y/8)node[below]{$\bZ$};
\end{circuitikz}
\caption*{(الف)}
\end{subfigure}
\begin{subfigure}{0.5\textwidth}
\centering
\begin{tikzpicture}
\draw(0,0) to [short,o-]++(\x/2,0)coordinate(kA) to [european resistor,l={$\bZ_1$}]++(\x,0)coordinate(kB);
\draw(kA) to [short,*-]++(0,\y/2) to [european resistor,l=${\bZ_2}$]++(\x,0) to [short,-*] (kB);
\draw(kB) to [european resistor,l={$\bZ_3$}]++(\x,0) to [european resistor,l={$\bZ_5$}]++(0,-\y) to [short,-o]++(-2.5*\x,0);
\draw(kB) to [european resistor,-*,l={$\bZ_4$}]++(0,-\y);
\draw[stealth-](\x/4,-\y/2) --++(-\x/4,0) --++(0,-\y/8)node[below]{$\bZ$};
\end{tikzpicture}
\caption*{(ب)}
\end{subfigure}
\caption{مثال \حوالہ{مثال_بدلتا_متعدد_رکاوٹ_مساوی_الف} کا دور۔}
\label{شکل_بدلتا_متعدد_رکاوٹ_مساوی_الف}
\end{figure}

حل:شکل \حوالہ{شکل_بدلتا_متعدد_رکاوٹ_مساوی_الف}-ب میں دور کے مختلف حصوں کی نشاندہی کی گئی ہے جن کا مساوی رکاوٹ آسانی سے حاصل کیا جا سکتا ہے۔ان حصوں کی رکاوٹ دریافت کرتے ہیں۔
\begin{align*}
\bZ_1&=8+j6 \, \si{\ohm}\\
\bZ_2&=2-j4 \, \si{\ohm}\\
\bZ_3&=4+j2-j6=4-j4 \, \si{\ohm}\\
\bZ_4&=6-j10+j12=6+j2 \,\si{\ohm}
\end{align*}
اور 
\begin{align*}
\frac{1}{\bZ_5}&=\frac{1}{j10}+\frac{1}{-j2}\\
&=\frac{1}{j10}-\frac{1}{j2}\\
&=\frac{j2-j10}{(j10)(j2)}\\
&=\frac{-j8}{-20} \, \si{\siemens}
\end{align*}
سے درج ذیل ملتا ہے۔
\begin{align*}
\bZ_5&=\frac{20}{j8}=-j\frac{5}{2} \, \si{\ohm}
\end{align*}
رکاوٹ \عددی{\bZ_3} اور \عددی{\bZ_5} سلسلہ وار جڑے ہیں لہٰذا ان کا مساوی رکاوٹ درج ذیل ہو گا۔
\begin{align*}
\bZ_{35}=\bZ_3+\bZ_5=4-j4-j\frac{5}{2}=4-j7.5 \, \si{\ohm}
\end{align*}
اب \عددی{\bZ_4} اور \عددی{\bZ_{35}} متوازی ہیں لہٰذا ان رکاوٹ کی فراوانی دریافت کرتے ہیں۔یوں
\begin{align*}
\bY_4&=\frac{1}{\bZ_4}\\
&=\frac{1}{6+j2}\\
&=\left(\frac{1}{6+j2}\right)\left( \frac{6-j2}{6-j2}\right)\\
&=\frac{6-j2}{36+4}\\
&=0.15-j 0.05
\end{align*}
اور
\begin{align*}
\bY_{35}&=\frac{1}{\bZ_{35}}\\
&=\frac{1}{4-j7.5}\\
&=\frac{4+j7.5}{4^2+7.5^2}\\
&=0.05536+j0.10381
\end{align*}
حاصل کرتے ہوئے درج ذیل لکھا جا سکتا ہے
\begin{align*}
\bY_{435}&=\bY_4+\bY_{35}\\
&=0.15-j 0.05+0.05536+j0.10381\\
&=0.20536+j0.05381 \, \si{\siemens}
\end{align*}
جس سے 
\begin{align*}
\bZ_{435}&=\frac{1}{\bY_{435}}\\
&=\frac{1}{0.20536+j0.05381}\\
&=4.55665-j1.19397 \, \si{\ohm}
\end{align*}
حاصل ہوتا ہے جو متوازی جڑے \عددی{\bZ_4} اور \عددی{\bZ_{35}} کا مساوی رکاوٹ ہے۔رکاوٹ \عددی{\bZ_1} اور \عددی{\bZ_2} متوازی جڑے ہیں۔ان کا مساوی رکاوٹ درج ذیل ہو گا۔
\begin{align*}
\bZ_{12}&=\frac{\bZ_1 \bZ_2}{\bZ_1+\bZ_2}\\
&=\frac{(8+j6)(2-j4)}{(8+j6)+(2-j4)}\\
&=3.46154-j2.69231 \, \si{\ohm}
\end{align*}
یوں شکل \حوالہ{شکل_بدلتا_متعدد_رکاوٹ_مساوی_الف} میں دیے دور کا مساوی مزاحمت درج ذیل ہو گا۔
\begin{align*}
\bZ&=\bZ_{12}+\bZ_{435}\\
&=3.46154-j2.69231+4.55665-j1.19397\\
&=8.01819-j3.88628 \, \si{\ohm}
\end{align*}
\انتہا{مثال}
%=================
\ابتدا{مشق}\شناخت{مشق_بدلتا_متعدد_متوازی_رکاوٹ_مساوی_الف}
شکل \حوالہ{شکل_بدلتا_متعدد_متوازی_رکاوٹ_مساوی_الف} میں \عددی{\bZ} حاصل کریں۔
\begin{figure}
\centering
\begin{tikzpicture}
\draw(0,0) to [inductor,o-,l={$j4 \, \si{\ohm}$}] ++(\x,0) to [resistor,l={$\SI{2}{\ohm}$}]++(0,-\y) to [capacitor,l={$-j6 \, \si{\ohm}$}]++(0,-\y) to [capacitor,l={$-j2$},-o]++(-\x,0);
\draw(\x,0) to [short,*-]++(\x,0) to [resistor,l={$\SI{8}{\ohm}$}]++(0,-\y) to [inductor,l={$j16\, \si{\ohm}$}]++(0,-\y) to [short,-*]++(-\x,0);
\draw[stealth-](\x/4,-\y)--++(-\x/4,0)--++(0,-\y/8)node[below]{$\bZ$};
\end{tikzpicture}
\caption{مشق \حوالہ{مشق_بدلتا_متعدد_متوازی_رکاوٹ_مساوی_الف} کا دور۔}
\label{شکل_بدلتا_متعدد_متوازی_رکاوٹ_مساوی_الف}
\end{figure}

جواب:\عددی{\bZ=\tfrac{24}{5}-j\tfrac{22}{5} \, \si{\ohm}}
\انتہا{مشق}
%============================

\حصہ{دوری سمتیات کے اشکال}
رکاوٹ کی قیمت تعدد پر منحصر ہوتی ہے۔یوں دور میں رو اور دباو کا دارومدار بھی تعدد پر ہو گا۔دوری سمتی اشکال کی مدد سے رو اور دباو پر تعدد کے اثر پر غور کرنے میں مدد ملتی ہے۔آئیں اس پر چند مثال دیکھیں۔
%=======================
\ابتدا{مثال}\شناخت{مثال_بدلتا_دوری_سمتیات_الف}
شکل \حوالہ{شکل_بدلتا_دوری_سمتیات_الف} میں \عددی{\hat{I}}، \عددی{\hat{V}_R}، \عددی{\hat{V}_L} اور \عددی{\hat{V}_m} کے دوری سمتیہ مختلف تعدد پر  کھینچیں۔تعدد \عددی{\omega=\SI{1000}{\radian\per\second}} پر \عددی{\hat{I}=5\phase{0^{\circ}}\, \si{\ampere}} کی صورت میں \عددی{\hat{V}_m} حاصل کریں۔

\begin{figure}
\centering
\begin{subfigure}{0.5\textwidth}
\centering
\begin{tikzpicture}[american voltages]
\draw(0,0) to [american voltage source,l={$\hat{V}_m$}]++(0,\y) to [short,i={$\hat{I}$}]++(\x/4,0) to [resistor,l={$\SI{2}{\ohm}$},v={$\hat{V}_R$}]++(\x,0) to [short]++(\x/4,0) to [inductor,l={$\SI{1.5}{\milli\henry}$},v={$\hat{V}_L$}]++(0,-\y) to [short]++(-\x-\x/2,0);
\end{tikzpicture}
\caption*{(الف)}
\end{subfigure}%
\begin{subfigure}{0.5\textwidth}
\centering
\begin{tikzpicture}
\draw[gray](0,0)--++(3,0);
\draw[gray](0,0)--++(0,2);
\draw[-latex](0,0)--++(1,0)node[below]{$\hat{I}$};
\draw[-latex](0,0)--++(2,0)node[below]{$\hat{V}_R$};
\draw[-latex](0,0)--++(0,1.5)node[left]{$\hat{V}_L$};
\end{tikzpicture}
\caption*{(ب)}
\end{subfigure}
\begin{subfigure}{0.5\textwidth}
\centering
\begin{tikzpicture}
\draw[gray](0,0)--++(3,0);
\draw[gray](0,0)--++(0,2);
\draw[-latex](0,0)--++(1,0)node[pos=0.5,below]{$\hat{I}$};
\draw[-latex](0,0)--++(2,0)node[below]{$\hat{V}_R$};
\draw[-latex](2,0)--++(0,1.5)node[pos=0.5,right]{$\hat{V}_L$};
\draw[-latex](0,0)--++(2,1.5)node[pos=0.75,above left]{$\hat{V}_m$};
\draw[dashed](2,1.5)--++(0,0.5);
\end{tikzpicture}
\caption*{(پ)}
\end{subfigure}%
\begin{subfigure}{0.5\textwidth}
\centering
\begin{tikzpicture}
\draw[gray](0,0)--++(3,0);
\draw[gray](0,0)--++(0,2);
\draw[-stealth]([shift={(0:0.6)}]0,0) arc (0:20:0.6);
\draw(2/3*20:0.8)node[right]{$\phi$};
\begin{scope}[rotate=20]
\draw[-latex](0,0)--++(1,0)node[pos=0.7,above]{$\hat{I}$};
\draw[-latex](0,0)--++(2,0)node[below]{$\hat{V}_R$};
\draw[-latex](2,0)--++(0,1.5)node[pos=0.5,right]{$\hat{V}_L$};
\draw[-latex](0,0)--++(2,1.5)node[pos=0.75,above left]{$\hat{V}_m$};
\draw[dashed](2,1.5)--++(0,0.5);
\end{scope}
\end{tikzpicture}
\caption*{(ت)}
\end{subfigure}
\caption{مثال \حوالہ{مثال_بدلتا_دوری_سمتیات_الف} کے اشکال۔}
\label{شکل_بدلتا_دوری_سمتیات_الف}
\end{figure}


حل:دوری سمتیات کے خط کسی ایک دوری سمتیہ کے حوالے سے کھینچے جاتے ہیں۔ہم \عددی{\hat{I}} کو حوالہ سمتیہ تصور کرتے ہوئے آگے بڑھتے ہیں۔مزید، ہم اس دوری سمتیہ کو صفر زاویے پر تصور کرتے ہیں یعنی ہم \عددی{\hat{I}=I_0\phase{0^{\circ}}} تصور کرتے ہیں۔تعدد \عددی{\omega} پر مزاحمتی رکاوٹ \عددی{\bZ_R=R} جبکہ  امالی رکاوٹ \عددی{\bZ_L=\omega L \phase{90^{\circ}}} ہو گی لہٰذا ان پرزوں پر دباو درج ذیل ہو گا۔
\begin{align*}
\hat{V}_R&=\hat{I} \bZ_R=I_0 R \phase{0^{\circ}}\\
\hat{V}_L&=\hat{I} \bZ_L=I_0 \omega L \phase{90^{\circ}}
\end{align*}
یوں مزاحمت پر دباو عین رو کے ہم زاویہ ہے جبکہ امالہ پر دباو، رو سے \عددی{90^{\circ}} آگے ہے۔شکل \حوالہ{شکل_بدلتا_دوری_سمتیات_الف}-ب میں ان دوری سمتیات کو دکھایا گیا ہے۔چونکہ مزاحمتی رکاوٹ کی قیمت پر تعدد کا کوئی اثر نہیں لہٰذا \عددی{\hat{V}_R} کی قیمت اور زاویہ تعدد تبدیل کرنے سے تبدیل نہیں ہوتے۔اس کے برعکس امالی رکاوٹ تعدد کے راست متناسب ہے لہٰذا تعدد بڑھانے سے \عددی{\bZ_L} کی قیمت بڑھے گی اور یوں \عددی{\hat{V}_L} کا حیطہ بھی بڑھے گا جبکہ اس کا زاویہ جوں کا توں رہے گا۔آپ دیکھ سکتے ہیں کہ \عددی{\omega=\SI{0}{\radian\per\second}} پر \عددی{\hat{V}_L} کی قیمت صفر ہو گی جبکہ تعدد بڑھانے سے \عددی{\hat{V}_L} کی نوک خیالی محدد پر رہتے ہوئے  مرکز سے دور ہو گی۔

شکل \حوالہ{شکل_بدلتا_دوری_سمتیات_الف}-الف سے درج ذیل لکھا جا سکتا ہے۔
\begin{align*}
\hat{V}_m=\hat{V}_R+\hat{V}_L
\end{align*}
شکل \حوالہ{شکل_بدلتا_دوری_سمتیات_الف}-پ میں اس سمتی جمع کو دکھایا گیا ہے جہاں دم سے سر جوڑنے کا طریقہ استعمال کیا گیا ہے۔تعدد کو کم یا زیادہ کرنے  سے \عددی{\hat{V}_L} کا حیطہ کم اور زیادہ ہو گا لہٰذا شکل میں \عددی{\hat{V}_m} کی نوک نقطہ دار لکیر پر حرکت کرے گی۔صفر تعدد کی صورت میں \عددی{\hat{V}_m=\hat{V}_R} ہو گا جبکہ لامتناہی  تعدد پر \عددی{\hat{V}_m} کا زاویہ تقریباً \عددی{90^{\circ}} ہو گا۔
 
\عددی{\omega=\SI{1000}{\radian\per\second}} اور \عددی{\hat{I}=5\phase{0^{\circ}}\, \si{\ampere}} کی صورت میں درج ذیل لکھا جا سکتا ہے
\begin{align*}
\hat{V}_R&=\hat{I} \bZ_R=(5\phase{0^{\circ}})(2)=10\phase{0^{\circ}} \, \si{\volt}\\
\hat{V}_L&=\hat{I} \bZ_L=(5\phase{0^{\circ}})(1000 \times 1.5\times 10^{-3})=7.5\phase{90^{\circ}}\, \si{\volt}
\end{align*}
جس سے منبع کا دباو درج ذیل ملتا ہے۔
\begin{align*}
\hat{V}_m&=10\phase{0^{\circ}}+7.5\phase{90^{\circ}}\\
&=10+j7.5\\
&=12.5\phase{36.87^{\circ}}
\end{align*}
یہی جواب شکل \حوالہ{شکل_بدلتا_دوری_سمتیات_الف}-پ سے  بھی ترسیمی طریقے سے حاصل کیا جا سکتا ہے۔

یہاں بتلاتا چلوں کہ حوالہ دوری سمتیہ کا زاویہ صفر درجے رکھنا ضروری نہیں ہے۔ہم \عددی{\hat{I}=I_0\phase{\phi}} لے سکتے ہیں۔ایسی صورت میں تمام سمتیات اسی زاویے سے گھوم جائیں گے۔شکل \حوالہ{شکل_بدلتا_دوری_سمتیات_الف}-ت میں ایسا ہی دکھایا گیا ہے۔
\انتہا{مثال}
%===============

\ابتدا{مثال}\شناخت{مثال_بدلتا_دوری_سمتیات_ب}
شکل \حوالہ{شکل_بدلتا_دوری_سمتیات_ب} میں \عددی{\hat{I}}، \عددی{\hat{V}_R}، \عددی{\hat{V}_L} اور \عددی{\hat{V}_m} کے دوری سمتیہ مختلف تعدد پر  کھینچیں۔

\begin{figure}
\centering
\begin{subfigure}{0.5\textwidth}
\centering
\begin{tikzpicture}[american voltages]
\draw(0,0) to [american voltage source,l={$\hat{V}_m$}]++(0,\y) to [short,i={$\hat{I}$}]++(\x/4,0) to [resistor,l={$\SI{2}{\ohm}$},v={$\hat{V}_R$}]++(\x,0) to [short]++(\x/4,0) to [capacitor,l={$\SI{150}{\micro\farad}$},v={$\hat{V}_L$}]++(0,-\y) to [short]++(-\x-\x/2,0);
\end{tikzpicture}
\caption*{(الف)}
\end{subfigure}
\begin{subfigure}{0.5\textwidth}
\centering
\begin{tikzpicture}
\draw[gray](0,0)--++(3,0);
\draw[gray](0,0.5)--(0,-2);
\draw[-latex](0,0)--++(1,0)node[above]{$\hat{I}$};
\draw[-latex](0,0)--++(2,0)node[above]{$\hat{V}_R$};
\draw[-latex](0,0)--++(0,-1.5)node[left]{$\hat{V}_C$};
\end{tikzpicture}
\caption*{(ب)}
\end{subfigure}%
\begin{subfigure}{0.5\textwidth}
\centering
\begin{tikzpicture}
\draw[gray](0,0)--++(3,0);
\draw[gray](0,0.5)--(0,-2);
\draw[-latex](0,0)--++(1,0)node[pos=0.5,above]{$\hat{I}$};
\draw[-latex](0,0)--++(2,0)node[above]{$\hat{V}_R$};
\draw[-latex](2,0)--++(0,-1.5)node[pos=0.5,right]{$\hat{V}_C$};
\draw[-latex](0,0)--++(2,-1.5)node[pos=0.75,below left]{$\hat{V}_m$};
\draw[dashed](2,-1.5)--++(0,-0.5);
\end{tikzpicture}
\caption*{(پ)}
\end{subfigure}
\caption{مثال \حوالہ{مثال_بدلتا_دوری_سمتیات_ب} کے اشکال۔}
\label{شکل_بدلتا_دوری_سمتیات_ب}
\end{figure}


حل:رو کو حوالہ لیتے ہیں۔یوں \عددی{\hat{I}=I_0\phase{0^{\circ}}} ہو گا۔تعدد \عددی{\omega} پر مزاحمتی رکاوٹ \عددی{\bZ_R=R} جبکہ  برق گیر کی رکاوٹ \عددی{\bZ_C=\tfrac{1}{\omega C} \phase{-90^{\circ}}} ہو گی لہٰذا ان پرزوں پر دباو درج ذیل ہو گا۔
\begin{align*}
\hat{V}_R&=\hat{I} \bZ_R=I_0 R \phase{0^{\circ}}\\
\hat{V}_C&=\hat{I} \bZ_C=\frac{I_0}{ \omega C} \phase{-90^{\circ}}
\end{align*}
یوں مزاحمت پر دباو عین رو کے ہم زاویہ ہے جبکہ برق گیر پر دباو، رو سے \عددی{90^{\circ}} پیچھے ہے۔شکل \حوالہ{شکل_بدلتا_دوری_سمتیات_ب}-ب میں ان دوری سمتیات کو دکھایا گیا ہے۔چونکہ مزاحمتی رکاوٹ کی قیمت پر تعدد کا کوئی اثر نہیں لہٰذا \عددی{\hat{V}_R} کی قیمت اور زاویہ تعدد تبدیل کرنے سے تبدیل نہیں ہوتے۔اس کے برعکس برق گیر رکاوٹ تعدد کے بالعکس متناسب ہے لہٰذا تعدد بڑھانے سے \عددی{\bZ_C} کی قیمت کم ہو گی اور یوں \عددی{\hat{V}_C} کا حیطہ بھی کم ہو گا جبکہ اس کا زاویہ جوں کا توں رہے گا۔آپ دیکھ سکتے ہیں کہ لامتناہی تعدد پر \عددی{\hat{V}_C} کی قیمت صفر ہو گی جبکہ تعدد کم کرنے سے \عددی{\hat{V}_C} کی نوک خیالی محدد پر رہتے ہوئے  مرکز سے دور ہو گی۔

شکل \حوالہ{شکل_بدلتا_دوری_سمتیات_ب}-الف سے درج ذیل لکھا جا سکتا ہے۔
\begin{align*}
\hat{V}_m=\hat{V}_R+\hat{V}_C
\end{align*}
شکل \حوالہ{شکل_بدلتا_دوری_سمتیات_ب}-پ میں اس سمتی جمع کو دکھایا گیا ہے جہاں دم سے سر جوڑنے کا طریقہ استعمال کیا گیا ہے۔تعدد کو کم یا زیادہ کرنے  سے \عددی{\hat{V}_C} کا حیطہ زیادہ اور کم ہو گا لہٰذا شکل میں \عددی{\hat{V}_m} کی نوک نقطہ دار لکیر پر حرکت کرے گی۔لامتناہی تعدد کی صورت میں \عددی{\hat{V}_m=\hat{V}_R} ہو گا جبکہ صفر  تعدد پر \عددی{\hat{V}_m} کا زاویہ تقریباً \عددی{-90^{\circ}} ہو گا۔ 
\انتہا{مثال}
%===============
\ابتدا{مثال}\شناخت{مثال_بدلتا_دوری_سمتیات_پ}
شکل \حوالہ{شکل_بدلتا_دوری_سمتیات_پ}-الف  میں دکھائے دور کے  \عددی{\hat{I}}، \عددی{\hat{V}_R}، \عددی{\hat{V}_L}، \عددی{\hat{V}_C} اور \عددی{\hat{V}_m} دوری سمتیہ مختلف تعدد پر کھینچیں۔

\begin{figure}
\centering
\begin{subfigure}{0.5\textwidth}
\centering
\begin{tikzpicture}[american voltages]
\draw(0,0) to [american voltage source,l={$\hat{V}_m$}]++(0,\y) to [short,i={$\hat{I}$}]++(\x/2,0)to [resistor,l={$\SI{8}{\ohm}$},v_={$\hat{V}_R$}]++(\x,0) to [short]++(\x/2,0) to [inductor,l={$\SI{4}{\milli\henry}$},v_={$\hat{V}_L$}]++(0,-\y) to [short]++(-\x/2,0) to [capacitor,l={$\SI{100}{\micro\farad}$},v_={$\hat{V}_C$}]++(-\x,0) to [short]++(-\x/2,0);
\end{tikzpicture}
\caption*{(الف)}
\end{subfigure}%
\begin{subfigure}{0.5\textwidth}
\centering
\begin{tikzpicture}
\draw[gray](0,0)--++(3,0);
\draw[gray](0,-2)--++(0,4);
\draw[-latex](0,0)--++(1,0)node[below]{$\hat{I}$};
\draw[-latex](0,0)--(2,0)node[below]{$\hat{V}_R$};
\draw[-latex](0,0)--(0,1.5)node[left]{$\hat{V}_L$};
\draw[-latex](0,0)--(0,-1)node[left]{$\hat{V}_C$};
\end{tikzpicture}
\caption*{(ب)}
\end{subfigure}
\begin{subfigure}{0.5\textwidth}
\centering
\begin{tikzpicture}
\draw[gray](0,0)--++(3,0);
\draw[gray](0,-0.75)--++(0,2.5);
\draw[-latex](0,0)--++(1,0)node[pos=0.5,below]{$\hat{I}$};
\draw[-latex](0,0)--(2,0)node[pos=0.75,below]{$\hat{V}_R$};
\draw[-latex](2,0)--++(0,1)node[pos=0.5,right]{$\hat{V}_L+\hat{V}_C$};
\draw[-latex](0,0)--(2,1)node[pos=0.6,above left]{$\hat{V}_m$};
\draw[dashed](2,1)--++(0,0.5);
\draw[dashed](2,0)--++(0,-0.5);
\end{tikzpicture}
\caption*{(پ)}
\end{subfigure}
\caption{مثال \حوالہ{مثال_بدلتا_دوری_سمتیات_پ} کے اشکال۔}
\label{شکل_بدلتا_دوری_سمتیات_پ}
\end{figure}

حل:یہاں بھی رو کو حوالہ دوری سمتیہ \عددی{\hat{I}=I_0\phase{0^{\circ}}\,\si{\ampere}} تصور کرتے ہیں۔مزاحمت کا دباو اسی سمت میں ہو گا جبکہ امالہ کا دباو \عددی{90^{\circ}} آگے اور برق گیر کا دباو \عددی{90^{\circ}} پیچھے ہو گا۔شکل \حوالہ{شکل_بدلتا_دوری_سمتیات_پ}-ب میں انہیں دکھایا گیا ہے۔

جن تعدد پر \عددی{\omega L > \tfrac{1}{\omega C}} ہو، ان تعدد پر امالہ کا دباو، برق گیر کے دباو سے زیادہ ہو گا لہٰذا \عددی{\hat{V}_L+\hat{V}_C} کا زاویہ \عددی{90^{\circ}} ہو گا یعنی ان کا مجموعی تاثیر امالی ہو گا۔شکل \حوالہ{شکل_بدلتا_دوری_سمتیات_پ}-پ میں ایسی ہی تعدد پر درج ذیل سمتی مجموعہ دکھایا گیا ہے۔
\begin{align*}
\hat{V}_m=\hat{V}_R+\hat{V}_L+\hat{V}_C
\end{align*}
جس تعدد پر \عددی{\omega L=\tfrac{1}{\omega C}} ہو یعنی \عددی{(\omega_0=\tfrac{1}{\sqrt{LC}})} اس تعدد پر \عددی{\hat{V}_L+\hat{V}_C=0} ہو گا لہٰذا درج بالا مجموعے سے \عددی{\hat{V}_m=\hat{V}_R} حاصل ہو گا۔تعدد \عددی{\omega_0} سلسلہ وار جڑے \عددی{LC} کی \اصطلاح{قدرتی تعدد}\فرہنگ{قدرتی تعدد} یا اس کی \اصطلاح{گھمکی تعدد}\فرہنگ{گھمکی تعدد}\حاشیہب{resonant frequency}\فرہنگ{resonant frequency} کہلاتی ہے۔تعدد کم اور زیادہ کرنے سے \عددی{\hat{V}_m} کی نوک نقطہ دار لکیر پر حرکت کرتی ہے۔عین \عددی{\omega_0} پر \عددی{\hat{V}_m=\hat{V}_R} ہو گا۔گھمکی تعدد سے زیادہ تعدد پر \عددی{\hat{V}_m} کی نوک، نقطہ دار لکیر پر رہتے ہوئے، افقی محدد سے اوپر ہو گی جبکہ \عددی{\omega_0} سے کم تعدد پر \عددی{\hat{V}_m} کی نوک، نقطہ دار لکیر پر رہتے ہوئے، افقی محدد سے نیچے ہو گی۔ شکل \حوالہ{شکل_بدلتا_دوری_سمتیات_پ}-پ  گھمکی تعدد سے زیادہ تعدد کی صورت حال دکھا رہا ہے۔ 
\انتہا{مثال} 
%================
\ابتدا{مثال}\شناخت{مثال_بدلتا_رو_بالمقابل_دباو}
گزشتہ مثال میں \عددی{\hat{V}_m=120\phase{40^{\circ}} \, \si{\volt}} ہے۔شکل \حوالہ{شکل_بدلتا_دوری_سمتیات_پ}-الف میں پرزوں کی قیمتیں استعمال کرتے ہوئے \عددی{\omega=\SI{1000}{\radian\per\second}} پر  \عددی{\hat{V}_m} اور \عددی{\hat{I}} دوری سمتیوں کے خط کھینچیں۔

حل:اس مرتبہ ہم \عددی{\hat{V}_m} کو حوالہ لیتے ہیں۔دی گئی تعدد پر درج ذیل ہو گا۔
\begin{align*}
\bZ_R&=\SI{8}{\ohm}\\
\bZ_L&=1000\times 0.004 \phase{90^{\circ}}=4\phase{90^{\circ}}\, \si{\ohm}\\
\bZ_C&=\frac{1}{1000 \times 100\times 10^{-6}} \phase{-90^{\circ}}=10\phase{-90^{\circ}} \, \si{\ohm}
\end{align*}
شکل \حوالہ{شکل_بدلتا_دوری_سمتیات_پ}-الف کو دیکھ کر درج ذیل لکھا جا سکتا ہے
\begin{align*}
\hat{V}_m=\hat{V}_R+\hat{V}_L+\hat{V}_C
\end{align*}
جس میں قیمتیں پُر کرتے ہوئے
\begin{align*}
120 \phase{40^{\circ}}&=\hat{I} \bZ_R+\hat{I}\bZ_L+\hat{I} \bZ_C\\
&=\hat{I}\left(8+4\phase{90^{\circ}}+10\phase{-90^{\circ}}\right)\\
&=\hat{I}\left(8+j4-j10\right)\\
&=\hat{I}\left(8-j6\right)\\
&=\hat{I} \left(10\phase{-36.87^{\circ}}\right)
\end{align*}
ملتا ہے۔اس مساوات سے درج ذیل حاصل ہوتا ہے۔
\begin{align*}
\hat{I}&=\frac{120\phase{40^{\circ}}}{10\phase{-36.87^{\circ}}}\\
&=12\phase{76.87^{\circ}}
\end{align*}
\عددی{\hat{V}_m} اور \عددی{\hat{I}} کو شکل \حوالہ{شکل_بدلتا_رو_بالمقابل_دباو}-الف میں دکھایا گیا ہے جہاں حیطوں کو درست تناسب سے نہیں دکھایا گیا ہے۔

دور کی قدرتی تعدد درج ذیل ہے
\begin{align*}
\omega_0&=\frac{1}{\sqrt{0.004\times 100 \times 10^{-6}}}=\SI{1581}{\radian\per\second}
\end{align*}
جبکہ دور کو \عددی{\SI{1000}{\radian\per\second}} پر حل کیا گیا ہے۔یہی وجہ ہے کہ دور برق گیر تاثیر رکھتا ہے اور رو منبع کے دباو سے \عددی{37.87^{\circ}} درجے  آگے ہے۔عین قدرتی تعدد پر
\begin{align*}
\bZ_L&=j1581\times 0.004=6.32\phase{90^{\circ}} \, \si{\ohm}\\
\bZ_C&=\frac{1}{j1581\times 100\times 10^{-6}}=6.32\phase{-90^{\circ}}\,\si{\ohm}
\end{align*}
حاصل ہوتے ہیں۔

عموماً حوالہ دوری سمتیہ کا زاویہ \عددی{0^{\circ}} رکھا جاتا ہے۔یوں اگر ہم منبع دباو کا زاویہ \عددی{40^{\circ}} کی جگہ \عددی{0^{\circ}} چنتے تب \عددی{\hat{V}_m=120\phase{0^{\circ}}} لکھا جاتا اور رو درج ذیل حاصل ہوتی۔
\begin{align*}
\hat{I}=\frac{\hat{V}_m}{\bZ}=\frac{120\phase{0^{\circ}}}{10\phase{-36.87^{\circ}}}=12\phase{36.87^{\circ}}\,\si{\ampere}
\end{align*}
انہیں شکل  \حوالہ{شکل_بدلتا_رو_بالمقابل_دباو}-ب میں دکھایا گیا ہے۔آپ دیکھ سکتے ہیں کہ شکل-الف کو  گھڑی کے گردش کی سمت میں \عددی{40^{\circ}} گھمانے سے شکل-ب ملتا ہے۔یوں حوالہ سمتیہ کا زاویہ تبدیل کرنے سے تمام دوری سمتیہ کی شکل گھوم جاتی ہے، البتہ انفرادی دوری سمتیات کے تعلق پر کوئی فرق نہیں پڑتا۔یوں شکل-الف اور شکل-ب دونوں میں رو \عددی{36.87^{\circ}} درجے دباو سے آگے ہے۔ 
\begin{figure}
\centering
\begin{subfigure}{0.5\textwidth}
\centering
\begin{tikzpicture}
\draw[gray](0,0)--++(4,0);
\draw[gray](0,0)--++(0,2.5);
\draw[-latex](0,0)--++(40:2.4)node[right]{$\hat{V}_m$};
\draw[-latex](0,0)--++(76.87:1.2)node[above]{$\hat{I}$};
\draw[-stealth]([shift={(0:0.7)}]0,0) arc (0:40:0.7);
\draw(2/3*40:0.8)node[right]{$40^{\circ}$};
\draw[-stealth]([shift={(40:0.8)}]0,0) arc (40:76.87:0.8);
\draw(3/4*76.87:0.9)node[right,rotate=40]{$36.87^{\circ}$};
\end{tikzpicture}
\caption*{(الف)}
\end{subfigure}%
\begin{subfigure}{0.5\textwidth}
\centering
\begin{tikzpicture}
\draw[gray](0,0)--++(4,0);
\draw[gray](0,0)--++(0,2.5);
\draw[-latex](0,0)--++(0:2.4)node[below]{$\hat{V}_m$};
\draw[-latex](0,0)--++(36.87:1.2)node[above]{$\hat{I}$};
\draw[-stealth]([shift={(0:0.7)}]0,0) arc (0:36.87:0.7);
\draw(2/3*40:0.8)node[right]{$36.87^{\circ}$};
\end{tikzpicture}
\caption*{(ب)}
\end{subfigure}%
\caption{مثال \حوالہ{مثال_بدلتا_رو_بالمقابل_دباو} کے دوری سمتیوں کے خط۔}
\label{شکل_بدلتا_رو_بالمقابل_دباو}
\end{figure}
\انتہا{مثال}
%===================
\ابتدا{مثال}\شناخت{مثال_بدلتا_متوازی_تینوں_پرزے_الف}
شکل \حوالہ{شکل_بدلتا_متوازی_تینوں_پرزے_الف} کے دور میں دیے تمام دوری سمتیات کے خط کھینچیں۔
\begin{figure}
\centering
\begin{subfigure}{0.7\textwidth}
\centering
\begin{tikzpicture}[american voltages]
\draw(0,0) to [american voltage source,l={$\hat{V}_m$}]++(0,\y) to [short,i={$\hat{I}_m$}]++(\x,0) to [short]++(2*\x,0)  to [capacitor,i>_={$\hat{I}_C$},l_={$C$},v^={$\hat{V}_C$}]++(0,-\y) to [short] (0,0);
\draw(\x,0) to [resistor,*-*,i<={$\hat{I}_R$},l={$R$},v={$\hat{V}_R$}]++(0,\y);
\draw(2*\x,0) to [inductor,*-*,i<={$\hat{I}_L$},l={$L$},v={$\hat{V}_L$}]++(0,\y);
\end{tikzpicture}
\caption*{(الف)}
\end{subfigure}%
\begin{subfigure}{0.3\textwidth}
\centering
\begin{tikzpicture}[american voltages]
\draw[gray](0,0)--++(3,0);
\draw[gray](0,-1.5)--(0,2);
\draw[-latex](0,0)--++(0:2.7)node[below]{$\hat{V}_m$};
\draw[-latex](0,0)--++(0:1.75)node[below]{$\hat{I}_R$};
\draw[-latex](0,0)--++(0,1.5)node[left]{$\hat{I}_C$};
\draw[-latex](0,0)--++(0,-1)node[left]{$\hat{I}_L$};
\end{tikzpicture}
\caption*{(ب)}
\end{subfigure}
\begin{subfigure}{0.5\textwidth}
\centering
\begin{tikzpicture}[american voltages]
\draw[gray](0,0)--++(3,0);
\draw[gray](0,-1.5)--(0,2);
\draw[-latex](0,0)--++(0:2.7)node[below]{$\hat{V}_m$};
\draw[-latex](0,0)--++(0:1.75)node[below]{$\hat{I}_R$};
\draw[-latex,gray](0,0)--++(0,1.5)node[left]{$\hat{I}_C$};
\draw[-latex,gray](0,0)--++(0,-1)node[left]{$\hat{I}_L$};
\draw[-latex](0,0)--++(0,0.5)node[left]{$\hat{I}_L+\hat{I}_C$};
\end{tikzpicture}
\caption*{(پ)}
\end{subfigure}%
\begin{subfigure}{0.5\textwidth}
\centering
\begin{tikzpicture}[american voltages]
\draw[gray](0,0)--++(3,0);
\draw[gray](0,-1.5)--(0,2);
\draw[-latex](0,0)--++(0:2.7)node[below]{$\hat{V}_m$};
\draw[-latex](0,0)--++(0:1.75)node[pos=0.8,below]{$\hat{I}_R$};
\draw[-latex](1.75,0)--++(0,0.5)node[right]{$\hat{I}_L+\hat{I}_C$};
\draw[-latex](0,0)--++(1.75,0.5)node[pos=0.7,above]{$\hat{I}_m$};
\draw[gray,dashed](1.75,0.5)--++(0,1.5);
\draw[gray,dashed](1.75,0)--++(0,-1.5);
\draw[-latex,gray](0,0)--++(1.75,1.5)node[pos=0.7,above]{$\hat{I}'_m$};
\draw[-latex,gray](0,0)--++(1.75,-1)node[pos=0.7,below]{$\hat{I}''_m$};
\end{tikzpicture}
\caption*{(ت)}
\end{subfigure}
\caption{مثال \حوالہ{مثال_بدلتا_متوازی_تینوں_پرزے_الف} کے اشکال۔}
\label{شکل_بدلتا_متوازی_تینوں_پرزے_الف}
\end{figure}

حل:دباو \عددی{\hat{V}_m} کو حوالہ دوری سمتیہ لیتے ہوئے اس کا زاویہ صفر درجے چنتے ہیں۔ تینوں پرزوں پر \عددی{\hat{V}_m}  دباو پایا جاتا ہے لہٰذا ان کی انفرادی رو درج ذیل ہوں گے۔
\begin{align*}
\hat{I}_R&=\frac{\hat{V}_m}{\bZ_R}=\frac{V_m\phase{0^{\circ}}}{R}=\frac{V_m}{R}\phase{0^{\circ}}\\
\hat{I}_L&=\frac{\hat{V}_m}{\bZ_L}=\frac{V_m\phase{0^{\circ}}}{\omega L \phase{90^{\circ}}}=\frac{V_m}{\omega L} \phase{-90^{\circ}}\\
\hat{I}_C&=\frac{\hat{V}_m}{\bZ_C}=\frac{V_m \phase{0^{\circ}}}{\frac{1}{\omega C} \phase{-90^{\circ}}}=\omega C V_m \phase{90^{\circ}}
\end{align*}
انہیں شکل \حوالہ{شکل_بدلتا_متوازی_تینوں_پرزے_الف}-ب میں دکھایا گیا ہے۔

قدرتی تعدد \عددی{\omega_0=\tfrac{1}{\sqrt{LC}}} پر امالی رکاوٹ اور برق گیری رکاوٹ کے  مقدار برابر \عددی{(\omega L=\tfrac{1}{\omega C})} ہوتے ہیں۔قدرتی تعدد پر \عددی{\hat{I}_m=\hat{I}_R} ہو گا۔قدرتی تعدد سے زیادہ تعدد پر \عددی{\bZ_L>\bZ_C} لہٰذا \عددی{I_C>I_L} ہو گا۔اس صورت حال کو شکل-پ میں دکھایا گیا ہے۔

کرخوف قانون رو سے درج ذیل لکھا جا سکتا ہے
\begin{align*}
\hat{I}_m=\hat{I}_R+\hat{I}_L+\hat{I}_C
\end{align*}
جسے \عددی{\omega>\omega_0}  کی صورت میں  شکل-ت میں دکھایا گیا ہے۔تعدد مزید بڑھانے سے \عددی{\hat{I}_m} کی نوک نقطہ دار لکیر پر رہتے ہوئے افقی محدد سے مزید دور ہو گی۔دوری سمتیہ \عددی{\hat{I}'_m} ایسی صورت کو ظاہر کرتی ہے۔

قدرتی تعدد سے کم تعدد \عددی{(\omega<\omega_0)} پر \عددی{\bZ_C>\bZ_L} اور \عددی{I_L>I_C} ہو گا لہٰذا دوری سمتیہ کی نوک نقطہ دار لکیر پر افقی محدد سے نیچے کی طرف ہو گی۔دوری سمتیہ \عددی{\hat{I}''_m} ایسی صورت کو ظاہر کرتی ہے۔
\انتہا{مثال}
%===================
\ابتدا{مشق}\شناخت{مشق_بدلتا_دو_پرزے_متوازی_الف}
شکل \حوالہ{شکل_بدلتا_دو_پرزے_متوازی_الف}-الف میں تمام رو اور دباو کے دوری سمتیات کے خط کھینچیں۔تقسیم رو کا کلیہ استعمال کیا جا سکتا ہے۔ جواب کو شکل-ب میں دکھایا گیا ہے۔
\begin{figure}
\centering
\begin{subfigure}{1\textwidth}
\centering
\begin{tikzpicture}[american voltages]
\draw(0,0) to [american current source,l={${\hat{I}_m=10\phase{60^{\circ}}\,\si{\ampere}}$}]++(0,\y) to [short]++(2*\x,0) to [inductor,l_={$j10 \, \si{\ohm}$},i>={$\hat{I}_L$}]++(0,-\y) to [short] (0,0);
\draw(\x,0) to [resistor,*-*,i<_={$\hat{I}_R$},l={$\SI{20}{\ohm}$}]++(0,\y);
\draw(2*\x,\y) to [short,*-o]++(\x/2,0)coordinate(kT);
\draw(2*\x,0) to [short,*-o]++(\x/2,0)coordinate(kB);
\draw($(kT)!0.5!(kB)$)node{$\begin{aligned}&+ \\& \hat{V}_0 \\ &- \end{aligned}$};
\end{tikzpicture}
\caption*{(الف)}
\end{subfigure}
\begin{subfigure}{1\textwidth}
\centering
\begin{tikzpicture}[american voltages]
\draw[gray](0,0)--++(4,0);
\draw[gray](0,0)--(0,2.5);
\draw[-latex](0,0)--++(60:2)node[right]{${\hat{I}_m=\SI{10}{\ampere}}$};
\draw[-latex](0,0)--++(123.43:0.894)node[left]{${\hat{I}_R=2\sqrt{5} \, \si{\ampere}}$};
\draw[-latex](0,0)--++(33.43:1.788)node[right]{${\hat{I}_L=4\sqrt{5} \, \si{\ampere}}$};
\draw[-latex](0,0)--++(123.43:3)node[left]{${\hat{V}_0=8\sqrt{125}\,\si{\volt}}$};
\draw[-stealth]([shift={(0:1)}]0,0) arc (0:33.43:1);
\draw(2/3*33.43:1.1)node[right]{$33.43^{\circ}$};
\draw[-stealth]([shift={(0:0.8)}]0,0) arc (0:60:0.8);
\draw(3/4*60:0.9)node[right,rotate=60]{$60^{\circ}$};
\draw[-stealth]([shift={(0:0.6)}]0,0) arc (0:123.43:0.6);
\draw(110:1.2)node[rotate=123.43]{$123.43^{\circ}$};
\end{tikzpicture}
\caption*{(ب)}
\end{subfigure}
\caption{مشق \حوالہ{مشق_بدلتا_دو_پرزے_متوازی_الف} کے اشکال۔}
\label{شکل_بدلتا_دو_پرزے_متوازی_الف}
\end{figure}

\انتہا{مشق}
%===============
\ابتدا{مشق}\شناخت{مشق_بدلتا_دو_پرزے_متوازی_ب}
شکل \حوالہ{شکل_بدلتا_دو_پرزے_متوازی_ب} میں برق گیر کی وہ قیمت دریافت کریں جس پر \عددی{v(t)=120\cos(5500t-30^{\circ}) \, \si{\volt}} اور \عددی{i(t)} ہم زاویہ ہوں گے۔اس تعدد پر \عددی{i(t)} دریافت کریں۔
\begin{figure}
\centering
\begin{tikzpicture}
\draw(0,0) to [american voltage source,l={$v(t)$}]++(0,\y) to [resistor,i_={$i(t)$},l={$\SI{100}{\ohm}$}]++(\x,0) to [inductor,l={$\SI{120}{\micro\henry}$}]++(\x,0) to [capacitor,l={$C$}]++(0,-\y) to [short] (0,0);
\end{tikzpicture}
\caption{مشق \حوالہ{مشق_بدلتا_دو_پرزے_متوازی_ب} کا دور۔}
\label{شکل_بدلتا_دو_پرزے_متوازی_ب}
\end{figure}

جواب:\عددی{C=\SI{275.48}{\micro\farad}}، \عددی{i(t)=1.2\cos(5500t-30^{\circ})\, \si{\ampere}}
\انتہا{مشق}
%===============

\حصہ{کرخوف مساوات}
یک سمتی رو ادوار کو کرخوف کے قوانین سے حل کرنا ہم گزشتہ بابوں میں دیکھ چکے ہیں۔قوانین کرخوف دوری سمتیات پر بھی لاگو ہوتے ہیں۔یوں بدلتی رو ادوار کو کرخوف مساوات سے بالکل یک سمتی رو ادوار کی طرح حل کیا جا سکتا ہے۔یک سمتی رو ادوار حل کرنے کے تمام ترکیب یعنی مسئلہ تھونن، مسئلہ نارٹن، مسئلہ تبادلہ منبع، مسئلہ نفاذ، تقسیم رو اور تقسیم دباو کو بدلتی رو ادوار حل کرنے کے لئے بھی استعمال کیا جاتا ہے۔بدلتی رو ادوار کی صورت میں مخلوط الجبرا کا استعمال کیا جاتا ہے۔گزشتہ حصے میں ہم نے چند سادہ مثال اسی طرح حل کئے۔آئیں نسبتاً مشکل ادوار حل کریں۔متعدد منبع کی صورت میں دوری سمتیات کی ترکیب صرف اس صورت میں قابل استعمال ہو گی جب تمام منبع کی تعدد یکساں ہو البتہ ان کے انفرادی زاویے مختلف ہو سکتے ہیں۔

%==================
\ابتدا{مثال}\شناخت{مثال_بدلتا-متعدد_پرزے_کرخوف_الف}
شکل \حوالہ{شکل_بدلتا-متعدد_پرزے_کرخوف_الف} میں تمام نا معلوم دباو اور رو دریافت کریں۔
\begin{figure}
\centering
\begin{subfigure}{0.5\textwidth}
\centering
\begin{tikzpicture}[american voltages]
\draw(0,0) to [american voltage source,l={$20\phase{30^{\circ}}\,\si{\volt}$}]++(0,\y) to [inductor,i>^={$\hat{I}_1$},l={$j10\,\si{\ohm}$}]++(\x,0) to [resistor,i={$\hat{I}_3$},l={$\SI{1}{\ohm}$}]++(\x,0) to [capacitor,v={$\hat{V}_2$},l={$-j8\,\si{\ohm}$}]++(0,-\y) to [short] (0,0);
\draw(\x,\y) to [resistor,v={$\hat{V}_1$},i={$\hat{I}_2$},l={$\SI{6}{\ohm}$},*-*]++(0,-\y);
\end{tikzpicture}
\caption*{(الف)}
\end{subfigure}%
\begin{subfigure}{0.5\textwidth}
\centering
\begin{tikzpicture}[american voltages]
\draw[gray](0,0)--(3,0);
\draw[gray](0,0)--(0,1);
\draw[-latex](0,0)--++(-33.163:2.395)node[right]{$\hat{I}_1$}coordinate(kA);
\draw[-latex](0,0)--++(-67.224:1.816)node[left]{$\hat{I}_2$}coordinate(kB);
\draw[-latex](0,0)--++(15.65:1.352)node[above]{$\hat{I}_3$}coordinate(kC);
\draw[-stealth]([shift={(0:0.8)}]0,0) arc (0:15.65:0.8);
%\draw(20:0.45)node[above,rotate=15]{$16^{\circ}$};
\draw[-stealth]([shift={(0:0.7)}]0,0) arc (0:-33.163:0.7);
\draw(-33.163*2/3:0.8)node[right]{$33^{\circ}$};
\draw[-stealth]([shift={(0:0.6)}]0,0) arc (0:-67.224:0.6);
\draw(-67.224*2/3:0.7)node[right,rotate={-50}]{$67^{\circ}$};
\draw[gray,dashed](kC)--(kA)--(kB);
\end{tikzpicture}
\caption*{(ب)}
\end{subfigure}%
\caption{مثال \حوالہ{مثال_بدلتا-متعدد_پرزے_کرخوف_الف} کے اشکال۔}
\label{شکل_بدلتا-متعدد_پرزے_کرخوف_الف}
\end{figure}

حل:ہم پہلے منبع سے جڑی کل رکاوٹ حاصل کرتے ہوئے \عددی{\hat{I}_1} دریافت کرتے ہیں جسے جانتے ہوئے \عددی{j2\,\si{\ohm}} امالہ کا دباو حاصل کیا جا سکتا ہے۔اس دباو کو \عددی{\hat{V}_m} سے منفی کرتے ہوئے \عددی{\hat{V}_1} حاصل کیا جائے گا۔اب \عددی{\hat{V}_1} جانتے ہوئے  \عددی{\hat{I}_2} حاصل کیا جا سکتا ہے جسے استعمال کرتے ہوئے \عددی{\hat{I}_3=\hat{I}_1-\hat{I}_2} لکھا جا سکتا ہے۔آخر میں \عددی{\hat{V}_2=\hat{I}_3 (-j8)} لکھتے ہوئے برق گیر کا دباو حاصل کیا جائے  گا۔

منبع کو درج ذیل رکاوٹ نظر آتی ہے۔
\begin{align*}
\bZ&=j10+\frac{6(1-j8)}{6+1-j8}\\
&=j10 +\frac{6-j48}{7-j8}\\
&=j10 +\left(\frac{6-j48}{7-j8}\right)\left(\frac{7+j8}{7+j8}\right)\\
&=j10+\frac{426-j288}{113}\\
&=3.7699+j7.4513\\
&=8.3507\phase{63.163^{\circ}}\,\si{\ohm}
\end{align*}
یوں درج ذیل لکھا جا سکتا ہے
\begin{align*}
\hat{I}_1&=\frac{\hat{V}_m}{\bZ}\\
&=\frac{20\phase{30^{\circ}}}{8.3507\phase{63.163^{\circ}}}\\
&=2.395\phase{-33.163^{\circ}}\, \si{\ampere}
\end{align*}
جس سے \عددی{\hat{V}_1} حاصل کرتے ہیں۔
\begin{align*}
\hat{V}_1&=\hat{V}_m-\hat{I}_1 (j2)\\
&=20\phase{30^{\circ}}-(2.395\phase{-33.163^{\circ}})( 2\phase{90^{\circ}})\\
&=4.219-j10.049\\
&=10.8987\phase{-67.224^{\circ}}\,\si{\volt}
\end{align*}
آپ \عددی{\hat{V}_1} کو یوں بھی حاصل کر سکتے ہیں۔
\begin{align*}
\hat{V}_1&=\frac{6(1-j8)}{6+1-j8} \hat{I}_1\\
&=10.8987\phase{-67.224^{\circ}}\,\si{\volt}
\end{align*}
اس کے علاوہ \عددی{\hat{V}_1} کو تقسیم دباو کے کلیے سے بھی حاصل کیا جا سکتا ہے یعنی
\begin{align*}
\hat{V}_1&=\left(\frac{\frac{6(1-j8)}{6+1-j8}}{j10+\frac{6(1-j8)}{6+1-j8}}\right) 20\phase{30^{\circ}}\\
&=10.8987\phase{-67.224^{\circ}}\,\si{\volt}
\end{align*}
دباو \عددی{\hat{V}_1} جانتے ہوئے \عددی{\hat{I}_2} حاصل کرتے ہیں۔
\begin{align*}
\hat{I}_2&=\frac{\hat{V}_1}{6}\\
&=\frac{10.8987\phase{-67.224^{\circ}}}{6}\\
&=1.816\phase{-67.224^{\circ}}\,\si{\ampere}
\end{align*}
یوں \عددی{\hat{I}_3} درج ذیل ہو گا۔
\begin{align*}
\hat{I}_3 &=\hat{I}_1-\hat{I}_2\\
&=2.395\phase{-33.163^{\circ}}-1.816\phase{-67.224^{\circ}}\\
&=(2.005-j1.310)-(0.703-j1.675)\\
&=1.302+j0.365\\
&=1.352\phase{15.65^{\circ}}\,\si{\ampere}
\end{align*}
آپ \عددی{\hat{I}_3} کو درج ذیل سے بھی حاصل کر سکتے ہیں۔
\begin{align*}
\hat{I}_3&=\frac{\hat{V}_1}{1-j8}\\
&=1.352\phase{15.65^{\circ}}\,\si{\ampere}
\end{align*}
برق گیر کا دباو حاصل کرتے ہیں۔
\begin{align*}
\hat{V}_2&=\hat{I}_3 (-j8)\\
&=(1.352\phase{15.65^{\circ}})(8\phase{-90^{\circ}})\\
&=10.816\phase{-74.35^{\circ}}\,\si{\volt}
\end{align*}
اس دباو کو تقسیم دباو کے کلیے سے بھی حاصل کیا جا سکتا ہے یعنی
\begin{align*}
\hat{V}_2&=\left(\frac{-j8}{1-j8}\right) \hat{V}_1\\
&=10.816\phase{-74.35^{\circ}}\,\si{\volt}
\end{align*}
اس کے علاوہ \عددی{\SI{1}{\ohm}}  مزاحمت میں \عددی{\hat{I}_3} گزرتی ہے۔یوں \عددی{\hat{V}_1} سے اس مزاحمت کی دباو منفی کرنے سے بھی برق گیر کا دباو حاصل کیا جا سکتا ہے یعنی
\begin{align*}
\hat{V}_2&=\hat{V}_1-(\hat{I}_3)(1)\\
&=10.816\phase{-74.35^{\circ}}\,\si{\volt}
\end{align*}
آپ نے دیکھا کہ آپ اپنے مرضی کی کوئی بھی ترکیب استعمال کرتے ہوئے جوابات حاصل کر سکتے ہیں۔شکل-ب میں دوری رو دکھائے گئے ہیں جہاں نقطہ دار لکیر قانون متوازی الاضلاع سے \عددی{\hat{I}_1=\hat{I}_2+\hat{I}_3} دکھاتی ہے۔
\انتہا{مثال}
%=========================
\ابتدا{مشق}\شناخت{مشق_بدلتا_کرخوف_بدلتی_رو_الف}
شکل \حوالہ{شکل_بدلتا_کرخوف_بدلتی_رو_الف} میں
\begin{align*}
v_1(t)&=10\cos(300t+30^{\circ}) \, \si{\volt}\\
v_2(t)&=30\cos(300t+60^{\circ})\,\si{\volt}
\end{align*}
ہیں۔ دوری سمتیات استعمال کرتے ہوئے \عددی{i(t)} حاصل کریں۔
\begin{figure}
\centering
\begin{tikzpicture}
\draw(0,0) to [american voltage source,l={$v_1(t)$}]++(0,\y) to [resistor,i={$i(t)$},l={$\SI{4}{\ohm}$}]++(\x,0) to [capacitor,l={$\SI{200}{\micro\farad}$}]++(\x,0) ++(\x,0) to [american voltage source,l_={$v_2(t)$}]++(-\x,0) ++(\x,0) to [inductor,l={$\SI{050}{\milli\henry}$}]++(\x,0) to [resistor,l={$\SI{2}{\ohm}$}]++(0,-\y) to [short] (0,0);
\end{tikzpicture}
\caption{مشق \حوالہ{مشق_بدلتا_کرخوف_بدلتی_رو_الف} کا دور۔}
\label{شکل_بدلتا_کرخوف_بدلتی_رو_الف}
\end{figure}

جواب:\عددی{i(t)=3.52\cos(300t-91.3^{\circ})\,\si{\ampere}}
\انتہا{مشق}
%============================
 \ابتدا{مشق}\شناخت{مشق_بدلتا_کرخوف_بدلتی_رو_ب}
شکل \حوالہ{شکل_بدلتا_کرخوف_بدلتی_رو_ب} میں \عددی{\hat{V}_0} دریافت کریں۔

\begin{figure}
\centering
\begin{tikzpicture}
\draw(0,0) to [american current source,l={$16\phase{0^{\circ}}\, \si{\ampere}$}]++(0,\y) to [resistor,l={$\SI{2}{\ohm}$}]++(\x,0) to [inductor,l={$j6\,\si{\ohm}$}]++(\x,0) to [capacitor,l_={$-j10\,\si{\ohm}$}]++(0,-\y) to [short] (0,0);
\draw(\x,\y) to [inductor,*-*,l_={$j8\,\si{\ohm}$}]++(0,-\y);
\draw(2*\x,\y) to [short,*-o]++(\x/2,0)coordinate(kT);
\draw(2*\x,0) to [short,*-o]++(\x/2,0)coordinate(kB);
\draw($(kT)!0.5!(kB)$) node {$\begin{aligned}&+ \\ & \hat{V}_0 \\ &-  \end{aligned}$};
\end{tikzpicture}
\caption{مشق \حوالہ{مشق_بدلتا_کرخوف_بدلتی_رو_ب} کا دور۔}
\label{شکل_بدلتا_کرخوف_بدلتی_رو_ب}
\end{figure}

جواب:\عددی{\hat{V}_0=320\phase{-90^{\circ}}\,\si{\volt}}
\انتہا{مشق}
%==============

 \ابتدا{مشق}\شناخت{مشق_بدلتا_کرخوف_بدلتی_رو_پ}
شکل \حوالہ{شکل_بدلتا_کرخوف_بدلتی_رو_پ} میں \عددی{\hat{V}_1} دریافت کریں۔

\begin{figure}
\centering
\begin{tikzpicture}
\draw(0,0) to [american current source,l_={$10\phase{30^{\circ}}\,\si{\ampere}$}]++(0,-2*\y) to [short]++(3*\x,0) to [american voltage source,l_={$40\phase{60^{\circ}}\,\si{\volt}$}]++(0,2*\y) to [resistor,l={$\SI{4}{\ohm}$}]++(-\x,0) to [capacitor,l={$-j20\,\si{\ohm}$}]++(-\x,0) to [short] (0,0);
\draw(\x,-2*\y)node[ground]{} to [resistor,*-,l={$\SI{5}{\ohm}$}]++(0,\y) to [inductor,-*,l={$j10\,\si{\ohm}$}]++(0,\y)node[above]{$\hat{V}_1$};
\end{tikzpicture}
\caption{مشق \حوالہ{مشق_بدلتا_کرخوف_بدلتی_رو_پ} کا دور۔}
\label{شکل_بدلتا_کرخوف_بدلتی_رو_پ}
\end{figure}

جواب:\عددی{\hat{V}_1=182.88\phase{-127.16^{\circ}}\,\si{\volt}}
\انتہا{مشق}
%==============

\حصہ{تجزیاتی تراکیب}
اس حصے میں ہم وہ تمام ترکیب استعمال کریں گے جن سے یک سمتی ادوار حل کیے گئے۔ایسا مثالوں کی مدد سے کیا جائے گا۔

%===================
\ابتدا{مثال}\شناخت{مثال_بدلتا_تراکیب_الف}
شکل \حوالہ{شکل_بدلتا_تراکیب_الف} کو ترکیب جوڑ سے حل کرتے ہوئے \عددی{\hat{V}_1} اور \عددی{\hat{V}_2} دریافت کریں۔
\begin{figure}
\centering
\begin{tikzpicture}
\draw(0,0) to [american voltage source,l={$20\phase{70^{\circ}}\,\si{\volt}$}]++(0,\y) to [inductor,l={$j2\,\si{\ohm}$}]++(\x,0)node[above]{$\hat{V}_1$} to [resistor,l={$\SI{10}{\ohm}$}]++(0,-\y)node[ground]{} to [short] (0,0);
\draw(\x,0) to [short,*-]++(2*\x,0) to [resistor,l_={$\SI{2}{\ohm}$}]++(0,\y) to [capacitor,l_={$-j4\,\si{\ohm}$}]++(-\x,0) to [american current source,-*,l_={$2\phase{30^{\circ}}\,\si{\ampere}$}]++(-\x,0);
\draw(2*\x,0) to [inductor,*-*,l_={$j6\,\si{\ohm}$}]++(0,\y)node[above]{$\hat{V}_2$};
\end{tikzpicture}
\caption{مثال \حوالہ{مثال_بدلتا_تراکیب_الف} کا دور۔}
\label{شکل_بدلتا_تراکیب_الف}
\end{figure}

حل:جوڑ \عددی{\hat{V}_1} اور \عددی{\hat{V}_2} پر کرخوف مساوات رو لکھتے ہیں۔
\begin{align}
\frac{\hat{V}_1-20\phase{70^{\circ}}}{j2}+\frac{\hat{V}_1}{10}-2\phase{30^{\circ}}&=0 \label{مساوات_بدلتا_جوڑ_ترکیب_الف}\\
2\phase{30^{\circ}}+\frac{\hat{V}_2}{j6}+\frac{\hat{V}_2}{2-j4}&=0\label{مساوات_بدلتا_جوڑ_ترکیب_ب}
\end{align}
مساوات \حوالہ{مساوات_بدلتا_جوڑ_ترکیب_الف} سے درج ذیل ملتا ہے
\begin{align*}
\hat{V}_1\left(\frac{1}{j2}+\frac{1}{10}\right)&=2\phase{30^{\circ}}+\frac{20\phase{70^{\circ}}}{j2}\\
&=1.7321+j+9.3970-j3.4202
\end{align*}
جسے حل کرتے ہیں۔
\begin{align*}
\hat{V}_1&=\frac{11.1291-j2.4202}{0.1-j0.5}\\
&=8.9347+j20.4713\\
&=22.3361\phase{66.42^{\circ}}\,\si{\volt}
\end{align*}
مساوات \حوالہ{مساوات_بدلتا_جوڑ_ترکیب_ب} سے درج ذیل حاصل ہوتا ہے۔
\begin{align*}
\hat{V}_2&=\frac{-2\phase{30^{\circ}}}{\frac{1}{j6}+\frac{1}{2-j4}}\\
&=-18.5885-j3.8038\\
&=18.97\phase{-168.43^{\circ}} \, \si{\volt}
\end{align*}
\انتہا{مثال}
%===================
\ابتدا{مثال}\شناخت{مثال_بدلتا_تراکیب_ب}
شکل \حوالہ{شکل_بدلتا_تراکیب_ب} کو دائری ترکیب سے حل کرتے ہوئے \عددی{\hat{I}_0} حاصل کریں۔
\begin{figure}
\centering
\begin{tikzpicture}
\draw(0,0) to [american current source,l_={$5\phase{60^{\circ}}\,\si{\ampere}$}]++(0,\yy) to [capacitor,l_={$-j4\,\si{\ohm}$}]++(-\xx,0)node[above]{$b$} to [inductor,l_={$j2\,\si{\ohm}$}]++(0,-\yy)node[below]{$a$} to [short] (0,0);
\draw(0,0) to [short,*-*]++(\xx,0)node[ground]{}node[above left]{$d$} to [resistor,l_={$\SI{10}{\ohm}$}]++(0,\yy)node[above]{$c$} to [american voltage source,-*,l_={$30\phase{-40^{\circ}}\,\si{\volt}$}]++(-\xx,0);
\draw(\xx,0) to [short]++(\xx,0)node[below]{$f$} to [resistor,i<_={$\hat{I}_0$},l_={$\SI{2}{\ohm}$}]++(0,\yy)node[above]{$e$} to [inductor,-*,l_={$j6\,\si{\ohm}$}]++(-\xx,0);
%current
\draw[stealth-] ([shift={(-150:\xx/4)}]-\xx/2,\yy/2) arc (-150:150:\xx/4);
\draw(-\xx/2,0)node[above]{$\hat{I}_1$};
\draw[stealth-] ([shift={(-120:\xx/4)}]\xx/2,\yy/2) arc (-120:120:\xx/4);
\draw(\xx/2,0)node[above]{$\hat{I}_2$};
\draw[stealth-] ([shift={(-120:\xx/4)}]\xx+\xx/2,\yy/2) arc (-120:120:\xx/4);
\draw(\xx+\xx/2,0)node[above]{$\hat{I}_3$};
\end{tikzpicture}
\caption{مثال \حوالہ{مثال_بدلتا_تراکیب_ب} کا دور۔}
\label{شکل_بدلتا_تراکیب_ب}
\end{figure}

حل:تین خانوں میں رو فرض کرتے ہوئے دائرہ \عددی{abcda} اور \عددی{dcefd} پر  کرخوف مساوات دباو لکھتے ہیں۔
\begin{align}
(j2-j4)\hat{I}_1+30\phase{-40^{\circ}}+10(\hat{I}_2-\hat{I}_3)&=0 \label{مساوات_بدلتا_دائری_ترکیب_الف}\\
10(\hat{I_3}-\hat{I}_2)+(2+6j)\hat{I}_3&=0  \label{مساوات_بدلتا_دائری_ترکیب_ب}
\end{align} 
چونکہ \عددی{\hat{I}_1} اور \عددی{\hat{I}_2} منبع رو سے گزرتے ہیں لہٰذا درج ذیل لکھا جائے گا۔
\begin{align}
\hat{I}_2-\hat{I}_1=5\phase{60^{\circ}}  \label{مساوات_بدلتا_دائری_ترکیب_پ}
\end{align}
درج بالا تین ہمزاد مساوات کو حل کرتے ہوئے تینوں دائروں کی رو حاصل ہو گی۔

مساوات \حوالہ{مساوات_بدلتا_دائری_ترکیب_پ} سے \عددی{\hat{I}_1=\hat{I}_2-5\phase{60^{\circ}}} لیتے ہوئے مساوات \حوالہ{مساوات_بدلتا_دائری_ترکیب_الف} میں پُر کرتے اور ترتیب دیتے ہوئے درج ذیل مساوات حاصل ہوتا ہے۔
\begin{gather}
\begin{aligned} \label{مساوات_بدلتا_دائری_ترکیب_ت}
(10-j2)\hat{I}_2-10\hat{I}_3&=-30\phase{-40^{\circ}}-(j2)(5\phase{60^{\circ}})\\
&=-14.3211+j14.2836
\end{aligned}
\end{gather}
مساوات \حوالہ{مساوات_بدلتا_دائری_ترکیب_ب} سے \عددی{\hat{I}_2=\left(\tfrac{12+6j}{10}\right)\hat{I}_3} لیتے ہوئے مساوات \حوالہ{مساوات_بدلتا_دائری_ترکیب_ت} میں پُر کرتے ہیں۔
\begin{align*}
(10-j2)\left(\frac{12+6j}{10}\right)\hat{I}_3-10\hat{I}_3&=-14.3211+j14.2836
\end{align*}
اس کو حل کرنے سے
\begin{align*}
\hat{I}_3&=\frac{-14.3211+j14.2836}{3.2+j3.6}\\
&=0.2411+j4.1924\\
&=4.1993\phase{86.71^{\circ}}\,\si{\ampere}
\end{align*}
حاصل ہوتا ہے۔ شکل سے ظاہر ہے کہ یہی \عددی{\hat{I}_0} ہے یعنی
\begin{align}
\hat{I}_0=\hat{I}_3=4.1993\phase{86.71^{\circ}}\,\si{\ampere}
\end{align}

رو \عددی{\hat{I}_3} جاننے کے بعد مساوات \حوالہ{مساوات_بدلتا_دائری_ترکیب_ب} سے \عددی{\hat{I}_2} حاصل کرتے ہیں۔
\begin{align*}
\hat{I}_2&=\left(\frac{12+6j}{10}\right)\hat{I}_3\\
&=\left(\frac{12+6j}{10}\right)(0.2411+j4.1924)\\
&=-2.2261+j5.1755\\
&=5.63\phase{113.27^{\circ}}\,\si{\volt}
\end{align*}
رو \عددی{\hat{I}_2} جانتے ہوئے مساوات \حوالہ{مساوات_بدلتا_دائری_ترکیب_پ} سے \عددی{\hat{I}_1} حاصل کرتے ہیں۔
\begin{align*}
\hat{I}_1&=\hat{I}_2-5\phase{60^{\circ}} \\
&=(-2.2261+j5.1755)-(2.5+j4.3301)\\
&=-4.7261+j0.8454\\
&=4.8\phase{169.86^{\circ}}\,\si{\volt}
\end{align*}
\انتہا{مثال}
%====================
\ابتدا{مثال}\شناخت{مثال_بدلتا_تراکیب_پ}
شکل \حوالہ{شکل_بدلتا_تراکیب_ب} کو مسئلہ نفاذ سے حل کریں۔
\begin{figure}
\centering
\begin{tikzpicture}
\draw(0,0) to [american current source,l_={$5\phase{60^{\circ}}\,\si{\ampere}$}]++(0,\yy)node[above]{$\hat{V}$} to [capacitor,l_={$-j4\,\si{\ohm}$}]++(-\xx,0) to [inductor,l_={$j2\,\si{\ohm}$}]++(0,-\yy) to [short] (0,0);
\draw(0,0) to [short,*-*]++(\xx,0)node[ground]{} to [resistor,l_={$\SI{10}{\ohm}$}]++(0,\yy) to [short,-*]++(-\xx,0);
\draw(\xx,0) to [short]++(\xx,0) to [resistor,i<_={$\hat{I}'_0$},l_={$\SI{2}{\ohm}$}]++(0,\yy) to [inductor,-*,l_={$j6\,\si{\ohm}$}]++(-\xx,0);
\end{tikzpicture}
\caption{مثال \حوالہ{مثال_بدلتا_تراکیب_پ} کا دور۔ منبع دباو کو قصر دور کیا گیا ہے۔}
\label{شکل_بدلتا_تراکیب_پ}
\end{figure}


حل:مسئلہ نفاذ میں تمام منبع کے انفرادی اثرات کا مجموعہ لیا جاتا ہے۔شکل \حوالہ{شکل_بدلتا_تراکیب_ب} میں منبع دباو کو قصر دور کرتے ہوئے  شکل \حوالہ{شکل_بدلتا_تراکیب_پ} حاصل ہوتا ہے۔منبع رو کے متوازی کل رکاوٹ \عددی{ \bZ_1} حاصل کرتے ہیں۔شکل کو دیکھتے ہوئے درج ذیل لکھا جا سکتا ہے
\begin{align*}
\frac{1}{\bZ_1}&=\frac{1}{j2-j4}+\frac{1}{10}+\frac{1}{2+j6}\\
&=\frac{3}{20}+j\frac{7}{20}
\end{align*}
جس سے
\begin{align*}
\bZ_1&=\frac{1}{\frac{3}{20}+j\frac{7}{20}}\\
&=\frac{30}{29}-j\frac{70}{29}\\
&=2.6261\phase{-66.8^{\circ}} \, \si{\ohm}
\end{align*}
حاصل ہوتا ہے۔یوں دباو \عددی{\hat{V}} درج ذیل ہو گا۔
\begin{align*}
\hat{V}&=(5\phase{60^{\circ}})(2.6261\phase{-66.8^{\circ}})=13.13\phase{-6.8^{\circ}}\,\si{\volt} 
\end{align*}
دباو جانتے ہوئے درکار رو درج ذیل لکھی جا سکتی ہے۔
\begin{gather}
\begin{aligned}\label{مساوات-بدلتا_منبع_رو_کی_رو}
\hat{I}'_0&=\frac{\hat{V}}{2+j6}\\
&=\frac{13.13\phase{-6.8^{\circ}}}{2+j6}\\
&=2.076\phase{-78.37^{\circ}} \, \si{\ampere}
\end{aligned}
\end{gather}
%
\begin{figure}
\centering
\begin{tikzpicture}
\draw(0,0) ++(0,\yy) to [capacitor,l_={$-j4\,\si{\ohm}$}]++(-\xx,0) to [inductor,l_={$j2\,\si{\ohm}$}]++(0,-\yy) to [short] (0,0);
\draw(0,0) to [short]++(\xx,0)node[ground]{} to [resistor,l_={$\SI{10}{\ohm}$}]++(0,\yy) to [american voltage source,i={$\hat{I}$},l_={$30\phase{-40^{\circ}}\,\si{\volt}$}]++(-\xx,0);
\draw(\xx,0) to [short]++(\xx,0) to [resistor,i<_={$\hat{I}''_0$},l_={$\SI{2}{\ohm}$}]++(0,\yy) to [inductor,-*,l_={$j6\,\si{\ohm}$}]++(-\xx,0);
\end{tikzpicture}
\caption{مثال \حوالہ{مثال_بدلتا_تراکیب_ب} کا دور۔منبع رو کو کھلا دور کیا گیا ہے۔}
\label{شکل_بدلتا_تراکیب_ت}
\end{figure}

آئیں اب منبع دباو سے پیدا رو حاصل کریں۔ایسا کرنے کی خاطر منبع رو کو کھلا دور کرتے ہوئے شکل \حوالہ{شکل_بدلتا_تراکیب_ت} حاصل کرتے ہیں۔اس شکل میں \عددی{\hat{I}} جانتے ہوئے تقسیم رو کے کلیے سے \عددی{\hat{I}''_0} حاصل کیا جا سکتا ہے۔آئیں پہلے \عددی{\hat{I}} حاصل کریں۔منبع رو کے ساتھ کل درج ذیل رکاوٹ جڑی ہے۔
\begin{align*}
\bZ_2&=j2-j4+\frac{10(2+j6)}{10+2+j6}\\
&=\frac{10}{3}+j\frac{4}{3}\\
&=3.59\phase{21.8^{\circ}}\,\si{\ohm}
\end{align*}
یوں رو \عددی{\hat{I}} درج ذیل ہو گی۔
\begin{align*}
\hat{I}&=\frac{30\phase{-40^{\circ}}}{3.59\phase{21.8^{\circ}}}\\
&=8.356\phase{-61.8^{\circ}}\,\si{\ampere}
\end{align*}
رو \عددی{\hat{I}} جانتے ہوئے تقسیم رو کے کلیے سے \عددی{\hat{I}''_0} حاصل کرتے ہیں جہاں منفی کی علامت اس لئے استعمال کی گئی ہے کہ  \عددی{\hat{I}} اور \عددی{\hat{I}''_0} کی سمتیں آپس میں الٹ ہیں۔
\begin{align*}
\hat{I}''_0&=-\left(\frac{10}{10+2+j6}\right) \hat{I}\\
&=-\left(\frac{10}{10+2+j6}\right) 8.356\phase{-61.8^{\circ}}\\
&=6.23\phase{91.63^{\circ}}\,\si{\ampere}
\end{align*}
شکل \حوالہ{شکل_بدلتا_تراکیب_پ} اور شکل \حوالہ{شکل_بدلتا_تراکیب_ت} میں حاصل کئے گئے رو کا مجموعہ درکار رو ہے یعنی
\begin{align*}
\hat{I}_0&=\hat{I}'_0+\hat{I}''_0\\
&=2.076\phase{-78.37^{\circ}}+6.23\phase{91.63^{\circ}}\\
&=4.1993\phase{86.71^{\circ}}\,\si{\ampere}
\end{align*}
\انتہا{مثال}
%====================
\ابتدا{مثال}\شناخت{مثال_بدلتا_تراکیب_ٹ}
شکل \حوالہ{شکل_بدلتا_تراکیب_ٹ}-الف کو تبادلہ منبع سے حل کرتے ہوئے \عددی{\hat{I}_0} حاصل کریں۔

\begin{figure}
\centering
\begin{subfigure}{1\textwidth}
\centering
\begin{tikzpicture}
\draw(0,\yy) to [american current source,l_={$10\phase{30^{\circ}}\,\si{\ampere}$}]++(0,-\yy)++(0,\yy) to [capacitor,l_={$-j4\,\si{\ohm}$}]++(-\xx,0) to [resistor,l_={$\SI{2}{\ohm}$}]++(0,-\yy) to [short] (0,0);
\draw(0,0) to [short,*-]++(\xx,0) to [capacitor,*-,l_={$-j12 \, \si{\ohm}$}]++(0,\yy) to [american voltage source,i={$\hat{I}_0$},-*,l_={$100\phase{60^{\circ}}$}]++(-\xx,0);
\draw(\xx,0) to [short]++(\xx,0) to [resistor,l_={$\SI{4}{\ohm}$}]++(0,\yy) to [inductor,-*,l_={$j8\,\si{\ohm}$}]++(-\xx,0);
\end{tikzpicture}
\caption*{(الف)}
\end{subfigure}
\begin{subfigure}{1\textwidth}
\centering
\begin{tikzpicture}
\draw(0,\yy) to [capacitor,l={$-j4\,\si{\ohm}$}]++(-\xx,0)  to [resistor,l_={$\SI{2}{\ohm}$}]++(-\xx,0) to [american voltage source,l={$44.79\phase{-33.43^{\circ}}\,\si{\volt}$}]++(0,-\yy) to [short] (\xx,0);
\draw(0,0) to [short,*-*]++(\xx,0) to [capacitor,l_={$-j12 \, \si{\ohm}$}]++(0,\yy) to [american voltage source,i={$\hat{I}_0$},-*,l_={$100\phase{60^{\circ}}$}]++(-\xx,0);
\draw(\xx,0) to [short]++(\xx,0) to [resistor,l_={$\SI{4}{\ohm}$}]++(0,\yy) to [inductor,-*,l_={$j8\,\si{\ohm}$}]++(-\xx,0);
\end{tikzpicture}
\caption*{(ب)}
\end{subfigure}
\begin{subfigure}{1\textwidth}
\centering
\begin{tikzpicture}
\draw(0,\yy) to [capacitor,l={$-j4\,\si{\ohm}$}]++(-\xx,0)  to [resistor,l_={$\SI{2}{\ohm}$}]++(-\xx,0) to [american voltage source,l={$44.79\phase{-33.43^{\circ}}\,\si{\volt}$}]++(0,-\yy) to [short] (\xx,0);
\draw(0,0) to [short,-*]++(\xx,0)++(0,\yy) to [american voltage source,i={$\hat{I}_0$},l_={$100\phase{60^{\circ}}$}]++(-\xx,0);
\draw(\xx,0) to [short]++(\xx,0) to [resistor,l_={$\SI{18}{\ohm}$}]++(0,\yy) to [inductor,-*,l_={$j6\,\si{\ohm}$}]++(-\xx,0);
\end{tikzpicture}
\caption*{(پ)}
\end{subfigure}
\caption{مثال \حوالہ{مثال_بدلتا_تراکیب_ٹ} کا دور۔}
\label{شکل_بدلتا_تراکیب_ٹ}
\end{figure}

حل:منبع رو کے متوازی رکاوٹ \عددی{2-j4} جڑی ہے۔ان کو نارٹن مساوی دور تصور کرتے ہوئے ان کی جگہ  تھونن مساوی  دور نسب کرتے ہوئے شکل-ب ملتا ہے جہاں درج ذیل متبادل منبع دباو نسب کیا گیا ہے۔
\begin{align*}
(10\phase{30^{\circ}})(2-j4)&=44.72\phase{-33.43^{\circ}}\,\si{\volt}
\end{align*}
شکل \حوالہ{شکل_بدلتا_تراکیب_ٹ}-ب میں دائیں جانب مساوی رکاوٹ درج ذیل ہے
\begin{align*}
\frac{-j12(4+j8)}{-j12+4+j8}=18+j6
\end{align*}
جسے استعمال کرتے ہوئے شکل-پ ملتی ہے۔شکل-پ کو دیکھ کر درج ذیل لکھا جا سکتا ہے۔
\begin{align*}
\hat{I}_0&=\frac{44.72\phase{-33.43^{\circ}}+100\phase{60^{\circ}}}{2-j4+j6+18}\\
&=\frac{87.3205+j61.9615}{20+j2}\\
&=5.33\phase{29.65^{\circ}}\,\si{\ampere}
\end{align*}
\انتہا{مثال}

\باب{برقرار برقی طاقت}

\حصہ{لمحاتی طاقت}
شکل \حوالہ{شکل_طاقت_پرزے_کو_منتقل} میں بوجھ \عددی{\bZ} کو بدلتی رو منبع  طاقت فراہم کرتا ہے۔اس عمومی دور کے برقرار دباو اور برقرار رو درج ذیل لکھے جا سکتے ہیں۔
\begin{gather}
\begin{aligned}\label{مساوات_طاقت_دباو_رو_عمومی_الف}
v(t)&=V_0\cos(\omega t +\phi_v)\\
i(t)&=I_0\cos(\omega t +\phi_i)
\end{aligned}
\end{gather}
یوں کسی بھی لمحہ بوجھ کو منتقل طاقت درج ذیل ہو گا
\begin{gather}
\begin{aligned}
p(t)&=v(t)i(t)\\
&=V_0 I_0  \cos(\omega t +\phi_v) \cos(\omega t +\phi_i)
\end{aligned}
\end{gather}
جس میں
\begin{align}
\cos \alpha \cos \beta=\frac{\cos(\alpha-\beta)+\cos(\alpha+\beta)}{2}
\end{align}
استعمال کرتے ہوئے
\begin{align}\label{مساوات_طاقت_لمحاتی_طاقت_الف}
p(t)=\frac{V_0 I_0}{2}\left[\cos(\phi_v-\phi_i)+\cos(2\omega t +\phi_v+\phi_i)\right]
\end{align}
ملتا ہے جہاں \عددی{\alpha=\omega t +\phi_v} اور \عددی{\beta=\omega t+\phi_i} لئے گئے ہیں۔آپ دیکھ سکتے ہیں کہ لمحاتی طاقت دو اجزاء کا مجموعہ ہے۔پہلا جزو مستقل طاقت ہے جو وقت کے ساتھ تبدیل نہیں ہوتا جبکہ دوسرا جزو دگنی تعدد کا بدلتی رو طاقت ہے۔  
%
\begin{figure}
\centering
\begin{tikzpicture}
\draw(0,0) to [american voltage source,l={$v(t)$}]++(0,\y) to [short,i={$i(t)$}]++(\x,0) to [european resistor,l={$\bZ$}]++(0,-\y) to [short](0,0);
\end{tikzpicture}
\caption{بدلتی رو دور۔}
\label{شکل_طاقت_پرزے_کو_منتقل}
\end{figure}
%======================
\ابتدا{مثال}\شناخت{مثال_طاقت_عمومی_الف}
شکل \حوالہ{شکل_طاقت_پرزے_کو_منتقل} میں برقرار دباو \عددی{v(t)=15\cos(100 t+45^{\circ})\,\si{\volt}} اور \عددی{\bZ=5\phase{20^{\circ}}\,\si{\ohm}} ہیں۔بوجھ کو منتقل لمحاتی طاقت دریافت کریں۔

حل:دوری سمتیات استعمال کرتے ہوئے
\begin{align*}
\hat{I}&=\frac{15\phase{45^{\circ}}}{5\phase{20^{\circ}}}\\
&=3\phase{25^{\circ}}\,\si{\ampere}
\end{align*}
یعنی
\begin{align*}
i(t)=3\cos(100 t +25^{\circ}) \, \si{\ampere}
\end{align*}
لکھا جا سکتا ہے۔یوں مساوات \حوالہ{مساوات_طاقت_لمحاتی_طاقت_الف} سے لمحاتی طاقت درج ذیل لکھی جا سکتی ہے۔
 \begin{align*}
p(t)&=22.5\left[\cos 20^{\circ}+\cos(200 t +70^{\circ})\right] \\
&=21.143+22.5\cos(200t+70^{\circ})\,\si{\watt}
\end{align*}
دباو، رو اور طاقت کے خط شکل \حوالہ{شکل_طاقت_عمومی_الف} میں دکھائے گئے ہیں۔درج بالا مساوات میں \عددی{\SI{21.143}{\watt}} مستقل طاقت ہے جو وقت کے ساتھ تبدیل نہیں ہوتا جبکہ \عددی{22.5\cos(200t+70^{\circ})\,\si{\watt}} بدلتی رو طاقت ہے جس کی تعدد \عددی{\SI{200}{\radian\per\second}} ہے۔
\begin{figure}
\centering
\begin{subfigure}{0.5\textwidth}
\centering
\begin{tikzpicture}
\begin{axis}[kStyleCircuitsA,small,xlabel=$\omega t$, xtick={90,180,270,360},xticklabels={$90^{\circ}$,$180^{\circ}$,$270^{\circ}$,$360^{\circ}$},]
\addplot[domain=0:370,samples=100]{15*cos(1*x+45)}node[pos=0.7,left]{$v(t)$};
\addplot[domain=0:370,samples=100]{3*cos(1*x+25)}node[pos=0.9,above]{$i(t)$};
\end{axis}%
\end{tikzpicture}%
\end{subfigure}%
\begin{subfigure}{0.5\textwidth}
\centering
\begin{tikzpicture}
\begin{axis}[kStyleCircuitsA,small,xlabel=$\omega t$, xtick={90,180,270,360},xticklabels={$90^{\circ}$,$180^{\circ}$,$270^{\circ}$,$360^{\circ}$},]
\addplot[domain=0:370,samples=100]{21.143+22.5*cos(2*x+25)}node[pos=0.4,left]{$p(t)$};
\end{axis}%
\end{tikzpicture}%
\end{subfigure}%
\caption{مثال \حوالہ{مثال_طاقت_عمومی_الف} کے اشکال۔}
\label{شکل_طاقت_عمومی_الف}
\end{figure}
\انتہا{مثال}
%==============
\ابتدا{مثال}
شکل \حوالہ{شکل_طاقت_پرزے_کو_منتقل} میں \عددی{v(t)=V_0\cos(\omega t +\phi_v)\,\si{\volt}} اور \عددی{\bZ=Z_0\phase{\phi_z}\,\si{\ohm}} ہیں۔رو دریافت کریں۔

حل:دوری سمتیات استعمال کرتے ہوئے
\begin{align*}
\hat{I}&=\frac{V_0\phase{\phi_v}}{Z_0\phase{\phi_z}}\\
&=\frac{V_0}{Z_0}\phase{\phi_v-\phi_z}
\end{align*}
لکھا جا سکتا ہے جس سے وقتی دائرہ کار میں رو درج ذیل حاصل ہوتی ہے۔
\begin{align}
i(t)=\frac{V_0}{Z_0} \cos(\omega t+\phi_v-\phi_z)
\end{align}
مساوات \حوالہ{مساوات_طاقت_دباو_رو_عمومی_الف} میں دیے عمومی رو کے ساتھ موازنہ کرتے ہوئے آپ دیکھ سکتے ہیں کہ \عددی{\phi_i} درحقیقت میں \عددی{\phi_v-\phi_z} کے برابر ہے  جسے درج ذیل لکھا جا سکتا ہے۔
\begin{align}\label{مساوات_طاقت_زاویہ_رکاوٹ_اور_طاقت}
\phi_v-\phi_i=\phi_z
\end{align}
\انتہا{مثال}
%=====================

\حصہ{اوسط طاقت}
دہراتے تفاعل (مثلاً سائن نما تفاعل) کے ایک دوری عرصے پر تکمل کو دوری عرصے سے تقسیم کرنے سے تفاعل کی اوسط قیمت حاصل ہوتی ہے۔یوں مساوات \حوالہ{مساوات_طاقت_دباو_رو_عمومی_الف} میں دیے دباو اور رو کی صورت میں بوجھ کو منتقل اوسط طاقت درج ذیل ہو گی
\begin{gather}
\begin{aligned}\label{مساوات_طاقت_اوسط_تکمل_الف}
P&=\frac{1}{T}\int_{t_0}^{t_0+T} p(t) \dif t\\
&=\frac{V_0 I_0}{T}\int_{t_0}^{t_0+T} \cos(\omega t +\phi_v) \cos(\omega t +\phi_i) \dif t
\end{aligned}
\end{gather}    
جہاں \عددی{t_0} کوئی بھی لمحہ ہو سکتا ہے جبکہ \عددی{T=\tfrac{2\pi}{\omega}} دباو یا رو کا دوری عرصہ ہے۔حقیقت میں ہم ایک دوری عرصے کی بجائے  \عددی{n} مکمل دوری عرصے پر تکمل لیتے ہوئے \عددی{n} دوری عرصے سے تقسیم کرتے ہوئے بھی اوسط قیمت حاصل کر سکتے ہیں۔یوں اوسط طاقت درج ذیل بھی لکھی جا سکتی ہے۔
\begin{align}
P&=\frac{V_0 I_0}{nT}\int_{t_0}^{t_0+nT} \cos(\omega t +\phi_v) \cos(\omega t +\phi_i) \dif t
\end{align} 
مساوات \حوالہ{مساوات_طاقت_لمحاتی_طاقت_الف} کی مدد سے مساوات \حوالہ{مساوات_طاقت_اوسط_تکمل_الف} درج ذیل لکھا جائے گا۔
\begin{gather}
\begin{aligned}\label{مساوات_طاقت_اوسط_تکمل_ب}
P&=\frac{V_0 I_0}{2T}\int_{t_0}^{t_0+T} \left[\cos(\phi_v-\phi_i)+\cos(2\omega t +\phi_v+\phi_i)\right] \dif t\\
&=\frac{V_0 I_0}{2T}\int_{t_0}^{t_0+T} \cos(\phi_v-\phi_i) \dif t+\frac{V_0 I_0}{2T}\int_{t_0}^{t_0+T} \cos(2\omega t +\phi_v+\phi_i)\dif t
\end{aligned}
\end{gather} 
درج بالا تکمل کے دو اجزاء کو باری باری حل کرتے ہیں۔پہلا جزو مستقل ہے لہٰذا اس کو تکمل کے باہر لکھتے ہوئے حل کرتے ہیں۔
\begin{align*}
\frac{V_0 I_0}{2T}\int_{t_0}^{t_0+T} \cos(\phi_v-\phi_i) \dif t&=\frac{V_0 I_0}{2T}\cos(\phi_v-\phi_i) \int_{t_0}^{t_0+T} \dif t\\
&=\left. \frac{V_0 I_0}{2T}\cos(\phi_v-\phi_i) t\right|_{t_0}^{t_0+T}\\
&=\frac{V_0 I_0}{2}\cos(\phi_v-\phi_i)
\end{align*}
اب مساوات \حوالہ{مساوات_طاقت_اوسط_تکمل_ب} کے دوسرے جزو کو حل کرتے ہیں
\begin{align*}
\frac{V_0 I_0}{2T}\int_{t_0}^{t_0+T} \cos(2\omega t +\phi_v+\phi_i)\dif t&=\left. \frac{V_0 I_0}{2T}\frac{ \sin(2\omega t +\phi_v+\phi_i)}{2\omega}\right|_{t_0}^{t_0+T}\\
&=0
\end{align*}
جہاں \عددی{\sin \alpha=\sin(\alpha+T)} کا استعمال کیا گیا ہے۔یوں مساوات \حوالہ{مساوات_طاقت_اوسط_تکمل_ب} سے درج ذیل اوسط طاقت حاصل  ہوتا ہے۔
\begin{align}\label{مساوات_طاقت_عمومی_الف}
P=\frac{V_0 I_0}{2}\cos(\phi_v-\phi_i)
\end{align}
چونکہ \عددی{\cos(\alpha)=\cos(-\alpha)} کے برابر ہے لہٰذا درج بالا مساوات میں کوسائن کا دلیل \عددی{\phi_v-\phi_i} یا \عددی{\phi_i-\phi_v} لکھا جا سکتا ہے۔مساوات \حوالہ{مساوات_طاقت_زاویہ_رکاوٹ_اور_طاقت} کو استعمال کرتے ہوئے درج بالا مساوات کو دوبارہ لکھتے ہیں۔
\begin{align}\label{مساوات_طاقت_عمومی_ب}
P=\frac{V_0 I_0}{2}\cos \phi_z
\end{align}
خالص مزاحمتی رکاوٹ \عددی{\bZ=R\phase{0^{\circ}}} کا زاویہ ہٹاو \عددی{0^{\circ}} ہوتا ہے لہٰذا \عددی{\cos 0^{\circ}=1} لیتے ہوئے مزاحمتی بوجھ کا طاقت
\begin{align}\label{مساوات_طاقت_مزاحمتی_طاقت_الف}
P_{\text{مزاحمتی}}=\frac{V_0 I_0}{2}
\end{align}
ہو گا جہاں \عددی{V_0} سے مراد مزاحمت کے دباو کا حیطہ ہے۔قانون اوہم سے درج بالا کو درج ذیل صورتوں میں بھی لکھا جا سکتا ہے۔
\begin{align}
P_{\text{مزاحمتی}}&=\frac{I^2_0 R}{2} \label{مساوات_طاقت_مزاحمتی_طاقت_ب}\\
P_{\text{مزاحمتی}}&=\frac{V^2_0}{2 R} \label{مساوات_طاقت_مزاحمتی_طاقت_پ}
\end{align}
درج بالا تینوں مساوات کا یک سمتی رو میں مزاحمتی ضیاع کے مساوات کے ساتھ موازنہ کرنے سے معلوم ہوتا ہے کہ موجودہ تینوں مساوات میں کسر کے نچلی جانب دو \عددی{(2)} کا اضافی عدد پایا جاتا ہے۔اس پر آگے جا کر مزید بات ہو گی۔

امالی متعاملیت کی رکاوٹ \عددی{\bZ_L=X_L\phase{90^{\circ}}} جبکہ برق گیر متعاملیت کی رکاوٹ \عددی{\bZ_C=X_C\phase{-90^{\circ}}} ہوتی ہے۔چونکہ \عددی{\cos (\mp 90^{\circ})=0} ہوتا ہے لہٰذا غیر مزاحمتی رکاوٹ کی طاقت صفر ہو گی۔
\begin{align}
P_{\text{متعاملی}} =0
\end{align}
چونکہ خالص متعامل پرزوں کو صفر اوسط طاقت منتقل ہوتی ہے لہٰذا انہیں \اصطلاح{بے ضیاع پرزے}\فرہنگ{بے ضیاع پرزے}\حاشیہب{lossless components}\فرہنگ{lossless components} کہتے ہیں۔دور کا متعامل حصہ، دوری عرصے کے کچھ حصے میں  دور سے طاقت حاصل کرتے ہوئے  ذخیرہ کرتا ہے  جبکہ دوری عرصے کے کسی دوسرے حصے میں اسی طاقت کو دور کو واپس کرتا ہے۔

%==========================
\ابتدا{مثال}\شناخت{مثال_طاقت_مزاحمت_امالہ_الف}
شکل \حوالہ{مشق_طاقت_مزاحمت_امالہ_الف} میں رکاوٹ کی اوسط طاقت دریافت کریں۔
\begin{figure}
\centering
\begin{tikzpicture}
\draw(0,0) to [american voltage source,l={$50\phase{30^{\circ}}\,\si{\volt}$}]++(0,2*\y) to [short,i={$\hat{I}$}]++(\x,0) to [resistor,l={$\SI{3}{\ohm}$}]++(0,-\y) to [inductor,l={$j6 \,\si{\ohm}$}]++(0,-\y) to [short] (0,0);
\end{tikzpicture}
\caption{مثال \حوالہ{مثال_طاقت_مزاحمت_امالہ_الف} کا دور۔}
\label{مشق_طاقت_مزاحمت_امالہ_الف}
\end{figure}

حل:رو درج ذیل ہے۔
\begin{align*}
\hat{I}&=\frac{50\phase{30^{\circ}}}{3+j6}=\frac{50\phase{30^{\circ}}}{3+j6}=\frac{50\phase{30^{\circ}}}{\sqrt{45}\phase{63.435^{\circ}}}=7.454\phase{-33.435^{\circ}}\,\si{\ampere}
\end{align*}
یوں
\begin{align*}
P&=\frac{V_0 I_0}{2}\cos(\phi_v-\phi_i)\\
&=\frac{(50)(7.454)}{2}\cos[30^{\circ}-(-33.435^{\circ})]\\
&=\SI{83.34}{\watt}
\end{align*}
ہو گا۔چونکہ طاقت صرف مزاحمت میں ضائع ہوتی ہے لہٰذا یہی جواب مساوات \حوالہ{مساوات_طاقت_مزاحمتی_طاقت_الف} سے بھی حاصل کیا جا سکتا ہے جہاں \عددی{V_0} سے مراد مزاحمت کے دباو کا حیطہ ہے۔تقسیم دباو سے مزاحمت کا دباو درج ذیل ہے
\begin{align*}
\hat{V}_R&=\left(\frac{3}{3+j6}\right)50\phase{30^{\circ}}=22.361\phase{-33.435^{\circ}}
\end{align*}
جس سے مزاحمت کا اوسط طاقت درج ذیل ہو گا۔
\begin{align*}
P&=\frac{V_0 I_0}{2}=\frac{(22.361)(7.454)}{2}=\SI{83.34}{\watt}
\end{align*}
اسی طرح مساوات \حوالہ{مساوات_طاقت_مزاحمتی_طاقت_ب} اور مساوات \حوالہ{مساوات_طاقت_مزاحمتی_طاقت_پ} بھی استعمال کیے جا سکتے ہیں
\begin{align*}
P&=\frac{I^2_0 R}{2}=\frac{(7.454^2)(3)}{2}=\SI{83.34}{\watt}\\
P&=\frac{V^2_0}{2 R}=\frac{(22.361^2)}{(2)(3)}=\SI{83.34}{\watt}
\end{align*}
\انتہا{مثال}
%==========================
\ابتدا{مثال}\شناخت{مثال_طاقت_مزاحمت_امالہ_ب}
شکل \حوالہ{شکل_طاقت_مزاحمت_امالہ_ب} میں منبع دباو کا اوسط طاقت حاصل کریں۔دور کے بقایا پرزوں کا اوسط طاقت بھی دریافت کریں۔
\begin{figure}
\centering
\begin{tikzpicture}
\draw(0,0) to [american voltage source,i^<={$\hat{I}_m$},l={$10\phase{30^{\circ}}\,\si{\volt}$}]++(0,2*\y) to [short]++(4*\x,0) to [capacitor,i>_={$\hat{I}_C$},l={$-j10\,\si{\ohm}$}]++(0,-2*\y) to [short] (0,0);
\draw(\x,0) to [inductor,i_<={$\hat{I}_L$},*-*,l={$j5\,\si{\ohm}$}]++(0,2*\y);
\draw(2*\x,0) to [resistor,i_<={$\hat{I}_R$},*-*,l={$\SI{2}{\ohm}$}]++(0,2*\y);
\draw(3*\x,0) to [inductor,*-,l={$j2\,\si{\ohm}$}]++(0,\y) to [resistor,i_<={$\hat{I}_Z$},-*,l={$\SI{2}{\ohm}$}]++(0,\y);
\end{tikzpicture}
\caption{مثال \حوالہ{مثال_طاقت_مزاحمت_امالہ_ب} کا دور۔}
\label{شکل_طاقت_مزاحمت_امالہ_ب}
\end{figure} 

حل:پہلے تمام رو دریافت کرتے ہیں۔شکل میں دباو کو دیکھتے ہوئے انفعالی رائج رو کے تحت رو کی سمتیں چننی گئی ہیں۔ 
\begin{align*}
\hat{I}_L&=\frac{10\phase{30^{\circ}}}{j5}=\frac{10\phase{30^{\circ}}}{5\phase{90^{\circ}}}=2\phase{-60^{\circ}}\\
\hat{I}_R&=\frac{10\phase{30^{\circ}}}{2}=\frac{10\phase{30^{\circ}}}{2\phase{0^{\circ}}}=5\phase{30^{\circ}}\\
\hat{I}_Z&=\frac{10\phase{30^{\circ}}}{2+j2}=\frac{10\phase{30^{\circ}}}{\sqrt{8}\phase{45^{\circ}}}=\frac{5}{\sqrt{2}}\phase{-15^{\circ}}\\
\hat{I}_C&=\frac{10\phase{30^{\circ}}}{-j10}=\frac{10\phase{30^{\circ}}}{10\phase{-90^{\circ}}}=1\phase{120^{\circ}}\\
\hat{I}_m&=-\left[\hat{I}_L+\hat{I}_R+\hat{I}_Z+\hat{I}_C\right]=8.27647\phase{-175.01689^{\circ}}
\end{align*}
یوں انفرادی شاخوں کے اوسط طاقت مساوات \حوالہ{مساوات_طاقت_عمومی_الف} یا مساوات \حوالہ{مساوات_طاقت_عمومی_ب} سے درج ذیل ہوں گے۔
\begin{align*}
P_L&=\frac{(30)(2)}{2}\cos(90^{\circ})&=\SI{0}{\watt}\\
P_R&=\frac{(30)(5)}{2}\cos(0^{\circ})&=\SI{75}{\watt}\\
P_Z&=\frac{(30)(\tfrac{5}{\sqrt{2}})}{2}\cos(45^{\circ})&=\SI{37.5}{\watt}\\
P_C&=\frac{(30)(1)}{2}\cos(90^{\circ})&=\SI{0}{\watt}\\
P_m&=\frac{(30)(8.27647)}{2}\cos[(30^{\circ}+175.01689^{\circ})]&=-\SI{112.5}{\watt}
\end{align*}
مثبت جواب طاقت کا ضیاع ہے جبکہ منفی جواب طاقت کی پیداوار ہے۔آپ دیکھ سکتے ہیں کہ منبع کی طاقتی پیداوار \عددی{\SI{112.5}{\watt}} ہے جو دور میں طاقت کے ضیاع 
\begin{align*}
P_L+P_R+P_Z+P_C=0+75+37.5+0=\SI{112.5}{\watt}
\end{align*}
کے عین برابر ہے۔
\انتہا{مثال}
%==========================
\ابتدا{مشق}\شناخت{مشق_طاقت_دریافت_کریں_الف}
شکل \حوالہ{شکل_طاقت_دریافت_کریں_الف} کے تمام مزاحمتوں میں ضائع ہونے والا اوسط طاقت دریافت کریں۔


\begin{figure}
\centering
\begin{tikzpicture}
\draw(0,0) to [american voltage source,l={$22\phase{-30^{\circ}}\,\si{\volt}$}]++(0,2*\y) to [capacitor,l={$-j2\,\si{\ohm}$}]++(\x,0) to [resistor,l={$\SI{4}{\ohm}$}]++(\x,0) to [inductor,l={$j1\,\si{\ohm}$}]++(0,-\y) to [resistor,l={$\SI{5}{\ohm}$}]++(0,-\y) to [short] (0,0);
\draw(2*\x,2*\y) to [short,*-]++(\x,0) to [inductor,l={$j10\,\si{\ohm}$}]++(0,-2*\y) to [short,-*]++(-\x,0);
\end{tikzpicture}
\caption{مشق \حوالہ{مشق_طاقت_دریافت_کریں_الف} کا دور۔}
\label{شکل_طاقت_دریافت_کریں_الف}
\end{figure}

جوابات:\عددی{P_{\SI{4}{\ohm}}=\SI{17.491}{\watt}}، \عددی{P_{\SI{5}{\ohm}}=\SI{14.975}{\watt}}
\انتہا{مشق}
%============================

%==========================
\ابتدا{مشق}\شناخت{مشق_طاقت_دریافت_کریں_ب}
شکل \حوالہ{شکل_طاقت_دریافت_کریں_ب} کے تمام مزاحمتوں میں ضائع ہونے والا اوسط طاقت دریافت کریں۔
\begin{figure}
\centering
\begin{tikzpicture}
\draw(0,0) to [american current source,l={$10\phase{45^{\circ}}\,\si{\ampere}$}]++(0,\yy) to [capacitor,l={$-j2\,\si{\ohm}$}]++(\xx,0) to [resistor,l={$\SI{4}{\ohm}$}]++(0,-\yy) to [short] (0,0);
\draw(0,\yy) to [resistor,*-,l_={$\SI{2}{\ohm}$}]++(-\xx,0) to [inductor,l_={$j4\,\si{\ohm}$}]++(0,-\yy) to [short,-*] (0,0);
\end{tikzpicture}
\caption{مشق \حوالہ{مشق_طاقت_دریافت_کریں_ب} کا دور۔}
\label{شکل_طاقت_دریافت_کریں_ب}
\end{figure}

جوابات:\عددی{P_{\SI{2}{\ohm}}=\SI{50}{\watt}}، \عددی{P_{\SI{4}{\ohm}}=\SI{100}{\watt}}
\انتہا{مشق}
%====================
\ابتدا{مشق}\شناخت{مشق_طاقت_دریافت_کریں_پ}
شکل \حوالہ{شکل_طاقت_دریافت_کریں_پ} کے تمام مزاحمتوں میں ضائع ہونے والا اوسط طاقت دریافت کریں۔
\begin{figure}
\centering
\begin{tikzpicture}
\draw(0,0) to [american voltage source,l={$20\phase{30^{\circ}}\,\si{\volt}$}]++(0,\yy) to [european resistor,l={$2+j2\,\si{\ohm}$}]++(\xx,0) to [european resistor,l={$6+j2\,\si{\ohm}$}]++(0,-\yy) to [short] (0,0);
\draw(\xx,0) to [short,*-]++(\xx,0) to [american voltage source,l_={$40\phase{60^{\circ}}\,\si{\volt}$}]++(0,\yy) to [european resistor,l_={$3-j4\,\si{\ohm}$},-*]++(-\xx,0);
\end{tikzpicture}
\caption{مشق \حوالہ{مشق_طاقت_دریافت_کریں_پ} کا دور۔}
\label{شکل_طاقت_دریافت_کریں_پ}
\end{figure}

جوابات:\عددی{P_{\SI{2}{\ohm}}=\SI{22.72}{\watt}} ،\عددی{P_{\SI{3}{\ohm}}=\SI{5.71}{\watt}}، \عددی{P_{\SI{6}{\ohm}}=\SI{11.42}{\watt}}
\انتہا{مشق}
%==============

ایک سے زیادہ منبع کی صورت میں آپ کسی بھی ترکیب کو استعمال کرتے ہوئے شاخوں کی رو اور جوڑ کے دباو حاصل کرتے ہوئے طاقت دریافت کر سکتے ہیں۔البتہ یاد رہے کہ ترکیب نفاذ سے طاقت کا تخمینہ نہیں لگایا جا سکتا چونکہ طاقت مربع دباو (یا مربع رو) کا تعلق رکھتا ہے جو غیر خطی تعلق ہے۔ 

%===============
\ابتدا{مشق}\شناخت{مشق_طاقت_دریافت_کریں_ت}
شکل \حوالہ{شکل_طاقت_دریافت_کریں_ت} میں اوسط طاقت کی پیداوار اور ضیاع معلوم کریں۔ 
\begin{figure}
\centering
\begin{tikzpicture}
\draw(0,0) to [american voltage source,l={$20\phase{30^{\circ}}\,\si{\volt}$}]++(0,\y) to [resistor,l={$\SI{2}{\ohm}$}]++(\x,0) to [inductor,l={$j4\,\si{\ohm}$}]++(\x,0);
\draw(0,0) to [short]++(2*\x,0) to [american voltage source,l_={$40\phase{0^{\circ}}\,\si{\volt}$}]++(0,\y);
\end{tikzpicture}
\caption{مشق \حوالہ{مشق_طاقت_دریافت_کریں_ت} کا دور۔}
\label{شکل_طاقت_دریافت_کریں_ت}
\end{figure}

\عددی{P_{20\phase{30^{\circ}}}=\SI{-25.36}{\watt}}، \عددی{P_{40\phase{0^{\circ}}}=\SI{-5.36}{\watt}}، \عددی{P_{\SI{2}{\ohm}}=\SI{30.72}{\watt}}
\انتہا{مشق}
%================

\ابتدا{مشق}\شناخت{مشق_طاقت_دریافت_کریں_ٹ}
شکل \حوالہ{شکل_طاقت_دریافت_کریں_ٹ} میں اوسط طاقت کی پیداوار اور ضیاع معلوم کریں۔ 
\begin{figure}
\centering
\begin{tikzpicture}
\draw(0,0) to [american voltage source,l={$30\phase{0^{\circ}}$}]++(0,\y) to [short]++(\x,0) to [inductor,l={$j8\,\si{\ohm}$}]++(0,-\y) to [short] (0,0);
\draw(\x,0) to [short,*-] ++(\x,0) to [inductor,l_={$j6\,\si{\ohm}$}]++(0,\y) to [capacitor,-*,l_={$-j10\,\si{\ohm}$}]++(-\x,0);
\end{tikzpicture}
\caption{مشق \حوالہ{مشق_طاقت_دریافت_کریں_ٹ} کا دور۔}
\label{شکل_طاقت_دریافت_کریں_ٹ}
\end{figure}

جواب:اوسط طاقت کی پیدا وار اور طاقت کا ضیاع صفر واٹ ہیں۔
\انتہا{مشق}
%================

\حصہ{زیادہ سے زیادہ اوسط طاقت منتقل کرنے کا مسئلہ}
یک سمتی رو ادوار میں ہم زیادہ سے زیادہ طاقت منتقل کرنے کے مسئلے پر ہم حصہ \حوالہ{حصہ_مسئلے_زیادہ_سے_زیادہ_طاقت_منتقل} میں غور کر چکے ہیں۔آئیں بدلتی رو کی صورت میں اسی مسئلے پر دوبارہ غور کریں۔

کسی بھی دور کا تھونن مساوی حاصل کیا جا سکتا ہے۔شکل \حوالہ{شکل_طاقت_زیادہ_سے_زیادہ_الف} میں تھونن مساوی دور کے ساتھ بوجھ جوڑا گیا ہے جہاں تھونن دباو کو \عددی{\hat{V}_{\text{کھلا}}} کہا گیا ہے۔ہم جاننا چاہتے ہیں کہ بوجھ کو کس صورت میں زیادہ سے زیادہ اوسط طاقت منتقل ہو گا۔  
\begin{figure}
\centering
\begin{tikzpicture}[american voltages]
\draw(0,0) to [american voltage source,l={$\hat{V}_{\text{کھلا}}$}]++(0,\y) to [european resistor,l={$\bZ_{\text{تھونن}}$}]++(\x,0) to [short,i={$\hat{I}_{\text{بوجھ}}$}]++(\x,0) to [european resistor,v={$\hat{V}_{\text{بوجھ}}$},l={$\bZ_{\text{تھونن}}$}]++(0,-\y) to [short] (0,0);
\end{tikzpicture}
\caption{زیادہ سے زیادہ اوسط طاقت منتقل کرنے کا مسئلہ۔}
\label{شکل_طاقت_زیادہ_سے_زیادہ_الف}
\end{figure}

شکل کو دیکھ کر درج ذیل لکھا جا سکتا ہے
\begin{align}\label{مساوات_طاقت_زیادہ_سے_زیادہ_طاقت_الف}
\hat{I}_{\text{بوجھ}} &=\frac{\hat{V}_{\text{کھلا}}}{\bZ_{\text{تھونن}}+\bZ_{\text{بوجھ}}}
\end{align}
جہاں
\begin{align*}
\bZ_{\text{تھونن}}&=R_{\text{تھونن}}+jX_{\text{تھونن}}\\
\bZ_{\text{بوجھ}}&=R_{\text{بوجھ}}+jX_{\text{بوجھ}}\\
\hat{V}_{\text{کھلا}}&=V_{\text{کھلا}} \phase{\phi_{\text{کھلا}}}
\end{align*}
ہیں۔درج بالا میں امالی رکاوٹ کی صورت میں \عددی{X} کی قیمت مثبت ہو گی جبکہ برق گیر رکاوٹ کی صورت میں اس کی قیمت منفی ہو گی۔یوں مساوات \حوالہ{مساوات_طاقت_زیادہ_سے_زیادہ_طاقت_الف} کو درج ذیل لکھا جا سکتا ہے
\begin{align*}
\hat{I}_{\text{بوجھ}}&=\frac{V_{\text{کھلا}} \phase{\phi_{\text{کھلا}}}}{R_{\text{تھونن}}+jX_{\text{تھونن}}+R_{\text{بوجھ}}+jX_{\text{بوجھ}}}
\end{align*}
جس کی حتمی قیمت درج ذیل ہے۔
\begin{align*}
I_{\text{بوجھ}}&=\frac{V_{\text{کھلا}}}{\sqrt{(R_{\text{تھونن}}+R_{\text{بوجھ}})^2+(X_{\text{تھونن}}+X_{\text{بوجھ}})^2}}
\end{align*}

بوجھ کو منتقل اوسط طاقت مساوات \حوالہ{مساوات_طاقت_مزاحمتی_طاقت_ب} کی مدد سے لکھتے ہیں۔
\begin{gather}
\begin{aligned}\label{مساوات_طاقت_زیادہ_سے_زیادہ_مساوات_الف}
P_{\text{بوجھ}}&=\frac{1}{2}I^2_{\text{بوجھ}} R_{\text{بوجھ}}\\
&=\frac{\frac{1}{2}V^2_{\text{کھلا}}\, R_{\text{بوجھ}}}{(R_{\text{تھونن}}+R_{\text{بوجھ}})^2+(X_{\text{تھونن}}+X_{\text{بوجھ}})^2}
\end{aligned}
\end{gather} 
ہم جانتے ہیں کہ \عددی{X} میں طاقت ضائع نہیں ہوتا لہٰذا اس کو اوسطاً صفر طاقت منتقل ہوتا ہے۔درج بالا مساوات میں کسر کے نچلے حصے
 میں \عددی{X_{\text{بوجھ}}+X_{\text{تھونن}}}  کی قیمت کم سے کم کرتے ہوئے طاقت بڑھائی جا سکتی ہے۔درج ذیل صورت میں اس قیمت کو صفر بنایا جا سکتا ہے۔
\begin{align}\label{مساوات_طاقت_زیادہ_سے_زیادہ_مساوات_ب}
X_{\text{بوجھ}}=-X_{\text{تھونن}}  \quad \quad \text{\RL{بوجھ کو زیادہ سے زیادہ طاقت کی منتقلی کا پہلا شرط}}
\end{align}
مساوات \حوالہ{مساوات_طاقت_زیادہ_سے_زیادہ_مساوات_ب} کے شرط پر پورا اترتے ہوئے مساوات \حوالہ{مساوات_طاقت_زیادہ_سے_زیادہ_مساوات_الف} کو درج ذیل لکھا جا سکتا ہے۔
\begin{align}\label{مساوات_طاقت_زیادہ_سے_زیادہ_مساوات_پ}
P_{\text{بوجھ}}&=\frac{V^2_{\text{کھلا}}\, R_{\text{بوجھ}}}{2(R_{\text{تھونن}}+R_{\text{بوجھ}})^2}
\end{align}
آئیں جانتے ہیں کہ کس قیمت کے \عددی{R_{\text{بوجھ}}} کو زیادہ سے زیادہ طاقت  منتقل ہو گی۔یہ جاننے کے لئے درج بالا مساوات کے تفرق کو صفر کے برابر پُر کرتے ہوئے  \عددی{R_{\text{بوجھ}}} کی درکار قیمت حاصل کرتے ہیں۔
\begin{align*}
\frac{\dif P_{\text{بوجھ}}}{\dif R_{\text{بوجھ}}} = \frac{V^2_{\text{بوجھ}}\left(R_{\text{تھونن}}+R_{\text{بوجھ}}\right)^2-2V_{\text{بوجھ}}^2 R_{\text{بوجھ}} \left(R_{\text{تھونن}}+R_{\text{بوجھ}}\right)}{2\left(R_{\text{تھونن}}+R_{\text{بوجھ}}\right)^4} =0
\end{align*}
اس سے
\begin{align}\label{مساوات_طاقت_زیادہ_سے_زیادہ_مساوات_ت}
R_{\text{بوجھ}}=R_{\text{تھونن}} \quad \quad \text{\RL{بوجھ کو زیادہ سے زیادہ طاقت کی منتقلی کا دوسرا شرط}}
\end{align}
حاصل ہوتا ہے۔اس نتیجے کے تحت بوجھ کو اس صورت زیادہ سے زیادہ طاقت منتقل ہو گی جب بوجھ کی مزاحمت دور کے تھونن مزاحمت کے برابر ہو۔مساوات \حوالہ{مساوات_طاقت_زیادہ_سے_زیادہ_مساوات_ب} اور مساوات \حوالہ{مساوات_طاقت_زیادہ_سے_زیادہ_مساوات_ت} کو استعمال کرتے ہوئے، بوجھ کو زیادہ سے زیادہ طاقت منتقل ہونے کی شرط کو درج ذیل لکھا جا سکتا ہے۔
\begin{gather}
\begin{aligned}
R_{\text{بوجھ}}+jX_{\text{بوجھ}} &=R_{\text{تھونن}}-jX_{\text{تھونن}}\\
\bZ_{\text{بوجھ}}=\bZ^*_{\text{تھونن}}
\end{aligned}
\end{gather}
آخر میں یہ بھی بتلاتا چلوں کہ مزاحمتی بوجھ \عددی{(X_L=0)} کی صورت میں مساوات \حوالہ{مساوات_طاقت_زیادہ_سے_زیادہ_مساوات_الف} کے تفرق کو صفر
\begin{align*}
\frac{\dif P_{\text{بوجھ}}}{\dif R_{\text{بوجھ}}}=0
\end{align*}
 کے برابر پر کرنے سے درج ذیل ملتا ہے۔
\begin{align}
R_{\text{بوجھ}}=\sqrt{R^2_{\text{تھونن}}+X^2_{\text{تھونن}}}
\end{align}

%=======================
\ابتدا{مثال}\شناخت{مثال_طاقت_زیادہ_سے-زیادہ_مثال_الف}
شکل \حوالہ{شکل_طاقت_زیادہ_سے-زیادہ_مثال_الف} میں بوجھ کے رکاوٹ کی وہ قیمت دریافت کریں جس پر بوجھ کو زیادہ سے زیادہ طاقت منتقل ہو گا۔اس طاقت کی قیمت بھی دریافت کریں۔
\begin{figure}
\centering
\begin{subfigure}{1\textwidth}
\centering
\begin{tikzpicture}
\draw(0,0) to [american voltage source,l={$20\phase{0^{\circ}}\,\si{\volt}$}]++(0,\y) to [inductor,l={$j2\,\si{\ohm}$}]++(\x,0) to [resistor,l={$\SI{6}{\ohm}$}]++(0,-\y) to [short] (0,0);
\draw(\x,0) to [short,*-]++(\x,0) to [european resistor,l_={$\bZ_{\text{بوجھ}}$}]++(0,\y) to [capacitor,-*,l_={$-j4\,\si{\ohm}$}]++(-\x,0);
\end{tikzpicture}
\caption*{(الف)}
\end{subfigure}
\begin{subfigure}{0.5\textwidth}
\centering
\begin{tikzpicture}
\draw(0,0) to [short]++(0,\y) to [inductor,l={$j2\,\si{\ohm}$}]++(\x,0) to [resistor,l={$\SI{6}{\ohm}$}]++(0,-\y) to [short] (0,0);
\draw(\x,0) to [short,*-o]++(\x,0)++(0,\y) to [capacitor,o-*,l_={$-j4\,\si{\ohm}$}]++(-\x,0);
\draw[stealth-] (2*\x,\y/2)--++(\x/8,0)--++(0,-\y/8)node[below]{$\bZ_{\text{تھونن}}$};
\end{tikzpicture}
\caption*{(ب)}
\end{subfigure}%
\begin{subfigure}{0.5\textwidth}
\centering
\begin{tikzpicture}
\draw(0,0) to [american voltage source,l={$20\phase{0^{\circ}}\,\si{\volt}$}]++(0,\y) to [inductor,l={$j2\,\si{\ohm}$}]++(\x,0) to [resistor,l={$\SI{6}{\ohm}$}]++(0,-\y) to [short] (0,0);
\draw(\x,0) to [short,*-o]++(\x,0) ++(0,\y) to [capacitor,o-*,l_={$-j4\,\si{\ohm}$}]++(-\x,0);
\draw(2*\x,\y/2)node{$\begin{aligned} &+ \\ &\hat{V}_{\text{کھلا}} \\ &- \end{aligned}$};
\end{tikzpicture}
\caption*{(پ)}
\end{subfigure}
\begin{subfigure}{1\textwidth}
\centering
\begin{tikzpicture}
\draw(0,0) to [american voltage source,l={$18.97\phase{-18.43^{\circ}}\,\si{\volt}$}]++(0,\y) to [european resistor,l={${0.6-j2.2\,\si{\ohm}}$}]++(\x,0) to [european resistor,i>^={$\hat{I}_{\text{بوجھ}}$},l={${0.6+j2.2\,\si{\ohm}}$}]++(0,-\y) to [short] (0,0);
\end{tikzpicture}
\caption*{(ت)}
\end{subfigure}
\caption{مثال \حوالہ{مثال_طاقت_زیادہ_سے-زیادہ_مثال_الف} کا دور۔}
\label{شکل_طاقت_زیادہ_سے-زیادہ_مثال_الف}
\end{figure}

حل:سب سے پہلے بوجھ کو ہٹاتے ہوئے بقایا دور کا تھونن مساوی حاصل کرنا ہو گا۔شکل-ب میں منبع دباو کو قصر دور کیا گیا ہے تا کہ تھونن مزاحمت حاصل کی جا سکے۔اسی طرح شکل-پ میں کھلے دور دباو کی نشاندہی کی گئی ہے۔ شکل-ب تھونن رکاوٹ لکھتے ہیں۔
\begin{align*}
\bZ_{\text{تھونن}}&=-j4+\frac{(6)(j2)}{6+j2}=\frac{3}{5}-j\frac{11}{5} \, \si{\ohm}
\end{align*}
یوں بوجھ کو زیادہ سے زیادہ طاقت کی منتقلی کے لئے ضروری ہے کہ بوجھ کی رکاوٹ درج ذیل ہو۔
\begin{align*}
\bZ_{\text{بوجھ}}=\frac{3}{5}+j\frac{11}{5} \, \si{\ohm}
\end{align*} 
شکل-پ میں برق گیر میں صفر رو ہے لہٰذا اس پر دباو بھی صفر ہو گا۔اس طرح مزاحمت پر دباو ہی تھونن دباو ہے جسے تقسیم دباو کے کلیے سے لکھتے ہیں۔
\begin{align*}
\hat{V}_{\text{کھلا}}&=\left(\frac{6}{6+j2}\right) (20\phase{0^{\circ}})=18.97\phase{-18.43^{\circ}}\,\si{\volt}
\end{align*}
شکل-ت میں تھونن مساوی دور کو بوجھ کے ساتھ جوڑ کر دکھایا گیا ہے جہاں سے رو حاصل کرتے ہیں۔
\begin{align*}
\hat{I}_{\text{بوجھ}}&=\frac{18.97\phase{-18.43^{\circ}}}{\frac{3}{5}-j\frac{11}{5}+\frac{3}{5}+j\frac{11}{5}}\\
&=15.81\phase{-18.43^{\circ}}\,\si{\ampere}
\end{align*}
یوں بوجھ کو منتقل طاقت درج ذیل ہو گا۔
\begin{align*}
P_{\text{بوجھ}}=\frac{(15.81^2)(0.6)}{2}=\SI{74.99}{\watt}
\end{align*}

\انتہا{مثال}
%=================

\باب{مقناطیسی جڑے ادوار}

\حصہ{مشترکہ امالہ}
شکل \حوالہ{شکل_مقناطیسی_خود_امالہ} میں \عددی{N} چکر کا \اصطلاح{لچھا}\فرہنگ{لچھا}\حاشیہب{coil}\فرہنگ{coil} دکھایا گیا ہے جس میں \عددی{i} رو گزر رہی ہے۔رو کے گزرنے سے لچھے میں \عددی{\phi} \اصطلاح{مقناطیسی بہاو}\فرہنگ{مقناطیسی بہاو}\فرہنگ{بہاو!مقناطیسی}\حاشیہب{magnetic flux}\فرہنگ{magnetic flux}\فرہنگ{flux!magnetic} پیدا ہوتا ہے۔مقناطیسی بہاو \عددی{\phi} لچھے کے تمام چکروں کے اندر سے گزرنے کی صورت میں لچھے کا ارتباط بہاو \عددی{\lambda} درج ذیل ہے۔
\begin{align}\label{مساوات_مشترک_ارتباط_بہاو_الف}
\lambda=N \phi
\end{align}
اس کتاب میں صرف خطی نظام پر غور کیا گیا ہے۔خطی صورت میں ارتباط بہاو اور رو کا تعلق درج ذیل ہے
\begin{align}\label{مساوات_مشترک_ارتباط_بہاو_ب}
\lambda=L i
\end{align}
جہاں مساوات کے مستقل \عددی{L} کو \اصطلاح{خود امالہ}\فرہنگ{امالہ!خود}\فرہنگ{خود امالہ}\حاشیہب{self inductance}\فرہنگ{inductance!self}\فرہنگ{self inductance} یا \اصطلاح{امالہ} کہتے ہیں۔باب \حوالہ{باب_برق_گیر_امالہ_گیر} میں ہم امالہ پر غور کر چکے ہیں۔درج بالا دو مساوات کو ملاتے ہوئے  بہاو اور رو کا تعلق ملتا ہے۔
\begin{align}
\phi=\frac{Li}{N}
\end{align}
قانون فیراڈے کے تحت بدلتی ارتباط بہاو لچھے میں امالی دباو پیدا کرتا ہے۔
\begin{align}
v=\frac{\dif \lambda}{\dif t}
\end{align}
مساوات \حوالہ{مساوات_مشترک_ارتباط_بہاو_ب} کو درج بالا مساوات میں پر کرتے ہیں۔
\begin{align*}
v=\frac{\dif \lambda}{\dif t}=\frac{\dif (Li)}{\dif t}=L\frac{\dif i}{\dif t}+i\frac{\dif L}{\dif t}
\end{align*}
مستقل امالہ کی صوت میں اس مساوات سے امالہ کی جانی پہچانی درج ذیل مساوات حاصل ہوتی ہے۔
\begin{align}
v=L\frac{\dif i}{\dif t}
\end{align}
اس کتاب میں مستقل امالہ پر ہی غور کیا جائے گا۔
%
\begin{figure}
\centering
\begin{tikzpicture}[american voltages]
\pgfmathsetmacro{\lx}{1}
\pgfmathsetmacro{\ly}{0.2}
\pgfmathsetmacro{\yDiv}{1+\ly*sin(1980)}
%
\draw[dashed,gray] (\lx,\y/2) circle (1 cm and 1.5 cm);
\draw(\lx+1,\y/2)node[fill=white]{$\phi$};
\draw[domain=-180:5.5*360,samples=500,variable=\t,mark position=0(kBot)]  plot ({\lx*cos(\t)},{(\t/1980+\ly*sin(\t))/\yDiv*\y})coordinate(kTop);
\draw(kBot) to [short]++(-2*\x,0) to [american current source,l={$i$}]++(0,\y+0.15) to [short]++(2*\x,0) to [short] (kTop);
\draw(-\lx,\y/2)node[left]{\RL{$N$ چکر}};
\draw(-2*\x,\y) to [open,v={$v$}]++(0,-\y);
\end{tikzpicture}
\caption{خود امالہ کی تعریف۔}
\label{شکل_مقناطیسی_خود_امالہ}
\end{figure}

\باب{کثیر دوری ادوار}
\حصہ{تین دوری نظام}
اب تک بدلتی رو طاقت کی بات کرتے ہوئے  ایک عدد منبع دباو کی بات کی جاتی رہی۔حقیقت میں بدلتی رو طاقت کی پیدا وار اور ترسیل تین دوری نظام سے کی جاتی ہے۔شکل \حوالہ{شکل_تین_دوری_تین_دوری_نظام} میں تین دوری نظام دکھایا گیا ہے جہاں تین عدد منبع استعمال کئے گئے ہیں جو آپس میں \عددی{120^{\circ}} زاویائی فاصلہ رکھتے ہیں۔تمام دباو کے حیطے یک برابر ہونے کی صورت میں اس کو \اصطلاح{متوازن تین دوری نظام}\فرہنگ{متوازن!تین دوری نظام}\فرہنگ{تین دور!متوازن}\حاشیہب{balanced three phase system}\فرہنگ{three phase balanced system} کہا جاتا ہے۔دکھائے گئے متوازن نظام کے دباو درج ذیل ہیں جن کے دوری سمتیات کو شکل-ب میں دکھایا گیا ہے۔
\begin{gather}
\begin{aligned}
\hat{V}_{an}&=230 \phase{0^{\circ}}\,\si{\volt} \rms\\
\hat{V}_{bn}&=230 \phase{-120^{\circ}}\,\si{\volt} \rms\\
\hat{V}_{cn}&=230 \phase{-240^{\circ}}\,\si{\volt} \rms\\
&=230 \phase{120^{\circ}}\,\si{\volt} \rms
\end{aligned}
\end{gather}
انہیں کو وقتی دائرہ کار میں درج ذیل لکھا جائے گا۔شکل-پ میں انہیں دکھایا گیا ہے۔
\begin{gather}
\begin{aligned}\label{مساوات_تین_دوری_ستارہ_الف}
v_{an}(t)&=230\sqrt{2} \cos\omega t \,\si{\volt}\\
v_{bn}(t)&=230\sqrt{2} \cos(\omega t-120^{\circ})\,\si{\volt}\\
v_{cn}(t)&=230\sqrt{2} \cos(\omega t +120^{\circ})\,\si{\volt}
\end{aligned}
\end{gather}
متوازن بوجھ کی صورت میں تینوں رو کے حیطے اور زاوئے بھی برابر ہوں گے لہٰذا انہیں درج ذیل لکھا جائے گا۔
\begin{gather}
\begin{aligned}
i_{an}(t)&=I_0 \cos(\omega t -\theta)\,\si{\ampere}\\
i_{bn}(t)&=I_0 \cos(\omega t-120^{\circ}-\theta)\,\si{\ampere}\\
i_{cn}(t)&=I_0 \cos(\omega t +120^{\circ}-\theta)\,\si{\ampere}
\end{aligned}
\end{gather}
%
\begin{figure}
\centering
\begin{subfigure}{1\textwidth}
\centering
\begin{tikzpicture}
\draw(0,0)to [short]++(0,2*\y)  to [american voltage source,l_={${\hat{V}_{an}=230\phase{0^{\circ}}\,\si{\volt}\rms}$}]++(0,\y)  to [short,-o]++(4*\x,0)node[right]{$a$};
\draw(1*\x,0) to [short]++(0,\y) to [american voltage source,l_={${\hat{V}_{bn}=230\phase{-120^{\circ}}\,\si{\volt}\rms}$}]++(0,\y) to [short,-o]++(3*\x,0)node[right]{$b$};
\draw(2*\x,0) to [american voltage source,l_={${\hat{V}_{cn}=230\phase{120^{\circ}}\,\si{\volt}\rms}$}] ++(0,\y)to [short,-o]++(2*\x,0)node[right]{$c$};
\draw(0,0) to [short,-*]++(1*\x,0) to [short,-*]++(1*\x,0) to [short,-o]++(2*\x,0)node[right]{$n$};
\end{tikzpicture}
\caption*{(الف)}
\end{subfigure}
\begin{subfigure}{0.4\textwidth}
\centering
\begin{tikzpicture}
\pgfmathsetmacro{\len}{\x}
\draw[-latex](0,0)--++(0:\len)node[right]{$\hat{V}_{an}$};
\draw[-latex](0,0)--++(-120:\len)node[left]{$\hat{V}_{bn}$};
\draw[-latex](0,0)--++(120:\len)node[left]{$\hat{V}_{cn}$};
\draw[stealth-stealth]([shift={(0:0.3)}]0,0) arc (0:120:0.3);
\draw[stealth-stealth]([shift={(-120:0.3)}]0,0) arc (-120:0:0.3);
\draw[stealth-stealth]([shift={(120:0.3)}]0,0) arc (120:240:0.3);
\draw(60:0.8)node{$120^{\circ}$};
\draw(-60:0.8)node{$120^{\circ}$};
\draw(180:0.8)node{$120^{\circ}$};
\end{tikzpicture}
\caption*{(ب)}
\end{subfigure}%
\begin{subfigure}{0.6\textwidth}
\centering
\begin{tikzpicture}
\begin{axis}[kStyleCircuitsA,small,xlabel=$\omega t$, xtick={90,180,270,360},xticklabels={$90^{\circ}$,$180^{\circ}$,$270^{\circ}$,$360^{\circ}$},ytick={10},yticklabels={$230\sqrt{2}\,\si{\volt}$},]
\addplot[domain=0:370,samples=100]{10*cos(1*x+0)}node[pos=0,above right]{$v_{an}$};
\addplot[domain=0:370,samples=100]{10*cos(1*x-120)}node[pos=0.35,above right]{$v_{bn}$};
\addplot[domain=0:370,samples=100]{10*cos(1*x+120)}node[pos=0.65,above right]{$v_{cn}$};
\end{axis}%
\end{tikzpicture}
\caption*{(پ)}
\end{subfigure}%
\caption{تین دوری نظام۔}
\label{شکل_تین_دوری_تین_دوری_نظام}
\end{figure} 

تینوں دباو کو عمومی شکل میں لکھتے ہوئے
\begin{gather}
\begin{aligned}
v_{an}(t)&=V_0 \cos\omega t \,\si{\volt}\\
v_{bn}(t)&=V_0 \cos(\omega t-120^{\circ})\,\si{\volt}\\
v_{cn}(t)&=V_0 \cos(\omega t +120^{\circ})\,\si{\volt}
\end{aligned}
\end{gather}

آگے بڑھنے سے پہلے درج ذیل مثال میں ایک اہم مساوات ثابت کرتے ہیں۔
%=================
\ابتدا{مثال}\شناخت{مثال_تین_دوری_تکونی_صفر_برابر_ہے}
درج ذیل مساوات کو ثابت کریں۔
\begin{align}
\cos \alpha+\cos(\alpha+120^{\circ})+\cos(\alpha-120^{\circ})&=0\label{مساوات_تین_دوری_تکونی_صفر_برابر_ہے}\\
\cos \alpha+\cos(\alpha-240^{\circ})+\cos(\alpha+240^{\circ})&=0\label{مساوات_تین_دوری_تکونی_صفر_برابر_ب}
\end{align}

حل:مساوات \حوالہ{مساوات_تین_دوری_تکونی_صفر_برابر_ہے} میں دوسرے اور تیسرے اجزاء کو درج ذیل لکھا جا سکتا ہے۔
\begin{align*}
\cos(\alpha+120^{\circ})&=\cos \alpha \cos 120^{\circ}-\sin \alpha \sin 120^{\circ}=-\frac{1}{2}\cos \alpha-\frac{\sqrt{3}}{2}\sin\alpha\\
\cos(\alpha-120^{\circ})&=\cos \alpha \cos 120^{\circ}+\sin \alpha \sin 120^{\circ}=-\frac{1}{2}\cos \alpha+\frac{\sqrt{3}}{2}\sin\alpha
\end{align*}
یوں تینوں اجزاء کا مجموعہ درج ذیل ہے۔
\begin{align*}
(\cos \alpha)+(-\frac{1}{2}\cos \alpha-\frac{\sqrt{3}}{2}\sin\alpha)+(-\frac{1}{2}\cos \alpha+\frac{\sqrt{3}}{2}\sin\alpha)=0
\end{align*}
آئیں اب مساوات \حوالہ{مساوات_تین_دوری_تکونی_صفر_برابر_ب} کو ثابت کریں۔مساوات کے دوسرے جزو میں \عددی{\cos(\alpha-240^{\circ})=\cos(\alpha+120^{\circ})} استعمال کرتے ہوئے اور تیسرے جزو میں \عددی{\cos(\alpha+240^{\circ})=\cos(\alpha-120^{\circ})} استعمال کرتے ہوئے مساوات \حوالہ{مساوات_تین_دوری_تکونی_صفر_برابر_ہے} ملتا ہے جسے ہم ثابت کر چکے ہیں۔

\انتہا{مثال}
%=================
تین دوری نظام میں علیحدہ علیحدہ دور کے لمحاتی طاقت لکھتے ہیں 
\begin{align*}
p_a(t)&=v_{an}i_{an}\\
&=V_0 I_0 \cos \omega t \cos(\omega t -\theta)\\
&=\frac{V_0 I_0}{2}[\cos \theta +\cos(2\omega t -\theta)]\\
p_b(t)&=v_{bn}i_{bn}\\
&=V_0 I_0 \cos(\omega t -120^{\circ})\cos(\omega t-120^{\circ} -\theta)\\
&=\frac{V_0 I_0}{2}[\cos \theta +\cos(2\omega t -\theta-240^{\circ})]\\
p_c(t)&=v_{cn}i_{cn}\\
&=V_0 I_0 \cos (\omega t +120^{\circ})\cos(\omega t+120^{\circ} -\theta)\\
&=\frac{V_0 I_0}{2}[\cos \theta +\cos(2\omega t -\theta+240^{\circ})]
\end{align*}
جہاں \عددی{\cos \alpha \cos \beta=\tfrac{1}{2} [\cos(\alpha-\beta)+\cos(\alpha+\beta)]} کا استعمال کیا گیا ہے۔یوں مکمل نظام کا لمحاتی طاقت \عددی{p(t)} درج بالا کا مجموعہ ہو گا۔
\begin{align*}
p(t)&=p_a(t)+p_b(t)+p_c(t)\\
&=\frac{V_0 I_0}{2}[3\cos \theta +\cos(2\omega t-\theta)+\cos(2\omega t -\theta-240^{\circ})+\cos(2\omega t -\theta+240^{\circ})]
\end{align*}
درج بالا مساوات میں \عددی{2\omega t -\theta=\alpha} لکھتے ہوئے اور مساوات \حوالہ{مساوات_تین_دوری_تکونی_صفر_برابر_ب} استعمال کرتے ہوئے آخری تین اجزاء کے مجموعے کو صفر کے برابر لکھا جا سکتا ہے۔یوں لمحاتی طاقت درج ذیل حاصل ہوتی ہے۔
\begin{align}\label{مساوات_تین_دوری_لمحاتی_طاقت_برقرار}
p(t)=\frac{3V_0 I_0}{2}\cos \theta =3 \Vrms \Irms \cos \theta \,\si{\watt}
\end{align}
آپ مساوات \حوالہ{مساوات_تین_دوری_لمحاتی_طاقت_برقرار} کا \عددی{p_a(t)=\frac{V_0 I_0}{2}[\cos \theta +\cos(2\omega t -\theta)]} کے ساتھ موازنہ کریں جو دگنی تعدد یعنی \عددی{2\omega} کے ساتھ تبدیل ہوتا ہے۔آپ دیکھ سکتے ہیں کہ تین دوری نظام میں لمحاتی طاقت برقرار رہتا ہے۔یہ انتہائی اہم نتیجہ ہے۔تین دور کا موٹر برقرار میکانی قوت پیدا کرے گا لہٰذا اس میں ترتراہٹ کم سے کم ہو گی جو میکانی خرابی کی وجہ بنتی ہے۔

\حصہ{ستارہ اور تکونی جوڑ}
مساوات \حوالہ{مساوات_تین_دوری_ستارہ_الف} میں لمحہ \عددی{t=0} پر \عددی{v_{an}} کی چوٹی پائی جاتی ہے۔ہم کہتے ہیں کہ \عددی{v_{an}} کا زاویائی ہٹاو صفر کے برابر ہے۔اگر \عددی{v_{an}} کا زاویائی ہٹاو \عددی{\theta} ہو تب تین دوری نظام کے دوری سمتیات درج ذیل ہوں گے۔
 \begin{gather}
\begin{aligned}
\hat{V}_{an}&=230 \phase{\theta}\,\si{\volt} \rms\\
\hat{V}_{bn}&=230 \phase{\theta-120^{\circ}}\,\si{\volt} \rms\\
\hat{V}_{cn}&=230 \phase{\theta-240^{\circ}}\,\si{\volt} \rms
\end{aligned}
\end{gather}
ایسی صورت میں شکل \حوالہ{شکل_تین_دوری_تین_دوری_نظام}-ب کے تینوں دوری سمتیات \عددی{\theta} زاویہ گھوم جائیں گے۔تین دوری نظام کی بات کرتے ہوئے ہم \عددی{v_{an}} کا زاویہ ہٹاو صفر کے برابر لیں گے تا کہ بار بار اس کا ذکر نہ کرنا پڑے۔ساتھ ہی ساتھ جیسا شکل \حوالہ{شکل_تین_دوری_تین_دوری_نظام}-ب میں دکھایا گیا ہے ہم \عددی{v_{bn}} کو \عددی{v_{an}} سے \عددی{120^{\circ}} پیچے اور \عددی{v_{cn}} کو \عددی{v_{bn}} سے \عددی{120^{\circ}}  پیچے تصور کریں گے۔ایسے نظام کو \عددی{abc}\فرہنگ{abc} نظام کہا جاتا ہے۔
\begin{figure}
\centering
\begin{subfigure}{0.5\textwidth}
\centering
\begin{tikzpicture}
\draw(0,0)node[left]{$n$} to [american voltage source,*-o,l={$\hat{V}_{an}$}]++(0:\x)node[right]{$a$};
\draw(0,0) to [american voltage source,-o,l={$\hat{V}_{bn}$}]++(-120:\x)node[left]{$b$};
\draw(0,0) to [american voltage source,-o,l={$\hat{V}_{cn}$}]++(120:\x)node[left]{$c$};
\end{tikzpicture}
\caption*{(الف)}
\end{subfigure}%
\begin{subfigure}{0.5\textwidth}
\centering
\begin{tikzpicture}
\draw[-latex](0,0)--++(0:\x)node[right]{$\hat{V}_{an}$};
\draw[-latex](0,0)--++(-120:\x)node[left]{$\hat{V}_{bn}$};
\draw[-latex](0,0)--++(120:\x)node[left]{$\hat{V}_{cn}$};
\draw[stealth-stealth]([shift={(0:0.3)}]0,0) arc (0:120:0.3);
\draw[stealth-stealth]([shift={(-120:0.3)}]0,0) arc (-120:0:0.3);
\draw[stealth-stealth]([shift={(120:0.3)}]0,0) arc (120:240:0.3);
\draw(60:0.8)node{$120^{\circ}$};
\draw(-60:0.8)node{$120^{\circ}$};
\draw(180:0.8)node{$120^{\circ}$};
\end{tikzpicture}
\caption*{(ب)}
\end{subfigure}%
\caption{ستارہ نما جوڑ۔}
\label{شکل_تین_دوری_ستارہ_نظام}
\end{figure}


شکل \حوالہ{شکل_تین_دوری_تین_دوری_نظام}-الف کے تین دوری \عددی{abc} نظام کو شکل \حوالہ{شکل_تین_دوری_ستارہ_نظام}-الف میں \اصطلاح{ستارہ نما جڑا}\فرہنگ{ستارہ نما جوڑ}\حاشیہب{star connected, Y connected}\فرہنگ{star connected}\فرہنگ{Y connected} دکھایا گیا ہے۔ساتھ ہی شکل-ب میں دوری سمتیات دکھائے گئے ہیں جو پہلے سے ستارہ شکل بناتے ہیں۔تین دوری نظام کو اس طرح کاغذ پر بناتے ہوئے مکمل معلومات بغیر لکھے دی جاتی ہے۔یوں شکل \حوالہ{شکل_تین_دوری_تین_دوری_نظام}-الف سے ظاہر ہے کہ \عددی{v_{an}} کا زاویہ ہٹاو صفر کے برابر ہے اور \عددی{v_{bn}} اس سے \عددی{120^{\circ}} پیچے ہے۔یوں ظاہر ہے کہ یہ نظام \عددی{abc} ہے۔ساتھ ہی آپ دیکھ سکتے ہیں کہ تینوں دباو کے حیطے برابر ہیں۔تینوں دباو کو نقطہ \عددی{n} سے ناپا جاتا ہے۔

دوری سمتیات کا مجموعہ حاصل کرتے وقت ایک دوری سمتیہ کی نوک کے ساتھ دوسری دوری سمتیہ کی دم ملائی جاتی ہے۔اس ترکیب کو استعمال کرتے ہوئے شکل \حوالہ{شکل_تین_دوری_دباو_مجموعہ_صفر} میں درج ذیل مساوات ثابت کی گئی ہے۔
\begin{align}
\hat{V}_{an}+\hat{V}_{bn}+\hat{V}_{cn}=0
\end{align} 

\begin{figure}
\centering
\begin{tikzpicture}
\draw[-latex](0,0)--++(0:\x)coordinate(ka)node[above,pos=0.7]{$\hat{V}_{an}$};
\draw[-latex](ka)--++(-120:\x)coordinate(kb)node[right,pos=0.7]{$\hat{V}_{bn}$};
\draw[-latex](kb)--++(120:\x)node[left,pos=0.7]{$\hat{V}_{cn}$};
\draw(-\x/2,-\y/2)node[left]{$\hat{V}_{an}+\hat{V}_{bn}+\hat{V}_{cn}=0$};
\end{tikzpicture}
\caption{تین دوری نظام کے تینوں دباو کا مجموعہ صفر کے برابر ہے۔}
\label{شکل_تین_دوری_دباو_مجموعہ_صفر}
\end{figure}

\باب{تعددی ردعمل}
گزشتہ بابوں میں ہم \عددی{RLC} ادوار کو حل کر چکے ہیں جہاں تعدد غیر متغیر تھی۔اس باب میں تعدد تبدیل کرتے ہوئے ادوار کا ردعمل بالمقابل تعدد دیکھا جائے گا۔آئیں شروع میں سادہ ترین پرزوں کا تعددی رد عمل دیکھیں۔سادہ ترین پرزے  مزاحمت، امالہ اور برق گیر ہیں۔تعددی رد عمل دیکھتے ہوئے سائن نما اشارات زیر استعمال لائے جائیں گے۔ 

شکل \حوالہ{شکل_تعددی_مزاحمتی_ردعمل}-الف میں مزاحمت دکھایا گیا ہے۔مزاحت کی رکاوٹ درج ذیل ہے۔
\begin{align}
Z_R=R\phase{0^{\circ}}
\end{align}
یوں مزاحمت کی رکاوٹ پر تعدد \عددی{\omega} کا کوئی اثر نہیں پایا جاتا۔مزاحمت کے رکاوٹ کی حتمی قیمت \عددی{\abs{\bZ_R}} تمام تعدد پر \عددی{R} کے برابر ہے جبکہ اس کا زاویائی ہٹاو \عددی{\phase{\bZ_R}} تمام تعدد پر صفر درجے رہتا ہے۔یہ حقائق شکل \حوالہ{شکل_تعددی_مزاحمتی_ردعمل}-ب اور شکل \حوالہ{شکل_تعددی_مزاحمتی_ردعمل}-پ میں دکھائے گئے ہیں۔
\begin{figure}
\centering
\begin{subfigure}{1\textwidth}
\centering
\begin{tikzpicture}
\draw(0,0) to [short,o-]++(\x,0) to [resistor,l_={$R$}]++(0,\y) to [short,-o]++(-\x,0);
\draw[stealth-](\x/4,\y/2)--++(-\x/4,0)--++(0,-\y/8)node[below]{$\bZ_R$};
\end{tikzpicture}
\caption*{(الف) مزاحمت کی رکاوٹ۔}
\end{subfigure}
\begin{subfigure}{0.5\textwidth}
\centering
\begin{tikzpicture}
\begin{axis}[kStyleCircuitsA,name=ka,small,
,xlabel=$\omega$,ylabel=$\abs{\bZ_R}$,ytick={0.5},yticklabels={$R$},ymax=0.75,ymin=0,xmin=0,xtick=\empty]
\addplot[domain=0:3,samples=2]{0.5};
\end{axis}%
\end{tikzpicture}%
\caption*{(ب) مزاحمتی رکاوٹ کی حتمی قیمت بالمقابل تعدد۔}
\end{subfigure}%
\begin{subfigure}{0.5\textwidth}
\centering
\begin{tikzpicture}
\begin{axis}[kStyleCircuitsA,name=ka,small,
,xlabel=$\omega$,ylabel=$\phase{\bZ_R}$,ytick={0.25},yticklabels={$0^{\circ}$},ymax=0.75,ymin=0,xmin=0,xtick=\empty]
\addplot[domain=0:3,samples=2]{0.25};
\end{axis}
\end{tikzpicture}%
\caption*{(پ) مزاحمتی رکاوٹ کا زاویہ بالمقابل تعدد۔}
\end{subfigure}%
\caption{مزاحمتی رکاوٹ کا تعدد ردعمل۔}
\label{شکل_تعددی_مزاحمتی_ردعمل}
\end{figure}

امالہ گیر کو شکل \حوالہ{شکل_تعددی_امالی_ردعمل}-الف میں دکھایا گیا ہے۔امالہ گیر کی رکاوٹ درج ذیل ہے۔
\begin{align}
\bZ_L&=j\omega L=\omega L\phase{90^{\circ}}
\end{align}
اس طرح امالہ گیر کے رکاوٹ کی حتمی قیمت تعدد بڑھانے سے بڑھتی ہے۔رکاوٹ کی مقدار کا تعدد کے ساتھ راست تنابی رشتہ ہے۔
\begin{align}
\abs{\bZ_L}=\omega L
\end{align}
صفر تعدد پر امالہ گیر کی رکاوٹ \عددی{\SI{0}{\ohm}} ہو جاتی ہے اور یہ قصر دور خاصیت رکھتا ہے جبکہ لامتناہی تعدد پر رکاوٹ کی مقدار لامتناہی ہو جاتی ہے اور امالہ گیر بطور کھلا دور عمل کرتا ہے۔امالی رکاوٹ کا زاویہ تمام تعدد پر \عددی{90^{\circ}} رہتا ہے۔
\begin{align}
\phase{\bZ_L}=90^{\circ}
\end{align}
شکل \حوالہ{شکل_تعددی_امالی_ردعمل}-ب اور شکل \حوالہ{شکل_تعددی_امالی_ردعمل}-پ میں ان حقائق کو دکھایا گیا ہے۔
\begin{figure}
\centering
\begin{subfigure}{1\textwidth}
\centering
\begin{tikzpicture}
\draw(0,0) to [short,o-]++(\x,0) to [inductor,l_={$L$}]++(0,\y) to [short,-o]++(-\x,0);
\draw[stealth-](\x/4,\y/2)--++(-\x/4,0)--++(0,-\y/8)node[below]{$\bZ_L$};
\end{tikzpicture}
\caption*{(الف) امالہ گیر کی رکاوٹ۔}
\end{subfigure}
\begin{subfigure}{0.5\textwidth}
\centering
\begin{tikzpicture}
\begin{axis}[kStyleCircuitsA,name=ka,small,
,xlabel=$\omega$,ylabel=$\abs{\bZ_L}$,ytick=\empty,ymin=0,xmin=0,xtick=\empty]
\addplot[domain=0:10,samples=10]{0.5*x};
\end{axis}%
\end{tikzpicture}%
\caption*{(ب) امالی رکاوٹ کی حتمی قیمت بالمقابل تعدد۔}
\end{subfigure}%
\begin{subfigure}{0.5\textwidth}
\centering
\begin{tikzpicture}
\begin{axis}[kStyleCircuitsA,name=ka,small,
,xlabel=$\omega$,ylabel=$\phase{\bZ_L}$,ytick={0.5},yticklabels={$90^{\circ}$},ymax=0.75,ymin=0,xmin=0,xtick=\empty]
\addplot[domain=0:3,samples=2]{0.5};
\end{axis}
\end{tikzpicture}%
\caption*{(پ) امالی رکاوٹ کا زاویہ بالمقابل تعدد۔}
\end{subfigure}%
\caption{امالی رکاوٹ کا تعدد ردعمل۔}
\label{شکل_تعددی_امالی_ردعمل}
\end{figure}

برق گیر کو شکل \حوالہ{شکل_تعددی_برق_گیری_ردعمل}-الف میں دکھایا گیا ہے۔برق گیر کی رکاوٹ درج ذیل ہے۔
\begin{align}
\bZ_C=\frac{1}{j\omega C}=\frac{1}{\omega C}\phase{-90^{\circ}}
\end{align}
اس طرح برق گیر کے رکاوٹ کی مقدار کا تعدد کے ساتھ بالعکس متناسب کا رشتہ ہے جبکہ اس کا زاویہ تمام تعدد پر \عددی{-90^{\circ}} رہتا ہے۔
\begin{align}
\abs{\bZ_C}&=\frac{1}{\omega C}\\
\phase{bZ_C}&=-90^{\circ}
\end{align}
ان تعلقات کو شکل \حوالہ{شکل_تعددی_برق_گیری_ردعمل}-ب اور شکل \حوالہ{شکل_تعددی_برق_گیری_ردعمل}-پ میں دکھایا گیا ہے۔ صفر تعدد پر برق گیر کی رکاوٹ لامتناہی ہو جاتی ہے لہٰذا یہ بطور کھلا دور عمل کرتا ہے جبکہ لامتناہی تعدد پر رکاوٹ کی مقدار صفر ہو جاتی ہے اور یہ قصر دور کردار ادا کرتا ہے۔
\begin{figure}
\centering
\begin{subfigure}{1\textwidth}
\centering
\begin{tikzpicture}
\draw(0,0) to [short,o-]++(\x,0) to [capacitor,l_={$C$}]++(0,\y) to [short,-o]++(-\x,0);
\draw[stealth-](\x/4,\y/2)--++(-\x/4,0)--++(0,-\y/8)node[below]{$\bZ_C$};
\end{tikzpicture}
\caption*{(الف) برق گیر کی رکاوٹ۔}
\end{subfigure}
\begin{subfigure}{0.5\textwidth}
\centering
\begin{tikzpicture}
\begin{axis}[kStyleCircuitsA,name=ka,small,
,xlabel=$\omega$,ylabel=$\abs{\bZ_C}$,ytick=\empty,ymin=0,xmin=0,xtick=\empty]
\addplot[domain=10:60,samples=100]{1/(0.5*x)};
\end{axis}%
\end{tikzpicture}%
\caption*{(ب) برق گیر رکاوٹ کی حتمی قیمت بالمقابل تعدد۔}
\end{subfigure}%
\begin{subfigure}{0.5\textwidth}
\centering
\begin{tikzpicture}
\begin{axis}[kStyleCircuitsA,name=ka,small,
,xlabel=$\omega$,ylabel=$\phase{\bZ_C}$,ytick={0.5},yticklabels={$-90^{\circ}$},ymax=0.75,xmin=0,xtick=\empty]
\addplot[domain=0:3,samples=2]{0.5};
\end{axis}
\end{tikzpicture}%
\caption*{(پ) برق گیر رکاوٹ کا زاویہ بالمقابل تعدد۔}
\end{subfigure}%
\caption{برق گیر رکاوٹ کا تعدد ردعمل۔}
\label{شکل_تعددی_برق_گیری_ردعمل}
\end{figure}

سادہ ترین پرزوں کو نپٹانے کے بعد ذرہ مشکل ادوار دیکھتے ہیں۔شکل میں مزاحمت، امالہ گیر اور برق گیر سلسلہ وار جڑے دکھائے گئے ہیں۔ان کی کل رکاوٹ \عددی{\bZ_s} لکھتے ہیں
\begin{align*}
\bZ_s&=\bZ_R+\bZ_L+\bZ_C\\
&=R+j\omega L+\frac{1}{j\omega C}\\
&=R+j\left(\omega L-\frac{1}{\omega C}\right)
\end{align*}
اس تفاعل کو شکل \حوالہ{شکل_تعددی_مزاحمت_املہ_برق_گیر_سلسلہ_وار_ردعمل}-ب اور شکل \حوالہ{شکل_تعددی_مزاحمت_املہ_برق_گیر_سلسلہ_وار_ردعمل}-پ میں دکھایا گیا ہے۔
%
\begin{figure}
\centering
\begin{subfigure}{1\textwidth}
\centering
\begin{tikzpicture}
\draw(0,0) to [resistor,o-,l={$R$}]++(\x,0) to [inductor,l={$L$}]++(\x,0) to  [capacitor,l={$C$}]++(0,-\y) to [short,-o]++(-2*\x,0);
\draw[stealth-](\x/4,-\y/2)--++(-\x/4,0)--++(0,-\y/8)node[below]{$\bZ_s$};
\end{tikzpicture}
\caption*{(الف) سلسلہ وار دور۔}
\end{subfigure}
\begin{subfigure}{0.5\textwidth}
\centering
\begin{tikzpicture}
\begin{axis}[kStyleCircuitsA,name=ka,small,
,xlabel=$\omega$,ylabel=$\abs{\bZ_C}$,ytick=\empty,ymin=0,xmin=0,xtick={1},xticklabels={$\omega_0$}]
\addplot[domain=0:5,samples=100]{sqrt((1-x^2)^2+x^2)/x};
\end{axis}%
\end{tikzpicture}%
\caption*{(ب) مقدار بالمقابل تعدد۔}
\end{subfigure}%
\begin{subfigure}{0.5\textwidth}
\centering
\begin{tikzpicture}
\begin{axis}[kStyleCircuitsA,name=ka,small,
,xlabel=$\omega$,ylabel=$\phase{\bZ_C}$,ytick={-90,0,90},yticklabels={$-90^{\circ}$,$0^{\circ}$,$90^{\circ}$},ymax=120,xmin=0,xtick={1},xticklabels={$\omega_0$}]
\addplot[domain=0.01:3,samples=100]{atan(x-1/x)};
\end{axis}
\end{tikzpicture}%
\caption*{(پ) زاویہ بالمقابل تعدد۔}
\end{subfigure}%
\caption{سلسلہ وار جڑے مزاحمت، امالہ گیر اور برق گیر کا تعدد ردعمل۔}
\label{شکل_تعددی_مزاحمت_املہ_برق_گیر_سلسلہ_وار_ردعمل}
\end{figure}
%======================
\ابتدا{مثال}\شناخت{مثال_تعددی_بوڈا_خط_مزاحمت_امالہ_برق_گیر_رو_الف}
شکل \حوالہ{شکل_تعددی_بوڈا_خط_مزاحمت_امالہ_برق_گیر_رو_الف}-الف میں مزاحمت پر دباو حاصل کریں۔اس کے مقدار بالمقابل تعدد اور زاویہ بالمقابل تعدد کے خط کھینچیں۔
\begin{figure}
\centering
\begin{subfigure}{1\textwidth}
\centering
\begin{tikzpicture}[american voltages]
\draw(0,0) to [capacitor,l={$\SI{4}{\milli\farad}$}]++(\x,0) to [inductor,l={$\SI{0.15}{\henry}$}]++(\x,0) to  [resistor,l={$\SI{10}{\ohm}$},v={$\hat{V}_R$}]++(0,-\y) to [short]++(-2*\x,0) to [american voltage source,l={$20\phase{0^{\circ}}\,\si{\volt}$}]++(0,\y);
\end{tikzpicture}
\caption*{(الف)}
\end{subfigure}
\begin{subfigure}{1\textwidth}
\centering
\includegraphics{figFreqMagRLCvoltA}
\caption*{(ب) مقدار بالمقابل تعدد کا خط۔}
\end{subfigure}
\begin{subfigure}{1\textwidth}
\centering
\includegraphics{figFreqPhaseRLCvoltA}
\caption*{(پ) زاویہ بالمقابل تعدد کا خط۔}
\end{subfigure}
\caption{مثال \حوالہ{مثال_تعددی_بوڈا_خط_مزاحمت_امالہ_برق_گیر_رو_الف} کا دور۔}
\label{شکل_تعددی_بوڈا_خط_مزاحمت_امالہ_برق_گیر_رو_الف}
\end{figure}

حل:دور سے مزاحمت کا دباو درج ذیل لکھا جا سکتا ہے
\begin{align*}
\hat{V}_R=\frac{(4)(20\phase{0^{\circ}})}{4+j(2\pi f 0.15-\frac{1}{2\pi f 0.004})}
\end{align*}
جو مخلوط تفاعل ہے۔اس کی حتمی مقدار \عددی{\hat{V}_R} بالمقابل تعدد \عددی{f} کو شکل-ب میں دکھایا گیا ہے۔اس ترسیم میں دونوں محور کی پیمائش \اصطلاح{لاگ}\فرہنگ{لاگ}\حاشیہب{log}\فرہنگ{log} میں ہے۔اس طرز کے ترسیم کو \اصطلاح{لاگ لاگ}\فرہنگ{لاگ لاگ!ترسیم}\فرہنگ{ترسیم!لاگ لاگ}\حاشیہب{log-log}\فرہنگ{log-log!graph}\فرہنگ{graph!log-log} ترسیم کہا جاتا  ہے۔مقدار بالمقابل تعدد کے خط عموماً لاگ لاگ محور پر دکھائے جاتے ہیں۔ زاویہ دباو \عددی{\phase{\hat{V}_R}} بالمقابل تعدد کو شکل-پ میں \اصطلاح{نیم لاگ}\فرہنگ{نیم لاگ!ترسیم}\فرہنگ{ترسیم!نیم لاگ}\حاشیہب{semilog}\فرہنگ{semilog!axis}\فرہنگ{axis!semilog} محور پر دکھایا گیا ہے۔کم تعدد پر دباو کا زاویہ \عددی{+90^{\circ}} جبکہ بلند تعدد پر زاویہ \عددی{-90^{\circ}} ہے۔
\انتہا{مثال}
%========================

یہاں لاگ لاگ اور نیم لاگ محور پر قیمتیں پڑھنا سیکھ لیں چونکہ اس باب میں انہیں کا استعمال ہو گا۔یوں شکل \حوالہ{شکل_تعددی_بوڈا_خط_مزاحمت_امالہ_برق_گیر_رو_الف}-ب میں  حتمی مقدار کی چوٹی \عددی{10^1} یعنی دس ہرٹز پر پائی جاتی ہے۔یہ چوٹی \عددی{10^1} یعنی دس وولٹ کو ظاہر کرتی ہے۔اسی طرح \عددی{\SI{e2}{\hertz}} یعنی سو ہرٹز پر دباو تقریباً  \عددی{\SI{1.6}{\volt}} ہے۔

\اصطلاح{سمعی}\فرہنگ{سمعی}\حاشیہب{audio}\فرہنگ{audio} اشارات کو \اصطلاح{عددی صورت}\فرہنگ{عددی صورت}\حاشیہب{digital form}\فرہنگ{digital form} میں تبدیل کرتے ہوئے کمپیوٹر میں ذخیرہ کیا جاتا ہے۔ انہیں کو دوبارہ \اصطلاح{مماثل صورت}\فرہنگ{مماثل صورت}\حاشیہب{analog form}\فرہنگ{analog form} میں تبدیل کرتے ہوئے سنا جا سکتا ہے۔آئیں ان اشارات پر ایک مثال دیکھیں۔

کمپیوٹر سے حاصل موسیقی کے مماثلی اشارات کی چوٹی \عددی{\SI{1.5}{\volt}} ہے۔ ہم چاہتے ہیں کہ \اصطلاح{سمعی دباو ایمپلیفائر}\فرہنگ{ایمپلیفائر!سمعی}\فرہنگ{سمعی ایمپلیفائر}\حاشیہب{voltage amplifier}\فرہنگ{voltage!amplifier} استعمال کرتے ہوئے \عددی{\SI{8}{\ohm}} کے \اصطلاح{سپیکر}\فرہنگ{سپیکر}\حاشیہب{loud speaker}\فرہنگ{speaker} کو \عددی{\SI{10}{\watt}} طاقت فراہم کی جائے۔ان حقائق سے ایمپلیفائر کے داخلی مماثل اشارہ کی موثر قیمت حاصل کرتے ہیں۔
\begin{align*}
v_m=\frac{1.5}{\sqrt{2}}=\SI{1.061}{\volt}\,\rms
\end{align*}
طاقت کے کلیے \عددی{P=\tfrac{\VrmsS}{R}} سے آٹھ اوہم کے سپیکر کو دس واٹ طاقت کے لئے درکار موثر دباو حاصل کرتے ہیں۔
\begin{align*}
v_0=\sqrt{(10)(8)}=\SI{8.944}{\volt}\,\rms
\end{align*} 
یوں ایمپلیفائر کی درکار افزائش دباو درج ذیل ہے۔
\begin{align*}
A_v=\frac{v_0}{v_m}=\frac{8.944}{1.061}=\SI{8.43}{\volt\per\volt}
\end{align*}
شکل \حوالہ{شکل_تعددی_افزائش_بالمقابل_تعددی_خط}-الف میں  ایمپلیفائر اور سپیکر دکھائے گئے ہیں جہاں \عددی{v_m} کمپیوٹر سے حاصل مماثل سمعی اشارہ ہے اور \عددی{A_v=\SI{10.53}{\volt\per\volt} کے برابر ہے}۔
\begin{figure}
\centering
\begin{subfigure}{1\textwidth}
\centering
\begin{tikzpicture}[american voltages]
\draw(0,0) to [american voltage source,l={$v_m$}]++(0,\y)  to [resistor,l={$\substack{ \displaystyle  R_m \hfill \\ \displaystyle \SI{50}{\ohm}}$}]++(\x,0) to [short]++(\x+\x/4,0) to [resistor,l_={$\substack{\displaystyle R_i \hfill \\ \displaystyle \SI{1}{\mega\ohm}}$},v^<={$v_i$}]++(0,-\y) to [short]++(-2*\x-\x/4,0);
\draw (\x+\x/4,0)node[ground]{} to [capacitor,*-*,l={$\substack{\displaystyle C_i \hfill\\ \displaystyle \SI{159.2}{\nano\farad}}$}]++(0,\y);
\draw(2*\x+\x/4,0) to [short,*-*]++(3/4*\x,0) to [american controlled voltage source,l_={$\substack{\displaystyle A_v v_i \\ \displaystyle 10.53 v_i}$}]++(0,\y) to [resistor,l={$\substack{\displaystyle R_o \hfill\\ \displaystyle \SI{2}{\ohm}}$}]++(\x,0) to [capacitor,l={$\substack{\displaystyle  C \hfill \\ \displaystyle \SI{796}{\micro\farad}}$}]++(\x,0) to [resistor,l={$\substack{\displaystyle  R_s \hfill\\ \displaystyle \SI{8}{\ohm}}$},v_<={$v_0$}]++(0,-\y)coordinate(kSp) to [short]++(-2*\x,0);
\draw(kSp)++(0.5,1/5*\y)node[]{سپیکر};
\end{tikzpicture}
\caption*{(الف)}
\end{subfigure}
\begin{subfigure}{1\textwidth}
\centering
\begin{tikzpicture}[american voltages]
\draw(0,0) to [american voltage source,l={$v_m$}]++(0,\y)  to [resistor,l={$R_m $}]++(\x,0) to [short]++(\x+\x/4,0) to [resistor,l_={$R_i$},v^<={$v_i$}]++(0,-\y) to [short]++(-2*\x-\x/4,0);
\draw (\x+\x/4,0)node[ground]{} to [capacitor,*-*,l={$\frac{1}{s C_i}$}]++(0,\y);
\draw(2*\x+\x/4,0) to [short,*-*]++(3/4*\x,0) to [american controlled voltage source,l_={$A_v v_i$}]++(0,\y) to [resistor,l={$R_o$}]++(\x,0) to [capacitor,l={$\frac{1}{s C}$}]++(\x,0) to [resistor,l={$R_s$},v_<={$v_0$}]++(0,-\y)coordinate(kSp) to [short]++(-2*\x,0);
\draw(kSp)++(0.5,1/5*\y)node[]{سپیکر};
\end{tikzpicture}
\caption*{(ب)}
\end{subfigure}
\begin{subfigure}{1\textwidth}
\centering
\begin{tikzpicture}
%axis
\draw(0,0)--++(5.5,0)node[below]{$f$};
\draw(0,0)--++(0,2.5)node[left]{$A_v$};
%gain
\draw(0.5,0.3)--++(0.5,1.5)--++(3,0)--++(0.5,-1.5);
\draw[dashed](1,1.8)--(1,0)node[below]{$\SI{20}{\hertz}$};
\draw[dashed](4,1.8)--(4,0)node[below]{$\SI{20}{\kilo\hertz}$};
\draw(0,1.8)--++(-0.1,0)node[left]{$\SI{8.43}{\volt\per\volt}$};
\end{tikzpicture}
\caption*{(پ)}
\end{subfigure}
\begin{subfigure}{1\textwidth}
\centering
\includegraphics{figFreqAudioAmplifier}
\caption*{(ت) ایمپلیفائر کی افزائش بالمقابل تعددی خط۔}
\end{subfigure}
\caption{ایمپلیفائر اور اس کی افزائش بالمقابل تعددی خط۔}
\label{شکل_تعددی_افزائش_بالمقابل_تعددی_خط}
\end{figure}
انسان \عددی{\SI{20}{\hertz}} تا \عددی{\SI{20}{\kilo\hertz}} کے سمعی اشارات سن سکتا ہے لہٰذا ہمارے ایمپلیفائر کو اس \اصطلاح{تعددی پٹی}\فرہنگ{تعددی پٹی}\حاشیہب{frequency band}\فرہنگ{frequency!band} کے اشارات کا حیطہ بڑھانا ہو گا۔حیطہ بڑھاتے ہوئے اصل آواز کی خاصیت تبدیل نہیں ہونی چاہیے۔اگر پوری تعددی پٹی پر ایمپلیفائر کی افزائش  کی قیمت یکساں ہو تب آواز کی خاصیت برقرار رہے گی۔یوں ہم چاہیں گے  \عددی{\SI{20}{\hertz}} تا \عددی{\SI{20}{\kilo\hertz}} پر ایمپلیفائر کی افزائش  \عددی{\SI{8.43}{\volt\per\volt}} رہے۔ایمپلیفائر کے افزائش بالمقابل تعددی خط  کو شکل-پ میں دکھایا گیا ہے۔

برق گیر کی رکاوٹ \عددی{\bZ_C=\tfrac{1}{j\omega C}} لکھی جاتی ہے جس میں \عددی{j\omega=s} پر کرتے ہوئے \عددی{\bZ_C=\tfrac{1}{sC}} لکھا جا سکتا ہے۔ایسا ہی کرتے ہوئے ایمپلیفائر کو دوبارہ شکل-پ میں دکھایا گیا ہے۔آپ میں سے کچھ طلبہ \عددی{s} کو پہچان گئے ہوں گے۔یہ \اصطلاح{لاپلاس بدل}\فرہنگ{لاپلاس بدل}\حاشیہب{Laplace transform}\فرہنگ{Laplace transform} کا متغیرہ ہے۔

آئیں شکل-ب کو حل کریں۔داخلی جانب بالائی جوڑ پر کرخوف مساوات رو لکھتے ہیں
\begin{align*}
\frac{v_i-v_m}{R_m}+sC_i v_i+\frac{v_i}{R_i}=0
\end{align*}
جس کو درج ذیل لکھا جا سکتا ہے۔
\begin{align*}
v_i\left(\frac{1}{R_m}+s C_i+\frac{1}{R_i}\right)=\frac{v_m}{R_m}
\end{align*}
اس میں قوسین کے اندر مزاحمتوں کو قریب قریب لکھتے ہوئے  \عددی{v_i} کے لئے حل کرتے ہیں۔
\begin{align*}
v_i&=\frac{v_m}{R_m\left(\frac{1}{R_m}+\frac{1}{R_i}+s C_i\right)}
\end{align*}
شکل \حوالہ{شکل_تعددی_افزائش_بالمقابل_تعددی_خط}-ب کے دائیں جانب تقسیم دباو کے کلیے سے \عددی{v_0} لکھتے ہیں۔
\begin{align*}
v_0=\frac{A_v v_i R_s}{R_o+R_s+\frac{1}{s C}}
\end{align*}
اس میں \عددی{v_i} کی قیمت پر کرتے ہیں
\begin{align*}
v_0&=\left(\frac{A_v R_s}{R_o+R_s+\frac{1}{sC}}\right)\frac{v_m}{R_m\left(\frac{1}{R_m}+\frac{1}{R_i}+s C_i\right)}\\
&=\left[\frac{sC R_s A_v}{1+sC(R_o+R_s)}\right]\frac{v_m}{R_m\left(\frac{1}{R_m}+\frac{1}{R_i}\right)\left(1+\frac{s C_i}{\frac{1}{R_m}+\frac{1}{R_i}}\right)}\\
&=\frac{R_s A_v v_m}{R_m\left(\frac{1}{R_m}+\frac{1}{R_i}\right)}\left[\frac{sC}{1+sC(R_o+R_s)}\right]\frac{1}{\left(1+\frac{s C_i}{\frac{1}{R_m}+\frac{1}{R_i}}\right)}
\end{align*}
جہاں دوسری قدم پر دائیں نچلی قوسین سے \عددی{\tfrac{1}{R_m}+\tfrac{1}{R_i}} باہر نکالا گیا اور تیسری قدم پر اسی کو پہلی قوسین کا حصہ بنایا گیا۔اس مساوات میں
\begin{align*}
\omega_{p1}&=\frac{1}{C(R_o+R_s)}\\
\omega_{p2}&=\frac{1}{C_i}\left(\frac{1}{R_m}+\frac{1}{R_i}\right)
\end{align*}
لکھتے ہوئے درج ذیل صاف ستھرا مساوات حاصل ہوتا ہے جہاں \عددی{\omega_{p1}} اور \عددی{\omega_{p2}} مساوات کے  \اصطلاح{قطب}\فرہنگ{قطب}\حاشیہب{pole}\فرہنگ{pole}  کہلاتے ہیں اور انہیں تعدد کی اکائی یعنی ہرٹز \عددی{\si{\hertz}} یا ریڈیئن فی سیکنڈ \عددی{\si{\radian\per\second}} میں ناپا جاتا ہے۔
\begin{align}
\bA_v(s)=\frac{v_0}{v_m}=\dfrac{R_s A_v }{R_m\left(\dfrac{1}{R_m}+\frac{1}{R_i}\right)}\dfrac{sC}{\left(1+\dfrac{s}{\omega_{p1}}\right)\left(1+\dfrac{s}{\omega_{p2}}\right)}
\end{align}
شکل-الف میں دی گئی مزاحمتوں اور برق گیروں کی قیمتیں استعمال کرتے ہوئے درج ذیل ملتا ہے۔
\begin{align*}
\omega_{p1}&=\frac{1}{796\times 10^{-6}(2+8)}=\SI{125.63}{\radian\per\second}\\
\omega_{p2}&=\frac{1}{159.2\times 10^{-9}}\left(\frac{1}{50}+\frac{1}{1000000}\right)=\SI{125.634}{\kilo\radian\per\second}\\
\frac{R_s A_v}{R_m\left(\frac{1}{R_m}+\frac{1}{R_i}\right)}&=\frac{8 \times 10.53}{50\left(\frac{1}{50}+\frac{1}{1000000}\right)}\approx 84.2
\end{align*}
یوں درج ذیل لکھا جا سکتا ہے۔
\begin{align}\label{مساوت_تعددی_تبادلی_تفاعل_ایمپلیفائر}
\bA_v(s)= 84.2 \dfrac{sC}{\left(1+\dfrac{s}{125.63}\right)\left(1+\dfrac{s}{125634}\right)}
\end{align}
آئیں اس میں واپس \عددی{s=j\omega=j2\pi f} پر کرتے ہیں۔
\begin{gather}
\begin{aligned}\label{مساوت_تعددی_انقطاعی_تعدد}
\bA_v(s)&= 84.2 \dfrac{j 2 \pi f \times 796\times 10^{-6}}{\left(1+\dfrac{j 2\pi f}{125.63}\right)\left(1+\dfrac{j 2 \pi f}{125634}\right)}\\
&= \dfrac{j 0.421f}{\left(1+\dfrac{j f}{20}\right)\left(1+\dfrac{j f}{20000}\right)}\\
\end{aligned}
\end{gather}
اس کے حتمی قیمت \عددی{\abs{\bA_v(s)}} بالمقابل تعدد \عددی{f} کو شکل \حوالہ{شکل_تعددی_افزائش_بالمقابل_تعددی_خط}-ت میں دکھایا گیا ہے۔
\begin{align*}
\abs{\bA_v(s)}=\frac{0.421 f }{\sqrt{1+\left(\frac{f}{20}\right)^2}\sqrt{1+\left(\frac{f}{20000}\right)^2}}
\end{align*}

شکل-ب میں \عددی{\SI{20}{\hertz}} کو \اصطلاح{پست انقطاعی تعدد}\فرہنگ{پست انقطاعی تعدد}\فرہنگ{تعدد!پست انقطاعی}\فرہنگ{انقطاعی!پست تعدد}\حاشیہب{low cut-off frequency}\فرہنگ{low cut-off frequency}\فرہنگ{cut-off!low frequency} اور \عددی{\SI{20}{\kilo\hertz}} کو \اصطلاح{بلند انقطاعی تعدد}\فرہنگ{بلند انقطاعی تعدد}\فرہنگ{تعدد!بلند انقطاعی}\فرہنگ{انقطاعی!بلند تعدد}\حاشیہب{high cut-off frequency}\فرہنگ{cut-off!high frequency}\فرہنگ{high cut-off frequency} کہتے ہیں۔ انہیں خط کے \اصطلاح{کونے کی تعدد}\فرہنگ{تعدد!کونے کی}\حاشیہب{corner frequencies}\فرہنگ{corner frequencies} بھی کہا جاتا ہے۔

شکل \حوالہ{شکل_تعددی_افزائش_بالمقابل_تعددی_خط}-ت میں انقطاعی تعدد کے مابین \اصطلاح{درمیانی تعدد خطے}\فرہنگ{درمیانی تعدد خطہ}\حاشیہب{mid-frequency range}\فرہنگ{mid-frequency range} میں ایمپلیفائر کی افزائش \عددی{\SI{8.41}{\volt\per\volt}} ہے جو ہمیں درکار تھی۔اس کو درمیانی تعدد پر افزائش کہتے ہیں۔البتہ انقطاعی تعدد کے قریب ایمپلیفائر کی افزائش گھٹ جاتی ہے۔یوں پست اور بلند انقطاعی تعدد پر افزائش \عددی{\SI{5.95}{\volt\per\volt}} ہے۔جس تعدد پر افزائش کی قیمت درمیانی تعدد کے افزائش کے \عددی{\tfrac{1}{\sqrt{2}}} گنا رہ جاتی ہے اس کو انقطاعی تعدد کہتے ہیں۔چونکہ طاقت \عددی{P=\tfrac{\VrmsS}{R}} کے برابر ہے لہٰذا دباو کی قیمت \عددی{\tfrac{1}{\sqrt{2}}} گنا ہو جانے سے طاقت کی قیمت نصف ہو جاتی ہے۔یوں انقطاعی تعدد اس تعدد کو کہتے ہیں جس پر اشارے کی طاقت نصف رہ جاتی ہے۔ہمارے ایمپلیفائر کی درمیانی تعدد پر افزائش  \عددی{\SI{8.41}{\volt\per\volt}} ہے۔اس کا \عددی{\tfrac{1}{\sqrt{2}}} گنا \عددی{8.4\times\tfrac{1}{\sqrt{2}}=\SI{5.95}{\volt\per\volt}}  ہے۔

حقیقت میں پرزوں کی قیمتیں یوں رکھی جائیں گی کہ پست انقطاعی تعدد \عددی{\SI{20}{\hertz}} سے کئی گنا کم ہو اور اسی طرح بلند انقطاعی تعدد \عددی{\SI{20}{\kilo\hertz}} سے کئی گنا زیادہ ہو۔یوں حقیقی ایمپلیفائر میں آپ انقطاعی تعدد کو \عددی{\SI{2}{\hertz}} اور \عددی{\SI{200}{\kilo\hertz}} رکھیں گے تا کہ پوری تعددی پٹی پر ایمپلیفائر سے درکار افزائش میسر ہو۔ 

مساوات \حوالہ{مساوت_تعددی_انقطاعی_تعدد} میں درمیانی تعددی پٹی پر انقطاعی تعدد سے دور  تعدد
\begin{align*}
\SI{20}{\hertz} \ll f \ll \SI{20000}{\hertz}
\end{align*}
کی صورت میں \عددی{\tfrac{f}{20000} \ll 1} اور \عددی{\tfrac{f}{20} \gg 1} ہو گا۔یوں مساوات \حوالہ{مساوت_تعددی_انقطاعی_تعدد} کے بائیں قوسین میں \عددی{1+\tfrac{jf}{20}=\tfrac{jf}{20}}اور دائیں قوسین میں \عددی{1+\tfrac{j f }{20000}=1} لکھتے ہوئے درج ذیل لکھا جا سکتا ہے جو درمیانی تعدد پر افزائش ہے۔
\begin{align*}
\bA_v(s)&\approx  \dfrac{j 0.421f}{\left(\dfrac{jf }{20}\right)\left(1\right)}=8.42  \quad \quad (\SI{20}{\hertz} \ll f \ll \SI{20}{\kilo\hertz})
\end{align*}
%=====================

\حصہ{جال}
کسی بھی دور میں متعدد پرزے اور تار پائے جاتے ہیں جسے پرزوں اور تاروں کا جال تصور کیا جا سکتا ہے۔یوں دور کو \اصطلاح{برقی جال} یا صرف \اصطلاح{جال}\فرہنگ{جال}\حاشیہب{network}\فرہنگ{network} بھی کہا جاتا ہے۔گزشتہ حصے میں ایمپلیفائر کی افزائش دباو \عددی{\bA_v(s)} کی بات کی گئی جو جال کے مختلف تفاعل \عددی{\bH(s)} میں سے ایک ہے۔ جال میں کسی مقام پر ردعمل اور داخلی اشارے کی تناسب کو \عددی{\bH(s)} لکھا جاتا ہے۔چونکہ جال میں عموماً ردعمل کو اس مقام پر نہیں ناپا جاتا جس پر داخلی اشارہ لاگو کیا گیا ہو لہٰذا \عددی{\bH(s)} کو  \اصطلاح{تبادلی تفاعل}\فرہنگ{تبادلی تفاعل}\حاشیہب{transfer function}\فرہنگ{transfer function} کہا جاتا ہے۔ داخلی اشارہ دباو یا رو کا ہو سکتا ہے۔اسی طرح ردعمل بھی دباو یا رو کی صورت میں ممکن ہے لہٰذا تبادلی تفاعل کے چار اقسام ممکنہ پائے جاتے ہیں جنہیں جدول \حوالہ{جدول_تعددی_جال_تبادلی_تفاعل} میں پیش کیا گیا ہے۔
\begin{table}\caption{جال کے تبادلی تفاعل}
\centering
\begin{tabular}{r r r l}
داخلی & خارجی& تبادلی تفاعل& علامت\\
\hline
دباو&دباو&افزائش دباو&$\bA_v(s)$ \\
رو&رو&افزائش رو&$\bA_i(s)$ \\
دباو&رو&موصل نما افزائش& $\bA_g(s)$\\
رو&دباو&دباو نما افزائش&$\bA_r(s)$ 
\end{tabular}\label{جدول_تعددی_جال_تبادلی_تفاعل}
\end{table} 

%=================
\ابتدا{مثال}\شناخت{مثال_تعددی_تبادلی_تفاعل_الف}
شکل \حوالہ{شکل_تعددی_تبادلی_تفاعل_الف} میں تبادلی تفاعل \عددی{\bA_i(s)=\tfrac{\hat{I}_2}{\hat{I}_m}} اور \عددی{\bA_r(s)=\tfrac{\hat{V}_2}{\hat{I}_m}} حاصل کریں۔
\begin{figure}
\centering
\begin{tikzpicture}[american voltages]
\draw(0,0) to [american current source,l={$\hat{I}_m$}]++(0,\y) to [short]++(\x,0) to [capacitor,l={$\frac{1}{sC}$}]++(\x,0) to [resistor,l={$R$},i>_={$\hat{I}_2$}]++(\x,0) to [inductor,l={$sL_2$},v={$\hat{V}_2$}]++(0,-\y) to [short] (0,0);
\draw(\x,0) to [inductor,*-*,l={$sL_2$}]++(0,\y);
\end{tikzpicture}
\caption{مثال \حوالہ{مثال_تعددی_تبادلی_تفاعل_الف} کا دور۔}
\label{شکل_تعددی_تبادلی_تفاعل_الف}
\end{figure}

حل:تقسیم رو کے کلیے سے درج ذیل لکھتے ہیں
\begin{align*}
\hat{I}_2=\frac{ sL_1 \hat{I}_m}{sL_1+\frac{1}{sC}+R+sL_2}
\end{align*}
جس سے افزائش رو کی تفاعل لکھتے ہیں۔
\begin{align*}
\bA_i(s)=\frac{\hat{I}_2}{\hat{I}_m}=\frac{ s^2 L_1 C}{s^2 (L_1+L_2)C +s RC +1}
\end{align*}
رو \عددی{\hat{I}_2} جانتے ہوئے \عددی{\hat{V}_2} لکھتے ہیں
\begin{align*}
\hat{V}_2&= sL_2\hat{I}_2\\
&=\frac{ s^3 L_1 L_2 C \hat{I}_m}{s^2 (L_1+L_2)C +s RC +1}
\end{align*}
جس سے مزاحمت نما افزائش لکھتے ہیں۔
\begin{align*}
\bA_r(s)&=\frac{\hat{V}_2}{\hat{I}_m}=\frac{ s^3 L_1 L_2 C}{s^2 (L_1+L_2)C +s RC +1}
\end{align*}
\انتہا{مثال}
%======================

\حصہ{صفر اور قطب}
درج بالا مثال میں ہم نے دیکھا کہ تبادلی تفاعل کو دو سلسلوں کا تناسب \عددی{\tfrac{A(s)}{B(s)}} لکھا جا سکتا ہے جن کا متغیرہ \عددی{s} ہے۔چونکہ ادوار میں پرزوں کی قیمت اور تابع یا غیر تابع منبع کی قیمت حقیقی اعداد ہوتے ہیں لہٰذا ان سلسلوں کے سر حقیقی  اعداد ہوں گے۔یوں کسی بھی جال کا تبادلی تفاعل درج ذیل لکھا جا سکتا ہے
\begin{align}\label{مساوات_تعددی_تبادلی_تفاعل_الف}
\bH(s)=\frac{A(s)}{B(s)}=\frac{a_m s^m+a_{m-1} s^{m-1}+\cdots+a_2 s^2+a_1 s+a_0}{b_n s^n+b_{n-1} s^{n-1}+\cdots+b_2 s^2+b_1 s+b_0}
\end{align}
جہاں شمار کنندہ تسلسل \عددی{m} درجے کا ہے جبکہ  نسب نما تسلسل \عددی{n} درجے کا ہے۔مساوات \حوالہ{مساوات_تعددی_تبادلی_تفاعل_الف} کو بذریعہ تجزی درج ذیل لکھا جا سکتا ہے
\begin{align}\label{مساوات_تعددی_تبادلی_تفاعل_ب}
\bH(s)=\frac{K_0 (s-z_1)(s-z_2)\cdots (s-z_m)}{(s-p_1)(s-p_2) \cdots (s-p_n)}
\end{align}
جہاں \عددی{K_0} مساوات کا مستقل، \عددی{z_1} تا \عددی{z_m} شمار کنندہ کے  حل  اور \عددی{p_1} تا \عددی{p_n} نسب نما کے حل ہیں۔یہاں غور کریں کہ اگر \عددی{s=z_1} ہو تب \عددی{\bH(s)=0} ہو گا۔اسی طرح  اگر \عددی{s} کی قیمت \عددی{z_1} تا \عددی{z_m} میں کسی ایک کے بھی برابر ہو تو  \عددی{\bH(s)=0} ہو گا۔یہی وجہ ہے کہ \عددی{z_1} تا \عددی{z_m} تفاعل کے \اصطلاح{صفر}\فرہنگ{صفر}\حاشیہب{zeroes}\فرہنگ{zeroes} کہلاتے ہیں۔اس کے برعکس اگر \عددی{s} کی قیمت \عددی{p_1} تا \عددی{p_n} میں کسی بھی ایک کے برابر ہو تب \عددی{\bH(s)} کی قیمت لامتناہی ہو گی۔اسی لئے \عددی{p_1} تا \عددی{p_n} تفاعل کے \اصطلاح{قطب}\فرہنگ{قطب}\حاشیہب{poles}\فرہنگ{poles} کہلاتے ہیں۔تفاعل کے صفر اور قطب مخلوط اعداد ہو سکتے ہیں۔مخلوط اعداد کی صورت میں ان کی جوڑیاں پائی جاتی ہیں جہاں جوڑی کے دونوں اعداد ایک دوسرے کے جوڑی دار مخلوط اعداد ہوتے ہیں۔ایسی جوڑی کے قوسین ضرب کرنے سے حقیقی سر والے تسلسل دیتے ہیں جو ادوار کو ظاہر کر سکتے ہیں۔مساوات \حوالہ{مساوات_تعددی_تبادلی_تفاعل_ب} کسی بھی خطی اور وقت کے ساتھ نہ بدلنے والے نظام کے تبادلی تفاعل لکھنے کا انتہائی اہم طریقہ ہے چونکہ اس کے قطبین کو دیکھ کر تفاعل کی خاصیت کے بارے میں جانا جا سکتا ہے۔ایسے نظام کے تبادلی تفاعل کو عموماً اسی صورت میں لکھا جاتا ہے۔
%==============
\ابتدا{مشق}
شکل \حوالہ{شکل_تعددی_افزائش_بالمقابل_تعددی_خط}-الف کا تبادلی تفاعل مساوات \حوالہ{مساوت_تعددی_تبادلی_تفاعل_ایمپلیفائر} میں دیا گیا ہے۔ اس کے صفر، قطب اور \عددی{K_0} دریافت کریں۔

جوابات:\عددی{z_1=\SI{0}{\hertz}}، \عددی{p_1=\SI{20}{\hertz}}، \عددی{p_2=\SI{20}{\kilo\hertz}}، \عددی{K_0=1.06\times 10^6}
\انتہا{مشق}
%======================
\ابتدا{مشق}
شکل \حوالہ{شکل_تعددی_افزائش_بالمقابل_تعددی_خط}-الف میں داخلی اشارے کو درپیش رکاوٹ دریافت کریں۔

جواب:\عددی{R_m+\frac{R_i}{1+sR_i C_i}}
\انتہا{مشق}
%=================
\ابتدا{مشق}
شکل میں تبادلی تفاعل \عددی{\tfrac{\hat{V}_0(s)}{\hat{V}_i(s)}} حاصل کریں۔
\begin{figure}
\centering
\begin{tikzpicture}
\draw(0,0) to [inductor,o-,l={$s L_1$}]++(\x,0) to [capacitor,l={$\frac{1}{sC_1}$}]++(\x,0) to [short]++(\x,0) to [short,-o]++(\x/2,0);
\draw(0,-\y) to [short,o-o]++(3*\x+\x/2,0);
\draw(2*\x,-\y) to [inductor,*-*,l={$sL_2$}]++(0,\y);
\draw(3*\x,-\y) to [capacitor,*-*,l={$\frac{1}{sC_2}$}]++(0,\y);
\draw(0,-\y/2)node{$\begin{aligned} &+ \\ &\hat{V}_i(s) \\ &- \end{aligned}$};
\draw(3*\x+\x/2,-\y/2)node{$\begin{aligned} &+ \\ &\hat{V}_0(s) \\& - \end{aligned}$};
\end{tikzpicture}
\caption{}
\label{}
\end{figure}

جواب:
\begin{align*}
\frac{\hat{V}_0(s)}{\hat{V}_i(s)}=\frac{s^2 L_2 C_1}{s^4 L_1 L_2 C_1 C_2 +s^2(L_1 C_1+L_2 C_2+L_2 C_1)+1}
\end{align*}
\انتہا{مشق}
%================
\حصہ{سائن نما تعددی تجزیہ}

\باب{لاپلاس بدل}

\حصہ{تعریف}
کسی تفاعل \عددی{f(t)} کا \اصطلاح{لاپلاس بدل}\فرہنگ{لاپلاس بدل}\حاشیہب{Laplace transform}\فرہنگ{Laplace transform} درج ذیل مساوات دیتا ہے
\begin{align}\label{مساوات_لاپلاس_بدل_تعارف}
\Laplace[f(t)]=\bF(s)=\int_0^{\infty} f(t) e^{-st}\dif t
\end{align}
جہاں \عددی{s} \اصطلاح{مخلوط تعدد}\فرہنگ{مخلوط تعدد}\فرہنگ{تعدد!مخلوط}\حاشیہب{complex frequency}\فرہنگ{complex!frequency}\فرہنگ{frequency!complex} ہے
\begin{align}
s=\sigma+j\omega
\end{align}
اور تفاعل \عددی{f(t)} کی قیمت \عددی{t<0} پر صفر کے برابر ہے۔
\begin{align}
f(t)=0\quad t<0
\end{align}
لاپلاس بدل سے ادوار کا حل \عددی{t\ge0} کے لئے حاصل کیا جاتا ہے جبکہ \عددی{t<0} کو ابتدائی حالت میں سمویا جاتا ہے۔ لاپلاس بدل وقتی دائرہ کار میں تفاعل \عددی{f(t)} کو تعددی دائرہ کار کے تفاعل \عددی{\bF(s)} میں تبدیل کرتی ہے۔ 

کسی تفاعل کا لاپلاس بدل اس صورت پایا جاتا ہے جب تفاعل درج ذیل شرط پر پورا اترتا ہو جہاں \عددی{\sigma} کوئی مثبت قیمت ہے۔
\begin{align}\label{مساوات_لاپلاس_شرط_بدل_پایا_جاتا_ہے}
\int_0^{\infty} e^{-\sigma t}\abs{f(t)}\dif t<\infty
\end{align}
لاپلاس بدل کے حصول میں \عددی{e^{-\sigma t}} کے ارتکازی جزو  کی بنا کئی ایسے کئی اہم تفاعل کے لاپلاس بدل پائے جاتے ہیں جن کے \اصطلاح{فوریئر بدل}\فرہنگ{فوریئر بدل}\حاشیہب{Fourier transform}\فرہنگ{Fourier transform} نہیں پائے جاتے۔برقی ادوار میں ایسے تفاعل استعمال کئے جاتے ہیں جن کے لاپلاس بدل پائے جاتے ہوں۔

\اصطلاح{الٹ لاپلاس بدل}\فرہنگ{الٹ لاپلاس بدل}\فرہنگ{لاپلاس بدل!الٹ}\حاشیہب{inverse Laplace transform}\فرہنگ{inverse Laplace transform}\فرہنگ{Laplace!inverse transform} درج ذیل مساوات دیتی ہے
\begin{align}
\Laplace^{-1}\left[\bF(s)\right]=f(t)=\frac{1}{2\pi j}\int_{\sigma_1-j\omega}^{\sigma+j\omega} \bF(s) e^{st} \dif s
\end{align}
 جہاں \عددی{\sigma_1} حقیقی ہے اور اس کی قیمت مساوات \حوالہ{مساوات_لاپلاس_شرط_بدل_پایا_جاتا_ہے} کے \عددی{\sigma} سے زیادہ ہے یعنی \عددی{\sigma_1>\sigma} ہے۔الٹ لاپلاس بدل تعددی دائرہ کار میں تفاعل \عددی{\bF(s)} کو وقتی دائرہ کار کے تفاعل \عددی{f(t)} میں تبدیل کرتی ہے۔

لاپلاس بدل آسانی سے حاصل ہوتا ہے جبکہ الٹ لاپلاس بدل مشکل سے حاصل ہوتا ہے۔ہم کئی تفاعل کے لاپلاس بدل حاصل کرتے ہوئے انہیں جدول میں جوڑیوں کی صورت میں لکھیں گے اور الٹ بدل کو اسی جدول سے دیکھ کر حاصل کریں گے۔کسی بھی وقتی تفاعل \عددی{f(t)} کا منفرد لاپلاس بدل \عددی{\bF(s)} پایا جاتا ہے لہٰذا دو مختلف وقتی تفاعل \عددی{f_1(t)} اور \عددی{f_2(t)} کے لاپلاس بدل کسی بھی صورت میں یکساں نہیں ہو سکتے ہیں۔یوں کسی بھی لاپلاس بدل \عددی{\bF(s)} کو سادہ ترین اجزاء میں تقسیم کرتے ہوئے ان کے الٹ بدل کو جدول سے پڑھا جاتا ہے۔تمام اجزاء کے الٹ لاپلاس بدل کا مجموعہ درکار وقتی تفاعل ہو گا۔ہم لاپلاس بدل کو \اصطلاح{جزوی کسری پھیلاو}\فرہنگ{جزوی کسری پھیلاو}\حاشیہب{partial fraction expansion}\فرہنگ{partial fraction expansion} کے ذریعہ اجزاء میں تقسیم کریں گے۔

\حصہ{تفاعل یکتائی}
برقی ادوار میں \اصطلاح{اکائی سیڑھی تفاعل}\فرہنگ{اکائی سیڑھی تفاعل}\فرہنگ{سیڑھی!تفاعل}\حاشیہب{unit step function}\فرہنگ{unit step function} \عددی{u(t)} اور  \اصطلاح{اکائی ضرب تفاعل}\فرہنگ{اکائی ضرب تفاعل}\حاشیہب{unit impulse function}\فرہنگ{impulse!function}  \عددی{\sigma(t)} نہایت اہم ہیں۔ایسے تفاعل جو یا تو خود کہیں غیر متناہی ہوں اور یا ان کا تفرق  کہیں غیر متناہی ہو کو \اصطلاح{یکتائی تفاعل}\فرہنگ{یکتائی تفاعل}\حاشیہب{singularity function}\فرہنگ{singularity function} کہتا ہے۔ اکائی سیڑھی تفاعل اور اکائی ضرب تفاعل یکتائی تفاعل ہیں۔اکائی سیڑھی تفاعل پر صفحہ \حوالہصفحہ{حصہ_عارضی_دھڑکن} پر حصہ \حوالہ{حصہ_عارضی_دھڑکن} میں ہم غور کر چکے ہیں۔

\begin{figure}
\centering
\begin{subfigure}{0.5\textwidth}
\centering
\begin{tikzpicture}
\draw[gray](-0.5,0)--(3,0)node[right]{$t$};
\draw[gray](0,-0.5)--(0,2)node[left]{$u(t)$};
\draw(-0.2,0)--(0,0)--(0,1)node[left]{$1$}--(3,1);
\end{tikzpicture}
\caption*{(الف)}
\end{subfigure}%
\begin{subfigure}{0.5\textwidth}
\centering
\begin{tikzpicture}
\draw[gray](-0.5,0)--(3,0)node[right]{$t$};
\draw[gray](0,-0.5)--(0,2)node[left]{$u(t-a)$};
\draw(-0.2,0)--(1,0)node[below]{$a$}--(1,1)--(3,1);
\draw[dashed] (1,1)--(0,1)node[left]{$1$};
\end{tikzpicture}
\caption*{(ب)}
\end{subfigure}%
\caption{اکائی سیڑھی تفاعل۔}
\label{شکل_لاپلاس_اکائی_سیڑھی_الف}
\end{figure}

شکل \حوالہ{شکل_لاپلاس_اکائی_سیڑھی_الف}-الف میں دکھایا گیا اکائی سیڑھی تفاعل درج ذیل لکھا جاتا ہے۔
\begin{align}
u(t)=
\begin{cases}
0 & t<0\\
1& t>0
\end{cases}
\end{align}
اکائی سیڑھی تفاعل \عددی{u(t)}، جیسے باب \حوالہ{باب_عارضی_رد_عمل} میں ذکر کیا گیا، لمحہ \عددی{t=\SI{0}{\second}} پر سوئچ چالو کرتے ہوئے  دور پر \عددی{\SI{1}{\volt}} یا \عددی{\SI{1}{\ampere}} لاگو کرنے کے مترادف ہے۔آئیں شکل \حوالہ{شکل_لاپلاس_اکائی_سیڑھی_الف}-الف میں دکھائے گئے  اکائی سیڑھی تفاعل کا لاپلاس بدل حاصل کریں۔
%=====================
\ابتدا{مثال}\شناخت{مثال_لاپلاس_اکائی_سیڑھی_بدل}
شکل \حوالہ{شکل_لاپلاس_اکائی_سیڑھی_الف} کے تفاعل کا لاپلاس بدل حاصل کریں۔

حل:مساوات \حوالہ{مساوات_لاپلاس_بدل_تعارف} کے استعمال سے شکل-الف کا لاپلاس بدل حاصل کرتے ہیں۔
\begin{align*}
\Laplace[u(t)]&=\int_0^{\infty} u(t) e^{-st}\dif t\\
&=\int_0^{\infty} 1e^{-st} \dif t\\
&=\left . \frac{e^{-st}}{-s}\right|_0^{\infty}\\
&=\frac{e^{-\infty s}-e^{-0 s}}{-s}\\
&=\frac{1}{s} \quad \sigma >0
\end{align*}
حاصل ہوتا ہے  جہاں آخری قدم پر \عددی{\sigma>0} کی بنا \عددی{e^{-\infty s}=0} لکھا گیا ہے۔ اس طرح اکائی سیڑھی تفاعل کا لاپلاس بدل درج ذیل ہے۔
\begin{align}
\Laplace[u(t)]=\bF(s)=\frac{1}{s}
\end{align}
شکل \حوالہ{شکل_لاپلاس_اکائی_سیڑھی_الف}-ب میں  وقت کے لحاظ سے منتقل ہوا اکائی سیڑھی تفاعل دکھایا گیا ہے جس کو \اصطلاح{وقتی منقولہ اکائی سیڑھی تفاعل}\فرہنگ{اکائی سیڑھی تفاعل!وقتی منقولہ}\حاشیہب{time-shifted unit step function}\فرہنگ{unit step function!time-shifted} کہتے ہیں۔آئیں اس کا لاپلاس بدل حاصل کریں۔
\begin{align*}
\Laplace[u(t-a)]&=\int_0^{\infty} u(t-a) e^{-st} \dif t\\
&=\int_0^a 0 e^{-st} \dif t+\int_a^{\infty} 1 e^{-st}\dif t\\
&=0+\left. \frac{e^{-st}}{-s}\right|_a^{\infty}\\
&=\frac{e^{-as}}{s} \quad \sigma >0
\end{align*} 
اس طرح وقتی منقولہ اکائی سیڑھی تفاعل کا لاپلاس بدل درج ذیل ہے۔
\begin{align}\label{مساوات_لاپلاس_منقولہ_اکائی_سیڑھی}
\Laplace[u(t-a)]=F(s)=\frac{e^{-as}}{s}
\end{align}
\انتہا{مثال}
%=============================
\ابتدا{مثال}\شناخت{مثال_لاپلاس_دھڑکن}
شکل \حوالہ{شکل_لاپلاس_دھڑکن}-الف میں دو عدد اکائی سیڑھی تفاعل سے دھڑکن کا حصول دکھایا گیا ہے۔دھڑکن کا لاپلاس بدل حاصل کریں۔شکل-ب میں وقت کے لحاظ سے منتقل شدہ دھڑکن دکھائی گئی ہے۔اس کا بھی لاپلاس بدل حاصل کریں۔ 
 \begin{figure}
\centering
\begin{subfigure}{0.5\textwidth}
\centering
\begin{tikzpicture}
\draw[gray](-0.5,0) --(3,0)node[below]{$t$};
\draw[gray](0,-2)--(0,2)node[left]{$u(t)$};
\draw (0,0)--(0,0)--(0,1)node[left]{$1$}--(3,1);
\draw (-0.25,0)--(1,0)node[above]{$a$}--(1,-1)--(3,-1);
\draw[dashed](1,-1)--(0,-1)node[left]{$-1$};
\end{tikzpicture}
\begin{tikzpicture}[yshift=-3cm]
\draw[gray](-0.5,0) --(3,0)node[below]{$t$};
\draw[gray](0,-0.25)--(0,2)node[left]{$u(t)$};
\draw (-0.25,0)--(0,0)--(0,1)node[left]{$1$}--(1,1) --(1,0)node[below]{$a$}--(3,0);
\end{tikzpicture}
\caption*{(الف)}
\end{subfigure}%
\begin{subfigure}{0.5\textwidth}
\centering
\begin{tikzpicture}
\draw[gray](-0.5,0) --(3,0)node[below]{$t$};
\draw[gray](0,-2)--(0,2)node[left]{$u(t)$};
\draw (-0.25,0)--(1,0)node[below]{$b$}--(1,1)--(3,1);
\draw (-0.25,0)--(2,0)node[above]{$c$}--(2,-1)--(3,-1);
\draw[dashed](1,1)--(0,1)node[left]{$1$};
\draw[dashed](2,-1)--(0,-1)node[left]{$-1$};
\end{tikzpicture}
\begin{tikzpicture}[yshift=-3cm]
\draw[gray](-0.5,0) --(3,0)node[below]{$t$};
\draw[gray](0,0)--(0,2)node[left]{$u(t)$};
\draw (-0.25,0)--(1,0)node[below]{$b$}--(1,1)--(2,1) --(2,0)node[below]{$c$}--(3,0);
\draw[dashed](1,1)--(0,1)node[left]{$1$};
\end{tikzpicture}
\caption*{(ب)}
\end{subfigure}%
\caption{مثال \حوالہ{مثال_لاپلاس_دھڑکن} کے  اشکال۔}
\label{شکل_لاپلاس_دھڑکن}
\end{figure}

حل:شکل \حوالہ{شکل_لاپلاس_دھڑکن}-الف کے دھڑکن کو درج ذیل لکھا جا سکتا ہے۔
\begin{align}
f(t)=
\begin{cases}
0& t<0\\
1& 0<t<a\\
0& t>a
\end{cases}
\end{align}
لہٰذا لاپلاس تکمل درج ذیل ہو گا
\begin{align*}
\Laplace[f(t)]&=\int_0^{\infty} f(t) e^{-st} \dif t\\
&=\int_0^a 1e^{-st} \dif t\\
&=\frac{1-e^{-as}}{s} \quad \sigma >0
\end{align*}
یعنی دھڑکن کا لاپلاس بدل
\begin{align}
\Laplace[f(t)]=\frac{1-e^{-as}}{s}
\end{align}
ہو گا۔شکل \حوالہ{شکل_لاپلاس_دھڑکن}-ب کے تفاعل کو اکائی سیڑھی تفاعل کا مجموعہ لکھتے ہوئے
\begin{align*}
f(t)=u(t-b)-u(t-c)
\end{align*} 
لاپلاس بدل لکھتے ہیں۔
\begin{align*}
\Laplace[f(t)]=\Laplace[u(t-b)]-\Laplace[u(t-c)]
\end{align*}
مساوات \حوالہ{مساوات_لاپلاس_منقولہ_اکائی_سیڑھی} کے استعمال سے درج بالا کو 
\begin{align}
\Laplace[f(t)]=\bF(s)=\frac{e^{-bs}-e^{-cs}}{s}
\end{align}
لکھ سکتے ہیں۔
\انتہا{مثال}
%===========================

\حصہء{اکائی ضرب تفاعل}
شکل \حوالہ{شکل_لاپلاس_اکائی_جھٹکا_تفاعل}-الف کے مستطیل کی چوڑائی \عددی{a} اور لمبائی \عددی{\tfrac{1}{a}} ہے لہٰذا اس کا رقبہ \عددی{(a\times \tfrac{1}{a}=1)}
 اکائی کے برابر ہے۔مستطیل کی چوڑائی لامتناہی کم \عددی{(a \to 0)} کرنے سے اس کی لمبائی لامتناہی بڑھ \عددی{(\tfrac{1}{a} \to \infty)} جائے گی البتہ اس کا رقبہ اکائی ہی رہے گا۔ایسا مستطیل جس کی چوڑائی صفر کے قریب تر اور رقبہ اکائی ہو کو \اصطلاح{اکائی ضرب تفاعل}\فرہنگ{اکائی ضرب تفاعل}\حاشیہب{unit impulse function}\فرہنگ{impulse!unit}\فرہنگ{unit impulse function} تصور کیا جا سکتا ہے۔لمحہ \عددی{t_0} پر پائے جانے والے اکائی ضرب تفاعل کو \عددی{\delta(t-t_0)} لکھا جاتا ہے جس کو ترسیمی طور پر شکل \حوالہ{شکل_لاپلاس_اکائی_جھٹکا_تفاعل}-ب میں دکھایا گیا ہے۔اکائی ضرب تفاعل کو کئی دیگر تفاعل سے بھی ظاہر کیا جا سکتا ہے۔

لکڑی پر کیل کو ہتھوڑی سے ضرب لگانے سے نہایت کم وقت کے لئے انتہائی زیادہ طاقت عمل میں آتا ہے اگرچہ ہتھوڑی کی توانائی محدود ہوتی ہے۔اگر ہتھوڑی کی توانائی ایک جاول ہوتی تو اس کو \اصطلاح{اکائی ضرب تفاعل} تصور کیا جا سکتا ہے۔اسی مشابہت سے ہم ایسے تفاعل کو \اصطلاح{اکائی ضرب تفاعل} کہیں گے۔

اکائی ضرب تفاعل کو الجبرائی صورت میں لکھتے ہیں۔
\begin{gather}
\begin{aligned}\label{مساوات_لاپلاس_اکائی_جھٹکا_تفاعل_الجبرائی_تعریف}
\delta(t-t_0)&=0 \quad t\ne t_0\\
\int_{t_0-\epsilon}^{t_0+\epsilon} \delta(t-t_0)\dif t &=1 \quad  \epsilon >0
\end{aligned}
\end{gather}
اکائی جھٹکے کی قیمت لمحہ \عددی{t=t_0} پر غیر معین ہے جبکہ اس لمحے کے علاوہ اس کی قیمت صفر کے برابر ہے البتہ جھٹکے کا رقبہ اکائی ہے۔ جھٹکے کے رقبے کو تفاعل کا \اصطلاح{زور}\فرہنگ{زور} بھی کہتے ہیں۔
 \begin{figure}
\centering
\begin{subfigure}{0.6\textwidth}
\centering
\begin{tikzpicture}
\draw[gray](0,0)--(6,0)node[below]{$t$};
\draw[gray](0,0)--(0,2.5)node[left]{$f(t)$};
\draw(0,0)--(2.5,0)node[shift={(-0.3,-0.3)}]{$t_0-\frac{a}{2}$}--(2.5,2)--(4,2)--(4,0)node[shift={(0.3,-0.3)}]{$t_0+\frac{a}{2}$}--(6,0);
\draw[stealth-stealth] (2.5,0)++(-0.3,0)--++(0,2)node[pos=0.5,fill=white]{$\frac{1}{a}$};
\draw(3.25,0)node[below]{$t_0$};
\end{tikzpicture}
\caption*{(الف)}
\end{subfigure}%
\begin{subfigure}{0.4\textwidth}
\centering
\begin{tikzpicture}
\draw[gray](0,0)--(3,0)node[below]{$t$};
\draw[gray](0,0)--(0,2.5)node[left]{$f(t)$};
 \draw[-latex](1.5,0)node[below]{$t_0$}--++(0,2)node[pos=0.7,right]{$\delta(t-t_0)$};
\end{tikzpicture}
\caption*{(ب)}
\end{subfigure}%
\caption{اکائی ضرب تفاعل۔}
\label{شکل_لاپلاس_اکائی_جھٹکا_تفاعل}
\end{figure}

اکائی ضرب تفاعل کی ایک اہم خاصیت جسے \اصطلاح{خاصیت نمونہ بندی}\فرہنگ{خاصیت نمونہ بندی}\فرہنگ{نمونہ بندی!خاصیت}\حاشیہب{sampling property}\فرہنگ{sampling property} کہتے ہیں کو درج ذیل تکمل سے سمجھا جا سکتا ہے
\begin{align*}
\int_0^{\infty} f(t) \delta(t-t_0) \dif t&=\int_{t_0-\epsilon}^{t_0+\epsilon} f(t_0) \delta(t-t_0) \dif t\\
&=f(t_0) \int_{t_0-\epsilon}^{t_0+\epsilon} \delta(t-t_0) \dif t\\
&= f(t_0)
\end{align*}
جہاں \عددی{t_0-\epsilon} تا \عددی{t_0+\epsilon} کے علاوہ \عددی{\delta(t-t_0)=0} ہے لہٰذا تکمل کے حدود یہی کر دیے گئے ہیں۔چونکہ \عددی{\epsilon \to 0} ہے لہٰذا ان حدود کے مابین کسی بھی تفاعل کی قیمت میں تبدیلی کو نظر انداز کرتے ہوئے تفاعل کی قیمت \عددی{f(t_0)} لی جا سکتی ہے۔غیر تغیر \عددی{f(t_0)} کو تکمل کے باہر لے جایا جا سکتا ہے۔یوں ہمارے پاس صرف \عددی{\delta(t-t_0)} کا تکمل رہ جاتا ہے جو مساوات \حوالہ{مساوات_لاپلاس_اکائی_جھٹکا_تفاعل_الجبرائی_تعریف} کے تحت اکائی کے برابر ہے۔درج بالا مساوات کو درج ذیل لکھا جا سکتا ہے جہاں سے واضح ہے کہ اکائی ضرب تفاعل \عددی{f(t)} کا نمونہ \عددی{t=t_0} پر حاصل کرتا ہے۔
\begin{align}\label{مساوات_لاپلاس_خاصیت_نمونہ_بندی}
\int_{t_1}^{t_2} f(t) \delta(t-t_0)=
\begin{cases}
f(t_0)& t_1<t_0<t_2\\
0&t_0<t_1, \, t_0>t_2 
\end{cases}
\end{align}

اگرچہ حقیقی دنیا میں ہم لمحاتی طور پر لامحدود قیمت کا دباو یا رو کسی دور پر لاگو نہیں کر سکتے  ہیں لہٰذا حقیقی دنیا میں اکائی ضرب تفاعل نہیں پایا جاتا ہے۔اس کے باوجود یہ ایک اہم تفاعل ہے جس کو استعمال کرتے ہوئے الجبرائی طور پر مختلف اعمال کا مطالعہ ممکن بنایا جاتا ہے۔مثال کے طور پر آسمانی بجلی کو اکائی ضرب تصور کیا جا سکتا ہے۔اسی طرح آواز کو عددی صورت میں تبدیل کرنے کے عمل پر غور کے لئے اس تفاعل کا سہارا لیا جاتا ہے۔\اصطلاح{مماثل سے عددی مبادل کار}\فرہنگ{مماثل سے عددی مبادل کار}\حاشیہب{analog to digital converter, ADC}\فرہنگ{analog to digital converter}\فرہنگ{ADC}  کی مدد سے مماثل اشارے کو عددی صورت میں  تبدیل کیا جاتا ہے۔انسانی کان \عددی{\SI{20}{\hertz}} تا \عددی{\SI{20}{\kilo\hertz}} تک کی آواز سن سکتا ہے۔\اصطلاح{اصول نِی کوسٹ}\فرہنگ{اصول نی کوسٹ}\فرہنگ{نی کوسٹ!اصول}\حاشیہب{Nyquist criterion}\فرہنگ{Nyquist!criterion}  کے تحت کسی بھی اشارے کی مکمل معلومات برقرار رکھنے کی خاطر اشارے کی بلند تر تعدد کی دگنی تعدد پر نمونہ حاصل کرنا ضروری ہے۔یہی وجہ ہے کہ انسانی آواز کے عددی نمونے \عددی{\SI{44.1}{\kilo\hertz}} پر حاصل کئے جاتے ہیں۔
%===================

\ابتدا{مثال}
اکائی ضرب تفاعل کا لاپلاس بدل حاصل کریں۔

حل:لاپلاس تکمل لکھتے ہیں۔
\begin{align*}
\Laplace[\delta(t-t_0)]&=\int_0^{\infty} \delta(t-t_0) e^{-st} \dif t\\
&=\int_{t_0-\epsilon}^{t_0+\epsilon} \delta(t-t_0) e^{-st} \dif t\\
&=e^{-s t_0} \int_{t_0-\epsilon}^{t_0+\epsilon} \delta(t-t_0) \dif t\\
&=e^{-st_0}
\end{align*}
اس جواب  کو مساوات \حوالہ{مساوات_لاپلاس_خاصیت_نمونہ_بندی} میں دی گئی خاصیت نمونہ بندی کی مدد سے بھی حاصل کیا جا سکتا ہے یعنی
  \begin{align*}
\Laplace[\delta(t-t_0)]&=\int_0^{\infty} \delta(t-t_0) e^{-st} \dif t
\end{align*}
میں \عددی{e^{-st}=f(t)} تصور کرتے ہوئے خاصیت نمونہ بندی استعمال کرتے  ہوئے درج ذیل لکھا جا سکتا ہے۔
\begin{align}
\Laplace[\delta(t-t_0)]=\bF(s)=e^{-st_0}
\end{align}
چونکہ \عددی{e^{-0 s} =1} کے برابر ہے لہٰذا درج بالا سے درج ذیل لکھا جا سکتا ہے۔
\begin{align}
\Laplace[\delta(t)]=\bF(s)=1
\end{align}
\انتہا{مثال}
%====================
\حصہ{لاپلاس بدل کی جوڑیاں}
آئیں کئی اہم لاپلاس بدل کی جوڑیاں حاصل کریں۔

%=======================
\ابتدا{مثال}
تفاعل \عددی{f(t)=t} کا لاپلاس بدل دریافت کریں۔

حل:لاپلاس تکمل استعمال کرتے ہیں۔
\begin{align*}
\bF(s)=\int_0^{\infty} t e^{-st} \dif t
\end{align*}
تکمل کو ٹکڑوں میں حاصل کرنے کی خاطر ہم 
\begin{align*}
u&=t\\
\dif v&=e^{-st} \dif t
\end{align*}
 لیتے ہیں۔یوں
\begin{align*}
\dif u &= \dif t\\
v&=\int e^{-st} \dif t=-\frac{e^{-st}}{-s}
\end{align*}
ہو گا لہٰذا
\begin{gather}
\begin{aligned}
\bF(s)&=\left. -\frac{t}{s}e^{-st}\right|_0^{\infty}+\int_0^{\infty} \frac{e^{-st}}{s} \dif t\\
&=\frac{1}{s^2} \quad \sigma>0
\end{aligned}
\end{gather}
حاصل ہوتا ہے۔
\انتہا{مثال}
%==========================
\ابتدا{مثال}
تفاعل \عددی{e^{at}} کا لاپلاس بدل حاصل کریں۔

حل:
\begin{align*}
\bF(s)&=\int_0^{\infty} e^{at} e^{-st} \dif t\\
&=\int_0^{\infty} e^{-(s-a)t} \dif t\\
&=\left. \frac{e^{-(s-a)t}}{-(s-a)}\right|_0^{\infty} \quad \sigma>0\\
&=\frac{1}{s-a}
\end{align*}
\انتہا{مثال}
%===========================
\ابتدا{مثال}
تفاعل \عددی{\cos \omega t} کا لاپلاس بدل حاصل کریں۔

حل:کوسائن کو \عددی{\tfrac{e^{+j \omega t}+e^{-j\omega t}}{2}} لکھتے ہوئے لاپلاس تکمل حل کرتے ہیں۔
\begin{align*}
\bF(s)&=\int_0^{\infty} \frac{e^{+j\omega t}+e^{-j\omega t}}{2} e^{-st} \dif t\\
&=\int_0^{\infty} \frac{e^{-(s-j\omega)t}+e^{-(s+j\omega)t}}{2} \dif t\\
&=\frac{1}{2}\left(\frac{1}{s-j\omega}+\frac{1}{s+j\omega}\right) \quad \sigma>0\\
&=\frac{s}{s^2+\omega^2}
\end{align*}
\انتہا{مثال}
%=======================

\ابتدا{مثال}
تفاعل \عددی{\sin \omega t} کا لاپلاس بدل حاصل کریں۔

حل:سائن کو \عددی{\tfrac{e^{+j \omega t}-e^{-j\omega t}}{j2}} لکھتے ہوئے لاپلاس تکمل حل کرتے ہیں۔
\begin{align*}
\bF(s)&=\int_0^{\infty} \frac{e^{+j\omega t}-e^{-j\omega t}}{j2} e^{-st} \dif t\\
&=\int_0^{\infty} \frac{e^{-(s-j\omega)t}-e^{-(s+j\omega)t}}{j2} \dif t\\
&=\frac{1}{j2}\left(\frac{1}{s-j\omega}-\frac{1}{s+j\omega}\right) \quad \sigma>0\\
&=\frac{\omega}{s^2+\omega^2}
\end{align*}
\انتہا{مثال}
%=======================

جدول \حوالہ{جدول_لاپلاس_بدل_جوڑیاں} میں کئی لاپلاس بدل کی جوڑیاں پیش کی گئی ہیں۔
\begin{table}
\caption{لاپلاس بدل کی جوڑیاں۔}
\label{جدول_لاپلاس_بدل_جوڑیاں}
\centering
\begin{tabular}{l l}
$F(s)$&$f(t)$\\[2ex] 
\hline 
$1$&$\delta(t)$ \\[2ex] 
$e^{-s t_0}$&$\delta(t-t_0)$ \\[2ex] 
$\frac{1}{s}$&$u(t)$ \\[2ex] 
$\frac{e^{-as}}{s}$&$u(t-a)$ \\[2ex] 
$\frac{1}{s^2}$&$t$ \\[2ex] 
$\frac{n!}{s^{n+1}}$&$t^n$ \\[2ex] 
$\frac{1}{s\pm a}$&$e^{\mp at}$ \\[2ex] 
$\frac{1}{(s+a)^2}$&$te^{-at}$ \\[2ex] 
$\frac{n!}{(s+a)^{n+1}}$&$t^n e^{-at}$ \\[2ex] 
$\frac{s}{s^2+\omega^2}$&$\cos \omega t$ \\[2ex] 
$\frac{\omega}{s^2+\omega^2}$&$\sin \omega t$ \\[2ex] 
$\frac{s+a}{(s+a)^2+\omega^2}$&$e^{-at} \cos \omega t$ \\[2ex] 
$\frac{\omega}{(s+a)^2+\omega^2}$&$e^{-at} \sin \omega t$ 
\end{tabular}
\end{table}

%======================
\ابتدا{مشق}
تفاعل \عددی{\cosh \omega t} کا لاپلاس بدل حاصل کریں۔

جواب:\عددی{\bF(s)=\tfrac{s}{s^2-\omega^2}}
\انتہا{مشق}
%================================
\ابتدا{مشق}
تفاعل \عددی{\sinh \omega t} کا لاپلاس بدل حاصل کریں۔

جواب:\عددی{\bF(s)=\tfrac{\omega}{s^2-\omega^2}}
\انتہا{مشق}
%=========================

\حصہ{خواص البدل}
لاپلاس بدل کے کئی مسئلوں پر اس حصے میں غور کیا جائے گا۔یہ مسئلے لاپلاس بدل کے خصوصیات بیان کرتے ہیں اور ان کی مدد سے لاپلاس بدل کا حصول نہایت عمدگی کے ساتھ ممکن ہوتا ہے۔

\حصہء{متناسب وقت}
\اصطلاح{مسئلہ متناسب وقت}\فرہنگ{مسئلہ!متناسب وقت}\فرہنگ{متناسب وقت!مسئلہ}\حاشیہب{time-scaling theorem}\فرہنگ{theorem!time-scaling}\فرہنگ{time-scaling theorem} کہتا ہے کہ
\begin{align}
\Laplace[f(at)]=\frac{1}{a} \bF\left(\frac{s}{a}\right) \quad a>0
\end{align}
آئیں اس نتیجے کو لاپلاس تکمل کے ذریعہ حاصل کریں۔
\begin{align*}
\Laplace[f(at)]&=\int_0^{\infty} f(at) e^{-st} \dif t
\end{align*}
اس میں \عددی{\lambda=at} لیتے ہوئے \عددی{\dif \lambda =a\dif t} لکھا جا سکتا ہے۔
\begin{align*}
\Laplace[f(at)]&=\int_0^{\infty} f(\lambda) e^{-\left(\frac{\lambda}{a}\right)s} \frac{\dif \lambda}{a}\\
&=\frac{1}{a}\int_0^{\infty} f(\lambda) e^{-\left(\frac{s}{a}\right)\lambda } \dif \lambda\\
&=\frac{1}{a} \bF\left(\frac{s}{a}\right) \quad a>0
\end{align*}

\حصہء{منقلی وقت}
\اصطلاح{مسئلہ منقلی وقت}\فرہنگ{مسئلہ!منقلی وقت}\فرہنگ{منقلی وقت!مسئلہ}\حاشیہب{time-shifting theorem}\فرہنگ{time-shifting theorem}\فرہنگ{theorem!time-shifting} کہتا ہے کہ
\begin{align}\label{مساوات_لاپلاس_مسئلہ_منتقلی_وقت}
\Laplace[f(t-t_0)u(t-t_0)]=e^{-t_0 s}\bF(s) \quad t_0\ge 0
\end{align}
آئیں مسئلہ منقولہ وقت کو لاپلاس تکمل سے حاصل کریں۔
\begin{align*}
\Laplace[f(t-t_0)u(t-t_0)]&=\int_{0}^{\infty} f(t-t_0) u(t-t_0)e^{-st} \dif t\\
&=\int_{t_0}^{\infty} f(t-t_0)e^{-st} \dif t\\
\end{align*}
اب اگر ہم \عددی{\lambda=t-t_0} لیں تو \عددی{\dif \lambda=\dif t} ہو گا
\begin{align*}
\Laplace[f(t-t_0)u(t-t_0)]&=\int_{0}^{\infty} f(\lambda)e^{-s\left(\lambda+t_0\right)} \dif \lambda\\
&=e^{-t_0 s}\int_0^{\infty}f(\lambda)e^{-s\lambda}\dif \lambda\\
&=e^{-t_0 s}\bF(s)
\end{align*}

\حصہء{منتقلی تعدد}
\اصطلاح{مسئلہ منتقلی تعدد}\فرہنگ{مسئلہ!منتقلی تعدد}\فرہنگ{منتقلی تعدد!مسئلہ}\حاشیہب{frequency-shifting theorem}\فرہنگ{theorem!frequency-shifting} کہتا ہے کہ
\begin{align}
\Laplace[e^{-at} f(t)]=\bF(s+a)
\end{align}
یعنی تفاعل کو \عددی{e^{-at}} سے ضرب دینے سے لاپلاس بدل کی تعدد تبدیل ہو کر \عددی{s+a} ہو جاتی ہے۔ اس مسئلے کو \اصطلاح{مسئلہ ترمیم تعدد}\فرہنگ{مسئلہ!ترمیم تعدد}\فرہنگ{ترمیم تعدد!مسئلہ}\حاشیہب{frequency modulation theorem}\فرہنگ{frequency modulation theorem}\فرہنگ{theorem!frequency modulation} بھی کہتے ہیں۔

جدول \حوالہ{جدول_لاپلاس_مسئلے} میں کئی مسئلے درج کئے گئے ہیں۔ آئیں ان کا استعمال دیکھیں۔
\begin{table}
\caption{لاپلاس بدل کے مسئلے۔}
\label{جدول_لاپلاس_مسئلے}
\centering
\begin{tabular}{l l r}
 $\bF(s)$&$f(t)$ & مسئلہ\\  [2ex] 
\hline
 $\bF_1(s)+\bF_2(s)$&$f_1(t)+f_2(2)$ & جمع و منفی\\[2ex] 
 $A\bF(s)$&$Af(t)$& متناسب مقدار\\[2ex] 
$\frac{1}{a}\bF(\frac{s}{a}),\, a>0$ & $ f(at)$ & متناسب وقت\\[2ex] 
 $e^{-t_0 s}\bF(s)$&$f(t-t_0) u(t-t_0), \, t_0>0$  & منتقلی وقت\\[2ex] 
 $e^{-t_0 s}\Laplace [f(t+t_0)]$&$f(t) u(t-t_0), \, t_0>0$  & منتقلی وقت\\[2ex] 
$\bF(s+a)$& $e^{-at} f(t)$  & منتقلی تعدد\\[2ex] 
$-\frac{\dif \bF(s)}{\dif s}$ & $t f(t)$  & وقت سے ضرب\\[2ex] 
$(-1)^n\frac{\dif{^n} \bF(s)}{\dif s^n}$ & $t^n f(t)$  & وقت سے ضرب\\[2ex] 
$\int_s^{\infty} \bF(\lambda) \dif \lambda $ & $\frac{f(t)}{t}$  & وقت سے تقسیم\\[2ex] 
$s^n \bF(s)-s^{n-1}f(0)-s^{n-2} f^1(0)-\cdots -s^0 f^{n-1}(0)$ & $\frac{\dif{^n} f(t)}{\dif t^n}$  & تفرق\\[2ex] 
$\frac{\bF(s)}{s}$ & $\int_0^t f(\lambda) \dif \lambda$  & تکمل\\[2ex] 
$\bF_1(s) \bF_2(s)$ & $\int_0^t f_1(\lambda) f_2(t-\lambda) \dif \lambda$  & الجھاو
\end{tabular}
\end{table}
%===============
\ابتدا{مثال}
تفاعل \عددی{\sin \omega t} کا لاپلاس بدل \عددی{\tfrac{\omega}{s^2+\omega^2}} ہے۔جدول \حوالہ{جدول_لاپلاس_مسئلے} میں مسئلہ منتقلی تعدد کی مدد سے \عددی{e^{-at}\sin \omega t} کا بدل دریافت کریں۔

حل:مسئلہ منتقلی تعدد کے تحت لاپلاس بدل میں \عددی{s} کی جگہ \عددی{s+a}  لکھا جائے گا لہٰذا جواب درج ذیل ہو گا۔
\begin{align*}
\Laplace[e^{-at} \sin \omega t]=\frac{\omega}{(s+a)^2+\omega^2}
\end{align*}
\انتہا{مثال}
%=========================
\ابتدا{مثال}
تفاعل \عددی{e^{-at}} کا لاپلاس بدل \عددی{\bF(s)=\tfrac{1}{s+a}} ہے۔مسئلہ ضرب وقت کی مدد سے \عددی{te^{-at}} کا لاپلاس بدل دریافت کریں۔

حل:مسئلہ ضرب وقت کے  تحت
\begin{align*}
\Laplace[te^{-at}] &=-\frac{\dif \bF(s)}{\dif s}\\
&=\frac{1}{(s+a)^2}
\end{align*}
ہو گا۔
\انتہا{مثال}
%=========================
تفاعل \عددی{f(t)=1} کا لاپلاس بدل \عددی{\bF(s)=\tfrac{1}{s}} ہے۔مسئلہ ضرب وقت کی مدد سے تفاعل \عددی{t} کا  لاپلاس بدل حاصل کریں۔

حل:مسئلہ ضرب وقت کے تحت جواب درج ذیل ہو گا۔
\begin{align*}
\Laplace[t]&=-\frac{\dif \bF(s)}{\dif s}\\
&=\frac{1}{s^2}
\end{align*}
%======================
\ابتدا{مشق}
تفاعل \عددی{\sin \omega t} کا لاپلاس بدل \عددی{\bF(s)=\tfrac{\omega}{s^2+\omega^2}} ہے۔مسئلہ ضرب وقت کی مدد سے تفاعل \عددی{t\sin \omega t} کا لاپلاس بدل حاصل کریں۔

جواب:\عددی{\tfrac{2\omega s}{(s^2+\omega^2)^2}}
\انتہا{مشق}
%============================
\ابتدا{مشق}
جدول \حوالہ{جدول_لاپلاس_بدل_جوڑیاں} سے \عددی{t} اور \عددی{e^{-2t}} کے لاپلاس بدل دیکھتے ہوئے جدول  \حوالہ{جدول_لاپلاس_مسئلے} کی مدد سے \عددی{t^2(t+e^{-2t})} کا لاپلاس بدل حاصل کریں۔

جواب:\عددی{\tfrac{6}{s^4}+\tfrac{2}{(s+2)^3}}
\انتہا{مشق}
%=============================
\ابتدا{مشق}
جدول \حوالہ{جدول_لاپلاس_بدل_جوڑیاں} سے \عددی{\sin \omega t} کا بدل دیکھتے ہوئے جدول  \حوالہ{جدول_لاپلاس_مسئلے} میں دئے مسئلہ تفرق کی مدد سے \عددی{\cos \omega t} کا لاپلاس بدل دریافت کریں۔

جواب:\عددی{\tfrac{s}{s^2+\omega^2}}
\انتہا{مشق}
%==========================

\حصہ{الٹ لاپلاس بدل کا حصول}
برقی ادوار حل کرتے ہوئے ہمیں جن لاپلاس بدل سے واسطہ پڑتا ہے انہیں دو کثیر رکنی کے کسر کی صورت میں لکھا جا سکتا ہے۔
\begin{align}\label{مساوات_لاپلاس_عمومی_الف}
\bF(s)=\frac{\bP(s)}{\bQ(s)}=\frac{a_m s^m+a_{m-1} s^{m-1}+\cdots+a_1 s+a_0}{b_n s^n+b_{n-1} s^{n-1}+\cdots+b_1 s+b_0}
\end{align}
شمار کنندہ \عددی{\bP(s)} کے جذر \عددی{-z_1} تا \عددی{-z_m}  کو تفاعل کے صفر کہتے ہیں جبکہ نسب نما \عددی{\bQ(s)} کے جذر  \عددی{-p_1} تا \عددی{-p_n} کو تفاعل کے قطب کہتے ہیں۔اگر \عددی{n\le m} ہو تب \عددی{\bP(s)} کو \عددی{\bQ(s)} سے تقسیم کرتے ہوئے
\begin{align}\label{مساوات_لاپلاس_عمومی_ب}
\bF(s)=\frac{\bP(s)}{\bQ(s)}=C_{m-n}s^{m-n}+\cdots+C_2 s^2+C_1 s+C_0+\frac{\bP_1(s)}{\bQ(s)}
\end{align}
لکھا جا سکتا ہے۔ہم \عددی{\tfrac{\bP_1(s)}{\bQ(s)}} کی \اصطلاح{جزوی کسری پھیلاو}\فرہنگ{جزوی کسری پھیلاو}\حاشیہب{partial fraction expansion}\فرہنگ{partial fraction expansion} کرنا چاہتے ہیں۔ایسا کرنے کی خاطر نسب نما \عددی{\bQ(s)} کے جذر پر غور کرنا ہو گا۔

\جزوحصہ{جزوی کسری پھیلاو}
\begin{itemize}
\item
اگر \عددی{\bQ(s)} کے جذر سادہ ہوں تب \عددی{\tfrac{\bP_1(s)}{\bQ(s)}} کو درج ذیل جزوی کسری صورت میں لکھا جا سکتا ہے۔
\begin{align}
\frac{\bP_1(s)}{\bQ(s)}=\frac{K_1}{s+p_1}+\frac{K_2}{s+p_2}+\cdots+\frac{K_n}{s+p_n}
\end{align}
\item
اگر \عددی{\bQ(s)} کے جذر میں مخلوط اعداد پائے جاتے ہوں تو یہ جوڑی دار مخلوط اعداد کی صورت میں ہوں گے۔ یوں ہر جوڑی کے لئے درج ذیل لکھنا ممکن ہو گا جہاں \عددی{K_1} اور \عددی{K^*_1} آپس میں جوڑی دار مخلوط اعداد ہیں۔
\begin{align}
\frac{\bP_1(s)}{\bQ_1(s) (s+\alpha-j\beta)(s+\alpha+j\beta)}=\frac{K_1}{s+\alpha-j\beta}+\frac{K^*_1}{s+\alpha+j\beta}+\cdots
\end{align}
\item
اگر \عددی{\bQ(s)} کے جذر میں \اصطلاح{ہم قطب}\فرہنگ{ہم قطب}\فرہنگ{قطب!ہم} \عددی{r} گنا پایا جاتا ہو تب اس قطب کی جزوی کسری پھیلاو درج ذیل ہو گی۔
  \begin{align}
\frac{\bP_1(s)}{\bQ_1(s) (s+p_1)^r}=\frac{K_{1}}{(s+p_1)}+\frac{K_{2}}{(s+p_1)^2}+\cdots +\frac{K_{r}}{(s+p_1)^r}+\cdots
\end{align}
\end{itemize}

لاپلاس بدل \عددی{\bF(s)} کی جزوی کسری پھیلاو کے بعد علیحدہ علیحدہ کسر کا الٹ لاپلاس بدل جدول سے پڑھا جا سکتا ہے۔تمام کسروں کے الٹ لاپلاس بدل کا مجموعہ \عددی{\bF(s)} کا الٹ لاپلاس بدل ہو گا۔

\حصہء{سادہ قطبین}
سادہ قطبین کی صورت میں لاپلاس بدل \عددی{\bF(s)} کا جزوی کسری پھیلاو درج ذیل ہے۔
\begin{align*}
\bF(s)=\frac{\bP}{\bQ}=\frac{\bP}{(s+p_1)(s+p_2)\cdots(s+p_n)}=\frac{K_1}{s+p_1}+\frac{K_2}{s+p_2}+\cdots +\frac{K_n}{s+p_n}
\end{align*}
مساوات کو \عددی{(s+p_i)} سے ضرب دیتے ہوئے
\begin{align*}
(s+p_1)\frac{\bP}{\bQ}=\frac{\bP}{(s+p_2)\cdots(s+p_n)}=K_1+\frac{(s+p_1)K_2}{s+p_2}+\cdots +\frac{(s+p_1)K_n}{s+p_n}
\end{align*}
اس میں \عددی{s=-p_1} پر کرنے سے
\begin{align*}
\left. (s+p_1)\frac{\bP}{\bQ}\right|_{s=-p_1}&=\frac{\bP}{(-p_1+p_2)\cdots(-p_1+p_n)}\\
&=K_1+\frac{(-p_1+p_1)K_2}{s+p_2}+\cdots +\frac{(-p_1+p_1)K_n}{s+p_n}
\end{align*}
یعنی
\begin{align*}
\left. (s+p_1)\frac{\bP}{\bQ}\right|_{s=-p_1}=\frac{\bP}{(-p_1+p_2)\cdots(-p_1+p_n)}=K_1
\end{align*}
حاصل ہوتا ہے۔اسی طرح جزوی کسری پھیلاو کے بقایا مستقل درج ذیل مساوات سے حاصل کئے جا سکتے ہیں۔
\begin{align}
K_i=\left. (s+p_i)\frac{\bP}{\bQ}\right|_{s=-p_i}=\left. (s+p_i) \bF \right|_{s=-p_i}
\end{align}
تمام \عددی{K_i} جانتے ہوئے الٹ لاپلاس بدل درج ذیل لکھا جا سکتا ہے۔
\begin{align}
\Laplace^{-1} \bF(s)=f(t)=\left(K_1 e^{-p_1 t}+K_2 e^{-p_2 t}+\cdots+K_n e^{-p_n t}\right) u(t)
\end{align}
%===================

\ابتدا{مثال}
لاپلاس تفاعل \عددی{\bF(s)=\tfrac{10(s+2)}{(s+4)(s+6)}} کے جزوی کسری پھیلاو حاصل کرتے ہوئے الٹ لاپلاس تفاعل \عددی{f(t)} دریافت کریں۔ 

حل:نسب نما کے قطبین سادہ ہیں لہٰذا درج ذیل لکھا جا سکتا ہے۔
\begin{align}\label{مساوات_لاپلاس_سادہ_قطبین_الف}
\bF(s)=\frac{10(s+2)}{(s+4)(s+6)}=\frac{K_1}{s+4}+\frac{K_2}{s+6}
\end{align}
مستقل \عددی{K_1} حاصل کرنے کی خاطر دونوں اطراف کو \عددی{(s+4)} سے ضرب دیتے ہوئے
\begin{align*}
\frac{10(s+2)}{(s+6)}=K_1+\frac{K_2(s+4)}{s+6}
\end{align*}
دونوں اطراف میں \عددی{s=-4} پر کرتے  
\begin{align*}
\frac{10(-4+2)}{(-4+6)}&=K_1+\frac{K_2(-4+4)}{s+6}
\end{align*}
ہوئے \عددی{K_1} کے لئے حل کرتے ہیں۔
\begin{align*}
K_1=-10
\end{align*}
یہی طریقہ کار \عددی{K_2} کے لئے بھی استعمال کرتے ہوئے مساوات \حوالہ{مساوات_لاپلاس_سادہ_قطبین_الف} کو \عددی{(s+6)} سے ضرب دیتے ہوئے
\begin{align*}
\frac{10(s+2)}{(s+4)}=\frac{K_1(s+6)}{s+4}+K_2
\end{align*}
دونوں اطراف میں \عددی{s=-6} پر کرتے ہیں۔

\begin{align*}
\frac{10(-6+2)}{(-6+4)}=\frac{K_1(-6+6)}{-6+4}+K_2
\end{align*}
یوں \عددی{K_2} حاصل ہوتا ہے۔
\begin{align*}
K_2=20
\end{align*}
اس طرح مساوات \حوالہ{مساوات_لاپلاس_سادہ_قطبین_الف} کے تفاعل کو درج ذیل لکھا جا سکتا ہے
\begin{align*}
\bF(s)=-\frac{10}{s+4}+\frac{20}{s+6}
\end{align*}
جس کا الٹ لاپلاس لیتے ہوئے وقتی دائرہ کار میں تفاعل لکھتے ہیں۔
\begin{align*}
\Laplace^{-1} \bF(s)=f(t)=\left(-10e^{-4t}+20e^{-6t}\right)u(t)
\end{align*}
\انتہا{مثال}
%========================
\ابتدا{مشق}
تفاعل \عددی{\bF(s)=\tfrac{5(s+6)}{(s+3)(s+5)}} دیا گیا ہے۔اس کا الٹ لاپلاس تفاعل حاصل کریں۔

جواب:\عددی{f(t)=(\frac{15}{2}e^{-3t}-\frac{5}{2}e^{-5t})u(t)}
\انتہا{مشق}
%========================

\ابتدا{مشق}
تفاعل \عددی{\bF(s)=\tfrac{(s^2+5s+1)}{s(s+2)(s+3)}} دیا گیا ہے۔اس کا الٹ لاپلاس تفاعل حاصل کریں۔

جواب:\عددی{f(t)=(\frac{1}{6}+\frac{7}{2}e^{-2t}-\frac{8}{3}e^{-3t})u(t)}
\انتہا{مشق}
%=============================

\حصہء{جوڑی دار مخلوط قطبین}
فرض کریں کہ \عددی{\bF(s)} میں جوڑی دار مخلوط قطبین کی ایک جوڑی پائی جاتی ہے۔ایسی صورت میں \عددی{\bF(s)} کی جزوی کسری پھیلاو کو درج ذیل لکھا جا سکتا ہے
\begin{align*}
\bF(s)=\frac{\bP}{\bQ_1 (s+\alpha-j\beta)(s+\alpha+j\beta)}=\frac{K_1}{s+\alpha-j\beta}+\frac{K^*_1}{s+\alpha+j\beta}+\cdots
\end{align*}
جہاں سادہ قطبین کی طرح \عددی{K_1} کو درج ذیل لکھا جا سکتا ہے۔
\begin{align}
\left. (s+\alpha-j\beta)\bF\right|_{s=-\alpha+j\beta}=K_1
\end{align}
مستقل \عددی{K^*_1} کو بھی اسی طرح حاصل کیا جا سکتا ہے البتہ ایسا کرنے کی ضرورت نہیں ہے چونکہ دونوں مستقل آپس میں جوڑی دار مخلوط اعداد ہیں۔ اس طرح اگر \عددی{K_1=K\phase{\theta}} ہو تو \عددی{K^*_1=K\phase{-\theta}} ہو گا اور \عددی{\bF(s)} کا جزوی کسری پھیلاو درج ذیل لکھا جائے گا۔
\begin{align*}
\bF(s)&=\frac{K\phase{\theta}}{s+\alpha-j\beta}+\frac{K\phase{-\theta}}{s+\alpha+j\beta}+\cdots\\
&=\frac{Ke^{j\theta}}{s+\alpha-j\beta}+\frac{Ke^{-j\theta}}{s+\alpha+j\beta}+\cdots
\end{align*}
یوں وقتی دائرہ کار میں تفاعل درج ذیل ہو گا
\begin{align*}
f(t)&=\Laplace^{-1}\bF(s)=Ke^{j\theta}e^{-(\alpha-j\beta)t}+Ke^{-j\theta}e^{-(\alpha+j\beta)t}+\cdots\\
&=Ke^{-\alpha t}\left(e^{j(\theta+\beta t)}+e^{-j(\theta+\beta t)}\right)+\cdots\\
&=2Ke^{-\alpha t}\cos(\beta t+\theta)+\cdots
\end{align*}
جہاں آخری قدم پر \عددی{\tfrac{e^{+jx}+e^{-jx}}{2}=\cos x} کا استعمال کیا گیا ہے۔
%===============

\ابتدا{مثال}
درج ذیل لاپلاس تفاعل کا الٹ لاپلاس بدل حاصل کریں۔
\begin{align*}
\bF(s)=\frac{10(s+4)}{s(s^2+2s+2)}
\end{align*}

حل:اس تفاعل کے نسب نما میں \عددی{s^2+2s+2} کے جذر \عددی{-1+j} اور \عددی{-1-j} ہیں لہٰذا تفاعل کو درج ذیل لکھا جا سکتا ہے۔
\begin{align*}
\bF(s)=\frac{10(s+4)}{s(s+1-j)(s+1+j)}
\end{align*}
اس کی جزوی کسری پھیلاو لکھتے ہیں۔
\begin{align*}
\bF(s)=\frac{K_1}{s}+\frac{K_2}{s+1-j}+\frac{K^*_2}{s+1+j}
\end{align*}
مساوات کے مستقل حاصل کرتے ہیں۔پہلے \عددی{K_1} حاصل کرتے ہیں۔
\begin{align*}
K_1&=\left. \frac{10(s+4)}{(s+1-j)(s+1+j)} \right|_{s=0}\\
&=\frac{10(0+4)}{(0+1-j)(0+1+j)} \\
&=20
\end{align*}
اسی طرح \عددی{K_2} حاصل کرتے ہیں۔
\begin{align*}
K_2&=\left. \frac{10(s+4)}{s(s+1+j)} \right|_{s=-1+j}\\
&=\frac{10(-1+j+4)}{(-1+j)(-1+j+1+j)} \\
&=-10+j5
\end{align*}
ہم جانتے ہیں کہ \عددی{K^*_2} درج بالا کا جوڑی دار مخلوط عدد یعنی \عددی{K^*_2=-10-j5} ہے۔ اس کے باوجود ہم اس کو حل کرتے ہوئے حاصل کرتے ہیں۔
 \begin{align*}
K^*_2&=\left. \frac{10(s+4)}{s(s+1-j)} \right|_{s=-1-j}\\
&=\frac{10(-1-j+4)}{(-1-j)(-1-j+1-j)} \\
&=-10-j5
\end{align*}
ان مستقل کو استعمال کرتے ہوئے  \عددی{\bF(s)} کا جزوی کسری پھیلاو  لکھتے ہیں۔
\begin{align*}
\bF(s)=\frac{20}{s}+\frac{(-10+j5)}{s+1-j}+\frac{(-10-j5)}{s+1+j}
\end{align*}
الٹ لاپلاس بدل لکھتے ہیں۔
\begin{align*}
f(t)&=\Laplace^{-1} \bF(s)=\left[20+(-10+j5)e^{-(1-j)t}+(-10-j5)e^{-(1+j)t}\right]u(t)\\
&=\left[20-10e^{-t} \left(e^{jt}+e^{-jt}\right)+j5e^{-t}\left(e^{jt}-e^{-jt}\right)\right]u(t)\\
&=\left(20-20e^{-t}\cos t-10e^{-t}\sin t\right)u(t)\\
&=\left[20-10\sqrt{5}e^{-t}\cos(t+26.56^{\circ})\right]u(t)
\end{align*}

آئیں حاصل جواب کا لاپلاس بدل حاصل کرتے ہوئے ثابت کریں کہ ہمارا جواب درست ہے۔ہم درج ذیل
\begin{align*}
f(t)=\left(20-20e^{-t}\cos t-10e^{-t}\sin t\right)u(t)
\end{align*}
 کا لاپلاس بدل  جدول \حوالہ{جدول_لاپلاس_بدل_جوڑیاں} کی مدد سے لکھتے ہوئے ترتیب دیتے ہیں۔
\begin{align*}
\bF(s)&=\frac{20}{s}-\frac{s+1}{(s+1)^2+1^2}-\frac{10}{(s+1)^2+1^1}\\
&=\frac{10(s+4)}{s(s^2+2s+2)}
\end{align*}
اصل لاپلاس بدل حاصل کرتے ہوئے ہم نے ثابت کیا کہ ہم نے صحیح وقتی تفاعل حاصل کیا ہے۔ 
\انتہا{مثال}
%===================
\ابتدا{مشق}
لاپلاس تفاعل \عددی{\bF(s)=\tfrac{2s+3}{s^2+6s+34}} دیا ہوا ہے۔ اس کا الٹ لاپلاس تفاعل \عددی{f(t)} حاصل کریں۔

جواب:\عددی{f(t)=e^{-3t}\left(2\cos 5t-\tfrac{3}{5}\sin 5t\right)u(t)}
\انتہا{مشق}
%=====================
\ابتدا{مشق}
تفاعل \عددی{\bF(s)=\tfrac{5(s+2)}{(s+3)(s^2+2s+5)}} کا الٹ لاپلاس تفاعل دریافت کریں۔

جواب:\عددی{f(t)=\left[-\tfrac{5}{8}e^{-3t}+\tfrac{1}{8}e^{-t}\left(5\cos 2t +15\sin 2t\right)\right]u(t)}
\انتہا{مشق}
%===================

\حصہء{کثیر ہم قطبین}
فرض کریں کہ \عددی{\bF(s)} میں \عددی{-p_1} قطب \عددی{r} مرتبہ پایا جاتا ہے۔ایسی صورت میں تفاعل کا جزوی کسری پھیلاو درج ذیل ہو گا۔
\begin{align*}
\bF(s)&=\frac{\bP}{\bQ_1 (s+p_1)^r}\\
&=\frac{K_r}{(s+p_1)^r}+\frac{K_{r-1}}{(s+p_1)^{r-1}}+\frac{K_{r-2}}{(s+p_1)^{r-2}}+\frac{K_{r-3}}{(s+p_1)^{r-3}}+\cdots \\
&\hspace{4cm}+\frac{K_3}{(s+p_1)^3}+\frac{K_2}{(s+p_1)^2}+\frac{K_1}{(s+p_1)^1}+\cdots
\end{align*}
مساوات کے دونوں اطراف کو \عددی{(s+p_1)^r} سے ضرب دیتے ہیں۔
\begin{gather}
\begin{aligned}\label{مساوات_لاپلاس_کثیر_قطبین_الف}
\frac{\bP}{\bQ_1}&=K_r+K_{r-1}(s+p_1)^{1}+K_{r-2}(s+p_1)^{2}+K_{r-3}(s+p_1)^{3}+\cdots \\
&\hspace{1cm}+K_3(s+p_1)^{r-3}+K_2(s+p_1)^{r-2}+K_1(s+p_1)^{r-1}+\cdots
\end{aligned}
\end{gather}
درج بالا مساوات میں \عددی{s=-p_1} پر کرنے سے \عددی{K_r} حاصل ہوتا ہے۔
\begin{align}
K_r=\left. \frac{\bP}{\bQ_1}\right|_{s=-p_1}=\left. (s+p_1)^r \bF\right|_{s=-p_1}
\end{align}
مساوات \حوالہ{مساوات_لاپلاس_کثیر_قطبین_الف} کا ایک مرتبہ تفرق لیتے ہوئے۔
\begin{gather}
\begin{aligned}\label{مساوات_لاپلاس_کثیر_قطبین_ب}
\frac{\dif }{\dif s}\left[\frac{\bP}{\bQ_1}\right]&=K_{r-1}+2K_{r-2}(s+p_1)^{1}+3K_{r-3}(s+p_1)^{2}+\cdots +(r-3)K_3(s+p_1)^{r-4}\\
&\hspace{1cm}+(r-2)K_2(s+p_1)^{r-3}+(r-1)K_1(s+p_1)^{r-2}+\cdots
\end{aligned}
\end{gather}
حاصل جواب میں \عددی{s=-p_1} پر کرنے سے \عددی{K_{r-1}} حاصل ہوتا ہے۔
\begin{align}
K_{r-1}=\left. \frac{1}{1!}\frac{\dif}{\dif s}\left[(s+p_1)^r \bF\right]\right|_{s=-p_1}
\end{align}
اسی طرح مساوات \حوالہ{مساوات_لاپلاس_کثیر_قطبین_الف} کا دو مرتبہ تفرق لے کر اس میں \عددی{s=-p_1} پر کرنے سے \عددی{K_{r-2}} حاصل ہو گا۔
 \begin{align}
K_{r-2}=\left. \frac{1}{2!}\frac{\dif^{\,2}}{\dif s^2}\left[(s+p_1)^r \bF\right]\right|_{s=-p_1}
\end{align}
یوں مستقل حاصل کرنے کی عمومی مساوات درج ذیل ہے۔
  \begin{align}
K_{r-m}=\left. \frac{1}{m!}\frac{\dif^{\,m}}{\dif s^m}\left[(s+p_1)^r \bF\right]\right|_{s=-p_1}
\end{align}
%===============

\ابتدا{مثال}
لاپلاس بدل \عددی{\bF(s)=\tfrac{s+1}{(s+2)^3(s+3)}} سے وقتی تفاعل حاصل کریں۔

حل:دیے گئے تفاعل کا جزوی کسری پھیلاو لکھتے ہیں۔
\begin{align}\label{مساوات_لاپلاس_مثال_کثیر_قطبین_الف}
\frac{s+1}{(s+2)^3(s+3)}=\frac{K_0}{s+3}+\frac{K_1}{(s+2)}+\frac{K_2}{(s+2)^2}+\frac{K_3}{(s+2)^3}
\end{align}
مستقل \عددی{K_0} حاصل کرنے کی خاطر مساوات کے دونوں اطراف کو \عددی{(s+3)} سے ضرب دیتے ہوئے
\begin{align*}
\frac{s+1}{(s+2)^3}=K_0+\frac{(s+3)K_1}{(s+2)}+\frac{(s+3)K_2}{(s+2)^2}+\frac{(s+3)K_3}{(s+2)^3}
\end{align*}
دونوں اطراف میں \عددی{s=-3} پر کرتے ہوئے
\begin{align*}
\frac{-3+1}{(-3+2)^3}&=K_0+\frac{(-3+3)K_1}{(-3+2)}+\frac{(-3+3)K_2}{(-3+2)^2}+\frac{(-3+3)K_3}{(-3+2)^3}
\end{align*}
 بائیں ہاتھ تین قوسین صفر کے برابر ہو جاتے ہیں لہٰذا
\begin{align*}
K_0=2
\end{align*}
حاصل ہوتا ہے۔مستقل \عددی{K_3} حاصل کرنے کی خاطر مساوات \حوالہ{مساوات_لاپلاس_مثال_کثیر_قطبین_الف} کے دونوں اطراف کو \عددی{(s+2)^3} سے ضرب دیتے ہیں۔
\begin{align}\label{مساوات_لاپلاس_مثال_کثیر_قطبین_ب}
\frac{s+1}{s+3}=\frac{(s+2)^3}{s+3}K_0+(s+2)^2 K_1+(s+2)K_2+K_3
\end{align}
اس میں \عددی{s=-2} پر کرنے سے \عددی{K_3} حاصل ہوتا ہے۔
\begin{align*}
K_3=-1
\end{align*}
مساوات \حوالہ{مساوات_لاپلاس_مثال_کثیر_قطبین_ب} کا ایک مرتبہ تفرق لینے کے بعد \عددی{s=-2} پر کرنے سے \عددی{K_2} حاصل ہو گا۔چونکہ ایسا کرتے ہوئے صرف \عددی{K_2} کا جزو باقی رہتا ہے لہٰذا بائیں ہاتھ بقایا اجزاء کا تفرق  لینے کی ضرورت نہیں ہے۔مساوات کے بائیں ہاتھ کا ایک درجی تفرق \عددی{\tfrac{2}{(s+3)^2}} ہے۔
\begin{align*}
K_2=\left. \frac{2}{(s+3)^2} \right|_{s=-2}=2
\end{align*}
مساوات \حوالہ{مساوات_لاپلاس_مثال_کثیر_قطبین_ب} کا دو درجی تفرق لینے کے بعد دونوں اطراف \عددی{s=-2} پر کرنے سے \عددی{K_1} حاصل ہوتا ہے۔بائیں ہاتھ کا دو درجی تفرق \عددی{\tfrac{-4}{(s+3)^2}} ہے جبکہ دائیں جانب \عددی{K_1} والے جزو کا دو درجی تفرق \عددی{2K_1} کے برابر ہے۔ 
\begin{align*}
2K_1=\left. -\frac{4}{(s+3)^3} \right|_{s=-2}
\end{align*}
اس سے درج ذیل حاصل ہوتا ہے۔
\begin{align*}
K_1=-2
\end{align*}
تمام مستقل جاننے کے بعد مساوات \حوالہ{مساوات_لاپلاس_مثال_کثیر_قطبین_الف} کو دوبارہ لکھتے ہیں۔
\begin{align*}
\bF(s)=\frac{2}{s+3}-\frac{2}{(s+2)}+\frac{2}{(s+2)^2}-\frac{1}{(s+2)^3}
\end{align*}
جدول \حوالہ{جدول_لاپلاس_بدل_جوڑیاں} کی مدد سے تمام اجزاء کے الٹ بدل لکھتے ہیں۔
\begin{align*}
f(t)=\Laplace^{-1}\bF(s)=\left(2e^{-3t}-2e^{-2t}+2te^{-2t}-\frac{t^2 e^{-2t}}{2}\right)u(t)
\end{align*}
\انتہا{مثال}
%==================
\ابتدا{مشق}
لاپلاس بدل \عددی{\bF(s)=\tfrac{s+2}{(s+3)^2}} کا الٹ لاپلاس بدل حاصل کریں۔

جواب:\عددی{f(t)=\left(e^{-3t}-te^{-3t}\right)u(t)}
\انتہا{مشق}
%=================
\ابتدا{مشق}
لاپلاس بدل \عددی{\bF(s)=\tfrac{s+1}{s^2(s+2)}} کا الٹ لاپلاس بدل حاصل کریں۔

جواب:\عددی{f(t)=\tfrac{1}{4}\left(1+2t-e^{-2t}\right)u(t)}
\انتہا{مشق}
%=================

\ابتدا{مشق}
لاپلاس بدل \عددی{\bF(s)=\tfrac{80}{s^2(s+2)^3}} کا الٹ لاپلاس بدل حاصل کریں۔

جواب:\عددی{f(t)=\left(10t^2e^{-2t}+20te^{-2t}+15e^{-2t}+10t-15\right)u(t)}
\انتہا{مشق}
%=================

باب \حوالہ{باب_عارضی_رد_عمل} میں دور کی امتیازی مساوات کی بات کی گئی۔آپ جلد دیکھیں گے کہ \عددی{\bQ(s)} دور کی امتیازی مساوات ہے۔ہم نے اس باب میں دیکھا کہ \عددی{\bF(s)} سے حاصل وقتی تفاعل کے اجزاء کا دارومدار \عددی{\bQ(s)} کے قطبین پر ہے۔سادہ قطب \عددی{\tfrac{1}{s+a}} کی صورت میں \عددی{e^{-at}}  حاصل ہو گا جو وقت کے ساتھ گھٹتا ہے۔اس کے برعکس سادہ قطب \عددی{\tfrac{1}{s-b}} کی صورت میں \عددی{e^{bt}} حاصل ہو گا جو وقت کے ساتھ مسلسل بڑھتا ہے۔حقیقت میں وقت کے ساتھ مسلسل بڑھتا دباو یا رو آخر کار دور کو تباہ کر دے گا لہٰذا ادوار تخلیق دیتے ہوئے ایسے قطبین پر کھڑی نظر رکھی جاتی ہے اور ان سے چھٹکارا حاصل کیا جاتا ہے۔ کثیر قطبین کی صورت میں \عددی{te^{-at}}، \عددی{t^2e^{-at}} وغیرہ حاصل ہوتے ہیں جبکہ مخلوط قطبین  \عددی{e^{-at}\cos (\omega t +\theta)} کو جنم دیتے ہیں جو وقت کے ساتھ گھٹتی سائن نما تفاعل ہے۔مخلوط قطبین کا حقیقی جزو صفر ہونے کی صورت میں خیالی قطبین کی جوڑی ملتی ہے جو مسلسل ارتعاش کرتے سائن نما تفاعل \عددی{\cos(\omega t +\theta)} کو جنم دیتے ہیں۔ یوں صرف قطبین کے بارے میں جان لینے سے دور کے بارے میں بہت کچھ کہا جا سکتا ہے۔

ہم نے  کہا کہ مساوات \حوالہ{مساوات_لاپلاس_عمومی_الف} کسی بھی دور کے لاپلاس بدل کو ظاہر کر سکتی ہے۔اگر \عددی{m\ge n} ہو تب اس  کو مساوات \حوالہ{مساوات_لاپلاس_عمومی_ب} کی صورت میں لکھا جا سکتا ہے جس میں مستقل \عددی{C_0} پایا جاتا ہے۔ جدول \حوالہ{جدول_لاپلاس_بدل_جوڑیاں} کے تحت \عددی{\bF(s)=1} کا الٹ لاپلاس بدل  اکائی ضرب تفاعل \عددی{\delta(t)} ہے لہٰذا \عددی{\bF(s)=C_0} کا الٹ لاپلاس بدل \عددی{C_0 \delta(t)} ہو گا۔ہم اس حقیقت پر تبصرہ کر چکے ہیں کہ حقیقی دنیا میں اکائی ضرب تفاعل نہیں پایا جاتا لہٰذا کسی بھی دور کا رد عمل اکائی ضرب تفاعل نہیں ہو سکتا۔اس سے ہم اخذ کر سکتے ہیں کہ کسی بھی حقیقی دور کے لاپلاس بدل میں \عددی{m<n} ہو گا۔

%=========================
\حصہ{تکمل الجھاو}
جدول \حوالہ{جدول_لاپلاس_مسئلے} میں لاپلاس مسئلہ الجھاو بیان کیا گیا ہے جس کے تحت 
\begin{align}\label{مساوات_لاپلاس_الجھاو_تعریف}
f(t)=f_1(t) \ast f_2(t)=\int_0^t f_1(t-\lambda)f_2(\lambda)\dif \lambda=\int_0^t f_1(\lambda)f_2(t-\lambda)\dif \lambda
\end{align}
کا لاپلاس بدل
\begin{align}\label{مساوات_لاپلاس_مسئلہ_الجھاو_تعریف}
\Laplace[f(t)]=\bF_1(s)\bF_2(s)
\end{align}
ہے جہاں
\begin{align*}
\Laplace[f_1(t)]&=\bF_1(s)\\
\Laplace[f_2(t)]&=\bF_2(s)
\end{align*}
 ہیں۔مساوات \حوالہ{مساوات_لاپلاس_الجھاو_تعریف} کو \اصطلاح{تکمل الجھاو}\فرہنگ{تکمل الجھاو}\فرہنگ{الجھاو}\فرہنگ{تکمل!الجھاو}\حاشیہب{convolution integral}\فرہنگ{convolution!integral}\فرہنگ{integral!convolution} کہتے ہیں۔تفاعل کی الجھاو نہایت اہم ہے جو ادوار اور  \اصطلاح{تجزیہ نظام}\فرہنگ{تجزیہ نظام}\حاشیہب{systems analysis}\فرہنگ{systems analysis} میں کلیدی کردار ادا کرتی ہے۔

لاپلاس بدل کی تعریف یعنی مساوات \حوالہ{مساوات_لاپلاس_بدل_تعارف} کو استعمال کرتے ہوئے مساوات \حوالہ{مساوات_لاپلاس_مسئلہ_الجھاو_تعریف} کو ثابت کرتے ہیں۔
\begin{align}\label{مساوات_لاپلاس_الجھاو_الف}
\Laplace[f(t)]&=\int_0^{\infty}\left[\int_0^t f_1(t-\lambda)f_2(\lambda) \dif \lambda\right]e^{-st}\dif t
\end{align}
اندرونی تکمل کے حدود کو صفر تا لامتناہی بنانے کی خاطر اندرونی تکمل کو \عددی{u(t-\lambda)} سے ضرب دیتے ہیں۔
\begin{align*}
u(t-\lambda)=
\begin{cases}
1& \lambda<t\\
0&\lambda>t
\end{cases}
\end{align*}
اندرونی تکمل کے اضافی احاطے یعنی \عددی{t} تا \عددی{\infty} میں چونکہ \عددی{u(t-\lambda)=0} ہے لہٰذا تکمل کی قیمت میں کوئی تبدیلی رونما نہیں ہو گی۔یوں مساوات \حوالہ{مساوات_لاپلاس_الجھاو_الف} کو درج ذیل
\begin{align*}
\Laplace[f(t)]&=\int_0^{\infty}\left[\int_0^{\infty} f_1(t-\lambda)f_2(\lambda) u(t-\lambda) \dif \lambda\right]e^{-st}\dif t
\end{align*}
یعنی
\begin{align*}
\Laplace[f(t)]&=\int_0^{\infty}f_2(\lambda)\left[\int_0^{\infty} f_1(t-\lambda) u(t-\lambda) e^{-st}\dif t\right]\dif \lambda
\end{align*}
لکھا جا سکتا ہے۔قوسین کے اندر تکمل مساوات \حوالہ{مساوات_لاپلاس_مسئلہ_منتقلی_وقت} میں دیا گیا مسئلہ منتقلی وقت ہے لہٰذا درج ذیل لکھا جا سکتا ہے۔
 \begin{align*}
\Laplace[f(t)]&=\int_0^{\infty}f_2(\lambda)\bF_1(s)e^{-s \lambda}\dif \lambda\\
&=\bF_1(s)\int_0^{\infty} f_2(\lambda) e^{-s\lambda} \dif \lambda\\
&=\bF_1(s) \bF_2(s)
\end{align*}
آپ دیکھ سکتے ہیں کہ وقتی دائرہ کار میں الجھاو، تعددی دائرہ کار میں ضرب کے مترادف ہے۔آئیں اس پر ایک مثال دیکھیں۔

%=============
\ابتدا{مثال}

\انتہا{مثال}
%======================= 

\باب{ادوار کا حل بذریعہ لاپلاس بدل}
\حصہ{ادوار کا حل}
لاپلاس بدل کا استعمال دیکھنے کی خاطر شکل \حوالہ{شکل_لاپالس_حل_امالہ_مزاحمت} میں \عددی{RL} دور کو حل کرتے ہوئے \عددی{i(t)} دریافت کرتے ہیں۔دور کی کرخوف مساوات لکھتے ہیں۔
\begin{align*}
v_d(t)=i(t)R+L\frac{\dif i(t)}{\dif t}
\end{align*}
اس دور کے  فطری حل اور جبری حل  کا مجموعہ درکار حل ہو گا۔لاپلاس بدل سے دور حل کرتے ہوئے مکمل حل ایک ہی بار میں حاصل ہوتا ہے۔درج بالا مساوات کے دونوں اطراف کا لاپلاس بدل لیتے ہیں۔
\begin{align*}
\Laplace\left[10 u(t)\right]& =R \Laplace[i(t)]+L\Laplace\left[\frac{\dif i(t)}{\dif t}\right]
\end{align*}   
صفحہ \حوالہصفحہ{جدول_لاپلاس_بدل_جوڑیاں} پر جدول \حوالہ{جدول_لاپلاس_بدل_جوڑیاں} اور صفحہ \حوالہصفحہ{جدول_لاپلاس_مسئلے} پر جدول \حوالہ{جدول_لاپلاس_مسئلے} کی مدد لیتے ہیں۔
\begin{align*}
\frac{10}{s}& =R \bI(s)+L[s\bI(s) -i(0)]
\end{align*} 
چونکہ \عددی{i(0)=\SI{0}{\ampere}} ہے لہٰذا
\begin{align*}
\frac{10}{s}& =R \bI(s)+sL\bI(s)
\end{align*} 
یعنی
\begin{align*}
\bI(s)=\frac{10}{s(sL+R)}
\end{align*}
یا
\begin{align*}
\bI(s)=\frac{10}{R}\left(\frac{1}{s}-\frac{1}{s+\frac{R}{L}}\right)
\end{align*}
حاصل ہوتا ہے جہاں جزوی کسری پھیلاو لکھی گئی ہے۔درج بالا سے وقتی تفاعل  لکھتے ہیں۔
\begin{align*}
i(t)=\frac{10}{R}\left(1-e^{-\frac{R}{L}t}\right)u(t)
\end{align*}
آپ نے دیکھا کہ مکمل حل یک وقت حاصل ہوتا ہے۔دور کی ابتدائی معلومات لاپلاس بدل لیتے وقت استعمال کی جاتی ہے۔

جیسا آپ نے دیکھا، لاپلاس بدل سے تفرقی و تکملی مساوات الجبرائی مساوات میں تبدیل ہو جاتی ہے جس سے درکار تفاعل کا لاپلاس بدل نہایت آسانی سے حاصل ہوتا ہے۔حاصل تفاعل کا الٹ لاپلاس بدل وقتی تفاعل دیتا ہے۔الٹ لاپالاس بدل جدول کی مدد سے حاصل کیا جاتا ہے۔
\begin{figure}
\centering
\begin{circuitikz}
\draw(0,0) to [american voltage source,l={${ v_d(t)=10u(t)}$}]++(0,\y) to [resistor,i^>={$i(t)$},l={$R$}]++(\x,0) to [inductor,l={$L$}]++(0,-\y) to [short] (0,0);
\end{circuitikz}
\caption{سلسلہ وار \عددی{RL} دور۔}
\label{شکل_لاپالس_حل_امالہ_مزاحمت}
\end{figure}

%=================
\حصہ{پرزوں کے مساوی لاپلاسی ادوار}
برقی پرزوں کی خصوصیات سے ان کے مساوی لاپلاسی ادوار حاصل کئے جا سکتے ہیں۔تمام پرزوں کے دباو بالمقابل رو تعلق لکھتے ہوئے انفعالی رائج سمت استعمال کئے گئے ہیں۔مزاحمت کے دباو اور رو کا تعلق
\begin{align}
v(t)=R i(t)
\end{align}
ہے۔دونوں اطراف کا لاپلاس بدل لیتے ہوئے اس تعلق کو درج ذیل لکھا جا سکتا ہے۔
\begin{align}
\bV(s)=R \bI(s)
\end{align}
شکل \حوالہ{شکل_لاپلاس_دور_مزاحمت_اظہار} میں مزاحمت کے دباو بالمقابل کا تعلق وقتی دائرہ کار اور مخلوط تعددی دائرہ کار میں دکھائے گئے ہیں۔
\begin{figure}
\centering
\begin{subfigure}{0.5\textwidth}
\centering
\begin{tikzpicture}
\draw(0,0) to [short,o-]++(\x,0) to [resistor,l_={$R$}]++(0,\y) to [short,-o,i<_={$i(t)$}]++(-\x,0);
\draw(0,0) to [short,o-]++(-\x/4,0)coordinate(kBL);
\draw(0,\y) to [short,o-]++(-\x/4,0);
\draw(kBL)++(0,-0.25) rectangle ++(-0.5,\y+0.5);
\draw(0,\y/2)node{$\begin{aligned}&+ \\&  v(t) \\ &- \end{aligned}$};
\end{tikzpicture}
\caption*{(الف)}
\end{subfigure}%
\begin{subfigure}{0.5\textwidth}
\centering
\begin{tikzpicture}
\draw(0,0) to [short,o-]++(\x,0) to [resistor,l_={$R$}]++(0,\y) to [short,-o,i<_={$\bI(s)$}]++(-\x,0);
\draw(0,0) to [short,o-]++(-\x/4,0)coordinate(kBL);
\draw(0,\y) to [short,o-]++(-\x/4,0);
\draw(kBL)++(0,-0.25) rectangle ++(-0.5,\y+0.5);
\draw(0,\y/2)node{$\begin{aligned}&+ \\&  \bV(s) \\ &- \end{aligned}$};
\end{tikzpicture}
\caption*{(ب)}
\end{subfigure}%
\caption{وقتی اور مخلوط تعددی دائرہ کار میں مزاحمت کا اظہار۔}
\label{شکل_لاپلاس_دور_مزاحمت_اظہار}
\end{figure}

برق گیر کے تعلقات
\begin{align}
v(t)&=\frac{1}{C}\int_0^t i(t) \dif t +v(0)\\
i(t)&=C\frac{\dif v(t)}{\dif t}
\end{align}
ہیں۔دونوں اطراف کا لاپلاس بدل لیتے ہوئے مخلوط تعددی دائرہ کار میں تعلقات حاصل ہوتے ہیں جنہیں شکل \حوالہ{شکل_لاپلاس_دور_برق_گیر_اظہار} میں دکھایا گیا ہے۔ابتدائی معومات سے پیدا منبع رو کی سمت اور منبع دباو کے قطب پر غور کریں۔ابتدائی رو کی سمت الٹ کرنے یا ابتدائی دباو کے قطب الٹ کرنے سے پیدا منبع رو کی سمت اور منبع دباو کے قطب الٹ ہوں گے۔
\begin{align}
\bV(s)&=\frac{\bI(s)}{sC}+\frac{v(0)}{s}\\
\bI(s)&=sC\bV(s)-Cv(0)
\end{align}
%
\begin{figure}
\centering
\begin{subfigure}{1\textwidth}
\centering
\begin{tikzpicture}[american voltages]
\draw(0,0) to [short,o-]++(\x,0) to [capacitor,l={$C$},v_>={$v(0)$}]++(0,\y) to [short,-o,i<_={$i(t)$}]++(-\x,0);
\draw(0,0) to [short,o-]++(-\x/4,0)coordinate(kBL);
\draw(0,\y) to [short,o-]++(-\x/4,0);
\draw(kBL)++(0,-0.25) rectangle ++(-0.5,\y+0.5);
\draw(0,\y/2)node{$\begin{aligned}&+ \\&  v(t) \\ &- \end{aligned}$};
\end{tikzpicture}
\caption*{(الف)}
\end{subfigure}
\begin{subfigure}{0.4\textwidth}
\centering
\begin{tikzpicture}
\draw(0,0) to [short,o-]++(\x,0) to [american voltage source,l_={$\frac{v(0)}{s}$}]++(0,3/4*\y) to [capacitor,l_={$\frac{1}{sC}$}]++(0,3/4*\y) to [short,-o,i<_={$\bI(s)$}]++(-\x,0);
\draw(0,0) to [short,o-]++(-\x/4,0)coordinate(kBL);
\draw(0,\y+\y/2) to [short,o-]++(-\x/4,0);
\draw(kBL)++(0,-0.25) rectangle ++(-0.5,\y+\y/2+0.5);
\draw(0,\y/2+\y/4)node{$\begin{aligned}&+ \\ \\ &  \bV(s) \\  \\&- \end{aligned}$};
\end{tikzpicture}
\caption*{(ب)}
\end{subfigure}%
\begin{subfigure}{0.6\textwidth}
\centering
\begin{tikzpicture}
\draw(0,0) to [short,o-]++(\x,0) to [capacitor,l_={$\frac{1}{sC}$}]++(0,\y) to [short,-o,i<_={$\bI(s)$}]++(-\x,0);
\draw(0,0) to [short,o-]++(-\x/4,0)coordinate(kBL);
\draw(\x,0) to [short,*-]++(\x,0) to [american current source,l_={$Cv(0)$}]++(0,\y) to [short,-*]++(-\x,0);
\draw(0,\y) to [short,o-]++(-\x/4,0);
\draw(kBL)++(0,-0.25) rectangle ++(-0.5,\y+0.5);
\draw(0,\y/2)node{$\begin{aligned}&+ \\&  \bV(s) \\ &- \end{aligned}$};
\end{tikzpicture}
\caption*{(پ)}
\end{subfigure}%
\caption{وقتی اور مخلوط تعددی دائرہ کار میں برق گیر کا اظہار۔}
\label{شکل_لاپلاس_دور_برق_گیر_اظہار}
\end{figure}

امالہ گیر کے تعلقات
\begin{align}
v(t)&=L\frac{\dif i(t)}{\dif t}\\
i(t)&=\frac{1}{L}\int_0^t v(t) \dif t+i(0)
\end{align}
ہیں جن سے
\begin{align}
\bV(s)&=sL\bI(s)-Li(0)\\
\bI(s)&=\frac{\bV(s)}{sL}+\frac{i(0)}{s}
\end{align}
حاصل ہوتے ہیں۔انہیں شکل \حوالہ{شکل_لاپلاس_دور_امالہ_گیر_اظہار} میں دکھایا گیا ہے۔یہاں بھی ابتدائی معلومات سے پیدا منبع کا دارومدار ابتدائی رو کی سمت اور ابتدائی دباو کے قطب  پر ہے۔ 
\begin{figure}
\centering
\begin{subfigure}{1\textwidth}
\centering
\begin{tikzpicture}
\draw(0,0) to [short,o-]++(\x,0) to [inductor,l_={$L$}]++(0,\y) to [short,-o,i<_={$i(t)$}]++(-\x,0);
\draw(0,0) to [short,o-]++(-\x/4,0)coordinate(kBL);
\draw(0,\y) to [short,o-]++(-\x/4,0);
\draw(kBL)++(0,-0.25) rectangle ++(-0.5,\y+0.5);
\draw(0,\y/2)node{$\begin{aligned}&+ \\&  v(t) \\ &- \end{aligned}$};
\draw[latex-](\x,0)++(-0.4,\y/4)--++(0,\y/2)node[pos=0.5,left]{$i(0)$};
\end{tikzpicture}
\caption*{(الف)}
\end{subfigure}
\begin{subfigure}{0.4\textwidth}
\centering
\begin{tikzpicture}
\draw(0,\y+\y/2) to [short,o-,i>^={$\bI(s)$}]++(\x,0) to [inductor,l={$sL$}]++(0,-3/4*\y) to [american voltage source,l={$L i(0)$}]++(0,-3/4*\y)  to [short,-o,]++(-\x,0);
\draw(0,0) to [short,o-]++(-\x/4,0)coordinate(kBL);
\draw(0,\y+\y/2) to [short,o-]++(-\x/4,0);
\draw(kBL)++(0,-0.25) rectangle ++(-0.5,\y+\y/2+0.5);
\draw(0,\y/2+\y/4)node{$\begin{aligned}&+ \\ \\ &  \bV(s) \\  \\&- \end{aligned}$};
\end{tikzpicture}
\caption*{(ب)}
\end{subfigure}%
\begin{subfigure}{0.6\textwidth}
\centering
\begin{tikzpicture}
\draw(0,0) to [short,o-]++(\x,0) to [inductor,l_={$sL$}]++(0,\y) to [short,-o,i<_={$\bI(s)$}]++(-\x,0);
\draw(0,0) to [short,o-]++(-\x/4,0)coordinate(kBL);
\draw(\x,\y) to [short,*-]++(\x,0) to [american current source,l={$\frac{i(0)}{s}$}]++(0,-\y) to [short,-*]++(-\x,0);
\draw(0,\y) to [short,o-]++(-\x/4,0);
\draw(kBL)++(0,-0.25) rectangle ++(-0.5,\y+0.5);
\draw(0,\y/2)node{$\begin{aligned}&+ \\&  \bV(s) \\ &- \end{aligned}$};
\end{tikzpicture}
\caption*{(پ)}
\end{subfigure}%
\caption{وقتی اور مخلوط تعددی دائرہ کار میں امالہ گیر کا اظہار۔}
\label{شکل_لاپلاس_دور_امالہ_گیر_اظہار}
\end{figure}

شکل \حوالہ{شکل_لاپلاس_استعمال_مشترک_امالہ_دباو_الف} میں دکھائے گئے مربوط لچھوں کے تعلق درج ذیل ہیں۔
\begin{align}
v_1(t)&=L_1 \frac{\dif i_1(t)}{\dif t}+M\frac{\dif i_2(t)}{\dif t}\\
v_2(t)&=L_2 \frac{\dif i_2(t)}{\dif t}+M\frac{i_1(t)}{\dif t}
\end{align} 
یہی مساوات \عددی{s} دائرہ کار میں درج ذیل لکھے جائیں گے۔
\begin{align}
\bV_1(s)&=sL_1 \bI_1(s)-L_1 i_1(0)+sM\bI_2(s)-M i_2(0)\\
\bV_2(s)&=sL_2 \bI_2(s)-L_2 i_2(0)+sM\bI_1(s)-M i_1(0)
\end{align}
%
\begin{figure}
\centering
\begin{subfigure}{1\textwidth}
\centering
\begin{circuitikz}
\draw(0,0) rectangle ++(-\boxW,\boxH);
\draw(0,0.25) to [short,-o]++(\x/4,0) to [short,o-]++(3/4*\x,0)coordinate(BL) to [inductor,l={$L_1$}]++(0,\y)coordinate(TL) to [short,-o,i<_={$i_1$}]++(-3/4*\x,0) to [short,o-]++(-\x/4,0);
\draw(\x+\x/3+\x,0.25) to [short,-o]++(-\x/4,0) to [short,o-]++(-3/4*\x,0)coordinate(BR) to [inductor,l_={$L_2$}]++(0,\y)coordinate(TR) to [short,-o,i<^={$i_2$}]++(3/4*\x,0) to [short,o-]++(\x/4,0);
\draw($(TL)!0.5!(TR)$)node[above]{$M$};
\draw(TL)++(-0.5,-0.5) node[circ]{}; 
\draw(TR)++(0.5,-0.5) node[circ]{}; 
\draw(2*\x+\x/3,0) rectangle ++(\boxW,\boxH);
\draw(\x/4,\boxH/2) node[]{$\begin{aligned} &+ \\ &v_1 \\ &-  \end{aligned}$};
\draw(2*\x+\x/3-\x/4,\boxH/2) node[]{$\begin{aligned} &+ \\ &v_2 \\ &-  \end{aligned}$};
\end{circuitikz}
\caption*{(الف)}
\end{subfigure}
\begin{subfigure}{1\textwidth}
\centering
\begin{circuitikz}
\draw(0,0) rectangle ++(-\boxW,\boxH);
\draw(0,0.25) to [short,-o]++(\x/2,0) to [short,o-]++(\x/2+\x,0)coordinate(BL) to [inductor,l={$sL_1$}]++(0,\y)coordinate(TL);
\draw(0,0.25+\y) to [short,-o]++(\x/2,0) to [short,i>^={$\bI_1(s)$}]++(\x/2,0) to [american voltage source]++(\x,0)coordinate(upL);
\draw(\x+\x/3+3*\x,0.25) to [short,-o]++(-\x/2,0) to [short,o-]++(-\x/2-\x,0)coordinate(BR) to [inductor,l_={$sL_2$}]++(0,\y)coordinate(TR);
 \draw(\x+\x/3+2*\x+\x,0.25+\y)to [short,-o]++(-\x/2,0) to [short,i>_={$\bI_2(s)$}]++(-\x/2,0) to [american voltage source]++(-\x,0)coordinate(upR);
\draw($(TL)!0.5!(TR)$)node[above]{$sM$};
\draw(TL)++(-0.5,-0.5) node[circ]{}; 
\draw(TR)++(0.5,-0.5) node[circ]{}; 
\draw(2*\x+\x/3+2*\x,0) rectangle ++(\boxW,\boxH);
\draw(\x/2+0.3,\boxH/2) node[]{$\begin{aligned} &+ \\ &\bV_1(s) \\ &-  \end{aligned}$};
\draw(2*\x+\x/3-\x/2+2*\x+0.3,\boxH/2) node[]{$\begin{aligned} &+ \\ &\bV_2(s) \\ &-  \end{aligned}$};
\draw(upL)++(-3/4*\x,0.9)node{$L_1i_1(0)+Mi_2(0)$};
\draw(upR)++(3/4*\x,0.9)node{$L_2i_2(0)+Mi_1(0)$};
\end{circuitikz}
\caption*{(ب)}
\end{subfigure}
\caption{مشترکہ امالہ کا لاپلاسی بدل۔}
\label{شکل_لاپلاس_استعمال_مشترک_امالہ_دباو_الف}
\end{figure}

تابع اور غیر تابع منبع دباو اور منبع رو کو بھی \عددی{s} دائرہ کار میں ظاہر کیا جا سکتا ہے
\begin{align}
\bV_1(s)&=\Laplace[v_1(t)]\\
\bI_2(s)&=\Laplace[i_2(t)]
\end{align}
اور اگر \عددی{v_1(t)=A_r i_2(t)} ہو جہاں \عددی{A_r} افزائش مزاحمت نما ہے تب 
 \begin{align}
\bV_1(s)=A_r \bI_2(s)
\end{align}
لکھا جا سکتا ہے۔

\حصہ{تجزیاتی تراکیب}
درج بالا حصے میں ہم نے برقی پرزوں کے \عددی{s} دائرہ کار میں مساوی ادوار حاصل کئے۔انہیں استعمال کرتے ہوئے ادوار حل کئے جا سکتے ہیں۔ایسا کرنے کی خاطر درج ذیل کرنا ہو گا۔

\begin{itemize}
\item
ابتدائی حالت جاننے کے لئے \عددی{t<\SI{0}{\second}} کے لئے دور حل کریں۔اگر \عددی{t<\SI{0}{\second}} میں دور برقرار حالت میں ہو تب برق گیر کو کھلے سر اور امالہ گیر کو قصر دور تصور کرتے ہوئے ابتدائی رو اور ابتدائی دباو حاصل کئے جا سکتے ہیں۔
\item
ابتدائی معلومات شامل کرتے ہوئے تمام پرزوں کی جگہ ان کے مساوی مخلوط تعددی دائرہ کار کے ادوار نسب کریں۔
\item
کسی بھی ترکیب کو استعمال کرتے ہوئے دور کو حل کریں۔جوابات \عددی{s} دائرہ کار میں ہوں گے۔
\item
الٹ لاپلاس بدل لیتے ہوئے وقتی دائرہ کار میں جوابات حاصل کریں۔
\end{itemize}
%==================
\ابتدا{مثال}\شناخت{مثال_لاپلاس_استعمال_مزاحمت_برق_گیر_الف}
لاپلاس بدل کی مدد سے شکل \حوالہ{شکل_لاپلاس_استعمال_مزاحمت_برق_گیر_الف}-الف میں \عددی{v_C(t)} حاصل کریں۔
\begin{figure}
\centering
\begin{subfigure}{0.5\textwidth}
\centering
\begin{tikzpicture}[american voltages]
\draw(0,0) to [american voltage source,l={$20u(t)$}]++(0,\y) to [resistor,l={$\SI{4}{\ohm}$}]++(\x,0) to [capacitor,l_={$\SI{2}{\farad}$},v^<={$v_C(t)$}]++(0,-\y) to [short](0,0);
\end{tikzpicture}
\caption*{(الف)}
\end{subfigure}%
\begin{subfigure}{0.5\textwidth}
\centering
\begin{tikzpicture}[american voltages]
\draw(0,0) to [american voltage source,l={$\frac{20}{s}$}]++(0,\y) to [resistor,l={$\SI{4}{\ohm}$}]++(\x,0) to [capacitor,l_={$\frac{1}{2s}$},v^<={$\bV_C(s)$}]++(0,-\y) to [short](0,0);
\end{tikzpicture}
\caption*{(ب)}
\end{subfigure}%
\caption{مثال \حوالہ{مثال_لاپلاس_استعمال_مزاحمت_برق_گیر_الف} کا دور۔}
\label{شکل_لاپلاس_استعمال_مزاحمت_برق_گیر_الف}
\end{figure}

حل:ابتدائی دباو \عددی{v_C(0)=\SI{0}{\volt}} ہے۔تمام پرزوں کی جگہ \عددی{s} دائرہ کار کے مساوی دور پر کرتے ہوئے شکل-ب حاصل ہوتا ہے۔شکل-ب میں تقسیم دباو کے کلیے سے برق گیر کا دباو لکھتے ہیں۔
\begin{align*}
\bV_C(s)&=\left(\frac{\frac{1}{2s}}{4+\frac{1}{2s}}\right)\frac{20}{s}\\
&=20\left(\frac{1}{s}-\frac{1}{s+\frac{1}{8}}\right)
\end{align*}
الٹ لاپلاس بدل لیتے ہوئے \عددی{v_C(t)} حاصل کرتے ہیں۔
\begin{align*}
v_C(t)=20\left(1-e^{-\frac{t}{8}}\right)u(t)
\end{align*}
\انتہا{مثال}
%====================
\ابتدا{مثال}\شناخت{مثال_لاپلاس_استعمال_دائری_مساوات}
شکل \حوالہ{شکل_لاپلاس_استعمال_دائری_مساوات} کے دائری مساوات اور مساوات جوڑ لکھیں۔
\begin{figure}
\centering
\begin{subfigure}{1\textwidth}
\centering
\begin{tikzpicture}[american voltages]
\draw(0,0) to [american voltage source,l={$v_A(t)$}]++(0,2*\y) to [capacitor,l={$C_1$},v={$v_1(0)$}]++(\x,0) to  [resistor,l={$R_1$}]++(\x,0) to [inductor,l={$L_2$}]++(\x,0)coordinate(ka) to [resistor,l={$R_2$}]++(\x,0) to [american voltage source,l={$v_B(t)$}]++(0,-2*\y) to [short](0,0);
\draw(2*\x,0)node[ground]{}coordinate(kb) to [inductor,*-,l={$L_1$}]++(0,\y) to [capacitor,-*,l={$C_2$},v={$v_2(0)$}]++(0,\y)node[above]{$v_0(t)$};
%initial currents
\draw[-latex] (ka)++(-\x/4,-0.3)node[below]{$i_2(0)$}--++(-\x/4,0);
\draw[-latex] (kb)++(0.3,\y/4)node[right]{$i_1(0)$}--++(0,\y/4);
\end{tikzpicture}
\caption*{(الف)}
\end{subfigure}
\begin{subfigure}{1\textwidth}
\centering
\begin{tikzpicture}[]
\draw(0,0) to [american voltage source,l={$\bV_A(s)$}]++(0,3.5*\y) to [capacitor,l={$\frac{1}{sC_1}$}]++(\x,0) ++(3/4*\x,0) to [american voltage source,l_={$\frac{v_1(0)}{s}$}]++(-3/4*\x,0)++(3/4*\x,0)to  [resistor,l={$R_1$}]++(\x,0) to [inductor,l={$sL_2$}]++(\x,0)++(3/4*\x,0) to [american voltage source,l_={$L_2 i_2(0)$}]++(-3/4*\x,0)++(3/4*\x,0) to [resistor,l={$R_2$}]++(\x,0) to [american voltage source,l={$\bV_B(s)$}]++(0,-3.5*\y) to [short](0,0);
\draw(2*\x+3/4*\x,0)node[ground]{} to [inductor,*-,l={$s L_1$}]++(0,\y) to [american voltage source,l={$L_1 i_1(0)$}]++(0,3/4*\y)++(0,3/4*\y) to [american voltage source,l_={$\frac{v_2(0)}{s}$}]++(0,-3/4*\y)++(0,3/4*\y)to [capacitor,-*,l={$\frac{1}{sC_2}$}]++(0,\y)node[above]{$\bV_0(s)$};
%currents
\draw[stealth-] ([shift={(-150:\x/2)}]1.375*\x,1.75*\y) arc (-150:150:\x/2);
\draw[] (1.375*\x,1.75*\y) node{$\bI_1(s)$};
\draw[stealth-] ([shift={(-150:\x/2)}]2.75*\x+1.375*\x,1.75*\y) arc (-150:150:\x/2);
\draw[] (2.75*\x+1.375*\x,1.75*\y) node{$\bI_2(s)$};
\end{tikzpicture}
\caption*{(ب)}
\end{subfigure}
\caption{مثال \حوالہ{مثال_لاپلاس_استعمال_دائری_مساوات} کا دور۔}
\label{شکل_لاپلاس_استعمال_دائری_مساوات}
\end{figure}

حل:لاپلاس بدل شکل \حوالہ{شکل_لاپلاس_استعمال_دائری_مساوات}-ب میں دکھایا گیا ہے جہاں سے کرخوف دائری مساوات لکھتے ہیں۔
\begin{align*}
\bI_1(s)\left[\frac{1}{sC_1}+R_1+\frac{1}{sC_2}+sL_1\right]-\bI_2(s)\left[\frac{1}{sC_2}+sL_1\right]&=\bV_A(s)-\frac{v_1(0)}{s}+\frac{v_2(0)}{s}-L_1 i_1(0)\\
-\bI_1(s)\left[sL_1+\frac{1}{sC_2}\right]+\bI_2(s)\left[sL_1+\frac{1}{sC_2}+sL_2+R_2\right]&=\bV_B(s)+L_1i_1(0)-\frac{v_2(0)}{s}-L_2i_2(0)
\end{align*} 
مساوات جوڑ لکھتے ہیں۔
\begin{align*}
\frac{\bV_0(s)-\bV_A(s)+\frac{v_1(0)}{s}}{R_1+\frac{1}{sC_1}}+\frac{\bV_0(s)+\frac{v_2(0)}{s}-L_1 i_1(0)}{\frac{1}{sC_2}+sL_1}+\frac{\bV_0(s)-L_2i_2(0)+\bV_B(s)}{sL_2+R_2}=0
\end{align*}
\انتہا{مثال}
%===================
\ابتدا{مثال}\شناخت{مثال_لاپلاس_استعمال_متعدد_طریقے_الف}
شکل \حوالہ{شکل_لاپلاس_استعمال_متعدد_طریقے_الف}-الف میں دور دیا گیا ہے۔اس کو ہم دائری ترکیب، ترکیب جوڑ، مسئلہ نفاذ، تبادلہ منبع اور مسئلہ تھونن کی مدد سے حل کرتے ہیں۔
\begin{figure}
\centering
\begin{subfigure}{0.5\textwidth}
\centering
\begin{tikzpicture}[american voltages]
\draw(0,0) to [american voltage source,l={$8u(t)\,\si{\volt}$}]++(0,2*\y) to [inductor,l={$\SI{2}{\henry}$}]++(\x,0) to [capacitor,l={$\SI{0.25}{\farad}$}]++(\x,0) to [resistor,l_={$\SI{6}{\ohm}$},v^<={$v_0(t)$}]++(0,-2*\y) to [short](0,0); 
\draw(\x,0)node[ground]{} to [american voltage source,*-,l={$2u(t)\,\si{\volt}$}]++(0,\y) to [resistor,-*,l={$\SI{2}{\ohm}$}]++(0,\y);
\end{tikzpicture}
\caption*{(الف)}
\end{subfigure}%
\begin{subfigure}{0.5\textwidth}
\centering
\begin{tikzpicture}[american voltages]
\draw(0,0) to [american voltage source,l={${\frac{8}{s}}$}]++(0,2*\y) to [inductor,l={$2s$}]++(\x,0) to [capacitor,l={$\frac{4}{s}$}]++(\x,0) to [resistor,l_={$6$},v^<={$\bV_0(s)$}]++(0,-2*\y) to [short](0,0); 
\draw(\x,0)node[ground]{} to [american voltage source,*-,l={$\frac{2}{s}$}]++(0,\y) to [resistor,-*,l={$2$}]++(0,\y)node[above]{$\bV_1(s)$};
%currents
\draw[stealth-]([shift={(-150:\x/4)}]\x/2,\y) arc (-150:150:\x/4);
\draw(\x/2,\y)node{$\bI_1(s)$};
\draw[stealth-]([shift={(-150:\x/4)}]\x+\x/3,\y) arc (-150:150:\x/4);
\draw(\x+\x/3,\y)node{$\bI_2(s)$};
\end{tikzpicture}
\caption*{(ب)}
\end{subfigure}
\caption{مثال \حوالہ{مثال_لاپلاس_استعمال_متعدد_طریقے_الف} کا دور۔}
\label{شکل_لاپلاس_استعمال_متعدد_طریقے_الف}
\end{figure}

حل: لاپلاس مساوی شکل-ب میں دکھایا گیا ہے۔ ہم جوڑ \عددی{\bV_1(s)} کو حاصل کرتے ہوئے \عددی{\bV_0(s)} کو تقسیم دباو کے کلیے سے حاصل کریں گے۔ مساوات جوڑ لکھتے ہیں
\begin{align*}
\frac{\bV_1(s)-\frac{8}{s}}{2s}+\frac{\bV_1(s)-\frac{2}{s}}{2}+\frac{\bV_1(s)}{6+\frac{4}{s}}=0
\end{align*}
جس سے
\begin{align*}
\bV_1(s)\left(\frac{1}{2s}+\frac{1}{2}+\frac{1}{6+\frac{4}{s}}\right)&=\frac{4}{s^2}+\frac{1}{s}
\end{align*}
یعنی
\begin{align*}
\bV_1(s)=\frac{2(s+4)(3s+2)}{s(4s^2+5s+2)}
\end{align*}
حاصل ہوتا ہے۔تقسیم دباو کے کلیے سے \عددی{\bV_0(s)} لکھتے ہیں۔
\begin{align*}
\bV_0(s)&=\left(\frac{6}{6+\frac{4}{s}}\right)\bV_1(s)\\
&=\left(\frac{6s}{6s+4}\right)\left[\frac{2(s+4)(3s+2)}{s(4s^2+5s+2)}\right]\\
&=\frac{6(s+4)}{4s^2+5s+2}
\end{align*}

اس دباو کا جزوی کسری پھیلاو لکھتے ہوئے وقتی تفاعل حاصل کرنا ہو گا۔ میں یہاں گزارش کروں گا ہوں کہ آپ صفحہ \حوالہصفحہ{مثال_تعددی_درست_تجزی} پر مثال \حوالہ{مثال_تعددی_درست_تجزی} کو ضرور دیکھیں۔
\begin{align*}
\bV_0(s)&=\frac{6(s+4)}{4(s^2+\frac{5}{4}s+\frac{1}{2})}\\
&=\frac{6(s+4)}{4(s+\frac{5}{8}+j\frac{\sqrt{7}}{8})(s+\frac{5}{8}-j\frac{\sqrt{7}}{8})}\\
&=\frac{K}{s+\frac{5}{8}+j\frac{\sqrt{7}}{8}}+\frac{K^*}{s+\frac{5}{8}-j\frac{\sqrt{7}}{8}}
\end{align*}
مستقل \عددی{K} اور \عددی{K^*} حاصل کرتے ہیں۔
\begin{align*}
K&=\left. \frac{6(s+4)}{4(s+\frac{5}{8}-j\frac{\sqrt{7}}{8})} \right|_{s=-\frac{5}{8}-j\frac{\sqrt{7}}{8}}\\
&=\frac{3}{4}+j\frac{81}{4\sqrt{7}}\\
K^*&=\left. \frac{6(s+4)}{4(s+\frac{5}{8}+j\frac{\sqrt{7}}{8})} \right|_{s=-\frac{5}{8}+j\frac{\sqrt{7}}{8}}\\
&=\frac{3}{4}-j\frac{81}{4\sqrt{7}}
\end{align*}
یوں درج ذیل لکھا جائے گا۔
\begin{align*}
\bV_0(s)&=\frac{\frac{3}{4}+j\frac{81}{4\sqrt{7}}}{s+\frac{5}{8}+j\frac{\sqrt{7}}{8}}+\frac{\frac{3}{4}-j\frac{81}{4\sqrt{7}}}{s+\frac{5}{8}-j\frac{\sqrt{7}}{8}}
\end{align*}
الٹ لاپلاس بدل لیتے ہیں۔
\begin{align*}
v_0(t)&=\left(\frac{3}{4}+j\frac{81}{4\sqrt{7}}\right)e^{-(\frac{5}{8}+j\frac{\sqrt{7}}{8})t}+\left(\frac{3}{4}-j\frac{81}{4\sqrt{7}}\right)e^{-(\frac{5}{8}-j\frac{\sqrt{7}}{8})t}\\
&=e^{-\frac{5}{8}t}\left[\frac{3}{4}\left(e^{-j\frac{\sqrt{7}}{8}t}+e^{j\frac{\sqrt{7}}{8}t}\right)+j\frac{81}{4\sqrt{7}}\left(e^{-j\frac{\sqrt{7}}{8}t}-e^{j\frac{\sqrt{7}}{8}t}\right)\right]\\
&=\frac{1}{4} e^{-\frac{5}{8}t}\left[6\cos\left(\frac{\sqrt{7} t}{8}\right)+\frac{162}{\sqrt{7}} \sin \left(\frac{\sqrt{7}t}{8}\right)\right] \,\si{\volt}
\end{align*}

آئیں یہی جواب دائری ترکیب سے حاصل کریں۔دائری مساوات لکھتے ہیں۔
\begin{align*}
\bI_1(s)\left(2s+2\right)-2\bI_2(s)&=\frac{8}{s}-\frac{2}{s}\\
-2\bI_1(s)+\bI_2(s)\left(2+\frac{4}{s}+6\right)&=\frac{2}{s}
\end{align*}
ان ہمزاد مساوات کا حل درج ذیل ہے
\begin{align*}
\bI_1(s)&=\frac{13s+6}{4s^3+5s^2+2s}\\
\bI_2(s)&=\frac{s+4}{4s^2+5s+2}
\end{align*}
جس سے خارجی دباو حاصل ہوتا ہے۔
\begin{align*}
\bV_0(s)=6\bI_2(s)=\frac{6(s+4)}{4s^2+5s+2}
\end{align*}
%
\begin{figure}
\centering
\begin{subfigure}{0.5\textwidth}
\centering
\begin{tikzpicture}[american voltages]
\draw(0,0) to [american voltage source,l={${\frac{8}{s}}$}]++(0,2*\y) to [inductor,l={$2s$}]++(\x,0) to [capacitor,l_={$\frac{4}{s}$}]++(\x,0) to [resistor,l_={$6$},v^<={$\bV'_0(s)$}]++(0,-2*\y) to [short](0,0); 
\draw(\x,0)node[ground]{}  to [resistor,*-*,l_={$2$}]++(0,2*\y)node[above]{$\bV'_1(s)$};
\draw [decorate,decoration={brace,amplitude=10pt,raise=4pt},yshift=0pt](\x-\x/8,2*\y+0.5) --++ (\x+\x/4,0) node [black,midway,yshift={0.8cm}] {\footnotesize $\bZ_1(s)$};
\end{tikzpicture}
\caption*{(الف)}
\end{subfigure}%
\begin{subfigure}{0.5\textwidth}
\centering
\begin{tikzpicture}[american voltages]
\draw(0,0) to [short]++(0,2*\y) to [inductor,l={$2s$}]++(\x,0) to [capacitor,l_={$\frac{4}{s}$}]++(\x,0) to [resistor,l_={$6$},v^<={$\bV''_0(s)$}]++(0,-2*\y) to [short](0,0); 
\draw(\x,0)node[ground]{} to [american voltage source,*-,l={$\frac{2}{s}$}]++(0,\y) to [resistor,-*,l={$2$}]++(0,\y)node[above]{$\bV''_1(s)$};
\end{tikzpicture}
\caption*{(ب)}
\end{subfigure}
\caption{مسئلہ نفاذ سے حل کرتے ہوئے باری باری ایک ایک منبع کو نافذ کیا گیا ہے}
\label{شکل_لاپلاس_استعمال_متعدد_طریقے_نفاذ}
\end{figure}

مسئلہ نفاذ سے اب اسی دور کو حل کرتے ہیں۔شکل \حوالہ{شکل_لاپلاس_استعمال_متعدد_طریقے_نفاذ} میں باری باری ایک ایک منبع کو لاگو کیا گیا ہے۔شکل \حوالہ{شکل_لاپلاس_استعمال_متعدد_طریقے_نفاذ}-الف کو دیکھ کر \عددی{\bZ_1(s} لکھتے ہیں۔
\begin{align*}
\bZ_1(s)=\frac{2(6+\frac{4}{s})}{2+6+\frac{4}{s}}=\frac{3s+2}{2s+1}
\end{align*}
یوں تقسیم دباو کے کلیے سے \عددی{\bV'_1(s)}  لکھا جا سکتا ہے۔
\begin{align*}
\bV'_1(s)&=\left(\frac{\bZ_1(s)}{2s+\bZ_1(s)}\right)\frac{8}{s}\\
&=\left(\frac{\frac{3s+2}{2s+1}}{2s+\frac{3s+2}{2s+1}}\right)\frac{8}{s}\\
&=\frac{\frac{8}{s}(3s+2)}{4s^2+5s+2}
\end{align*}
تقسیم دباو کے کلیے کو دوبارہ استعمال کرتے ہوئے \عددی{\bV'_1(s)} سے \عددی{\bV''_0(s)} لکھتے ہیں۔
\begin{align*}
\bV'_0(s)&=\left(\frac{6}{6+\frac{4}{s}}\right)\bV'_1(s)\\
&=\left(\frac{3s}{3s+2}\right)\frac{\frac{8}{s}(3s+2)}{4s^2+5s+2}\\
&=\frac{24}{4s^2+5s+2}
\end{align*}
اب شکل \حوالہ{شکل_لاپلاس_استعمال_متعدد_طریقے_نفاذ}-ب سے دوسرے منبع سے پیدا \عددی{\bV''_0(s)} حاصل کرتے ہیں۔یہاں \عددی{2s} اور \عددی{(6+\tfrac{4}{s})} متوازی جڑے ہیں جن کے مساوی کو \عددی{\bZ_2(s)} کہہ کر حاصل کرتے ہیں۔
\begin{align*}
\bZ_2(s)&=\frac{2s(6+\frac{4}{s})}{2s+6+\frac{4}{s}}\\
&=\frac{2s(3s+2)}{s^2+3s+2}
\end{align*}
یوں تقسیم دباو کے کلیے سے درج ذیل لکھا جا سکتا ہے
\begin{align*}
\bV''_1(s)&=\left(\frac{\bZ_2(s)}{2+\bZ_2(s)}\right)\frac{2}{s}\\
&=\left(\frac{\frac{2s(3s+2)}{s^2+3s+2}}{2+\frac{2s(3s+2)}{s^2+3s+2}}\right)\frac{2}{s}\\
&=\frac{2(3s+2)}{4s^2+5s+2}
\end{align*}
اور ایک مرتبہ دوبارہ تقسیم دباو سے 
\begin{align*}
\bV''_0(s)&=\left(\frac{6}{6+\frac{4}{s}}\right)\bV''_1(s)\\
&=\left(\frac{3s}{3s+2}\right)\frac{2(3s+2)}{4s^2+5s+2}\\
&=\frac{6s}{4s^2+5s+2}
\end{align*}
حاصل ہوتا ہے۔یوں دونوں منبع کی موجودگی میں \عددی{\bV_0(s)=\bV'_0(s)+\bV''_0(s)} ہو گا۔
\begin{align*}
\bV_0(s)&=\frac{24}{4s^2+5s+2}+\frac{6s}{4s^2+5s+2}\\
&=\frac{6(s+4)}{4s^2+5s+2}
\end{align*}

%  

آئیں اب شکل \حوالہ{شکل_لاپلاس_استعمال_متعدد_طریقے_الف}-الف کو تبادلہ منبع سے حل کریں۔دونوں منبع دباو کے مساوی منبع رو نسب کرتے ہوئے شکل \حوالہ{شکل_لاپلاس_استعمال_تبادلہ_منبع_الف}-الف ملتا ہے جہاں منبع دباو \عددی{\tfrac{8}{s}} اور اس کے سلسلہ وار \عددی{2s} کو منبع رو \عددی{\tfrac{8/s}{2s}=\tfrac{4}{s^2}} جس کے متوازی \عددی{2s} جڑا ہے میں تبدیل کیا گیا ہے۔اسی طرح منبع دباو \عددی{\tfrac{2}{s}} اور سلسلہ وار \عددی{2} کو منبع رو \عددی{\tfrac{2/s}{2}=\tfrac{1}{s}} میں تبدیل کیا گیا ہے جس کے متوازی \عددی{2} نسب ہے۔

شکل \حوالہ{شکل_لاپلاس_استعمال_تبادلہ_منبع_الف}-الف میں متوازی جڑے منبع رو کا مساوی منبع رو \عددی{\tfrac{4}{s^2}+\tfrac{1}{s}=\tfrac{s+4}{s^2}} ہے۔اسی طرح منبع کے متوازی \عددی{2} اور \عددی{2s} مل کر \عددی{\tfrac{2(2s)}{2+2s}=\tfrac{2s}{s+1}} دیتے ہیں۔یوں شکل-ب حاصل ہوتا ہے۔

شکل \حوالہ{شکل_لاپلاس_استعمال_تبادلہ_منبع_الف}-ب میں منبع رو \عددی{\tfrac{s+4}{s^2}} اور متوازی رکاوٹ \عددی{\tfrac{2s}{s+1}} کو سلسلہ وار جڑے منبع دباو \عددی{(\tfrac{s+4}{s^2})(\tfrac{2s}{s+1})=\tfrac{2(s+4)}{s(s+1)}} اور رکاوٹ \عددی{\tfrac{2s}{s+1}} میں تبدیل کرتے ہوئے شکل-پ حاصل ہوتی ہے جس سے تقسیم دباو کے کلیے سے \عددی{\bV_0(s)} لکھتے ہیں۔
\begin{align*}
\bV_0(s)&=\left(\frac{6}{\frac{2s}{s+1}+\frac{4}{s}+6}\right)\frac{2(s+4)}{s(s+1)}\\
&=\frac{6(s+4)}{4s^2+5s+2}
\end{align*}
%
\begin{figure}
\centering
\begin{subfigure}{1\textwidth}
\centering
\begin{tikzpicture}[american voltages]
\draw(0,0) to [american current source,l={$\frac{4}{s^2}$}]++(0,\y) to [short]++(3*\x,0) to [capacitor,l={$\frac{4}{s}$}]++(\x,0) to [resistor,l={$6$},v_<={$\bV_0(s)$}]++(0,-\y) to [short](0,0);
\draw(\x,0) to [inductor,*-*,l={$2s$}]++(0,\y);
\draw(2*\x,0) to [american current source,*-*,l={$\frac{1}{s}$}]++(0,\y);
\draw(3*\x,0) to [resistor,*-*,l={$2$}]++(0,\y);
\end{tikzpicture}
\caption*{(الف)}
\end{subfigure}
\begin{subfigure}{0.5\textwidth}
\centering
\begin{tikzpicture}[american voltages]
\draw(0,0) to [american current source,l={$\frac{s+4}{s^2}$}]++(0,\y) to [short]++(1*\x,0) to [capacitor,l={$\frac{4}{s}$}]++(\x,0) to [resistor,l={$6$},v_<={$\bV_0(s)$}]++(0,-\y) to [short](0,0);
\draw(\x,0) to [european resistor,*-*,l={$\frac{2s}{s+1}$}]++(0,\y);
\end{tikzpicture}
\caption*{(ب)}
\end{subfigure}%
\begin{subfigure}{0.5\textwidth}
\centering
\begin{tikzpicture}[american voltages]
\draw(0,0) to [american voltage source,l={$\frac{2(s+4)}{s(s+1)}$}]++(0,\y) to [european resistor,l={$\frac{2s}{s+1}$}]++(1*\x,0) to [capacitor,l={$\frac{4}{s}$}]++(\x,0) to [resistor,l={$6$},v_<={$\bV_0(s)$}]++(0,-\y) to [short](0,0);
\end{tikzpicture}
\caption*{(پ)}
\end{subfigure}
\caption{منبع دباو کی جگہ منبع رو نسب کیا گیا ہے۔}
\label{شکل_لاپلاس_استعمال_تبادلہ_منبع_الف}
\end{figure}

مسئلہ تھونن سے حل کرنے کی خاطر شکل \حوالہ{شکل_لاپلاس_استعمال_متعدد_طریقے_الف}-الف میں سلسلہ وار جڑے \عددی{\SI{6}{\ohm}} اور \عددی{\SI{0.25}{\farad}} کو بوجھ تصور کرتے ہوئے بقایا دور کا تھونن مساوی حاصل کرتے ہیں۔تھونن دباو شکل \حوالہ{شکل_لاپلاس_استعمال_متعدد_تھونن}-الف اور تھونن رکاوٹ شکل-ب سے حاصل کی جائے گی۔شکل-الف سے درج ذیل لکھتے
\begin{align*}
\bI(s)&=\frac{\frac{8}{s}-\frac{2}{s}}{2s+2}\\
&=\frac{3}{s(s+1)}
\end{align*}
ہوئے تھونن دباو حاصل کی جا سکتی ہے۔
\begin{align*}
\bV_{\text{تھونن}}&=\frac{2}{s}+2\bI(s)\\
&=\frac{2}{s}+\frac{6}{s(s+1)}\\
&=\frac{2(s+4)}{s+1}
\end{align*}
شکل-ب سے تھونن رکاوٹ حاصل کرتے ہیں۔
 \begin{align*}
\bZ_{\text{تھونن}}&=\frac{(2)(2s)}{2+2s}\\
&=\frac{2s}{s+1}
\end{align*}
تھونن دباو اور تھونن رکاوٹ استعمال کرتے ہوئے تھونن دور حاصل ہوتا ہے جس کے ساتھ بوجھ جوڑتے ہوئے  شکل \حوالہ{شکل_لاپلاس_استعمال_متعدد_تھونن}-پ حاصل ہوتی ہے جہاں سے تقسیم دباو کے کلیے سے \عددی{\bV_0(s)} حاصل ہو گا۔
\begin{align*}
\bV_0(s)&=\left(\frac{6}{\frac{2s}{s+1}+\frac{4}{s}+6}\right)\frac{2(s+4)}{s(s+1)}\\
&=\frac{6(s+4)}{4s^2+5s+2}
\end{align*}
%
\begin{figure}
\centering
\begin{subfigure}{0.5\textwidth}
\centering
\begin{tikzpicture}[american voltages]
\draw(0,0) to [american voltage source,l={${\frac{8}{s}}$}]++(0,2*\y) to [inductor,l={$2s$}]++(\x,0) to [short,-o]++(\x/2,0);
\draw(\x,0)node[circ]{} to [american voltage source,*-,l={$\frac{2}{s}$}]++(0,\y) to [resistor,-*,l={$2$}]++(0,\y);
\draw(0,0) to [short,-o]++(\x+\x/2,0);
\draw[stealth-]([shift={(-150:\x/4)}]\x/2,\y) arc (-150:150:\x/4);
\draw[](\x/2,\y) node{$\bI (s)$};
\draw(\x+\x/2,\y)node[]{$\begin{aligned} &+ \\ \\ \\ &\bV_{\text{تھونن}} \\ \\ \\ &- \end{aligned}$};
\end{tikzpicture}
\caption*{(الف)}
\end{subfigure}%
\begin{subfigure}{0.5\textwidth}
\centering
\begin{tikzpicture}[american voltages]
\draw(0,0) to [short]++(0,2*\y) to [inductor,l={$2s$}]++(\x,0) to [short,-o]++(\x/2,0);
\draw(\x,0)node[circ]{} to [short]++(0,\y) to [resistor,-*,l={$2$}]++(0,\y);
\draw(0,0) to [short,-o]++(\x+\x/2,0);
\draw[latex-] (\x+\x/4,\y)--++(\x/4,0)--++(0,-\y/8)node[below]{$\bZ_{\text{تھونن}}$};
\end{tikzpicture}
\caption*{(ب)}
\end{subfigure}
\begin{subfigure}{1\textwidth}
\centering
\begin{tikzpicture}[american voltages]
\draw(0,0) to [american voltage source,l={$\frac{2(s+4)}{s(s+1)}$}]++(0,2*\y) to [european resistor,l={$\frac{2s}{s+1}$}]++(\x,0) to [capacitor,l={$\frac{4}{s}$}]++(\x,0) to [resistor,l_={$6$},v^<={$\bV_0(s)$}]++(0,-2*\y) to [short](0,0); 
\end{tikzpicture}
\caption*{(پ)}
\end{subfigure}
\caption{مثال \حوالہ{مثال_لاپلاس_استعمال_متعدد_طریقے_الف} کے دور کا تھونن سے حل۔}
\label{شکل_لاپلاس_استعمال_متعدد_تھونن}
\end{figure}
\انتہا{مثال}
%==========
\ابتدا{مشق}
شکل \حوالہ{شکل_لاپلاس_استعمال_متعدد_طریقے_الف}-الف کو مسئلہ نارٹن سے حل کریں۔
\انتہا{مشق}
%=================
\ابتدا{مثال}\شناخت{مثال_لاپلاس_استعمال_مخلوط_جوڑ}
شکل \حوالہ{شکل_لاپلاس_استعمال_مخلوط_جوڑ}-الف میں \عددی{v_0(t)} دریافت کریں۔
\begin{figure}
\centering
\begin{subfigure}{1\textwidth}
\centering
\begin{tikzpicture}[american voltages]
\draw(0,0) to [american controlled current source,l={$4i(t)$}]++(0,\y) to [short]++(\x,0) to [american voltage source,l={$10u(t)$}]++(\x,0) to [resistor,l={$\SI{1}{\ohm}$}]++(\x,0) to [capacitor,l_={$\SI{1}{\farad}$},v^<={$v_0(t)$}]++(0,-\y) to [short] (0,0);
\draw(\x,0) to [capacitor,*-*,l={$\SI{0.5}{\farad}$}]++(0,\y)node[above]{$v_1(t)$};
\draw(2*\x,0) to [resistor,*-*,l={$\SI{6}{\ohm}$},i<_={$i(t)$}]++(0,\y)node[above]{$v_2(t)$};
\draw[dashed,gray] (3/4*\x,3/4*\y) rectangle ++(1.5*\x,3/4*\y)node[right]{\RL{مخلوط جوڑ}};
\end{tikzpicture}
\caption*{(الف)}
\end{subfigure}
\begin{subfigure}{1\textwidth}
\centering
\begin{tikzpicture}[american voltages]
\draw(0,0) to [american controlled current source,l={$4\bI(s)$}]++(0,\y) to [short]++(\x,0) to [american voltage source,l={$\frac{10}{s}$}]++(\x,0) to [resistor,l={$1$}]++(\x,0) to [capacitor,l_={$\frac{1}{s}$},v^<={$\bV_0(s)$}]++(0,-\y) to [short] (0,0);
\draw(\x,0) to [capacitor,*-*,l={$\frac{2}{s}$}]++(0,\y)node[above]{$\bV_1(s)$};
\draw(2*\x,0) to [resistor,*-*,l={$6$},i<_={$\bI(s)$}]++(0,\y)node[above]{$\bV_2(s)$};
\draw[dashed,gray] (3/4*\x,3/4*\y) rectangle ++(1.5*\x,3/4*\y+0.2)node[right]{\RL{مخلوط جوڑ}};
\end{tikzpicture}
\caption*{(ب)}
\end{subfigure}
\caption{مثال \حوالہ{مثال_لاپلاس_استعمال_مخلوط_جوڑ} کا دور۔}
\label{شکل_لاپلاس_استعمال_مخلوط_جوڑ}
\end{figure}

حل: اگر \عددی{v_2(t)} معلوم کیا جائے تو \عددی{v_0(t)} کو تقسیم دباو کے کلیے سے حاصل کیا جا سکتا ہے۔اس دور میں مخلوط جوڑ پایا جاتا ہے لہٰذا مساوات جوڑ کی تعداد کم ہو گی۔ شکل-ب میں لاپلاس بدل  دکھایا گیا ہے جس سے کرخوف مساوات جوڑ لکھتے ہیں
\begin{align*}
\frac{\bV_2(s)}{6}+\frac{\bV_2(s)}{1+\frac{1}{s}}+\frac{\bV_2(s)-\frac{10}{s}}{\frac{2}{s}}-4\bI(s)=0
\end{align*}
جہاں
\begin{align*}
\bI(s)=\frac{\bV_2(s)}{6}
\end{align*}
ہے لہٰذا
\begin{align*}
\frac{\bV_2(s)}{6}+\frac{\bV_2(s)}{1+\frac{1}{s}}+\frac{\bV_2(s)-\frac{10}{s}}{\frac{2}{s}}-\frac{4\bV_2(s)}{6}=0
\end{align*}
یعنی
\begin{align*}
\frac{\bV_2(s)}{6}+\frac{s\bV_2(s)}{s+1}+\frac{s\bV_2(s)-10}{2}-\frac{2\bV_2(s)}{3}=0
\end{align*}
یا
\begin{align*}
\bV_2(s)=\frac{10(s+1)}{s^2+2s-1}
\end{align*}
حاصل ہوتا ہے۔تقسیم دباو کے کلیے سے درکار جواب لکھتے ہیں۔
\begin{align*}
\bV_0(s)&=\bV_2(s)\left(\frac{\frac{1}{s}}{1+\frac{1}{s}}\right)\\
&=\frac{10(s+1)}{s^2+2s-1}\left(\frac{\frac{1}{s}}{1+\frac{1}{s}}\right)\\
&=\frac{10}{s^2+2s-1}
\end{align*}
جزوی کسری پھیلاو حاصل کرتے ہوئے  وقتی دائرہ کار میں دباو حاصل ہو گا۔  نسب نما کے جذر \عددی{-1\mp \sqrt{2}} ہیں لہٰذا درج ذیل لکھا جا سکتا ہے
\begin{align*}
\bV_0(s)&=\frac{10}{(s+1-\sqrt{2})(s+1+\sqrt{2})}\\
&=\frac{K_1}{s+1-\sqrt{2}}+\frac{K_2}{s+1+\sqrt{2}}
\end{align*}
جس سے 
\begin{align*}
K_1&=\left.\frac{10}{s+1+\sqrt{2}} \right|_{s=-1+\sqrt{2}}\\
&=\frac{5}{\sqrt{2}}\\
K_2&=\left.\frac{10}{s+1-\sqrt{2}} \right|_{s=-1-\sqrt{2}}\\
&=-\frac{5}{\sqrt{2}}
\end{align*}
حاصل ہوتے ہیں۔یوں
\begin{align*}
\bV_0(s)=\frac{5}{\sqrt{2}}\left(\frac{1}{s+1-\sqrt{2}}-\frac{1}{s+1+\sqrt{2}}\right)
\end{align*}
لکھ کر الٹ لاپلاس بدل لیتے ہوئے درکار دباو حاصل ہو گا۔
\begin{align*}
v_0(t)&=\frac{5}{\sqrt{2}}\left[e^{-(1-\sqrt{2})t}-e^{-(1+\sqrt{2})t}\right]u(t)\\
&=5\sqrt{2}e^{-t}\sinh (\sqrt{2}t) u(t) \, \si{\volt}
\end{align*}
\انتہا{مثال}
%===================
\ابتدا{مشق}\شناخت{مثال_لاپلاس_استعمال_مشق_الف}
شکل \حوالہ{شکل_لاپلاس_استعمال_مشق_الف} میں \عددی{i_0(t)} بذریعہ مساوات جوڑ دریافت کریں۔
\begin{figure}
\centering
\begin{tikzpicture}
\draw(0,0) to [american current source,l={$4u(t)\,\si{\ampere}$}]++(0,\y) to [short]++(\x,0) to [american voltage source,l={$6u(t)\,\si{\volt}$}]++(\x,0) to [short]++(\x,0) to [resistor,l={$\SI{2}{\ohm}$},i={$i_0(t)$}]++(0,-\y) to [short](0,0);
\draw(\x,0) to [inductor,*-*,l={$\SI{2}{\henry}$}]++(0,\y);
\draw(2*\x,0) to [capacitor,*-*,l={$\SI{0.25}{\farad}$}]++(0,\y);
\end{tikzpicture}
\caption{مثال \حوالہ{مثال_لاپلاس_استعمال_مشق_الف} کا دور۔}
\label{شکل_لاپلاس_استعمال_مشق_الف}
\end{figure}

جواب:\عددی{i_0(t)=[e^{-t}(5\sin t-3\cos t)+3]u(t) \, \si{\ampere}}
\انتہا{مشق}
%===================
\ابتدا{مشق}\شناخت{مثال_لاپلاس_استعمال_مشق_ب}
شکل \حوالہ{شکل_لاپلاس_استعمال_مشق_ب} میں \عددی{v_0(t)} بذریعہ مساوات جوڑ دریافت کریں۔
\begin{figure}
\centering
\begin{tikzpicture}[american voltages]
\draw(0,0) to [american current source,l={$2u(t)\,\si{\ampere}$}]++(0,2*\y) to [short]++(\x,0) to [resistor,l={$\SI{2}{\ohm}$}]++(\x,0) to [short]++(\x,0) to [resistor,l_={$\SI{6}{\ohm}$},v^<={$v_0(t)$}]++(0,-2*\y) to [short](0,0);
\draw(\x,0) to [capacitor,*-*,l={$\SI{0.2}{\farad}$}]++(0,2*\y);
\draw(2*\x,0) to [american voltage source,*-,l={$12u(t)\,\si{\volt}$}]++(0,\y) to [inductor,-*,l={$\SI{4}{\henry}$}]++(0,\y);
\end{tikzpicture}
\caption{مثال \حوالہ{مثال_لاپلاس_استعمال_مشق_ب} کا دور۔}
\label{شکل_لاپلاس_استعمال_مشق_ب}
\end{figure}

جواب:\عددی{v_0(t)=\left[e^{-\frac{t}{2}}\left(7.24\sin \frac{\sqrt{11}}{4}t-12\cos \frac{\sqrt{11}}{4}t\right)+12\right]u(t)\,\si{\volt}}
\انتہا{مشق}
%=============================
\ابتدا{مشق}\شناخت{مثال_لاپلاس_استعمال_مشق_پ}
شکل \حوالہ{شکل_لاپلاس_استعمال_مشق_پ} میں \عددی{v_0(t)} بذریعہ دائری مساوات دریافت کریں۔
\begin{figure}
\centering
\begin{tikzpicture}[american voltages]
\draw(0,0) to [resistor,l={$2$}]++(0,\y) to [inductor,l={$4s$}]++(\x,0) to [capacitor,l={$\frac{2}{s}$}]++(\x,0) to [american voltage source,l={$\frac{6}{s}$}]++(\x,0) to [resistor,l_={$4$},v^<={$v_0(t)$}]++(0,-\y) to [short](0,0);
\draw(\x,0) to [american current source,*-*,l_={$\frac{3}{s}$}]++(0,\y);
%currents
\draw[stealth-] ([shift={(-150:\x/4)}]\x/2,\y/2) arc (-150:150:\x/4);
\draw(\x/2,\y/2) node{$\bI_1(s)$};
\draw[stealth-] ([shift={(-150:\x/4)}]\x+\x,\y/2) arc (-150:150:\x/4);
\draw(\x+\x,\y/2) node{$\bI_2(s)$};
\end{tikzpicture}
\caption{مثال \حوالہ{مثال_لاپلاس_استعمال_مشق_پ} اور مثال \حوالہ{مثال_لاپلاس_استعمال_مشق_ت} کا دور۔}
\label{شکل_لاپلاس_استعمال_مشق_پ}
\end{figure}

جواب:\عددی{v_0(t)=12e^{-\frac{t}{2}}\,\si{\volt}}
\انتہا{مشق}
%=================
\ابتدا{مشق}\شناخت{مثال_لاپلاس_استعمال_مشق_ت}
مسئلہ تھونن کی مدد سے شکل \حوالہ{شکل_لاپلاس_استعمال_مشق_پ} میں  \عددی{v_0(t)} حاصل کریں۔
\انتہا{مشق}
%==========================

لاپلاس بدل کی مدد سے کچھ ادوار ہم حل کر چکے جن میں ابتدائی رو اور دباو صفر تھے۔ آئیں اب چند ایسے ادوار دیکھیں جن میں ابتدائی رو یا ابتدائی دباو پایا جاتا ہو۔اس طرز کے ادوار  ہم پہلے باب \حوالہ{باب_عارضی_رد_عمل} میں حل کر چکے ہیں۔اس باب کے شروع میں ابتدائی رو اور ابتدائی دباو کو شامل کرتے ہوئے پرزوں کے لاپلاس بدل حاصل کئے گئے نہیں شکل \حوالہ{شکل_لاپلاس_دور_مزاحمت_اظہار}، شکل \حوالہ{شکل_لاپلاس_دور_برق_گیر_اظہار} اور شکل \حوالہ{شکل_لاپلاس_دور_امالہ_گیر_اظہار} میں دکھایا گیا ہے۔انہیں کو استعمال کرتے ہوئے ادوار حل کئے جائیں گے۔
%======================

\ابتدا{مثال}\شناخت{مثال_لاپلاس_استعمال_عارضی_ردعمل_الف}
شکل \حوالہ{شکل_لاپلاس_استعمال_عارضی_ردعمل_الف} میں ازل سے ایک سوئچ منقطع اور ایک سوئچ چالو ہے۔عین \عددی{t=\SI{0}{\second}} پر چالو سوئچ کو منقطع کر دیا جاتا ہے جبکہ منقطع سوئچ کو چالو کر دیا جاتا ہے۔لمحہ \عددی{t<0} پر دور کو حل کرتے ہوئے ابتدائی دباو اور ابتدائی رو حاصل کرتے ہوئے \عددی{t\ge 0} پر \عددی{i_0(t)} دریافت کریں۔
\begin{figure}
\centering
\begin{subfigure}{1\textwidth}
\centering
\begin{tikzpicture}
\draw(0,0) to [american voltage source,l={$\SI{6}{\volt}$}]++(0,2*\y) to [cspst,l={${t=0}$}]++(\x,0) to [resistor,l={$\SI{2}{\ohm}$}]++(\x,0) to [short]++(\x,0) to [resistor,l={$\SI{4}{\ohm}$}]++(\x,0) to [inductor,l={$\SI{3}{\henry}$},i={$i(t)$}]++(0,-2*\y) to [short] (0,0);
\draw(2*\x,0) to [capacitor,*-*,l={$\SI{0.2}{\farad}$}]++(0,2*\y);
\draw(3*\x,0) to [american voltage source,*-,l={$\SI{2}{\volt}$}]++(0,\y) to [ospst,-*,l={${t=0}$}]++(0,\y);
\end{tikzpicture}
\caption*{(الف)}
\end{subfigure}
\begin{subfigure}{0.4\textwidth}
\centering
\begin{tikzpicture}
\draw(0,0) to [american voltage source,l_={$\SI{2}{\volt}$}]++(0,\y) to [resistor,l={$\SI{4}{\ohm}$}]++(\x,0) to [short,i={$i_L(0)$}]++(0,-\y) to [short]++(-\x,0) to [short,*-o]++(-\x,0);
\draw(0,\y) to [short,*-o]++(-\x,0);
\draw(-\x,\y/2)node{$\begin{aligned} &+ \\ &v_C(0) \\ &- \end{aligned}$};
\end{tikzpicture}
\caption*{(ب)}
\end{subfigure}%
\begin{subfigure}{0.6\textwidth}
\centering
\begin{tikzpicture}
\draw(0,0) to [american voltage source,l={$\frac{6}{s}$}]++(0,2*\y)  to [resistor,l={$2$}]++(\x,0) to [resistor,l={$4$}]++(\x,0) to [inductor,l={$3s$},i={$\bI(s)$}]++(0,-\y) to [american voltage source,l={$1.5$}]++(0,-\y)to [short] (0,0);
\draw(\x,0)node[ground]{} to [american voltage source,*-,l={$\frac{2}{s}$}]++(0,\y)to [capacitor,-*,l={$\frac{5}{s}$}]++(0,\y)node[above]{$\bV_1(s)$};
\end{tikzpicture}
\caption*{(پ)}
\end{subfigure}
\caption{مثال \حوالہ{مثال_لاپلاس_استعمال_عارضی_ردعمل_الف} کا دور۔}
\label{شکل_لاپلاس_استعمال_عارضی_ردعمل_الف}
\end{figure}

حل: لمحہ \عددی{t<0} پر برق گیر کو کھلے دور جبکہ امالہ گیر کو قصر دور تصور کرتے ہوئے شکل-ب حاصل ہوتا ہے جہاں سے امالہ گیر کی ابتدائی رو \عددی{i_L(0)} اور برق گیر کا ابتدائی دباو \عددی{v_C(0)} حاصل ہوتے ہیں۔
\begin{align*}
i_L(0)&=\frac{2}{4}=\SI{0.5}{\ampere}\\
v_C(0)&=\SI{2}{\volt}
\end{align*}
ابتدائی معلومات کو شامل کرتے ہوئے پرزوں کے لاپلاس مساوی دور پر کرنے سے  لمحہ \عددی{t\ge 0} کے لئے شکل حاصل ہوتا ہے۔مساوات جوڑ لکھتے ہیں
\begin{align*}
\frac{\bV_1(s)-\frac{6}{s}}{2}+\frac{\bV_1(s)-\frac{2}{s}}{\frac{5}{s}}+\frac{\bV_1(s)+1.5}{3s}=0
\end{align*}
جس سے 
\begin{align*}
\bV_1(s)&=\frac{12s^2+91s+120}{s(6s^2+23s+30)}
\end{align*}
حاصل ہوتا ہے۔یوں رو درج ذیل ہے
\begin{align*}
\bI(s)&=\frac{\bV_1(s)}{3s+4}\\
&=\frac{12s^2+91s+120}{s(s+4)(6s^2+23s+30)}
\end{align*}
الٹ لاپلاس بدل لیتے ہوئے درج ذیل ملتا ہے۔
\begin{align*}
i(t)=\left[e^{-\frac{23}{12}t}\left(\frac{44}{\sqrt{191}} \sin \frac{\sqrt{191} t}{12}-2\cos\frac{\sqrt{191} t}{12} \right)+4\right]u(t)\,\si{\ampere}
\end{align*}
\انتہا{مثال}
%==============
\ابتدا{مثال}\شناخت{مثال_لاپلاس_استعمال_عارضی_ردعمل_ب}
شکل \حوالہ{شکل_لاپلاس_استعمال_عارضی_ردعمل_ب} میں ازل سے چالو سوئچ کو لمحہ \عددی{} پر منقطع کیا جاتا ہے۔سوئچ منقطع ہونے کے بعد کی رو \عددی{i(t)} دریافت کریں۔

\begin{figure}
\centering
\begin{subfigure}{1\textwidth}
\centering
\begin{tikzpicture}
\draw(0,0) to [american voltage source,l={$\SI{5}{\volt}$}]++(0,\y) to [ospst,l={${t=0}$}]++(\x,0) to [resistor,l={$\SI{5}{\ohm}$}]++(\x,0) to [resistor,l={$\SI{15}{\ohm}$}]++(\x,0) to [inductor,l={$\SI{3}{\henry}$},i={$i(t)$}]++(\x,0);
\draw(0,0) to [short]++(4*\x,0) to [american voltage source,l_={$\SI{15}{\volt}$}]++(0,\y);
\draw(2*\x,0) to [capacitor,*-*,l={$\SI{0.5}{\farad}$}]++(0,\y);
\end{tikzpicture}
\caption*{(الف)}
\end{subfigure}
\begin{subfigure}{0.4\textwidth}
\centering
\begin{tikzpicture}
\draw(0,0) to [american voltage source,l={$\SI{5}{\volt}$}]++(0,\y) to [resistor,l={$\SI{5}{\ohm}$}]++(\x,0) to [resistor,l={$\SI{15}{\ohm}$},i={$i_L(0)$}]++(\x,0);
\draw(0,0) to [short]++(2*\x,0) to [american voltage source,l_={$\SI{15}{\volt}$}]++(0,\y);
\draw(\x,0) to [short,*-o]++(0,\y/8);
\draw(\x,\y) to [short,*-o]++(0,-\y/8);
\draw(\x+0.3,\y/2)node{$\begin{aligned} &+ \\ &v_C(0) \\ &- \end{aligned}$};
\end{tikzpicture}
\caption*{(ب)}
\end{subfigure}%
\begin{subfigure}{0.6\textwidth}
\centering
\begin{tikzpicture}
\draw(0,0)  to [american voltage source,l={$\frac{7.5}{s}$}] ++(0,\y)to [capacitor,l={$\frac{2}{s}$}]++(0,\y)to [resistor,l={$\frac{15}{s}$}]++(\x,0) to [inductor,l={$3s$},i={$\bI(s)$}]++(\x,0) ;
\draw(0,0) to [short]++(2*\x,0) to [american voltage source,l_={$\frac{15}{s}$}]++(0,\y)to [american voltage source,l_={$1.5$}]++(0,\y);
\end{tikzpicture}
\caption*{(پ)}
\end{subfigure}
\caption{مثال \حوالہ{مثال_لاپلاس_استعمال_عارضی_ردعمل_ب} کا دور۔}
\label{شکل_لاپلاس_استعمال_عارضی_ردعمل_ب}
\end{figure}

حل:چالو سوئچ کی صورت میں برق گیر کو کھلا دور اور امالہ گیر کو قصر دور تصور کرتے ہوئے شکل-ب حاصل ہوتی ہے جہاں سے  امالہ گیر کی ابتدائی رو \عددی{i_L(0)} اور برق گیر کی ابتدائی دباو \عددی{v_C(0)} حاصل کرتے ہیں۔
\begin{align*}
i_L(0)&=\frac{10-20}{5+15}=\SI{-0.5}{\ampere}\\
v_C(0)&=\frac{5\times 15+15\times 5 }{5+15}=\SI{7.5}{\volt}
\end{align*}
ابتدائی معلومات کو استعمال کرتے ہوئے، سوئچ منقطع ہونے کے بعد کا لاپلاس بدل دور شکل-پ میں دکھایا گیا ہے۔ابتدائی رو منفی ہونے کی وجہ سے امالہ کے لاپلاسی اظہار میں \عددی{\SI{1.5}{\volt}} منبع کے قطبین شکل \حوالہ{شکل_لاپلاس_دور_امالہ_گیر_اظہار} کے الٹ ہیں۔ شکل \حوالہ{شکل_لاپلاس_استعمال_عارضی_ردعمل_ب}-ب سے \عددی{\bI(s)} لکھتے ہیں۔
\begin{align*}
\bI(s)&=\frac{\frac{7.5}{s}-1.5-\frac{15}{s}}{\frac{2}{s}+15+3s}\\
&=\frac{-(s+5)}{2(s^2+5s+\frac{2}{3})}\\
&=\frac{-(s+5)}{2(s+\frac{5}{2}-\frac{\sqrt{201}}{6})(s+\frac{5}{2}+\frac{\sqrt{201}}{6})}
\end{align*}
اس کا الٹ لاپلاس بدل لیتے ہوئے درج ذیل حاصل ہوتا ہے۔
 \begin{align*}
i(t)=-e^{-\frac{5}{2}t}\left[\frac{45}{6\sqrt{201}} \sinh \left(\frac{\sqrt{201}}{6}t\right)+\frac{1}{2}\cosh \left(\frac{\sqrt{201}}{6}t\right)\right]u(t)\,\si{\ampere}
\end{align*}
\انتہا{مثال}
%================
\ابتدا{مشق}\شناخت{مشق_لاپلاس_استعمال_سوئچ_الف}
شکل \حوالہ{شکل_لاپلاس_استعمال_سوئچ_الف} میں \عددی{i_0(t)} حاصل کریں۔
\begin{figure}
\centering
\begin{tikzpicture}
\draw(0,0) to [american voltage source,l={$\SI{10}{\volt}$}]++(0,\y) to [ospst,l={${t=0}$}]++(\x,0) to [resistor,l={$\SI{8}{\ohm}$}]++(\x,0) to [resistor,l={$\SI{4}{\ohm}$}]++(\x,0) to [inductor,l={$\SI{4}{\henry}$},i={$i(t)$}]++(\x,0);
\draw(0,0) to [short]++(4*\x,0) to [american voltage source,l_={$\SI{12}{\volt}$}]++(0,\y);
\draw(2*\x,0) to [capacitor,*-*,l={$\SI{1}{\farad}$}]++(0,\y);
\end{tikzpicture}
\caption{مشق \حوالہ{مشق_لاپلاس_استعمال_سوئچ_الف} کا دور۔}
\label{شکل_لاپلاس_استعمال_سوئچ_الف}
\end{figure}

جواب:\عددی{i_0(t)=-\frac{e^{-\frac{t}{2}}}{6}(1+\frac{t}{2})u(t)\,\si{\ampere}}
\انتہا{مشق}
%====================
\ابتدا{مشق}\شناخت{مشق_لاپلاس_استعمال_سوئچ_ب}
شکل \حوالہ{شکل_لاپلاس_استعمال_سوئچ_ب}-الف میں \عددی{v_0(t)} حاصل کریں۔شکل-ب میں داخلی دباو کی مستطیل صورت دی گئی ہے۔
\begin{figure}
\centering
\begin{subfigure}{0.6\textwidth}
\centering
\begin{tikzpicture}[american voltages]
\draw(0,0) to [american voltage source,l={$v_d(t)$}]++(0,\y) to [resistor,l={$\SI{4}{\ohm}$}]++(\x,0) to [inductor,l={$\SI{4}{\henry}$}]++(\x,0) to [resistor,l={$\SI{2}{\ohm}$},v={$v_0(t)$}]++(0,-\y) to [short](0,0);
\draw(\x,0) to [resistor,*-*,l={$\SI{2}{\ohm}$}]++(0,\y);
\end{tikzpicture}
\caption*{(الف)}
\end{subfigure}%
\begin{subfigure}{0.4\textwidth}
\centering
\begin{tikzpicture}
\draw(0,0)--++(0,2)node[left]{$v_d(t)$};
\draw(0,0)--++(3,0)node[below]{$t\,(\si{\second})$};
\draw(0,0)--++(0,1)node[left]{$\SI{20}{\volt}$}--++(1.5,0)--++(0,-1)node[below]{$2$}--(3,0);
\end{tikzpicture}
\caption*{(ب)}
\end{subfigure}%
\caption{مشق \حوالہ{مشق_لاپلاس_استعمال_سوئچ_ب} کا دور۔}
\label{شکل_لاپلاس_استعمال_سوئچ_ب}
\end{figure}

جواب:\عددی{v_0(t)=4(1-e^{-\frac{5}{6}t})u(t)+4(1-e^{-(\frac{5}{6}-2)t})u(t-2) \, \si{\volt}}
\انتہا{مشق}
%=================

\حصہ{تبادلی تفاعل جال}
دور میں کسی بھی دباو یا رو اور داخلی اشارے کے تناسب کو جال کی \اصطلاح{تبادلی تفاعل}\فرہنگ{تبادلی تفاعل}\حاشیہب{transfer function}\فرہنگ{transfer function} یا \اصطلاح{تفاعل جال}\فرہنگ{تفاعل!جال}\فرہنگ{جال!تفاعل}\حاشیہب{network function}\فرہنگ{network!function}\فرہنگ{function!network}  کہتے ہیں۔اگر دونوں متغیرات دباو ہوں تب تبادلی تفاعل \اصطلاح{افزائش دباو}\فرہنگ{افزائش!دباو}\فرہنگ{دباو!افزائش}\حاشیہب{voltage gain}\فرہنگ{voltage!gain}\فرہنگ{gain!voltage} کہلاتا ہے، اگر دونوں متغیرات رو ہوں تب اس کو \اصطلاح{افزائش رو}\فرہنگ{افزائش!رو}\فرہنگ{رو!افزائش}\حاشیہب{current gain}\فرہنگ{current!gain}\فرہنگ{gain!current} کہتے ہیں۔اسی طرح دباو اور رو کے تناسب کو \اصطلاح{افزائش مزاحمت نما}\فرہنگ{افزائش!مزاحمت نما}\فرہنگ{مزاحمت نما!افزائش}\حاشیہب{transresistance gain}\فرہنگ{transcresistance!gain}\فرہنگ{gain!transresistance} کہتے ہیں جبکہ رو اور دباو کے تناسب کو \اصطلاح{افزائش موصلیت نما}\فرہنگ{افزائش!موصلیت نما}\فرہنگ{موصلیت نما!افزائش}\حاشیہب{transconductance gain}\فرہنگ{transconductance!gain}\فرہنگ{gain!transconductance} کہتے ہیں۔تبادلی تفاعل کے حصول میں ابتدائی دباو اور ابتدائی رو کو صفر لیا جاتا ہے۔

فرض کریں کہ کسی دور کا تبادلی تفاعل درج ذیل مساوات دیتی ہے جہاں \عددی{x_d(t)} داخلی اشارہ اور \عددی{y_0(t)} خارجی اشارہ ہیں۔
\begin{multline*}
b_n\frac{\dif^{\, n} y_0(t)}{\dif t^{n}}+b_{n-1}\frac{\dif^{\, n-1} y_0(t)}{\dif t^{n-1}}+\cdots+b_{1}\frac{\dif^{\, 1} y_0(t)}{\dif t^{1}}+b_0 y_0(t)=\\
a_m\frac{\dif^{\, m} x_d(t)}{\dif t^{m}}+a_{m-1}\frac{\dif^{\, m-1} x_d(t)}{\dif t^{m-1}}+\cdots+a_{1}\frac{\dif^{\, 1} x_d(t)}{\dif t^{1}}+a_0 x_d(t)
\end{multline*}
تمام ابتدائی معلومات صفر ہونے کی صورت میں درج بالا کا لاپلاس بدل درج ذیل ہو گا
\begin{multline*}
\left(b_n s^n+b_{n-1}s^{n-1}+\cdots +b_1 s+b_0\right){\textup{Y}_0(s)}=\\
\left(a_m s^m+a_{m-1}s^{m-1}+\cdots+a_1 s+a_0\right){\textup{X}_d(s)}
\end{multline*}
جس سے تبادلی تفاعل \عددی{\bH(s)} 
\begin{align*}
\bH(s)=\frac{{\textup{Y}_0(s)}}{{\textup{X}_d(s)}}=\frac{a_m s^m+a_{m-1}s^{m-1}+\cdots+a_1 s+a_0}{b_n s^n+b_{n-1}s^{n-1}+\cdots +b_1 s+b_0}
\end{align*}
یا
\begin{align}\label{مساوات_لاپلاس_استعمال_عمومی_الف}
{\textup{Y}_0(s)}=\bH(s){\textup{X}_d(s)}
\end{align}
لکھتے ہیں۔

مساوات \حوالہ{مساوات_لاپلاس_استعمال_عمومی_الف} کہتی ہے کہ تبادلی تفاعل \عددی{\bH(s)} اور داخلی تفاعل \عددی{{\textup{X}_d}} کا حاصل ضرب خارجی تفاعل \عددی{{\textup{Y}_0(s)}} کے برابر ہے۔یوں \عددی{x_d(t)=\delta(t)} کی صورت میں چونکہ \عددی{{\textup{X}_d(s)}=1} ہے لہٰذا \عددی{{\textup{Y}_0(s)}=\bH(s)} ہو گا۔
\begin{align}\label{مساوات_لاپلاس_استعمال_تبادلی_تفاعل_اور_خارجی_اشارہ}
{\textup{Y}_0(s)}=\bH(s)\quad  \delta(t)
\end{align}
یہ ایک اہم نتیجہ ہے جس کے تحت کسی بھی دور پر اکائی ضرب تفاعل لاگو کرتے ہوئے خارجی اشارے سے دور کا تبادلی تفاعل حاصل کیا جا سکتا ہے۔ایک بار دور کا تبادلی تفاعل معلوم ہو جائے اس کے بعد کسی بھی داخلی اشارے پر دور کا ردعمل مساوات \حوالہ{مساوات_لاپلاس_استعمال_عمومی_الف} سے حاصل کیا جا سکتا ہے۔اکائی ضرب تفاعل لاگو کرتے ہوئے خارجی ردعمل \عددی{h(t)} دے گا جس کا لاپلاس بدل لیتے ہوئے \عددی{\bH(s)} حاصل کیا جائے گا۔چونکہ تجزیہ گاہ\فرہنگ{تجزیہ گاہ}\حاشیہب{lab}\فرہنگ{lab} میں اکائی ضرب تفاعل پیدا کرنا مشکل بلکہ ناممکن کام ہے لہٰذا ہم دور پر اکائی سیڑھی تفاعل لاگو کرتے ہوئے تبادلی تفاعل حاصل کر سکتے ہیں۔چونکہ \عددی{u(t)} کا لاپلاس بدل \عددی{\tfrac{1}{s}} ہے لہٰذا دور پر اکائی سیڑھی تفاعل لاگو کرتے ہوئے مساوات \حوالہ{مساوات_لاپلاس_استعمال_عمومی_الف} کے تحت درج ذیل لکھا جا سکتا ہے۔
\begin{align}
{\textup{Y}_0(s)}=\frac{\bH(s)}{s} \quad u(t)
\end{align}
یوں اکائی سیڑھی تفاعل لاگو کرتے ہوئے دور کا خارجی اشارہ \عددی{y_0(t)} ناپا جاتا ہے۔خارجی اشارے کا لاپلاس بدل \عددی{{\textup{Y}_0(s)}} دے گا۔درج بالا مساوات کے تحت \عددی{\bH(s)={\textup{Y}_0(s)}} کے برابر ہے۔اس کو یوں بھی بیان کیا جا سکتا ہے کہ ناپے گئے خارجی اشارے کے تفرق \عددی{\tfrac{\dif y_0(t)}{\dif t}} کا لاپلاس بدل نظام کا تبادلی تفاعل \عددی{\bH(s)} ہو گا۔

%==============

\ابتدا{مثال}
دور کا اکائی ضرب تفاعل رد عمل \عددی{\bH(s)=\tfrac{2}{s+5}} ہے۔داخلی اشارہ \عددی{v_d(t)=3e^{-4t}u(t)\,\si{\volt}} لاگو کرتے ہوئے خارجی اشارہ \عددی{v_0(t)} دریافت کریں۔

حل:داخلی اشارے کا لاپلاس بدل لکھتے ہیں۔
\begin{align*}
\bV_d(s)=\frac{3}{s+4}
\end{align*}
یوں مساوات استعمال کرتے ہوئے
\begin{align*}
\bV_0(s)&=\bH(s) \bV_d(s)\\
&=\frac{6}{(s+5)(s+4)}\\
&=\frac{6}{s+4}-\frac{6}{s+5}
\end{align*}
الٹ لاپلاس بدل لیتے ہوئے خارجی اشارہ حاصل کرتے ہیں۔
\begin{align*}
v_0(t)=6\left(e^{-4t}-e^{-5t}\right)u(t)\,\si{\volt}
\end{align*}
\انتہا{مثال}
%=================

تبادلی تفاعل کے قطب سے دور کے ردعمل کے بارے میں بہت کچھ جانا جاتا ہے۔ہم ایک درجی اور دو درجہ ادوار پر باب \حوالہ{باب_عارضی_رد_عمل} میں غور کر چکے ہیں۔یہاں نتائج کو دوبارہ پیش کرتے ہیں۔ایک عدد امالہ گیر یا برق گیر کی صورت میں ردعمل \عددی{y(t)=y_0e^{-\frac{t}{\tau}}} صورت رکھتا ہے جہاں \عددی{\tau} دور کا وقتی مستقل ہے۔دو درجی ادوار کا ردعمل دور کے \اصطلاح{امتیازی مساوات}
\begin{align*}
s^2+2\zeta \omega_0 s+\omega_0^2=0
\end{align*}
کے قطبین پر منحصر ہوتا ہے۔یاد رہے کہ تبادلی تفاعل کا نسب نما امتیازی مساوات کہلاتا ہے۔امتیازی مساوات میں \عددی{ \zeta} \اصطلاح{تقصیری مستقل} اور \عددی{\omega_0} \اصطلاح{بلا تقصیر قدرتی تعدد} ہے اور یہی دو قیمتیں ردعمل کی تین ممکنہ صورتیں تعین کرتی ہیں۔
%=====================
\begin{description}
\جزو{زیادہ تقصیر:} امتیازی مساوات میں \عددی{\zeta>1} اور مساوات  کے جذر
\begin{align*}
s_1&=-\zeta\omega_0-\omega_0\sqrt{\zeta^2-1}\\
s_2&=-\zeta\omega_0+\omega_0\sqrt{\zeta^2-1}
\end{align*}
ہیں لہٰذا جال کا ردعمل درج ذیل ہے۔
\begin{align*}
y(t)=K_1e^{-(\zeta\omega_0+\omega_0\sqrt{\zeta^2-1})t}+K_2e^{-(\zeta\omega_0-\omega_0\sqrt{\zeta^2-1})t}
\end{align*}
\جزو{کم تقصیر:} امتیازی مساوات میں \عددی{\zeta<1} اور مساوات  کے جذر
\begin{align*}
s_1&=-\zeta\omega_0-j\omega_0\sqrt{1-\zeta^2}\\
s_2&=-\zeta\omega_0+j\omega_0\sqrt{1-\zeta^2}
\end{align*}
ہیں لہٰذا جال کا ردعمل درج ذیل ہے۔
\begin{align*}
y(t)&=K_1e^{-(\zeta\omega_0+j\omega_0\sqrt{1-\zeta^2})t}+K_2e^{-(\zeta\omega_0-j\omega_0\sqrt{1-\zeta^2})t}\\
&=Ke^{-\zeta \omega_0 t} \cos (\omega_0\sqrt{1-\zeta^2} t+\phi)
\end{align*}
\جزو{فاصل تقصیر:} امتیازی مساوات میں \عددی{\zeta=1} اور مساوات  کے جذر
\begin{align*}
s_1=s_2=-\omega_0\\
\end{align*}
ہیں لہٰذا جال کا ردعمل درج ذیل ہے۔
\begin{align*}
y(t)&=K_1e^{-\omega_0 t}+K_2 te^{-\omega_0 t}
\end{align*}
\end{description}
%====================

جال کے قطبین اور صفروں کو عموماً \اصطلاح{مخلوط سطح}\فرہنگ{مخلوط!سطح}\فرہنگ{سطح!مخلوط}\حاشیہب{complex plane}\فرہنگ{complex!plane}\فرہنگ{plane!complex} یا \عددی{s} سطح\فرہنگ{plane!s}\فرہنگ{s!plane} پر دکھایا جاتا ہے۔مخلوط سطح کے افقی محور پر \عددی{\sigma} اور عمودی محور پر \عددی{j\omega} رکھتے ہوئے مخلوط تعدد \عددی{s=\sigma+j\omega} دکھایا جاتا ہے۔اس سطح پر صفر کو \عددی{0} جبکہ قطبین کو \عددی{\times} سے ظاہر کیا جاتا ہے۔ اگرچہ ہمیں تفاعل جال کے  غیر لامتناہی قطبین اور صفر سے غرض ہے، یہاں اس حقیقت کا ذکر کرنا ضروری ہے کہ کسی بھی معقول تفاعل میں قطبین اور صفر کی تعداد برابر پائی جاتی ہے۔یوں اگر تبادلی تفاعل میں \عددی{n>m} ہو تب \عددی{n-m} صفر لامتناہی پر پائے جاتے ہیں اور اگر \عددی{n<m} ہو تب \عددی{m-n} قطب لامتناہی پر پائے جاتے ہیں۔
\begin{figure}
\centering
\begin{subfigure}{0.5\textwidth}
\centering
\begin{tikzpicture}
%axis
\draw(-2.25,0)--(2.25,0)node[below]{$\sigma$};
\draw(0,-2.25)--(0,2.25)node[left]{$j\omega$};
%
\draw(-0.5,0)node[cross out,draw=black]{};
\draw(-1.5,0)node[cross out,draw=black]{};
\end{tikzpicture}
\caption*{(الف) مخلوط سطح پر حقیقی سادہ قطبین کا اظہار۔}
\end{subfigure}%
\begin{subfigure}{0.5\textwidth}
\centering
\begin{tikzpicture}
\begin{axis}[kStyleCircuitsA,small,xlabel={$t$},ylabel={$y(t)$},xmin=0,ymin=0,ymax=2.5,xtick=\empty,ytick=\empty]
\addplot[mark=none,color=black,domain=0:4]{e^(-0.5*x)+e^(-1.5*x)};
\end{axis}
\end{tikzpicture}
\caption*{(ب) حقیقی منفی سادہ قطبین سے پیدا ردعمل۔}
\end{subfigure}
\begin{subfigure}{0.5\textwidth}
\centering
\begin{tikzpicture}
%axis
\draw(-2.25,0)--(2.25,0)node[below]{$\sigma$};
\draw(0,-2.25)--(0,2.25)node[left]{$j\omega$};
%
\draw(-1,0.5)node[cross out,draw=black]{};
\draw(-1,-0.5)node[cross out,draw=black]{};
\end{tikzpicture}
\caption*{(پ) مخلوط سطح پر جوڑی دار مخلوط قطبین کا اظہار۔}
\end{subfigure}%
\begin{subfigure}{0.5\textwidth}
\centering
\begin{tikzpicture}
\begin{axis}[kStyleCircuitsA,small,xlabel={$t$},ylabel={$y(t)$},xmin=0,xtick=\empty,ytick=\empty]
\addplot[mark=none,color=black,domain=0:10,samples=100]{2*e^(-0.1*x)*cos(50*x)};
\end{axis}
\end{tikzpicture}
\caption*{(ت) مخلوط جوڑی دار قطبین سے پیدا ردعمل۔}
\end{subfigure}
\begin{subfigure}{0.5\textwidth}
\centering
\begin{tikzpicture}
%axis
\draw(-2.25,0)--(2.25,0)node[below]{$\sigma$};
\draw(0,-2.25)--(0,2.25)node[left]{$j\omega$};
%
\draw(-1,0.15)node[cross out,draw=black]{};
\draw(-1,-0.15)node[cross out,draw=black]{};
\end{tikzpicture}
\caption*{(ٹ) کثیر ہم رقمی حقیقی منفی قطبین۔}
\end{subfigure}%
\begin{subfigure}{0.5\textwidth}
\centering
\begin{tikzpicture}
\begin{axis}[kStyleCircuitsA,small,xlabel={$t$},ylabel={$y(t)$},xmin=0,ymin=0,ymax=2.5,xtick=\empty,ytick=\empty]
\addplot[mark=none,color=black,domain=0:4]{e^(-1*x)+e^(-1*x)};
\end{axis}
\end{tikzpicture}
\caption*{(ث) کثیر ہم رقمی حقیقی منفی قطبین سے پیدا ردعمل۔}
\end{subfigure}
\caption{قطبین اور ردعمل}
\label{شکل_لاپلاس_استعمال_قطبین_اور_ردعمل}
\end{figure}

شکل \حوالہ{شکل_لاپلاس_استعمال_قطبین_اور_ردعمل} میں سادہ اور علیحدہ قطبین، مخلوط قطبین اور کثیر  ہم رقمی قطبین مخلوط سطح پر دکھائے گئے ہیں۔شکل-ٹ میں دو عدد ہم رقمی قطبین کو علیحدہ علیحدہ کر کے دکھایا گیا ہے۔حقیقت میں یہ دونوں حقیقی محور پر ایک ہی نقطے پر پائے جاتے ہیں۔ساتھ ہی ساتھ ان سے حاصل ردعمل بھی دکھایا گیا ہے۔سادہ اور علیحدہ قطبین کے تفاعل  کی شرح تبدیلی کم ہوتی ہے لہٰذا اس کو صفر تک پہنچنے میں زیادہ وقت لگتا ہے۔ مخلوط قطبین کے تفاعل کی شرح تبدیلی زیادہ ہوتی ہے البتہ یہ صفر پر پہنچ کر دوسری جانب نکل جاتا ہے۔یوں مخلوط قطبین کا تفاعل \اصطلاح{مقصور سائن نما}\فرہنگ{مقصور سائن نما}\فرہنگ{سائن نما!مقصور}\حاشیہب{damped sinusoidal}\فرہنگ{damped!sinusoidal} ہوتا ہے۔ کثیر ہم رقمی قطبین کا ردعمل ان دونوں کے درمیان ہے۔یہ تیز تر ممکنہ رفتار سے صفر تک پہنچتا ہے، البتہ اتنا تیز نہیں کہ صفر پر رکھ نہ سکے اور دوسری جانب نکل جائے۔
%======================

\begin{figure}
\centering
\includegraphics{figLaplaceApplicationComplexPlane}
\caption{مخلوط سطح پر مختلف قطبین اور ان کے تفاعل کے ردعمل۔}
\label{شکل_لاپلاس_استعمال_مختلف_قطبین}
\end{figure}

شکل \حوالہ{شکل_لاپلاس_استعمال_مختلف_قطبین} میں مخلوط سطح پر مختلف تفاعل اور تفاعل کے  قطبین  دکھائے گئے۔اس شکل سے کئی حقائق کی وضاحت ہوتی ہے لہٰذا اس پر کچھ وقت صرف کرتے ہیں۔فرض کریں کہ \عددی{p_1} تا \عددی{p_5} بالترتیب \عددی{f_1(t)} تا \عددی{f_5(t)} تفاعل کو ظاہر کرتے ہیں۔مخلوط قطبین جوڑیوں میں پائے جاتے ہیں۔یوں \عددی{p_2} اور \عددی{p^*_2} مخلوط جوڑی ہے جو \عددی{f_2(t)} کو ظاہر کرتے ہیں۔حقیقی جزو صفر ہونے کی صورت میں خیالی قطبین کی جوڑی مثلاً \عددی{p_3} اور \عددی{p^*_3} ملتی ہے۔قطب کا حقیقی جزو اگر مثبت ہو تو تفاعل مسلسل بڑھتا ہے اور اگر حقیقی  جزو منفی ہو تب تفاعل مسلسل گھٹتا ہے۔یوں \عددی{f_4(t)} یا \عددی{f_5(t)} مسلسل بڑھتے تفاعل ہیں جبکہ \عددی{f_1(t)} اور \عددی{f_2(t)} مسلسل گھٹتے تفاعل ہیں۔مسلسل بڑھتا تفاعل غیر متوازن صورت حال کو ظاہر کرتی ہے جو حقیقی دنیا میں زیادہ دیر برقرار نہیں رہ سکتی جیسے مسلسل بڑھتی رو آخر کار کسی نہ کسی چیز کو تباہ کر کے ہی رہے گی۔مسلسل گھٹتا تفاعل متوازن صورت حال کو ظاہر کرتی ہے۔یوں خیالی محور کے دائیں جانب قطب غیر متوازن جبکہ محور کے بائیں جانب قطب متوازن نظام کو ظاہر کرتی ہے۔کسی بھی نظام کی تخلیق کے دوران مخلوط سطح میں قطبین کے مقام پر کھڑی نظر رکھی جاتی ہے اور خیالی محور کے دائیں جانب قطبین سے ہر صورت چھٹکارا حاصل کیا جاتا ہے۔قطب کا خیالی جزو صفر نہ ہونے کی صورت میں تفاعل سائن نما ہو گا لہٰذا \عددی{f_5(t)} مسلسل بڑھتا سائن نما تفاعل ہے جبکہ  \عددی{f_2(t)} مسلسل گھٹتا سائن نما یعنی \اصطلاح{مقصور سائن نما}\فرہنگ{مقصور سائن نما}\فرہنگ{سائن نما!مقصور}\حاشیہب{damped sinusoidal}\فرہنگ{damped!sinusoidal} ہے۔خیالی قطبین کی جوڑی سائن نما تفاعل کو ظاہر کرتی ہے لہٰذا \عددی{f_3(t)} برقرار حیطے کا سائن نما تفاعل ہے۔حقیقی محور سے جتنا دور جایا جائے، تعدد اتنی بڑھتی ہے لہٰذا \عددی{f_5(t)} سے \عددی{f_2(t)} کا تعدد زیادہ ہے اور \عددی{f_3(t)} کا تعدد اس سے بھی زیادہ ہے۔اسی طرح خیالی محور سے جتنا دور جایا جائے، بڑھنے یا گھٹنے کی شرح اتنی بڑھتی ہے لہٰذا \عددی{f_4(t)} کے بڑھنے کی شرح سے \عددی{f_2(t)} کے گھٹنے کی شرح زیادہ ہو گی جبکہ \عددی{f_5(t)} اس سے زیادہ اور \عددی{f_1(t)} تمام سے زیادہ تیزی سے تبدیل ہو گا۔
%================
\ابتدا{مثال}\شناخت{مثال_لاپلاس_استعمال_تغیر_پذیر_برق_گیر_الف}
شکل \حوالہ{شکل_لاپلاس_استعمال_تغیر_پذیر_برق_گیر_الف}-الف میں \اصطلاح{تغیر پذیر برق گیر}\فرہنگ{تغیر پذیر برق گیر}\فرہنگ{برق گیر!تغیر پذیر} استعمال کیا گیا ہے۔خارجی دباو \عددی{v_C(t)} کو \عددی{C=\SI{1}{\farad}}، \عددی{C=\SI{4}{\farad}} اور \عددی{C=\SI{10}{\farad}} کے لئے حاصل کریں۔
\begin{figure}
\centering
\begin{subfigure}{1\textwidth}
\centering
\begin{tikzpicture}[american voltages]
\draw(0,0) to [american voltage source,l={$u(t) \, \si{\volt}$}]++(0,\y) to [resistor,l={$\SI{1}{\ohm}$}]++(\x,0) to [inductor,l={$\SI{1}{\henry}$}]++(\x,0);
\draw(0,0) to [short]++(2*\x,0) to [vC={$C$},v_>={$v_C(t)$}]++(0,\y);
\end{tikzpicture}
\caption*{(الف)}
\end{subfigure}
\begin{subfigure}{1\textwidth}
\centering
\begin{tikzpicture}[american voltages]
\draw(0,0) to [american voltage source,l={$\frac{1}{s}$}]++(0,\y) to [resistor,l={$1$}]++(\x,0) to [inductor,l={$s$}]++(\x,0);
\draw(0,0) to [short]++(2*\x,0) to [vC={$\frac{1}{sC}$},v_>={$\bV_C(s)$}]++(0,\y);
\end{tikzpicture}
\caption*{(ب)}
\end{subfigure}%
\caption{مثال \حوالہ{مثال_لاپلاس_استعمال_تغیر_پذیر_برق_گیر_الف} کا دور۔}
\label{شکل_لاپلاس_استعمال_تغیر_پذیر_برق_گیر_الف}
\end{figure}

شکل \حوالہ{شکل_لاپلاس_استعمال_تغیر_پذیر_برق_گیر_الف}-ب میں لاپلاس بدل دور دکھایا گیا ہے جس سے تقسیم دباو کے کلیے سے خارجی دباو لکھتے ہیں۔
\begin{align*}
\bV_C(s)&=\left(\frac{\frac{1}{sC}}{1+s+\frac{1}{sC}}\right)\frac{1}{s}\\
&=\frac{\frac{1}{C}}{s(s^2+s+\frac{1}{C})}
\end{align*}
\عددی{C=\SI{1}{\farad}}  کے لئے \عددی{\bV_C(s)} کے مساوات کو حل کرتے ہیں۔
\begin{align*}
\bV_C(s)&=\frac{1}{s(s^2+s+1)}\\
&=\frac{1}{s}-\frac{\frac{1}{6}(3+j\sqrt{3})}{s+\frac{1}{2}+j\frac{\sqrt{3}}{2}}-\frac{\frac{1}{6}(3-j\sqrt{3})}{s+\frac{1}{2}-j\frac{\sqrt{3}}{2}}
\end{align*}
امتیازی مساوات کے جذر یعنی \عددی{\bV_C(s)} کے قطبین \عددی{p_1=0}، \عددی{p_2=-\tfrac{1}{2}-j\tfrac{\sqrt{3}}{2}} اور
 \عددی{p_3=-\tfrac{1}{2}+j\tfrac{\sqrt{3}}{2}} ہیں۔یوں ایک عدد حقیقی اور مخلوط جوڑی دار قطبین پائے جاتے ہیں جو \اصطلاح{کم مقصور صورت}\فرہنگ{مقصور!کم}\فرہنگ{underdamped} حال ہے۔الٹ لاپلاس بدل سے وقتی دائرہ کار میں خارجی دباو حاصل کرتے ہیں۔
\begin{align*}
v_C(t)=\left[1-e^{-\frac{t}{2}}\left(\cos \frac{\sqrt{3}t}{2}+\frac{1}{\sqrt{3}}\sin\frac{\sqrt{3}t}{2}\right)\right]u(t)\,\si{\volt}
\end{align*}
\عددی{C=\SI{4}{\farad}}  کے لئے \عددی{\bV_C(s)} کے مساوات کو حل کرتے ہیں۔
\begin{align*}
\bV_C(s)&=\frac{0.25}{s(s^2+s+0.25)}\\
&=\frac{0.25}{s(s+\frac{1}{2})^2}\\
&=\frac{1}{s}-\frac{1}{s+\frac{1}{2}}-\frac{1}{2(s+\frac{1}{2})^2}
\end{align*}
یہاں تینوں قطبین حقیقی ہیں جن میں \عددی{p=-\tfrac{1}{2}} کثیر رقمی قطب ہے جو \اصطلاح{فاصل مقصور حال}\فرہنگ{مقصور!فاصل}\فرہنگ{critical damped} کو ظاہر کرتی ہے۔الٹ لاپلاس لیتے ہوئے \عددی{v_C(t)} حاصل  کرتے ہیں۔
\begin{align*}
v_C(t)=\left(1-e^{-\frac{t}{2}}-\frac{t}{2}e^{-\frac{t}{2}}\right)u(t) \,\si{\volt}
\end{align*}
\عددی{C=\SI{10}{\farad}}  کے لئے \عددی{\bV_C(s)} کے مساوات کو حل کرتے ہیں۔
\begin{align*}
\bV_C(s)&=\frac{0.1}{s(s^2+s+0.1)}\\
&=\frac{1}{s}+\frac{0.145}{s+0.887}-\frac{1.145}{s+0.113}
\end{align*}
اس مساوات کے قطبین \عددی{p_1=0}، \عددی{p_2=-0.887} اور \عددی{p_3=-0.113} ہیں۔یوں تینوں سادہ علیحدہ علیحدہ حقیقی قطبین ہیں لہٰذا تفاعل کا ردعمل \اصطلاح{زیادہ مقصور}\فرہنگ{مقصور!زیادہ}\فرہنگ{damped!over} ہو گا۔الٹ لاپلاس بدل سے \عددی{v_C(t)} حاصل کرتے ہیں۔
\begin{align*}
v_C(t)=\left(1+0.145e^{-0.887t}-1.145e^{-0.113t}\right)u(t)\,\si{\volt}
\end{align*}
\انتہا{مثال}
%==================

\باب{فوریئر تجزیہ}

\حصہ{تکونیاتی فوریئر تسلسل}
\اصطلاح{دوری تفاعل}\فرہنگ{دوری!تفاعل}\فرہنگ{تفاعل!دوری}\حاشیہب{periodic function}\فرہنگ{periodic!function} سے مراد وہ تفاعل ہے جو درج ذیل مساوات پر پورا اترتا ہے جہاں \عددی{T_0}  \اصطلاح{دوری عرصہ}\فرہنگ{دوری عرصہ}\حاشیہب{time period}\فرہنگ{time period} کہلاتی ہے۔
\begin{align}
f(t)=f(t+nT_0), \quad n=\mp 1, \mp 2, \mp3, \cdots
\end{align}
درج بالا مساوات کہتی ہے کہ  کسی بھی لمحہ \عددی{t} پر دوری تفاعل کی قیمت \عددی{f(t)} اور اس لمحے سے \عددی{T_0} وقت بعد تفاعل کی قیمت \عددی{f(t+T_0)} برابر ہیں۔شکل \حوالہ{شکل_فوریئر_دوری_عرصہ} میں اس کی وضاحت  کی گئی ہے۔
\begin{figure}
\centering
\begin{tikzpicture}
\begin{axis}[kStyleCircuitsA,xtick={115,475},xticklabels={$t$,$t+T_0$},ytick=\empty]
\pgfmathsetmacro{\kt}{115}
\pgfmathsetmacro{\ktt}{\kt+360}
\addplot[domain=0:720,samples=100]{sin(x)+1/3*sin(3*x)};
\addplot[] plot coordinates {(\kt,{sin(\kt)+1/3*sin(3*\kt)+0.1}) (\kt,{sin(\kt)+1/3*sin(3*\kt)+0.3})};
\addplot[] plot coordinates {(\ktt,{sin(\ktt)+1/3*sin(3*\ktt)+0.1}) (\ktt,{sin(\ktt)+1/3*sin(3*\ktt)+0.3})};
\addplot[stealth-stealth] plot coordinates {(\kt,{sin(\kt)+1/3*sin(3*\kt)+0.2}) (\ktt,{sin(\ktt)+1/3*sin(3*\ktt)+0.2})}node[pos=0.5,fill=white]{$T_0$};
\addplot[] plot coordinates {(\kt,{sin(\kt)+1/3*sin(3*\kt)})}node[circ]{};
\addplot[] plot coordinates {(\ktt,{sin(\ktt)+1/3*sin(3*\ktt)})}node[circ]{};
\end{axis}
\end{tikzpicture}
\caption{دوری عرصہ۔}
\label{شکل_فوریئر_دوری_عرصہ}
\end{figure}
دوری عرصے کو سیکنڈ \عددی{(\si{\second})} میں ناپا جاتا ہے۔دوری عرصہ \عددی{T_0} اور \اصطلاح{تعدد}\فرہنگ{تعدد} \عددی{f_0} کا تعلق درج ذیل ہے جہاں تعدد کو \اصطلاح{ہرٹز}\فرہنگ{ہرٹز}\حاشیہب{Hertz, \si{\hertz}}\فرہنگ{Hertz}\فرہنگ{\si{\hertz}} \عددی{(\si{\hertz})} میں ناپا جاتا ہے۔
\begin{align}
f_0=\frac{1}{T_0}
\end{align}
\اصطلاح{زاویائی تعدد}\فرہنگ{زاویائی تعدد}\فرہنگ{تعدد!زاویائی} \عددی{\omega_0} اور تعدد \عددی{f_0} کا تعلق درج ذیل ہے۔
\begin{align}
\omega_0=2\pi f_0
\end{align}
زاویائی تعدد کو ریڈیئن فی سیکنڈ \عددی{(\si{\radian \per\second})} میں ناپا جاتا ہے۔شکل \حوالہ{شکل_فوریئر_چند_دوری_امواج} میں چند \اصطلاح{دوری امواج}\فرہنگ{دوری!موج}\فرہنگ{موج!دوری}\حاشیہب{periodic wave}\فرہنگ{periodic!wave} دکھائے گئے ہیں۔
\begin{figure}
\centering
\begin{subfigure}{0.5\textwidth}
\centering
\begin{tikzpicture}
\begin{axis}[small,xmin=0,xtick={360,720,1080},xticklabels={$T_0$, $2T_0$,$3T_0$},ytick={-1,1},yticklabels={$-A$,$A$}]
\addplot[domain=0:1080,samples=100]{cos(x)};
\end{axis}
\end{tikzpicture}
\caption*{(الف) سائن نما موج۔}
\end{subfigure}%
\begin{subfigure}{0.5\textwidth}
\centering
\begin{tikzpicture}
\begin{axis}[small,xmin=0,xtick={1,3,6},xticklabels={$T_a$, $T_0$,$2T_0$},ytick={1},yticklabels={$A$}]
\addplot[] plot coordinates {(0,0) (0,1) (1,1) (1,0) (3,0) (3,1) (4,1) (4,0)(6,0) (6,1) (7,1) (7,0)};
\end{axis}%
\end{tikzpicture}
\caption*{(ب) مستطیل موج۔}
\end{subfigure}
\begin{subfigure}{0.5\textwidth}
\centering
\begin{tikzpicture}
\begin{axis}[small,xmin=0,xtick={3,6,9},xticklabels={$T_0$, $2T_0$,$3T_0$},ytick={1},yticklabels={$A$}]
\addplot[] plot coordinates {(0,0) (3,1) (3,0) (6,1) (6,0) (9,1) (9,0)};
\end{axis}%
\end{tikzpicture}
\caption*{(پ) دندان موج۔}
\end{subfigure}%
\begin{subfigure}{0.5\textwidth}
\centering
\begin{tikzpicture}
\begin{axis}[small,xmin=0,xtick={2,4,6},xticklabels={$T_0$, $2T_0$,$3T_0$},ytick={1},yticklabels={$A$}]
\addplot[] plot coordinates {(0,0) (1,1) (2,0) (3,1) (4,0) (5,1) (6,0)};
\end{axis}%
\end{tikzpicture}
\caption*{(ت) تکونی موج۔}
\end{subfigure}%
\caption{چند دوری امواج۔}
\label{شکل_فوریئر_چند_دوری_امواج}
\end{figure}

کسی بھی دوری تفاعل کو بطور درج ذیل (تکونیاتی) \اصطلاح{فوریئر تسلسل}\فرہنگ{فوریئر تسلسل!تکونیاتی}\فرہنگ{تسلسل!فوریئر}\حاشیہب{trignometric Fourier series}\فرہنگ{Fourier series!trignometric} لکھا\حاشیہد{جین بپٹسٹ یوسف فوریئر نے حرارتی توانائی کے بہاو پر غور کے دوران اس تسلسل کو دریافت کیا۔} جا سکتا ہے
\begin{gather}
\begin{aligned}\label{مساوات_فوریئر_تسلسل_سائن_نما_الف}
f(t)&=a_0+\sum_{n=1}^{\infty} [a_n \cos (n \omega_0 t) +b_n \sin (n \omega_0 t)]\\
&=a_0+a_1\cos \omega_0 t+a_2 \cos (2\omega_0 t)+a_3\cos (3\omega_0 t)+\cdots\\
&\phantom{=a_0\,\,}+b_1\sin \omega_0t+b_2\sin (2\omega_0t)+b_3 \sin (3\omega_0t)+\cdots
\end{aligned}
\end{gather}
جہاں \عددی{a_0}، \عددی{a_1}،\عددی{a_2}، \عددی{b_1} وغیرہ تسلسل کے \اصطلاح{عددی سر}\فرہنگ{عددی سر}\حاشیہب{coefficients}\فرہنگ{coefficients} کہلاتے ہیں۔فوریئر تسلسل کی اوسط قیمت \عددی{a_0} کے برابر ہے۔ایک دوری عرصہ \عددی{T_0} میں \عددی{\cos \omega_0 t} یا \عددی{\sin \omega_0 t} کی ایک لہر، \عددی{\cos (2\omega_0 t)} یا \عددی{\sin (2\omega_0 t)} کی دو لہریں اور \عددی{\cos (m\omega_0 t)} یا \عددی{\sin (m\omega_0 t)} کی \عددی{m} لہریں پوری آتی ہیں۔اس حقیقت کو شکل \حوالہ{شکل_فوریئر_ارکان_تعداد_فی_دوری_عرصہ} میں دکھایا گیا ہے جہاں وضاحت کی خاطر امواج کے حیطے مختلف رکھے گئے ہیں۔فوریئر تسلسل میں \عددی{a_1\cos \omega_0 t +b_1\sin \omega_0 t} \اصطلاح{بنیادی رکن}\فرہنگ{بنیادی رکن}\فرہنگ{ہارمونی!بنیادی رکن}\حاشیہب{fundamental component}\فرہنگ{fundamental component} یا \اصطلاح{پہلا ہارمونی رکن} کہلاتا ہے،   \عددی{a_2\cos (2\omega_0 t) +b_2\sin (2\omega_0 t)} \اصطلاح{دوسرا ہارمونی رکن}\فرہنگ{دوسرا ہارمونی رکن}\فرہنگ{ہارمونی!دوسرا رکن}\حاشیہب{second harmonic}\فرہنگ{harmonic!second} کہلاتا ہے،  \عددی{a_3\cos (3\omega_0 t) +b_3\sin (3\omega_0 t)} تیسرا ہارمونی رکن اور اسی طرح \عددی{a_m\cos (m\omega_0 t) +b_m\sin (m\omega_0 t)} ایم ہارمونی رکن کہلاتا ہے۔
\begin{figure}
\centering
\begin{tikzpicture}
\begin{axis}[kStyleCircuitsA,xlabel={$t\,(\si{\second})$},xtick={90,180,270,360},xticklabels={$\frac{1}{4}T_0$,$\frac{1}{2}T_0$,$\frac{3}{2}T_0$,$T_0$},ytick={1,-1},yticklabels={$+A_0$,$-A_0$}]
\addplot[mark=none,color=black,domain=0:360,samples=100]{sin(x)}node[pos=0.4,pin=45:{$A_0\sin \omega_0 t$}]{};
\addplot[mark=none,color=black,domain=0:360,samples=100]{0.5*sin(2*x)}node[pos=0.7,pin=45:{$\frac{A_0}{2}\sin 2\omega_0 t$}]{};
\end{axis}
\end{tikzpicture}
\caption{ایک دوری عرصہ میں فوریئر تسلسل کے ارکان کی تعداد۔}
\label{شکل_فوریئر_ارکان_تعداد_فی_دوری_عرصہ}
\end{figure}
%======================
ہم یہاں اصل رک کر چند حقائق اور تکملات پر غور کرتے ہیں جو فوریئر تسلسل میں کلیدی کردار ادا کرتے ہیں۔

آپ دو سمتیوں کے \اصطلاح{نقطہ ضرب}\فرہنگ{نقطہ ضرب}\فرہنگ{ضرب!نقطہ}\حاشیہب{dot product}\فرہنگ{dot product} سے خوب واقف ہیں۔سمتیہ \سمتیہ{A} اور \سمتیہ{B} کا نقطہ ضرب یا \اصطلاح{غیر سمتی ضرب}\فرہنگ{غیر سمتی ضرب}\فرہنگ{ضرب!غیر سمتی}\حاشیہب{scalar product}\فرہنگ{scalar product} درج ذیل ہے جہاں دونوں سمتیوں کے مابین زاویہ \عددی{\theta} ہے۔
\begin{align}
\kvec{A} \cdot \kvec{B}=A B \cos \theta
\end{align} 
آپس میں \اصطلاح{عمودی}\فرہنگ{عمودی}\حاشیہب{orthogonal}\فرہنگ{orthogonal} سمتیوں کے مابین \عددی{\theta=90^{\circ}} ہونے کی بدولت \عددی{\kvec{A} \cdot \kvec{B}=0} ہوتا ہے جبکہ کسی بھی سمتیہ کے خود نقطہ ضرب کا جذر اس کے حیطے  کے برابر ہوتا ہے۔
\begin{gather}
\begin{aligned}
 \abs{\kvec{A}}=\sqrt{\kvec{A} \cdot \kvec{A}}
\end{aligned}
\end{gather}
اسی سوچ کے ساتھ تفاعل کا نقطہ ضرب بیان کیا جاتا ہے۔

اگر تفاعل \عددی{f(t) \ne 0} اور \عددی{g(t)\ne 0} کے حاصل ضرب کا تکمل \عددی{a \le t \le b} فاصلے پر صفر کے برابر ہو
\begin{align}
\int_a^b f(t)g(t) \dif t=0
\end{align}
تو \عددی{a\le t\le b} فاصلے پر ان تفاعل کو آپس میں \اصطلاح{عمودی} تصور کیا جاتا ہے۔یاد رہے کہ دونوں تفاعل از خود \اصطلاح{غیر سمتی}\فرہنگ{غیر سمتی}\حاشیہب{scalar}\فرہنگ{scalar} اور غیر صفر ہیں۔

کسی بھی مقدار کا مربع مثبت ہوتا ہے لہٰذا تفاعل کا مربع \عددی{f^2(t)} ہر نقطے پر مثبت ہو گا۔ فاصلہ \عددی{a \le t \le b}  پر تفاعل کے \اصطلاح{معیار}\فرہنگ{معیار}\حاشیہب{norm}\فرہنگ{norm} \عددی{\parallel f(t) \parallel} سے مراد 
\begin{align}
\parallel f(t) \parallel =\sqrt{\int_a^b f^2(t) \dif t}
\end{align}
ہے۔
%===================
\ابتدا{مثال}
ثابت کریں کہ \عددی{0 \le t \le T_0}  فاصلے پر \عددی{\cos (m\omega_0 t)} اور \عددی{\cos (n\omega_0 t)} آپس میں عمودی ہیں جہاں \عددی{m=1,2,3,\cdots} اور \عددی{n=1,2,3,\cdots} ممکن ہیں لیکن \عددی{m\ne n} ہے۔

حل:دیے گئے فاصلے پر دونوں تفاعل کے حاصل ضرب کا تکمل لیتے ہیں۔
\begin{align*}
\int_0^{T_0} \cos (m\omega_0 t) \cos (n\omega_0 t) \dif t&=\int_0^{T_0} \frac{\cos\left[(m+n)\frac{2\pi}{T_0} t\right]+\cos\left[(m-n)\frac{2\pi}{T_0} t\right]}{2}\dif t\\
&=\left.\frac{\sin\left[(m+n)\frac{2\pi}{T_0} t\right]}{2(m+n)\frac{2\pi}{T_0} }+\frac{\sin\left[(m-n)\frac{2\pi}{T_0} t\right]}{2(m-n)\frac{2\pi}{T_0} }\right|_0^{T_0}\\
&=\frac{\sin[(m+n)2\pi]}{2(m+n)\frac{2\pi}{T_0} }+\frac{\sin[(m-n)2\pi]}{2(m-n)\frac{2\pi}{T_0} }\\
&\quad \quad \quad \quad -\frac{\sin[(m+n)0]}{2(m+n)\frac{2\pi}{T_0} }-\frac{\sin[(m-n)0]}{2(m-n)\frac{2\pi}{T_0} }
\end{align*}
چونکہ \عددی{m} اور \عددی{n} عدد صحیح ہیں لہٰذا \عددی{m+n} اور \عددی{m-n} بھی عدد صحیح ہوں گے لہٰذا \عددی{\sin[(m+n)2\pi]=0} اور \عددی{\sin[(m-n)2\pi]=0}  ہوں گے۔اس طرح درج ذیل حاصل ہوتا ہے جو عمودی تفاعل کو ظاہر کرتی ہے۔
\begin{align}\label{مساوات_فوریئر_مختلف_تکمل_الف}
\int_0^{T_0} \cos (m\omega_0 t) \cos (n\omega_0 t) \dif t=0\quad (m\ne n)
\end{align}
\انتہا{مثال}
%===================

\ابتدا{مثال}
ثابت کریں کہ \عددی{0 \le t \le T_0} فاصلے پر \عددی{\sin (m\omega_0 t)} اور \عددی{\sin (n\omega_0 t)} آپس میں عمودی ہیں جہاں \عددی{m=1,2,3,\cdots} اور \عددی{n=1,2,3,\cdots} ممکن ہیں لیکن \عددی{m\ne n} ہے۔

حل:دیے گئے فاصلے پر دونوں تفاعل کے حاصل ضرب کا تکمل لیتے ہیں۔
\begin{align*}
\int_0^{T_0} \sin (m\omega_0 t) \sin (n\omega_0 t) \dif t&=\int_0^{T_0} \frac{\cos\left[(m-n)\frac{2\pi}{T_0} t\right]-\cos\left[(m+n)\frac{2\pi}{T_0} t\right]}{2}\dif t\\
&=\left.\frac{\sin\left[(m-n)\frac{2\pi}{T_0} t\right]}{2(m-n)\frac{2\pi}{T_0} }-\frac{\sin\left[(m+n)\frac{2\pi}{T_0} t\right]}{2(m+n)\frac{2\pi}{T_0} }\right|_0^{T_0}\\
&=\frac{\sin[(m-n)2\pi]}{2(m-n)\frac{2\pi}{T_0} }-\frac{\sin[(m+n)2\pi]}{2(m+n)\frac{2\pi}{T_0} }\\
&\quad \quad \quad \quad -\frac{\sin[(m-n)0]}{2(m-n)\frac{2\pi}{T_0} }+\frac{\sin[(m+n)0]}{2(m+n)\frac{2\pi}{T_0} }
\end{align*}
چونکہ \عددی{m} اور \عددی{n} عدد صحیح ہیں لہٰذا \عددی{m+n} اور \عددی{m-n} بھی عدد صحیح ہوں گے لہٰذا \عددی{\sin[(m+n)2\pi]=0} اور \عددی{\sin[(m-n)2\pi]=0}  ہوں گے۔اس طرح درج ذیل حاصل ہوتا ہے جو عمودی تفاعل کو ظاہر کرتی ہے۔
\begin{align}\label{مساوات_فوریئر_مختلف_تکمل_ب}
\int_0^{T_0} \sin (m\omega_0 t) \sin (n\omega_0 t) \dif t=0\quad (m\ne n)
\end{align}
\انتہا{مثال}
%===================

\ابتدا{مثال}
ثابت کریں کہ \عددی{0 \le t \le T_0} فاصلے پر \عددی{\cos (m\omega_0 t)} اور \عددی{\sin (n\omega_0 t)} آپس میں عمودی ہیں جہاں \عددی{m=1,2,3,\cdots} اور \عددی{n=1,2,3,\cdots} ممکن ہیں۔

حل:دیے گئے فاصلے پر دونوں تفاعل کے حاصل ضرب کا تکمل لیتے ہیں۔
\begin{align*}
\int_0^{T_0} \cos (m\omega_0 t) \sin (n\omega_0 t) \dif t&=\frac{1}{2}\int_0^{T_0} \sin\left[(m+n)\frac{2\pi}{T_0} t\right]-\sin\left[(m-n)\frac{2\pi}{T_0} t\right]\dif t\\
&=\left.-\frac{\cos\left[(m+n)\frac{2\pi}{T_0} t\right]}{2(m+n)\frac{2\pi}{T_0}}+\frac{\cos\left[(m-n)\frac{2\pi}{T_0} t\right]}{2(m-n)\frac{2\pi}{T_0}}\right|_0^{T_0}\\
&=-\frac{\cos[(m+n)2\pi]}{2(m+n)\frac{2\pi}{T_0}}+\frac{\cos[(m-n)2\pi]}{2(m-n)\frac{2\pi}{T_0}}\\
&\quad \quad \quad \quad +\frac{\cos[(m+n)0]}{2(m+n)\frac{2\pi}{T_0}}-\frac{\cos[(m-n)0]}{2(m-n)\frac{2\pi}{T_0}}
\end{align*}
چونکہ \عددی{m} اور \عددی{n} عدد صحیح ہیں لہٰذا \عددی{m+n} اور \عددی{m-n} بھی عدد صحیح ہوں گے لہٰذا \عددی{\cos(m+n)2\pi=1} اور \عددی{\cos(m-n)2\pi=1}  ہوں گے۔اس طرح درج ذیل حاصل ہوتا ہے جو عمودی تفاعل کو ظاہر کرتی ہے۔
\begin{align}\label{مساوات_فوریئر_مختلف_تکمل_پ}
\int_0^{T_0} \cos (m\omega_0 t) \sin (n\omega_0 t) \dif t=0\quad (m\ne n)
\end{align}
\انتہا{مثال}
%===================
\ابتدا{مثال}
تفاعل \عددی{f(t)=\cos (m\omega_0 t)} کا معیار \عددی{0 \le t \le T_0} فاصلے پر حاصل کریں جہاں  \عددی{{m=1,2,3,\cdots}} ممکن ہے۔ 

حل:دیے گئے فاصلے پر معیار کو تکمل سے حاصل کرتے ہیں۔
\begin{align*}
\parallel f(t) \parallel^2&=\int_0^{T_0} \cos^2 \left(m\frac{2\pi}{T_0} t\right) \dif t\\
&=\frac{1}{2}\int_0^{T_0}\left[ 1+\cos \left(2m\frac{2\pi}{T_0} t\right)\right] \dif t\\
&=\left. \frac{t}{2}+\frac{\sin \left(2m\frac{2\pi}{T_0} t\right)}{4m\frac{2\pi}{T_0}}\right|_0^{T_0}\\
&=\frac{T_0}{2}+\frac{\sin 4m\pi}{4m\frac{2\pi}{T_0}}-\frac{0}{2}-\frac{\sin 0}{4m\frac{2\pi}{T_0}}\\
&=\frac{T_0}{2}
\end{align*}
دونوں اطراف کا جذر لیتے ہوئے  \عددی{0 \le t \le T_0} فاصلے پر معیار ملتا ہے۔
\begin{align}\label{مساوات_فوریئر_مختلف_تکمل_ت}
\parallel \cos (m\omega_0 t) \parallel =\sqrt{\int_0^{T_0} \cos^2  (m\omega_0 t) \dif t}=\sqrt{\frac{T_0}{2}}
\end{align}
\انتہا{مثال}
%==================

\ابتدا{مشق}
تفاعل \عددی{f(t)=\sin m\omega_0 t} کا معیار \عددی{0 \le t \le T_0} فاصلے پر درج ذیل ہے جہاں \عددی{m=1,2,3,\cdots} ممکن ہے۔ اس معیار کو حاصل کریں۔
\begin{align}\label{مساوات_فوریئر_مختلف_تکمل_ٹ}
\parallel \sin (m\omega_0 t) \parallel =\sqrt{\int_0^{T_0} \sin^2 (m\omega_0 t) \dif t}=\sqrt{\frac{T_0}{2}}
\end{align}
\انتہا{مشق}
%==================
\ابتدا{مشق}
درج ذیل دو مساوات کو ثابت کریں جہاں \عددی{m=1,2,3,\cdots} ممکن ہے۔
\begin{align}
\int_0^{T_0} \cos (m\omega_0 t) \dif t&=0 \label{مساوات_فوریئر_مختلف_تکمل_ث}\\
\int_0^{T_0} \sin (m\omega_0 t) \dif t&=0\label{مساوات_فوریئر_مختلف_تکمل_ج}
\end{align}
\انتہا{مشق}
%====================
مساوات \حوالہ{مساوات_فوریئر_مختلف_تکمل_الف}، مساوات \حوالہ{مساوات_فوریئر_مختلف_تکمل_ب} اور مساوات \حوالہ{مساوات_فوریئر_مختلف_تکمل_پ} مل کر ثابت کرتے ہیں کہ فوریئر تسلسل میں استعمال ہونے والا ہر تفاعل بقایا تمام تفاعل کے ساتھ \عددی{0 \le t \le T_0} فاصلے پر عمودی ہے۔یوں \عددی{\cos (3\omega_0 t)} کو مثال بناتے ہوئے ہم دیکھتے ہیں کہ یہ \عددی{\cos \omega_0 t}، \عددی{\cos(2\omega_0 t)}،\عددی{\cos(4\omega_0 t)}، \عددی{\sin\omega_0 t}، \عددی{\sin(2\omega_0 t)}، \عددی{\sin(3\omega_0 t)} وغیرہ کے ساتھ عمودی ہے۔

%====================
درج بالا تکملات حاصل کرنے کے بعد اصل مضمون یعنی فوریئر تسلسل پر دوبارہ آتے ہیں۔مساوات \حوالہ{مساوات_فوریئر_مختلف_تکمل_الف} تا مساوات \حوالہ{مساوات_فوریئر_مختلف_تکمل_ج} کو استعمال کرتے ہوئے مساوات  \حوالہ{مساوات_فوریئر_تسلسل_سائن_نما_الف} کے عددی سر \عددی{a_0, a_1,a_2,b_1,\cdots} حاصل کئے جا سکتے ہیں۔آئیں ایسا ہی کریں۔

عددی سر \عددی{a_0} کی قیمت دریافت کرنے کی خاطر ہم  مساوات  \حوالہ{مساوات_فوریئر_تسلسل_سائن_نما_الف} کا تکمل \عددی{0 \le t \le T_0} فاصلے پر لیتے ہیں
\begin{align*}
\int_0^{T_0}f(t) \dif t&=\int_0^{T_0} a_0 \dif t+\sum_{n=1}^{\infty}  \int_0^{T_0}(a_n\cos n \omega_0 t +b_n \sin n \omega_0 t) \dif t\\
&=a_0 T_0
\end{align*}
جہاں مساوات \حوالہ{مساوات_فوریئر_مختلف_تکمل_ث} اور مساوات \حوالہ{مساوات_فوریئر_مختلف_تکمل_ج} کو استعمال کرتے ہوئے مجموعے میں دیے تمام تکمل کو صفر کے برابر پر کیا گیا ہے۔ یوں درج ذیل حاصل ہوتا ہے۔
\begin{align}\label{مساوات_فوریئر_عددی_سر_الف}
a_0=\frac{1}{T_0}\int_0^{T_0}f(t) \dif t
\end{align}
مساوات \حوالہ{مساوات_فوریئر_عددی_سر_الف} کہتا ہے کہ \عددی{a_0} تفاعل \عددی{f(t)} کی اوسط قیمت ہے۔


عددی سر \عددی{a_m} حاصل کرنے کی خاطر مساوات \حوالہ{مساوات_فوریئر_تسلسل_سائن_نما_الف} کے دونوں اطراف کو \عددی{\cos (m\omega_0t)} سے ضرب دیتے ہوئے ایک دوری عرصے پر تکمل کرتے ہیں۔ہم تکمل کو \عددی{0 \le t \le T_0} پر حاصل کرتے ہیں۔
\begin{multline}\label{مساوات_فوریئر_عددی_سر_کا_حصول_الف}
\int_0^{T_0}f(t) \cos( m \omega_0 t) \dif t=\\
\int_0^{T_0} a_0 \cos (m\omega_0 t) \dif t+\sum_{n=1}^{\infty}  \int_0^{T_0} a_n\cos (n \omega_0 t)  \cos (m \omega_0 t) \dif t \\
+\sum_{n=1}^{\infty} \int_0^{T_0} b_n \sin (n \omega_0 t) \cos (m\omega_0 t) \dif t
\end{multline}
دائیں ہاتھ پہلا تکمل مساوات \حوالہ{مساوات_فوریئر_مختلف_تکمل_ث} کی بنا صفر کے برابر ہے جبکہ مساوات \حوالہ{مساوات_فوریئر_مختلف_تکمل_پ} کے تحت تیسرا تکمل صفر کے برابر ہے۔آئیں دوسرے تکمل پر غور کریں۔
\begin{multline*}
\sum_{n=1}^{\infty}  \int_0^{T_0} a_n\cos n \omega_0 t  \cos m \omega_0 t \dif t =\\
\int_0^{T_0} \cos (m\omega_0 t)\left[a_1 \cos \omega_0 t+a_2\cos (2\omega_0 t)+\cdots \right.\\
\left.+a_{m-1}\cos[(m-1)\omega_0 t]+a_m\cos(m\omega_0 t)+\cdots \right] \dif t
\end{multline*}
اب اگر \عددی{n \ne m} ہو تب مساوات \حوالہ{مساوات_فوریئر_مختلف_تکمل_الف} کے تحت تکمل صفر کے برابر ہو گا۔البتہ \عددی{n=m} کی صورت میں مساوات \حوالہ{مساوات_فوریئر_مختلف_تکمل_ت} کو استعمال کرتے ہوئے
\begin{align*}
\int_0^{T_0}a_m \cos^2 (m\omega_0 t) \dif t=a_m\frac{T_0}{2}
\end{align*} 
حاصل ہوتا ہے۔ان قیمتوں کو مساوات \حوالہ{مساوات_فوریئر_عددی_سر_کا_حصول_الف} میں پر کرتے ہوئے درج ذیل حاصل ہوتا ہے۔
\begin{align}\label{مساوات_فوریئر_عددی_سر_ب}
a_m=\frac{2}{T_0}\int_0^{T_0}f(t) \cos (m \omega_0 t) \dif t
\end{align}

عددی سر \عددی{b_m} حاصل کرنے کی خاطر مساوات \حوالہ{مساوات_فوریئر_تسلسل_سائن_نما_الف} کے دونوں اطراف کو \عددی{\sin (m\omega_0 t)} سے ضرب دیتے ہوئے ایک دوری عرصے پر تکمل کرتے ہیں۔ہم تکمل کو \عددی{0 \le t \le T_0} پر حاصل کرتے ہیں۔
\begin{multline}\label{مساوات_فوریئر_عددی_سر_کا_حصول_ب}
\int_0^{T_0}f(t) \sin (m \omega_0 t) \dif t=\\
\int_0^{T_0} a_0 \sin (m\omega_0 t) \dif t+\sum_{n=1}^{\infty}  \int_0^{T_0} a_n\cos (n \omega_0 t)  \sin (m \omega_0 t) \dif t \\
+\sum_{n=1}^{\infty} \int_0^{T_0} b_n \sin (n \omega_0 t) \sin (m\omega_0 t) \dif t
\end{multline}
دائیں ہاتھ پہلا تکمل مساوات \حوالہ{مساوات_فوریئر_مختلف_تکمل_ج} کی بنا صفر کے برابر ہے جبکہ مساوات \حوالہ{مساوات_فوریئر_مختلف_تکمل_پ} کے تحت دوسرا تکمل صفر کے برابر ہے۔آئیں تیسرے تکمل پر غور کریں۔
\begin{multline*}
\sum_{n=1}^{\infty} \int_0^{T_0} b_n \sin (n \omega_0 t) \sin (m\omega_0 t) \dif t=\\
\int_0^{T_0} \sin (m\omega_0 t)\left[b_1 \sin \omega_0 t+b_2\sin (2\omega_0 t)+\cdots \right.\\
\left.+b_{m-1}\sin[(m-1)\omega_0 t]+b_m\sin(m\omega_0 t)+\cdots \right] \dif t
\end{multline*}
اب اگر \عددی{n \ne m} ہو تب مساوات \حوالہ{مساوات_فوریئر_مختلف_تکمل_ب} کے تحت تکمل صفر کے برابر ہو گا۔البتہ \عددی{n=m} کی صورت میں مساوات \حوالہ{مساوات_فوریئر_مختلف_تکمل_ٹ} کو استعمال کرتے ہوئے
\begin{align*}
\int_0^{T_0}b_m \sin^2 (m\omega_0 t) \dif t=b_m\frac{T_0}{2}
\end{align*} 
حاصل ہوتا ہے۔ان قیمتوں کو مساوات \حوالہ{مساوات_فوریئر_عددی_سر_کا_حصول_الف} میں پر کرتے ہوئے درج ذیل حاصل ہوتا ہے۔
\begin{align}\label{مساوات_فوریئر_عددی_سر_پ}
b_m=\frac{2}{T_0}\int_0^{T_0}f(t) \sin (m \omega_0 t) \dif t
\end{align}
مساوات \حوالہ{مساوات_فوریئر_عددی_سر_الف}، مساوات \حوالہ{مساوات_فوریئر_عددی_سر_ب} اور مساوات \حوالہ{مساوات_فوریئر_عددی_سر_پ} فوریئر تکمل کے عددی سر دیتے ہیں۔انہیں یہاں اکٹھے پیش کرتے ہیں۔
\begin{gather}
\begin{aligned}\label{مساوات_فوریئر_عددی_سر_ت}
a_0&=\frac{1}{T_0}\int_0^{T_0}f(t) \dif t\\
a_m&=\frac{2}{T_0}\int_0^{T_0}f(t) \cos (m \omega_0 t) \dif t\\
b_m&=\frac{2}{T_0}\int_0^{T_0}f(t) \sin (m \omega_0 t) \dif t
\end{aligned}
\end{gather}
%============

\ابتدا{مثال}\شناخت{مثال_فوریئر_دندان_موج_الف}
شکل \حوالہ{شکل_فوریئر_دندان_موج_الف}-الف میں دکھائے گئے \اصطلاح{دندان موج}\فرہنگ{دندان موج}\فرہنگ{موج!دندان}\فرہنگ{saw tooth} کا فوریئر تسلسل حاصل کریں۔دو، پانچ اور پچاس فوریئر ارکان استعمال کرتے ہوئے  موج کا خط کھینچیں۔آپ دیکھ سکتے ہیں کہ موج کا دوری عرصہ \عددی{T_0=\SI{3}{\second}} ہے۔
\begin{figure}
\centering
\begin{subfigure}{0.5\textwidth}
\centering
\begin{tikzpicture}
\begin{axis}[small,xlabel={$t\,(\second)$},ylabel style={rotate=-90},xtick={3,6},xticklabels={$3$,$6$},ytick={0,1},yticklabels={$0$,$1$},ymax=1.1]
\addplot[] plot coordinates {(0,0.5)(0,0)  (3,1)  (3,0) (6,1) (6,0) (7.5,0.5)};
\end{axis}
\end{tikzpicture}
\caption*{(الف) دندان موج۔}
\end{subfigure}%
\begin{subfigure}{0.5\textwidth}
\centering
\includegraphics{figFourierSawTooth2}
\caption*{(ب) دو ہارمونی ارکان شامل کئے گئے ہیں۔}
\end{subfigure}
\begin{subfigure}{0.5\textwidth}
\centering
\includegraphics{figFourierSawTooth5}
\caption*{(پ) پانچ ہارمونی ارکان شامل کئے گئے ہیں۔}
\end{subfigure}%
\begin{subfigure}{0.5\textwidth}
\centering
\includegraphics{figFourierSawTooth50}
\caption*{(ت) پچاس ہارمونی ارکان شامل کئے گئے ہیں۔}
\end{subfigure}%
\caption{مثال \حوالہ{مثال_فوریئر_دندان_موج_الف} کی دندان موج۔}
\label{شکل_فوریئر_دندان_موج_الف}
\end{figure}

حل:شکل میں دکھائی گئی موج \عددی{(0,0)} سے \عددی{(3,1)} تک بالکل سیدھی لکیر کی مانند ہے جس کی ڈھلوان 
\begin{align*}
\text{ڈھلوان}=\frac{y_2-y_1}{x_2-x_1}=\frac{1-0}{3-0}=\frac{1}{3}
\end{align*}
ہے لہٰذا اس سیدھے حصے کی مساوات درج ذیل لکھی جا سکتی ہے جہاں لکیر پر کسی بھی نقطے کے کارتیسی محدد مساوات میں پر کرنے سے \عددی{c} کی قیمت حاصل کی جا سکتی ہے۔
\begin{align*}
y=\frac{x}{3}+c
\end{align*}
ہم درج بالا میں \عددی{(0,0)} پر کرتے ہوئے
\begin{align*}
0=\frac{0}{3}+c
\end{align*}
 \عددی{c=0} حاصل کرتے ہیں لہٰذا سیدھی حصے کی مساوات \عددی{y=\tfrac{x}{3}} یعنی
\begin{align}
f(t)=\frac{t}{3}
\end{align}
ہے جہاں کارتیسی نظام کے \عددی{x} محور پر \عددی{t} اور \عددی{y} محور پر \عددی{f(t)} پر کئے گئے ہیں۔

مساوات \حوالہ{مساوات_فوریئر_عددی_سر_ت} سے فوریئر تسلسل کے عددی سر حاصل کرتے ہیں۔
\begin{align*}
a_0&=\frac{1}{T_0}\int_0^{T_0} f(t) \dif t\\
&=\frac{1}{3}\int_0^3 \frac{t}{3} \dif t\\
&=\left. \frac{1}{3} \frac{t^2}{6} \right|_0^3\\
&=\frac{1}{2}
\end{align*}
چونکہ \عددی{a_0} تفاعل کی اوسط قیمت کے برابر ہے لہٰذا یہی جواب تکون کے رقبے \عددی{\tfrac{1}{2}\times 3\times 1=\tfrac{3}{2}} اور قاعدہ \عددی{3} سے حاصل کی جا سکتی ہے یعنی
\begin{align*}
\text{اوسط}&=\frac{\text{رقبہ}}{\text{قاعدہ}}=\frac{\frac{3}{2}}{3}=\frac{1}{2}
\end{align*}
عددی سر \عددی{a_m} حاصل کرتے ہیں۔
\begin{align*}
a_m&=\frac{2}{T_0}\int_0^{T_0} f(t)\cos (m\omega_0 t) \dif t\\
&=\frac{2}{3}\int_0^3 \frac{t}{3} \cos (m \frac{2\pi}{3} t) \dif t\\
&=\left. \frac{2}{9}t \frac{\sin(\frac{2\pi}{3}mt)}{\frac{2\pi}{3}m}+\frac{2}{9}\frac{\cos(\frac{2\pi}{3}mt)}{\left(\frac{2\pi}{3}m\right)^2}\right|_0^3\\
&=0
\end{align*}
اس کا مطلب ہے کہ دندان موج کی فوریئر تسلسل میں کوئی کوسائن تفاعل نہیں پایا جاتا۔

عددی سر \عددی{b_m} حاصل کرتے ہیں۔
\begin{align*}
b_m&=\frac{2}{T_0}\int_0^{T_0} f(t)\sin(m\omega_0 t) \dif t\\
&=\frac{2}{3}\int_0^3 \frac{t}{3}\sin (m\frac{2\pi}{3} t) \dif t\\
&=\left.-\frac{2}{9}t\frac{\cos(\frac{2\pi}{3}mt)}{\frac{2\pi}{3}m}+\frac{2}{9}\frac{\sin(\frac{2\pi}{3}mt)}{\left(\frac{2\pi}{3}m\right)^2}\right|_0^3\\
&=-\frac{1}{m\pi}
\end{align*}
یوں \عددی{m=1,2,3,\cdots} پر کرتے ہوئے عددی سر حاصل ہوتے ہیں یعنی
\begin{align*}
b_1&=-\frac{1}{\pi}\\
b_2&=-\frac{1}{2\pi}\\
b_3&=-\frac{1}{3\pi}\\
&\vdots
\end{align*}
لہٰذا فوریئر تسلسل درج ذیل لکھی جائے گی۔
\begin{align}\label{مساوات_فوریئر_دندان_موج}
f(t)=\frac{1}{2}-\frac{1}{\pi}\left[\sin \omega_0 t+\frac{1}{2} \sin (2\omega_0 t) +\frac{1}{3} \sin (3\omega_0 t)+\cdots\right]
\end{align}
شکل \حوالہ{شکل_فوریئر_دندان_موج_الف}-ب میں  مساوات \حوالہ{مساوات_فوریئر_دندان_موج} کو \عددی{m=2} تک استعمال کرتے ہوئے خط کھینچا گیا ہے۔شکل-پ میں پانچ ہارمونی ارکان استعمال کئے گئے ہیں جبکہ شکل-ت میں پچاس ہارمونی ارکان استعمال کئے گئے ہیں۔آپ دیکھ سکتے ہیں کہ ارکان بڑھانے سے اصل موج کے قریب تر خط حاصل کیا جا سکتا ہے۔
\انتہا{مثال}
%========================
\ابتدا{مثال}\شناخت{مثال_فوریئر_مستطیل_موج}
آئیں شکل \حوالہ{شکل_فوریئر_مستطیل_موج_الف}-الف  میں دکھائے گئے  دوری مستطیل موج کا فوریئر تسلسل حاصل کریں جس میں دوری عرصے کو \عددی{T} لکھا گیا ہے۔
\begin{figure}
\centering
\begin{subfigure}{0.5\textwidth}
\centering
\begin{tikzpicture}
\begin{axis}[small,xlabel={$t$},ylabel style={rotate=-90},xtick={-2,-1,0,1,2,3},xticklabels={$-\frac{T}{2}$,$-\frac{T}{4}$,$0$,$\frac{T}{4}$,$\frac{T}{2}$,$\frac{3T}{4}$},ytick={-1,1},yticklabels={$-V_0$,$V_0$}]
\addplot[] plot coordinates {(-4,1) (-3,1) (-3,-1) (-1,-1) (-1,1) (1,1) (1,-1) (3,-1) (3,1)(4,1)};
\end{axis}
\end{tikzpicture}
\caption*{(الف) مستطیل موج۔}
\end{subfigure}%
\begin{subfigure}{0.5\textwidth}
\centering
\includegraphics{figFourierSquareWave5}
\caption*{(ب) ایک، تین اور پانچ ہارمونی ارکان کا مجموعہ یعنی \عددی{m=5} ہے۔}
\end{subfigure}
\begin{subfigure}{0.5\textwidth}
\centering
\includegraphics{figFourierSquareWave10}
\caption*{((پ) \عددی{m=9} تک ارکان کا مجموعہ۔}
\end{subfigure}%
\begin{subfigure}{0.5\textwidth}
\centering
\includegraphics{figFourierSquareWave50}
\caption*{(ت) \عددی{m=49} تک ارکان کا مجموعہ۔}
\end{subfigure}
\caption{مثال \حوالہ{مثال_فوریئر_مستطیل_موج} کی مستطیل موج۔}
\label{شکل_فوریئر_مستطیل_موج_الف}
\end{figure}

حل:افقی محور کے دونوں اطراف برابر موج پائی جاتی ہے لہٰذا اس کی اوسط قیمت صفر ہو گی اور یوں \عددی{a_0=0} ہو گا۔آئیں یہی جواب مساوات \حوالہ{مساوات_فوریئر_عددی_سر_ت} سے حاصل کریں۔اس مرتبہ ہم دوری عرصے کو \عددی{-\tfrac{T}{2} \le t \le \tfrac{T}{2}} لیتے ہیں۔شکل کو دیکھ معلوم ہوتا ہے کہ \عددی{-\tfrac{T}{4} \le t \le \tfrac{T}{4}} تفاعل کی قیمت \عددی{V_0} ہے جبکہ
 \عددی{-\frac{T}{2}\le t \le -\tfrac{T}{4}} اور \عددی{\tfrac{T}{4} \le t \le \tfrac{T}{2}} پر تفاعل کی قیمت \عددی{-V_0} ہے۔
\begin{align*}
a_0&=\frac{1}{T}\int_0^T f(t) \dif t\\
&=\frac{1}{T} \left(-V_0 \int_{-\frac{T}{2}}^{-\frac{T}{4}} \dif t+V_0\int_{-\frac{T}{4}}^{\frac{T}{4}}\dif t -V_0\int_{\frac{T}{4}}^{\frac{T}{2}} \dif t\right)\\
&=\frac{1}{T}\left[-V_0\left(-\frac{T}{4}+\frac{T}{2}\right)+V_0\left(\frac{T}{4}+\frac{T}{4}\right)-V_0\left(\frac{T}{2}-\frac{T}{4}\right)\right]\\
&=0
\end{align*}
کوسائن کے عددی سر \عددی{a_m} کو مساوات \حوالہ{مساوات_فوریئر_عددی_سر_ت} کی مدد سے حاصل کرتے ہیں۔مستقل \عددی{V_0} کو تکمل کے باہر لکھا گیا ہے۔
\begin{align*}
a_m&=\frac{2}{T}\int_0^T f(t)\cos(m\omega_0 t) \dif t\\
&=-\frac{2}{T}V_0\int_{-\frac{T}{2}}^{-\frac{T}{4}} \cos (\frac{2\pi m}{T} t) \dif t+\frac{2}{T}V_0\int_{-\frac{T}{4}}^{\frac{T}{4}} \cos (\frac{2\pi m}{T} t) \dif t-\frac{2}{T}V_0\int_{\frac{T}{4}}^{\frac{T}{2}} \cos (\frac{2\pi m}{T} t) \dif t\\
&=-\frac{2V_0}{T}\left.\frac{\sin (\frac{2\pi m}{T} t)}{\frac{2\pi m}{T}}\right|_{-\frac{T}{2}}^{-\frac{T}{4}}+\frac{2V_0}{T}\left.\frac{\sin (\frac{2\pi m}{T} t)}{\frac{2\pi m}{T}}\right|_{-\frac{T}{4}}^{\frac{T}{4}}-\frac{2V_0}{T}\left.\frac{\sin (\frac{2\pi m}{T} t)}{\frac{2\pi m}{T}}\right|_{\frac{T}{4}}^{\frac{T}{2}}\\
&=\frac{4V_0}{m\pi} \sin(\frac{m\pi}{2})
\end{align*}
اس سے درج ذیل عددی سر لکھے جا سکتے ہیں۔
\begin{align*}
a_1&=\frac{4V_0}{1\pi} \sin(\frac{1\pi}{2})=\frac{4V_0}{\pi}\\
a_2&=\frac{4V_0}{2\pi} \sin(\frac{2\pi}{2})=0\\
a_3&=\frac{4V_0}{3\pi} \sin(\frac{3\pi}{2})=-\frac{4V_0}{3\pi}\\
a_4&=\frac{4V_0}{4\pi} \sin(\frac{4\pi}{2})=0\\
a_5&=\frac{4V_0}{5\pi} \sin(\frac{5\pi}{2})=\frac{4V_0}{5\pi}\\
&\vdots
\end{align*}
سائن کے عددی سر \عددی{b_m} کو مساوات \حوالہ{مساوات_فوریئر_عددی_سر_ت} کی مدد سے حاصل کرتے ہیں۔مستقل \عددی{V_0} کو تکمل کے باہر لکھا گیا ہے۔
\begin{align*}
b_m&=\frac{2}{T}\int_0^T f(t)\sin(m\omega_0 t) \dif t\\
&=-\frac{2}{T}V_0\int_{-\frac{T}{2}}^{-\frac{T}{4}} \sin (\frac{2\pi m}{T} t) \dif t+\frac{2}{T}V_0\int_{-\frac{T}{4}}^{\frac{T}{4}} \sin (\frac{2\pi m}{T} t) \dif t-\frac{2}{T}V_0\int_{\frac{T}{4}}^{\frac{T}{2}} \sin (\frac{2\pi m}{T} t) \dif t\\
&=\frac{2V_0}{T}\left.\frac{\cos (\frac{2\pi m}{T} t)}{\frac{2\pi m}{T}}\right|_{-\frac{T}{2}}^{-\frac{T}{4}}-\frac{2V_0}{T}\left.\frac{\cos (\frac{2\pi m}{T} t)}{\frac{2\pi m}{T}}\right|_{-\frac{T}{4}}^{\frac{T}{4}}+\frac{2V_0}{T}\left.\frac{\cos (\frac{2\pi m}{T} t)}{\frac{2\pi m}{T}}\right|_{\frac{T}{4}}^{\frac{T}{2}}\\
&=0
\end{align*}
اس معلومات کو استعمال کرتے ہوئے مستطیل موج کی فوریئر مساوات لکھتے ہیں۔
\begin{align}\label{مساوات_فوریئر_مستطیل_موج}
f(t)=\frac{4V_0}{\pi}\left[\cos \omega_0 t-\frac{1}{3}\cos(3\omega_0 t)+\frac{1}{5}\cos(5\omega_0 t)-\frac{1}{7}\cos(7\omega_0 t)+\cdots\right]
\end{align}
مختلف تعداد میں فوریئر تسلسل کے ارکان شامل کرتے ہوئے تفاعل کو شکل \حوالہ{شکل_فوریئر_مستطیل_موج_الف}-ب تا شکل \حوالہ{شکل_فوریئر_مستطیل_موج_الف}-ت میں دکھایا گیا ہے۔
\انتہا{مثال}
%=======================

مثال \حوالہ{مثال_فوریئر_مستطیل_موج} میں مستطیل موج کی فوریئر تسلسل  حاصل کی گئی۔آئیں تسلسل کے ایک رکن سے شروع کرتے ہوئے دیکھیں کہ اس میں مزید ارکان شامل کرتے ہوئے مستطیل موج کیسے حاصل ہوتی ہے۔شکل \حوالہ{شکل_فوریئر_ابھرتا_مستطیل}-الف میں  مساوات \حوالہ{مساوات_فوریئر_مستطیل_موج} کا پہلا ہارمونی رکن \عددی{\tfrac{4V_0}{\pi}\cos\omega_0 t} اور تیسرا ہارمونی رکن \عددی{-\frac{4V_0}{3\pi}\cos(3\omega_0t)} ہلکی سیاہی میں دکھائے گئے ہیں۔دونوں سائن نما صورت رکھتے ہیں جس کا مستطیل سے دور دور تک کوئی واسطہ نہیں ہے۔اسی شکل میں دونوں کے مجموعے کو گہری سیاہی میں دکھایا گیا ہے۔آپ دیکھ سکتے ہیں کہ دو سائن نما امواج مل کر ایسی شکل بناتے ہیں جو مستطیل زیادہ اور سائن نما کم نظر آتا ہے۔مستطیل موج کی چوٹی \عددی{V_0} ہے جبکہ پہلے ہارمونی رکن کی چوٹی \عددی{\tfrac{4V_0}{\pi}=1.27V_0} ہے۔تیسرا ہارمونی رکن اس چوٹی کو نیچے کھینچتا ہے۔اسی طرح مستطیل موج \عددی{\mp \tfrac{T}{4}} پر یکدم قیمت تبدیل کرتی ہے جبکہ پہلا ہارمونی جزو نہایت صبروتحمل کے ساتھ منفی چوٹی سے مثبت چوٹی اور مثبت چوٹی سے منفی چوٹی پہنچتی ہے۔ یہاں بھی تیسرا ہارمونی رکن پہلے رکن کے اطراف کو کھینچ کر ان کی ڈھلوان بڑھاتی ہے۔

\begin{figure}
\centering
\begin{subfigure}{0.5\textwidth}
\centering
\begin{tikzpicture}
\begin{axis}[small,xlabel={$t$},ylabel style={rotate=-90},xtick={-2,-1,0,1,2},xticklabels={$-\frac{T}{2}$,$-\frac{T}{4}$,$0$,$\frac{T}{4}$,$\frac{T}{2}$},ytick={0,1},yticklabels={$0$,$V_0$}]
\addplot[gray,domain=-1:1,samples=100]{4/pi*cos(90*x)};
\addplot[gray,domain=-1:1,samples=100]{-4/(3*pi)*cos(3*90*x)};
\addplot[black,domain=-1:1,samples=100]{4/pi*cos(90*x)-4/(3*pi)*cos(3*90*x)};
\end{axis}
\end{tikzpicture}
\caption*{(الف) پہلا اور تیسرا ہارمونی رکن مل کر مستطیل صورت بنانے کی کوشش کرتے ہیں۔}
\end{subfigure}%
\begin{subfigure}{0.5\textwidth}
\centering
\begin{tikzpicture}
\begin{axis}[small,xlabel={$t$},ylabel style={rotate=-90},xtick={-2,-1,0,1,2},xticklabels={$-\frac{T}{2}$,$-\frac{T}{4}$,$0$,$\frac{T}{4}$,$\frac{T}{2}$},ytick={0,1},yticklabels={$0$,$V_0$}]
\addplot[gray,domain=-1:1,samples=100]{4/pi*cos(90*x)-4/(3*pi)*cos(3*90*x)};
\addplot[gray,domain=-1:1,samples=100]{4/(5*pi)*cos(5*90*x)};
\addplot[black,domain=-1:1,samples=100]{4/pi*cos(90*x)-4/(3*pi)*cos(3*90*x)+4/(5*pi)*cos(5*90*x)};
\end{axis}
\end{tikzpicture}
\caption*{(ب) پہلے،تیسرا اور پانچواں ہارمونی ارکان مل کر مستطیل شکل بناتے ہیں۔}
\end{subfigure}
\caption{بتدریج زیادہ ارکان شامل کرتے ہوئے مستطیل موج کی صورت ابھرتے ہوئے دیکھتے ہیں۔}
\label{شکل_فوریئر_ابھرتا_مستطیل}
\end{figure}

شکل \حوالہ{شکل_فوریئر_ابھرتا_مستطیل}-الف میں تیسرا رکن زیادہ جزبات میں آ کر پہلی رکن کی چوٹی ضرورت سے زیادہ نیچے کھینچ  دیتا ہے۔شکل-ب میں پہلے اور تیسرے ارکان سے حاصل موج کو ہلکی سیاہی میں دکھایا گیا ہے۔ساتھ ہی ساتھ پانچویں رکن کو بھی ہلکی سیاہی میں دکھایا گیا ہے۔ان کے مجموعے کو گہری سیاہی میں دکھایا گیا ہے۔آپ دیکھ سکتے ہیں پانچواں رکن ضرورت سے زیادہ نیچے کھینچی گئی چوٹی کو معمولی اٹھاتا ہے تا کہ یہ \عددی{V_0} کے قریب ہو جائے۔اسی طرح یہ رکن بھی موج کے اطراف کی ڈھلوان بڑھاتا ہے۔فوریئر تسلسل کے بقایا ارکان بھی اسی طرح مدد کرتے ہوئے اطراف کو زیادہ عمودی اور چوٹی کو بالکل چپٹی بنانے میں مدد دیتے ہیں حتٰی کہ ہمیں بالکل مستطیل موج نظر آتی ہے۔

شکل \حوالہ{شکل_فوریئر_مستطیل_موج_الف}-ب، پ اور ت میں آپ دیکھتے ہیں کہ فوریئر تسلسل سے حاصل موج \عددی{\mp \tfrac{T}{4}} پر درکار قیمت سے تجاوز کرتے ہوئے آگے نکل جاتی ہے۔تسلسل میں ارکان کی تعداد بڑھانے سے ان تجاوزات کا خاتمہ نہیں ہوتا۔
%======================
\ابتدا{مشق}
شکل \حوالہ{شکل_فوریئر_مستطیل_موج_الف}-الف میں عددی سر حاصل کرتے ہوئے تکملات کو \عددی{-\tfrac{T}{4}\le t\le \tfrac{3T}{4}} پر حاصل کرتے ہوئے فوریئر تسلسل حاصل کریں۔

جواب:عددی سر حاصل کرتے ہوئے دوری موج کے کسی بھی حصے پر مسلسل ایک دوری عرصے پر تکمل حاصل کیا جا سکتا ہے۔جوابات میں کوئی فرق نہیں پایا جاتا۔
\انتہا{مشق}
%=======================
\ابتدا{مشق}\شناخت{مشق_فوریئر_مشق_مستطیل}
شکل \حوالہ{شکل_فوریئر_مشق_مستطیل}-الف میں دکھائے گئے مستطیل موج کی فوریئر تسلسل حاصل کریں۔
\begin{figure}
\centering
\begin{subfigure}{0.5\textwidth}
\centering
\begin{tikzpicture}
\begin{axis}[small,xlabel={$t$},ylabel style={rotate=-90},xtick={0,1,2},xticklabels={$0$,$\frac{T}{2}$,$T$},ytick={-1,0,1},yticklabels={$-V_0$,$0$,$V_0$}]
\addplot[] plot coordinates{(-0.25,-1) (0,-1) (0,1) (1,1) (1,-1) (2,-1) (2,1) (3,1) (3,-1) (3.25,-1)};
\end{axis}
\end{tikzpicture}
\caption*{(الف) مستطیل موج۔}
\end{subfigure}%
\begin{subfigure}{0.5\textwidth}
\centering
\begin{tikzpicture}
\begin{axis}[small,xlabel={$t$},ylabel style={rotate=-90},xtick={-1,0,1,2,3,4},xticklabels={$-\frac{T}{4}$,$0$,$\frac{T}{4}$,$\frac{T}{2}$,$\frac{3}{4}T$,$T$},ytick={-1,0,1},yticklabels={$-V_0$,$0$,$V_0$}]
\addplot[] plot coordinates{(-2,0)(-1,-1) (1,1) (3,-1) (5,1) };
\end{axis}
\end{tikzpicture}
\caption*{(ب) تکونی موج۔}
\end{subfigure}
\caption{مشق \حوالہ{مشق_فوریئر_مشق_مستطیل} اور مشق \حوالہ{مشق_فوریئر_مشق_تکونی} کے امواج۔}
\label{شکل_فوریئر_مشق_مستطیل}
\end{figure}

جواب:
$f(t)=\frac{4V}{\pi}\left[\sin \omega_0 t+\frac{1}{3}\sin(3\omega_0 t)+\frac{1}{5}\sin(5\omega_0 t)+\cdots\right]$
\انتہا{مشق}
%=======================
\ابتدا{مشق}\شناخت{مشق_فوریئر_مشق_تکونی}
شکل \حوالہ{شکل_فوریئر_مشق_مستطیل}-ب میں دکھائے گئے تکونی موج کی فوریئر تسلسل حاصل کریں۔پہلے \عددی{-\tfrac{T}{4} \le t \le \tfrac{T}{4}} اور 
\عددی{\tfrac{T}{4} \le t \le\tfrac{3T}{4}} سیدھے حصوں کے مساوات حاصل کریں۔

جوابات:\عددی{f_1(t)=\tfrac{4V_0}{T}t}، \عددی{f_2(t)=2V_0(1-2\tfrac{t}{T})}، \\
$f(t)=\tfrac{8V_0}{\pi^2}\left[\sin\omega_0 t-\tfrac{1}{3^2}\sin(3\omega_0 t)+\tfrac{1}{5^2}\sin(5\omega_0 t)-\cdots\right]$
\انتہا{مشق}
%=========================
\ابتدا{مشق}\شناخت{مشق_فوریئر_مختلف_تفاعل_مشق}
شکل \حوالہ{شکل_فوریئر_مختلف_تفاعل_مشق} میں دیے تفاعل کی فوریئر تسلسل حاصل کریں۔
\begin{figure}
\centering
\begin{tikzpicture}
\begin{axis}[xtick={0,1,2,3,4},xticklabels={$0$,$1$,$2$,$3$,$4$},ytick={0,1,2},yticklabels={$0$,$1$,$2$}]
\addplot[] plot coordinates {(0,0) (1,0) (1,1) (2,1) (2,2) (3,2) (3,0) (4,0) (4,1) (5,1) (5,2) (6,2) (6,0)};
\end{axis}
\end{tikzpicture}
\caption{مشق \حوالہ{مشق_فوریئر_مختلف_تفاعل_مشق} کا تفاعل۔}
\label{شکل_فوریئر_مختلف_تفاعل_مشق}
\end{figure}

جواب:
\begin{align*}
1-\tfrac{3}{\pi}\left[\sin \omega_0 t+\tfrac{1}{2}\sin (2\omega_0 t)+\tfrac{1}{4}\sin(4\omega_0 t)+\tfrac{1}{5}\sin(5\omega_0 t)+\tfrac{1}{7}\sin(7\omega_0 t)+\cdots\right]
\end{align*}
\انتہا{مشق}
%======================================================================
%===============================

دوری سمتیہ کا حقیقی جزو اصل تفاعل ہوتا ہے لہٰذا دوری سمتیہ \عددی{(a_m-jb_m)e^{jm\omega t}} درج ذیل حقیقی تفاعل کو ظاہر کرتی ہے۔
\begin{gather}
\begin{aligned}\label{مساوات_فوریئر_دوری_سمتی_الف}
\left. (a_m-jb_m)e^{jm\omega_ t}\right|_{\text{حقیقی}}&=\left. (a_m-jb_m)[\cos(m\omega_0 t)+j\sin (m\omega_ t) ] \right|_{\text{حقیقی}}\\
&=\left. a_m\cos(m\omega_0 t)+ja_m\sin(m\omega_0 t)-jb_m\cos(m\omega_0 t)+b_m\sin (m\omega_0 t)\right|_{\text{حقیقی}}\\
&=a_m\cos (m\omega_0 t)+b_m\sin (m\omega_0 t)
\end{aligned}
\end{gather}
مساوات \حوالہ{مساوات_فوریئر_دوری_سمتی_الف} کو استعمال کرتے ہوئے  مساوات \حوالہ{مساوات_فوریئر_تسلسل_سائن_نما_الف} کے فوریئر تسلسل کو
\begin{gather}
\begin{aligned}
f(t)&=a_0+\sum_{n=1}^{\infty} \left. (a_n-jb_n)e^{jn\omega_0 t}\right|_{\text{حقیقی}}\\
&=a_0+\sum_{n=1}^{\infty} \left. \kx{D}_n e^{jn\omega_0 t}\right|_{\text{حقیقی}}
\end{aligned}
\end{gather}
لکھا جا سکتا ہے  جہاں
\begin{align}
\kx{D}_n=D_n\phase{\theta_n}=a_n-jb_n
\end{align}
کے برابر ہے۔
%================================

\حصہ{قوت نمائی فوریئر تسلسل}
فوریئر تسلسل کی قوت نمائی صورت درج ذیل ہے۔
\begin{align}\label{مساوات_فوریئر_قوت_نمائی_الف}
f(t)=\sum_{n=-\infty}^{\infty} \kx{c}_ne^{jn\omega_0 t}
\end{align}
مساوات \حوالہ{مساوات_فوریئر_قوت_نمائی_الف} سے مساوات \حوالہ{مساوات_فوریئر_تسلسل_سائن_نما_الف} حاصل کرتے ہیں۔درج بالا کو پھیلا کر لکھتے ہیں۔
\begin{multline*}
f(t)=\cdots+\kx{c}_{-3}e^{-3j\omega_0 t}+\kx{c}_{-2}e^{-2j\omega_0 t}+\kx{c}_{-1}e^{-1j\omega_0 t}+\kx{c}_0\\
+\kx{c}_{1}e^{1j\omega_0 t}+\kx{c}_{2}e^{2j\omega_0 t}+\kx{c}_{3}e^{3j\omega_0 t}+\cdots
\end{multline*}
اس کو ترتیب دے کر جوڑیوں کی صورت میں لکھتے ہیں
\begin{multline*}
f(t)=\kx{c}_0+\kx{c}_{1}e^{1j\omega_0 t}+\kx{c}_{-1}e^{-1j\omega_0 t}+\kx{c}_{2}e^{2j\omega_0 t}+\kx{c}_{-2}e^{-2j\omega_0 t}+\kx{c}_{3}e^{3j\omega_0 t}+\kx{c}_{-3}e^{-3j\omega_0 t}\cdots
\end{multline*}
جس کو مجموعے کی صورت میں لکھا جا سکتا ہے۔
\begin{gather}
\begin{aligned}
f(t)&=\kx{c}_0+\sum_{n=1}^{\infty} \kx{c}_ne^{jn\omega_0 t}+\kx{c}_{-n} e^{-jn\omega_0 t}\\
&=\kx{c}_0+\sum_{n=1}^{\infty} \kx{c}_n[\cos (n\omega_0 t)+j\sin(n\omega_0 t)]+\kx{c}_{-n}[\cos (n\omega_0 t)-j\sin(n\omega_0 t)]\\
&=\kx{c}_0+\sum_{n=1}^{\infty}(\kx{c}_n+\kx{c}_{-n})\cos(n\omega_0 t)+j(\kx{c}_n-\kx{c}_{-n})\sin(n\omega_0t)
\end{aligned}
\end{gather}
درج بالا مساوات اور مساوات \حوالہ{مساوات_فوریئر_تسلسل_سائن_نما_الف} صرف اس صورت برابر ہوں گے جب
\begin{gather}
\begin{aligned}\label{مساوات_فوریئر_مستقل_تعلق}
a_n&=\kx{c}_n+\kx{c}_{-n}\\
b_n&=j(\kx{c}_n-\kx{c}_{-n})
\end{aligned}
\end{gather}
ہوں۔اب تصور کریں کہ \عددی{\kx{c}_n=e+jf} اور \عددی{\kx{c}_{-n}=g+jh} ہیں تب درج بالا کے تحت
\begin{align*}
a_n&=e+jf+g+jh\\
b_n&=j(e+jf-g-jh)
\end{align*}
ہو گا۔اب \عددی{a_n} اور \عددی{b_n} حقیقی قیمتیں ہیں لہٰذا پہلی مساوات میں \عددی{f=-h} ہو گا اور دوسری مساوات میں \عددی{e=g} ہو گا۔اس طرح
\begin{align*}
a_n=e+g=2e\\
b_n=h-f=2h
\end{align*}
اور
\begin{align*}
\kx{c}_n&=e+jf\\
\kx{c}_{-n}&=e-jf
\end{align*}
ہوں گے۔آپ نے دیکھا کہ \عددی{\kx{c}_n} اور \عددی{\kx{c}_{-n}} آپس میں مخلوط جوڑی ہے یعنی
\begin{align}
\kx{c}_{-n}=\kx{c}^*_n
\end{align}
مساوات \حوالہ{مساوات_فوریئر_مستقل_تعلق} سے درج ذیل لکھا جا سکتا ہے۔
\begin{align}
2\kx{c}_n=a_n-jb_n
\end{align}
%====================================
فوریئر تسلسل کے تین اقسام کے عددی سر کا تعلق درج ذیل ہے۔
\begin{align}
\kx{D}_n=D_n\phase{\theta_n}=2\kx{c}_n=a_n-jb_n
\end{align}

قوت نمائی فوریئر تسلسل کا عددی سر \عددی{\kx{c}_m} حاصل کرنے کی خاطر مساوات \حوالہ{مساوات_فوریئر_قوت_نمائی_الف} کے دونوں اطراف کو \عددی{e^{-jm\omega_0 t}} سے ضرب دیتے ہوئے \عددی{0 \le t \le T_0} پر ان کا تکمل حاصل کیا جاتا ہے۔
\begin{align}\label{مساوات_فوریئر_عددی_سر_قوت_نما_الف}
\int_0^{T_0}f(t)e^{-jm\omega_0 t} \dif t=\sum_{n=-\infty}^{\infty} \int_0^{T_0}\kx{c}_ne^{j(n-m)\omega_0 t} \dif t
\end{align}
اگر \عددی{n\ne m} ہو تب
\begin{align*}
\int_0^{T_0} \kx{c}_ne^{j(n-m)\omega_0 t} \dif t&=\left. \frac{\kx{c}_ne^{j(n-m)\omega_0 t}}{j(n-m)\omega_0}\right|_0^{T_0}\\
&=\frac{\kx{c}_n \left[e^{j(n-m)2\pi}-e^{0}\right]}{j(n-m)\omega_0}\\
&=0
\end{align*}
ملتا ہے جہاں آخری قدم پر \عددی{e^{j(n-m)2\pi}=\cos[(n-m)2\pi]+j\sin[(n-m)2\pi]=1} اور \عددی{e^0=1} کا استعمال کیا گیا ہے۔اس کے برعکس \عددی{n=m} کی صورت میں \عددی{\kx{c}_n} کو \عددی{\kx{c}_m} لکھا جا سکتا ہے اور
\begin{align*}
\int_0^{T_0} \kx{c}_m \dif t=T_0 \kx{c}_m
\end{align*}
ہو گا لہٰذا مساوات \حوالہ{مساوات_فوریئر_عددی_سر_قوت_نما_الف} کو درج ذیل لکھا جا سکتا ہے۔
\begin{align}\label{مساوات_فوریئر_عددی_سر_قوت_نما_ب}
\kx{c}_m=\frac{1}{T_0} \int_0^{T_0} f(t) e^{-jm\omega_0 t} \dif t
\end{align}
%======================================================================
\ابتدا{مثال}\شناخت{مثال_فوریئر_دندان_دوبارہ}
ہم شکل \حوالہ{شکل_فوریئر_مشق_مستطیل}-الف کے مستطیل تفاعل کا تکونیاتی فوریئر تسلسل حاصل کر چکے ہیں۔آئیں اس کی قوت نمائی فوریئر تسلسل حاصل کریں۔تفاعل کو شکل \حوالہ{شکل_فوریئر_دندان_دوبارہ} میں دوبارہ دکھایا گیا ہے۔
\begin{figure}
\centering
\begin{tikzpicture}
\begin{axis}[small,xlabel={$t$},ylabel style={rotate=-90},xtick={0,1,2},xticklabels={$0$,$\frac{T}{2}$,$T$},ytick={-1,0,1},yticklabels={$-V_0$,$0$,$V_0$}]
\addplot[] plot coordinates{(-0.25,-1) (0,-1) (0,1) (1,1) (1,-1) (2,-1) (2,1) (3,1) (3,-1) (3.25,-1)};
\end{axis}
\end{tikzpicture}
\caption{مثال \حوالہ{مثال_فوریئر_دندان_دوبارہ} کا تفاعل۔}
\label{شکل_فوریئر_دندان_دوبارہ}
\end{figure}%

حل: مساوات \حوالہ{مساوات_فوریئر_عددی_سر_قوت_نما_ب} استعمال کرتے ہوئے  \عددی{\kx{c}_0} حاصل کرتے ہیں۔
\begin{align*}
\kx{c}_0&=\frac{1}{T}\int_0^{T} f(t) \dif t\\
&=\frac{1}{T}\int_0^{\frac{T}{2}} V_0 \dif t+\frac{1}{T}\int_{\frac{T}{2}}^{T}(-V_0)\dif t\\
&=0
\end{align*}
اسی طرح \عددی{\kx{c}_m} حاصل کرتے ہیں۔
\begin{align*}
\kx{c}_m&=\frac{1}{T} \int_0^{T} f(t) e^{-jm\omega_0 t} \dif t\\
&=\frac{V_0}{T}\int_0^{\frac{T}{2}} e^{-jm\omega_0 t} \dif t-\frac{V_0}{T}\int_{\frac{T}{2}}^T e^{-jm\omega_0 t} \dif t\\
&=\left. \frac{V_0 e^{-jm\omega_0 t}}{-jm\omega_0 T}\right|_{0}^{\frac{T}{2}}-\left. \frac{V_0 e^{-jm\omega_0 t}}{-jm\omega_0 T}\right|_{\frac{T}{2}}^{T}\\
&=\frac{jV_0}{m\pi}(\cos m\pi -1)\quad \substack{-\infty \le m \le \infty\\ m \ne 0}
\end{align*}
جس سے درج ذیل لکھا جا سکتا ہے۔
\begin{align*}
\kx{c}_1&=\kx{c}^*_1=-\frac{j2V_0}{\pi}\\
\kx{c}_2&=\kx{c}^*_2=0\\
\kx{c}_3&=\kx{c}^*_3=-\frac{j2V_0}{3\pi}\\
\kx{c}_4&=\kx{c}^*_4=0\\
\kx{c}_5&=\kx{c}^*_5=-\frac{j2V_0}{5\pi}
\end{align*}
یوں شکل میں دیے مستطیل تفاعل کی فوریئر تسلسل درج ذیل ہو گی۔
\begin{align} \label{مساوات_فوریئر_مستطیل_دوسرا_جواب}
f(t)&=\sum_{\substack{n=-\infty \\ n=\text{طاق}\\ n \ne 0} }^{\infty} -\frac{j2V_0}{n\pi}e^{jn\omega_0 t}
\end{align}
\انتہا{مثال}
%===================================
\ابتدا{مشق}
مساوات \حوالہ{مساوات_فوریئر_مستطیل_دوسرا_جواب} میں \عددی{\kx{c}_ne^{jn\omega_0 t}+\kx{c}^*_ne^{-jn\omega_0t}} اکٹھے کرتے ہوئے مشق \حوالہ{مشق_فوریئر_مشق_مستطیل} میں دیا جواب حاصل کریں۔
\انتہا{مشق}
%======================================
\ابتدا{مشق}\شناخت{مشق_فوریئر_جفت_تفاعل_قوت_نمائی_تسلسل_الف}
شکل \حوالہ{شکل_فوریئر_جفت_تفاعل_قوت_نمائی_تسلسل_الف}-الف میں دیے تفاعل کے قوت نمائی فوریئر تسلسل کے عددی سر معلوم کریں۔
\begin{figure}
\centering
\begin{subfigure}{0.5\textwidth}
\centering
\begin{tikzpicture}
\begin{axis}[small,xtick={-4,-2,-1,0,1,2,4,5,7,8},xticklabels={$-4$,$-2$,$-1$,$0$,$1$,$2$,$4$,$5$,$7$,$8$},ytick={2,4},yticklabels={$2$,$4$}]
\addplot[] plot coordinates { (-4,0) (-2,0) (-2,4) (-1,4) (-1,2) (1,2) (1,4) (2,4) (2,0) (4,0) (4,4) (5,4) (5,2) (7,2) (7,4) (8,4) (8,0)};
\end{axis}
\end{tikzpicture}
\caption*{(الف)}
\end{subfigure}%
\begin{subfigure}{0.5\textwidth}
\centering
\begin{tikzpicture}
\begin{axis}[small,xtick={0,1,3,4,6,7},xticklabels={$0$,$1$,$3$,$4$,$6$,$7$},ytick={0,1},yticklabels={$0$,$1$}]
\addplot[] plot coordinates { (0,0) (0,1) (1,1) (1,0) (3,0) (3,1) (4,1) (4,0) (6,0) (6,1) (7,1) (7,0) };
\end{axis}
\end{tikzpicture}
\caption*{(ب)}
\end{subfigure}%
\caption{مشق \حوالہ{مشق_فوریئر_جفت_تفاعل_قوت_نمائی_تسلسل_الف} اور مشق \حوالہ{مشق_فوریئر_جفت_تفاعل_قوت_نمائی_تسلسل_ب} کے تفاعل۔}
\label{شکل_فوریئر_جفت_تفاعل_قوت_نمائی_تسلسل_الف}
\end{figure}

جوابات:
\begin{align*}
\kx{c}_0&=\frac{1}{2}\\
\kx{c}_n&=\frac{2}{n\pi}\left[2\sin \frac{2\pi n}{3}-\sin\frac{n\pi}{3}\right]
\end{align*}
\انتہا{مشق}
%================

\ابتدا{مشق}\شناخت{مشق_فوریئر_جفت_تفاعل_قوت_نمائی_تسلسل_ب}
شکل \حوالہ{شکل_فوریئر_جفت_تفاعل_قوت_نمائی_تسلسل_الف}-ب میں دیے تفاعل کے قوت نمائی فوریئر تسلسل کے عددی سر معلوم کریں۔

جوابات:
\begin{align*}
\kx{c}_0&=\frac{1}{3}\\
\kx{c}_n&=\frac{1-e^{-j\frac{2}{3}n\pi}}{j2n\pi}
\end{align*}
\انتہا{مشق}
%======================================================================
\حصہ{تشاکل تفاعل}
آپ نے مختلف تفاعل کے فوریئر تسلسل دیکھے۔ان میں کئی ایسے تھے جن کے یا تمام \عددی{a_m} اور یا تمام \عددی{b_m} صفر کے برابر تھے۔آئیں اس کی وجہ سمجھیں اور تکملات حل کرنے سے پہلے یہ دریافت کرنا سیکھیں کہ آیا فوریئر تسلسل میں \عددی{a_m} اور یا تمام \عددی{b_m} صفر کے برابر ہوں گے۔فوریئر تسلسل کے ارکان کا دارومدار تفاعل کی شکل و صورت پر ہے۔ تین قسم کے تشاکل تفاعل پائے جاتے ہیں۔ان پر باری باری غور کرتے ہیں۔

\جزوحصہ{جفت تشاکل تفاعل}
جفت تفاعل سے مراد ایسا تفاعل ہے جو درج ذیل مساوات پر پورا اترتا ہو۔
\begin{align}
f(t)=f(-t)
\end{align}
جفت  تفاعل عمودی محور کے دونوں اطراف یکساں دکھائی دیتا ہے۔جفت تفاعل کی اہم مثال \عددی{\cos(n\omega_0 t)} ہے۔آپ جانتے ہیں کہ \عددی{\cos(\theta)=\cos(-\theta)} ہوتا ہے لہٰذا یہ جفت تفاعل ہے۔شکل \حوالہ{شکل_فوریئر_مستطیل_موج_الف}-الف بھی جفت تفاعل ہے۔ آئیں جفت تفاعل کے فوریئر تسلسل کے عددی سر حاصل کریں۔

مساوات \حوالہ{مساوات_فوریئر_عددی_سر_ت} میں تکمل کو \عددی{-\tfrac{T_0}{2}\le  t \le \tfrac{T_0}{2}} لیتے ہوئے یہاں دوبارہ پیش کرتے ہیں۔
\begin{gather}
\begin{aligned}\label{مساوات_فوریئر_عددی_سر_جفت}
a_0&=\frac{1}{T_0}\int_{-\frac{T_0}{2}}^{\frac{T_0}{2}}f(t) \dif t\\
a_m&=\frac{2}{T_0}\int_{-\frac{T_0}{2}}^{\frac{T_0}{2}}f(t) \cos (m \omega_0 t) \dif t\\
b_m&=\frac{2}{T_0}\int_{-\frac{T_0}{2}}^{\frac{T_0}{2}}f(t) \sin (m \omega_0 t) \dif t
\end{aligned}
\end{gather}
\عددی{a_0} کی مساوات کو درج ذیل لکھا جا سکتا ہے۔
\begin{align*}
a_0&=\frac{1}{T_0}\int_{-\frac{T_0}{2}}^0 f(t) \dif t+\frac{1}{T_0}\int_0^{\frac{T_0}{2}} f(t) \dif t
\end{align*}
ان میں پہلے تکمل میں متغیرہ کو تبدیل کرتے ہوئے \عددی{t=-x} لکھنے سے \عددی{f(t)=f(-x)} اور \عددی{\dif t=-\dif x} لکھے جائیں گے اور تکمل کے حدود \عددی{\tfrac{T_0}{2} \le t \le 0} ہوں گے۔چونکہ تفاعل جفت ہے لہٰذا \عددی{f(-x)=f(x)} ہو گا۔یوں درج ذیل لکھا جا سکتا ہے
\begin{gather}
\begin{aligned}
a_0&=-\frac{1}{T_0}\int_{\frac{T_0}{2}}^0 f(-x) \dif x+\frac{1}{T_0}\int_0^{\frac{T_0}{2}} f(t) \dif t\\
&=\frac{1}{T_0}\int_0^{\frac{T_0}{2}} f(x) \dif x+\frac{1}{T_0}\int_0^{\frac{T_0}{2}} f(t) \dif t\\
&=\frac{2}{T_0}\int_0^{\frac{T_0}{2}} f(t) \dif t
\end{aligned}
\end{gather}
جہاں آخری قدم پر دونوں تکمل میں صرف متغیرات کی علامت مختلف ہے لہٰذا ان کی قیمتیں برابر ہیں۔

\عددی{a_m} کو بھی اسی طرح حاصل کرتے ہیں۔
\begin{gather}
\begin{aligned}
a_m&=\frac{2}{T_0}\int_{-\frac{T_0}{2}}^{0}f(t) \cos (m \omega_0 t) \dif t+\frac{2}{T_0}\int_{0}^{\frac{T_0}{2}}f(t) \cos (m \omega_0 t) \dif t\\
&=-\frac{2}{T_0}\int_{\frac{T_0}{2}}^{0}f(-x) \cos (-m \omega_0 x) \dif x+\frac{2}{T_0}\int_{0}^{\frac{T_0}{2}}f(t) \cos (m \omega_0 t) \dif t\\
&=\frac{2}{T_0}\int_0^{\frac{T_0}{2}}f(x) \cos (m \omega_0 x) \dif x+\frac{2}{T_0}\int_{0}^{\frac{T_0}{2}}f(t) \cos (m \omega_0 t) \dif t\\
&=\frac{4}{T_0}\int_{0}^{\frac{T_0}{2}}f(t) \cos (m \omega_0 t) \dif t
\end{aligned}
\end{gather}
آخر میں \عددی{b_m} کو اسی ترکیب سے حاصل کرتے ہیں
\begin{gather}
\begin{aligned}
b_m&=\frac{2}{T_0}\int_{-\frac{T_0}{2}}^{0}f(t) \sin (m \omega_0 t) \dif t+\frac{2}{T_0}\int_{0}^{\frac{T_0}{2}}f(t) \sin (m \omega_0 t) \dif t\\
&=-\frac{2}{T_0}\int_{\frac{T_0}{2}}^{0}f(-x) \sin(-m \omega_0 x) \dif x+\frac{2}{T_0}\int_{0}^{\frac{T_0}{2}}f(t) \sin (m \omega_0 t) \dif t\\
&=\frac{2}{T_0}\int_0^{\frac{T_0}{2}}f(x) \sin (-m \omega_0 x) \dif x+\frac{2}{T_0}\int_{0}^{\frac{T_0}{2}}f(t) \cos (m \omega_0 t) \dif t\\
&=0
\end{aligned}
\end{gather}
جہاں آخری قدم پر \عددی{\sin(-m\omega_0 t)=-\sin(m\omega_0 t)} کا استعمال کیا گیا ہے۔

آپ نے دیکھا کہ جفت تفاعل کی صورت میں فوریئر تسلسل کے \عددی{b_m=0} ہیں لہٰذا انہیں حاصل کرنے کی ضرورت نہیں ہے۔
%================================
\جزوحصہ{طاق تشاکل تفاعل}
طاق تفاعل سے مراد ایسا تفاعل ہے جو درج ذیل مساوات پر پورا اترتا ہو۔
\begin{align}
f(-t)=-f(t)
\end{align}
طاق تفاعل کی مثال \عددی{\sin(m\omega_0 t)} ہے۔ہم جفت تفاعل کی طرح طاق تفاعل کے عددی سر حاصل کرتے ہوئے درج ذیل نتائج پر پہنچتے ہیں۔
\begin{gather}
\begin{aligned}
a_0&=0\\
a_m&=0\\
b_m&=\frac{4}{T_0}\int_{0}^{\frac{T_0}{2}}f(t) \sin (m \omega_0 t) \dif t
\end{aligned}
\end{gather}
یوں طاق تفاعل کے فوریئر تسلسل کے صرف \عددی{b_m} عددی سر حاصل کرنے کی ضرورت پیش آئے گی۔
%=================================

\ابتدا{مثال}\شناخت{مثال_فوریئر_جفت_مستطیل}
شکل \حوالہ{شکل_فوریئر_جفت_طاق_مستطیل}-الف میں جفت تفاعل دکھایا گیا ہے۔اس کے فوریئر تکونیاتی تسلسل کے عددی سر حاصل کریں۔
\begin{figure}
\centering
\begin{subfigure}{0.5\textwidth}
\centering
\begin{tikzpicture}
\begin{axis}[small,xtick={-1,0,1,3,5},xticklabels={$-1$,$0$,$1$,$3$,$5$},ytick={0,1},yticklabels={$0$,$1$}]
\addplot[] plot coordinates {(-1,0) (-1,1) (1,1) (1,0) (3,0) (3,1) (5,1) (5,0)};
\end{axis}
\end{tikzpicture}
\caption*{(الف)}
\end{subfigure}%
\begin{subfigure}{0.5\textwidth}
\centering
\begin{tikzpicture}
\begin{axis}[small,xtick={0,2,4,6},xticklabels={$0$,$2$,$4$,$6$},ytick={0,1},yticklabels={$0$,$1$}]
\addplot[] plot coordinates {(0,0) (0,1) (2,1) (2,0) (4,0) (4,1) (6,1) (6,0)};
\end{axis}
\end{tikzpicture}
\caption*{(ب)}
\end{subfigure}%
\caption{مثال \حوالہ{مثال_فوریئر_جفت_مستطیل} اور مثال \حوالہ{مثال_فوریئر_طاق_مستطیل} کے اشکال۔}
\label{شکل_فوریئر_جفت_طاق_مستطیل}
\end{figure}

حل:چونکہ دیا تفاعل جفت ہے لہٰذا \عددی{b_m=0} ہوں گے۔بقایا عددی سر دریافت کرتے ہیں یعنی
\begin{align*}
a_0&=\frac{1}{4}\int_{-1}{1} \dif t\\
&=\frac{1}{2}
\end{align*}
اور
\begin{align*}
a_n&=\frac{2}{4}\int_{-1}^{1} \cos (n\omega_0 t) \dif t\\
&=\frac{\sin \frac{n\pi}{2}}{\frac{n\pi}{2}}
\end{align*}
\انتہا{مثال}
%=================================

\ابتدا{مثال}\شناخت{مثال_فوریئر_طاق_مستطیل}
شکل \حوالہ{شکل_فوریئر_جفت_طاق_مستطیل}-ب میں طاق تفاعل دکھایا گیا ہے۔اس کے فوریئر تکونیاتی تسلسل کے عددی سر حاصل کریں۔

حل:چونکہ دیا تفاعل طاق ہے لہٰذا \عددی{a_m=0} ہوں گے۔بقایا عددی سر دریافت کرتے ہیں یعنی
\begin{align*}
a_0&=\frac{1}{4}\int_{0}{2} \dif t\\
&=\frac{1}{2}
\end{align*}
اور
\begin{align*}
b_n&=\frac{2}{4}\int_{0}^{2} \sin (n\omega_0 t) \dif t\\
&=\frac{1-\cos n\pi}{n\pi}
\end{align*}
\انتہا{مثال}
%=================================

\حصہ{منتقلی وقت}
فرض کریں کہ ایک تفاعل جس کی فوریئر تسلسل درج ذیل ہے
\begin{align}
f(t)=\sum_{n=-\infty}^{\infty} \kx{c}_n e^{jn\omega_0 t}
\end{align}
کو وقت کے لحاض سے منتقل کیا جاتا ہے۔تفاعل \عددی{f(t)} کو \عددی{t_0} سیکنڈ تاخیر سے  \عددی{f(t-t_0)} لکھا جاتا ہے۔
\begin{gather}
\begin{aligned}\label{مساوات_فوریئر_تبادلہ_وقت}
f(t-t_0)&=\sum_{n=-\infty}^{\infty} \kx{c}_n e^{jn\omega_0 (t-t_0)}\\
&=\sum_{n=-\infty}^{\infty} \left(\kx{c}_n e^{-jn\omega_0 t_0}\right) e^{jn\omega_0 t}
\end{aligned}
\end{gather}
چونکہ \عددی{e^{-jn\omega_0 t_0}} سے مراد زاویائی فاصلہ ہے لہٰذا وقت میں منتقل تفاعل \عددی{f(t-t_0)} کے فوریئر عددی سر اصل تفاعل \عددی{f(t)} کے عددی سر ہوتے ہیں جن میں تعدد کے راست متناسب زاویائی ہٹاو \عددی{e^{-jn\omega_0 t_0}}  پایا جاتا ہے۔ اس طرح وقتی دائرہ کار میں تبادلے سے  تعددی دائرہ کار میں مراد  زاویائی تبادلہ ہے۔

%============
\ابتدا{مثال}\شناخت{مثال_فوریئر_دندان_منتقل_دوبارہ}
مثال \حوالہ{مثال_فوریئر_دندان_دوبارہ} میں ہم شکل \حوالہ{شکل_فوریئر_دندان_دوبارہ} کے تفاعل \عددی{f(t)} کا قوت نمائی فوریئر تسلسل حاصل کر چکے ہیں۔ اس تفاعل کو \عددی{\tfrac{T}{4}} بائیں منتقل کرتے ہوئے شکل \حوالہ{شکل_فوریئر_دندان_منتقل_دوبارہ} یعنی \عددی{f(t+\tfrac{T}{4})} حاصل ہوتا ہے جس کی قوت نمائی فوریئر تسلسل درکار ہے۔
\begin{figure}
\centering
\begin{tikzpicture}
\begin{axis}[small,xlabel={$t$},ylabel style={rotate=-90},xtick={-0.5,0,0.5,1.5},xticklabels={$-\frac{T}{4}$,$0$,$\frac{T}{4}$,$\frac{3T}{4}$},ytick={-1,0,1},yticklabels={$-V_0$,$0$,$V_0$}]
\addplot[] plot coordinates{(-0.75,-1) (-0.5,-1) (-0.5,1) (0.5,1) (0.5,-1) (1.5,-1) (1.5,1) (2.5,1) (2.5,-1) (2.75,-1)};
\end{axis}
\end{tikzpicture}
\caption{مثال \حوالہ{مثال_فوریئر_دندان_منتقل_دوبارہ} کا تفاعل۔}
\label{شکل_فوریئر_دندان_منتقل_دوبارہ}
\end{figure}%
حل: اس کو تبادلہ وقت کے کلیے سے حل کرتے ہیں۔مساوات \حوالہ{مساوات_فوریئر_تبادلہ_وقت} کے ذریعہ نئے عددی سر حاصل کرتے ہیں۔چونکہ \عددی{t_0=-\tfrac{T}{4}} ہے لہٰذا زاویائی ہٹاو
\begin{align*}
-jn\omega_0 t_0=jn \frac{2\pi}{T} \frac{T}{4}=jn\frac{\pi}{2}
\end{align*}
ہو گا۔یوں بنیادی ہارمونی رکن کے عددی سر میں \عددی{90^{\circ}} کا زاویائی ہٹاو پایا جائے گا۔ مساوات \حوالہ{مساوات_فوریئر_مستطیل_دوسرا_جواب} میں ان زاویائی ہٹاو کو شامل کرتے ہوئے فوریئر تسلسل لکھتے ہیں۔
\begin{align}
f\left(t+\frac{T}{4}\right)&=\sum_{\substack{n=-\infty \\ n=\text{طاق}\\ n \ne 0} }^{\infty} -\frac{j2V_0}{n\pi}e^{jn\left(\omega_0 +\frac{\pi}{2}\right)t}
\end{align}
\انتہا{مثال}
%============

\حصہ{تخلیق موج}
دو یا دو سے زیادہ امواج کا مجموعہ لیتے ہوئے دیگر امواج حاصل کئے جا سکتے ہیں۔یوں شکل \حوالہ{شکل_فوریئر_تخلیق_موج}-الف اور شکل-ب مل کر شکل-پ دیتے ہیں۔انفرادی امواج کے فوریئر تسلسل کا مجموعہ حاصل موج کی فوریئر تسلسل دے گی۔ فوریئر تسلسل میں تعددی ارکان کا دارومدار وقت کے لحاض سے موج کی صورت پر منحصر ہوتا ہے نا کہ موج کی حتمی قیمت پر۔یوں جس نسبت سے موج کا حیطہ تبدیل کیا جائے اسی نسبت سے تسلسل کو ضرب دیتے ہوئے کم یا زیادہ حیطے کی موج کا تسلسل حاصل کیا جا سکتا ہے۔ 
\begin{figure}
\centering
\begin{subfigure}{0.5\textwidth}
\centering
\begin{tikzpicture}
\begin{axis}[small,ytick={0,1},yticklabels={$0$,$1$},xtick={-2,0,2,4,8},xticklabels={$-2$,$0$,$2$,$4$,$8$},ymax=2.2,xmin=-2.5,xmax=8.5]
\addplot[] plot coordinates {(-2.5,0)(-2,0) (-2,1) (2,1) (2,0) (4,0) (4,1) (8,1) (8,0)(8.5,0)};
\end{axis}
\end{tikzpicture}
\caption*{(الف)}
\end{subfigure}%
\begin{subfigure}{0.5\textwidth}
\centering
\begin{tikzpicture}
\begin{axis}[small,ytick={0,1},yticklabels={$0$,$1$},xtick={-1,0,1,5,7},xticklabels={$-1$,$0$,$1$,$5$,$7$},ymax=2.2,xmin=-2.5,xmax=8.5]
\addplot[] plot coordinates {(-2.5,0)(-1,0) (-1,1) (1,1) (1,0) (5,0) (5,1) (7,1) (7,0) (8.5,0)};
\end{axis}
\end{tikzpicture}
\caption*{(ب)}
\end{subfigure}
\begin{subfigure}{0.5\textwidth}
\centering
\begin{tikzpicture}
\begin{axis}[small,ytick={0,1,2},yticklabels={$0$,$1$,$2$},xtick={-2,-1,0,1,2,4,5,7,8},xticklabels={$-2$,$-1$,$0$,$1$,$2$,$4$,$5$,$7$,$8$},ymax=2.2,xmin=-2.5,xmax=8.5]
\addplot[] plot coordinates {(-2.5,0)(-2,0) (-2,1) (-1,1) (-1,2) (1,2) (1,1) (2,1) (2,0) (4,0) (4,1) (5,1) (5,2) (7,2) (7,1) (8,1) (8,0)(8.5,0)};
\end{axis}
\end{tikzpicture}
\caption*{(پ)}
\end{subfigure}
\caption{دو یا دو سے زیادہ امواج کے جمع و منفی سے نئی موج کی تخلیق۔}
\label{شکل_فوریئر_تخلیق_موج}
\end{figure}

فوریئر تسلسل میں \عددی{\sum a_n \cos (n\omega_0 t)} جفت موج کو ظاہر کرتی ہے جبکہ \عددی{\sum b_n \sin(n\omega_0 t)} طاق موج کو ظاہر کرتی ہے لہٰذا کسی بھی موج کو جفت موج اور طاق موج کا مجموعہ تصور کیا جا سکتا ہے۔
%================

\ابتدا{مشق}
شکل \حوالہ{شکل_فوریئر_تخلیق_موج}-الف اور ب کی فوریئر قوت نمائی تسلسل کے عددی سر حاصل کریں۔ان کے مجموعے سے شکل-پ کی تسلسل کے عددی سر حاصل کریں۔

جواب:\عددی{\kx{c}_0=1}، \عددی{\kx{c}_n=\tfrac{\sin\tfrac{2n\pi}{3}+\sin\tfrac{n\pi}{3}}{n\pi}}
\انتہا{مشق}
%======================
\ابتدا{مشق}\شناخت{مشق_تخلیق_موج_ب}
شکل \حوالہ{شکل_تخلیق_موج_ب}-ال اور ب کا مجموعہ شکل-پ ہے۔شکل-الف اور ب کے فوریئر تسلسل حاصل کرتے ہوئے شکل-پ کا تسلسل حاصل کریں۔
\begin{figure}
\centering
\begin{subfigure}{0.5\textwidth}
\centering
\begin{tikzpicture}
\begin{axis}[small,ytick={-1,0,1},yticklabels={$-A$,$0$,$A$},xtick={-1,0,1},xticklabels={$-\frac{T}{2}$,$0$,$\frac{T}{2}$},ymax=2.2,xmin=-1.5,xmax=3]
\addplot[] plot coordinates {(-1.5,0)(-1,-1) (0,1) (1,-1) (2,1) (3,-1)};
\end{axis}
\end{tikzpicture}
\caption*{(الف)}
\end{subfigure}%
\begin{subfigure}{0.5\textwidth}
\centering
\begin{tikzpicture}
\begin{axis}[small,ytick={-1,0,1},yticklabels={$-A$,$0$,$A$},xtick={-1,0,1},xticklabels={$-\frac{T}{2}$,$0$,$\frac{T}{2}$},ymax=2.2,xmin=-1.5,xmax=3]
\addplot[] plot coordinates {(-1.5,-1) (-1,-1) (-1,1) (0,1) (0,-1) (1,-1) (1,1) (2,1) (2,-1) (3,-1)};
\end{axis}
\end{tikzpicture}
\caption*{(ب)}
\end{subfigure}
\begin{subfigure}{0.5\textwidth}
\centering
\begin{tikzpicture}
\begin{axis}[small,ylabel={$f(t)$},ylabel style ={rotate=-90},ytick={-2,0,2},yticklabels={$-2A$,$0$,$2A$},xtick={-1,0,1},xticklabels={$-\frac{T}{2}$,$0$,$\frac{T}{2}$},ymax=2.2,xmin=-1.5,xmax=3]
\addplot[] plot coordinates {(-1.5,-1)(-1,-2)(-1,0)(0,2) (0,0) (1,-2) (1,0) (2,2) (2,0) (3,-2)};
\end{axis}
\end{tikzpicture}
\caption*{(پ)}
\end{subfigure}
\caption{مشق \حوالہ{مشق_تخلیق_موج_ب} کے اشکال۔}
\label{شکل_تخلیق_موج_ب}
\end{figure}

جواب:
$f(t)=\sum_{\substack{n=1 \\ \text{طاق}}}^{\infty} \left[\frac{4A}{n\pi}\sin(n\omega_0 t)-\frac{8A}{n^2\pi^2} \cos(n\omega_0 t)\right]$
\انتہا{مشق}
%==========================
\حصہ{تعددی طیف}
تفاعل \عددی{f(t)} کے عددی سر کی مقدار بالمقابل تعدد کے ترسیم کو \اصطلاح{مقداری طیف}\فرہنگ{مقداری!طیف}\فرہنگ{طیف!مقداری}\حاشیہب{amplitude spectrum}\فرہنگ{spectrum!amplitude}\فرہنگ{amplitude!spectrum} کہتے ہیں جبکہ عددی سر کی زاویائی ہٹاو بالمقابل تعدد کے ترسیم کو \اصطلاح{زاویائی ہٹاو طیف}\فرہنگ{زاویائی ہٹاو!طیف}\فرہنگ{طیف!زاویائی ہٹاو}\حاشیہب{phase spectrum}\فرہنگ{phase!spectrum}\فرہنگ{spectrum!phase} کہتے ہیں۔ چونکہ طیف انفرادی لکیروں پر مشتمل ہے لہٰذا اسے \اصطلاح{انفرادی لکیری طیف}\فرہنگ{انفرادی لکیری طیف}\فرہنگ{طیف!انفرادی لکیری}\حاشیہب{discrete line spectra}\فرہنگ{discrete line spectra}
\فرہنگ{spectrum!discrete line} کہتے ہیں۔انفرادی لکیری طیف تفاعل کی تعددی مواد کے بارے میں معلومات دیتی ہے۔

%=====================
\ابتدا{مثال}\شناخت{مثال_فوریئر_طیف_دیا_گیا_ہے}
مشق \حوالہ{مشق_تخلیق_موج_ب} میں شکل \حوالہ{شکل_تخلیق_موج_ب}-پ کا تسلسل حاصل کیا گیا جو \عددی{A=5} کی صورت میں درج ذیل ہو گا۔
\begin{align*}
f(t)=\sum_{\substack{n=1 \\ \text{طاق}}}^{\infty} \left[\frac{20}{n\pi}\sin(n\omega_0 t)-\frac{40}{n^2\pi^2} \cos(n\omega_0 t)\right]
\end{align*}
تفاعل کی مقداری طیف اور زاویائی ہٹاو طیف درکار ہے۔

حل:چونکہ \عددی{D_n\phase{\theta_n}=a_n-jb_n} ہے لہٰذا پہلے چار \عددی{D_n\phase{\theta_n}} درج ذیل ہوں گے۔
\begin{align*}
D_1\phase{\theta_1}&=-\frac{40}{\pi^2}-j\frac{20}{\pi}=7.5\phase{-122^{\circ}}\\
D_3\phase{\theta_1}&=-\frac{40}{9\pi^2}-j\frac{20}{3\pi}=2.2\phase{-102^{\circ}}\\
D_5\phase{\theta_1}&=-\frac{40}{25\pi^2}-j\frac{20}{5\pi}=1.3\phase{-97^{\circ}}\\
D_7\phase{\theta_1}&=-\frac{40}{49\pi^2}-j\frac{20}{7\pi}=0.91\phase{-95^{\circ}}\\
\end{align*}
مقداری طیف اور زاویائی ہٹاو طیف کو شکل \حوالہ{شکل_فوریئر_طیف_دیا_گیا_ہے} میں دکھایا گیا ہے۔
\begin{figure}
\centering
\begin{subfigure}{0.5\textwidth}
\centering
\begin{tikzpicture}
\begin{axis}[small,axis lines*=middle,ylabel={$D_n$},ylabel style={rotate=-90},xlabel={$\omega_0$},xtick={1,3,5,7},xticklabels={$\omega_0$,$3\omega_0$,$5\omega_0$,$7\omega_0$},ytick={1,2,3,4,5,6,7,8},yticklabels={$1$,$2$,$3$,$4$,$5$,$6$,$7$,$8$},ymin=0,y label style={at={(axis description cs:0,1.1)}}]
\addplot[] plot coordinates {(1,0) (1,7.5)};
\addplot[] plot coordinates {(3,0) (3,2.2)};
\addplot[] plot coordinates {(5,0) (5,1.3)};
\addplot[] plot coordinates {(7,0) (7,0.91)};
\end{axis}
\end{tikzpicture}
\caption*{(الف) مقداری طیف۔}
\end{subfigure}%
\begin{subfigure}{0.5\textwidth}
\centering
\begin{tikzpicture}
\begin{axis}[small,axis lines*=middle,ylabel={$\theta_n$},ylabel style={rotate=-90},xlabel={$\omega_0$},xtick={1,3,5,7},xticklabels={$\omega_0$,$3\omega_0$,$5\omega_0$,$7\omega_0$},ytick={-20,-40,-60,-80,-100,-120,-140},yticklabels={$-20^{\circ}$,$-40^{\circ}$,$-60^{\circ}$,$-80^{\circ}$,$-100^{\circ}$,$-120^{\circ}$,$-140^{\circ}$},ymax=20 ,x tick label style={above},x label style={at={(axis description cs:0.5,1.2)}},x tick style={draw=none},y label style={at={(axis description cs:0,1.05)}}]
\addplot[] plot coordinates {(1,0) (1,-122)};
\addplot[] plot coordinates {(3,0) (3,-102)};
\addplot[] plot coordinates {(5,0) (5,-97)};
\addplot[] plot coordinates {(7,0) (7,-95)};
\end{axis}
\end{tikzpicture}
\caption*{(ب) زاویائی ہٹاو طیف۔}
\end{subfigure}
\caption{مثال \حوالہ{مثال_فوریئر_طیف_دیا_گیا_ہے} کے طیف۔}
\label{شکل_فوریئر_طیف_دیا_گیا_ہے}
\end{figure}
\انتہا{مثال}
%==================
\ابتدا{مشق}\شناخت{مشق_فوریئر_طیف_الف}
شکل \حوالہ{شکل_فوریئر_طیف_الف} میں دیے تفاعل کے \عددی{D_n} عددی سر حاصل کریں۔
\begin{figure}
\centering
\begin{tikzpicture}
\begin{axis}[small,xlabel={$t$},ylabel style={rotate=-90},xtick={-1,0,1,2,3},xticklabels={$-1$,$0$,$1$,$2$,$3$},ytick={0,1},yticklabels={$0$,$1$}]
\addplot[] plot coordinates {(-1,0) (-1,1) (0,0) (0,1) (1,0) (1,1) (2,0) (2,1) (3,0)};
\end{axis}
\end{tikzpicture}
\caption{مشق \حوالہ{مشق_فوریئر_طیف_الف} کا تفاعل۔}
\label{شکل_فوریئر_طیف_الف}
\end{figure}
جوابات:
\begin{align*}
a_0&=\frac{1}{2}\\
D_1&=-\frac{j}{\pi}\\
D_2&=-\frac{j}{2\pi}\\
D_3&=-\frac{j}{3\pi}\\
D_4&=-\frac{j}{4\pi}
\end{align*}
\انتہا{مشق}
%==================
\ابتدا{مشق}\شناخت{مشق_فوریئر_طیف_دیا_گیا}
انفرادی لکیری طیف شکل \حوالہ{شکل_فوریئر_طیف_دیا_گیا} میں دکھائے گئے ہیں۔تفاعل کی تکونیاتی تسلسل لکھیں۔
\begin{figure}
\centering
\begin{subfigure}{0.5\textwidth}
\centering
\begin{tikzpicture}
\begin{axis}[small,ylabel={$D_n$},ylabel style={rotate=-90},xlabel={$f\,(\si{\hertz})$},xtick={0,20,40,60,80},xticklabels={$0$,$20$,$40$,$60$,$80$},ytick=\empty,,ymin=0,ymax=1.7,y label style={at={(axis description cs:0,1.1)}}]
\addplot[] plot coordinates {(0,0) (0,0.25)}node[circ]{}node[above]{$0.25$};
\addplot[] plot coordinates {(20,0) (20,1.35)}node[circ]{}node[above]{$1.35$};
\addplot[] plot coordinates {(40,0) (40,1)}node[circ]{}node[above]{$1$};
\addplot[] plot coordinates {(60,0) (60,0.5)}node[circ]{}node[above]{$0.5$};
\addplot[] plot coordinates {(80,0) (80,0.35)}node[circ]{}node[above]{$0.35$};
\end{axis}
\end{tikzpicture}
\caption*{(الف) مقداری طیف۔}
\end{subfigure}%
\begin{subfigure}{0.5\textwidth}
\centering
\begin{tikzpicture}
\begin{axis}[small,axis lines*=middle,ylabel={$\theta_n$},ylabel style={rotate=-90},xlabel={$f\,(\si{\hertz})$},xtick={0,20,40,60,80},xticklabels={$0$,$20$,$40$,$60$,$80$},ytick={-45,-90,-135},yticklabels={$-45^{\circ}$,$-90^{\circ}$,$-135^{\circ}$},ymax=20 ,x tick label style={above},x label style={at={(axis description cs:0.5,1.2)}},x tick style={draw=none},y label style={at={(axis description cs:0,1.05)}}]
\addplot[] plot coordinates {(20,0) (20,-135)};
\addplot[] plot coordinates {(40,0) (40,-90)};
\addplot[] plot coordinates {(60,0) (60,-45)};
\addplot[] plot coordinates {(80,0) (80,-135)};
\end{axis}
\end{tikzpicture}
\caption*{(ب) زاویائی ہٹاو طیف۔}
\end{subfigure}
\caption{مشق \حوالہ{مشق_فوریئر_طیف_دیا_گیا} کے طیف۔}
\label{شکل_فوریئر_طیف_دیا_گیا}
\end{figure}

جواب:
 \begin{multline*}
f(t)=0.25+1.35\cos(40\pi t-135^{\circ})+1\cos(80\pi t-90^{\circ})\\
+0.5\cos(120\pi t-45^{\circ})+0.35\cos(160\pi t-135^{\circ})+\cdots
\end{multline*}
\انتہا{مشق}
%===================

\حصہ{برقرار حال برقی جال}
کسی دور پر سائن نما دباو مسلط کرتے ہوئے دور میں مختلف مقامات پر دباو اور رو حاصل کرنا ہم دیکھ چکے ہیں۔فرض کریں کہ کسی دور پر  دوری دباو \عددی{v(t)} مسلط کی جاتی ہے۔ایسے دور کو حل کرنے  کی خاطر ہم مسلط دباو کا فوریئر تسلسل حاصل کرتے ہیں۔فوریئر تسلسل کا ہر رکن سائن نما دباو ہو گا۔انفرادی ہارمونی دباو کے لئے دور کو تعددی دائرہ کار میں حل کیا جاتا ہے۔ ان جوابات کا وقتی دائرہ کار میں مطلوبہ قیمت حاصل کی جاتی ہے۔تمام جوابات کا مجموعہ درکار جواب ہوتا ہے۔

\جزوحصہ{اوسط طاقت}
جیسا اوپر ذکر کیا گیا، دور پر دوری دباو یا دوری رو مسلط کرنے سے مختلف مقامات پر دباو اور رو پیدا ہوں گے جنہیں تسلسل کی صورت میں درج ذیل لکھا جا سکتا ہے۔ 
\begin{align}
v(t)&=V_{DC}+\sum_{n=1}^{\infty} V_n \cos (n\omega_0 t-\theta_{v,n})\\
i(t)&=I_{DC}+\sum_{n=1}^{\infty} I_n \cos (n\omega_0 t-\theta_{i,n})
\end{align}
غیر فعال رائج سمت استعمال کرتے ہوئے فرض کریں کہ کسی پرزے پر دباو اور اس میں رو درج بالا مساوات دیتے ہیں۔یوں اس پرزے کی اوسط طاقت درج ذیل ہو گی۔
\begin{align}\label{مساوات_فوریئر_طاقت_الف}
P&=\frac{1}{T}\int_0^T v(t)i(t) \dif t
\end{align}
درج بالا مساوات میں دو فوریئر تسلسل کے حاصل ضرب کا تکمل لیا گیا ہے۔ دو عدد فوریئر تسلسل کے حاصل ضرب  میں تین قسم کے ارکان پائے جاتے ہیں۔ایک رکن \عددی{V_{DC}I_{DC}} ہے جس کا تکمل تقسیم \عددی{T} از خود \عددی{V_{DC}I_{DC}} کے برابر ہے۔ دوسرا قسم وہ ہے جو \عددی{V_{DC} I_n\cos(n\omega_0 t -\theta_{i,n})} یا \عددی{I_{DC} V_n\cos(n\omega_0 t -\theta_{v,n})} صورت رکھتے ہیں۔ آپ جانتے ہیں کہ سائن نما تفاعل کا ایک دوری عرصے پر تکمل صفر کے برابر ہوتا ہے لہٰذا ایسے تمام ارکان صفر کے برابر ہوں گے۔تیسرا اور سب سے زیادہ تعداد میں  ارکان \عددی{\cos(m\omega_0 t -\theta_{v,n}) \cos(n\omega_0 t-\theta_{i,n})} صورت رکھتے ہیں۔آپ جانتے ہیں کہ \عددی{\cos(n\omega_0 t)} اور \عددی{\sin(n\omega_0 t)} آپس میں عمودی تفاعل ہیں لہٰذا \عددی{m \ne n} کی صورت میں تمام 
 \عددی{\cos(m\omega_0 t -\theta_{v,n}) \cos(n\omega_0 t-\theta_{i,n})} کا ایک دوری عرصے پر تکمل صفر کے برابر ہو گا۔یوں صرف ایسے ارکان غیر صفر ہوں گے جن میں دباو کی تعدد  اور رو کی تعدد برابر ہو یعنی
\begin{align*}
\sum_{n=1}^{\infty} \frac{1}{T}\int_0^T V_n I_n \cos(n\omega_0 t-\theta_{v,n})\cos(n\omega_0 t-\theta_{i,n}) \dif t=\sum_{n=1}^{\infty} \frac{V_n I_n}{2} \cos(\theta_{v,n}-\theta_{i,n})
\end{align*}
اس طرح اوسط طاقت درج ذیل ہو گی۔
\begin{align}
P=V_{DC}I_{DC}+\sum_{n=1}^{\infty}\frac{ V_n I_n}{2} \cos(\theta_{v,n}-\theta_{i,n})
\end{align}
درج بالا مساوات کا مطلب ہے کہ تمام انفرادی ہارمونی اجزاء کے اوسط طاقت کا مجموعہ کل اوسط طاقت دیتا ہے۔
%====================
\ابتدا{مثال}\شناخت{مثال_فوریئر_دور_حل_الف}
شکل \حوالہ{شکل_فوریئر_دور_حل_الف}-الف پر درج ذیل داخلی دباو \عددی{v_d(t)} مسلط کی گئی ہے۔خارجی دباو \عددی{v_0(t)} حاصل کریں۔
\begin{align*}
v_d(t)=\sum_{\substack{n=1 \\ \text{طاق}}}^{\infty} \frac{20}{n\pi}\sin (2nt)-\frac{40}{n^2\pi^2}\cos(2nt)
\end{align*}
% 
\begin{figure}
\centering
\begin{subfigure}{0.5\textwidth}
\centering
\begin{tikzpicture}[american voltages]
\draw(0,0) to [american voltage source,l={$v_d(t)$}]++(0,\y) to [resistor,l={$\SI{2}{\ohm}$}]++(\x,0) to [short]++(\x,0) to [resistor,l_={$\SI{4}{\ohm}$},v^<={$v_0(t)$}]++(0,-\y) to [short](0,0);
\draw(\x,0) to [capacitor,*-*,l={$\SI{1}{\farad}$}]++(0,\y);
\end{tikzpicture}
\caption*{(الف)}
\end{subfigure}
\begin{subfigure}{0.5\textwidth}
\centering
\begin{tikzpicture}[american voltages]
\draw(0,0) to [american voltage source,l={$\bV_d(s)$}]++(0,\y) to [resistor,l={$\SI{2}{\ohm}$}]++(\x,0) to [short]++(\x,0) to [resistor,l_={$\SI{4}{\ohm}$},v^<={$\bV_0(t)$}]++(0,-\y) to [short](0,0);
\draw(\x,0) to [capacitor,*-*,l={$\frac{1}{s}$}]++(0,\y);
\end{tikzpicture}
\caption*{(ب)}
\end{subfigure}
\caption{مثال \حوالہ{مثال_فوریئر_دور_حل_الف} کا دور۔}
\label{شکل_فوریئر_دور_حل_الف}
\end{figure}

حل:داخلی دباو کو درج ذیل لکھا جا سکتا ہے۔
 \begin{multline}\label{مساوات_فوریئر_لاگو_دباو_الف}
v_d(t)=7.5\cos(2t-122^{\circ})+2.2\cos(6t-102^{\circ})\\
+1.3\cos(10t-97^{\circ})+0.91\cos(14t-95^{\circ})+\cdots
\end{multline}
شکل \حوالہ{شکل_فوریئر_دور_حل_الف}-ب میں تعددی دائرہ کار میں دور کو دکھایا گیا ہے جس میں متوازی جڑے برق گیر اور مزاحمت کی
 رکاوٹ \عددی{\tfrac{4(1/j\omega)}{4+1/j\omega}=\tfrac{4}{1+j4\omega}} ہے  لہٰذا تقسیم دباو سے خارجی دباو درج ذیل لکھا جا سکتا ہے۔
\begin{align}\label{مساوات_فوریئر_لاگو_دباو_ب}
\bV_0(\omega)=\frac{\frac{4}{1+j4\omega}}{2+\frac{4}{1+j4\omega}} \bV_d(s)=\frac{2}{3+j4\omega} \bV_d
\end{align}
مساوات \حوالہ{مساوات_فوریئر_لاگو_دباو_الف} کے پہلے رکن سے  \عددی{\omega_0=\SI{2}{\radian\per\second}} دیکھا جا سکتا ہے لہٰذا تسلسل کے پہلے رکن سے پیدا خارجی دباو درج بالا مساوات سے
\begin{align*}
\bV_0(\omega_0)=\left(\frac{2}{3+j8}\right) 7.5\phase{-122^{\circ}} =1.7556\phase{-191.4^{\circ}}
\end{align*}
ہو گا۔داخلی دباو کے تسلسل میں اگلے رکن یعنی تیسرے ہارمونی جزو کی تعدد \عددی{\omega=3\omega_0=\SI{6}{\radian \per\second}} جبکہ  پانچویں ہارمونی رکن کی تعدد \عددی{5\omega_0} ہے۔یوں ان ارکان سے پیدا خارجی دباو کو مساوات \حوالہ{مساوات_فوریئر_لاگو_دباو_ب} میں درست تعدد پر کرتے ہوئے حاصل کی جا سکتی ہے۔
\begin{align*}
\bV_0(3\omega_0)=\frac{\frac{4}{1+j24}}{2+\frac{4}{1+j24}} 2.2\phase{-102^{\circ}}=0.1819\phase{-184.9^{\circ}}\\
\bV_0(5\omega_0)=\frac{\frac{4}{1+j40}}{2+\frac{4}{1+j40}} 1.3\phase{-97^{\circ}}=0.0648\phase{-182.7^{\circ}}\\
\bV_0(7\omega_0)=\frac{\frac{4}{1+j56}}{2+\frac{4}{1+j56}} 0.91\phase{-95^{\circ}}=0.0325\phase{-181.9^{\circ}}
\end{align*}
یوں برقرار خارجی دباو درج ذیل ہو گا۔
 \begin{multline}
v_d(t)=1.7556\cos(2t-191.4^{\circ})+0.1819\cos(6t-184.9^{\circ})\\
+0.0648\cos(10t-182.7^{\circ})+0.0325\cos(14t-181.9^{\circ})+\cdots
\end{multline}
\انتہا{مثال}
%==================
\ابتدا{مثال}\شناخت{مثال_فوریئر_اوسط_طاقت_الف}
شکل \حوالہ{شکل_فوریئر_اوسط_طاقت_الف} میں سلسلہ وار \عددی{RLC} پر درج ذیل دباو \عددی{v_d(t)} مسلط کیا گیا ہے۔مزاحمت میں اوسط طاقت کا ضیاع دریافت کریں۔
\begin{align*}
v_d(t)=33+2\cos(100t-30^{\circ})+1.1\cos(200t-45^{\circ})+0.6\cos(300t-60^{\circ})+\cdots
\end{align*}
%
\begin{figure}
\centering
\begin{tikzpicture}
\draw(0,0) to [american voltage source,l={$v_d(t)$}]++(0,\y) to [resistor,l={$\SI{2}{\ohm}$},i={$i(t)$}]++(\x,0) to [inductor,l={$\SI{20}{\milli\henry}$}]++(\x,0) to [capacitor,l={$\SI{10}{\milli\farad}$}]++(0,-\y) to [short] (0,0);
\end{tikzpicture}
\caption{مثال \حوالہ{مثال_فوریئر_اوسط_طاقت_الف} کا دور۔}
\label{شکل_فوریئر_اوسط_طاقت_الف}
\end{figure}

حل:برق گیر یک سے سمتی رو نہیں گزرتی لہٰذا داخلی دباو کا یک سمتی جزو یعنی \عددی{\SI{33}{\volt}} کوئی رو نہیں پیدا کر پائے گا اور \عددی{I_{DC}=0} ہو گا۔پہلے ہارمونی جزو سے \عددی{\omega_0=\SI{100}{\radian \per\second}} دیکھا جا سکتا ہے۔داخلی دباو کے تسلسل کے بقایا ارکان کو باری باری حل کرتے  ہوئے رو حاصل کرتے ہیں۔
\begin{align*}
\bI(\omega_0)=\frac{2\phase{-30^{\circ}}}{2+j100\times 0.02+\frac{1}{j100\times 0.01}}=0.89\phase{-56.6^{\circ}}\\
\bI(2\omega_0)=\frac{1.1\phase{-45^{\circ}}}{2+j200\times 0.02+\frac{1}{j200\times 0.01}}=0.27\phase{-105.3^{\circ}}\\
\bI(3\omega_0)=\frac{0.6\phase{-60^{\circ}}}{2+j300\times 0.02+\frac{1}{j300\times 0.01}}=0.099\phase{-130.6^{\circ}}
\end{align*}
یوں رو
\begin{multline*}
i(t)=0.89\cos(100t-56.6^{\circ})+0.27\cos(200t-105.3^{\circ})\\
+0.099\cos(300t-130.6^{\circ})+\cdots
\end{multline*}
ہو گی۔دور میں صرف مزاحمت طاقت ضائع کرتی ہے لہٰذا پورے دور کا ضیاع مزاحمتی ضیاع ہو گا۔دور میں کل اوسط طاقت کا ضیاع درج ذیل ہے
\begin{multline*}
P=\frac{2\times 0.89}{2}\cos(56.6^{\circ}-30^{\circ})+\frac{1.1\times 0.27}{2}\cos(105.3^{\circ}-45^{\circ})\\
+\frac{0.6\times 0.099}{2}\cos(130.6^{\circ}-60^{\circ})+\cdots
\end{multline*}
یعنی
\begin{align*}
P\approx\SI{0.61}{\watt}
\end{align*}
\انتہا{مثال}
%===================
\ابتدا{مشق}
ایک دور کے داخلی دباو اور داخلی رو درج ذیل ہیں۔دور میں اوسط طاقت کا ضیاع دریافت کریں۔
\begin{align*}
v_d(t)=20+7\cos(100t-30^{\circ})+5\cos(200t-45^{\circ})+2\cos(300t-60^{\circ})+\cdots\\
i_d(t)=5+3\cos(100t+40^{\circ})+1\cos(200t-45^{\circ})+0.2\cos(300t-70^{\circ})+\cdots
\end{align*}

جواب:\عددی{P=\SI{108.99}{\watt}}
\انتہا{مشق}
%====================
\ابتدا{مشق}\شناخت{مشق_فوریئر_مشق_رو}
شکل \حوالہ{شکل_فوریئر_مشق_رو} میں داخلی رو  دریافت کریں۔داخلی دباو درج ذیل ہے۔
\begin{align*}
v_d(t)=5\cos(50t-30^{\circ})+4\cos(100t+45^{\circ})+2\cos(150t-10^{\circ})\,\si{\volt}
\end{align*}
%
\begin{figure}
\centering
\begin{tikzpicture}
\draw(0,0) to [american voltage source,l={$v_d(t)$}]++(0,\y) to [resistor,l={$\SI{4}{\ohm}$}]++(\x,0) to [inductor,l={$\SI{0.01}{\henry}$}]++(0,-\y) to [short](0,0);
\draw(\x,0) to [short,*-]++(\x,0) to [capacitor,l_={$\SI{0.01}{\farad}$}]++(0,\y) to [short,-*]++(-\x,0);
\end{tikzpicture}
\caption{مشق \حوالہ{مشق_فوریئر_مشق_رو} کا دور۔}
\label{شکل_فوریئر_مشق_رو}
\end{figure}

جواب: متوازی امالہ اور برق گیر کی قدرتی تعدد \عددی{\SI{100}{\radian\per\second}} ہے جس پر ان کی رکاوٹ لامتناہی ہو جاتی ہے لہٰذا رو میں \عددی{\SI{100}{\radian\per\second}} کا جزو نہیں پایا جاتا۔\\
 \عددی{i_d(t)=1.23\cos(50t-39.5^{\circ})+0.48\cos(150t+6.7^{\circ})\,\si{\ampere}}
\انتہا{مشق}
%=====================

\حصہ{فوریئر بدل}
ہم دوری تفاعل کو فوریئر تسلسل سے ظاہر کرنا دیکھ چکے ہیں۔آئیں اب \اصطلاح{غیر دوری}\فرہنگ{غیر دوری}\فرہنگ{دوری!غیر}\حاشیہب{aperiodic}\فرہنگ{aperiodic} تفاعل کو ظاہر کرنے پر غور کریں۔

شکل \حوالہ{شکل_فوریئر_دوری_غیر_دوری}-الف میں غیر دوری تفاعل \عددی{f(t)} دکھایا گیا ہے۔شکل-ب میں اس تفاعل کو \عددی{T} دورانیے پر دہراتے ہوئے تفاعل \عددی{f_d(t)} حاصل کیا گیا ہے۔آپ جانتے ہیں کہ شکل-ب کے دوری تفاعل کو ہم قوت نمائی تسلسل سے ظاہر کر سکتے ہیں
\begin{align}
f_d(t)=\sum_{n=-\infty}^{\infty} \kx{c}_n e^{jn\omega_0 t}
\end{align}
جہاں
\begin{align}\label{مساوات_فوریئر_لکیری_طیف_عددی_سر}
\kx{c}_n=\frac{1}{T}\int_{-\frac{T}{2}}^{\frac{T}{2}} f_d(t) e^{-jn\omega_0 t} \dif t
\end{align}
اور
\begin{align}
\omega_0=\frac{2\pi}{T}
\end{align}
ہیں۔شکل \حوالہ{شکل_فوریئر_دوری_غیر_دوری}-ب میں \عددی{T \to \infty} کرنے سے  شکل-الف حاصل ہوتا ہے یعنی تفاعل غیر دوری ہو گا۔ایسی صورت میں \عددی{-T}، \عددی{T}، \عددی{2T} وغیرہ پر پائے جانے والے حصے لامتناہی پر پائے جائیں گے۔
% 
\begin{figure}
\centering
\begin{subfigure}{0.5\textwidth}
\centering
\begin{tikzpicture}
\begin{axis}[small,axis lines*=middle,xtick={-120,0,120},xticklabels={$-\frac{T}{2}$,$0$,$\frac{T}{2}$},ytick=\empty,ymin=0,xmin=-360,xmax=360,ylabel={$f(t)$},xlabel={$t$},ylabel style={rotate=-90,at={(axis description cs:0.5,0.9)}}]
\addplot[domain=-90:90,samples=100]{cos(x)+1/2*sin(2*x)};
\end{axis}
\end{tikzpicture}
\caption*{(الف) غیر دوری تفاعل۔}
\end{subfigure}%
\begin{subfigure}{0.5\textwidth}
\centering
\begin{tikzpicture}
\begin{axis}[small,axis lines*=middle,xtick={-240,-120,0,120,240},xticklabels={$-T$,$-\frac{T}{2}$,$0$,$\frac{T}{2}$,$T$},ytick=\empty,ymin=0,xmin=-360,xmax=360,ylabel={$f_d(t)$},xlabel={$t$},ylabel style={rotate=-90,at={(axis description cs:0.5,0.9)}}]
\addplot[domain=-90:90,samples=100]{cos(x)+1/2*sin(2*x)};
\addplot[domain=240-90:240+90,samples=100]{cos(x-240)+1/2*sin(2*(x-240))};
\addplot[domain=-240-90:-240+90,samples=100]{cos(x+240)+1/2*sin(2*(x+240))};
\end{axis}
\end{tikzpicture}
\caption*{(ب) دوری تفاعل۔}
\end{subfigure}
\caption{دوری اور غیر دوری تفاعل۔}
\label{شکل_فوریئر_دوری_غیر_دوری}
\end{figure}

دوری تفاعل کے لکیری طیف میں لکیریں ہارمونی تعدد \عددی{n\omega_0} پر پائی جاتی ہے لہٰذا دو قریبی لکیروں کے مابین تعددی فاصلہ
\begin{align}
\Delta \omega=(n+1)\omega_0-n\omega_0 =\omega_0=\frac{2\pi}{T}
\end{align}
ہو گا۔شکل \حوالہ{شکل_فوریئر_بدل_الف}-الف میں ان حقائق کی وضاحت کی گئی ہے۔دوری فاصلہ \عددی{T} بڑھانے سے طیفی لکیروں کے مابین تعددی فاصلہ کم ہو گا۔شکل-ب اور پ میں ایسا دکھایا گیا ہے۔جیسا شکل-ت میں دکھایا گیا ہے، \عددی{T \to \infty} کرنے سے \عددی{\Delta \omega \to \dif \omega} ہو گا، طیف اپنی لکیری خاصیت کھو دے گا اور یہ ایک مسلسل طیف کی صورت اختیار کر لیگا۔ایسی صورت میں طیف انفرادی تعدد \عددی{n\omega_0} کی بجائے تمام تعدد \عددی{\omega} پر پایا جائے گا لہٰذا \عددی{n\omega_0} کو \عددی{\omega} فرض کیا جا سکتا ہے یعنی
\begin{align}\label{مساوات_فوریئر_لامتناہی_دوری_عرصہ_تعددی_فرق}
n \omega_0 = \omega
\end{align}
%
 \begin{figure}
\centering
\begin{subfigure}{0.5\textwidth}
\centering
\begin{tikzpicture}
\draw(0,0)--++(4,0)node[below]{$\omega$};
\draw(0,0)--++(0,2)node[left]{$D_n$};
\foreach \kx in {1,2,3,4,5}{\draw(\kx*0.72,0)--++(0,{1.75*cos(\kx*12.6)})node[circ]{};}
\draw(0.72,0)node[below]{$1\omega_0$};
\draw(0.72*2,0)node[below]{$2\omega_0$};
\draw(0.72*3,-0.1)--++(0,-0.3);
\draw(0.72*4,-0.1)--++(0,-0.3);
\draw[stealth-stealth](0.72*3,-0.2)--(0.72*4,-0.2)node[pos=0.5,below]{$\Delta \omega$};
\end{tikzpicture}
\caption*{طیفی لکیروں کے مابین فاصلہ \عددی{\omega_0} ہے۔}
\end{subfigure}%
\begin{subfigure}{0.5\textwidth}
\centering
\begin{tikzpicture}
\draw(0,0)--++(4,0)node[below]{$\omega$};
\draw(0,0)--++(0,2)node[left]{$D_n$};
\foreach \kx in {1,2,...,18}{\draw(\kx*0.2,0)--++(0,{1.75*cos(\kx*3.5)})node[circ]{};}
\end{tikzpicture}
\caption*{(ب) دوری عرصہ بڑھانے سے طیفی\\ لکیروں کے مابین فاصلہ کم ہوتا ہے۔}
\end{subfigure}
\begin{subfigure}{0.5\textwidth}
\centering
\begin{tikzpicture}
\draw(0,0)--++(4,0)node[below]{$\omega$};
\draw(0,0)--++(0,2)node[left]{$D_n$};
\foreach \kx in {1,2,...,36}{\draw(\kx*0.1,0)--++(0,{1.75*cos(\kx*1.75)})node[circ]{};}
\end{tikzpicture}
\caption*{(پ) دوری عرصہ بہت بڑھانے سے طیفی لکیروں\\ کے مابین فاصلہ نہایت کم ہو جاتا ہے۔}
\end{subfigure}%
\begin{subfigure}{0.5\textwidth}
\centering
\begin{tikzpicture}
\draw(0,0)--++(4,0)node[below]{$\omega$};
\draw(0,0)--++(0,2)node[left]{$\Fourier(\omega)$};
\draw[shade,domain=0:63,variable=\t] plot ({\t*2/35},{1.75*cos(\t)}) --(3.6,0)--(0,0)--(0,1.75);
\end{tikzpicture}
\caption*{(ت) لامتناہی دوری عرصے کی صورت میں طیفی لکیریں آپس\\ میں مل جاتی ہیں اور ان میں فرق کرنا ممکن نہیں رہتا۔}
\end{subfigure}
\caption{لکیری طیف سے مسلسل طیف کا حصول۔}
\label{شکل_فوریئر_بدل_الف}
\end{figure}

چونکہ \عددی{T \to \infty} کرنے سے مساوات \حوالہ{مساوات_فوریئر_لکیری_طیف_عددی_سر} میں \عددی{\kx{c}_n \to 0} ہوں گے لہٰذا ہم \عددی{\kx{c}_n T} پر نظر رکھتے ہوئے آگے بڑھتے ہیں۔
\begin{align*}
\kx{c}_n T=\int_{-\frac{T}{2}}^{\frac{T}{2}} f_d(t) e^{-jn\omega_0 t}\dif t
\end{align*}
دوری عرصے کی حد لامتناہی کرتے ہوئے درج بالا مساوات کو درج ذیل لکھا جا سکتا ہے
\begin{align*}
\lim_{T \to \infty} (\kx{c}_n T)&= \lim_{T \to \infty}\int_{-\frac{T}{2}}^{\frac{T}{2}} f_d(t) e^{-jn\omega_0 t}\dif t\\
&=\int_{-\infty}^{\infty} f(t) e^{-j\omega t}\dif t
\end{align*}
جہاں مندرجہ بالا بحث کو مد نظر رکھتے ہوئے، دوسری قدم پر دوری تفاعل \عددی{f_d(t)} کی جگہ غیر دوری تفاعل\عددی{f(t)} پر کیا گیا ہے اور مساوات \حوالہ{مساوات_فوریئر_لامتناہی_دوری_عرصہ_تعددی_فرق} کا استعمال کیا گیا ہے۔ اس تکمل کو \اصطلاح{فوریئر بدل}\فرہنگ{فوریئر بدل}\فرہنگ{بدل!فوریئر}\حاشیہب{Fourier transform}\فرہنگ{Fourier transform} کہتے اور \عددی{\Fourier(\omega)} سے ظاہر کیا جاتا ہے۔
\begin{align}\label{مساوات_فوریئر_الف}
\bF(\omega)=\int_{-\infty}^{\infty} f(t) e^{-j\omega t}\dif t
\end{align}
اسی طرح دوری تفاعل کو درج ذیل لکھا جا سکتا ہے
\begin{align*}
f_d(t)&=\sum_{n=-\infty}^{\infty} \kx{c}_n e^{jn\omega_0 t}\\
&=\sum_{n=-\infty}^{\infty} (\kx{c}_n T) e^{jn\omega_0 t} \frac{1}{T}\\
&=\sum_{n=-\infty}^{\infty} (\kx{c}_n T) e^{jn\omega_0 t} \frac{\Delta \omega}{2\pi}
\end{align*}
جس کو \عددی{T \to \infty} کی صورت میں درج ذیل لکھا جا سکتا ہے
\begin{align}\label{مساوات_فوریئر_ب}
f(t)=\frac{1}{2\pi}\int_{-\infty}^{\infty} \bF(\omega)e^{j\omega t} \dif \omega
\end{align}
جہاں مساوات \حوالہ{مساوات_فوریئر_لامتناہی_دوری_عرصہ_تعددی_فرق} کا استعمال کیا گیا ہے۔

مساوات \حوالہ{مساوات_فوریئر_الف} اور مساوات \حوالہ{مساوات_فوریئر_ب} فوریئر جوڑی کہلاتے ہیں۔چونکہ \عددی{f(t)} کا \اصطلاح{فوریئر بدل}\فرہنگ{فوریئر!بدل}\حاشیہب{Fourier transform}\فرہنگ{Fourier!transform} \عددی{\Fourier(\omega)} ہے لہٰذا \عددی{\Fourier(\omega)} کا \اصطلاح{الٹ فوریئر بدل}\فرہنگ{فوریئر!الٹ بدل}\حاشیہب{inverser Fourier transform}\فرہنگ{Fourier!inverse transform} \عددی{f(t)} ہے۔فوریئر جوڑی کو اکٹھے لکھتے ہیں۔
\begin{gather}
\begin{aligned}\label{مساوات_فوریئر_پ}
\bF(\omega)=\Fourier[f(t)]&=\int_{-\infty}^{\infty} f(t) e^{-j\omega t}\dif t\\
f(t)=\Fourier^{-1}[\bF(\omega)]&=\frac{1}{2\pi}\int_{-\infty}^{\infty} \bF(\omega)e^{j\omega t} \dif \omega
\end{aligned}
\end{gather}
%==========

\حصہء{چند اہم فوریئر بدل جوڑیاں}
چند فوریئر بدل کی جوڑیاں حاصل کرتے ہیں۔
%=======================
\ابتدا{مثال}\شناخت{مثال_فوریئر_مستطیل_کا_بدل}
شکل \حوالہ{شکل_فوریئر_مستطیل_کا_بدل}-الف میں دیے مستطیل تفاعل \عددی{f(t)} کی فوریئر بدل \عددی{\bF(\omega)} حاصل کریں۔
\begin{figure}
\centering
\begin{subfigure}{0.5\textwidth}
\centering
\begin{tikzpicture}
\begin{axis}[small,axis lines*=middle,ytick=\empty,xtick={-0.5,0,0.5},xticklabels={$-\frac{\delta}{2}$,$0$,$\frac{\delta}{2}$},xlabel={$t$},xmin=-6,xmax=6,ymin=0,ymax=1.5,ylabel={$v(t)$},ylabel style={rotate=-90},ylabel style={at={(axis description cs:0.5,0.9)}},xlabel style={at={(axis description cs:1.05,0)}}]
\addplot[] plot coordinates {(-0.5,0) (-0.5,1) (0.5,1) (0.5,0)};
\addplot[]plot coordinates{(-1.5,1)}node[font=\small]{$V$};
\end{axis}
\end{tikzpicture}
\caption*{(الف)}
\end{subfigure}%
\begin{subfigure}{0.5\textwidth}
\centering
\begin{tikzpicture}
\begin{axis}[small,axis lines*=middle,ytick=\empty,xtick={-5,-0.5,0,0.5,5},xticklabels={$-T$,$-\frac{\delta}{2}$,$0$,$\frac{\delta}{2}$,$T$},xlabel={$t$},xmin=-6,xmax=6,ymin=0,ymax=1.5,ylabel={$v_d(t)$},ylabel style={rotate=-90},ylabel style={at={(axis description cs:0.5,0.9)}},xlabel style={at={(axis description cs:1.05,0)}}]
\addplot[] plot coordinates {(-0.5,0) (-0.5,1) (0.5,1) (0.5,0)};
\addplot[] plot coordinates {(5-0.5,0) (5-0.5,1) (5+0.5,1) (5+0.5,0)};
\addplot[] plot coordinates {(-5-0.5,0) (-5-0.5,1) (-5+0.5,1) (-5+0.5,0)};
\addplot[stealth-stealth] plot coordinates {(0,0.5) (5,0.5)}node[pos=0.5,above]{$T=5\delta$};
\addplot[]plot coordinates{(-1.5,1)}node[font=\small]{$V$};
\end{axis}
\end{tikzpicture}
\caption*{(ب)}
\end{subfigure}
\begin{subfigure}{0.5\textwidth}
\centering
\begin{tikzpicture}
\begin{axis}[axis lines*=middle,ytick={1},yticklabels={$V \delta$},xtick={-360,-180,0,180,360},xticklabels={$-\frac{4\pi}{\delta}$,$-\frac{2\pi}{\delta}$,$0$,$\frac{2\pi}{\delta}$,$\frac{4\pi}{\delta}$},xlabel={$\omega$},xlabel style={at={(axis description cs:1.05,0.2)}},ylabel={$\bF(\omega)$},ylabel style={rotate=-90},ylabel style={at={(axis description cs:0.5,1)}},ymax=1.3]
\addplot [domain = -600:600, samples = 100]{sin(x)/ (x*pi/180) };
\end{axis}
\end{tikzpicture}
\caption*{(پ)}
\end{subfigure}
\begin{subfigure}{0.5\textwidth}
\centering
\begin{tikzpicture}
\begin{axis}[axis lines*=middle,ytick={1},yticklabels={$\frac{V \delta}{T}$},xlabel={$\omega$},xlabel style={at={(axis description cs:1.05,0.2)}},ylabel={$\kx{c}_n$},ylabel style={rotate=-90},ylabel style={at={(axis description cs:0.5,1)}},ymax=1.3,xtick=\empty]
\addplot [domain = -600:600, samples = 100]{sin(x)/ (x*pi/180) }node[pos=0.55,pin=45:{غلاف}]{};
%
\addplot[] plot coordinates {(36,0) (36,{sin(36)/ (36*pi/180) })};
\addplot[] plot coordinates {(72,0) (72,{sin(72)/ (72*pi/180) })};
\addplot[] plot coordinates {(108,0) (108,{sin(108)/ (108*pi/180) })};
\addplot[] plot coordinates {(144,0) (144,{sin(144)/ (144*pi/180) })};
\addplot[] plot coordinates {(216,0) (216,{sin(216)/ (216*pi/180) })};
\addplot[] plot coordinates {(252,0) (252,{sin(252)/ (252*pi/180) })};
\addplot[] plot coordinates {(288,0) (288,{sin(288)/ (288*pi/180) })};
\addplot[] plot coordinates {(324,0) (324,{sin(324)/ (324*pi/180) })};
%

\addplot[] plot coordinates {(-36,0) (-36,{sin(-36)/ (-36*pi/180) })};
\addplot[] plot coordinates {(-72,0) (-72,{sin(-72)/ (-72*pi/180) })};
\addplot[] plot coordinates {(-108,0) (-108,{sin(-108)/ (-108*pi/180) })};
\addplot[] plot coordinates {(-144,0) (-144,{sin(-144)/ (-144*pi/180) })};
\addplot[] plot coordinates {(-216,0) (-216,{sin(-216)/ (-216*pi/180) })};
\addplot[] plot coordinates {(-252,0) (-252,{sin(-252)/ (-252*pi/180) })};
\addplot[] plot coordinates {(-288,0) (-288,{sin(-288)/ (-288*pi/180) })};
\addplot[] plot coordinates {(-324,0) (-324,{sin(-324)/ (-324*pi/180) })};
%
\addplot[]plot coordinates {(180,-0.1) (180,-0.2)};
\addplot[stealth-stealth]plot coordinates {(0,-0.15) (180,-0.15)}node[pos=0.5,below]{$\frac{2\pi}{\delta}$};
\addplot[-stealth]plot coordinates {(18,0.3) (72,0.3)};
\addplot[stealth-]plot coordinates {(108,0.3) (180,0.3) (200,0.35)}node[right]{$\frac{2\pi}{T}$};
\end{axis}
\end{tikzpicture}
\caption*{(ت)}
\end{subfigure}
\caption{مثال \حوالہ{مثال_فوریئر_مستطیل_کا_بدل} کا تفاعل۔}
\label{شکل_فوریئر_مستطیل_کا_بدل}
\end{figure}

حل:مساوات \حوالہ{مساوات_فوریئر_پ} استعمال کرتے ہوئے فوریئر بدل حاصل کرتے ہیں۔
\begin{align*}
\bF(\omega)&=\int_{-\frac{\delta}{2}}^{\frac{\delta}{2}} V e^{-j\omega t} \dif t\\
&=\left. \frac{V e^{-j\omega t}}{-j\omega}\right|_{-\frac{\delta}{2}}^{\frac{\delta}{2}}\\
&=V\frac{e^{-j\omega \frac{\delta}{2}}-e^{+j\omega \frac{\delta}{2}}}{j\omega}\\
&=V \delta \frac{\sin \frac{\omega \delta}{2}}{\frac{\omega \delta}{2}}
\end{align*}
یوں وقتی دائرہ کار کے تفاعل \عددی{f(t)}
\begin{align}
f(t)=
\begin{cases}
0 & -\infty < t \le -\frac{\delta}{2}\\
V & -\frac{\delta}{2} < t < \frac{\delta}{2}\\
0& \frac{\delta}{2} \le t < \infty
\end{cases}
\end{align}
کا فوریئر بدل \عددی{\bF(\omega)} درج ذیل ہے۔
\begin{align}
\bF(\omega)=\Fourier[f(t)]=V \delta \frac{\sin \frac{\omega \delta}{2}}{\frac{\omega \delta}{2}}
\end{align}
اس مثال پر مزید غور کرتے ہیں۔شکل \حوالہ{شکل_فوریئر_مستطیل_کا_بدل}-الف کے تفاعل کو شکل-ب میں \عددی{T} فاصلے کے دورانیے پر دہراتے ہوئے  دوری تفاعل \عددی{f_d(t)} حاصل کیا گیا ہے۔دوری تفاعل \عددی{f_d(t)} کے فوریئر تسلسل کے عددی سر درج ذیل ہیں۔
\begin{align}
\kx{c}_n=\frac{V \delta}{T} \frac{\sin\frac{n\omega_0 \delta}{2}}{\frac{n\omega_0 \delta}{2}}
\end{align}
شکل \حوالہ{شکل_فوریئر_مستطیل_کا_بدل}-ت میں لکیری طیف دکھائی گئی ہے۔آپ دیکھ سکتے ہیں کہ  لکیری طیف کے \اصطلاح{غلاف}\فرہنگ{غلاف}\حاشیہب{envelope}\فرہنگ{envelope} کی شکل اور مسلسل طیف کی شکل بالکل یکساں ہیں۔

اس مثال کے نتائج اور مساوات سے ظاہر ہے کہ \عددی{T \to \infty} کرنے سے دوری تفاعل تبدیل ہو کر غیر دوری تفاعل بن جاتا ہے۔ساتھ ہی ساتھ جیسے جیسے \عددی{T} بڑھتا ہے ویسے ویسے طیفی لکیر قریب ہوتے ہیں اور ان کا حیطہ کم  ہوتا ہے حتٰی کہ آخر کار لکیری طیف مسلسل طیف میں تبدیل ہو جاتا ہے۔چونکہ فوریئر تسلسل مخصوص تعدد پر اشارے کا حیطہ اور زاویائی ہٹاو دیتا ہے لہٰذا فوریئر بدل بھی اشارے کی تعددی معلومات دیتا ہے۔ 
\انتہا{مثال}
%=======================
\ابتدا{مثال}
اکائی ضرب تفاعل \عددی{\delta(t-a)} اور \عددی{\delta(t)} کا فوریئر بدل حاصل کریں۔

حل:اکائی ضرب تفاعل کا فوریئر بدل تکمل سے حاصل کرتے ہیں۔
\begin{gather}
\begin{aligned}
\bF(\omega)&=\int_0^{\infty} \delta(t-a)e^{-j\omega t} \dif t\\
&=e^{-j\omega a}
\end{aligned}
\end{gather}
تکمل کو حل میں اکائی ضرب تفاعل کی \اصطلاح{نمونہ بندی}\فرہنگ{نمونہ بندی!خاصیت}\فرہنگ{sampling!property} خاصیت استعمال کی گئی۔ درج بالا میں \عددی{a=0} پر کرنے سے اکائی ضرب تفاعل \عددی{\delta(t)} کا فوریئر بدل ملتا ہے۔
\begin{align}
\Fourier[\delta(t)]=1
\end{align}
آپ نے دیکھا کہ اکائی ضرب تفاعل \عددی{\delta(t)} کا فوریئر بدل ایک مستقل مقدار ہے جو تعدد کے ساتھ تبدیل نہیں ہوتا۔یہ اکائی ضرب تفاعل کی ایک اہم خصوصیت ہے۔
 
\انتہا{مثال}
%======================
\ابتدا{مثال}
تفاعل \عددی{e^{j\omega_0 t}} کا فوریئر بدل حاصل کریں۔

حل:یہاں اگر \عددی{\bF(\omega)=2\pi \delta(t-t_0)} لیا جائے تب \عددی{f(t)} درج ذیل ہو گا
\begin{gather}
\begin{aligned}
f(t)&=\frac{1}{2\pi}\int_{-\infty}^{\infty} 2\pi \delta(t-t_0)e^{j\omega t}\dif \omega\\
&=e^{j\omega_0 t}
\end{aligned}
\end{gather} 
جہاں \عددی{\delta(t-t_0)} کی نمونہ بندی کی خاصیت استعمال کی گئی۔یوں \عددی{f(t)=e^{j\omega_0 t}} اور \عددی{\bF(\omega)=2\pi\delta(t-t_0)} فوریئر بدل جوڑی ہیں۔ 
\انتہا{مثال}
%=======================
\ابتدا{مشق}
تفاعل \عددی{\cos \omega t} کا فوریئر بدل دریافت کریں۔

جواب:\عددی{\bF(\omega)=\pi \delta(\omega-\omega_0)+\pi\delta(\omega+\omega_0)}
\انتہا{مشق}
%======================
چند اہم فوریئر بدل جوڑیوں کو جدول \حوالہ{جدول_فوریئر_بدل_جوڑیاں} میں اکٹھے کیا گیا ہے۔
\begin{table}
\caption{فوریئر بدل جوڑیاں۔}
\label{جدول_فوریئر_بدل_جوڑیاں}
\centering
\begin{tabular}{l l}
$\Fourier(\omega)$ & $f(t)$ \Bstrut \\
\hline
$2\pi A \delta(\omega)$&$A$ \Tstrut \\[2ex]
$1$&$\delta(t)$\\[1ex]
$e^{-j\omega a}$&$\delta(t-t_0)$\\[1ex]
 $2\pi \delta(\omega-\omega_0)$ &$e^{j\omega_0 t}$ \\[1ex]
 $j\pi \delta(\omega+\omega_0)-j\pi\delta(\omega-\omega_0)$ &$\sin \omega_0 t$ \\[1ex]
 $\pi \delta(\omega-\omega_0)+\pi\delta(\omega+\omega_0)$ &$\cos \omega_0 t$ \\[1ex]
 $\frac{1}{a+j\omega}$&$e^{-at}u(t), \, a>0$  \\[1ex]
 $\frac{2a}{a^2+\omega^2}$&$e^{-a\abs{t}}u(t), \, a>0 $  \\[1ex]
 $\frac{\omega_0}{(j\omega+0)^2+\omega_0^2}$ &$e^{-at} \sin \omega_0 t\, u(t), \, a>0$ \\[1ex]
 $\frac{j\omega+a}{(j\omega+0)^2+\omega_0^2}$ &$e^{-at} \cos \omega_0 t\, u(t), \, a>0$ 
\end{tabular}
\end{table}

%=============
\حصہ{فوریئر بدل کے خواص}
فوریئر بدل کے چند مخصوص مسئلوں کو جدول \حوالہ{جدول_فوریئر_بدل_مسئلے} میں پیش کیا گیا ہے۔ان میں سے \اصطلاح{مسئلہ وقتی الجھاو}\فرہنگ{مسئلہ!وقتی الجھاو}\فرہنگ{وقتی الجھاو!مسئلہ}\حاشیہب{time convolution theorem}\فرہنگ{time convolution!theorem}\فرہنگ{theorem!time convolution} کو ثابت کرتے ہیں۔جدول میں دیے بقایا مسئلے بھی انتہائی آسانی سے ثابت کئے جا سکتے ہیں۔
 
تفاعل \عددی{f(t)} کا فوریئر بدل درج ذیل ہے۔
\begin{align}
\Fourier[f(t)]=\bF(\omega)=\int_{-\infty}^{\infty}f(t) e^{-j\omega t} \dif t
\end{align}
فرض کریں کہ 
\begin{align*}
\Fourier[f_1(t)]&=\bF_1(\omega)\\
\Fourier[f_2(t)]&=\bF_2(\omega)
\end{align*}
ہیں تب
\begin{align*}
\Fourier\left[\int_{-\infty}^{\infty}f_1(x)f_2(t-x)\dif x\right]&=\int_{t=-\infty}^{\infty}\int_{x=-\infty}^{\infty}f_1(x)f_2(t-x)\dif x  e^{-j\omega t}\dif t\\
&=\int_{x=-\infty}^{\infty}f_1(x)\int_{t=-\infty}^{\infty}f_2(t-x)  e^{-j\omega t}\dif t \dif x
\end{align*}
اب  اگر ہم \عددی{u=t-x} پر کریں  تب درج ذیل لکھا جا سکتا ہے۔
\begin{align*}
\Fourier\left[\int_{-\infty}^{\infty}f_1(x)f_2(t-x)\dif x\right]&=\int_{x=-\infty}^{\infty}f_1(x)\int_{t=-\infty}^{\infty}f_2(u)  e^{-j\omega (u+x)}\dif u \dif x\\
&=\int_{x=-\infty}^{\infty}f_1(x) e^{-j\omega x}\dif x \int_{t=-\infty}^{\infty}f_2(u)  e^{-j\omega u}\dif u \\
&=\bF_1(\omega) \bF_2(\omega)
\end{align*}
شکل \حوالہ{شکل_فوریئر_وقتی_الجھاو} کو دیکھتے ہوئے مسئلہ وقتی الجھاو کہتا ہے کہ اگر \عددی{\kB{V}_d(\omega)=\Fourier[v_d(t)]}،
 \عددی{\kB{H}(\omega)=\Fourier[h(t)]} اور \عددی{\kB{V}_0(\omega)=\Fourier[v_0(t)]} ہوں تب
\begin{align}\label{مساوات_فوریئر_حل_دور}
\kB{V}_0(\omega)=\bH(\omega) \kB{V}_d(\omega)
\end{align}
ہو گا۔
\begin{figure}
\centering
\begin{tikzpicture}
\draw(0,0) rectangle ++(\x,\y/2);
\draw(0,\y/4)--++(-\x/2,0)node[left]{$\kB{V}_d(\omega)$};
\draw[-latex](\x,\y/4)--++(\x/2,0)node[right]{$\kB{V}_0(\omega)=\bH(\omega) \kB{V}_d(\omega)$};
\draw(\x/2,\y/4)node{$\bH(\omega)$};
\end{tikzpicture}
\caption{وقتی الجھاو۔}
\label{شکل_فوریئر_وقتی_الجھاو}
\end{figure}
%================
\begin{table}
\caption{فوریئر بدل کے مسئلے۔}
\label{جدول_فوریئر_بدل_مسئلے}
\centering
\begin{tabular}{l l r}
$\Fourier(\omega)$ & $f(t)$ & مسئلہ \Bstrut \\ 
\hline
$A\bF(\omega) $& $Af(t)$ & خطیت\Tstrut \\ [2ex]
$\bF_1(\omega)\mp \bF_2(\omega) $& $f_1(t)\mp f_2(t)$&\\[2ex]
$\frac{1}{a}\bF\left(\frac{\omega}{a}\right), \, a>0 $& $f(at)$ & مسئلہ متناسب وقت\\[2ex]
$e^{-j\omega t_0} \bF(\omega)$ &$ f(t-t_0)$ & مسئلہ منتقلی وقت\\[2ex]
$\bF(\omega-\omega_0)$ & $e^{j\omega t_0} f(t)$ & ترمیم تعدد\\[2ex]
$(j\omega)^n \bF(\omega)$ &$\frac{\dif^{\,n} f(t)}{\dif t^n}$ & \\[2ex]
$(j)^n \frac{\dif^{\, n} \bF(\omega)}{\dif \omega^n}$&$t^n f(t)$&تفرق\\[2ex]
$\bF_1(\omega)\bF_2(\omega)$ & $\int_{-\infty}^{\infty} f_1(x) f_2(t-x) \dif x$ & \\[2ex]
$\frac{1}{2\pi} \int_{-\infty}^{\infty} \bF_1(x)\bF_2(\omega-x) \dif x$&$f_1(t) f_2(t)$&الجھاو  
\end{tabular}
\end{table}
%==================

\ابتدا{مشق}\شناخت{مشق_فوریئر_دور_کا_حل}
شکل \حوالہ{شکل_فوریئر_دور_کا_حل} میں داخلی دباو \عددی{v_d(t)=e^{-t}u(t)\,\si{\volt}}،  دور کا اکائی ضرب رد عمل \عددی{h(t)=e^{-2t}u(t)} جبکہ ابتدائی معلومات صفر ہیں۔خارجی دباو \عددی{v_0(t)} دریافت کریں۔
\begin{figure}
\centering
\begin{tikzpicture}
\draw(0,0) rectangle ++(\x,\y/2);
\draw(0,\y/4)--++(-\x/2,0)node[left]{$v_d(t)$};
\draw[-latex](\x,\y/4)--++(\x/2,0)node[right]{$v_0(t)$};
\draw(\x/2,\y/4)node{$h(t)$};
\end{tikzpicture}
\caption{مشق \حوالہ{مشق_فوریئر_دور_کا_حل} کا دور۔}
\label{شکل_فوریئر_دور_کا_حل}
\end{figure}


جواب:\عددی{v_0(t)=(e^{-t}-e^{-2t})u(t)\,\si{\volt}}
\انتہا{مشق}
%================
\ابتدا{مشق}\شناخت{مشق_فوریئر_دور_کا_حل_ب}
شکل \حوالہ{شکل_فوریئر_دور_کا_حل_ب} میں \عددی{v_d(t)=42\cos 3t \,\si{\volt}} ہے۔فوریئر بدل کے طریقے سے \عددی{v_0(t)} حاصل کریں۔
\begin{figure}
\centering
\begin{tikzpicture}[american voltages]
\draw(0,0) to [american voltage source,l={$v_d(t)$}]++(0,\y) to [resistor,l={$\SI{2}{\ohm}$}]++(\x,0) to [capacitor,l={$\SI{0.2}{\farad}$}]++(\x,0) to [resistor,l_={$\SI{1}{\ohm}$},v^<={$v_0(t)$}]++(0,-\y) to [short] (0,0);
\draw (\x,0) to [inductor,*-*,l={$\SI{0.4}{\henry}$}]++(0,\y);
\end{tikzpicture}
\caption{مشق \حوالہ{مشق_فوریئر_دور_کا_حل_ب} کا دور۔}
\label{شکل_فوریئر_دور_کا_حل_ب}
\end{figure}

جواب:\عددی{v_0(t)=12.57\cos(10t+86.2^{\circ})\,\si{\volt}}
\انتہا{مشق}
%=================

\حصہ{مسئلہ پارسیوال}
\اصطلاح{مسئلہ پارسیوال}\فرہنگ{مسئلہ!پارسیوال}\حاشیہب{Parseval's theorem}\فرہنگ{Parseval's theorem}\فرہنگ{theorem!Parseval} کی الجبرائی صورت درج ذیل ہے۔
\begin{align}
\int_{-\infty}^{\infty} f^2(t) \dif t=\frac{1}{2\pi}\int_{-\infty}^{\infty} \abs{\bF(\omega)}^2\dif \omega
\end{align}
اس تعلق کو حاصل کرتے ہیں۔
\begin{align*}
\int_{-\infty}^{\infty} f^2(t) \dif t&=\int_{-\infty}^{\infty} f(t) \frac{1}{2\pi} \int_{-\infty}^{\infty} \bF(\omega) e^{j\omega t} \dif \omega \dif t\\
&= \frac{1}{2\pi} \int_{-\infty}^{\infty}\bF(\omega)\int_{-\infty}^{\infty}  f(t) e^{-j(-\omega) t} \dif t \dif \omega \\
&= \frac{1}{2\pi} \int_{-\infty}^{\infty}\bF(\omega)\bF(-\omega) \dif \omega \\
&=\frac{1}{2\pi}  \int_{-\infty}^{\infty} \bF(\omega)\bF^*(\omega) \dif \omega \\
&=\frac{1}{2\pi}  \int_{-\infty}^{\infty} \abs{\bF(\omega)}^2\dif \omega 
\end{align*}
فرض کریں کہ \عددی{\SI{1}{\ohm}} کی مزاحمت میں رو \عددی{f(t)} ہے۔اس مزاحمت میں طاقت \عددی{f^2(t)} اور توانائی کا ضیاع \عددی{\int f^2(t) \dif t} ہو گا۔مسئلہ پارسیوال کہتا ہے کہ \عددی{\SI{1}{\ohm}} کی مزاحمتی ضیاع یا تقابل پذیر ضیاع کو وقتی دائرہ کار یا تعددی دائرہ کار میں حاصل کیا جا سکتا ہے۔ 

پٹی گزار چھلنی جو \عددی{\omega_1 \le \omega \le \omega_2} کے تعدد گزارتا ہو درج ذیل توانائی گزارے گی۔
\begin{align}
W=\int_{-\infty}^{\infty} f^2(t) \dif t&=\frac{1}{2\pi}  \int_{\omega_1}^{\omega_2} \abs{\bF(\omega)}^2\dif \omega 
\end{align}
یوں \عددی{\Delta \omega} کی باریک تعددی پٹی میں درج ذیل توانائی پائی جائے گی۔
\begin{align}
\Delta W=\frac{1}{2\pi} \abs{\bF(\omega)}^2 \Delta \omega
\end{align}
اگرچہ توانائی کو پارسیوال تکمل کے دونوں اطراف سے حاصل کیا جا سکتا ہے، تعددی دائرہ کار میں توانائی کسی مخصوص تعددی پٹی میں بھی حاصل کی جا سکتی ہے۔
%==============
\ابتدا{مثال}\شناخت{مثال_فوریئر_مزاحمت_امالہ}
شکل \حوالہ{شکل_فوریئر_مزاحمت_امالہ} میں \عددی{v_d(t)=10e^{-4t} u(t)\,\si{\volt}} کی صورت میں \عددی{v_0(t)} فوریئر طریقے سے حاصل کریں۔ اسی دور کو فوریئر طریقے سے \عددی{v_d(t)=15\cos 5t\,\si{\volt}} کے لئے بھی حل کریں۔ 
\begin{figure}
\centering
\begin{tikzpicture}[american voltages]
\draw(0,0) to [american voltage source,l={$v_d(t)$}]++(0,\y) to [resistor,l={$\SI{4}{\ohm}$}]++(\x,0) to [capacitor,l_={$\SI{0.5}{\farad}$},v^<={$v_0(t)$}]++(0,-\y) to [short] (0,0);
\end{tikzpicture}
\caption{مثال \حوالہ{مثال_فوریئر_مزاحمت_امالہ} اور مثال \حوالہ{مثال_فوریئر_مزاحمت_برق_گیر} کا دور۔}
\label{شکل_فوریئر_مزاحمت_امالہ}
\end{figure}

حل:داخلی دباو \عددی{v_d(t)} کا فوریئر بدل جدول \حوالہ{جدول_فوریئر_بدل_جوڑیاں} سے لکھتے ہیں۔
\begin{align*}
\bV_d(\omega)=\frac{10}{4+j\omega}
\end{align*}
دور کا \عددی{\bH(\omega)} درج ذیل ہے۔
\begin{align*}
\bH(\omega)=\frac{1}{1+j2\omega}
\end{align*}
مساوات \حوالہ{مساوات_فوریئر_حل_دور} کے تحت
\begin{align*}
\kB{V}_0(\omega)&=\bH(\omega) \kB{V}_d(\omega)\\
&=\frac{10}{(1+j2\omega)(4+j\omega)}\\
&=\frac{10}{7(j\omega+0.5)}-\frac{10}{7(j\omega+4)}
\end{align*}
ہو گا۔جدول \حوالہ{جدول_فوریئر_بدل_جوڑیاں} سے الٹ فوریئر بدل لیتے ہوئے درج ذیل حاصل ہوتا ہے۔
\begin{align*}
v_0(t)=\frac{10}{7} e^{-0.5t}u(t)-\frac{10}{7}e^{-4t}u(t) \,\si{\volt}
\end{align*}
آئیں اب \عددی{v_d(t)=15\cos 5t \,\si{\volt}} کا فوریئر بدل جدول سے لکھیں۔
\begin{align*}
\bV_d(\omega)&=15\pi\delta(\omega-5)+15\pi\delta(\omega+5)
\end{align*}
خارجی دباو لکھتے ہیں۔
\begin{align*}
\bV_0(\omega)&=\bH(\omega)\bV_d(\omega)\\
&=\frac{15\pi}{1+j2\omega}[\delta(\omega-5)+\delta(\omega+5)]
\end{align*}
فوریئر الٹ بدل لیتے ہیں
\begin{align*}
v_0(t)&=\frac{1}{2\pi}\int_{-\infty}^{\infty} 15\pi \frac{\delta(\omega-5)+\delta(\omega+5)}{1+j2\omega} e^{j\omega t} \dif \omega\\
&=7.5 \frac{e^{j5t}}{1+j10}+7.5\frac{e^{-j5t}}{1-j10}\\
&=\frac{7.5e^{5t}}{\sqrt{101}e^{j84.3^{\circ}}}+\frac{7.5e^{-5t}}{\sqrt{101}e^{-j84.3^{\circ}}}\\
&=1.493\cos(5t-84.3^{\circ})\,\si{\volt}
\end{align*}
جہاں تکمل کو نمونہ بندی کی خاصیت سے حاصل کیا گیا۔
\انتہا{مثال}
%================
\ابتدا{مثال}\شناخت{مثال_فوریئر_مزاحمت_برق_گیر}
شکل \حوالہ{شکل_فوریئر_مزاحمت_امالہ} میں  \عددی{v_d(t)=10e^{-4t} u(t)\,\si{\volt}} کی صورت میں داخلی اور خارجی تقابل پذیر توانائی دریافت کریں۔

حل:اکائی اوہم داخلی توانائی حاصل کرتے ہیں۔
\begin{align*}
W_d&=\int_0^{\infty} f^2(t) \dif t\\
&=\int_0^{\infty} 100 e^{-8t} \dif t\\
&=\left. \frac{100e^{-8t}}{-8}\right|_{0}^{\infty}\\
&=\SI{12.5}{\joule}
\end{align*}
خارجی توانائی کو مسئلہ پارسیوال سے حاصل کرتے ہیں۔گزشتہ مثال میں درج ذیل حاصل کیا گیا
 \begin{align*}
\kB{V}_0(\omega)&=\frac{10}{(j2\omega+1)(j\omega+4)}
\end{align*}
لہٰذا
\begin{align*}
\abs{\kB{V}_0}^2&=\frac{100}{(4\omega^2+1)(\omega^2+16)}\\
&=\frac{100}{63(\omega^2+0.25)}-\frac{100}{63(\omega^2+16)}
\end{align*}
ہو گا اور تقابل پذیر خارجی توانائی
\begin{align*}
W_d&=\frac{1}{2\pi}\int_{-\infty}^{\infty}\abs{\bV_0(\omega)}^2 \dif \omega\\
&=\frac{50}{63\pi}\left(\int_{-\infty}^{\infty}\frac{\dif \omega}{\omega^2+0.5^2}-\int_{-\infty}^{\infty}\frac{\dif \omega}{\omega^2+4^2} \right)\\
&=\frac{50}{63\pi}\left(\left. \frac{1}{0.5}\tan^{-1} \frac{\omega}{0.5} \right|_{-\infty}^{\infty}-\left. \frac{1}{4}\tan^{-1} \frac{\omega}{4} \right|_{-\infty}^{\infty}\right)\\
&=\frac{50}{63\pi}\left(\frac{\pi}{0.5}-\frac{\pi}{4}\right)\\
&=\frac{25}{18}\,\si{\joule}
\end{align*}
ہو گی۔
\انتہا{مثال}
%==============================
\ابتدا{مثال}
تفاعل \عددی{f(t)=e^{-at}u(t)} کی اکائی اوہم توانائی دریافت کریں۔وہ  انقطاعی تعدد \عددی{\omega_0} دریافت کریں جس سے کم تعدد کی پٹی میں \عددی{\SI{95}{\percent}} توانائی پائی جاتی ہو۔

حل:اکائی اوہم توانائی حاصل کرتے ہیں۔
\begin{align*}
W&=\int_{-\infty}^{\infty} f^2(t)\dif t\\
&=\int_{-\infty}^{\infty} e^{-2at}u(t) \dif t\\
&=&=\int_{0}^{\infty} e^{-2at} \dif t\\
&=\left. \frac{e^{-2at}}{-2a}  \right|_{0}^{\infty}\dif t\\
&=\frac{1}{2a}
\end{align*}
تفاعل کا فوریئر بدل درج ذیل ہے
\begin{align*}
\bF(\omega)=\frac{1}{j\omega+a}
\end{align*}
لہٰذا 
\begin{align*}
\frac{1}{2\pi}\int_{-\infty}^{\infty} \abs{\bF(\omega)}^2 \dif \omega=\frac{1}{2\pi}\int_{-\infty}^{\infty}\frac{\dif \omega}{\omega^2+a^2}  =\frac{1}{2a}
\end{align*}
ہو گا۔ درج بالا کو ثابت کیا جا سکتا ہے۔چونکہ \عددی{\bF(\omega)} جفت تفاعل ہے لہٰذا
\begin{align*}
\frac{1}{2\pi}\int_{-\infty}^{\infty}\frac{\dif \omega}{\omega^2+a^2}&=\frac{2}{2\pi}\int_{0}^{\infty}\frac{\dif \omega}{\omega^2+a^2}\\
&=\left. \frac{1}{\pi a} \tan^{-1}\frac{\omega}{a}\right|_{0}^{\infty}\\
&=\frac{1}{2a}
\end{align*}
ہم تعدد \عددی{\omega_0} جاننا چاہتے ہیں جس کے لئے درج ذیل لکھا جا سکتا ہے
\begin{align*}
\frac{0.95}{2a}&=\frac{1}{\pi}\int_0^{\omega_0} \frac{\dif \omega}{\omega^2+a^2}\\
&=\left. \frac{1}{\pi a} \tan^{-1}\frac{\omega}{a} \right|_0^{\omega_0}\\
&=\frac{1}{\pi a} \tan^{-1} \frac{\omega_0}{a}
\end{align*}
جس سے
\begin{align*}
\omega_0=\SI{12.706}{\radian}
\end{align*}
حاصل ہوتا ہے لہٰذا \عددی{\SI{95}{\percent}} توانائی \عددی{0 \le \omega \le \SI{12.706}{\radian}} کی پٹی میں پائی جاتی ہے۔ 
\انتہا{مثال}
%=============================
\ابتدا{مشق}
تفاعل \عددی{f(t)=e^{-4t}u(t)} کی اکائی اوہم توانائی وقتی دائرہ کار اور تعددی دائرہ کار میں حاصل کریں۔

جواب:\عددی{\SI{125}{\milli\joule}}
\انتہا{مشق}
%===============================

\ابتدا{مشق}
تفاعل \عددی{f(t)=e^{-4t}u(t)} کی اکائی اوہم توانائی \عددی{0 \le \omega \le \SI{1}{\radian}} کی پٹی میں حاصل کریں۔

جواب:\عددی{\SI{19.49}{\milli\joule}}
\انتہا{مشق}
%===============================
\ابتدا{مشق}\شناخت{مشق_فوریئر_امالی_دور_الف}
شکل \حوالہ{شکل_فوریئر_امالی_دور_الف} میں \عددی{v_0(t)} کی اکائی اوہم توانائی دریافت کریں۔
\begin{figure}
\centering
\begin{tikzpicture}[american voltages]
\draw(0,0) to [american voltage source,l={$6e^{-3t} u(t)$}]++(0,\y) to [resistor,l={$\SI{2}{\ohm}$}]++(\x,0) to [inductor,l_={$\SI{1}{\henry}$},v^<={$v_0(t)$}]++(0,-\y) to [short] (0,0);
\end{tikzpicture}
\caption{مشق \حوالہ{مشق_فوریئر_امالی_دور_الف} کا دور۔}
\label{شکل_فوریئر_امالی_دور_الف}
\end{figure}

جواب:\عددی{W=\SI{3.6}{\joule}}
\انتہا{مشق}
%===============================
\ابتدا{مثال}\شناخت{مثال_فوریئر_نشریات}
بے تار نشریات میں بلند تعدد \عددی{v_c(t)=A\cos \omega_c t} کے سائن نما سواری پر سمعی اشارہ \عددی{f_s} سوار کیا جاتا ہے۔ سمعی اشارے کی تعدد \عددی{\SI{20}{\hertz} \le f_s \le \SI{20000}{\hertz}} ہوتی ہے جبکہ \عددی{f_c} کی قیمت اس کی نسبت انتہائی زیادہ ہوتی ہے۔سمعی اشارے \عددی{v_s(t)} سے سواری اشارے کا حیطہ ترمیم کرتے ہوئے حیطہ سوار اشارہ حاصل کیا جاتا ہے جس کی الجبرائی صورت درج ذیل ہے۔
\begin{align}
v(t)=[A+v_s(t)]\cos(\omega_c t)
\end{align}
اینٹینا کے ذریعہ \عددی{v(t)} کو نشر کیا جاتا ہے۔اشارہ حاصل کرنے والا اینٹینا \عددی{v(t)} کا دھندلا نقش  حاصل کرتا ہے جس کا حیطہ نہایت کم لیکن صورت ہوبہو \عددی{v(t)} جیسے ہوتی ہے۔

آئیں سمعی اشارہ \عددی{v_s(t)}
\begin{align}
v_s(t)=\cos \omega_s t
\end{align}
اور  نشر اشارہ \عددی{v(t)}
\begin{align}
v(t)=[1+\cos \omega_s t] \cos \omega_c t
\end{align}
کے لکیری طیف حاصل کریں جہاں \عددی{f_s=\SI{1}{\kilo\hertz}}، \عددی{f_c=\SI{1260}{\kilo\hertz}} اور \عددی{A=1} ہیں۔پشاور شہر کے ریڈیو اسٹیشن کی نشریات \عددی{\SI{1260}{\kilo\hertz}} پر ہوتی ہے۔ 

سمعی اشارے کے فوریئر بدل
\begin{align}\label{مساوات_فوریئر_سمعی_اشارہ}
\bV_s(\omega)=\Fourier[\cos \omega_s t]=\pi \delta(\omega-\omega_s)+\pi\delta(\omega+\omega_s)
\end{align}
کو شکل \حوالہ{شکل_فوریئر_نشریات}-الف میں دکھایا گیا ہے۔نشر اشارے کو درج ذیل لکھتے
\begin{align*}
v(t)=[1+v_s(t)]\cos \omega_c t=\cos \omega_c t+v_s(t) \cos \omega_c t
\end{align*}
ہیں جس میں \عددی{\cos \omega_c t} کا فوریئر بدل
\begin{align}
\Fourier[\cos \omega_c t]=\pi \delta(\omega-\omega_c)+\pi\delta(\omega+\omega_c)
\end{align}
ہے جبکہ
\begin{align*}
v_s(t)\cos \omega_c t=v_s(t) \frac{e^{j\omega_c t}+e^{-j\omega_c t}}{2}
\end{align*}
لکھتے ہوئے  فوریئر بدل کے مسئلہ ترمیم کی مدد سے درج ذیل لکھا جا سکتا ہے۔
\begin{align*}
\Fourier[v_s(t)\cos \omega-c t]=\frac{1}{2}\left[\bV_s(\omega-\omega_c)+\bV_s(\omega+\omega_c)\right]
\end{align*}
مساوات \حوالہ{مساوات_فوریئر_سمعی_اشارہ} میں دیے سمعی اشارے کا بدل استعمال کرتے ہوئے یوں درج ذیل لکھا جا سکتا ہے۔
\begin{multline*}
\Fourier[v_s(t)\cos \omega_c t]=\frac{\pi}{2}\left[\delta(\omega-\omega_s-\omega_c)+\delta(\omega+\omega_s-\omega_c)\right.\\
\left.+\delta(\omega-\omega_s+\omega_c)+\delta(\omega+\omega_s+\omega_c)\right]
\end{multline*}
ان تمام کو استعمال کرتے ہوئے نشر اشارے کا بدل لکھتے ہیں جسے شکل \حوالہ{شکل_فوریئر_نشریات}-ب میں دکھایا گیا ہے۔
\begin{multline}
\bV(\omega)=\frac{\pi}{2}\left[2\delta(\omega-\omega_c)+2\delta(\omega+\omega_c)+\delta(\omega-\omega_s-\omega_c)\right.\\
\left.+\delta(\omega+\omega_s-\omega_c)+\delta(\omega-\omega_s+\omega_c)+\delta(\omega+\omega_s+\omega_c)\right]
\end{multline}
%
\begin{figure}
\centering
\begin{subfigure}{1\textwidth}
\centering
\begin{tikzpicture}
\draw(-2,0)--(2.5,0)node[right]{$f \,(\si{\hertz})$};
\draw(0,0)--++(0,2.5)node[left]{$\bV_s(\omega)$};
\draw[-latex](-1.5,0)node[below]{$-1000$}--++(0,1.5);
\draw[-latex](1.5,0)node[below]{$1000$}--++(0,1.5);
\draw(0,1.5)--++(-0.1,0)node[left]{$\pi$};
\end{tikzpicture}
\caption*{(الف) سمعی اشارے کا لکیری طیف۔}
\end{subfigure}
\begin{subfigure}{1\textwidth}
\centering
\begin{tikzpicture}
\draw(-5,0)--(5.5,0)node[right]{$f \,(\si{\kilo\hertz})$};
\draw(0,0)--++(0,2.5)node[left]{$\bV(\omega)$};
\draw[-latex](-3,0)node[below]{$-1260$}--++(0,1.5);
\draw[-latex](-4.5,0)node[below]{$-1261$}--++(0,0.75);
\draw[-latex](-1.5,0)node[below]{$-1259$}--++(0,0.75);
\draw[-latex](3,0)node[below]{$1260$}--++(0,1.5);
\draw[-latex](4.5,0)node[below]{$1261$}--++(0,0.75);
\draw[-latex](1.5,0)node[below]{$1259$}--++(0,0.75);
\draw(0,1.5)--++(-0.1,0)node[left]{$\pi$};
\draw(0,0.75)--++(-0.1,0)node[left]{$\frac{\pi}{2}$};
\end{tikzpicture}
\caption*{(ب) نشر اشارے کا لکیری طیف۔}
\end{subfigure}%
\caption{مثال \حوالہ{مثال_فوریئر_نشریات} کے لکیری طیف۔}
\label{شکل_فوریئر_نشریات}
\end{figure}
\انتہا{مثال}
%=================================
\ابتدا{مثال}\شناخت{مشق_فوریئر_انورٹر}
انجنیئرنگ کی ڈگری لینے کے بعد میری پہلی ذمہ داری سائن نما \اصطلاح{انورٹر}\فرہنگ{انورٹر}\حاشیہب{inverter}\فرہنگ{inverter} کی تخلیق\حاشیہد{میں نے کام کا آغاز اسلام آباد کے نجی ادارے \تحریر{rwr} (موجودہ نام) سے کیا جہاں میں نے سائن نما انورٹر پر ہی انجنیئرنگ تخلیق سیکھی۔} تھی۔انورٹر یک سمتی دباو کو بدلتے دباو میں تبدیل کرتا ہے۔غیر سائن نما انورٹر مستطیلی دباو پیدا کرتا ہے جس کو شکل \حوالہ{شکل_فوریئر_انورٹر} میں دکھایا گیا ہے۔آئیں اس پر غور کریں۔
\begin{figure}
\centering
\begin{tikzpicture}
\begin{axis}[xlabel={$t$},ylabel={$v(t)$},ylabel style={rotate=-90},ytick={-1,1},yticklabels={$-V_0$,$V_0$},xtick={-1.5,-0.75,0,0.75,1.5},xticklabels={$-\frac{T}{2}$,$-\frac{T}{4}$,$0$,$\frac{T}{4}$,$\frac{T}{2}$}]
\addplot[] plot coordinates{(-2.5,0)(-2,0)(-2,-1)(-1,-1)(-1,0)(-0.5,0)(-0.5,1) (0.5,1)(0.5,0)(1,0)(1,-1)(2,-1)(2,0) (2.5,0)(2.5,1)(3.5,1)(3.5,0)(4,0)};
\addplot[stealth-stealth] plot coordinates {(-0.5,0.5) (0.5,0.5)}node[pos=0.5,above]{$2\delta$};
\addplot[stealth-stealth] plot coordinates {(1,-0.5) (2,-0.5)}node[pos=0.5,above]{$2\delta$};
\end{axis}
\end{tikzpicture}
\caption{مشق \حوالہ{مشق_فوریئر_انورٹر} کے انورٹر کا مستطیلی دباو۔}
\label{شکل_فوریئر_انورٹر}
\end{figure}    

شکل میں دکھائے گئے دباو کی اوسط قیمت صفر ہے لہٰذا \عددی{a_0=0} ہو گا۔ساتھ ہی چونکہ دباو جفت ہے لہٰذا \عددی{b_n=0} ہوں گے۔آئیں \عددی{a_n} دریافت کریں۔
\begin{align*}
a_n&=\frac{4}{T}\int_0^{\frac{T}{2}} v(t)\cos (n\omega_0 t)\dif t\\
&=\frac{4}{T}\left[\int_0^{\delta} V_0\cos(n\omega_0 t) \dif t-\int_{\frac{T}{2}-\delta}^{\frac{T}{2}}V_0 \cos(n\omega_0 t)\dif t\right]\\
&=\frac{4V_0}{T}\left. \frac{\sin(n\omega_0 t)}{n\omega_0}\right|_{0}^{\delta}-\frac{4V_0}{T}\left. \frac{\sin(n\omega_0 t)}{n\omega_0}\right|_{\frac{T}{2}-\delta}^{\frac{T}{2}}\\
&=\frac{4V_0}{n\omega_0 T}\sin(n\omega_0 \delta)-\frac{4V_0}{n\omega_0 T}\sin\left(n\omega_0 \frac{T}{2}\right)+\frac{4V_0}{n\omega_0 T}\sin[n\omega_0 (\frac{T}{2}-\delta)]
\end{align*}
اس میں \عددی{\omega_0=\tfrac{2\pi}{T}} پر کرتے ہیں۔
\begin{align*}
a_n&=\frac{4V_0}{n\frac{2\pi}{T} T}\sin(n\frac{2\pi}{T}\delta)-\frac{4V_0}{n\frac{2\pi}{T} T}\sin\left(n\frac{2\pi}{T} \frac{T}{2}\right)+\frac{4V_0}{n\frac{2\pi}{T} T}\sin[n\frac{2\pi}{T} (\frac{T}{2}-\delta)]\\
&=\frac{2V_0}{n\pi}\sin(n\frac{2\pi}{T}\delta)-\frac{2V_0}{n\pi}\sin n\pi+\frac{2V_0}{n\pi}\sin[n\frac{2\pi}{T} (\frac{T}{2}-\delta)]\\
&=\frac{2V_0}{n\pi}\sin\left(\frac{2n\pi\delta}{T}\right)+\frac{2V_0}{n\pi}\sin\left(n\pi-\frac{2n\pi\delta}{T}\right)
\end{align*}
یہاں دائیں ہاتھ پر \عددی{\sin(\alpha-\beta)=\sin \alpha\cos \beta-\cos\alpha\sin\beta} استعمال کرتے ہیں۔
\begin{align*}
a_n&=\frac{2V_0}{n\pi}\sin\left(\frac{2n\pi\delta}{T}\right)+\frac{2V_0}{n\pi}\left[\sin n\pi \cos(\frac{2n\pi\delta}{T})-\cos n\pi \sin(\frac{2n\pi\delta}{T})\right]\\
&=\frac{2V_0}{n\pi}\sin\left(\frac{2n\pi\delta}{T}\right)-\frac{2V_0}{n\pi}\cos n\pi \sin(\frac{2n\pi\delta}{T})\\
&=\frac{2V_0}{n\pi}\sin\left(\frac{2n\pi\delta}{T}\right)(1-\cos n\pi)
\end{align*}
اس مساوات میں \عددی{n=1,2,3,\cdots} پر کرتے ہوئے چند عددی سر لکھتے ہیں۔
\begin{align*}
a_1&=\frac{4V_0}{\pi}\sin\frac{2\pi\delta}{T}\\
a_2&=0\\
a_3&=\frac{4V_0}{3\pi}\sin\frac{6\pi\delta}{T}\\
a_4&=0\\
a_5&=\frac{4V_0}{5\pi}\sin\frac{10\pi\delta}{T}
\end{align*}
ہم درج بالا عددی سر کی معلومات استعمال کرتے ہوئے \عددی{\delta} یوں رکھ سکتے ہیں کہ مخصوص عددی سر صفر کے برابر ہو جائے مثلاً \عددی{a_3} کی قیمت صفر کرنے کی خاطر \عددی{\tfrac{6\pi\delta}{T}=\pi} کرنا ہو گا یعنی \عددی{\delta=\tfrac{T}{3}} رکھا جائے گا۔
\انتہا{مثال}
%==================================
\ابتدا{مثال}\شناخت{مثال_فوریئر_دندانہ_چھلنی_متعدد}
شکل \حوالہ{شکل_فوریئر_دندانہ_چھلنی_متعدد}-الف میں \عددی{RLC} دندانہ چھلنی دکھائی گئی ہے۔سلسلہ وار \عددی{LC} کی قدرتی تعدد \عددی{\omega_0=\tfrac{1}{\sqrt{LC}}} پر ان کی سلسلہ وار رکاوٹ صفر ہو جاتی ہے لہٰذا \عددی{\omega_0} تعدد کا اشارہ خارجی جانب نہیں پایا جاتا۔اب تصور کریں کہ \عددی{\SI{500}{\hertz}} پر چلنے والے نظام کے قریب واپڈا کے \عددی{\SI{50}{\hertz}} دباو پر چلنے والی مشین \عددی{\SI{50}{\hertz}} اور اس کے دیگر ہارمونی تعدد کا برقی شور پیدا کرتی ہے جو \عددی{\SI{1}{\kilo\hertz}} کی نظام میں مداخلت پیدا کرتی ہے۔ان میں پہلا، دوسرا اور تیسرا ہارمونی شور زیادہ مسئلہ پیدا کرتے ہیں جن سے چھٹکارا ضروری ہے۔ 
\begin{figure}
\centering
\begin{subfigure}{1\textwidth}
\centering
\begin{tikzpicture}[american voltages]
\draw(0,0) to [american voltage source,l={$v_d(t)$}]++(0,2*\y) to [resistor,l={$R$}]++(\x,0) to [inductor,l_={$L$}]++(0,-\y) to [capacitor,l_={$C$}]++(0,-\y) to [short] (0,0);
\draw(\x,2*\y) to [short,*-o]++(\x,0);
\draw(\x,0) to [short,*-o]++(\x,0);
\draw(2*\x+0.25,\y) node{$\begin{aligned}&+ \\ \\ &v_0(t) \\ \\ &-  \end{aligned}$};
\end{tikzpicture}
\caption*{(الف)}
\end{subfigure}
\begin{subfigure}{1\textwidth}
\centering
\begin{tikzpicture}[american voltages]
\draw(0,0) to [american voltage source,l={$v_d(t)$}]++(0,2*\y) to [resistor,l={$R$}]++(\x,0) to [inductor,l_={$L_1$}]++(0,-\y) to [capacitor,l_={$C_1$}]++(0,-\y) to [short] (0,0);
\draw(\x,2*\y) to [short,*-]++(\x,0) to [inductor,l_={$L_2$}]++(0,-\y) to [capacitor,l_={$C_2$}]++(0,-\y) to [short,-*]++(-\x,0);
\draw(2*\x,2*\y) to [short,*-]++(\x,0) to [inductor,l_={$L_3$}]++(0,-\y) to [capacitor,l_={$C_3$}]++(0,-\y) to [short,-*]++(-\x,0);
\draw(3*\x,2*\y) to [short,*-o]++(\x,0);
\draw(3*\x,0) to [short,*-o]++(\x,0);
\draw(4*\x+0.25,\y) node{$\begin{aligned}&+ \\ \\ &v_0(t) \\ \\ &-  \end{aligned}$};
\end{tikzpicture}
\caption*{(ب)}
\end{subfigure}
\caption{مثال \حوالہ{مثال_فوریئر_دندانہ_چھلنی_متعدد} کی دندانہ چھلنی۔}
\label{شکل_فوریئر_دندانہ_چھلنی_متعدد}
\end{figure}

شکل \حوالہ{شکل_فوریئر_دندانہ_چھلنی_متعدد}-ب میں تین عدد \عددی{LC} جوڑیاں استعمال کی گئی ہیں جن میں \عددی{2\pi f_1=\tfrac{1}{\sqrt{L_1 C_1}}} رکھتے ہوئے \عددی{f_1=\SI{50}{\hertz}} کو رد کیا جاتا ہے۔اسی طرح \عددی{f_2=\SI{100}{\hertz}} کی دوسری ہارمونی اور \عددی{f_3=\SI{150}{\hertz}} کی تیسری ہارمونی کو  
\عددی{2\pi f_2=\tfrac{1}{\sqrt{L_2 C_2}}} اور \عددی{2\pi f_3=\tfrac{1}{\sqrt{L_3 C_3}}} چنتے ہوئے رد کیا جاتا ہے۔ 
\انتہا{مثال}

%=========================
\حصہء{سوالات}

%========================
\ابتدا{سوال}\شناخت{سوال_فوریئر_چکور_الف}
شکل \حوالہ{شکل_سوال_فوریئر_چکور_الف}-الف کے  عددی سر حاصل کرتے ہوئے اس کی تکونیاتی فوریئر تسلسل لکھیں۔
\begin{figure}
\centering
\begin{subfigure}{0.5\textwidth}
\centering
\begin{tikzpicture}
\begin{axis}[small,xtick={0,0.2,1,1.2},xticklabels={$0$,$0.2$,$1$,$1.2$},ytick={0,1},yticklabels={$0$,$1$},xlabel={$t$},ylabel={$v(t)$},ylabel style={rotate=-90},ylabel style={at={(axis description cs:0,1.05)}}]
\addplot[] plot coordinates {(-0.1,0) (0,0) (0,1) (0.2,1) (0.2,0) (1,0) (1,1) (1.2,1) (1.2,0) (1.5,0)};
\end{axis}
\end{tikzpicture}
\caption*{(الف)}
\end{subfigure}%
\begin{subfigure}{0.5\textwidth}
\centering
\begin{tikzpicture}
\begin{axis}[small,xtick={0,3,6,9,12},xticklabels={$0$,$3$,$6$,$9$,$12$},ytick={-1,0,1},yticklabels={$-1$,$0$,$1$},xlabel={$t$},ylabel={$i(t)$},ylabel style={rotate=-90},ylabel style={at={(axis description cs:0,1.05)}}]
\addplot[] plot coordinates {(-0.5,-1) (0,-1)(0,1)(3,1)(3,-1)(6,-1) (6,1) (9,1) (9,-1) (12,-1) (12,1) (12.4,1)};
\end{axis}
\end{tikzpicture}
\caption*{(ب)}
\end{subfigure}%
\caption{سوال \حوالہ{سوال_فوریئر_چکور_الف} کی موج۔}
\label{شکل_سوال_فوریئر_چکور_الف}
\end{figure}

جوابات:\عددی{a_0=0.2}، \عددی{a_n=\tfrac{1}{n\pi}\sin \tfrac{2\pi n}{5}}، \عددی{b_n=\tfrac{1}{n\pi}(1-\cos \tfrac{2\pi n}{5})}، \\
\begin{multline*}
v(t)=0.2+0.3027\cos (2\pi t) +0.2199\sin (2\pi t)+0.0935\cos(4\pi t)\\
+0.2879\sin(4\pi t)-0.0623\cos(6\pi t)+0.1919\sin(6\pi t)+\cdots
\end{multline*}
\انتہا{سوال}
%=========================
\ابتدا{سوال}\شناخت{سوال_فوریئر_چکور_ب}
شکل \حوالہ{شکل_سوال_فوریئر_چکور_الف}-ب کے تکونی فوریئر تسلسل کے عددی سر حاصل کریں۔تسلسل کے شروع کے چند ارکان لکھیں۔

جوابات:\عددی{a_0=0}، \عددی{a_n=0}، \عددی{b_n=\tfrac{2}{n\pi}(1-(-1)^n)}،
\begin{align*}
i(t)=\tfrac{4}{\pi}[\sin(\tfrac{\pi}{2}t)+\tfrac{1}{3}\sin(\tfrac{3\pi}{2}t)+\tfrac{1}{5}\sin(\tfrac{5\pi}{2}t)+\cdots]
\end{align*}
\انتہا{سوال}
%==========================
\ابتدا{سوال}\شناخت{سوال_فوریئر_چکور_پ}
شکل \حوالہ{شکل_سوال_فوریئر_چکور_پ}-الف کے  عددی سر حاصل کرتے ہوئے اس کی تکونیاتی فوریئر تسلسل لکھیں۔
\begin{figure}
\centering
\begin{subfigure}{0.5\textwidth}
\centering
\begin{tikzpicture}
\begin{axis}[small,xtick={0,1,2,3,4,5},xticklabels={$0$,$1$,$2$,$3$,$4$,$5$},ytick={0,1,2},yticklabels={$0$,$1$,$2$},xlabel={$t$},ylabel={$v(t)$},ylabel style={rotate=-90},ylabel style={at={(axis description cs:0,1.05)}}]
\addplot[] plot coordinates {(-0.5,0)(0,0)(0,2)(1,2)(1,1)(2,1) (2,0) (3,0)(3,2)(4,2)(4,1)(5,1) (5,0) (5.5,0)};
\end{axis}
\end{tikzpicture}
\caption*{(الف)}
\end{subfigure}%
\begin{subfigure}{0.5\textwidth}
\centering
\begin{tikzpicture}
\begin{axis}[small,xtick={0,1,2,3,4,5,6,7},xticklabels={$0$,$1$,$2$,$3$,$4$,$5$,$6$,$7$},ytick={0,1,2},yticklabels={$0$,$1$,$2$},xlabel={$t$},ylabel={$i(t)$},ylabel style={rotate=-90},ylabel style={at={(axis description cs:0,1.05)}}]
\addplot[] plot coordinates {(-0.5,0)(0,0)(0,1)(1,1)(1,2)(2,2) (2,1) (3,1)(3,0)(4,0)(4,1)(5,1) (5,2) (6,2) (6,1) (7,1) (7,0) (7.5,0)};
\end{axis}
\end{tikzpicture}
\caption*{(ب)}
\end{subfigure}%
\caption{سوال \حوالہ{سوال_فوریئر_چکور_پ} کی موج۔}
\label{شکل_سوال_فوریئر_چکور_پ}
\end{figure}

جوابات:
\begin{align*}
a_0&=1\\
a_n&=0\\
b_n&=\tfrac{3}{\pi}, \tfrac{3}{2\pi}, 0, \tfrac{3}{4\pi}, \tfrac{3}{5\pi}, 0, \tfrac{3}{7\pi}, \cdots\\
i(t)&=1+\tfrac{3}{\pi}(\sin \tfrac{2\pi t}{3}+\tfrac{1}{2}\sin \tfrac{4\pi t}{3}+\tfrac{1}{4}\sin \tfrac{8\pi t}{3}+\cdots)
\end{align*}
\انتہا{سوال}
%=========================
\ابتدا{سوال}\شناخت{سوال_فوریئر_چکور_ت}
شکل \حوالہ{شکل_سوال_فوریئر_چکور_پ}-ب کے تکونی فوریئر تسلسل کے عددی سر حاصل کریں۔تسلسل کے شروع کے چند ارکان لکھیں۔

جوابات:
\begin{align*}
a_0&=1\\
a_n&=-\tfrac{2}{\pi},0,\tfrac{2}{3\pi},0,-\tfrac{2}{5\pi},0,\tfrac{2}{7\pi},\cdots\\
b_n&=\tfrac{2}{\pi},0,\tfrac{2}{3\pi},0,\tfrac{2}{5\pi},0,\tfrac{2}{7\pi},\cdots\\
i(t)&=1-\tfrac{2}{\pi}(\cos \tfrac{\pi t}{2}-\sin \tfrac{\pi t}{2}-\tfrac{1}{3}\cos \tfrac{3\pi t}{2}-\tfrac{1}{3}\sin \tfrac{\pi t}{2}+\cdots  )
\end{align*}
\انتہا{سوال}
%==========================
\ابتدا{سوال}\شناخت{سوال_فوریئر_چکور_ٹ}
شکل \حوالہ{شکل_سوال_فوریئر_چکور_ٹ}-الف کے  عددی سر حاصل کرتے ہوئے اس کی تکونیاتی فوریئر تسلسل لکھیں۔
\begin{figure}
\centering
\begin{subfigure}{0.5\textwidth}
\centering
\begin{tikzpicture}
\begin{axis}[small,xtick={0,3,6,9},xticklabels={$0$,$3$,$6$,$9$},ytick={0,1},yticklabels={$0$,$1$},xlabel={$t$},ylabel={$v(t)$},ylabel style={rotate=-90},ylabel style={at={(axis description cs:0,1.05)}}]
\addplot[] plot coordinates {(0,0)(3,1)(3,0)(6,1)(6,0)(9,1)(9,0)};
\end{axis}
\end{tikzpicture}
\caption*{(الف)}
\end{subfigure}%
\begin{subfigure}{0.5\textwidth}
\centering
\begin{tikzpicture}
\begin{axis}[small,xtick={0,2,3,4,6,7,8},xticklabels={$0$,$2$,$3$,$4$,$6$,$7$,$8$},ytick={0,2},yticklabels={$0$,$2$},xlabel={$t$},ylabel={$i(t)$},ylabel style={rotate=-90},ylabel style={at={(axis description cs:0,1.05)}}]
\addplot[] plot coordinates {(-0.5,0)(0,0)(2,2)(3,2)(3,0)(4,0)(6,2)(7,2)(7,0)(7.5,0)};
\end{axis}
\end{tikzpicture}
\caption*{(ب)}
\end{subfigure}%
\caption{سوال \حوالہ{سوال_فوریئر_چکور_ٹ} کی موج۔}
\label{شکل_سوال_فوریئر_چکور_ٹ}
\end{figure}

جوابات:
\begin{align*}
a_0&=\tfrac{9}{4}\\
a_n&=0\\
b_n&=-\tfrac{9}{2n\pi}\\
v(t)&=\tfrac{9}{4}-\tfrac{9}{2\pi}\sum_{n=1}^{\infty} \tfrac{1}{n}\sin \tfrac{2n\pi t}{3}
\end{align*}
\انتہا{سوال}
%=========================
\ابتدا{سوال}\شناخت{سوال_فوریئر_چکور_ث}
شکل \حوالہ{شکل_سوال_فوریئر_چکور_ٹ}-ب کے تکونی فوریئر تسلسل کے عددی سر حاصل کریں۔

جوابات:
\begin{align*}
a_0&=1\\
a_n&=-\tfrac{2}{\pi}(1+\tfrac{2}{\pi}), \tfrac{2}{3\pi}(1-\tfrac{2}{3\pi}), -\tfrac{2}{5\pi}(1+\tfrac{2}{5\pi}), \tfrac{2}{7\pi}(1-\tfrac{2}{7\pi}),-\cdots\\
b_n&=0,\tfrac{1}{\pi},0, -\tfrac{1}{2\pi},0,\tfrac{1}{3\pi},0,-\cdots
\end{align*}
\انتہا{سوال}
%==========================
\ابتدا{سوال}\شناخت{سوال_فوریئر_چکور_ج}
شکل \حوالہ{شکل_سوال_فوریئر_چکور_ج}-الف کے  عددی سر حاصل کریں۔
\begin{figure}
\centering
\begin{subfigure}{0.5\textwidth}
\centering
\begin{tikzpicture}
\begin{axis}[small,xtick={-2,-1,0,1,2,4,5,7,8},xticklabels={$-2$,$-1$,$0$,$1$,$2$,$4$,$5$,$7$,$8$},ytick={0,1,4},yticklabels={$0$,$1$,$4$},xlabel={$t$},ylabel={$v(t)$},ylabel style={rotate=-90},ylabel style={at={(axis description cs:0,1.05)}}]
\addplot[] plot coordinates {(-2.5,0)(-2,0)(-2,1)(-1,1)(-1,4)(1,4)(1,1)(2,1)(2,0) (4,0)(4,1)(5,1)(5,4)(7,4)(7,1)(8,1)(8,0) (8.5,0)};
\end{axis}
\end{tikzpicture}
\caption*{(الف)}
\end{subfigure}%
\begin{subfigure}{0.5\textwidth}
\centering
\begin{tikzpicture}
\begin{axis}[small,xtick={-2,0,2,4,6},xticklabels={$-2$,$0$,$2$,$4$,$6$},ytick={0,1},yticklabels={$0$,$1$},xlabel={$t$},ylabel={$i(t)$},ylabel style={rotate=-90},ylabel style={at={(axis description cs:0,1.05)}}]
\addplot[] plot coordinates {(-2,0)(0,1)(2,0)(4,1) (6,0)};
\end{axis}
\end{tikzpicture}
\caption*{(ب)}
\end{subfigure}%
\caption{سوال \حوالہ{سوال_فوریئر_چکور_ج} کی موج۔}
\label{شکل_سوال_فوریئر_چکور_ج}
\end{figure}

جوابات:
\begin{align*}
a_0&=\tfrac{5}{3}\\
a_n&=\tfrac{2}{n\pi} (\sin \tfrac{2n\pi}{3}+3\sin\tfrac{n\pi}{3})\\
b_n&=0
\end{align*}
\انتہا{سوال}
%=========================
\ابتدا{سوال}\شناخت{سوال_فوریئر_چکور_چ}
شکل \حوالہ{شکل_سوال_فوریئر_چکور_ج}-ب کے تکونی فوریئر تسلسل کے عددی سر حاصل کریں۔

جوابات:
\begin{align*}
a_0&=\tfrac{1}{3}\\
a_n&=\tfrac{3}{n^2\pi^2}(1-\cos \tfrac{2n\pi}{3})\\
b_n&=0
\end{align*}
\انتہا{سوال}
%==========================


%============================
\ابتدا{سوال}\شناخت{سوال_فوریئر_سمت_کار_الف}
نصف لہر \اصطلاح{سمت کار}\فرہنگ{سمت کار!نصف لہر}\حاشیہب{half wave rectifier}\فرہنگ{rectifier!half wave} کے ذریعہ بدلتے دباو کے مثبت حصے حاصل کرتے ہوئے شکل \حوالہ{شکل_سوال_فوریئر_سمت_کار_الف}-الف حاصل ہوتا ہے۔اس کا فوریئر تسلسل حاصل کریں۔ 
\begin{figure}
\centering
\begin{subfigure}{0.5\textwidth}
\centering
\begin{tikzpicture}
\begin{axis}[small,xtick={0,180,360,540},xticklabels={$0$,$\pi$,$2\pi$,$3\pi$},ytick={0,1},yticklabels={$0$,$A$},xlabel={$t$},ylabel={$v(t)$},ylabel style={rotate=-90},ylabel style={at={(axis description cs:0,1.05)}}]
\addplot[domain=0:180]{sin(x)};
\addplot[domain=360:540]{sin(x)};
\addplot[] plot coordinates {(-50,0) (0,0)};
\addplot[] plot coordinates {(180,0) (360,0)};
\addplot[] plot coordinates {(540,0) (580,0)};
\end{axis}
\end{tikzpicture}
\caption*{(الف)}
\end{subfigure}%
\begin{subfigure}{0.5\textwidth}
\centering
\begin{tikzpicture}
\begin{axis}[small,xtick={0,180,360,540},xticklabels={$0$,$\pi$,$2\pi$,$3\pi$},ytick={0,1},yticklabels={$0$,$A$},xlabel={$t$},ylabel={$v(t)$},ylabel style={rotate=-90},ylabel style={at={(axis description cs:0,1.05)}}]
\addplot[domain=0:540,samples=100]{abs(sin(x))};
\end{axis}
\end{tikzpicture}
\caption*{(ب)}
\end{subfigure}%
\caption{سوال \حوالہ{سوال_فوریئر_سمت_کار_الف} کی موج۔}
\label{شکل_سوال_فوریئر_سمت_کار_الف}
\end{figure}

جواب:
$v(t)=\tfrac{A}{\pi}+\tfrac{A}{2}\sin \omega t-\sum\limits_{\substack{n=2\\ \text{جفت}}}^{\infty} \tfrac{2A}{(n^2-1)\pi} \cos n\omega t$
\انتہا{سوال}
%===================================
\ابتدا{سوال}\شناخت{سوال_فوریئر_سمت_کار_ب}
مکمل لہر \اصطلاح{سمت کار}\فرہنگ{سمت کار!مکمل لہر}\حاشیہب{full wave rectifier}\فرہنگ{rectifier!full wave} کے ذریعہ بدلتے دباو کے مثبت حصے حاصل کرتے ہوئے شکل \حوالہ{شکل_سوال_فوریئر_سمت_کار_الف}-ب حاصل ہوتا ہے۔اس کا فوریئر تسلسل حاصل کریں۔ 

جواب:
$v(t)=\tfrac{2A}{\pi}- \sum \limits_{n=1}^{\infty} \tfrac{4A}{(4n^2-1)\pi}\cos n\omega t$
\انتہا{سوال}
%==============================
%============================
\ابتدا{سوال}\شناخت{سوال_فوریئر_سمت_کار_پ}
شکل \حوالہ{شکل_سوال_فوریئر_سمت_کار_پ}-الف میں دیے تفاعل کا فوریئر تسلسل لکھیں۔
\begin{figure}
\centering
\begin{subfigure}{0.5\textwidth}
\centering
\begin{tikzpicture}
\begin{axis}[small,xtick={-1,0,1},xticklabels={$-\frac{T}{2}$,$0$,$\frac{T}{2}$},ytick={-1,0,1},yticklabels={$-A$,$0$,$A$},xlabel={$t$},ylabel={$f(t)$},ylabel style={rotate=-90},ylabel style={at={(axis description cs:0,1.05)}}]
\addplot[] plot coordinates {(-3,-1)(-1,1)(-1,-1)(1,1)(1,-1)(3,1)(3,-1)};
\end{axis}
\end{tikzpicture}
\caption*{(الف)}
\end{subfigure}%
\begin{subfigure}{0.5\textwidth}
\centering
\begin{tikzpicture}
\begin{axis}[small,xtick={-2,0,2},xticklabels={$-\frac{T}{2}$,$0$,$\frac{T}{2}$},ytick={-1,0,1},yticklabels={$-A$,$0$,$A$},xlabel={$t$},ylabel={$f(t)$},ylabel style={rotate=-90},ylabel style={at={(axis description cs:0,1.05)}}]
\addplot[] plot coordinates {(-4,0)(-3,1)(-1,-1)(1,1)(3,-1)(4,0)};
\end{axis}
\end{tikzpicture}
\caption*{(ب)}
\end{subfigure}%
\caption{سوال \حوالہ{سوال_فوریئر_سمت_کار_پ} کی موج۔}
\label{شکل_سوال_فوریئر_سمت_کار_پ}
\end{figure}

جواب:
$f(t)=\sum \limits_{n=1}^{\infty} \tfrac{2A}{n\pi}\sin n\omega t$
\انتہا{سوال}
%===================================
\ابتدا{سوال}\شناخت{سوال_فوریئر_سمت_کار_ت}
شکل \حوالہ{شکل_سوال_فوریئر_سمت_کار_پ}-ب میں دیے تفاعل کا فوریئر تسلسل لکھیں۔

جواب:
$f(t)=\sum \limits_{\substack{n=1\\ \text{طاق}}}^{\infty} \tfrac{8A}{n^2\pi^2} \sin \tfrac{n\pi}{2} \sin n\omega t$
\انتہا{سوال}
%==============================
\ابتدا{سوال}\شناخت{سوال_فوریئر_سمت_کار_ٹ}
شکل \حوالہ{شکل_سوال_فوریئر_سمت_کار_ٹ}-الف میں دیے تفاعل کا فوریئر تسلسل لکھیں۔
\begin{figure}
\centering
\begin{subfigure}{0.5\textwidth}
\centering
\begin{tikzpicture}
\begin{axis}[small,xtick={0,1,2},xticklabels={$0$,$\frac{T}{2}$,$T$},ytick={-1,0,1},yticklabels={$-A$,$0$,$A$},xlabel={$t$},ylabel={$f(t)$},ylabel style={rotate=-90},ylabel style={at={(axis description cs:0,1.05)}}]
\addplot[] plot coordinates {(0,-1)(0,1)(1,1)(1,-1)(2,-1)(2,1)(3,1)(3,-1)};
\end{axis}
\end{tikzpicture}
\caption*{(الف)}
\end{subfigure}%
\begin{subfigure}{0.5\textwidth}
\centering
\begin{tikzpicture}
\begin{axis}[small,xtick={0,1,2},xticklabels={$0$,$\frac{T}{2}$,$T$},ytick={-1,0,1},yticklabels={$-A$,$0$,$A$},xlabel={$t$},ylabel={$f(t)$},ylabel style={rotate=-90},ylabel style={at={(axis description cs:0,1.05)}}]
\addplot[] plot coordinates {(0,0)(1,1)(2,0)(3,1)(4,0)(5,1)(6,0)};
\end{axis}
\end{tikzpicture}
\caption*{(ب)}
\end{subfigure}%
\caption{سوال \حوالہ{سوال_فوریئر_سمت_کار_ٹ} کی موج۔}
\label{شکل_سوال_فوریئر_سمت_کار_ٹ}
\end{figure}

جواب:
$f(t)=\sum \limits_{\substack{n=1 \\ \text{طاق}}}^{\infty} \tfrac{4A}{n\pi} \sin n\omega t$
\انتہا{سوال}
%===================================
\ابتدا{سوال}\شناخت{سوال_فوریئر_سمت_کار_ث}
شکل \حوالہ{شکل_سوال_فوریئر_سمت_کار_ٹ}-ب میں دیے تفاعل کا فوریئر تسلسل لکھیں۔

جواب:
$f(t)=\tfrac{A}{2}-\sum\limits_{\substack{n=-\infty \\ n \ne 0 \\ \text{طاق}}}^{\infty} \tfrac{2A}{n^2 \pi^2} e^{jn\omega t}$
\انتہا{سوال}
%==============================


\ابتدا{سوال}\شناخت{سوال_فوریئر_حل_دور_الف}
شکل \حوالہ{شکل_سوال_فوریئر_حل_دور_الف} میں داخلی دباو \عددی{v_d(t)=10\cos(1000t)+5\cos(2000t)\,\si{\volt}} ہے۔رو \عددی{i_0(t)} حاصل کریں۔
\begin{figure}
\centering
\begin{tikzpicture}
\draw(0,0) to [american voltage source ,l={$v_d(t)$}]++(0,\y) to [short]++(2*\x,0) to [capacitor,l={$\SI{1}{\micro\farad}$},i={$i_0(t)$}]++(0,-\y) to [short] (0,0);
\draw(\x,0) to [resistor,*-*,l={$\SI{1}{\kilo\ohm}$}]++(0,\y);
\end{tikzpicture}
\caption{سوال \حوالہ{سوال_فوریئر_حل_دور_الف} کا دور۔}
\label{شکل_سوال_فوریئر_حل_دور_الف}
\end{figure}

جواب:\عددی{i_0(t)=7.07\cos(1000t+45^{\circ})+4.47\cos(2000+26.6^{\circ})\,\si{\ampere}}
\انتہا{سوال}
%==========================
\ابتدا{سوال}
شکل \حوالہ{شکل_سوال_فوریئر_حل_دور_الف} میں داخلی دباو چکور موج ہے جس کا حیطہ \عددی{\SI{10}{\volt}} اور جس کا دوری عرصہ \عددی{T=\SI{5}{\milli\second}} ہے۔اس داخلی دباو کو شکل \حوالہ{شکل_سوال_فوریئر_سمت_کار_ٹ}-الف میں دکھایا گیا ہے۔رو \عددی{i_0(t)} حاصل کریں۔
\انتہا{سوال}
%==========================
\ابتدا{سوال}
ایک دور کے داخلی دباو اور داخلی رو درج ذیل ہیں۔ دور کو منتقل طاقت حاصل کریں۔
\begin{align*}
v_d(t)&=10+6\cos(100t+20^{\circ})+4\cos(300t+80^{\circ})\,\si{\volt}\\
i_d(t)&=2-3\cos(100t-40^{\circ})+2\cos(200t+30^{\circ})\\
&\quad \quad\quad\quad +3\cos(300t+60^{\circ})+8\cos(400t+40^{\circ})\,\si{\ampere}
\end{align*} 

جواب:\عددی{\SI{21.14}{\watt}}
\انتہا{سوال}
%===========================
\ابتدا{سوال}\شناخت{سوال_فوریئر_حل_دور_ت}
شکل \حوالہ{شکل_سوال_فوریئر_حل_دور_ت} میں دور کو مہیا کل طاقت دریافت کریں۔داخلی دباو درج ذیل ہے۔
\begin{align*}
v_d(t)=40+30\cos(314t+45^{\circ})+20\cos(628-60^{\circ})\,\si{\volt}
\end{align*}
% 
\begin{figure}
\centering
\begin{tikzpicture}
\draw(0,0) to [american voltage source ,l={$v_d(t)$}]++(0,\y) to [resistor,l={$\SI{20}{\ohm}$}]++(\x,0) to [inductor,l={$\SI{20}{\milli\henry}$}]++(\x,0) to [short]++(\x,0) to [capacitor,l={$\SI{20}{\micro\farad}$}]++(0,-\y) to [short] (0,0);
\draw(2*\x,0) to [resistor,*-*,l={$\SI{10}{\ohm}$}]++(0,\y);
\end{tikzpicture}
\caption{سوال \حوالہ{سوال_فوریئر_حل_دور_ت} کا دور۔}
\label{شکل_سوال_فوریئر_حل_دور_ت}
\end{figure}

جواب:\عددی{\SI{68.29}{\watt}}
\انتہا{سوال}
%==========================
\ابتدا{سوال}
درج ذیل تفاعل کے فوریئر بدل حاصل کریں۔
\begin{align*}
f(t)&=e^{-3t}\cos 6t \, u(t)\\
f(t)&=e^{-3t}\sin 6t \, u(t)
\end{align*}

جوابات:
\begin{align*}
\bF(\omega)&=\tfrac{-(j\omega +3)}{\omega^2-6j\omega-45}\\
\bF(\omega)&=\tfrac{-6}{\omega^2-6j\omega-45}
\end{align*}
\انتہا{سوال}
%=============================

\ابتدا{سوال}
درج ذیل تفاعل کے فوریئر بدل حاصل کریں۔
\begin{align*}
f(t)&=e^{-2\abs{t}}\cos 4t
\end{align*}

جوابات:
\begin{align*}
\bF(\omega)&=\tfrac{-(j2\omega+4)}{\omega^2-j4\omega-20}
\end{align*}
\انتہا{سوال}
%=============================

\باب{ریاضی نمونے}

حصہ{فراوانی نمونہ}
شکل \حوالہ{شکل_نمونہ_ڈبہ_دور} میں دو جوڑی سروں والا ڈبہ دور دکھایا گیا ہے۔دور کے داخلی سروں کو بائیں ہاتھ اور خارجی سروں کو دائیں ہاتھ دکھایا جاتا ہے لہٰذا \عددی{AB} داخلی اور \عددی{CD} خارجی سرے ہیں۔داخلی اور خارجی سروں پر دباو کے قطب اور رو کی سمتیں دکھائی گئی ہیں۔یوں نچلے سروں کو حوالہ سرا لیا جاتا ہے اور دونوں اطراف سے ڈبے میں رو داخل ہوتی ہے۔
\begin{figure}
\centering
\begin{tikzpicture}
\draw(0,0) rectangle ++(\xx,\yy);
\draw(0,\yy-\yy/8) to [short,-o,i<_={$\bI_1$}]++(-\xx/2,0)node[left]{$A$};
\draw(0,\yy/8) to [short,-o]++(-\xx/2,0)node[left]{$B$};
\draw(\xx,\yy-\yy/8) to [short,-o,i<^={$\bI_2$}]++(\xx/2,0)node[right]{$C$};
\draw(\xx,\yy/8) to [short,-o]++(\xx/2,0)node[right]{$D$};
\draw(\xx/2,\yy/2) node{خطی دور};
\draw(-\xx/2,\yy/2)node{$\begin{aligned} &+ \\ &\bV_1 \\ &- \end{aligned}$};
\draw(\xx+\xx/2,\yy/2)node{$\begin{aligned} &+ \\ &\bV_2 \\ &- \end{aligned}$};
\end{tikzpicture}
\caption{دو جوڑی سروں والا ڈبہ دور۔}
\label{شکل_نمونہ_ڈبہ_دور}
\end{figure}

 داخلی متغیرات مثلا \عددی{\bI_1} اور \عددی{\bV_1} کو زیر نوشت میں \عددی{1} سے ظاہر کیا جاتا ہے  جبکہ خارجی متغیرات کو زیر نوشت میں \عددی{2} سے ظاہر کیا جاتا ہے۔ڈبہ دور خطی دور ہے جس میں غیر تابع منبع نہیں پائے جاتے لہٰذا \عددی{\bI_1} اور \عددی{\bI_2} حاصل کرتے ہوئے مسئلہ نفاذ استعمال کیا جا سکتا ہے۔ یوں \عددی{\bV_1} اور \عددی{\bV_2} سے پیدا داخلی جانب رو کا مجموعہ \عددی{\bI_1} ہو گا اور اسی طرح خارجی جانب دونوں اطراف کے دباو سے پیدا رو کا مجموعہ \عددی{\bI_2} ہو گا یعنی
\begin{gather}
\begin{aligned}\label{مساوات_نمونہ_الف}
\bI_1&=y_{11}\bV_1+y_{12}\bV_2\\
\bI_2&=y_{21}\bV_1+y_{22}\bV_2
\end{aligned}
\end{gather}  
جہاں \عددی{y_{11}}، \عددی{y_{12}} وغیرہ مستقل ہیں جنہیں سیمنز \عددی{\si{\siemens}} میں ناپا جاتا ہے۔ان مساوات کو قالب کی شکل میں لکھتے ہیں۔\عددی{y_{11}}، \عددی{y_{12}}، \عددی{y_{21}} اور \عددی{y_{22}} کو \عددی{Y} مقدار کہتے ہیں۔اگر \عددی{Y} کی قیمتیں معلوم ہوں تب ڈبہ دور کی خارجی بالمقابل داخلی تعلقات مکمل طور پر تعین کی جا سکتی ہیں۔ 
\begin{align}\label{مساوات_نمونہ_ب}
\begin{bmatrix}
\bI_1 \\
\bI_2
\end{bmatrix}
=
\begin{bmatrix}
y_{11} & y_{12}\\
y_{21} & y_{22}
\end{bmatrix}
\begin{bmatrix}
\bV_1\\
\bV_2
\end{bmatrix}
\end{align}

مساوات \حوالہ{مساوات_نمونہ_الف} میں خارجی سروں کو قصر دور کرنے سے \عددی{\bV_2=0} ہو گا اور یوں \عددی{y_{11}} کو درج ذیل لکھا جا سکتا ہے۔
\begin{align}
y_{11}=\left. \frac{\bI_1}{\bV_1} \right|_{\bV_2=0}
\end{align}
\عددی{y_{11}} کو \اصطلاح{قصر دور داخلی فراوانی}\فرہنگ{قصر دور!داخلی فراوانی}\فرہنگ{داخلی فراوانی!قصر دور}\حاشیہب{short-circuit input admittance}\فرہنگ{admittance,short-circuit input} کہتے ہیں۔بقایا مقدار بھی اسی طرح حاصل کیے جا سکتے ہیں۔
\begin{gather}
\begin{aligned}
y_{12}&=\left. \frac{\bI_1}{\bV_2} \right|_{\bV_1=0}\\
y_{21}&=\left. \frac{\bI_2}{\bV_1} \right|_{\bV_2=0}\\
y_{22}&=\left. \frac{\bI_2}{\bV_2} \right|_{\bV_1=0}
\end{aligned}
\end{gather}
\عددی{y_{12}} اور \عددی{y_{21}} کو \اصطلاح{قصر دور فراوانی نما}\فرہنگ{قصر دور!فراوانی نما}\حاشیہب{short-circuit transadmittance}\فرہنگ{transadmittance!short-circuit} کہا جاتا ہے جبکہ \عددی{y_{22}} کو \اصطلاح{قصر دور خارجی فراوانی}\حاشیہب{short-circuit output admittance}\فرہنگ{output admittance!short-circuit} کہتے ہیں۔درج بالا مساوات کو استعمال کرتے ہوئے کسی بھی نامعلوم دور کے \عددی{Y} مقدار تجرباتی طور ناپے جا سکتا ہیں۔ 
%================
\ابتدا{مثال}\شناخت{مثال_نمونہ_مزاحمتی_دور_الف}
شکل \حوالہ{شکل_نمونہ_مزاحمتی_دور_الف} میں دور دکھایا گیا ہے۔اس کے \عددی{Y} مقدار دریافت کریں۔
\begin{figure}
\centering
\begin{subfigure}{1\textwidth}
\centering
\begin{tikzpicture}
\draw(0,0) to [short,o-,i={$\bI_1$}]++(\x,0) to [resistor,l={$\SI{4}{\ohm}$}]++(\x,0) to [short,,i<={$\bI_2$},-o]++(\x/2,0);
\draw(0,-\y) to [short,o-o]++(2*\x+\x/2,0);
\draw(\x,0) to [resistor,*-*,l={$\SI{2}{\ohm}$}]++(0,-\y);
\draw(0,-\y/2)node{$\begin{aligned}  &+ \\ &\bV_1 \\ &- \end{aligned}$};
\draw(2*\x+\x/2,-\y/2)node{$\begin{aligned}  &+ \\ &\bV_2 \\ &- \end{aligned}$};
\end{tikzpicture}
\caption*{(الف)}
\end{subfigure}
\begin{subfigure}{0.5\textwidth}
\centering
\begin{tikzpicture}
\draw(0,0) to [short,o-,i={$\bI_1$}]++(\x,0) to [resistor,l={$\SI{4}{\ohm}$}]++(\x,0) to[short,i<={$\bI_2$}]++(\x/4,0)to  [short]++(0,-\y);
\draw(0,-\y) to [short,o-]++(2*\x+\x/4,0);
\draw(\x,0) to [resistor,*-*,l={$\SI{2}{\ohm}$}]++(0,-\y);
\draw(0,-\y/2)node{$\begin{aligned}  &+ \\ &\bV_1 \\ &- \end{aligned}$};
\end{tikzpicture}
\caption*{(ب)}
\end{subfigure}%
\begin{subfigure}{0.5\textwidth}
\centering
\begin{tikzpicture}
\draw(0,0) to [short,i={$\bI_1$}]++(\x,0) to [resistor,l={$\SI{4}{\ohm}$}]++(\x,0) to [short,,i<={$\bI_2$},-o]++(\x/2,0);
\draw(0,-\y) to [short,-o]++(2*\x+\x/2,0);
\draw(\x,0) to [resistor,*-*,l={$\SI{2}{\ohm}$}]++(0,-\y);
\draw(0,0) --++(0,-\y);
\draw(2*\x+\x/2,-\y/2)node{$\begin{aligned}  &+ \\ &\bV_2 \\ &- \end{aligned}$};
\end{tikzpicture}
\caption*{(پ)}
\end{subfigure}
\begin{subfigure}{1\textwidth}
\centering
\begin{tikzpicture}
\draw(0,0) to [short,o-,i={$\bI_1$}]++(\x,0) to [resistor,l={$\SI{4}{\ohm}$}]++(\x,0) to [short,,i<={$\bI_2$},-o]++(\x/2,0);
\draw(0,-\y) to [short,o-o]++(2*\x+\x/2,0);
\draw(\x,0) to [resistor,*-*,l={$\SI{2}{\ohm}$}]++(0,-\y);
\draw(0,-\y) to [short,o-]++(-\x/2,0) to [american current source,l={$\SI{2}{\ampere}$}]++(0,\y) to [short,-o]++(\x/2,0);
\draw(2*\x+\x/2,0) to [short,o-]++(\x/2,0)  to [resistor,l={$\SI{3}{\ohm}$}]++(0,-\y) to [short,-o]++(-\x/2,0);
\draw(0,-\y/2)node{$\begin{aligned}  &+ \\ &\bV_1 \\ &- \end{aligned}$};
\draw(2*\x+\x/2,-\y/2)node{$\begin{aligned}  &+ \\ &\bV_2 \\ &- \end{aligned}$};
\end{tikzpicture}
\caption*{(ت)}
\end{subfigure}
\caption{مثال \حوالہ{مثال_نمونہ_مزاحمتی_دور_الف} کا دور۔}
\label{شکل_نمونہ_مزاحمتی_دور_الف}
\end{figure}

حل:\عددی{y_{11}} حاصل کرنے کی خاطر خارجی سروں کو قصر دور کرتے ہوئے داخلی جانب \عددی{\bV_1} مسلط کرتے ہیں۔شکل-ب میں ایسا دکھایا گیا ہے جہاں سے 
\begin{align*}
\bI_1&=\frac{\bV_1}{\frac{2\times 4}{2+4}}=\frac{3}{4} \bV_1
\end{align*}
لکھتے ہوئے
\begin{align*}
y_{11}&=\left. \frac{\bI_1}{\bV_1}\right|_{\bV_2=0}=\frac{3}{4}\, \si{\siemens}
\end{align*}
حاصل ہوتا ہے۔شکل-ب سے \عددی{y_{21}} بھی حاصل کیا جا سکتا ہے۔دور کو دیکھ کر درج ذیل لکھا جا سکتا ہے
\begin{align*}
\bI_2=-\frac{\bV_1}{4}
\end{align*}
لہٰذا
\begin{align*}
y_{21}=\left. \frac{\bI_2}{\bV_1}\right|_{\bV_2=0}=-\frac{1}{4}\,\si{\siemens}
\end{align*}
ہو گا۔


\عددی{y_{12}} حاصل کرنے کی خاطر داخلی سروں کو قصر دور کرتے ہوئے شکل-پ حاصل ہوتا ہے جس میں \عددی{\SI{2}{\ohm}} کے مزاحمت کو ہٹایا جا سکتا ہے۔اس دور سے درج ذیل لکھا جا سکتا ہے
\begin{align*}
\bI_1=-\frac{\bV_2}{4}
\end{align*} 
لہٰذا
\begin{align*}
y_{12}=\left.\frac{\bV_2}{\bI_1} \right|_{\bV_1=0}=-\frac{1}{4}\,\si{\siemens}
\end{align*}
ہو گا۔شکل-پ سے درج ذیل
\begin{align*}
\bI_2=\frac{\bV_2}{4}
\end{align*}
 لکھتے ہوئے
\begin{align*}
y_{22}=\left.\frac{\bI_2}{\bV_2} \right|_{\bV_1=0}=\frac{1}{4} \, \si{\siemens}
\end{align*}
حاصل ہوتا ہے۔

ان معلومات کو استعمال کرتے ہوئے مساوات \حوالہ{مساوات_نمونہ_الف} لکھتے ہیں
\begin{gather}
\begin{aligned}\label{مساوات_نمونہ_مثال_الف}
\bI_1&=\frac{3}{4}\bV_1-\frac{1}{4}\bV_2\\
\bI_2&=-\frac{1}{4}\bV_1+\frac{1}{4}\bV_2
\end{aligned}
\end{gather}
جنہیں قالب کی شکل میں لکھتے ہیں جو اس دور کو مکمل طور ظاہر کرتی ہے۔
\begin{align*}
\begin{bmatrix}
\bI_1\\
\bI_2
\end{bmatrix}
=
\begin{bmatrix}
\frac{3}{4} & -\frac{1}{4}\\
-\frac{1}{4} & \frac{1}{4}
\end{bmatrix}
\begin{bmatrix}
\bV_1\\
\bV_2
\end{bmatrix}
\end{align*}

اس مثال کو مکمل کرنے کی غرض سے  شکل \حوالہ{شکل_نمونہ_مزاحمتی_دور_الف}-الف کے داخلی جانب منبع رو اور خارجی جانب \عددی{\SI{3}{\ohm}} نسب کرتے ہوئے حل کرتے ہیں۔شکل-ت میں اسے دکھایا گیا ہے جہاں
\begin{align*}
\bI_1&=\SI{2}{\ampere}\\
\bV_2&=-3\bI_2
\end{align*}
ہیں۔انہیں مساوات \حوالہ{مساوات_نمونہ_مثال_الف} میں پر کرتے ہوئے 
\begin{align*}
\begin{bmatrix}
\frac{3}{4} & -\frac{1}{4}\\
-\frac{1}{4}& \frac{1}{3}+\frac{1}{4}
\end{bmatrix}
\begin{bmatrix}
\bV_1\\
\bV_2
\end{bmatrix}
=
\begin{bmatrix}
2\\
0
\end{bmatrix}
\end{align*}
ملتا ہے جو عین کرخوف مساوات جوڑ ہیں۔ان سے 
\begin{align*}
\bV_1&=\frac{28}{9}\, \si{\volt}\\
\bV_2&=\frac{4}{3} \, \si{\volt}
\end{align*}
حاصل ہوتا ہے۔ 
\انتہا{مثال}
%===================
\ابتدا{مشق}\شناخت{مشق_نمونہ_مشق_الف}
شکل \حوالہ{شکل_نمونہ_مشق_الف} میں دیے دور کے \عددی{Y} مقدار دریافت کریں۔
\begin{figure}
\centering
\begin{tikzpicture}
\draw(0,0) to [short,o-]++(\x/2,0) to [resistor,l={$\SI{40}{\ohm}$}]++(\x,0) to [short,-o]++(\x/2,0);
\draw(0,-\y) to [short,o-o]++(2*\x,0);
\draw(\x/2,0) to [resistor,*-*,l={$\SI{20}{\ohm}$}]++(0,-\y);
\draw(\x+\x/2,0) to [resistor,*-*,l={$\SI{10}{\ohm}$}]++(0,-\y);
\end{tikzpicture}
\caption{مشق \حوالہ{مشق_نمونہ_مشق_الف} کا دور۔}
\label{شکل_نمونہ_مشق_الف}
\end{figure}

جوابات:\عددی{y_{11}=\tfrac{3}{40}}، \عددی{y_{12}=-\tfrac{1}{40}}، \عددی{y_{21}=-\tfrac{1}{40}} اور \عددی{y_{22}=\tfrac{1}{8}}
\انتہا{مشق}
%==================
\ابتدا{مشق}
شکل \حوالہ{شکل_نمونہ_مشق_الف} میں داخلی جانب \عددی{\SI{3}{\ampere}} کا منبع رو نسب کیا جاتا ہے جبکہ خارجی جانب \عددی{\SI{30}{\ohm}} کا مزاحمت نسب کیا جاتا ہے۔گزشتہ مشق کے \عددی{Y} مقدار استعمال کرتے ہوئے \عددی{\bI_2} دریافت کریں۔

جواب:\عددی{\bI_2=-\tfrac{2}{9}\,\si{\ampere}}
\انتہا{مشق}
%===================

\حصہ{رکاوٹی نمانہ}
گزشتہ حصے میں ہم نے بے منبع دور  کو \عددی{Y} نمونے سے ظاہر کیا۔اس حصے میں دور کے داخلی دباو \عددی{\bV_1} کو داخلی رو \عددی{\bI_1} اور خارجی رو \عددی{\bI_2} کا پیدا کردہ دباو تصور کرتے ہیں۔اسی طرح خارجی دباو کو بھی انہیں رو کا پیدا کردہ دباو تصور کرتے ہیں۔یوں 
\begin{gather}
\begin{aligned}
\bV_1&=z_{11}\bI_1+z_{12}\bI_2\\
\bV_2&=z_{21} \bI_1+z_{22} \bI_2
\end{aligned}
\end{gather}
یا
\begin{align}
\begin{bmatrix}
\bV_1\\
\bV_2
\end{bmatrix}
=
\begin{bmatrix}
z_{11} & z_{12}\\
z_{21}& z_{22}
\end{bmatrix}
\begin{bmatrix}
\bI_1\\
\bI_2
\end{bmatrix}
\end{align}
لکھا جا سکتا ہے۔بالکل \عددی{Y} کی طرح \عددی{Z} مقدار درج ذیل لکھے جا سکتے ہیں۔
\begin{gather}
\begin{aligned} 
z_{11}&=\left. \frac{\bV_1}{\bI_1}\right|_{\bI_2=0}\\
z_{12}&=\left. \frac{\bV_1}{\bI_2}\right|_{\bI_1=0}\\
z_{21}&=\left. \frac{\bV_2}{\bI_1}\right|_{\bI_2=0}\\
z_{22}&=\left. \frac{\bV_2}{\bI_2}\right|_{\bI_1=0}
\end{aligned}
\end{gather}
یاد رہے کہ رو کو صفر کرنے کی خاطر دور کو کھلے سر کیا جاتا ہے۔اس طرح \عددی{z_{11}} کو \اصطلاح{کھلے سر داخلی رکاوٹ}\فرہنگ{کھلے سر!داخلی رکاوٹ}\فرہنگ{داخلی رکاوٹ!کھلے سر}\حاشیہب{open-circuit input impedance}\فرہنگ{open circuit!input impedance}، \عددی{z_{12}} اور \عددی{z_{21}} کو \اصطلاح{کھلے سر رکاوٹ نما}\فرہنگ{کھلے سر!رکاوٹ نما}\فرہنگ{رکاوٹ نما!کھلے سر}\حاشیہب{open-circuit transimpedance}\فرہنگ{transimpedance!open-circuit} اور \عددی{z_{22}} کو \اصطلاح{کھلے سر خارجی رکاوٹ}\فرہنگ{خارجی رکاوٹ!کھلے سر}\فرہنگ{کھلے سر!خارجی رکاوٹ}\حاشیہب{open-circuit output impedance}\فرہنگ{output impedance!open-circuit} کہتے ہیں۔




%\include{./tex/emtEndOfBookTableDivergenceCurlGradientLaplacian}
%\include{./tex/toDoList}
%\include{./tex/emtQuestions}
\backmatter

\cleardoublepage
%\include{./tex/emtDataTables}        %appendices
%\include{./tex/emtLinearAlgebra}
%\include{./tex/emtCoordinatesRelations}

\renewcommand*{\indexname}{فرہنگ}      %this command has to be placed right here just before printindex command
\cleardoublepage
\addcontentsline{toc}{chapter}{فرہنگ}
\printindex


\end{urdufont}
\end{document}
